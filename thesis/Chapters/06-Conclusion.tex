\chapter[Conclusion]{Conclusion}
\label{cp:conclusion}

\vspace*{-2ex} % TODO: check alignment
In this thesis, four novel methods for the dynamic initiation of new tuning phases in AutoPas were introduced. Additionally, reasonable default values for the corresponding control parameters were derived empirically.
It was shown that our trigger strategies can decrease simulation runtime across most typical scenarios when using \texttt{full-search} as the tuning strategy.
Especially in settings with low variance in iteration runtime, all strategies except the naive approach reduce the number of tuning phases without any significant decrease in optimality of the selected configurations.
The most promising candidates were shown to be the strategies based on sample averages, due to their resilience to noise in the live input data. These strategies lead to speedups of up to \qty{47}{\percent} under optimal conditions.
The regression strategy investigated was more susceptible to the aforementioned fluctuations in the input data, but nonetheless outperformed static tuning.

As new tuning strategies are introduced, however, the dynamic initiation of tuning intervals will likely become less relevant for single-process applications. In particular, tuning strategies based on machine learning lead to cheap tuning phases \cite{Newcome2025}, which in turn significantly diminishes the achievable speedups. One application in which a dynamic approach as ours might still be of use, is in MPI-parallel setups: As each rank runs its own AutoPas instance, it can trigger tuning phases independently from other ranks. In scenarios with heterogeneous particle distribution over the whole domain, the best configuration for a specific rank is likely to change separate from other parts of the domain. Thus, our proposed method may still be advantageous.

Possible future work may explore the analysis of additional live simulation statistics, either as single parameter or hybrid strategies. The introduction of new triggering algorithms providing more stability, such as linear regression using the Theil-Sen estimator, should be considered. More advanced methods such as digital filters have to be implemented efficiently, as not to inflate per-iteration overhead.
Another interesting subject for further research could be the combination of static and dynamic tuning intervals; at fixed, but much shorter intervals, a dynamic trigger would evaluate whether to start a new tuning phase or not.

In summary, the dynamic initiation of tuning phases has been shown to be lightweight method of decreasing unnecessary tuning phases, whilst still ensuring good configuration fit. Using more efficient tuning strategies reduces the performance gains in single-process applications, but some benefits remain in MPI-parallel settings.
