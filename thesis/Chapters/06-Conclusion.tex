\chapter[Conclusion]{Conclusion}
\label{cp:conclusion}

%{
%\parindent0pt
%This chapter will shortly summarize the results which were discussed in \autoref{cp:results}. Additionally, an outlook concerning tuning strategies is given. %TODO
%}
%
%\section{Summary of Findings and Outlook}

In this thesis, four new methods to dynamically initiate new tuning phases in AutoPas were introduced.
As shown in \autoref{cp:results}, our trigger strategies decrease the simulation run time across almost all tested scenarios when using \texttt{full-search} as the tuning strategy. 
Especially in settings in which the optimal configuration does not change rapidly, our method reduces the amount of iterations spent in tuning phases without any significant decrease in the optimality of the chosen configurations. %TODO: check this


The \texttt{TimeBasedSplitTrigger} strategy was shown to be the most promising candidate, which is due to its resilience against high variability in iteration run time. Using this trigger with optimal parameters led to speedups of up to \qty{76}{\percent}. Other strategies also perform well in scenarios \textellipsis %TODO

%TODO
TODO

However, as new tuning strategies are introduced, the dynamic initiation of tuning intervals will likely become less relevant for single-node applications. For example, the use of tuning strategies based on machine learning leads to very cheap tuning phases \cite{Newcome2025}, which in turn significantly diminishes the achievable speedups observed when comparing to tuning with \texttt{full-search}. One application in which a dynamic approach as ours might still be useful, is in shared memory setups: As each node runs its own autotuner instance, it can trigger tuning phases independently from other nodes. In scenarios like exploding-liquid this could still be advantageous.

Possible future work could explore the analysis of \texttt{LiveInfo} parameters, either as single parameter strategies or as hybrid triggers. The introduction of novel trigger methods providing more stability, such as using the Theil-Sen estimator, should be considered.
% TODO: polynomial functions?