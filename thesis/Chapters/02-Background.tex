\chapter[AutoPas]{AutoPas}
\label{cp:autopas}

{
	\parindent0pt
	This chapter examines the particle simulation library AutoPas and provides an overview of its architecture and features. In \autoref{sec:background}, the concept of autotuning is introduced, as well as the  introduces the \texttt{md-flexible} application.
	The various configuration parameters available for the simulation are introduced thereafter in \autoref{sec:config_params}. Additionally, \autoref{sec:tuning_strategies} shortly outlines the different tuning strategies for the selection of the optimal combination of these parameters.
	
	The contents of this chapter are mainly drawn from the works introducing AutoPas, specifically the publications by Gratl et al. \cite{Gratl2019, Gratl2021} and Seckler et al. \cite{Seckler2021}, as well as the AutoPas documentation \cite{AutoPas2025}.
}


\section{Background}
\label{sec:background}
% LJ functor
% Autotuning
% md flexible
% Simulation Loop? \cite{Hospital2015}

\section{Configuration Parameters}
\label{sec:config_params}
\subsection{Containers}
% only a selection of all containers
\begin{description}[leftmargin=!,labelwidth=\widthof{\textbf{Verlet Cluster List }}]
	\item[\textbf{Direct Sum}] TODO
	\item[\textbf{Linked Cells}] TODO
	\item[\textbf{Verlet Lists}] TODO % TODO
	\item[\textbf{Verlet Cluster Lists}] TODO % TODO
\end{description}
\subsection{Traversals}
\subsection{Additional Parameters}
Additionally, there are a number of additional configuration parameters that do not fall into the aforementioned classes. They are given below.

\begin{description}[leftmargin=!,labelwidth=\widthof{\textbf{Cell size factor }}]
	\item[\textbf{Data Layout}] This option concerns the memory layout of the particle structures. As each particle has multiple attributes associated to it, all particles together can be laid out either as an Array of Structures (AoS) or a Structure of Arrays (SoA). In the AoS layout, all particles are laid out after each other, in the SoA layout, each attribute type is grouped into an array, where each entry corresponds to the value of this attribute for a specific particle. \autoref{fig:aos_soa} illustrates both principles.
	\item[\textbf{Newton3}] As mentioned in \autoref{sec:newton}, Newton's third law states that $F_a = -F_b$ for two bodies $a, b$ exerting forces on each other. This allows for optimizing pairwise interactions, as only one force has to be computed. However, this approach is not always beneficial as it may limit parallelization -- once the force is evaluated, both particles must be updated at once.
	\item[\textbf{Cell Size Factor}] TODO % TODO
\end{description}

\begin{figure}[htpb]
	% TODO: color scheme
	\begin{center}
		\begin{tikzpicture}
			\def\sz{2em}
			\def\pad{0.125em}
			\def\brpad{3.25ex}
			\def\lblspc{2em}
			\def\rowspc{8ex}
			\tikzset{
				lbl/.style={draw=none, minimum width=\sz, minimum height=\sz, text depth=1pt},
				attr/.style={draw, minimum width=\sz, minimum height=\sz, anchor=west, text depth=1pt},
				br/.style={decorate,decoration={brace, amplitude=1ex, raise=0.5ex}},
				p1/.style={draw=black, fill=lightblue},
				p2/.style={draw=black, fill=lightviolet},
				p3/.style={draw=black, fill=lightred},
			}

			\node[lbl] (aos) {\textbf{AoS}};

			\node[attr, p1, right=\lblspc of aos] (x1) {$r_x^{(1)}$};
			\node[attr, p1, right=-\pgflinewidth of x1] (y1) {$r_y^{(1)}$};
			\node[attr, p1, right=-\pgflinewidth of y1] (z1) {$r_z^{(1)}$};
			\node[attr, p1, right=-\pgflinewidth of z1] (a1) {\textellipsis};
			\draw [br] ([xshift=\pad]x1.north west) -- ([xshift=-\pad]a1.north east) node[midway, yshift=\brpad]{Particle 1};

			\node[attr, p2, right=-\pgflinewidth of a1] (x2) {$r_x^{(2)}$};
			\node[attr, p2, right=-\pgflinewidth of x2] (y2) {$r_y^{(2)}$};
			\node[attr, p2, right=-\pgflinewidth of y2] (z2) {$r_z^{(2)}$};
			\node[attr, p2, right=-\pgflinewidth of z2] (a2) {\textellipsis};
			\draw [br] ([xshift=\pad]x2.north west) -- ([xshift=-\pad]a2.north east) node[midway, yshift=\brpad]{Particle 2};

			\node[attr, p3, right=-\pgflinewidth of a2] (x3) {$r_x^{(3)}$};
			\node[attr, p3, right=-\pgflinewidth of x3] (y3) {$r_y^{(3)}$};
			\node[attr, p3, right=-\pgflinewidth of y3] (z3) {$r_z^{(3)}$};
			\node[attr, p3, right=-\pgflinewidth of z3] (a3) {\textellipsis};
			\draw [br] ([xshift=\pad]x3.north west) -- ([xshift=-\pad]a3.north east) node[midway, yshift=\brpad]{Particle 3};
			\node[attr,right=-\pgflinewidth of a3] {\textellipsis};

			\node[lbl, below=\rowspc of aos] (soa) {\textbf{SoA}};

			\node[attr, p1, right=\lblspc of soa] (x1d) {$r_x^{(1)}$};
			\node[attr, p2, right=-\pgflinewidth of x1d] (x2d) {$r_x^{(2)}$};
			\node[attr, p3, right=-\pgflinewidth of x2d] (x3d) {$r_x^{(3)}$};
			\node[attr, right=-\pgflinewidth of x3d] (a1d) {\textellipsis};
			\draw [br] ([xshift=\pad]x1d.north west) -- ([xshift=-\pad]a1d.north east) node[midway, yshift=\brpad]{$r_x^{(i)}$};

			\node[attr, p1, right=-\pgflinewidth of a1d] (y1d) {$r_y^{(1)}$};
			\node[attr, p2, right=-\pgflinewidth of y1d] (y2d) {$r_y^{(2)}$};
			\node[attr, p3, right=-\pgflinewidth of y2d] (y3d) {$r_y^{(3)}$};
			\node[attr, right=-\pgflinewidth of y3d] (a2d) {\textellipsis};
			\draw [br] ([xshift=\pad]y1d.north west) -- ([xshift=-\pad]a2d.north east) node[midway, yshift=\brpad]{$r_y^{(i)}$};

			\node[attr, p1, right=-\pgflinewidth of a2d] (z1d) {$r_z^{(1)}$};
			\node[attr, p2, right=-\pgflinewidth of z1d] (z2d) {$r_z^{(2)}$};
			\node[attr, p3, right=-\pgflinewidth of z2d] (z3d) {$r_z^{(3)}$};
			\node[attr, right=-\pgflinewidth of z3d] (a3d) {\textellipsis};
			\draw [br] ([xshift=\pad]z1d.north west) -- ([xshift=-\pad]a3d.north east) node[midway, yshift=\brpad]{$r_z^{(i)}$};
			\node[attr,right=-\pgflinewidth of a3d] {\textellipsis};

		\end{tikzpicture}
	\end{center}
	\caption{Comparison between the Array of Structures (AoS) and Structure of Arrays (SoA) memory layouts. The $r^{(i)}$'s correspond to the position vector of the $i$th particle.}
	\label{fig:aos_soa}
\end{figure}

\section{Tuning Strategies}
\label{sec:tuning_strategies}
% keep this rather short

AutoPas provides a variety of different tuning strategies. These are used in the sampling and selection of the new optimal configuration in the tuning phases of the autotuner. As they are not particularly relevant to the topics discussed in this paper, only selected strategies are presented.
\begin{description}[leftmargin=!,labelwidth=\widthof{\textbf{PredictiveTuning }}]
	\item[\textbf{FullSearch}] The default tuning strategy is an exhaustive search over all possible configurations, i.e. all combinations of parameters. The optimal configuration is thereby guaranteed to be trialed at some point. However, the space of all possible configurations grows exponentially in the number of parameters, of which many configurations may be highly suboptimal. Other tuning strategies are therefore more suitable for most scenarios.
	\item[\textbf{PredictiveTuning}] TODO
	\item[\textbf{FuzzyTuning}] TODO \cite{Lerchner2024}
	\item[\textbf{SlowConfigFilter}] TODO
\end{description}

