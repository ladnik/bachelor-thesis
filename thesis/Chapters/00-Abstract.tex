\thispagestyle{plain}

\pdfbookmark[1]{Abstract}{abstract}
\chapter*{Abstract}

Particle simulations have become an important tool in current research and are used across different fields of study. AutoPas is a particle simulation library, that intends on providing a more accessible interface for researchers. The library finds the most efficient approach on simulating particle interactions without requiring the user to have expert knowledge. To achieve this, the simulation state is observed at fixed iteration intervals at which a new optimal configuration state is searched for. This thesis presents a dynamic approach to find optimal points at which a search of the configuration space is to be initiated, as to reduce possible overhead that might occur in static tuning.

We observe \textellipsis

\guideinfo{In the \textit{Abstract} section, provide a concise summary of your project, highlighting the key points. Begin with a brief statement of the problem or objective, followed by a description of your approach or methodology. Summarise the main results or findings, emphasising their significance or implications. Conclude with a sentence or two on the overall contribution or impact of your work. Keep the abstract clear and focused, ideally within 150-250 words, to give readers a quick understanding of your research and its importance.}

\keywordsen{Keyword A, Keyword B, Keyword C.}

\MediaOptionLogicBlank

\chapter*{Zusammenfassung}

\guideinfo{Na secção \textit{Resumo}, asdf}

\keywordspt{Palavra-Chave A, Palavra-Chave B, Palavra-Chave C.}

\MediaOptionLogicBlank