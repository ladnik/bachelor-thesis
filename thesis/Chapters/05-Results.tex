\chapter[Results]{Results}
\label{cp:results}

{
	\parindent0pt
	\textellipsis
}

% TODO: How do results differ based on tuning strategy? Full-search vs. others
\section{Experimental Setup}
The data analyzed in this chapter was collected on the Linux-Cluster of the Leibniz\-/Rechenzentrum\footnote{\href{https://www.lrz.de/}{https://www.lrz.de/}}. The nodes in the \texttt{cm4} cluster consist of processors in the Sapphire Rapids family (Intel\textsuperscript{\textregistered} Xeon\textsuperscript{\textregistered} Platinum 8480+)  with \qty{2.1}{\gibi \byte} of memory per logical CPU and \qty{488}{\gibi \byte} per node \cite{LSC2025}. For benchmarking purposes, the AutoPas library and \texttt{md-flexible} were compiled with  Spack GCC 13.2.0 and Intel MPI 2021.12.0 on commit \texttt{46bb925c8c5827faee8691afefd9f714630f20f1}.

The scripts used to generate the Slurm jobs and configuration files can be found in the repository of this thesis\footnote{\href{https://github.com/ladnik/bachelor-thesis}{https://github.com/ladnik/bachelor-thesis}}.

% TODO: add version compiler/mpi version, add commit hash

% Refer to Appendix for additional data
% Refer to Input files
\section{Choice of Simulation Statistics}
As referred to before in \autoref{sec:avail_sim_stats}, there are several simulation statistics available upon which we could base our trigger strategies. Even iteration runtimes themselves are subdivided into the time spent on computing interactions, traversing remainders and rebuilding neighbor lists. % TODO: explain remainder traversal?
In this thesis, we will consider the sum of these times with the exception of the rebuilding measurements. This choice can be justified by inspecting runtime data we collected. As shown in \autoref{fig:rebuild_times}, the rebuild times smooth out the overall measurements and thus decrease the effectivity at which a scenario change can be detected.
Moreover, the rebuilding of neighbor lists happens at a fixed \texttt{rebuildFrequency}, which leads to problems in trigger strategies with a low number of samples, as the rebuild iterations greatly outweigh all non-rebuild iterations.

\begin{figure}[htpb]
	\centering
%	% This file was created with matplot2tikz v0.4.0.
\documentclass[crop,tikz]{standalone}

\input{../plot_preamble.tex}

\begin{document}
\begin{tikzpicture}

	\definecolor{darkgray176}{RGB}{176,176,176}
	\definecolor{orange}{RGB}{255,165,0}

	\begin{axis}[
			tick align=outside,
			tick pos=left,
			title={equilibrium\_dynamic\_TimeBasedRegression\_1.25\_500
					Rebuild and Non-Rebuild times},
			x grid style={darkgray176},
			xlabel={iteration},
			xmajorgrids,
			xmin=1268.05, xmax=20890.95,
			xtick style={color=black},
			xtick={0,2500,5000,7500,10000,12500,15000,17500,20000,22500},
			xticklabels={
					\(\displaystyle {0}\),
					\(\displaystyle {2500}\),
					\(\displaystyle {5000}\),
					\(\displaystyle {7500}\),
					\(\displaystyle {10000}\),
					\(\displaystyle {12500}\),
					\(\displaystyle {15000}\),
					\(\displaystyle {17500}\),
					\(\displaystyle {20000}\),
					\(\displaystyle {22500}\)
				},
			y grid style={darkgray176},
			ylabel={time in ns},
			ymajorgrids,
			ymin=-54432.95, ymax=1143091.95,
			ytick style={color=black},
			ytick={-200000,0,200000,400000,600000,800000,1000000,1200000},
			yticklabels={
					\(\displaystyle {\ensuremath{-}0.2}\),
					\(\displaystyle {0.0}\),
					\(\displaystyle {0.2}\),
					\(\displaystyle {0.4}\),
					\(\displaystyle {0.6}\),
					\(\displaystyle {0.8}\),
					\(\displaystyle {1.0}\),
					\(\displaystyle {1.2}\)
				}
		]
		\addplot [draw=orange, fill=orange, mark=*, only marks]
		table{%
				x  y
				2160 890452
				2161 0
				2162 0
				2163 0
				2164 0
				2165 0
				2166 0
				2167 0
				2168 0
				2169 0
				2170 0
				2171 0
				2172 0
				2173 0
				2174 0
				2175 1005033
				2176 0
				2177 0
				2178 0
				2179 0
				2180 0
				2181 0
				2182 0
				2183 0
				2184 0
				2185 0
				2186 0
				2187 0
				2188 0
				2189 0
				2190 991619
				2191 0
				2192 0
				2193 0
				2194 0
				2195 0
				2196 0
				2197 0
				2198 0
				2199 0
				2200 0
				2201 0
				2202 0
				2203 0
				2204 0
				2205 1041715
				2206 0
				2207 0
				2208 0
				2209 0
				2210 0
				2211 0
				2212 0
				2213 0
				2214 0
				2215 0
				2216 0
				2217 0
				2218 0
				2219 0
				2220 994385
				2221 0
				2222 0
				2223 0
				2224 0
				2225 0
				2226 0
				2227 0
				2228 0
				2229 0
				2230 0
				2231 0
				2232 0
				2233 0
				2234 0
				2235 1039050
				2236 0
				2237 0
				2238 0
				2239 0
				2240 0
				2241 0
				2242 0
				2243 0
				2244 0
				2245 0
				2246 0
				2247 0
				2248 0
				2249 0
				2250 988523
				2251 0
				2252 0
				2253 0
				2254 0
				2255 0
				2256 0
				2257 0
				2258 0
				2259 0
				2260 0
				2261 0
				2262 0
				2263 0
				2264 0
				2265 1006843
				2266 0
				2267 0
				2268 0
				2269 0
				2270 0
				2271 0
				2272 0
				2273 0
				2274 0
				2275 0
				2276 0
				2277 0
				2278 0
				2279 0
				2280 997793
				2281 0
				2282 0
				2283 0
				2284 0
				2285 0
				2286 0
				2287 0
				2288 0
				2289 0
				2290 0
				2291 0
				2292 0
				2293 0
				2294 0
				2295 996527
				2296 0
				2297 0
				2298 0
				2299 0
				2300 0
				2301 0
				2302 0
				2303 0
				2304 0
				2305 0
				2306 0
				2307 0
				2308 0
				2309 0
				2310 990977
				2311 0
				2312 0
				2313 0
				2314 0
				2315 0
				2316 0
				2317 0
				2318 0
				2319 0
				2320 0
				2321 0
				2322 0
				2323 0
				2324 0
				2325 1001033
				2326 0
				2327 0
				2328 0
				2329 0
				2330 0
				2331 0
				2332 0
				2333 0
				2334 0
				2335 0
				2336 0
				2337 0
				2338 0
				2339 0
				2340 994299
				2341 0
				2342 0
				2343 0
				2344 0
				2345 0
				2346 0
				2347 0
				2348 0
				2349 0
				2350 0
				2351 0
				2352 0
				2353 0
				2354 0
				2355 992416
				2356 0
				2357 0
				2358 0
				2359 0
				2360 0
				2361 0
				2362 0
				2363 0
				2364 0
				2365 0
				2366 0
				2367 0
				2368 0
				2369 0
				2370 1033151
				2371 0
				2372 0
				2373 0
				2374 0
				2375 0
				2376 0
				2377 0
				2378 0
				2379 0
				2380 0
				2381 0
				2382 0
				2383 0
				2384 0
				2385 1013073
				2386 0
				2387 0
				2388 0
				2389 0
				2390 0
				2391 0
				2392 0
				2393 0
				2394 0
				2395 0
				2396 0
				2397 0
				2398 0
				2399 0
				2400 986857
				2401 0
				2402 0
				2403 0
				2404 0
				2405 0
				2406 0
				2407 0
				2408 0
				2409 0
				2410 0
				2411 0
				2412 0
				2413 0
				2414 0
				2415 986716
				2416 0
				2417 0
				2418 0
				2419 0
				2420 0
				2421 0
				2422 0
				2423 0
				2424 0
				2425 0
				2426 0
				2427 0
				2428 0
				2429 0
				2430 1019809
				2431 0
				2432 0
				2433 0
				2434 0
				2435 0
				2436 0
				2437 0
				2438 0
				2439 0
				2440 0
				2441 0
				2442 0
				2443 0
				2444 0
				2445 995491
				2446 0
				2447 0
				2448 0
				2449 0
				2450 0
				2451 0
				2452 0
				2453 0
				2454 0
				2455 0
				2456 0
				2457 0
				2458 0
				2459 0
				2460 1015357
				2461 0
				2462 0
				2463 0
				2464 0
				2465 0
				2466 0
				2467 0
				2468 0
				2469 0
				2470 0
				2471 0
				2472 0
				2473 0
				2474 0
				2475 1012545
				2476 0
				2477 0
				2478 0
				2479 0
				2480 0
				2481 0
				2482 0
				2483 0
				2484 0
				2485 0
				2486 0
				2487 0
				2488 0
				2489 0
				2490 1011335
				2491 0
				2492 0
				2493 0
				2494 0
				2495 0
				2496 0
				2497 0
				2498 0
				2499 0
				2500 0
				2501 0
				2502 0
				2503 0
				2504 0
				2505 1007638
				2506 0
				2507 0
				2508 0
				2509 0
				2510 0
				2511 0
				2512 0
				2513 0
				2514 0
				2515 0
				2516 0
				2517 0
				2518 0
				2519 0
				2520 1012852
				2521 0
				2522 0
				2523 0
				2524 0
				2525 0
				2526 0
				2527 0
				2528 0
				2529 0
				2530 0
				2531 0
				2532 0
				2533 0
				2534 0
				2535 1001322
				2536 0
				2537 0
				2538 0
				2539 0
				2540 0
				2541 0
				2542 0
				2543 0
				2544 0
				2545 0
				2546 0
				2547 0
				2548 0
				2549 0
				2550 996145
				2551 0
				2552 0
				2553 0
				2554 0
				2555 0
				2556 0
				2557 0
				2558 0
				2559 0
				2560 0
				2561 0
				2562 0
				2563 0
				2564 0
				2565 996359
				2566 0
				2567 0
				2568 0
				2569 0
				2570 0
				2571 0
				2572 0
				2573 0
				2574 0
				2575 0
				2576 0
				2577 0
				2578 0
				2579 0
				2580 978716
				2581 0
				2582 0
				2583 0
				2584 0
				2585 0
				2586 0
				2587 0
				2588 0
				2589 0
				2590 0
				2591 0
				2592 0
				2593 0
				2594 0
				2595 1006661
				2596 0
				2597 0
				2598 0
				2599 0
				2600 0
				2601 0
				2602 0
				2603 0
				2604 0
				2605 0
				2606 0
				2607 0
				2608 0
				2609 0
				2610 999312
				2611 0
				2612 0
				2613 0
				2614 0
				2615 0
				2616 0
				2617 0
				2618 0
				2619 0
				2620 0
				2621 0
				2622 0
				2623 0
				2624 0
				2625 1007332
				2626 0
				2627 0
				2628 0
				2629 0
				2630 0
				2631 0
				2632 0
				2633 0
				2634 0
				2635 0
				2636 0
				2637 0
				2638 0
				2639 0
				2640 1007191
				2641 0
				2642 0
				2643 0
				2644 0
				2645 0
				2646 0
				2647 0
				2648 0
				2649 0
				2650 0
				2651 0
				2652 0
				2653 0
				2654 0
				2655 1030486
				2656 0
				2657 0
				2658 0
				2659 0
				2660 0
				2661 0
				2662 0
				2663 0
				2664 0
				2665 0
				2666 0
				2667 0
				2668 0
				2669 0
				2670 1029766
				2671 0
				2672 0
				2673 0
				2674 0
				2675 0
				2676 0
				2677 0
				2678 0
				2679 0
				2680 0
				2681 0
				2682 0
				2683 0
				2684 0
				2685 1006984
				2686 0
				2687 0
				2688 0
				2689 0
				2690 0
				2691 0
				2692 0
				2693 0
				2694 0
				2695 0
				2696 0
				2697 0
				2698 0
				2699 0
				2700 1000567
				2701 0
				2702 0
				2703 0
				2704 0
				2705 0
				2706 0
				2707 0
				2708 0
				2709 0
				2710 0
				2711 0
				2712 0
				2713 0
				2714 0
				2715 1014704
				2716 0
				2717 0
				2718 0
				2719 0
				2720 0
				2721 0
				2722 0
				2723 0
				2724 0
				2725 0
				2726 0
				2727 0
				2728 0
				2729 0
				2730 1009044
				2731 0
				2732 0
				2733 0
				2734 0
				2735 0
				2736 0
				2737 0
				2738 0
				2739 0
				2740 0
				2741 0
				2742 0
				2743 0
				2744 0
				2745 981230
				2746 0
				2747 0
				2748 0
				2749 0
				2750 0
				2751 0
				2752 0
				2753 0
				2754 0
				2755 0
				2756 0
				2757 0
				2758 0
				2759 0
				2760 1008489
				2761 0
				2762 0
				2763 0
				2764 0
				2765 0
				2766 0
				2767 0
				2768 0
				2769 0
				2770 0
				2771 0
				2772 0
				2773 0
				2774 0
				2775 1006220
				2776 0
				2777 0
				2778 0
				2779 0
				2780 0
				2781 0
				2782 0
				2783 0
				2784 0
				2785 0
				2786 0
				2787 0
				2788 0
				2789 0
				2790 1048963
				2791 0
				2792 0
				2793 0
				2794 0
				2795 0
				2796 0
				2797 0
				2798 0
				2799 0
				2800 0
				2801 0
				2802 0
				2803 0
				2804 0
				2805 987939
				2806 0
				2807 0
				2808 0
				2809 0
				2810 0
				2811 0
				2812 0
				2813 0
				2814 0
				2815 0
				2816 0
				2817 0
				2818 0
				2819 0
				2820 978107
				2821 0
				2822 0
				2823 0
				2824 0
				2825 0
				2826 0
				2827 0
				2828 0
				2829 0
				2830 0
				2831 0
				2832 0
				2833 0
				2834 0
				2835 991872
				2836 0
				2837 0
				2838 0
				2839 0
				2840 0
				2841 0
				2842 0
				2843 0
				2844 0
				2845 0
				2846 0
				2847 0
				2848 0
				2849 0
				2850 996305
				2851 0
				2852 0
				2853 0
				2854 0
				2855 0
				2856 0
				2857 0
				2858 0
				2859 0
				2860 0
				2861 0
				2862 0
				2863 0
				2864 0
				2865 1001844
				2866 0
				2867 0
				2868 0
				2869 0
				2870 0
				2871 0
				2872 0
				2873 0
				2874 0
				2875 0
				2876 0
				2877 0
				2878 0
				2879 0
				2880 1038477
				2881 0
				2882 0
				2883 0
				2884 0
				2885 0
				2886 0
				2887 0
				2888 0
				2889 0
				2890 0
				2891 0
				2892 0
				2893 0
				2894 0
				2895 995671
				2896 0
				2897 0
				2898 0
				2899 0
				2900 0
				2901 0
				2902 0
				2903 0
				2904 0
				2905 0
				2906 0
				2907 0
				2908 0
				2909 0
				2910 986087
				2911 0
				2912 0
				2913 0
				2914 0
				2915 0
				2916 0
				2917 0
				2918 0
				2919 0
				2920 0
				2921 0
				2922 0
				2923 0
				2924 0
				2925 1001205
				2926 0
				2927 0
				2928 0
				2929 0
				2930 0
				2931 0
				2932 0
				2933 0
				2934 0
				2935 0
				2936 0
				2937 0
				2938 0
				2939 0
				2940 1005636
				2941 0
				2942 0
				2943 0
				2944 0
				2945 0
				2946 0
				2947 0
				2948 0
				2949 0
				2950 0
				2951 0
				2952 0
				2953 0
				2954 0
				2955 1030456
				2956 0
				2957 0
				2958 0
				2959 0
				2960 0
				2961 0
				2962 0
				2963 0
				2964 0
				2965 0
				2966 0
				2967 0
				2968 0
				2969 0
				2970 1047249
				2971 0
				2972 0
				2973 0
				2974 0
				2975 0
				2976 0
				2977 0
				2978 0
				2979 0
				2980 0
				2981 0
				2982 0
				2983 0
				2984 0
				2985 997174
				2986 0
				2987 0
				2988 0
				2989 0
				2990 0
				2991 0
				2992 0
				2993 0
				2994 0
				2995 0
				2996 0
				2997 0
				2998 0
				2999 0
				3000 1001978
				3001 0
				3002 0
				3003 0
				3004 0
				3005 0
				3006 0
				3007 0
				3008 0
				3009 0
				3010 0
				3011 0
				3012 0
				3013 0
				3014 0
				3015 1031458
				3016 0
				3017 0
				3018 0
				3019 0
				3020 0
				3021 0
				3022 0
				3023 0
				3024 0
				3025 0
				3026 0
				3027 0
				3028 0
				3029 0
				3030 1006755
				3031 0
				3032 0
				3033 0
				3034 0
				3035 0
				3036 0
				3037 0
				3038 0
				3039 0
				3040 0
				3041 0
				3042 0
				3043 0
				3044 0
				3045 1004394
				3046 0
				3047 0
				3048 0
				3049 0
				3050 0
				3051 0
				3052 0
				3053 0
				3054 0
				3055 0
				3056 0
				3057 0
				3058 0
				3059 0
				3060 1006334
				3061 0
				3062 0
				3063 0
				3064 0
				3065 0
				3066 0
				3067 0
				3068 0
				3069 0
				3070 0
				3071 0
				3072 0
				3073 0
				3074 0
				3075 993515
				3076 0
				3077 0
				3078 0
				3079 0
				3080 0
				3081 0
				3082 0
				3083 0
				3084 0
				3085 0
				3086 0
				3087 0
				3088 0
				3089 0
				3090 1012982
				3091 0
				3092 0
				3093 0
				3094 0
				3095 0
				3096 0
				3097 0
				3098 0
				3099 0
				3100 0
				3101 0
				3102 0
				3103 0
				3104 0
				3105 1010048
				3106 0
				3107 0
				3108 0
				3109 0
				3110 0
				3111 0
				3112 0
				3113 0
				3114 0
				3115 0
				3116 0
				3117 0
				3118 0
				3119 0
				3120 1003680
				3121 0
				3122 0
				3123 0
				3124 0
				3125 0
				3126 0
				3127 0
				3128 0
				3129 0
				3130 0
				3131 0
				3132 0
				3133 0
				3134 0
				3135 997370
				3136 0
				3137 0
				3138 0
				3139 0
				3140 0
				3141 0
				3142 0
				3143 0
				3144 0
				3145 0
				3146 0
				3147 0
				3148 0
				3149 0
				3150 1046409
				3151 0
				3152 0
				3153 0
				3154 0
				3155 0
				3156 0
				3157 0
				3158 0
				3159 0
				3160 0
				3161 0
				3162 0
				3163 0
				3164 0
				3165 1007768
				3166 0
				3167 0
				3168 0
				3169 0
				3170 0
				3171 0
				3172 0
				3173 0
				3174 0
				3175 0
				3176 0
				3177 0
				3178 0
				3179 0
				3180 1009401
				3181 0
				3182 0
				3183 0
				3184 0
				3185 0
				3186 0
				3187 0
				3188 0
				3189 0
				3190 0
				3191 0
				3192 0
				3193 0
				3194 0
				3195 1004683
				3196 0
				3197 0
				3198 0
				3199 0
				3200 0
				3201 0
				3202 0
				3203 0
				3204 0
				3205 0
				3206 0
				3207 0
				3208 0
				3209 0
				3210 994081
				3211 0
				3212 0
				3213 0
				3214 0
				3215 0
				3216 0
				3217 0
				3218 0
				3219 0
				3220 0
				3221 0
				3222 0
				3223 0
				3224 0
				3225 991286
				3226 0
				3227 0
				3228 0
				3229 0
				3230 0
				3231 0
				3232 0
				3233 0
				3234 0
				3235 0
				3236 0
				3237 0
				3238 0
				3239 0
				3240 997440
				3241 0
				3242 0
				3243 0
				3244 0
				3245 0
				3246 0
				3247 0
				3248 0
				3249 0
				3250 0
				3251 0
				3252 0
				3253 0
				3254 0
				3255 982008
				3256 0
				3257 0
				3258 0
				3259 0
				3260 0
				3261 0
				3262 0
				3263 0
				3264 0
				3265 0
				3266 0
				3267 0
				3268 0
				3269 0
				3270 1025556
				3271 0
				3272 0
				3273 0
				3274 0
				3275 0
				3276 0
				3277 0
				3278 0
				3279 0
				3280 0
				3281 0
				3282 0
				3283 0
				3284 0
				3285 1006407
				3286 0
				3287 0
				3288 0
				3289 0
				3290 0
				3291 0
				3292 0
				3293 0
				3294 0
				3295 0
				3296 0
				3297 0
				3298 0
				3299 0
				3300 1027745
				3301 0
				3302 0
				3303 0
				3304 0
				3305 0
				3306 0
				3307 0
				3308 0
				3309 0
				3310 0
				3311 0
				3312 0
				3313 0
				3314 0
				3315 1016584
				3316 0
				3317 0
				3318 0
				3319 0
				3320 0
				3321 0
				3322 0
				3323 0
				3324 0
				3325 0
				3326 0
				3327 0
				3328 0
				3329 0
				3330 1002645
				3331 0
				3332 0
				3333 0
				3334 0
				3335 0
				3336 0
				3337 0
				3338 0
				3339 0
				3340 0
				3341 0
				3342 0
				3343 0
				3344 0
				3345 989328
				3346 0
				3347 0
				3348 0
				3349 0
				3350 0
				3351 0
				3352 0
				3353 0
				3354 0
				3355 0
				3356 0
				3357 0
				3358 0
				3359 0
				3360 988166
				3361 0
				3362 0
				3363 0
				3364 0
				3365 0
				3366 0
				3367 0
				3368 0
				3369 0
				3370 0
				3371 0
				3372 0
				3373 0
				3374 0
				3375 989187
				3376 0
				3377 0
				3378 0
				3379 0
				3380 0
				3381 0
				3382 0
				3383 0
				3384 0
				3385 0
				3386 0
				3387 0
				3388 0
				3389 0
				3390 987393
				3391 0
				3392 0
				3393 0
				3394 0
				3395 0
				3396 0
				3397 0
				3398 0
				3399 0
				3400 0
				3401 0
				3402 0
				3403 0
				3404 0
				3405 999686
				3406 0
				3407 0
				3408 0
				3409 0
				3410 0
				3411 0
				3412 0
				3413 0
				3414 0
				3415 0
				3416 0
				3417 0
				3418 0
				3419 0
				3420 998243
				3421 0
				3422 0
				3423 0
				3424 0
				3425 0
				3426 0
				3427 0
				3428 0
				3429 0
				3430 0
				3431 0
				3432 0
				3433 0
				3434 0
				3435 998720
				3436 0
				3437 0
				3438 0
				3439 0
				3440 0
				3441 0
				3442 0
				3443 0
				3444 0
				3445 0
				3446 0
				3447 0
				3448 0
				3449 0
				3450 1009136
				3451 0
				3452 0
				3453 0
				3454 0
				3455 0
				3456 0
				3457 0
				3458 0
				3459 0
				3460 0
				3461 0
				3462 0
				3463 0
				3464 0
				3465 1011655
				3466 0
				3467 0
				3468 0
				3469 0
				3470 0
				3471 0
				3472 0
				3473 0
				3474 0
				3475 0
				3476 0
				3477 0
				3478 0
				3479 0
				3480 999418
				3481 0
				3482 0
				3483 0
				3484 0
				3485 0
				3486 0
				3487 0
				3488 0
				3489 0
				3490 0
				3491 0
				3492 0
				3493 0
				3494 0
				3495 1028256
				3496 0
				3497 0
				3498 0
				3499 0
				3500 0
				3501 0
				3502 0
				3503 0
				3504 0
				3505 0
				3506 0
				3507 0
				3508 0
				3509 0
				3510 981786
				3511 0
				3512 0
				3513 0
				3514 0
				3515 0
				3516 0
				3517 0
				3518 0
				3519 0
				3520 0
				3521 0
				3522 0
				3523 0
				3524 0
				3525 989883
				3526 0
				3527 0
				3528 0
				3529 0
				3530 0
				3531 0
				3532 0
				3533 0
				3534 0
				3535 0
				3536 0
				3537 0
				3538 0
				3539 0
				3540 986026
				3541 0
				3542 0
				3543 0
				3544 0
				3545 0
				3546 0
				3547 0
				3548 0
				3549 0
				3550 0
				3551 0
				3552 0
				3553 0
				3554 0
				3555 1031166
				3556 0
				3557 0
				3558 0
				3559 0
				3560 0
				3561 0
				3562 0
				3563 0
				3564 0
				3565 0
				3566 0
				3567 0
				3568 0
				3569 0
				3570 1014171
				3571 0
				3572 0
				3573 0
				3574 0
				3575 0
				3576 0
				3577 0
				3578 0
				3579 0
				3580 0
				3581 0
				3582 0
				3583 0
				3584 0
				3585 1011755
				3586 0
				3587 0
				3588 0
				3589 0
				3590 0
				3591 0
				3592 0
				3593 0
				3594 0
				3595 0
				3596 0
				3597 0
				3598 0
				3599 0
				3600 994337
				3601 0
				3602 0
				3603 0
				3604 0
				3605 0
				3606 0
				3607 0
				3608 0
				3609 0
				3610 0
				3611 0
				3612 0
				3613 0
				3614 0
				3615 986043
				3616 0
				3617 0
				3618 0
				3619 0
				3620 0
				3621 0
				3622 0
				3623 0
				3624 0
				3625 0
				3626 0
				3627 0
				3628 0
				3629 0
				3630 1005335
				3631 0
				3632 0
				3633 0
				3634 0
				3635 0
				3636 0
				3637 0
				3638 0
				3639 0
				3640 0
				3641 0
				3642 0
				3643 0
				3644 0
				3645 986883
				3646 0
				3647 0
				3648 0
				3649 0
				3650 0
				3651 0
				3652 0
				3653 0
				3654 0
				3655 0
				3656 0
				3657 0
				3658 0
				3659 0
				3660 1029223
				3661 0
				3662 0
				3663 0
				3664 0
				3665 0
				3666 0
				3667 0
				3668 0
				3669 0
				3670 0
				3671 0
				3672 0
				3673 0
				3674 0
				3675 1000541
				3676 0
				3677 0
				3678 0
				3679 0
				3680 0
				3681 0
				3682 0
				3683 0
				3684 0
				3685 0
				3686 0
				3687 0
				3688 0
				3689 0
				3690 997299
				3691 0
				3692 0
				3693 0
				3694 0
				3695 0
				3696 0
				3697 0
				3698 0
				3699 0
				3700 0
				3701 0
				3702 0
				3703 0
				3704 0
				3705 1004233
				3706 0
				3707 0
				3708 0
				3709 0
				3710 0
				3711 0
				3712 0
				3713 0
				3714 0
				3715 0
				3716 0
				3717 0
				3718 0
				3719 0
				3720 992384
				3721 0
				3722 0
				3723 0
				3724 0
				3725 0
				3726 0
				3727 0
				3728 0
				3729 0
				3730 0
				3731 0
				3732 0
				3733 0
				3734 0
				3735 998235
				3736 0
				3737 0
				3738 0
				3739 0
				3740 0
				3741 0
				3742 0
				3743 0
				3744 0
				3745 0
				3746 0
				3747 0
				3748 0
				3749 0
				3750 998776
				3751 0
				3752 0
				3753 0
				3754 0
				3755 0
				3756 0
				3757 0
				3758 0
				3759 0
				3760 0
				3761 0
				3762 0
				3763 0
				3764 0
				3765 994296
				3766 0
				3767 0
				3768 0
				3769 0
				3770 0
				3771 0
				3772 0
				3773 0
				3774 0
				3775 0
				3776 0
				3777 0
				3778 0
				3779 0
				3780 1029666
				3781 0
				3782 0
				3783 0
				3784 0
				3785 0
				3786 0
				3787 0
				3788 0
				3789 0
				3790 0
				3791 0
				3792 0
				3793 0
				3794 0
				3795 991695
				3796 0
				3797 0
				3798 0
				3799 0
				3800 0
				3801 0
				3802 0
				3803 0
				3804 0
				3805 0
				3806 0
				3807 0
				3808 0
				3809 0
				3810 990154
				3811 0
				3812 0
				3813 0
				3814 0
				3815 0
				3816 0
				3817 0
				3818 0
				3819 0
				3820 0
				3821 0
				3822 0
				3823 0
				3824 0
				3825 999567
				3826 0
				3827 0
				3828 0
				3829 0
				3830 0
				3831 0
				3832 0
				3833 0
				3834 0
				3835 0
				3836 0
				3837 0
				3838 0
				3839 0
				3840 996912
				3841 0
				3842 0
				3843 0
				3844 0
				3845 0
				3846 0
				3847 0
				3848 0
				3849 0
				3850 0
				3851 0
				3852 0
				3853 0
				3854 0
				3855 1000578
				3856 0
				3857 0
				3858 0
				3859 0
				3860 0
				3861 0
				3862 0
				3863 0
				3864 0
				3865 0
				3866 0
				3867 0
				3868 0
				3869 0
				3870 1031508
				3871 0
				3872 0
				3873 0
				3874 0
				3875 0
				3876 0
				3877 0
				3878 0
				3879 0
				3880 0
				3881 0
				3882 0
				3883 0
				3884 0
				3885 1023951
				3886 0
				3887 0
				3888 0
				3889 0
				3890 0
				3891 0
				3892 0
				3893 0
				3894 0
				3895 0
				3896 0
				3897 0
				3898 0
				3899 0
				3900 973705
				3901 0
				3902 0
				3903 0
				3904 0
				3905 0
				3906 0
				3907 0
				3908 0
				3909 0
				3910 0
				3911 0
				3912 0
				3913 0
				3914 0
				3915 994083
				3916 0
				3917 0
				3918 0
				3919 0
				3920 0
				3921 0
				3922 0
				3923 0
				3924 0
				3925 0
				3926 0
				3927 0
				3928 0
				3929 0
				3930 1009087
				3931 0
				3932 0
				3933 0
				3934 0
				3935 0
				3936 0
				3937 0
				3938 0
				3939 0
				3940 0
				3941 0
				3942 0
				3943 0
				3944 0
				3945 1019098
				3946 0
				3947 0
				3948 0
				3949 0
				3950 0
				3951 0
				3952 0
				3953 0
				3954 0
				3955 0
				3956 0
				3957 0
				3958 0
				3959 0
				3960 988802
				3961 0
				3962 0
				3963 0
				3964 0
				3965 0
				3966 0
				3967 0
				3968 0
				3969 0
				3970 0
				3971 0
				3972 0
				3973 0
				3974 0
				3975 982713
				3976 0
				3977 0
				3978 0
				3979 0
				3980 0
				3981 0
				3982 0
				3983 0
				3984 0
				3985 0
				3986 0
				3987 0
				3988 0
				3989 0
				3990 995788
				3991 0
				3992 0
				3993 0
				3994 0
				3995 0
				3996 0
				3997 0
				3998 0
				3999 0
				4000 0
				4001 0
				4002 0
				4003 0
				4004 0
				4005 978099
				4006 0
				4007 0
				4008 0
				4009 0
				4010 0
				4011 0
				4012 0
				4013 0
				4014 0
				4015 0
				4016 0
				4017 0
				4018 0
				4019 0
				4020 1000221
				4021 0
				4022 0
				4023 0
				4024 0
				4025 0
				4026 0
				4027 0
				4028 0
				4029 0
				4030 0
				4031 0
				4032 0
				4033 0
				4034 0
				4035 986839
				4036 0
				4037 0
				4038 0
				4039 0
				4040 0
				4041 0
				4042 0
				4043 0
				4044 0
				4045 0
				4046 0
				4047 0
				4048 0
				4049 0
				4050 989707
				4051 0
				4052 0
				4053 0
				4054 0
				4055 0
				4056 0
				4057 0
				4058 0
				4059 0
				4060 0
				4061 0
				4062 0
				4063 0
				4064 0
				4065 997186
				4066 0
				4067 0
				4068 0
				4069 0
				4070 0
				4071 0
				4072 0
				4073 0
				4074 0
				4075 0
				4076 0
				4077 0
				4078 0
				4079 0
				4080 1000514
				4081 0
				4082 0
				4083 0
				4084 0
				4085 0
				4086 0
				4087 0
				4088 0
				4089 0
				4090 0
				4091 0
				4092 0
				4093 0
				4094 0
				4095 1037740
				4096 0
				4097 0
				4098 0
				4099 0
				4100 0
				4101 0
				4102 0
				4103 0
				4104 0
				4105 0
				4106 0
				4107 0
				4108 0
				4109 0
				4110 1049627
				4111 0
				4112 0
				4113 0
				4114 0
				4115 0
				4116 0
				4117 0
				4118 0
				4119 0
				4120 0
				4121 0
				4122 0
				4123 0
				4124 0
				4125 997225
				4126 0
				4127 0
				4128 0
				4129 0
				4130 0
				4131 0
				4132 0
				4133 0
				4134 0
				4135 0
				4136 0
				4137 0
				4138 0
				4139 0
				4140 1002269
				4141 0
				4142 0
				4143 0
				4144 0
				4145 0
				4146 0
				4147 0
				4148 0
				4149 0
				4150 0
				4151 0
				4152 0
				4153 0
				4154 0
				4155 999788
				4156 0
				4157 0
				4158 0
				4159 0
				4160 0
				4161 0
				4162 0
				4163 0
				4164 0
				4165 0
				4166 0
				4167 0
				4168 0
				4169 0
				4170 1022706
				4171 0
				4172 0
				4173 0
				4174 0
				4175 0
				4176 0
				4177 0
				4178 0
				4179 0
				4180 0
				4181 0
				4182 0
				4183 0
				4184 0
				4185 998391
				4186 0
				4187 0
				4188 0
				4189 0
				4190 0
				4191 0
				4192 0
				4193 0
				4194 0
				4195 0
				4196 0
				4197 0
				4198 0
				4199 0
				4200 1024477
				4201 0
				4202 0
				4203 0
				4204 0
				4205 0
				4206 0
				4207 0
				4208 0
				4209 0
				4210 0
				4211 0
				4212 0
				4213 0
				4214 0
				4215 991950
				4216 0
				4217 0
				4218 0
				4219 0
				4220 0
				4221 0
				4222 0
				4223 0
				4224 0
				4225 0
				4226 0
				4227 0
				4228 0
				4229 0
				4230 990471
				4231 0
				4232 0
				4233 0
				4234 0
				4235 0
				4236 0
				4237 0
				4238 0
				4239 0
				4240 0
				4241 0
				4242 0
				4243 0
				4244 0
				4245 994902
				4246 0
				4247 0
				4248 0
				4249 0
				4250 0
				4251 0
				4252 0
				4253 0
				4254 0
				4255 0
				4256 0
				4257 0
				4258 0
				4259 0
				4260 1007770
				4261 0
				4262 0
				4263 0
				4264 0
				4265 0
				4266 0
				4267 0
				4268 0
				4269 0
				4270 0
				4271 0
				4272 0
				4273 0
				4274 0
				4275 992477
				4276 0
				4277 0
				4278 0
				4279 0
				4280 0
				4281 0
				4282 0
				4283 0
				4284 0
				4285 0
				4286 0
				4287 0
				4288 0
				4289 0
				4290 1028460
				4291 0
				4292 0
				4293 0
				4294 0
				4295 0
				4296 0
				4297 0
				4298 0
				4299 0
				4300 0
				4301 0
				4302 0
				4303 0
				4304 0
				4305 1035212
				4306 0
				4307 0
				4308 0
				4309 0
				4310 0
				4311 0
				4312 0
				4313 0
				4314 0
				4315 0
				4316 0
				4317 0
				4318 0
				4319 0
				4320 993672
				4321 0
				4322 0
				4323 0
				4324 0
				4325 0
				4326 0
				4327 0
				4328 0
				4329 0
				4330 0
				4331 0
				4332 0
				4333 0
				4334 0
				4335 982870
				4336 0
				4337 0
				4338 0
				4339 0
				4340 0
				4341 0
				4342 0
				4343 0
				4344 0
				4345 0
				4346 0
				4347 0
				4348 0
				4349 0
				4350 1008146
				4351 0
				4352 0
				4353 0
				4354 0
				4355 0
				4356 0
				4357 0
				4358 0
				4359 0
				4360 0
				4361 0
				4362 0
				4363 0
				4364 0
				4365 989306
				4366 0
				4367 0
				4368 0
				4369 0
				4370 0
				4371 0
				4372 0
				4373 0
				4374 0
				4375 0
				4376 0
				4377 0
				4378 0
				4379 0
				4380 1005219
				4381 0
				4382 0
				4383 0
				4384 0
				4385 0
				4386 0
				4387 0
				4388 0
				4389 0
				4390 0
				4391 0
				4392 0
				4393 0
				4394 0
				4395 984572
				4396 0
				4397 0
				4398 0
				4399 0
				4400 0
				4401 0
				4402 0
				4403 0
				4404 0
				4405 0
				4406 0
				4407 0
				4408 0
				4409 0
				4410 1032386
				4411 0
				4412 0
				4413 0
				4414 0
				4415 0
				4416 0
				4417 0
				4418 0
				4419 0
				4420 0
				4421 0
				4422 0
				4423 0
				4424 0
				4425 1033826
				4426 0
				4427 0
				4428 0
				4429 0
				4430 0
				4431 0
				4432 0
				4433 0
				4434 0
				4435 0
				4436 0
				4437 0
				4438 0
				4439 0
				4440 1010703
				4441 0
				4442 0
				4443 0
				4444 0
				4445 0
				4446 0
				4447 0
				4448 0
				4449 0
				4450 0
				4451 0
				4452 0
				4453 0
				4454 0
				4455 1018482
				4456 0
				4457 0
				4458 0
				4459 0
				4460 0
				4461 0
				4462 0
				4463 0
				4464 0
				4465 0
				4466 0
				4467 0
				4468 0
				4469 0
				4470 993343
				4471 0
				4472 0
				4473 0
				4474 0
				4475 0
				4476 0
				4477 0
				4478 0
				4479 0
				4480 0
				4481 0
				4482 0
				4483 0
				4484 0
				4485 1024654
				4486 0
				4487 0
				4488 0
				4489 0
				4490 0
				4491 0
				4492 0
				4493 0
				4494 0
				4495 0
				4496 0
				4497 0
				4498 0
				4499 0
				4500 1004141
				4501 0
				4502 0
				4503 0
				4504 0
				4505 0
				4506 0
				4507 0
				4508 0
				4509 0
				4510 0
				4511 0
				4512 0
				4513 0
				4514 0
				4515 999388
				4516 0
				4517 0
				4518 0
				4519 0
				4520 0
				4521 0
				4522 0
				4523 0
				4524 0
				4525 0
				4526 0
				4527 0
				4528 0
				4529 0
				4530 1008471
				4531 0
				4532 0
				4533 0
				4534 0
				4535 0
				4536 0
				4537 0
				4538 0
				4539 0
				4540 0
				4541 0
				4542 0
				4543 0
				4544 0
				4545 990744
				4546 0
				4547 0
				4548 0
				4549 0
				4550 0
				4551 0
				4552 0
				4553 0
				4554 0
				4555 0
				4556 0
				4557 0
				4558 0
				4559 0
				4560 1021966
				4561 0
				4562 0
				4563 0
				4564 0
				4565 0
				4566 0
				4567 0
				4568 0
				4569 0
				4570 0
				4571 0
				4572 0
				4573 0
				4574 0
				4575 991157
				4576 0
				4577 0
				4578 0
				4579 0
				4580 0
				4581 0
				4582 0
				4583 0
				4584 0
				4585 0
				4586 0
				4587 0
				4588 0
				4589 0
				4590 1008253
				4591 0
				4592 0
				4593 0
				4594 0
				4595 0
				4596 0
				4597 0
				4598 0
				4599 0
				4600 0
				4601 0
				4602 0
				4603 0
				4604 0
				4605 1003387
				4606 0
				4607 0
				4608 0
				4609 0
				4610 0
				4611 0
				4612 0
				4613 0
				4614 0
				4615 0
				4616 0
				4617 0
				4618 0
				4619 0
				4620 998221
				4621 0
				4622 0
				4623 0
				4624 0
				4625 0
				4626 0
				4627 0
				4628 0
				4629 0
				4630 0
				4631 0
				4632 0
				4633 0
				4634 0
				4635 990919
				4636 0
				4637 0
				4638 0
				4639 0
				4640 0
				4641 0
				4642 0
				4643 0
				4644 0
				4645 0
				4646 0
				4647 0
				4648 0
				4649 0
				4650 1001088
				4651 0
				4652 0
				4653 0
				4654 0
				4655 0
				4656 0
				4657 0
				4658 0
				4659 0
				4660 0
				4661 0
				4662 0
				4663 0
				4664 0
				4665 1004104
				4666 0
				4667 0
				4668 0
				4669 0
				4670 0
				4671 0
				4672 0
				4673 0
				4674 0
				4675 0
				4676 0
				4677 0
				4678 0
				4679 0
				4680 996657
				4681 0
				4682 0
				4683 0
				4684 0
				4685 0
				4686 0
				4687 0
				4688 0
				4689 0
				4690 0
				4691 0
				4692 0
				4693 0
				4694 0
				4695 985141
				4696 0
				4697 0
				4698 0
				4699 0
				4700 0
				4701 0
				4702 0
				4703 0
				4704 0
				4705 0
				4706 0
				4707 0
				4708 0
				4709 0
				4710 993809
				4711 0
				4712 0
				4713 0
				4714 0
				4715 0
				4716 0
				4717 0
				4718 0
				4719 0
				4720 0
				4721 0
				4722 0
				4723 0
				4724 0
				4725 983244
				4726 0
				4727 0
				4728 0
				4729 0
				4730 0
				4731 0
				4732 0
				4733 0
				4734 0
				4735 0
				4736 0
				4737 0
				4738 0
				4739 0
				4740 990733
				4741 0
				4742 0
				4743 0
				4744 0
				4745 0
				4746 0
				4747 0
				4748 0
				4749 0
				4750 0
				4751 0
				4752 0
				4753 0
				4754 0
				4755 986362
				4756 0
				4757 0
				4758 0
				4759 0
				4760 0
				4761 0
				4762 0
				4763 0
				4764 0
				4765 0
				4766 0
				4767 0
				4768 0
				4769 0
				4770 1006197
				4771 0
				4772 0
				4773 0
				4774 0
				4775 0
				4776 0
				4777 0
				4778 0
				4779 0
				4780 0
				4781 0
				4782 0
				4783 0
				4784 0
				4785 991215
				4786 0
				4787 0
				4788 0
				4789 0
				4790 0
				4791 0
				4792 0
				4793 0
				4794 0
				4795 0
				4796 0
				4797 0
				4798 0
				4799 0
				4800 1021387
				4801 0
				4802 0
				4803 0
				4804 0
				4805 0
				4806 0
				4807 0
				4808 0
				4809 0
				4810 0
				4811 0
				4812 0
				4813 0
				4814 0
				4815 1000731
				4816 0
				4817 0
				4818 0
				4819 0
				4820 0
				4821 0
				4822 0
				4823 0
				4824 0
				4825 0
				4826 0
				4827 0
				4828 0
				4829 0
				4830 1023934
				4831 0
				4832 0
				4833 0
				4834 0
				4835 0
				4836 0
				4837 0
				4838 0
				4839 0
				4840 0
				4841 0
				4842 0
				4843 0
				4844 0
				4845 995923
				4846 0
				4847 0
				4848 0
				4849 0
				4850 0
				4851 0
				4852 0
				4853 0
				4854 0
				4855 0
				4856 0
				4857 0
				4858 0
				4859 0
				4860 996738
				4861 0
				4862 0
				4863 0
				4864 0
				4865 0
				4866 0
				4867 0
				4868 0
				4869 0
				4870 0
				4871 0
				4872 0
				4873 0
				4874 0
				4875 1003350
				4876 0
				4877 0
				4878 0
				4879 0
				4880 0
				4881 0
				4882 0
				4883 0
				4884 0
				4885 0
				4886 0
				4887 0
				4888 0
				4889 0
				4890 1044410
				4891 0
				4892 0
				4893 0
				4894 0
				4895 0
				4896 0
				4897 0
				4898 0
				4899 0
				4900 0
				4901 0
				4902 0
				4903 0
				4904 0
				4905 995564
				4906 0
				4907 0
				4908 0
				4909 0
				4910 0
				4911 0
				4912 0
				4913 0
				4914 0
				4915 0
				4916 0
				4917 0
				4918 0
				4919 0
				4920 994442
				4921 0
				4922 0
				4923 0
				4924 0
				4925 0
				4926 0
				4927 0
				4928 0
				4929 0
				4930 0
				4931 0
				4932 0
				4933 0
				4934 0
				4935 990611
				4936 0
				4937 0
				4938 0
				4939 0
				4940 0
				4941 0
				4942 0
				4943 0
				4944 0
				4945 0
				4946 0
				4947 0
				4948 0
				4949 0
				4950 988629
				4951 0
				4952 0
				4953 0
				4954 0
				4955 0
				4956 0
				4957 0
				4958 0
				4959 0
				4960 0
				4961 0
				4962 0
				4963 0
				4964 0
				4965 997596
				4966 0
				4967 0
				4968 0
				4969 0
				4970 0
				4971 0
				4972 0
				4973 0
				4974 0
				4975 0
				4976 0
				4977 0
				4978 0
				4979 0
				4980 1027685
				4981 0
				4982 0
				4983 0
				4984 0
				4985 0
				4986 0
				4987 0
				4988 0
				4989 0
				4990 0
				4991 0
				4992 0
				4993 0
				4994 0
				4995 1029925
				4996 0
				4997 0
				4998 0
				4999 0
				5000 0
				5001 0
				5002 0
				5003 0
				5004 0
				5005 0
				5006 0
				5007 0
				5008 0
				5009 0
				5010 1003774
				5011 0
				5012 0
				5013 0
				5014 0
				5015 0
				5016 0
				5017 0
				5018 0
				5019 0
				5020 0
				5021 0
				5022 0
				5023 0
				5024 0
				5025 1039117
				5026 0
				5027 0
				5028 0
				5029 0
				5030 0
				5031 0
				5032 0
				5033 0
				5034 0
				5035 0
				5036 0
				5037 0
				5038 0
				5039 0
				5040 1036231
				5041 0
				5042 0
				5043 0
				5044 0
				5045 0
				5046 0
				5047 0
				5048 0
				5049 0
				5050 0
				5051 0
				5052 0
				5053 0
				5054 0
				5055 1030205
				5056 0
				5057 0
				5058 0
				5059 0
				5060 0
				5061 0
				5062 0
				5063 0
				5064 0
				5065 0
				5066 0
				5067 0
				5068 0
				5069 0
				5070 997313
				5071 0
				5072 0
				5073 0
				5074 0
				5075 0
				5076 0
				5077 0
				5078 0
				5079 0
				5080 0
				5081 0
				5082 0
				5083 0
				5084 0
				5085 998437
				5086 0
				5087 0
				5088 0
				5089 0
				5090 0
				5091 0
				5092 0
				5093 0
				5094 0
				5095 0
				5096 0
				5097 0
				5098 0
				5099 0
				5100 982273
				5101 0
				5102 0
				5103 0
				5104 0
				5105 0
				5106 0
				5107 0
				5108 0
				5109 0
				5110 0
				5111 0
				5112 0
				5113 0
				5114 0
				5115 993753
				5116 0
				5117 0
				5118 0
				5119 0
				5120 0
				5121 0
				5122 0
				5123 0
				5124 0
				5125 0
				5126 0
				5127 0
				5128 0
				5129 0
				5130 1001619
				5131 0
				5132 0
				5133 0
				5134 0
				5135 0
				5136 0
				5137 0
				5138 0
				5139 0
				5140 0
				5141 0
				5142 0
				5143 0
				5144 0
				5145 1036037
				5146 0
				5147 0
				5148 0
				5149 0
				5150 0
				5151 0
				5152 0
				5153 0
				5154 0
				5155 0
				5156 0
				5157 0
				5158 0
				5159 0
				5160 993723
				5161 0
				5162 0
				5163 0
				5164 0
				5165 0
				5166 0
				5167 0
				5168 0
				5169 0
				5170 0
				5171 0
				5172 0
				5173 0
				5174 0
				5175 997912
				5176 0
				5177 0
				5178 0
				5179 0
				5180 0
				5181 0
				5182 0
				5183 0
				5184 0
				5185 0
				5186 0
				5187 0
				5188 0
				5189 0
				5190 1008061
				5191 0
				5192 0
				5193 0
				5194 0
				5195 0
				5196 0
				5197 0
				5198 0
				5199 0
				5200 0
				5201 0
				5202 0
				5203 0
				5204 0
				5205 1008838
				5206 0
				5207 0
				5208 0
				5209 0
				5210 0
				5211 0
				5212 0
				5213 0
				5214 0
				5215 0
				5216 0
				5217 0
				5218 0
				5219 0
				5220 996568
				5221 0
				5222 0
				5223 0
				5224 0
				5225 0
				5226 0
				5227 0
				5228 0
				5229 0
				5230 0
				5231 0
				5232 0
				5233 0
				5234 0
				5235 991288
				5236 0
				5237 0
				5238 0
				5239 0
				5240 0
				5241 0
				5242 0
				5243 0
				5244 0
				5245 0
				5246 0
				5247 0
				5248 0
				5249 0
				5250 1016665
				5251 0
				5252 0
				5253 0
				5254 0
				5255 0
				5256 0
				5257 0
				5258 0
				5259 0
				5260 0
				5261 0
				5262 0
				5263 0
				5264 0
				5265 993187
				5266 0
				5267 0
				5268 0
				5269 0
				5270 0
				5271 0
				5272 0
				5273 0
				5274 0
				5275 0
				5276 0
				5277 0
				5278 0
				5279 0
				5280 995831
				5281 0
				5282 0
				5283 0
				5284 0
				5285 0
				5286 0
				5287 0
				5288 0
				5289 0
				5290 0
				5291 0
				5292 0
				5293 0
				5294 0
				5295 1045140
				5296 0
				5297 0
				5298 0
				5299 0
				5300 0
				5301 0
				5302 0
				5303 0
				5304 0
				5305 0
				5306 0
				5307 0
				5308 0
				5309 0
				5310 1011542
				5311 0
				5312 0
				5313 0
				5314 0
				5315 0
				5316 0
				5317 0
				5318 0
				5319 0
				5320 0
				5321 0
				5322 0
				5323 0
				5324 0
				5325 1000659
				5326 0
				5327 0
				5328 0
				5329 0
				5330 0
				5331 0
				5332 0
				5333 0
				5334 0
				5335 0
				5336 0
				5337 0
				5338 0
				5339 0
				5340 1011439
				5341 0
				5342 0
				5343 0
				5344 0
				5345 0
				5346 0
				5347 0
				5348 0
				5349 0
				5350 0
				5351 0
				5352 0
				5353 0
				5354 0
				5355 996426
				5356 0
				5357 0
				5358 0
				5359 0
				5360 0
				5361 0
				5362 0
				5363 0
				5364 0
				5365 0
				5366 0
				5367 0
				5368 0
				5369 0
				5370 1027824
				5371 0
				5372 0
				5373 0
				5374 0
				5375 0
				5376 0
				5377 0
				5378 0
				5379 0
				5380 0
				5381 0
				5382 0
				5383 0
				5384 0
				5385 1011677
				5386 0
				5387 0
				5388 0
				5389 0
				5390 0
				5391 0
				5392 0
				5393 0
				5394 0
				5395 0
				5396 0
				5397 0
				5398 0
				5399 0
				5400 999889
				5401 0
				5402 0
				5403 0
				5404 0
				5405 0
				5406 0
				5407 0
				5408 0
				5409 0
				5410 0
				5411 0
				5412 0
				5413 0
				5414 0
				5415 990303
				5416 0
				5417 0
				5418 0
				5419 0
				5420 0
				5421 0
				5422 0
				5423 0
				5424 0
				5425 0
				5426 0
				5427 0
				5428 0
				5429 0
				5430 1003449
				5431 0
				5432 0
				5433 0
				5434 0
				5435 0
				5436 0
				5437 0
				5438 0
				5439 0
				5440 0
				5441 0
				5442 0
				5443 0
				5444 0
				5445 999923
				5446 0
				5447 0
				5448 0
				5449 0
				5450 0
				5451 0
				5452 0
				5453 0
				5454 0
				5455 0
				5456 0
				5457 0
				5458 0
				5459 0
				5460 1007028
				5461 0
				5462 0
				5463 0
				5464 0
				5465 0
				5466 0
				5467 0
				5468 0
				5469 0
				5470 0
				5471 0
				5472 0
				5473 0
				5474 0
				5475 1019340
				5476 0
				5477 0
				5478 0
				5479 0
				5480 0
				5481 0
				5482 0
				5483 0
				5484 0
				5485 0
				5486 0
				5487 0
				5488 0
				5489 0
				5490 1018127
				5491 0
				5492 0
				5493 0
				5494 0
				5495 0
				5496 0
				5497 0
				5498 0
				5499 0
				5500 0
				5501 0
				5502 0
				5503 0
				5504 0
				5505 986269
				5506 0
				5507 0
				5508 0
				5509 0
				5510 0
				5511 0
				5512 0
				5513 0
				5514 0
				5515 0
				5516 0
				5517 0
				5518 0
				5519 0
				5520 1019726
				5521 0
				5522 0
				5523 0
				5524 0
				5525 0
				5526 0
				5527 0
				5528 0
				5529 0
				5530 0
				5531 0
				5532 0
				5533 0
				5534 0
				5535 1004759
				5536 0
				5537 0
				5538 0
				5539 0
				5540 0
				5541 0
				5542 0
				5543 0
				5544 0
				5545 0
				5546 0
				5547 0
				5548 0
				5549 0
				5550 991098
				5551 0
				5552 0
				5553 0
				5554 0
				5555 0
				5556 0
				5557 0
				5558 0
				5559 0
				5560 0
				5561 0
				5562 0
				5563 0
				5564 0
				5565 1056125
				5566 0
				5567 0
				5568 0
				5569 0
				5570 0
				5571 0
				5572 0
				5573 0
				5574 0
				5575 0
				5576 0
				5577 0
				5578 0
				5579 0
				5580 1010080
				5581 0
				5582 0
				5583 0
				5584 0
				5585 0
				5586 0
				5587 0
				5588 0
				5589 0
				5590 0
				5591 0
				5592 0
				5593 0
				5594 0
				5595 993405
				5596 0
				5597 0
				5598 0
				5599 0
				5600 0
				5601 0
				5602 0
				5603 0
				5604 0
				5605 0
				5606 0
				5607 0
				5608 0
				5609 0
				5610 999701
				5611 0
				5612 0
				5613 0
				5614 0
				5615 0
				5616 0
				5617 0
				5618 0
				5619 0
				5620 0
				5621 0
				5622 0
				5623 0
				5624 0
				5625 1007422
				5626 0
				5627 0
				5628 0
				5629 0
				5630 0
				5631 0
				5632 0
				5633 0
				5634 0
				5635 0
				5636 0
				5637 0
				5638 0
				5639 0
				5640 995943
				5641 0
				5642 0
				5643 0
				5644 0
				5645 0
				5646 0
				5647 0
				5648 0
				5649 0
				5650 0
				5651 0
				5652 0
				5653 0
				5654 0
				5655 991976
				5656 0
				5657 0
				5658 0
				5659 0
				5660 0
				5661 0
				5662 0
				5663 0
				5664 0
				5665 0
				5666 0
				5667 0
				5668 0
				5669 0
				5670 1006202
				5671 0
				5672 0
				5673 0
				5674 0
				5675 0
				5676 0
				5677 0
				5678 0
				5679 0
				5680 0
				5681 0
				5682 0
				5683 0
				5684 0
				5685 995499
				5686 0
				5687 0
				5688 0
				5689 0
				5690 0
				5691 0
				5692 0
				5693 0
				5694 0
				5695 0
				5696 0
				5697 0
				5698 0
				5699 0
				5700 989433
				5701 0
				5702 0
				5703 0
				5704 0
				5705 0
				5706 0
				5707 0
				5708 0
				5709 0
				5710 0
				5711 0
				5712 0
				5713 0
				5714 0
				5715 1000803
				5716 0
				5717 0
				5718 0
				5719 0
				5720 0
				5721 0
				5722 0
				5723 0
				5724 0
				5725 0
				5726 0
				5727 0
				5728 0
				5729 0
				5730 985830
				5731 0
				5732 0
				5733 0
				5734 0
				5735 0
				5736 0
				5737 0
				5738 0
				5739 0
				5740 0
				5741 0
				5742 0
				5743 0
				5744 0
				5745 1024367
				5746 0
				5747 0
				5748 0
				5749 0
				5750 0
				5751 0
				5752 0
				5753 0
				5754 0
				5755 0
				5756 0
				5757 0
				5758 0
				5759 0
				5760 998670
				5761 0
				5762 0
				5763 0
				5764 0
				5765 0
				5766 0
				5767 0
				5768 0
				5769 0
				5770 0
				5771 0
				5772 0
				5773 0
				5774 0
				5775 986369
				5776 0
				5777 0
				5778 0
				5779 0
				5780 0
				5781 0
				5782 0
				5783 0
				5784 0
				5785 0
				5786 0
				5787 0
				5788 0
				5789 0
				5790 992758
				5791 0
				5792 0
				5793 0
				5794 0
				5795 0
				5796 0
				5797 0
				5798 0
				5799 0
				5800 0
				5801 0
				5802 0
				5803 0
				5804 0
				5805 1001854
				5806 0
				5807 0
				5808 0
				5809 0
				5810 0
				5811 0
				5812 0
				5813 0
				5814 0
				5815 0
				5816 0
				5817 0
				5818 0
				5819 0
				5820 1013579
				5821 0
				5822 0
				5823 0
				5824 0
				5825 0
				5826 0
				5827 0
				5828 0
				5829 0
				5830 0
				5831 0
				5832 0
				5833 0
				5834 0
				5835 1040709
				5836 0
				5837 0
				5838 0
				5839 0
				5840 0
				5841 0
				5842 0
				5843 0
				5844 0
				5845 0
				5846 0
				5847 0
				5848 0
				5849 0
				5850 1010479
				5851 0
				5852 0
				5853 0
				5854 0
				5855 0
				5856 0
				5857 0
				5858 0
				5859 0
				5860 0
				5861 0
				5862 0
				5863 0
				5864 0
				5865 996144
				5866 0
				5867 0
				5868 0
				5869 0
				5870 0
				5871 0
				5872 0
				5873 0
				5874 0
				5875 0
				5876 0
				5877 0
				5878 0
				5879 0
				5880 1007818
				5881 0
				5882 0
				5883 0
				5884 0
				5885 0
				5886 0
				5887 0
				5888 0
				5889 0
				5890 0
				5891 0
				5892 0
				5893 0
				5894 0
				5895 1007009
				5896 0
				5897 0
				5898 0
				5899 0
				5900 0
				5901 0
				5902 0
				5903 0
				5904 0
				5905 0
				5906 0
				5907 0
				5908 0
				5909 0
				5910 997846
				5911 0
				5912 0
				5913 0
				5914 0
				5915 0
				5916 0
				5917 0
				5918 0
				5919 0
				5920 0
				5921 0
				5922 0
				5923 0
				5924 0
				5925 990825
				5926 0
				5927 0
				5928 0
				5929 0
				5930 0
				5931 0
				5932 0
				5933 0
				5934 0
				5935 0
				5936 0
				5937 0
				5938 0
				5939 0
				5940 993328
				5941 0
				5942 0
				5943 0
				5944 0
				5945 0
				5946 0
				5947 0
				5948 0
				5949 0
				5950 0
				5951 0
				5952 0
				5953 0
				5954 0
				5955 1003532
				5956 0
				5957 0
				5958 0
				5959 0
				5960 0
				5961 0
				5962 0
				5963 0
				5964 0
				5965 0
				5966 0
				5967 0
				5968 0
				5969 0
				5970 1041483
				5971 0
				5972 0
				5973 0
				5974 0
				5975 0
				5976 0
				5977 0
				5978 0
				5979 0
				5980 0
				5981 0
				5982 0
				5983 0
				5984 0
				5985 979311
				5986 0
				5987 0
				5988 0
				5989 0
				5990 0
				5991 0
				5992 0
				5993 0
				5994 0
				5995 0
				5996 0
				5997 0
				5998 0
				5999 0
				6000 999164
				6001 0
				6002 0
				6003 0
				6004 0
				6005 0
				6006 0
				6007 0
				6008 0
				6009 0
				6010 0
				6011 0
				6012 0
				6013 0
				6014 0
				6015 995154
				6016 0
				6017 0
				6018 0
				6019 0
				6020 0
				6021 0
				6022 0
				6023 0
				6024 0
				6025 0
				6026 0
				6027 0
				6028 0
				6029 0
				6030 993539
				6031 0
				6032 0
				6033 0
				6034 0
				6035 0
				6036 0
				6037 0
				6038 0
				6039 0
				6040 0
				6041 0
				6042 0
				6043 0
				6044 0
				6045 1004702
				6046 0
				6047 0
				6048 0
				6049 0
				6050 0
				6051 0
				6052 0
				6053 0
				6054 0
				6055 0
				6056 0
				6057 0
				6058 0
				6059 0
				6060 1037315
				6061 0
				6062 0
				6063 0
				6064 0
				6065 0
				6066 0
				6067 0
				6068 0
				6069 0
				6070 0
				6071 0
				6072 0
				6073 0
				6074 0
				6075 990480
				6076 0
				6077 0
				6078 0
				6079 0
				6080 0
				6081 0
				6082 0
				6083 0
				6084 0
				6085 0
				6086 0
				6087 0
				6088 0
				6089 0
				6090 987853
				6091 0
				6092 0
				6093 0
				6094 0
				6095 0
				6096 0
				6097 0
				6098 0
				6099 0
				6100 0
				6101 0
				6102 0
				6103 0
				6104 0
				6105 994886
				6106 0
				6107 0
				6108 0
				6109 0
				6110 0
				6111 0
				6112 0
				6113 0
				6114 0
				6115 0
				6116 0
				6117 0
				6118 0
				6119 0
				6120 1026467
				6121 0
				6122 0
				6123 0
				6124 0
				6125 0
				6126 0
				6127 0
				6128 0
				6129 0
				6130 0
				6131 0
				6132 0
				6133 0
				6134 0
				6135 992957
				6136 0
				6137 0
				6138 0
				6139 0
				6140 0
				6141 0
				6142 0
				6143 0
				6144 0
				6145 0
				6146 0
				6147 0
				6148 0
				6149 0
				6150 995975
				6151 0
				6152 0
				6153 0
				6154 0
				6155 0
				6156 0
				6157 0
				6158 0
				6159 0
				6160 0
				6161 0
				6162 0
				6163 0
				6164 0
				6165 1025506
				6166 0
				6167 0
				6168 0
				6169 0
				6170 0
				6171 0
				6172 0
				6173 0
				6174 0
				6175 0
				6176 0
				6177 0
				6178 0
				6179 0
				6180 997148
				6181 0
				6182 0
				6183 0
				6184 0
				6185 0
				6186 0
				6187 0
				6188 0
				6189 0
				6190 0
				6191 0
				6192 0
				6193 0
				6194 0
				6195 987558
				6196 0
				6197 0
				6198 0
				6199 0
				6200 0
				6201 0
				6202 0
				6203 0
				6204 0
				6205 0
				6206 0
				6207 0
				6208 0
				6209 0
				6210 1020379
				6211 0
				6212 0
				6213 0
				6214 0
				6215 0
				6216 0
				6217 0
				6218 0
				6219 0
				6220 0
				6221 0
				6222 0
				6223 0
				6224 0
				6225 1018606
				6226 0
				6227 0
				6228 0
				6229 0
				6230 0
				6231 0
				6232 0
				6233 0
				6234 0
				6235 0
				6236 0
				6237 0
				6238 0
				6239 0
				6240 992161
				6241 0
				6242 0
				6243 0
				6244 0
				6245 0
				6246 0
				6247 0
				6248 0
				6249 0
				6250 0
				6251 0
				6252 0
				6253 0
				6254 0
				6255 994263
				6256 0
				6257 0
				6258 0
				6259 0
				6260 0
				6261 0
				6262 0
				6263 0
				6264 0
				6265 0
				6266 0
				6267 0
				6268 0
				6269 0
				6270 1002503
				6271 0
				6272 0
				6273 0
				6274 0
				6275 0
				6276 0
				6277 0
				6278 0
				6279 0
				6280 0
				6281 0
				6282 0
				6283 0
				6284 0
				6285 995301
				6286 0
				6287 0
				6288 0
				6289 0
				6290 0
				6291 0
				6292 0
				6293 0
				6294 0
				6295 0
				6296 0
				6297 0
				6298 0
				6299 0
				6300 980546
				6301 0
				6302 0
				6303 0
				6304 0
				6305 0
				6306 0
				6307 0
				6308 0
				6309 0
				6310 0
				6311 0
				6312 0
				6313 0
				6314 0
				6315 1022444
				6316 0
				6317 0
				6318 0
				6319 0
				6320 0
				6321 0
				6322 0
				6323 0
				6324 0
				6325 0
				6326 0
				6327 0
				6328 0
				6329 0
				6330 1031089
				6331 0
				6332 0
				6333 0
				6334 0
				6335 0
				6336 0
				6337 0
				6338 0
				6339 0
				6340 0
				6341 0
				6342 0
				6343 0
				6344 0
				6345 1030654
				6346 0
				6347 0
				6348 0
				6349 0
				6350 0
				6351 0
				6352 0
				6353 0
				6354 0
				6355 0
				6356 0
				6357 0
				6358 0
				6359 0
				6360 996364
				6361 0
				6362 0
				6363 0
				6364 0
				6365 0
				6366 0
				6367 0
				6368 0
				6369 0
				6370 0
				6371 0
				6372 0
				6373 0
				6374 0
				6375 1011300
				6376 0
				6377 0
				6378 0
				6379 0
				6380 0
				6381 0
				6382 0
				6383 0
				6384 0
				6385 0
				6386 0
				6387 0
				6388 0
				6389 0
				6390 1007444
				6391 0
				6392 0
				6393 0
				6394 0
				6395 0
				6396 0
				6397 0
				6398 0
				6399 0
				6400 0
				6401 0
				6402 0
				6403 0
				6404 0
				6405 1045013
				6406 0
				6407 0
				6408 0
				6409 0
				6410 0
				6411 0
				6412 0
				6413 0
				6414 0
				6415 0
				6416 0
				6417 0
				6418 0
				6419 0
				6420 990293
				6421 0
				6422 0
				6423 0
				6424 0
				6425 0
				6426 0
				6427 0
				6428 0
				6429 0
				6430 0
				6431 0
				6432 0
				6433 0
				6434 0
				6435 1020335
				6436 0
				6437 0
				6438 0
				6439 0
				6440 0
				6441 0
				6442 0
				6443 0
				6444 0
				6445 0
				6446 0
				6447 0
				6448 0
				6449 0
				6450 1029399
				6451 0
				6452 0
				6453 0
				6454 0
				6455 0
				6456 0
				6457 0
				6458 0
				6459 0
				6460 0
				6461 0
				6462 0
				6463 0
				6464 0
				6465 1007223
				6466 0
				6467 0
				6468 0
				6469 0
				6470 0
				6471 0
				6472 0
				6473 0
				6474 0
				6475 0
				6476 0
				6477 0
				6478 0
				6479 0
				6480 1005702
				6481 0
				6482 0
				6483 0
				6484 0
				6485 0
				6486 0
				6487 0
				6488 0
				6489 0
				6490 0
				6491 0
				6492 0
				6493 0
				6494 0
				6495 987439
				6496 0
				6497 0
				6498 0
				6499 0
				6500 0
				6501 0
				6502 0
				6503 0
				6504 0
				6505 0
				6506 0
				6507 0
				6508 0
				6509 0
				6510 989962
				6511 0
				6512 0
				6513 0
				6514 0
				6515 0
				6516 0
				6517 0
				6518 0
				6519 0
				6520 0
				6521 0
				6522 0
				6523 0
				6524 0
				6525 1030918
				6526 0
				6527 0
				6528 0
				6529 0
				6530 0
				6531 0
				6532 0
				6533 0
				6534 0
				6535 0
				6536 0
				6537 0
				6538 0
				6539 0
				6540 999458
				6541 0
				6542 0
				6543 0
				6544 0
				6545 0
				6546 0
				6547 0
				6548 0
				6549 0
				6550 0
				6551 0
				6552 0
				6553 0
				6554 0
				6555 1008880
				6556 0
				6557 0
				6558 0
				6559 0
				6560 0
				6561 0
				6562 0
				6563 0
				6564 0
				6565 0
				6566 0
				6567 0
				6568 0
				6569 0
				6570 990764
				6571 0
				6572 0
				6573 0
				6574 0
				6575 0
				6576 0
				6577 0
				6578 0
				6579 0
				6580 0
				6581 0
				6582 0
				6583 0
				6584 0
				6585 999626
				6586 0
				6587 0
				6588 0
				6589 0
				6590 0
				6591 0
				6592 0
				6593 0
				6594 0
				6595 0
				6596 0
				6597 0
				6598 0
				6599 0
				6600 1014446
				6601 0
				6602 0
				6603 0
				6604 0
				6605 0
				6606 0
				6607 0
				6608 0
				6609 0
				6610 0
				6611 0
				6612 0
				6613 0
				6614 0
				6615 1045261
				6616 0
				6617 0
				6618 0
				6619 0
				6620 0
				6621 0
				6622 0
				6623 0
				6624 0
				6625 0
				6626 0
				6627 0
				6628 0
				6629 0
				6630 1024177
				6631 0
				6632 0
				6633 0
				6634 0
				6635 0
				6636 0
				6637 0
				6638 0
				6639 0
				6640 0
				6641 0
				6642 0
				6643 0
				6644 0
				6645 1002589
				6646 0
				6647 0
				6648 0
				6649 0
				6650 0
				6651 0
				6652 0
				6653 0
				6654 0
				6655 0
				6656 0
				6657 0
				6658 0
				6659 0
				6660 993598
				6661 0
				6662 0
				6663 0
				6664 0
				6665 0
				6666 0
				6667 0
				6668 0
				6669 0
				6670 0
				6671 0
				6672 0
				6673 0
				6674 0
				6675 1003659
				6676 0
				6677 0
				6678 0
				6679 0
				6680 0
				6681 0
				6682 0
				6683 0
				6684 0
				6685 0
				6686 0
				6687 0
				6688 0
				6689 0
				6690 1008622
				6691 0
				6692 0
				6693 0
				6694 0
				6695 0
				6696 0
				6697 0
				6698 0
				6699 0
				6700 0
				6701 0
				6702 0
				6703 0
				6704 0
				6705 993032
				6706 0
				6707 0
				6708 0
				6709 0
				6710 0
				6711 0
				6712 0
				6713 0
				6714 0
				6715 0
				6716 0
				6717 0
				6718 0
				6719 0
				6720 985985
				6721 0
				6722 0
				6723 0
				6724 0
				6725 0
				6726 0
				6727 0
				6728 0
				6729 0
				6730 0
				6731 0
				6732 0
				6733 0
				6734 0
				6735 989901
				6736 0
				6737 0
				6738 0
				6739 0
				6740 0
				6741 0
				6742 0
				6743 0
				6744 0
				6745 0
				6746 0
				6747 0
				6748 0
				6749 0
				6750 995532
				6751 0
				6752 0
				6753 0
				6754 0
				6755 0
				6756 0
				6757 0
				6758 0
				6759 0
				6760 0
				6761 0
				6762 0
				6763 0
				6764 0
				6765 1008118
				6766 0
				6767 0
				6768 0
				6769 0
				6770 0
				6771 0
				6772 0
				6773 0
				6774 0
				6775 0
				6776 0
				6777 0
				6778 0
				6779 0
				6780 993051
				6781 0
				6782 0
				6783 0
				6784 0
				6785 0
				6786 0
				6787 0
				6788 0
				6789 0
				6790 0
				6791 0
				6792 0
				6793 0
				6794 0
				6795 990234
				6796 0
				6797 0
				6798 0
				6799 0
				6800 0
				6801 0
				6802 0
				6803 0
				6804 0
				6805 0
				6806 0
				6807 0
				6808 0
				6809 0
				6810 1004972
				6811 0
				6812 0
				6813 0
				6814 0
				6815 0
				6816 0
				6817 0
				6818 0
				6819 0
				6820 0
				6821 0
				6822 0
				6823 0
				6824 0
				6825 983794
				6826 0
				6827 0
				6828 0
				6829 0
				6830 0
				6831 0
				6832 0
				6833 0
				6834 0
				6835 0
				6836 0
				6837 0
				6838 0
				6839 0
				6840 999498
				6841 0
				6842 0
				6843 0
				6844 0
				6845 0
				6846 0
				6847 0
				6848 0
				6849 0
				6850 0
				6851 0
				6852 0
				6853 0
				6854 0
				6855 1041495
				6856 0
				6857 0
				6858 0
				6859 0
				6860 0
				6861 0
				6862 0
				6863 0
				6864 0
				6865 0
				6866 0
				6867 0
				6868 0
				6869 0
				6870 991333
				6871 0
				6872 0
				6873 0
				6874 0
				6875 0
				6876 0
				6877 0
				6878 0
				6879 0
				6880 0
				6881 0
				6882 0
				6883 0
				6884 0
				6885 1010828
				6886 0
				6887 0
				6888 0
				6889 0
				6890 0
				6891 0
				6892 0
				6893 0
				6894 0
				6895 0
				6896 0
				6897 0
				6898 0
				6899 0
				6900 1018249
				6901 0
				6902 0
				6903 0
				6904 0
				6905 0
				6906 0
				6907 0
				6908 0
				6909 0
				6910 0
				6911 0
				6912 0
				6913 0
				6914 0
				6915 1030975
				6916 0
				6917 0
				6918 0
				6919 0
				6920 0
				6921 0
				6922 0
				6923 0
				6924 0
				6925 0
				6926 0
				6927 0
				6928 0
				6929 0
				6930 994837
				6931 0
				6932 0
				6933 0
				6934 0
				6935 0
				6936 0
				6937 0
				6938 0
				6939 0
				6940 0
				6941 0
				6942 0
				6943 0
				6944 0
				6945 990289
				6946 0
				6947 0
				6948 0
				6949 0
				6950 0
				6951 0
				6952 0
				6953 0
				6954 0
				6955 0
				6956 0
				6957 0
				6958 0
				6959 0
				6960 988648
				6961 0
				6962 0
				6963 0
				6964 0
				6965 0
				6966 0
				6967 0
				6968 0
				6969 0
				6970 0
				6971 0
				6972 0
				6973 0
				6974 0
				6975 995040
				6976 0
				6977 0
				6978 0
				6979 0
				6980 0
				6981 0
				6982 0
				6983 0
				6984 0
				6985 0
				6986 0
				6987 0
				6988 0
				6989 0
				6990 1022526
				6991 0
				6992 0
				6993 0
				6994 0
				6995 0
				6996 0
				6997 0
				6998 0
				6999 0
				7000 0
				7001 0
				7002 0
				7003 0
				7004 0
				7005 984733
				7006 0
				7007 0
				7008 0
				7009 0
				7010 0
				7011 0
				7012 0
				7013 0
				7014 0
				7015 0
				7016 0
				7017 0
				7018 0
				7019 0
				7020 1006858
				7021 0
				7022 0
				7023 0
				7024 0
				7025 0
				7026 0
				7027 0
				7028 0
				7029 0
				7030 0
				7031 0
				7032 0
				7033 0
				7034 0
				7035 992992
				7036 0
				7037 0
				7038 0
				7039 0
				7040 0
				7041 0
				7042 0
				7043 0
				7044 0
				7045 0
				7046 0
				7047 0
				7048 0
				7049 0
				7050 994835
				7051 0
				7052 0
				7053 0
				7054 0
				7055 0
				7056 0
				7057 0
				7058 0
				7059 0
				7060 0
				7061 0
				7062 0
				7063 0
				7064 0
				7065 1002827
				7066 0
				7067 0
				7068 0
				7069 0
				7070 0
				7071 0
				7072 0
				7073 0
				7074 0
				7075 0
				7076 0
				7077 0
				7078 0
				7079 0
				7080 1009008
				7081 0
				7082 0
				7083 0
				7084 0
				7085 0
				7086 0
				7087 0
				7088 0
				7089 0
				7090 0
				7091 0
				7092 0
				7093 0
				7094 0
				7095 999847
				7096 0
				7097 0
				7098 0
				7099 0
				7100 0
				7101 0
				7102 0
				7103 0
				7104 0
				7105 0
				7106 0
				7107 0
				7108 0
				7109 0
				7110 1013592
				7111 0
				7112 0
				7113 0
				7114 0
				7115 0
				7116 0
				7117 0
				7118 0
				7119 0
				7120 0
				7121 0
				7122 0
				7123 0
				7124 0
				7125 1010690
				7126 0
				7127 0
				7128 0
				7129 0
				7130 0
				7131 0
				7132 0
				7133 0
				7134 0
				7135 0
				7136 0
				7137 0
				7138 0
				7139 0
				7140 1017044
				7141 0
				7142 0
				7143 0
				7144 0
				7145 0
				7146 0
				7147 0
				7148 0
				7149 0
				7150 0
				7151 0
				7152 0
				7153 0
				7154 0
				7155 1003623
				7156 0
				7157 0
				7158 0
				7159 0
				7160 0
				7161 0
				7162 0
				7163 0
				7164 0
				7165 0
				7166 0
				7167 0
				7168 0
				7169 0
				7170 997907
				7171 0
				7172 0
				7173 0
				7174 0
				7175 0
				7176 0
				7177 0
				7178 0
				7179 0
				7180 0
				7181 0
				7182 0
				7183 0
				7184 0
				7185 1006406
				7186 0
				7187 0
				7188 0
				7189 0
				7190 0
				7191 0
				7192 0
				7193 0
				7194 0
				7195 0
				7196 0
				7197 0
				7198 0
				7199 0
				7200 1009760
				7201 0
				7202 0
				7203 0
				7204 0
				7205 0
				7206 0
				7207 0
				7208 0
				7209 0
				7210 0
				7211 0
				7212 0
				7213 0
				7214 0
				7215 1000558
				7216 0
				7217 0
				7218 0
				7219 0
				7220 0
				7221 0
				7222 0
				7223 0
				7224 0
				7225 0
				7226 0
				7227 0
				7228 0
				7229 0
				7230 990636
				7231 0
				7232 0
				7233 0
				7234 0
				7235 0
				7236 0
				7237 0
				7238 0
				7239 0
				7240 0
				7241 0
				7242 0
				7243 0
				7244 0
				7245 990541
				7246 0
				7247 0
				7248 0
				7249 0
				7250 0
				7251 0
				7252 0
				7253 0
				7254 0
				7255 0
				7256 0
				7257 0
				7258 0
				7259 0
				7260 991942
				7261 0
				7262 0
				7263 0
				7264 0
				7265 0
				7266 0
				7267 0
				7268 0
				7269 0
				7270 0
				7271 0
				7272 0
				7273 0
				7274 0
				7275 995297
				7276 0
				7277 0
				7278 0
				7279 0
				7280 0
				7281 0
				7282 0
				7283 0
				7284 0
				7285 0
				7286 0
				7287 0
				7288 0
				7289 0
				7290 1009003
				7291 0
				7292 0
				7293 0
				7294 0
				7295 0
				7296 0
				7297 0
				7298 0
				7299 0
				7300 0
				7301 0
				7302 0
				7303 0
				7304 0
				7305 1004987
				7306 0
				7307 0
				7308 0
				7309 0
				7310 0
				7311 0
				7312 0
				7313 0
				7314 0
				7315 0
				7316 0
				7317 0
				7318 0
				7319 0
				7320 998842
				7321 0
				7322 0
				7323 0
				7324 0
				7325 0
				7326 0
				7327 0
				7328 0
				7329 0
				7330 0
				7331 0
				7332 0
				7333 0
				7334 0
				7335 1031180
				7336 0
				7337 0
				7338 0
				7339 0
				7340 0
				7341 0
				7342 0
				7343 0
				7344 0
				7345 0
				7346 0
				7347 0
				7348 0
				7349 0
				7350 1024060
				7351 0
				7352 0
				7353 0
				7354 0
				7355 0
				7356 0
				7357 0
				7358 0
				7359 0
				7360 0
				7361 0
				7362 0
				7363 0
				7364 0
				7365 998245
				7366 0
				7367 0
				7368 0
				7369 0
				7370 0
				7371 0
				7372 0
				7373 0
				7374 0
				7375 0
				7376 0
				7377 0
				7378 0
				7379 0
				7380 999458
				7381 0
				7382 0
				7383 0
				7384 0
				7385 0
				7386 0
				7387 0
				7388 0
				7389 0
				7390 0
				7391 0
				7392 0
				7393 0
				7394 0
				7395 999512
				7396 0
				7397 0
				7398 0
				7399 0
				7400 0
				7401 0
				7402 0
				7403 0
				7404 0
				7405 0
				7406 0
				7407 0
				7408 0
				7409 0
				7410 993275
				7411 0
				7412 0
				7413 0
				7414 0
				7415 0
				7416 0
				7417 0
				7418 0
				7419 0
				7420 0
				7421 0
				7422 0
				7423 0
				7424 0
				7425 991575
				7426 0
				7427 0
				7428 0
				7429 0
				7430 0
				7431 0
				7432 0
				7433 0
				7434 0
				7435 0
				7436 0
				7437 0
				7438 0
				7439 0
				7440 1006620
				7441 0
				7442 0
				7443 0
				7444 0
				7445 0
				7446 0
				7447 0
				7448 0
				7449 0
				7450 0
				7451 0
				7452 0
				7453 0
				7454 0
				7455 994243
				7456 0
				7457 0
				7458 0
				7459 0
				7460 0
				7461 0
				7462 0
				7463 0
				7464 0
				7465 0
				7466 0
				7467 0
				7468 0
				7469 0
				7470 1035680
				7471 0
				7472 0
				7473 0
				7474 0
				7475 0
				7476 0
				7477 0
				7478 0
				7479 0
				7480 0
				7481 0
				7482 0
				7483 0
				7484 0
				7485 995276
				7486 0
				7487 0
				7488 0
				7489 0
				7490 0
				7491 0
				7492 0
				7493 0
				7494 0
				7495 0
				7496 0
				7497 0
				7498 0
				7499 0
				7500 997792
				7501 0
				7502 0
				7503 0
				7504 0
				7505 0
				7506 0
				7507 0
				7508 0
				7509 0
				7510 0
				7511 0
				7512 0
				7513 0
				7514 0
				7515 1029270
				7516 0
				7517 0
				7518 0
				7519 0
				7520 0
				7521 0
				7522 0
				7523 0
				7524 0
				7525 0
				7526 0
				7527 0
				7528 0
				7529 0
				7530 1003414
				7531 0
				7532 0
				7533 0
				7534 0
				7535 0
				7536 0
				7537 0
				7538 0
				7539 0
				7540 0
				7541 0
				7542 0
				7543 0
				7544 0
				7545 1056363
				7546 0
				7547 0
				7548 0
				7549 0
				7550 0
				7551 0
				7552 0
				7553 0
				7554 0
				7555 0
				7556 0
				7557 0
				7558 0
				7559 0
				7560 1031681
				7561 0
				7562 0
				7563 0
				7564 0
				7565 0
				7566 0
				7567 0
				7568 0
				7569 0
				7570 0
				7571 0
				7572 0
				7573 0
				7574 0
				7575 1005906
				7576 0
				7577 0
				7578 0
				7579 0
				7580 0
				7581 0
				7582 0
				7583 0
				7584 0
				7585 0
				7586 0
				7587 0
				7588 0
				7589 0
				7590 1036314
				7591 0
				7592 0
				7593 0
				7594 0
				7595 0
				7596 0
				7597 0
				7598 0
				7599 0
				7600 0
				7601 0
				7602 0
				7603 0
				7604 0
				7605 982277
				7606 0
				7607 0
				7608 0
				7609 0
				7610 0
				7611 0
				7612 0
				7613 0
				7614 0
				7615 0
				7616 0
				7617 0
				7618 0
				7619 0
				7620 997634
				7621 0
				7622 0
				7623 0
				7624 0
				7625 0
				7626 0
				7627 0
				7628 0
				7629 0
				7630 0
				7631 0
				7632 0
				7633 0
				7634 0
				7635 1010646
				7636 0
				7637 0
				7638 0
				7639 0
				7640 0
				7641 0
				7642 0
				7643 0
				7644 0
				7645 0
				7646 0
				7647 0
				7648 0
				7649 0
				7650 1002183
				7651 0
				7652 0
				7653 0
				7654 0
				7655 0
				7656 0
				7657 0
				7658 0
				7659 0
				7660 0
				7661 0
				7662 0
				7663 0
				7664 0
				7665 992818
				7666 0
				7667 0
				7668 0
				7669 0
				7670 0
				7671 0
				7672 0
				7673 0
				7674 0
				7675 0
				7676 0
				7677 0
				7678 0
				7679 0
				7680 988597
				7681 0
				7682 0
				7683 0
				7684 0
				7685 0
				7686 0
				7687 0
				7688 0
				7689 0
				7690 0
				7691 0
				7692 0
				7693 0
				7694 0
				7695 997996
				7696 0
				7697 0
				7698 0
				7699 0
				7700 0
				7701 0
				7702 0
				7703 0
				7704 0
				7705 0
				7706 0
				7707 0
				7708 0
				7709 0
				7710 1013185
				7711 0
				7712 0
				7713 0
				7714 0
				7715 0
				7716 0
				7717 0
				7718 0
				7719 0
				7720 0
				7721 0
				7722 0
				7723 0
				7724 0
				7725 988649
				7726 0
				7727 0
				7728 0
				7729 0
				7730 0
				7731 0
				7732 0
				7733 0
				7734 0
				7735 0
				7736 0
				7737 0
				7738 0
				7739 0
				7740 990705
				7741 0
				7742 0
				7743 0
				7744 0
				7745 0
				7746 0
				7747 0
				7748 0
				7749 0
				7750 0
				7751 0
				7752 0
				7753 0
				7754 0
				7755 1005710
				7756 0
				7757 0
				7758 0
				7759 0
				7760 0
				7761 0
				7762 0
				7763 0
				7764 0
				7765 0
				7766 0
				7767 0
				7768 0
				7769 0
				7770 987515
				7771 0
				7772 0
				7773 0
				7774 0
				7775 0
				7776 0
				7777 0
				7778 0
				7779 0
				7780 0
				7781 0
				7782 0
				7783 0
				7784 0
				7785 1002785
				7786 0
				7787 0
				7788 0
				7789 0
				7790 0
				7791 0
				7792 0
				7793 0
				7794 0
				7795 0
				7796 0
				7797 0
				7798 0
				7799 0
				7800 1003514
				7801 0
				7802 0
				7803 0
				7804 0
				7805 0
				7806 0
				7807 0
				7808 0
				7809 0
				7810 0
				7811 0
				7812 0
				7813 0
				7814 0
				7815 986942
				7816 0
				7817 0
				7818 0
				7819 0
				7820 0
				7821 0
				7822 0
				7823 0
				7824 0
				7825 0
				7826 0
				7827 0
				7828 0
				7829 0
				7830 1013329
				7831 0
				7832 0
				7833 0
				7834 0
				7835 0
				7836 0
				7837 0
				7838 0
				7839 0
				7840 0
				7841 0
				7842 0
				7843 0
				7844 0
				7845 1005520
				7846 0
				7847 0
				7848 0
				7849 0
				7850 0
				7851 0
				7852 0
				7853 0
				7854 0
				7855 0
				7856 0
				7857 0
				7858 0
				7859 0
				7860 998940
				7861 0
				7862 0
				7863 0
				7864 0
				7865 0
				7866 0
				7867 0
				7868 0
				7869 0
				7870 0
				7871 0
				7872 0
				7873 0
				7874 0
				7875 1005105
				7876 0
				7877 0
				7878 0
				7879 0
				7880 0
				7881 0
				7882 0
				7883 0
				7884 0
				7885 0
				7886 0
				7887 0
				7888 0
				7889 0
				7890 988113
				7891 0
				7892 0
				7893 0
				7894 0
				7895 0
				7896 0
				7897 0
				7898 0
				7899 0
				7900 0
				7901 0
				7902 0
				7903 0
				7904 0
				7905 987052
				7906 0
				7907 0
				7908 0
				7909 0
				7910 0
				7911 0
				7912 0
				7913 0
				7914 0
				7915 0
				7916 0
				7917 0
				7918 0
				7919 0
				7920 984581
				7921 0
				7922 0
				7923 0
				7924 0
				7925 0
				7926 0
				7927 0
				7928 0
				7929 0
				7930 0
				7931 0
				7932 0
				7933 0
				7934 0
				7935 998813
				7936 0
				7937 0
				7938 0
				7939 0
				7940 0
				7941 0
				7942 0
				7943 0
				7944 0
				7945 0
				7946 0
				7947 0
				7948 0
				7949 0
				7950 1031797
				7951 0
				7952 0
				7953 0
				7954 0
				7955 0
				7956 0
				7957 0
				7958 0
				7959 0
				7960 0
				7961 0
				7962 0
				7963 0
				7964 0
				7965 992619
				7966 0
				7967 0
				7968 0
				7969 0
				7970 0
				7971 0
				7972 0
				7973 0
				7974 0
				7975 0
				7976 0
				7977 0
				7978 0
				7979 0
				7980 978202
				7981 0
				7982 0
				7983 0
				7984 0
				7985 0
				7986 0
				7987 0
				7988 0
				7989 0
				7990 0
				7991 0
				7992 0
				7993 0
				7994 0
				7995 982883
				7996 0
				7997 0
				7998 0
				7999 0
				8000 0
				8001 0
				8002 0
				8003 0
				8004 0
				8005 0
				8006 0
				8007 0
				8008 0
				8009 0
				8010 1031497
				8011 0
				8012 0
				8013 0
				8014 0
				8015 0
				8016 0
				8017 0
				8018 0
				8019 0
				8020 0
				8021 0
				8022 0
				8023 0
				8024 0
				8025 980159
				8026 0
				8027 0
				8028 0
				8029 0
				8030 0
				8031 0
				8032 0
				8033 0
				8034 0
				8035 0
				8036 0
				8037 0
				8038 0
				8039 0
				8040 1013142
				8041 0
				8042 0
				8043 0
				8044 0
				8045 0
				8046 0
				8047 0
				8048 0
				8049 0
				8050 0
				8051 0
				8052 0
				8053 0
				8054 0
				8055 1028976
				8056 0
				8057 0
				8058 0
				8059 0
				8060 0
				8061 0
				8062 0
				8063 0
				8064 0
				8065 0
				8066 0
				8067 0
				8068 0
				8069 0
				8070 988374
				8071 0
				8072 0
				8073 0
				8074 0
				8075 0
				8076 0
				8077 0
				8078 0
				8079 0
				8080 0
				8081 0
				8082 0
				8083 0
				8084 0
				8085 1032298
				8086 0
				8087 0
				8088 0
				8089 0
				8090 0
				8091 0
				8092 0
				8093 0
				8094 0
				8095 0
				8096 0
				8097 0
				8098 0
				8099 0
				8100 993509
				8101 0
				8102 0
				8103 0
				8104 0
				8105 0
				8106 0
				8107 0
				8108 0
				8109 0
				8110 0
				8111 0
				8112 0
				8113 0
				8114 0
				8115 1017681
				8116 0
				8117 0
				8118 0
				8119 0
				8120 0
				8121 0
				8122 0
				8123 0
				8124 0
				8125 0
				8126 0
				8127 0
				8128 0
				8129 0
				8130 1032511
				8131 0
				8132 0
				8133 0
				8134 0
				8135 0
				8136 0
				8137 0
				8138 0
				8139 0
				8140 0
				8141 0
				8142 0
				8143 0
				8144 0
				8145 1005420
				8146 0
				8147 0
				8148 0
				8149 0
				8150 0
				8151 0
				8152 0
				8153 0
				8154 0
				8155 0
				8156 0
				8157 0
				8158 0
				8159 0
				8160 1011656
				8161 0
				8162 0
				8163 0
				8164 0
				8165 0
				8166 0
				8167 0
				8168 0
				8169 0
				8170 0
				8171 0
				8172 0
				8173 0
				8174 0
				8175 983123
				8176 0
				8177 0
				8178 0
				8179 0
				8180 0
				8181 0
				8182 0
				8183 0
				8184 0
				8185 0
				8186 0
				8187 0
				8188 0
				8189 0
				8190 1025802
				8191 0
				8192 0
				8193 0
				8194 0
				8195 0
				8196 0
				8197 0
				8198 0
				8199 0
				8200 0
				8201 0
				8202 0
				8203 0
				8204 0
				8205 974484
				8206 0
				8207 0
				8208 0
				8209 0
				8210 0
				8211 0
				8212 0
				8213 0
				8214 0
				8215 0
				8216 0
				8217 0
				8218 0
				8219 0
				8220 988443
				8221 0
				8222 0
				8223 0
				8224 0
				8225 0
				8226 0
				8227 0
				8228 0
				8229 0
				8230 0
				8231 0
				8232 0
				8233 0
				8234 0
				8235 988616
				8236 0
				8237 0
				8238 0
				8239 0
				8240 0
				8241 0
				8242 0
				8243 0
				8244 0
				8245 0
				8246 0
				8247 0
				8248 0
				8249 0
				8250 984501
				8251 0
				8252 0
				8253 0
				8254 0
				8255 0
				8256 0
				8257 0
				8258 0
				8259 0
				8260 0
				8261 0
				8262 0
				8263 0
				8264 0
				8265 979012
				8266 0
				8267 0
				8268 0
				8269 0
				8270 0
				8271 0
				8272 0
				8273 0
				8274 0
				8275 0
				8276 0
				8277 0
				8278 0
				8279 0
				8280 988429
				8281 0
				8282 0
				8283 0
				8284 0
				8285 0
				8286 0
				8287 0
				8288 0
				8289 0
				8290 0
				8291 0
				8292 0
				8293 0
				8294 0
				8295 992059
				8296 0
				8297 0
				8298 0
				8299 0
				8300 0
				8301 0
				8302 0
				8303 0
				8304 0
				8305 0
				8306 0
				8307 0
				8308 0
				8309 0
				8310 1028344
				8311 0
				8312 0
				8313 0
				8314 0
				8315 0
				8316 0
				8317 0
				8318 0
				8319 0
				8320 0
				8321 0
				8322 0
				8323 0
				8324 0
				8325 1036848
				8326 0
				8327 0
				8328 0
				8329 0
				8330 0
				8331 0
				8332 0
				8333 0
				8334 0
				8335 0
				8336 0
				8337 0
				8338 0
				8339 0
				8340 1037834
				8341 0
				8342 0
				8343 0
				8344 0
				8345 0
				8346 0
				8347 0
				8348 0
				8349 0
				8350 0
				8351 0
				8352 0
				8353 0
				8354 0
				8355 991281
				8356 0
				8357 0
				8358 0
				8359 0
				8360 0
				8361 0
				8362 0
				8363 0
				8364 0
				8365 0
				8366 0
				8367 0
				8368 0
				8369 0
				8370 1030158
				8371 0
				8372 0
				8373 0
				8374 0
				8375 0
				8376 0
				8377 0
				8378 0
				8379 0
				8380 0
				8381 0
				8382 0
				8383 0
				8384 0
				8385 997369
				8386 0
				8387 0
				8388 0
				8389 0
				8390 0
				8391 0
				8392 0
				8393 0
				8394 0
				8395 0
				8396 0
				8397 0
				8398 0
				8399 0
				8400 995167
				8401 0
				8402 0
				8403 0
				8404 0
				8405 0
				8406 0
				8407 0
				8408 0
				8409 0
				8410 0
				8411 0
				8412 0
				8413 0
				8414 0
				8415 1028318
				8416 0
				8417 0
				8418 0
				8419 0
				8420 0
				8421 0
				8422 0
				8423 0
				8424 0
				8425 0
				8426 0
				8427 0
				8428 0
				8429 0
				8430 980050
				8431 0
				8432 0
				8433 0
				8434 0
				8435 0
				8436 0
				8437 0
				8438 0
				8439 0
				8440 0
				8441 0
				8442 0
				8443 0
				8444 0
				8445 1036057
				8446 0
				8447 0
				8448 0
				8449 0
				8450 0
				8451 0
				8452 0
				8453 0
				8454 0
				8455 0
				8456 0
				8457 0
				8458 0
				8459 0
				8460 993195
				8461 0
				8462 0
				8463 0
				8464 0
				8465 0
				8466 0
				8467 0
				8468 0
				8469 0
				8470 0
				8471 0
				8472 0
				8473 0
				8474 0
				8475 1042499
				8476 0
				8477 0
				8478 0
				8479 0
				8480 0
				8481 0
				8482 0
				8483 0
				8484 0
				8485 0
				8486 0
				8487 0
				8488 0
				8489 0
				8490 996374
				8491 0
				8492 0
				8493 0
				8494 0
				8495 0
				8496 0
				8497 0
				8498 0
				8499 0
				8500 0
				8501 0
				8502 0
				8503 0
				8504 0
				8505 993542
				8506 0
				8507 0
				8508 0
				8509 0
				8510 0
				8511 0
				8512 0
				8513 0
				8514 0
				8515 0
				8516 0
				8517 0
				8518 0
				8519 0
				8520 998407
				8521 0
				8522 0
				8523 0
				8524 0
				8525 0
				8526 0
				8527 0
				8528 0
				8529 0
				8530 0
				8531 0
				8532 0
				8533 0
				8534 0
				8535 998846
				8536 0
				8537 0
				8538 0
				8539 0
				8540 0
				8541 0
				8542 0
				8543 0
				8544 0
				8545 0
				8546 0
				8547 0
				8548 0
				8549 0
				8550 1005786
				8551 0
				8552 0
				8553 0
				8554 0
				8555 0
				8556 0
				8557 0
				8558 0
				8559 0
				8560 0
				8561 0
				8562 0
				8563 0
				8564 0
				8565 995725
				8566 0
				8567 0
				8568 0
				8569 0
				8570 0
				8571 0
				8572 0
				8573 0
				8574 0
				8575 0
				8576 0
				8577 0
				8578 0
				8579 0
				8580 1013122
				8581 0
				8582 0
				8583 0
				8584 0
				8585 0
				8586 0
				8587 0
				8588 0
				8589 0
				8590 0
				8591 0
				8592 0
				8593 0
				8594 0
				8595 1052476
				8596 0
				8597 0
				8598 0
				8599 0
				8600 0
				8601 0
				8602 0
				8603 0
				8604 0
				8605 0
				8606 0
				8607 0
				8608 0
				8609 0
				8610 1013458
				8611 0
				8612 0
				8613 0
				8614 0
				8615 0
				8616 0
				8617 0
				8618 0
				8619 0
				8620 0
				8621 0
				8622 0
				8623 0
				8624 0
				8625 992134
				8626 0
				8627 0
				8628 0
				8629 0
				8630 0
				8631 0
				8632 0
				8633 0
				8634 0
				8635 0
				8636 0
				8637 0
				8638 0
				8639 0
				8640 992720
				8641 0
				8642 0
				8643 0
				8644 0
				8645 0
				8646 0
				8647 0
				8648 0
				8649 0
				8650 0
				8651 0
				8652 0
				8653 0
				8654 0
				8655 974575
				8656 0
				8657 0
				8658 0
				8659 0
				8660 0
				8661 0
				8662 0
				8663 0
				8664 0
				8665 0
				8666 0
				8667 0
				8668 0
				8669 0
				8670 1042032
				8671 0
				8672 0
				8673 0
				8674 0
				8675 0
				8676 0
				8677 0
				8678 0
				8679 0
				8680 0
				8681 0
				8682 0
				8683 0
				8684 0
				8685 1001872
				8686 0
				8687 0
				8688 0
				8689 0
				8690 0
				8691 0
				8692 0
				8693 0
				8694 0
				8695 0
				8696 0
				8697 0
				8698 0
				8699 0
				8700 986581
				8701 0
				8702 0
				8703 0
				8704 0
				8705 0
				8706 0
				8707 0
				8708 0
				8709 0
				8710 0
				8711 0
				8712 0
				8713 0
				8714 0
				8715 1033735
				8716 0
				8717 0
				8718 0
				8719 0
				8720 0
				8721 0
				8722 0
				8723 0
				8724 0
				8725 0
				8726 0
				8727 0
				8728 0
				8729 0
				8730 976242
				8731 0
				8732 0
				8733 0
				8734 0
				8735 0
				8736 0
				8737 0
				8738 0
				8739 0
				8740 0
				8741 0
				8742 0
				8743 0
				8744 0
				8745 993174
				8746 0
				8747 0
				8748 0
				8749 0
				8750 0
				8751 0
				8752 0
				8753 0
				8754 0
				8755 0
				8756 0
				8757 0
				8758 0
				8759 0
				8760 998846
				8761 0
				8762 0
				8763 0
				8764 0
				8765 0
				8766 0
				8767 0
				8768 0
				8769 0
				8770 0
				8771 0
				8772 0
				8773 0
				8774 0
				8775 983791
				8776 0
				8777 0
				8778 0
				8779 0
				8780 0
				8781 0
				8782 0
				8783 0
				8784 0
				8785 0
				8786 0
				8787 0
				8788 0
				8789 0
				8790 1007084
				8791 0
				8792 0
				8793 0
				8794 0
				8795 0
				8796 0
				8797 0
				8798 0
				8799 0
				8800 0
				8801 0
				8802 0
				8803 0
				8804 0
				8805 995862
				8806 0
				8807 0
				8808 0
				8809 0
				8810 0
				8811 0
				8812 0
				8813 0
				8814 0
				8815 0
				8816 0
				8817 0
				8818 0
				8819 0
				8820 982793
				8821 0
				8822 0
				8823 0
				8824 0
				8825 0
				8826 0
				8827 0
				8828 0
				8829 0
				8830 0
				8831 0
				8832 0
				8833 0
				8834 0
				8835 983979
				8836 0
				8837 0
				8838 0
				8839 0
				8840 0
				8841 0
				8842 0
				8843 0
				8844 0
				8845 0
				8846 0
				8847 0
				8848 0
				8849 0
				8850 965637
				8851 0
				8852 0
				8853 0
				8854 0
				8855 0
				8856 0
				8857 0
				8858 0
				8859 0
				8860 0
				8861 0
				8862 0
				8863 0
				8864 0
				8865 1032558
				8866 0
				8867 0
				8868 0
				8869 0
				8870 0
				8871 0
				8872 0
				8873 0
				8874 0
				8875 0
				8876 0
				8877 0
				8878 0
				8879 0
				8880 1040173
				8881 0
				8882 0
				8883 0
				8884 0
				8885 0
				8886 0
				8887 0
				8888 0
				8889 0
				8890 0
				8891 0
				8892 0
				8893 0
				8894 0
				8895 1043451
				8896 0
				8897 0
				8898 0
				8899 0
				8900 0
				8901 0
				8902 0
				8903 0
				8904 0
				8905 0
				8906 0
				8907 0
				8908 0
				8909 0
				8910 1025310
				8911 0
				8912 0
				8913 0
				8914 0
				8915 0
				8916 0
				8917 0
				8918 0
				8919 0
				8920 0
				8921 0
				8922 0
				8923 0
				8924 0
				8925 991354
				8926 0
				8927 0
				8928 0
				8929 0
				8930 0
				8931 0
				8932 0
				8933 0
				8934 0
				8935 0
				8936 0
				8937 0
				8938 0
				8939 0
				8940 984746
				8941 0
				8942 0
				8943 0
				8944 0
				8945 0
				8946 0
				8947 0
				8948 0
				8949 0
				8950 0
				8951 0
				8952 0
				8953 0
				8954 0
				8955 1003500
				8956 0
				8957 0
				8958 0
				8959 0
				8960 0
				8961 0
				8962 0
				8963 0
				8964 0
				8965 0
				8966 0
				8967 0
				8968 0
				8969 0
				8970 998379
				8971 0
				8972 0
				8973 0
				8974 0
				8975 0
				8976 0
				8977 0
				8978 0
				8979 0
				8980 0
				8981 0
				8982 0
				8983 0
				8984 0
				8985 993420
				8986 0
				8987 0
				8988 0
				8989 0
				8990 0
				8991 0
				8992 0
				8993 0
				8994 0
				8995 0
				8996 0
				8997 0
				8998 0
				8999 0
				9000 993684
				9001 0
				9002 0
				9003 0
				9004 0
				9005 0
				9006 0
				9007 0
				9008 0
				9009 0
				9010 0
				9011 0
				9012 0
				9013 0
				9014 0
				9015 990656
				9016 0
				9017 0
				9018 0
				9019 0
				9020 0
				9021 0
				9022 0
				9023 0
				9024 0
				9025 0
				9026 0
				9027 0
				9028 0
				9029 0
				9030 992146
				9031 0
				9032 0
				9033 0
				9034 0
				9035 0
				9036 0
				9037 0
				9038 0
				9039 0
				9040 0
				9041 0
				9042 0
				9043 0
				9044 0
				9045 995616
				9046 0
				9047 0
				9048 0
				9049 0
				9050 0
				9051 0
				9052 0
				9053 0
				9054 0
				9055 0
				9056 0
				9057 0
				9058 0
				9059 0
				9060 1030493
				9061 0
				9062 0
				9063 0
				9064 0
				9065 0
				9066 0
				9067 0
				9068 0
				9069 0
				9070 0
				9071 0
				9072 0
				9073 0
				9074 0
				9075 1037220
				9076 0
				9077 0
				9078 0
				9079 0
				9080 0
				9081 0
				9082 0
				9083 0
				9084 0
				9085 0
				9086 0
				9087 0
				9088 0
				9089 0
				9090 976113
				9091 0
				9092 0
				9093 0
				9094 0
				9095 0
				9096 0
				9097 0
				9098 0
				9099 0
				9100 0
				9101 0
				9102 0
				9103 0
				9104 0
				9105 997739
				9106 0
				9107 0
				9108 0
				9109 0
				9110 0
				9111 0
				9112 0
				9113 0
				9114 0
				9115 0
				9116 0
				9117 0
				9118 0
				9119 0
				9120 989210
				9121 0
				9122 0
				9123 0
				9124 0
				9125 0
				9126 0
				9127 0
				9128 0
				9129 0
				9130 0
				9131 0
				9132 0
				9133 0
				9134 0
				9135 996777
				9136 0
				9137 0
				9138 0
				9139 0
				9140 0
				9141 0
				9142 0
				9143 0
				9144 0
				9145 0
				9146 0
				9147 0
				9148 0
				9149 0
				9150 989394
				9151 0
				9152 0
				9153 0
				9154 0
				9155 0
				9156 0
				9157 0
				9158 0
				9159 0
				9160 0
				9161 0
				9162 0
				9163 0
				9164 0
				9165 997069
				9166 0
				9167 0
				9168 0
				9169 0
				9170 0
				9171 0
				9172 0
				9173 0
				9174 0
				9175 0
				9176 0
				9177 0
				9178 0
				9179 0
				9180 991156
				9181 0
				9182 0
				9183 0
				9184 0
				9185 0
				9186 0
				9187 0
				9188 0
				9189 0
				9190 0
				9191 0
				9192 0
				9193 0
				9194 0
				9195 990602
				9196 0
				9197 0
				9198 0
				9199 0
				9200 0
				9201 0
				9202 0
				9203 0
				9204 0
				9205 0
				9206 0
				9207 0
				9208 0
				9209 0
				9210 999107
				9211 0
				9212 0
				9213 0
				9214 0
				9215 0
				9216 0
				9217 0
				9218 0
				9219 0
				9220 0
				9221 0
				9222 0
				9223 0
				9224 0
				9225 998750
				9226 0
				9227 0
				9228 0
				9229 0
				9230 0
				9231 0
				9232 0
				9233 0
				9234 0
				9235 0
				9236 0
				9237 0
				9238 0
				9239 0
				9240 992521
				9241 0
				9242 0
				9243 0
				9244 0
				9245 0
				9246 0
				9247 0
				9248 0
				9249 0
				9250 0
				9251 0
				9252 0
				9253 0
				9254 0
				9255 1004780
				9256 0
				9257 0
				9258 0
				9259 0
				9260 0
				9261 0
				9262 0
				9263 0
				9264 0
				9265 0
				9266 0
				9267 0
				9268 0
				9269 0
				9270 998974
				9271 0
				9272 0
				9273 0
				9274 0
				9275 0
				9276 0
				9277 0
				9278 0
				9279 0
				9280 0
				9281 0
				9282 0
				9283 0
				9284 0
				9285 1002875
				9286 0
				9287 0
				9288 0
				9289 0
				9290 0
				9291 0
				9292 0
				9293 0
				9294 0
				9295 0
				9296 0
				9297 0
				9298 0
				9299 0
				9300 979868
				9301 0
				9302 0
				9303 0
				9304 0
				9305 0
				9306 0
				9307 0
				9308 0
				9309 0
				9310 0
				9311 0
				9312 0
				9313 0
				9314 0
				9315 993085
				9316 0
				9317 0
				9318 0
				9319 0
				9320 0
				9321 0
				9322 0
				9323 0
				9324 0
				9325 0
				9326 0
				9327 0
				9328 0
				9329 0
				9330 991863
				9331 0
				9332 0
				9333 0
				9334 0
				9335 0
				9336 0
				9337 0
				9338 0
				9339 0
				9340 0
				9341 0
				9342 0
				9343 0
				9344 0
				9345 999605
				9346 0
				9347 0
				9348 0
				9349 0
				9350 0
				9351 0
				9352 0
				9353 0
				9354 0
				9355 0
				9356 0
				9357 0
				9358 0
				9359 0
				9360 1006678
				9361 0
				9362 0
				9363 0
				9364 0
				9365 0
				9366 0
				9367 0
				9368 0
				9369 0
				9370 0
				9371 0
				9372 0
				9373 0
				9374 0
				9375 992641
				9376 0
				9377 0
				9378 0
				9379 0
				9380 0
				9381 0
				9382 0
				9383 0
				9384 0
				9385 0
				9386 0
				9387 0
				9388 0
				9389 0
				9390 987705
				9391 0
				9392 0
				9393 0
				9394 0
				9395 0
				9396 0
				9397 0
				9398 0
				9399 0
				9400 0
				9401 0
				9402 0
				9403 0
				9404 0
				9405 1001803
				9406 0
				9407 0
				9408 0
				9409 0
				9410 0
				9411 0
				9412 0
				9413 0
				9414 0
				9415 0
				9416 0
				9417 0
				9418 0
				9419 0
				9420 994865
				9421 0
				9422 0
				9423 0
				9424 0
				9425 0
				9426 0
				9427 0
				9428 0
				9429 0
				9430 0
				9431 0
				9432 0
				9433 0
				9434 0
				9435 1034458
				9436 0
				9437 0
				9438 0
				9439 0
				9440 0
				9441 0
				9442 0
				9443 0
				9444 0
				9445 0
				9446 0
				9447 0
				9448 0
				9449 0
				9450 1003983
				9451 0
				9452 0
				9453 0
				9454 0
				9455 0
				9456 0
				9457 0
				9458 0
				9459 0
				9460 0
				9461 0
				9462 0
				9463 0
				9464 0
				9465 1046611
				9466 0
				9467 0
				9468 0
				9469 0
				9470 0
				9471 0
				9472 0
				9473 0
				9474 0
				9475 0
				9476 0
				9477 0
				9478 0
				9479 0
				9480 1050190
				9481 0
				9482 0
				9483 0
				9484 0
				9485 0
				9486 0
				9487 0
				9488 0
				9489 0
				9490 0
				9491 0
				9492 0
				9493 0
				9494 0
				9495 1054840
				9496 0
				9497 0
				9498 0
				9499 0
				9500 0
				9501 0
				9502 0
				9503 0
				9504 0
				9505 0
				9506 0
				9507 0
				9508 0
				9509 0
				9510 1037882
				9511 0
				9512 0
				9513 0
				9514 0
				9515 0
				9516 0
				9517 0
				9518 0
				9519 0
				9520 0
				9521 0
				9522 0
				9523 0
				9524 0
				9525 1010914
				9526 0
				9527 0
				9528 0
				9529 0
				9530 0
				9531 0
				9532 0
				9533 0
				9534 0
				9535 0
				9536 0
				9537 0
				9538 0
				9539 0
				9540 1045826
				9541 0
				9542 0
				9543 0
				9544 0
				9545 0
				9546 0
				9547 0
				9548 0
				9549 0
				9550 0
				9551 0
				9552 0
				9553 0
				9554 0
				9555 1035108
				9556 0
				9557 0
				9558 0
				9559 0
				9560 0
				9561 0
				9562 0
				9563 0
				9564 0
				9565 0
				9566 0
				9567 0
				9568 0
				9569 0
				9570 992323
				9571 0
				9572 0
				9573 0
				9574 0
				9575 0
				9576 0
				9577 0
				9578 0
				9579 0
				9580 0
				9581 0
				9582 0
				9583 0
				9584 0
				9585 991086
				9586 0
				9587 0
				9588 0
				9589 0
				9590 0
				9591 0
				9592 0
				9593 0
				9594 0
				9595 0
				9596 0
				9597 0
				9598 0
				9599 0
				9600 1004116
				9601 0
				9602 0
				9603 0
				9604 0
				9605 0
				9606 0
				9607 0
				9608 0
				9609 0
				9610 0
				9611 0
				9612 0
				9613 0
				9614 0
				9615 1006644
				9616 0
				9617 0
				9618 0
				9619 0
				9620 0
				9621 0
				9622 0
				9623 0
				9624 0
				9625 0
				9626 0
				9627 0
				9628 0
				9629 0
				9630 1051633
				9631 0
				9632 0
				9633 0
				9634 0
				9635 0
				9636 0
				9637 0
				9638 0
				9639 0
				9640 0
				9641 0
				9642 0
				9643 0
				9644 0
				9645 985083
				9646 0
				9647 0
				9648 0
				9649 0
				9650 0
				9651 0
				9652 0
				9653 0
				9654 0
				9655 0
				9656 0
				9657 0
				9658 0
				9659 0
				9660 1015068
				9661 0
				9662 0
				9663 0
				9664 0
				9665 0
				9666 0
				9667 0
				9668 0
				9669 0
				9670 0
				9671 0
				9672 0
				9673 0
				9674 0
				9675 1008510
				9676 0
				9677 0
				9678 0
				9679 0
				9680 0
				9681 0
				9682 0
				9683 0
				9684 0
				9685 0
				9686 0
				9687 0
				9688 0
				9689 0
				9690 1036929
				9691 0
				9692 0
				9693 0
				9694 0
				9695 0
				9696 0
				9697 0
				9698 0
				9699 0
				9700 0
				9701 0
				9702 0
				9703 0
				9704 0
				9705 999022
				9706 0
				9707 0
				9708 0
				9709 0
				9710 0
				9711 0
				9712 0
				9713 0
				9714 0
				9715 0
				9716 0
				9717 0
				9718 0
				9719 0
				9720 1005230
				9721 0
				9722 0
				9723 0
				9724 0
				9725 0
				9726 0
				9727 0
				9728 0
				9729 0
				9730 0
				9731 0
				9732 0
				9733 0
				9734 0
				9735 1006737
				9736 0
				9737 0
				9738 0
				9739 0
				9740 0
				9741 0
				9742 0
				9743 0
				9744 0
				9745 0
				9746 0
				9747 0
				9748 0
				9749 0
				9750 1016772
				9751 0
				9752 0
				9753 0
				9754 0
				9755 0
				9756 0
				9757 0
				9758 0
				9759 0
				9760 0
				9761 0
				9762 0
				9763 0
				9764 0
				9765 1055594
				9766 0
				9767 0
				9768 0
				9769 0
				9770 0
				9771 0
				9772 0
				9773 0
				9774 0
				9775 0
				9776 0
				9777 0
				9778 0
				9779 0
				9780 1057348
				9781 0
				9782 0
				9783 0
				9784 0
				9785 0
				9786 0
				9787 0
				9788 0
				9789 0
				9790 0
				9791 0
				9792 0
				9793 0
				9794 0
				9795 996667
				9796 0
				9797 0
				9798 0
				9799 0
				9800 0
				9801 0
				9802 0
				9803 0
				9804 0
				9805 0
				9806 0
				9807 0
				9808 0
				9809 0
				9810 1005302
				9811 0
				9812 0
				9813 0
				9814 0
				9815 0
				9816 0
				9817 0
				9818 0
				9819 0
				9820 0
				9821 0
				9822 0
				9823 0
				9824 0
				9825 1002817
				9826 0
				9827 0
				9828 0
				9829 0
				9830 0
				9831 0
				9832 0
				9833 0
				9834 0
				9835 0
				9836 0
				9837 0
				9838 0
				9839 0
				9840 993622
				9841 0
				9842 0
				9843 0
				9844 0
				9845 0
				9846 0
				9847 0
				9848 0
				9849 0
				9850 0
				9851 0
				9852 0
				9853 0
				9854 0
				9855 1016564
				9856 0
				9857 0
				9858 0
				9859 0
				9860 0
				9861 0
				9862 0
				9863 0
				9864 0
				9865 0
				9866 0
				9867 0
				9868 0
				9869 0
				9870 984966
				9871 0
				9872 0
				9873 0
				9874 0
				9875 0
				9876 0
				9877 0
				9878 0
				9879 0
				9880 0
				9881 0
				9882 0
				9883 0
				9884 0
				9885 987265
				9886 0
				9887 0
				9888 0
				9889 0
				9890 0
				9891 0
				9892 0
				9893 0
				9894 0
				9895 0
				9896 0
				9897 0
				9898 0
				9899 0
				9900 985487
				9901 0
				9902 0
				9903 0
				9904 0
				9905 0
				9906 0
				9907 0
				9908 0
				9909 0
				9910 0
				9911 0
				9912 0
				9913 0
				9914 0
				9915 996367
				9916 0
				9917 0
				9918 0
				9919 0
				9920 0
				9921 0
				9922 0
				9923 0
				9924 0
				9925 0
				9926 0
				9927 0
				9928 0
				9929 0
				9930 990824
				9931 0
				9932 0
				9933 0
				9934 0
				9935 0
				9936 0
				9937 0
				9938 0
				9939 0
				9940 0
				9941 0
				9942 0
				9943 0
				9944 0
				9945 1010062
				9946 0
				9947 0
				9948 0
				9949 0
				9950 0
				9951 0
				9952 0
				9953 0
				9954 0
				9955 0
				9956 0
				9957 0
				9958 0
				9959 0
				9960 1005460
				9961 0
				9962 0
				9963 0
				9964 0
				9965 0
				9966 0
				9967 0
				9968 0
				9969 0
				9970 0
				9971 0
				9972 0
				9973 0
				9974 0
				9975 990953
				9976 0
				9977 0
				9978 0
				9979 0
				9980 0
				9981 0
				9982 0
				9983 0
				9984 0
				9985 0
				9986 0
				9987 0
				9988 0
				9989 0
				9990 993668
				9991 0
				9992 0
				9993 0
				9994 0
				9995 0
				9996 0
				9997 0
				9998 0
				9999 0
				10000 0
				10001 0
				10002 0
				10003 0
				10004 0
				10005 1011812
				10006 0
				10007 0
				10008 0
				10009 0
				10010 0
				10011 0
				10012 0
				10013 0
				10014 0
				10015 0
				10016 0
				10017 0
				10018 0
				10019 0
				10020 988800
				10021 0
				10022 0
				10023 0
				10024 0
				10025 0
				10026 0
				10027 0
				10028 0
				10029 0
				10030 0
				10031 0
				10032 0
				10033 0
				10034 0
				10035 1009566
				10036 0
				10037 0
				10038 0
				10039 0
				10040 0
				10041 0
				10042 0
				10043 0
				10044 0
				10045 0
				10046 0
				10047 0
				10048 0
				10049 0
				10050 997437
				10051 0
				10052 0
				10053 0
				10054 0
				10055 0
				10056 0
				10057 0
				10058 0
				10059 0
				10060 0
				10061 0
				10062 0
				10063 0
				10064 0
				10065 999498
				10066 0
				10067 0
				10068 0
				10069 0
				10070 0
				10071 0
				10072 0
				10073 0
				10074 0
				10075 0
				10076 0
				10077 0
				10078 0
				10079 0
				10080 1010058
				10081 0
				10082 0
				10083 0
				10084 0
				10085 0
				10086 0
				10087 0
				10088 0
				10089 0
				10090 0
				10091 0
				10092 0
				10093 0
				10094 0
				10095 1006819
				10096 0
				10097 0
				10098 0
				10099 0
				10100 0
				10101 0
				10102 0
				10103 0
				10104 0
				10105 0
				10106 0
				10107 0
				10108 0
				10109 0
				10110 1052104
				10111 0
				10112 0
				10113 0
				10114 0
				10115 0
				10116 0
				10117 0
				10118 0
				10119 0
				10120 0
				10121 0
				10122 0
				10123 0
				10124 0
				10125 996166
				10126 0
				10127 0
				10128 0
				10129 0
				10130 0
				10131 0
				10132 0
				10133 0
				10134 0
				10135 0
				10136 0
				10137 0
				10138 0
				10139 0
				10140 1026387
				10141 0
				10142 0
				10143 0
				10144 0
				10145 0
				10146 0
				10147 0
				10148 0
				10149 0
				10150 0
				10151 0
				10152 0
				10153 0
				10154 0
				10155 1016491
				10156 0
				10157 0
				10158 0
				10159 0
				10160 0
				10161 0
				10162 0
				10163 0
				10164 0
				10165 0
				10166 0
				10167 0
				10168 0
				10169 0
				10170 1016746
				10171 0
				10172 0
				10173 0
				10174 0
				10175 0
				10176 0
				10177 0
				10178 0
				10179 0
				10180 0
				10181 0
				10182 0
				10183 0
				10184 0
				10185 1028056
				10186 0
				10187 0
				10188 0
				10189 0
				10190 0
				10191 0
				10192 0
				10193 0
				10194 0
				10195 0
				10196 0
				10197 0
				10198 0
				10199 0
				10200 1018604
				10201 0
				10202 0
				10203 0
				10204 0
				10205 0
				10206 0
				10207 0
				10208 0
				10209 0
				10210 0
				10211 0
				10212 0
				10213 0
				10214 0
				10215 1019452
				10216 0
				10217 0
				10218 0
				10219 0
				10220 0
				10221 0
				10222 0
				10223 0
				10224 0
				10225 0
				10226 0
				10227 0
				10228 0
				10229 0
				10230 999569
				10231 0
				10232 0
				10233 0
				10234 0
				10235 0
				10236 0
				10237 0
				10238 0
				10239 0
				10240 0
				10241 0
				10242 0
				10243 0
				10244 0
				10245 1009192
				10246 0
				10247 0
				10248 0
				10249 0
				10250 0
				10251 0
				10252 0
				10253 0
				10254 0
				10255 0
				10256 0
				10257 0
				10258 0
				10259 0
				10260 1023015
				10261 0
				10262 0
				10263 0
				10264 0
				10265 0
				10266 0
				10267 0
				10268 0
				10269 0
				10270 0
				10271 0
				10272 0
				10273 0
				10274 0
				10275 1010762
				10276 0
				10277 0
				10278 0
				10279 0
				10280 0
				10281 0
				10282 0
				10283 0
				10284 0
				10285 0
				10286 0
				10287 0
				10288 0
				10289 0
				10290 1029429
				10291 0
				10292 0
				10293 0
				10294 0
				10295 0
				10296 0
				10297 0
				10298 0
				10299 0
				10300 0
				10301 0
				10302 0
				10303 0
				10304 0
				10305 994563
				10306 0
				10307 0
				10308 0
				10309 0
				10310 0
				10311 0
				10312 0
				10313 0
				10314 0
				10315 0
				10316 0
				10317 0
				10318 0
				10319 0
				10320 1003557
				10321 0
				10322 0
				10323 0
				10324 0
				10325 0
				10326 0
				10327 0
				10328 0
				10329 0
				10330 0
				10331 0
				10332 0
				10333 0
				10334 0
				10335 1020687
				10336 0
				10337 0
				10338 0
				10339 0
				10340 0
				10341 0
				10342 0
				10343 0
				10344 0
				10345 0
				10346 0
				10347 0
				10348 0
				10349 0
				10350 1002340
				10351 0
				10352 0
				10353 0
				10354 0
				10355 0
				10356 0
				10357 0
				10358 0
				10359 0
				10360 0
				10361 0
				10362 0
				10363 0
				10364 0
				10365 1019220
				10366 0
				10367 0
				10368 0
				10369 0
				10370 0
				10371 0
				10372 0
				10373 0
				10374 0
				10375 0
				10376 0
				10377 0
				10378 0
				10379 0
				10380 1047470
				10381 0
				10382 0
				10383 0
				10384 0
				10385 0
				10386 0
				10387 0
				10388 0
				10389 0
				10390 0
				10391 0
				10392 0
				10393 0
				10394 0
				10395 1018686
				10396 0
				10397 0
				10398 0
				10399 0
				10400 0
				10401 0
				10402 0
				10403 0
				10404 0
				10405 0
				10406 0
				10407 0
				10408 0
				10409 0
				10410 1021227
				10411 0
				10412 0
				10413 0
				10414 0
				10415 0
				10416 0
				10417 0
				10418 0
				10419 0
				10420 0
				10421 0
				10422 0
				10423 0
				10424 0
				10425 1029839
				10426 0
				10427 0
				10428 0
				10429 0
				10430 0
				10431 0
				10432 0
				10433 0
				10434 0
				10435 0
				10436 0
				10437 0
				10438 0
				10439 0
				10440 999147
				10441 0
				10442 0
				10443 0
				10444 0
				10445 0
				10446 0
				10447 0
				10448 0
				10449 0
				10450 0
				10451 0
				10452 0
				10453 0
				10454 0
				10455 1014111
				10456 0
				10457 0
				10458 0
				10459 0
				10460 0
				10461 0
				10462 0
				10463 0
				10464 0
				10465 0
				10466 0
				10467 0
				10468 0
				10469 0
				10470 988971
				10471 0
				10472 0
				10473 0
				10474 0
				10475 0
				10476 0
				10477 0
				10478 0
				10479 0
				10480 0
				10481 0
				10482 0
				10483 0
				10484 0
				10485 1014393
				10486 0
				10487 0
				10488 0
				10489 0
				10490 0
				10491 0
				10492 0
				10493 0
				10494 0
				10495 0
				10496 0
				10497 0
				10498 0
				10499 0
				10500 1005504
				10501 0
				10502 0
				10503 0
				10504 0
				10505 0
				10506 0
				10507 0
				10508 0
				10509 0
				10510 0
				10511 0
				10512 0
				10513 0
				10514 0
				10515 1005671
				10516 0
				10517 0
				10518 0
				10519 0
				10520 0
				10521 0
				10522 0
				10523 0
				10524 0
				10525 0
				10526 0
				10527 0
				10528 0
				10529 0
				10530 994493
				10531 0
				10532 0
				10533 0
				10534 0
				10535 0
				10536 0
				10537 0
				10538 0
				10539 0
				10540 0
				10541 0
				10542 0
				10543 0
				10544 0
				10545 1045925
				10546 0
				10547 0
				10548 0
				10549 0
				10550 0
				10551 0
				10552 0
				10553 0
				10554 0
				10555 0
				10556 0
				10557 0
				10558 0
				10559 0
				10560 1006181
				10561 0
				10562 0
				10563 0
				10564 0
				10565 0
				10566 0
				10567 0
				10568 0
				10569 0
				10570 0
				10571 0
				10572 0
				10573 0
				10574 0
				10575 1007042
				10576 0
				10577 0
				10578 0
				10579 0
				10580 0
				10581 0
				10582 0
				10583 0
				10584 0
				10585 0
				10586 0
				10587 0
				10588 0
				10589 0
				10590 993303
				10591 0
				10592 0
				10593 0
				10594 0
				10595 0
				10596 0
				10597 0
				10598 0
				10599 0
				10600 0
				10601 0
				10602 0
				10603 0
				10604 0
				10605 1005421
				10606 0
				10607 0
				10608 0
				10609 0
				10610 0
				10611 0
				10612 0
				10613 0
				10614 0
				10615 0
				10616 0
				10617 0
				10618 0
				10619 0
				10620 1014988
				10621 0
				10622 0
				10623 0
				10624 0
				10625 0
				10626 0
				10627 0
				10628 0
				10629 0
				10630 0
				10631 0
				10632 0
				10633 0
				10634 0
				10635 998537
				10636 0
				10637 0
				10638 0
				10639 0
				10640 0
				10641 0
				10642 0
				10643 0
				10644 0
				10645 0
				10646 0
				10647 0
				10648 0
				10649 0
				10650 990972
				10651 0
				10652 0
				10653 0
				10654 0
				10655 0
				10656 0
				10657 0
				10658 0
				10659 0
				10660 0
				10661 0
				10662 0
				10663 0
				10664 0
				10665 1006150
				10666 0
				10667 0
				10668 0
				10669 0
				10670 0
				10671 0
				10672 0
				10673 0
				10674 0
				10675 0
				10676 0
				10677 0
				10678 0
				10679 0
				10680 1084569
				10681 0
				10682 0
				10683 0
				10684 0
				10685 0
				10686 0
				10687 0
				10688 0
				10689 0
				10690 0
				10691 0
				10692 0
				10693 0
				10694 0
				10695 996375
				10696 0
				10697 0
				10698 0
				10699 0
				10700 0
				10701 0
				10702 0
				10703 0
				10704 0
				10705 0
				10706 0
				10707 0
				10708 0
				10709 0
				10710 1000054
				10711 0
				10712 0
				10713 0
				10714 0
				10715 0
				10716 0
				10717 0
				10718 0
				10719 0
				10720 0
				10721 0
				10722 0
				10723 0
				10724 0
				10725 1001281
				10726 0
				10727 0
				10728 0
				10729 0
				10730 0
				10731 0
				10732 0
				10733 0
				10734 0
				10735 0
				10736 0
				10737 0
				10738 0
				10739 0
				10740 1009074
				10741 0
				10742 0
				10743 0
				10744 0
				10745 0
				10746 0
				10747 0
				10748 0
				10749 0
				10750 0
				10751 0
				10752 0
				10753 0
				10754 0
				10755 1008678
				10756 0
				10757 0
				10758 0
				10759 0
				10760 0
				10761 0
				10762 0
				10763 0
				10764 0
				10765 0
				10766 0
				10767 0
				10768 0
				10769 0
				10770 1022867
				10771 0
				10772 0
				10773 0
				10774 0
				10775 0
				10776 0
				10777 0
				10778 0
				10779 0
				10780 0
				10781 0
				10782 0
				10783 0
				10784 0
				10785 987545
				10786 0
				10787 0
				10788 0
				10789 0
				10790 0
				10791 0
				10792 0
				10793 0
				10794 0
				10795 0
				10796 0
				10797 0
				10798 0
				10799 0
				10800 1014041
				10801 0
				10802 0
				10803 0
				10804 0
				10805 0
				10806 0
				10807 0
				10808 0
				10809 0
				10810 0
				10811 0
				10812 0
				10813 0
				10814 0
				10815 1003163
				10816 0
				10817 0
				10818 0
				10819 0
				10820 0
				10821 0
				10822 0
				10823 0
				10824 0
				10825 0
				10826 0
				10827 0
				10828 0
				10829 0
				10830 1005875
				10831 0
				10832 0
				10833 0
				10834 0
				10835 0
				10836 0
				10837 0
				10838 0
				10839 0
				10840 0
				10841 0
				10842 0
				10843 0
				10844 0
				10845 1011391
				10846 0
				10847 0
				10848 0
				10849 0
				10850 0
				10851 0
				10852 0
				10853 0
				10854 0
				10855 0
				10856 0
				10857 0
				10858 0
				10859 0
				10860 1035227
				10861 0
				10862 0
				10863 0
				10864 0
				10865 0
				10866 0
				10867 0
				10868 0
				10869 0
				10870 0
				10871 0
				10872 0
				10873 0
				10874 0
				10875 998484
				10876 0
				10877 0
				10878 0
				10879 0
				10880 0
				10881 0
				10882 0
				10883 0
				10884 0
				10885 0
				10886 0
				10887 0
				10888 0
				10889 0
				10890 1002456
				10891 0
				10892 0
				10893 0
				10894 0
				10895 0
				10896 0
				10897 0
				10898 0
				10899 0
				10900 0
				10901 0
				10902 0
				10903 0
				10904 0
				10905 1000177
				10906 0
				10907 0
				10908 0
				10909 0
				10910 0
				10911 0
				10912 0
				10913 0
				10914 0
				10915 0
				10916 0
				10917 0
				10918 0
				10919 0
				10920 998421
				10921 0
				10922 0
				10923 0
				10924 0
				10925 0
				10926 0
				10927 0
				10928 0
				10929 0
				10930 0
				10931 0
				10932 0
				10933 0
				10934 0
				10935 1005833
				10936 0
				10937 0
				10938 0
				10939 0
				10940 0
				10941 0
				10942 0
				10943 0
				10944 0
				10945 0
				10946 0
				10947 0
				10948 0
				10949 0
				10950 1024898
				10951 0
				10952 0
				10953 0
				10954 0
				10955 0
				10956 0
				10957 0
				10958 0
				10959 0
				10960 0
				10961 0
				10962 0
				10963 0
				10964 0
				10965 993551
				10966 0
				10967 0
				10968 0
				10969 0
				10970 0
				10971 0
				10972 0
				10973 0
				10974 0
				10975 0
				10976 0
				10977 0
				10978 0
				10979 0
				10980 1006639
				10981 0
				10982 0
				10983 0
				10984 0
				10985 0
				10986 0
				10987 0
				10988 0
				10989 0
				10990 0
				10991 0
				10992 0
				10993 0
				10994 0
				10995 1023151
				10996 0
				10997 0
				10998 0
				10999 0
				11000 0
				11001 0
				11002 0
				11003 0
				11004 0
				11005 0
				11006 0
				11007 0
				11008 0
				11009 0
				11010 1002651
				11011 0
				11012 0
				11013 0
				11014 0
				11015 0
				11016 0
				11017 0
				11018 0
				11019 0
				11020 0
				11021 0
				11022 0
				11023 0
				11024 0
				11025 1012533
				11026 0
				11027 0
				11028 0
				11029 0
				11030 0
				11031 0
				11032 0
				11033 0
				11034 0
				11035 0
				11036 0
				11037 0
				11038 0
				11039 0
				11040 1035397
				11041 0
				11042 0
				11043 0
				11044 0
				11045 0
				11046 0
				11047 0
				11048 0
				11049 0
				11050 0
				11051 0
				11052 0
				11053 0
				11054 0
				11055 999780
				11056 0
				11057 0
				11058 0
				11059 0
				11060 0
				11061 0
				11062 0
				11063 0
				11064 0
				11065 0
				11066 0
				11067 0
				11068 0
				11069 0
				11070 1040342
				11071 0
				11072 0
				11073 0
				11074 0
				11075 0
				11076 0
				11077 0
				11078 0
				11079 0
				11080 0
				11081 0
				11082 0
				11083 0
				11084 0
				11085 1016993
				11086 0
				11087 0
				11088 0
				11089 0
				11090 0
				11091 0
				11092 0
				11093 0
				11094 0
				11095 0
				11096 0
				11097 0
				11098 0
				11099 0
				11100 1021657
				11101 0
				11102 0
				11103 0
				11104 0
				11105 0
				11106 0
				11107 0
				11108 0
				11109 0
				11110 0
				11111 0
				11112 0
				11113 0
				11114 0
				11115 1031524
				11116 0
				11117 0
				11118 0
				11119 0
				11120 0
				11121 0
				11122 0
				11123 0
				11124 0
				11125 0
				11126 0
				11127 0
				11128 0
				11129 0
				11130 1029517
				11131 0
				11132 0
				11133 0
				11134 0
				11135 0
				11136 0
				11137 0
				11138 0
				11139 0
				11140 0
				11141 0
				11142 0
				11143 0
				11144 0
				11145 996963
				11146 0
				11147 0
				11148 0
				11149 0
				11150 0
				11151 0
				11152 0
				11153 0
				11154 0
				11155 0
				11156 0
				11157 0
				11158 0
				11159 0
				11160 1001679
				11161 0
				11162 0
				11163 0
				11164 0
				11165 0
				11166 0
				11167 0
				11168 0
				11169 0
				11170 0
				11171 0
				11172 0
				11173 0
				11174 0
				11175 1060213
				11176 0
				11177 0
				11178 0
				11179 0
				11180 0
				11181 0
				11182 0
				11183 0
				11184 0
				11185 0
				11186 0
				11187 0
				11188 0
				11189 0
				11190 1023038
				11191 0
				11192 0
				11193 0
				11194 0
				11195 0
				11196 0
				11197 0
				11198 0
				11199 0
				11200 0
				11201 0
				11202 0
				11203 0
				11204 0
				11205 1024214
				11206 0
				11207 0
				11208 0
				11209 0
				11210 0
				11211 0
				11212 0
				11213 0
				11214 0
				11215 0
				11216 0
				11217 0
				11218 0
				11219 0
				11220 1036267
				11221 0
				11222 0
				11223 0
				11224 0
				11225 0
				11226 0
				11227 0
				11228 0
				11229 0
				11230 0
				11231 0
				11232 0
				11233 0
				11234 0
				11235 1000743
				11236 0
				11237 0
				11238 0
				11239 0
				11240 0
				11241 0
				11242 0
				11243 0
				11244 0
				11245 0
				11246 0
				11247 0
				11248 0
				11249 0
				11250 1042114
				11251 0
				11252 0
				11253 0
				11254 0
				11255 0
				11256 0
				11257 0
				11258 0
				11259 0
				11260 0
				11261 0
				11262 0
				11263 0
				11264 0
				11265 1010863
				11266 0
				11267 0
				11268 0
				11269 0
				11270 0
				11271 0
				11272 0
				11273 0
				11274 0
				11275 0
				11276 0
				11277 0
				11278 0
				11279 0
				11280 1026129
				11281 0
				11282 0
				11283 0
				11284 0
				11285 0
				11286 0
				11287 0
				11288 0
				11289 0
				11290 0
				11291 0
				11292 0
				11293 0
				11294 0
				11295 1015662
				11296 0
				11297 0
				11298 0
				11299 0
				11300 0
				11301 0
				11302 0
				11303 0
				11304 0
				11305 0
				11306 0
				11307 0
				11308 0
				11309 0
				11310 1029312
				11311 0
				11312 0
				11313 0
				11314 0
				11315 0
				11316 0
				11317 0
				11318 0
				11319 0
				11320 0
				11321 0
				11322 0
				11323 0
				11324 0
				11325 1001256
				11326 0
				11327 0
				11328 0
				11329 0
				11330 0
				11331 0
				11332 0
				11333 0
				11334 0
				11335 0
				11336 0
				11337 0
				11338 0
				11339 0
				11340 1018189
				11341 0
				11342 0
				11343 0
				11344 0
				11345 0
				11346 0
				11347 0
				11348 0
				11349 0
				11350 0
				11351 0
				11352 0
				11353 0
				11354 0
				11355 1012639
				11356 0
				11357 0
				11358 0
				11359 0
				11360 0
				11361 0
				11362 0
				11363 0
				11364 0
				11365 0
				11366 0
				11367 0
				11368 0
				11369 0
				11370 1017058
				11371 0
				11372 0
				11373 0
				11374 0
				11375 0
				11376 0
				11377 0
				11378 0
				11379 0
				11380 0
				11381 0
				11382 0
				11383 0
				11384 0
				11385 1013063
				11386 0
				11387 0
				11388 0
				11389 0
				11390 0
				11391 0
				11392 0
				11393 0
				11394 0
				11395 0
				11396 0
				11397 0
				11398 0
				11399 0
				11400 1008056
				11401 0
				11402 0
				11403 0
				11404 0
				11405 0
				11406 0
				11407 0
				11408 0
				11409 0
				11410 0
				11411 0
				11412 0
				11413 0
				11414 0
				11415 1051569
				11416 0
				11417 0
				11418 0
				11419 0
				11420 0
				11421 0
				11422 0
				11423 0
				11424 0
				11425 0
				11426 0
				11427 0
				11428 0
				11429 0
				11430 1014046
				11431 0
				11432 0
				11433 0
				11434 0
				11435 0
				11436 0
				11437 0
				11438 0
				11439 0
				11440 0
				11441 0
				11442 0
				11443 0
				11444 0
				11445 1020089
				11446 0
				11447 0
				11448 0
				11449 0
				11450 0
				11451 0
				11452 0
				11453 0
				11454 0
				11455 0
				11456 0
				11457 0
				11458 0
				11459 0
				11460 1039623
				11461 0
				11462 0
				11463 0
				11464 0
				11465 0
				11466 0
				11467 0
				11468 0
				11469 0
				11470 0
				11471 0
				11472 0
				11473 0
				11474 0
				11475 1009959
				11476 0
				11477 0
				11478 0
				11479 0
				11480 0
				11481 0
				11482 0
				11483 0
				11484 0
				11485 0
				11486 0
				11487 0
				11488 0
				11489 0
				11490 1001688
				11491 0
				11492 0
				11493 0
				11494 0
				11495 0
				11496 0
				11497 0
				11498 0
				11499 0
				11500 0
				11501 0
				11502 0
				11503 0
				11504 0
				11505 1018670
				11506 0
				11507 0
				11508 0
				11509 0
				11510 0
				11511 0
				11512 0
				11513 0
				11514 0
				11515 0
				11516 0
				11517 0
				11518 0
				11519 0
				11520 1013976
				11521 0
				11522 0
				11523 0
				11524 0
				11525 0
				11526 0
				11527 0
				11528 0
				11529 0
				11530 0
				11531 0
				11532 0
				11533 0
				11534 0
				11535 998266
				11536 0
				11537 0
				11538 0
				11539 0
				11540 0
				11541 0
				11542 0
				11543 0
				11544 0
				11545 0
				11546 0
				11547 0
				11548 0
				11549 0
				11550 1005170
				11551 0
				11552 0
				11553 0
				11554 0
				11555 0
				11556 0
				11557 0
				11558 0
				11559 0
				11560 0
				11561 0
				11562 0
				11563 0
				11564 0
				11565 1019376
				11566 0
				11567 0
				11568 0
				11569 0
				11570 0
				11571 0
				11572 0
				11573 0
				11574 0
				11575 0
				11576 0
				11577 0
				11578 0
				11579 0
				11580 1058536
				11581 0
				11582 0
				11583 0
				11584 0
				11585 0
				11586 0
				11587 0
				11588 0
				11589 0
				11590 0
				11591 0
				11592 0
				11593 0
				11594 0
				11595 1019670
				11596 0
				11597 0
				11598 0
				11599 0
				11600 0
				11601 0
				11602 0
				11603 0
				11604 0
				11605 0
				11606 0
				11607 0
				11608 0
				11609 0
				11610 1023267
				11611 0
				11612 0
				11613 0
				11614 0
				11615 0
				11616 0
				11617 0
				11618 0
				11619 0
				11620 0
				11621 0
				11622 0
				11623 0
				11624 0
				11625 1015180
				11626 0
				11627 0
				11628 0
				11629 0
				11630 0
				11631 0
				11632 0
				11633 0
				11634 0
				11635 0
				11636 0
				11637 0
				11638 0
				11639 0
				11640 1023025
				11641 0
				11642 0
				11643 0
				11644 0
				11645 0
				11646 0
				11647 0
				11648 0
				11649 0
				11650 0
				11651 0
				11652 0
				11653 0
				11654 0
				11655 1032028
				11656 0
				11657 0
				11658 0
				11659 0
				11660 0
				11661 0
				11662 0
				11663 0
				11664 0
				11665 0
				11666 0
				11667 0
				11668 0
				11669 0
				11670 1070306
				11671 0
				11672 0
				11673 0
				11674 0
				11675 0
				11676 0
				11677 0
				11678 0
				11679 0
				11680 0
				11681 0
				11682 0
				11683 0
				11684 0
				11685 1001426
				11686 0
				11687 0
				11688 0
				11689 0
				11690 0
				11691 0
				11692 0
				11693 0
				11694 0
				11695 0
				11696 0
				11697 0
				11698 0
				11699 0
				11700 1017175
				11701 0
				11702 0
				11703 0
				11704 0
				11705 0
				11706 0
				11707 0
				11708 0
				11709 0
				11710 0
				11711 0
				11712 0
				11713 0
				11714 0
				11715 1016381
				11716 0
				11717 0
				11718 0
				11719 0
				11720 0
				11721 0
				11722 0
				11723 0
				11724 0
				11725 0
				11726 0
				11727 0
				11728 0
				11729 0
				11730 1011249
				11731 0
				11732 0
				11733 0
				11734 0
				11735 0
				11736 0
				11737 0
				11738 0
				11739 0
				11740 0
				11741 0
				11742 0
				11743 0
				11744 0
				11745 1021740
				11746 0
				11747 0
				11748 0
				11749 0
				11750 0
				11751 0
				11752 0
				11753 0
				11754 0
				11755 0
				11756 0
				11757 0
				11758 0
				11759 0
				11760 1065631
				11761 0
				11762 0
				11763 0
				11764 0
				11765 0
				11766 0
				11767 0
				11768 0
				11769 0
				11770 0
				11771 0
				11772 0
				11773 0
				11774 0
				11775 1023242
				11776 0
				11777 0
				11778 0
				11779 0
				11780 0
				11781 0
				11782 0
				11783 0
				11784 0
				11785 0
				11786 0
				11787 0
				11788 0
				11789 0
				11790 1028302
				11791 0
				11792 0
				11793 0
				11794 0
				11795 0
				11796 0
				11797 0
				11798 0
				11799 0
				11800 0
				11801 0
				11802 0
				11803 0
				11804 0
				11805 1021730
				11806 0
				11807 0
				11808 0
				11809 0
				11810 0
				11811 0
				11812 0
				11813 0
				11814 0
				11815 0
				11816 0
				11817 0
				11818 0
				11819 0
				11820 1020016
				11821 0
				11822 0
				11823 0
				11824 0
				11825 0
				11826 0
				11827 0
				11828 0
				11829 0
				11830 0
				11831 0
				11832 0
				11833 0
				11834 0
				11835 1009392
				11836 0
				11837 0
				11838 0
				11839 0
				11840 0
				11841 0
				11842 0
				11843 0
				11844 0
				11845 0
				11846 0
				11847 0
				11848 0
				11849 0
				11850 1019536
				11851 0
				11852 0
				11853 0
				11854 0
				11855 0
				11856 0
				11857 0
				11858 0
				11859 0
				11860 0
				11861 0
				11862 0
				11863 0
				11864 0
				11865 1036123
				11866 0
				11867 0
				11868 0
				11869 0
				11870 0
				11871 0
				11872 0
				11873 0
				11874 0
				11875 0
				11876 0
				11877 0
				11878 0
				11879 0
				11880 1005428
				11881 0
				11882 0
				11883 0
				11884 0
				11885 0
				11886 0
				11887 0
				11888 0
				11889 0
				11890 0
				11891 0
				11892 0
				11893 0
				11894 0
				11895 1005544
				11896 0
				11897 0
				11898 0
				11899 0
				11900 0
				11901 0
				11902 0
				11903 0
				11904 0
				11905 0
				11906 0
				11907 0
				11908 0
				11909 0
				11910 1019200
				11911 0
				11912 0
				11913 0
				11914 0
				11915 0
				11916 0
				11917 0
				11918 0
				11919 0
				11920 0
				11921 0
				11922 0
				11923 0
				11924 0
				11925 1023312
				11926 0
				11927 0
				11928 0
				11929 0
				11930 0
				11931 0
				11932 0
				11933 0
				11934 0
				11935 0
				11936 0
				11937 0
				11938 0
				11939 0
				11940 1044687
				11941 0
				11942 0
				11943 0
				11944 0
				11945 0
				11946 0
				11947 0
				11948 0
				11949 0
				11950 0
				11951 0
				11952 0
				11953 0
				11954 0
				11955 1067258
				11956 0
				11957 0
				11958 0
				11959 0
				11960 0
				11961 0
				11962 0
				11963 0
				11964 0
				11965 0
				11966 0
				11967 0
				11968 0
				11969 0
				11970 1028812
				11971 0
				11972 0
				11973 0
				11974 0
				11975 0
				11976 0
				11977 0
				11978 0
				11979 0
				11980 0
				11981 0
				11982 0
				11983 0
				11984 0
				11985 1021979
				11986 0
				11987 0
				11988 0
				11989 0
				11990 0
				11991 0
				11992 0
				11993 0
				11994 0
				11995 0
				11996 0
				11997 0
				11998 0
				11999 0
				12000 1005341
				12001 0
				12002 0
				12003 0
				12004 0
				12005 0
				12006 0
				12007 0
				12008 0
				12009 0
				12010 0
				12011 0
				12012 0
				12013 0
				12014 0
				12015 1000227
				12016 0
				12017 0
				12018 0
				12019 0
				12020 0
				12021 0
				12022 0
				12023 0
				12024 0
				12025 0
				12026 0
				12027 0
				12028 0
				12029 0
				12030 1025817
				12031 0
				12032 0
				12033 0
				12034 0
				12035 0
				12036 0
				12037 0
				12038 0
				12039 0
				12040 0
				12041 0
				12042 0
				12043 0
				12044 0
				12045 1017269
				12046 0
				12047 0
				12048 0
				12049 0
				12050 0
				12051 0
				12052 0
				12053 0
				12054 0
				12055 0
				12056 0
				12057 0
				12058 0
				12059 0
				12060 1021758
				12061 0
				12062 0
				12063 0
				12064 0
				12065 0
				12066 0
				12067 0
				12068 0
				12069 0
				12070 0
				12071 0
				12072 0
				12073 0
				12074 0
				12075 1021059
				12076 0
				12077 0
				12078 0
				12079 0
				12080 0
				12081 0
				12082 0
				12083 0
				12084 0
				12085 0
				12086 0
				12087 0
				12088 0
				12089 0
				12090 1039340
				12091 0
				12092 0
				12093 0
				12094 0
				12095 0
				12096 0
				12097 0
				12098 0
				12099 0
				12100 0
				12101 0
				12102 0
				12103 0
				12104 0
				12105 1029771
				12106 0
				12107 0
				12108 0
				12109 0
				12110 0
				12111 0
				12112 0
				12113 0
				12114 0
				12115 0
				12116 0
				12117 0
				12118 0
				12119 0
				12120 1026149
				12121 0
				12122 0
				12123 0
				12124 0
				12125 0
				12126 0
				12127 0
				12128 0
				12129 0
				12130 0
				12131 0
				12132 0
				12133 0
				12134 0
				12135 1022521
				12136 0
				12137 0
				12138 0
				12139 0
				12140 0
				12141 0
				12142 0
				12143 0
				12144 0
				12145 0
				12146 0
				12147 0
				12148 0
				12149 0
				12150 1067748
				12151 0
				12152 0
				12153 0
				12154 0
				12155 0
				12156 0
				12157 0
				12158 0
				12159 0
				12160 0
				12161 0
				12162 0
				12163 0
				12164 0
				12165 1066954
				12166 0
				12167 0
				12168 0
				12169 0
				12170 0
				12171 0
				12172 0
				12173 0
				12174 0
				12175 0
				12176 0
				12177 0
				12178 0
				12179 0
				12180 1037190
				12181 0
				12182 0
				12183 0
				12184 0
				12185 0
				12186 0
				12187 0
				12188 0
				12189 0
				12190 0
				12191 0
				12192 0
				12193 0
				12194 0
				12195 1044222
				12196 0
				12197 0
				12198 0
				12199 0
				12200 0
				12201 0
				12202 0
				12203 0
				12204 0
				12205 0
				12206 0
				12207 0
				12208 0
				12209 0
				12210 1049368
				12211 0
				12212 0
				12213 0
				12214 0
				12215 0
				12216 0
				12217 0
				12218 0
				12219 0
				12220 0
				12221 0
				12222 0
				12223 0
				12224 0
				12225 1036486
				12226 0
				12227 0
				12228 0
				12229 0
				12230 0
				12231 0
				12232 0
				12233 0
				12234 0
				12235 0
				12236 0
				12237 0
				12238 0
				12239 0
				12240 1039483
				12241 0
				12242 0
				12243 0
				12244 0
				12245 0
				12246 0
				12247 0
				12248 0
				12249 0
				12250 0
				12251 0
				12252 0
				12253 0
				12254 0
				12255 1039611
				12256 0
				12257 0
				12258 0
				12259 0
				12260 0
				12261 0
				12262 0
				12263 0
				12264 0
				12265 0
				12266 0
				12267 0
				12268 0
				12269 0
				12270 1028028
				12271 0
				12272 0
				12273 0
				12274 0
				12275 0
				12276 0
				12277 0
				12278 0
				12279 0
				12280 0
				12281 0
				12282 0
				12283 0
				12284 0
				12285 1032084
				12286 0
				12287 0
				12288 0
				12289 0
				12290 0
				12291 0
				12292 0
				12293 0
				12294 0
				12295 0
				12296 0
				12297 0
				12298 0
				12299 0
				12300 1017380
				12301 0
				12302 0
				12303 0
				12304 0
				12305 0
				12306 0
				12307 0
				12308 0
				12309 0
				12310 0
				12311 0
				12312 0
				12313 0
				12314 0
				12315 1018982
				12316 0
				12317 0
				12318 0
				12319 0
				12320 0
				12321 0
				12322 0
				12323 0
				12324 0
				12325 0
				12326 0
				12327 0
				12328 0
				12329 0
				12330 1020395
				12331 0
				12332 0
				12333 0
				12334 0
				12335 0
				12336 0
				12337 0
				12338 0
				12339 0
				12340 0
				12341 0
				12342 0
				12343 0
				12344 0
				12345 1010680
				12346 0
				12347 0
				12348 0
				12349 0
				12350 0
				12351 0
				12352 0
				12353 0
				12354 0
				12355 0
				12356 0
				12357 0
				12358 0
				12359 0
				12360 1000501
				12361 0
				12362 0
				12363 0
				12364 0
				12365 0
				12366 0
				12367 0
				12368 0
				12369 0
				12370 0
				12371 0
				12372 0
				12373 0
				12374 0
				12375 1017467
				12376 0
				12377 0
				12378 0
				12379 0
				12380 0
				12381 0
				12382 0
				12383 0
				12384 0
				12385 0
				12386 0
				12387 0
				12388 0
				12389 0
				12390 1018482
				12391 0
				12392 0
				12393 0
				12394 0
				12395 0
				12396 0
				12397 0
				12398 0
				12399 0
				12400 0
				12401 0
				12402 0
				12403 0
				12404 0
				12405 1022346
				12406 0
				12407 0
				12408 0
				12409 0
				12410 0
				12411 0
				12412 0
				12413 0
				12414 0
				12415 0
				12416 0
				12417 0
				12418 0
				12419 0
				12420 998953
				12421 0
				12422 0
				12423 0
				12424 0
				12425 0
				12426 0
				12427 0
				12428 0
				12429 0
				12430 0
				12431 0
				12432 0
				12433 0
				12434 0
				12435 1044409
				12436 0
				12437 0
				12438 0
				12439 0
				12440 0
				12441 0
				12442 0
				12443 0
				12444 0
				12445 0
				12446 0
				12447 0
				12448 0
				12449 0
				12450 1050672
				12451 0
				12452 0
				12453 0
				12454 0
				12455 0
				12456 0
				12457 0
				12458 0
				12459 0
				12460 0
				12461 0
				12462 0
				12463 0
				12464 0
				12465 1006104
				12466 0
				12467 0
				12468 0
				12469 0
				12470 0
				12471 0
				12472 0
				12473 0
				12474 0
				12475 0
				12476 0
				12477 0
				12478 0
				12479 0
				12480 1017698
				12481 0
				12482 0
				12483 0
				12484 0
				12485 0
				12486 0
				12487 0
				12488 0
				12489 0
				12490 0
				12491 0
				12492 0
				12493 0
				12494 0
				12495 1021953
				12496 0
				12497 0
				12498 0
				12499 0
				12500 0
				12501 0
				12502 0
				12503 0
				12504 0
				12505 0
				12506 0
				12507 0
				12508 0
				12509 0
				12510 1026043
				12511 0
				12512 0
				12513 0
				12514 0
				12515 0
				12516 0
				12517 0
				12518 0
				12519 0
				12520 0
				12521 0
				12522 0
				12523 0
				12524 0
				12525 1041239
				12526 0
				12527 0
				12528 0
				12529 0
				12530 0
				12531 0
				12532 0
				12533 0
				12534 0
				12535 0
				12536 0
				12537 0
				12538 0
				12539 0
				12540 1063925
				12541 0
				12542 0
				12543 0
				12544 0
				12545 0
				12546 0
				12547 0
				12548 0
				12549 0
				12550 0
				12551 0
				12552 0
				12553 0
				12554 0
				12555 1017820
				12556 0
				12557 0
				12558 0
				12559 0
				12560 0
				12561 0
				12562 0
				12563 0
				12564 0
				12565 0
				12566 0
				12567 0
				12568 0
				12569 0
				12570 1060609
				12571 0
				12572 0
				12573 0
				12574 0
				12575 0
				12576 0
				12577 0
				12578 0
				12579 0
				12580 0
				12581 0
				12582 0
				12583 0
				12584 0
				12585 1058957
				12586 0
				12587 0
				12588 0
				12589 0
				12590 0
				12591 0
				12592 0
				12593 0
				12594 0
				12595 0
				12596 0
				12597 0
				12598 0
				12599 0
				12600 1023695
				12601 0
				12602 0
				12603 0
				12604 0
				12605 0
				12606 0
				12607 0
				12608 0
				12609 0
				12610 0
				12611 0
				12612 0
				12613 0
				12614 0
				12615 1047725
				12616 0
				12617 0
				12618 0
				12619 0
				12620 0
				12621 0
				12622 0
				12623 0
				12624 0
				12625 0
				12626 0
				12627 0
				12628 0
				12629 0
				12630 1017286
				12631 0
				12632 0
				12633 0
				12634 0
				12635 0
				12636 0
				12637 0
				12638 0
				12639 0
				12640 0
				12641 0
				12642 0
				12643 0
				12644 0
				12645 1036403
				12646 0
				12647 0
				12648 0
				12649 0
				12650 0
				12651 0
				12652 0
				12653 0
				12654 0
				12655 0
				12656 0
				12657 0
				12658 0
				12659 0
				12660 1025445
				12661 0
				12662 0
				12663 0
				12664 0
				12665 0
				12666 0
				12667 0
				12668 0
				12669 0
				12670 0
				12671 0
				12672 0
				12673 0
				12674 0
				12675 1014011
				12676 0
				12677 0
				12678 0
				12679 0
				12680 0
				12681 0
				12682 0
				12683 0
				12684 0
				12685 0
				12686 0
				12687 0
				12688 0
				12689 0
				12690 1035531
				12691 0
				12692 0
				12693 0
				12694 0
				12695 0
				12696 0
				12697 0
				12698 0
				12699 0
				12700 0
				12701 0
				12702 0
				12703 0
				12704 0
				12705 1017068
				12706 0
				12707 0
				12708 0
				12709 0
				12710 0
				12711 0
				12712 0
				12713 0
				12714 0
				12715 0
				12716 0
				12717 0
				12718 0
				12719 0
				12720 1018206
				12721 0
				12722 0
				12723 0
				12724 0
				12725 0
				12726 0
				12727 0
				12728 0
				12729 0
				12730 0
				12731 0
				12732 0
				12733 0
				12734 0
				12735 1010400
				12736 0
				12737 0
				12738 0
				12739 0
				12740 0
				12741 0
				12742 0
				12743 0
				12744 0
				12745 0
				12746 0
				12747 0
				12748 0
				12749 0
				12750 1001786
				12751 0
				12752 0
				12753 0
				12754 0
				12755 0
				12756 0
				12757 0
				12758 0
				12759 0
				12760 0
				12761 0
				12762 0
				12763 0
				12764 0
				12765 994466
				12766 0
				12767 0
				12768 0
				12769 0
				12770 0
				12771 0
				12772 0
				12773 0
				12774 0
				12775 0
				12776 0
				12777 0
				12778 0
				12779 0
				12780 1017083
				12781 0
				12782 0
				12783 0
				12784 0
				12785 0
				12786 0
				12787 0
				12788 0
				12789 0
				12790 0
				12791 0
				12792 0
				12793 0
				12794 0
				12795 1026611
				12796 0
				12797 0
				12798 0
				12799 0
				12800 0
				12801 0
				12802 0
				12803 0
				12804 0
				12805 0
				12806 0
				12807 0
				12808 0
				12809 0
				12810 1015379
				12811 0
				12812 0
				12813 0
				12814 0
				12815 0
				12816 0
				12817 0
				12818 0
				12819 0
				12820 0
				12821 0
				12822 0
				12823 0
				12824 0
				12825 1067955
				12826 0
				12827 0
				12828 0
				12829 0
				12830 0
				12831 0
				12832 0
				12833 0
				12834 0
				12835 0
				12836 0
				12837 0
				12838 0
				12839 0
				12840 1078039
				12841 0
				12842 0
				12843 0
				12844 0
				12845 0
				12846 0
				12847 0
				12848 0
				12849 0
				12850 0
				12851 0
				12852 0
				12853 0
				12854 0
				12855 1016368
				12856 0
				12857 0
				12858 0
				12859 0
				12860 0
				12861 0
				12862 0
				12863 0
				12864 0
				12865 0
				12866 0
				12867 0
				12868 0
				12869 0
				12870 1004090
				12871 0
				12872 0
				12873 0
				12874 0
				12875 0
				12876 0
				12877 0
				12878 0
				12879 0
				12880 0
				12881 0
				12882 0
				12883 0
				12884 0
				12885 1018737
				12886 0
				12887 0
				12888 0
				12889 0
				12890 0
				12891 0
				12892 0
				12893 0
				12894 0
				12895 0
				12896 0
				12897 0
				12898 0
				12899 0
				12900 1006903
				12901 0
				12902 0
				12903 0
				12904 0
				12905 0
				12906 0
				12907 0
				12908 0
				12909 0
				12910 0
				12911 0
				12912 0
				12913 0
				12914 0
				12915 1079451
				12916 0
				12917 0
				12918 0
				12919 0
				12920 0
				12921 0
				12922 0
				12923 0
				12924 0
				12925 0
				12926 0
				12927 0
				12928 0
				12929 0
				12930 996081
				12931 0
				12932 0
				12933 0
				12934 0
				12935 0
				12936 0
				12937 0
				12938 0
				12939 0
				12940 0
				12941 0
				12942 0
				12943 0
				12944 0
				12945 1072565
				12946 0
				12947 0
				12948 0
				12949 0
				12950 0
				12951 0
				12952 0
				12953 0
				12954 0
				12955 0
				12956 0
				12957 0
				12958 0
				12959 0
				12960 1071262
				12961 0
				12962 0
				12963 0
				12964 0
				12965 0
				12966 0
				12967 0
				12968 0
				12969 0
				12970 0
				12971 0
				12972 0
				12973 0
				12974 0
				12975 1049301
				12976 0
				12977 0
				12978 0
				12979 0
				12980 0
				12981 0
				12982 0
				12983 0
				12984 0
				12985 0
				12986 0
				12987 0
				12988 0
				12989 0
				12990 1035257
				12991 0
				12992 0
				12993 0
				12994 0
				12995 0
				12996 0
				12997 0
				12998 0
				12999 0
				13000 0
				13001 0
				13002 0
				13003 0
				13004 0
				13005 1027099
				13006 0
				13007 0
				13008 0
				13009 0
				13010 0
				13011 0
				13012 0
				13013 0
				13014 0
				13015 0
				13016 0
				13017 0
				13018 0
				13019 0
				13020 1038765
				13021 0
				13022 0
				13023 0
				13024 0
				13025 0
				13026 0
				13027 0
				13028 0
				13029 0
				13030 0
				13031 0
				13032 0
				13033 0
				13034 0
				13035 1019465
				13036 0
				13037 0
				13038 0
				13039 0
				13040 0
				13041 0
				13042 0
				13043 0
				13044 0
				13045 0
				13046 0
				13047 0
				13048 0
				13049 0
				13050 1006130
				13051 0
				13052 0
				13053 0
				13054 0
				13055 0
				13056 0
				13057 0
				13058 0
				13059 0
				13060 0
				13061 0
				13062 0
				13063 0
				13064 0
				13065 1069000
				13066 0
				13067 0
				13068 0
				13069 0
				13070 0
				13071 0
				13072 0
				13073 0
				13074 0
				13075 0
				13076 0
				13077 0
				13078 0
				13079 0
				13080 1072480
				13081 0
				13082 0
				13083 0
				13084 0
				13085 0
				13086 0
				13087 0
				13088 0
				13089 0
				13090 0
				13091 0
				13092 0
				13093 0
				13094 0
				13095 1076365
				13096 0
				13097 0
				13098 0
				13099 0
				13100 0
				13101 0
				13102 0
				13103 0
				13104 0
				13105 0
				13106 0
				13107 0
				13108 0
				13109 0
				13110 1046611
				13111 0
				13112 0
				13113 0
				13114 0
				13115 0
				13116 0
				13117 0
				13118 0
				13119 0
				13120 0
				13121 0
				13122 0
				13123 0
				13124 0
				13125 1035557
				13126 0
				13127 0
				13128 0
				13129 0
				13130 0
				13131 0
				13132 0
				13133 0
				13134 0
				13135 0
				13136 0
				13137 0
				13138 0
				13139 0
				13140 1036711
				13141 0
				13142 0
				13143 0
				13144 0
				13145 0
				13146 0
				13147 0
				13148 0
				13149 0
				13150 0
				13151 0
				13152 0
				13153 0
				13154 0
				13155 1014398
				13156 0
				13157 0
				13158 0
				13159 0
				13160 0
				13161 0
				13162 0
				13163 0
				13164 0
				13165 0
				13166 0
				13167 0
				13168 0
				13169 0
				13170 1056818
				13171 0
				13172 0
				13173 0
				13174 0
				13175 0
				13176 0
				13177 0
				13178 0
				13179 0
				13180 0
				13181 0
				13182 0
				13183 0
				13184 0
				13185 1040313
				13186 0
				13187 0
				13188 0
				13189 0
				13190 0
				13191 0
				13192 0
				13193 0
				13194 0
				13195 0
				13196 0
				13197 0
				13198 0
				13199 0
				13200 1060838
				13201 0
				13202 0
				13203 0
				13204 0
				13205 0
				13206 0
				13207 0
				13208 0
				13209 0
				13210 0
				13211 0
				13212 0
				13213 0
				13214 0
				13215 1041562
				13216 0
				13217 0
				13218 0
				13219 0
				13220 0
				13221 0
				13222 0
				13223 0
				13224 0
				13225 0
				13226 0
				13227 0
				13228 0
				13229 0
				13230 1027657
				13231 0
				13232 0
				13233 0
				13234 0
				13235 0
				13236 0
				13237 0
				13238 0
				13239 0
				13240 0
				13241 0
				13242 0
				13243 0
				13244 0
				13245 1062193
				13246 0
				13247 0
				13248 0
				13249 0
				13250 0
				13251 0
				13252 0
				13253 0
				13254 0
				13255 0
				13256 0
				13257 0
				13258 0
				13259 0
				13260 1013370
				13261 0
				13262 0
				13263 0
				13264 0
				13265 0
				13266 0
				13267 0
				13268 0
				13269 0
				13270 0
				13271 0
				13272 0
				13273 0
				13274 0
				13275 1031164
				13276 0
				13277 0
				13278 0
				13279 0
				13280 0
				13281 0
				13282 0
				13283 0
				13284 0
				13285 0
				13286 0
				13287 0
				13288 0
				13289 0
				13290 1034474
				13291 0
				13292 0
				13293 0
				13294 0
				13295 0
				13296 0
				13297 0
				13298 0
				13299 0
				13300 0
				13301 0
				13302 0
				13303 0
				13304 0
				13305 1055657
				13306 0
				13307 0
				13308 0
				13309 0
				13310 0
				13311 0
				13312 0
				13313 0
				13314 0
				13315 0
				13316 0
				13317 0
				13318 0
				13319 0
				13320 1042536
				13321 0
				13322 0
				13323 0
				13324 0
				13325 0
				13326 0
				13327 0
				13328 0
				13329 0
				13330 0
				13331 0
				13332 0
				13333 0
				13334 0
				13335 1043885
				13336 0
				13337 0
				13338 0
				13339 0
				13340 0
				13341 0
				13342 0
				13343 0
				13344 0
				13345 0
				13346 0
				13347 0
				13348 0
				13349 0
				13350 1045831
				13351 0
				13352 0
				13353 0
				13354 0
				13355 0
				13356 0
				13357 0
				13358 0
				13359 0
				13360 0
				13361 0
				13362 0
				13363 0
				13364 0
				13365 1044330
				13366 0
				13367 0
				13368 0
				13369 0
				13370 0
				13371 0
				13372 0
				13373 0
				13374 0
				13375 0
				13376 0
				13377 0
				13378 0
				13379 0
				13380 1054512
				13381 0
				13382 0
				13383 0
				13384 0
				13385 0
				13386 0
				13387 0
				13388 0
				13389 0
				13390 0
				13391 0
				13392 0
				13393 0
				13394 0
				13395 1024195
				13396 0
				13397 0
				13398 0
				13399 0
				13400 0
				13401 0
				13402 0
				13403 0
				13404 0
				13405 0
				13406 0
				13407 0
				13408 0
				13409 0
				13410 1031125
				13411 0
				13412 0
				13413 0
				13414 0
				13415 0
				13416 0
				13417 0
				13418 0
				13419 0
				13420 0
				13421 0
				13422 0
				13423 0
				13424 0
				13425 1026116
				13426 0
				13427 0
				13428 0
				13429 0
				13430 0
				13431 0
				13432 0
				13433 0
				13434 0
				13435 0
				13436 0
				13437 0
				13438 0
				13439 0
				13440 1027636
				13441 0
				13442 0
				13443 0
				13444 0
				13445 0
				13446 0
				13447 0
				13448 0
				13449 0
				13450 0
				13451 0
				13452 0
				13453 0
				13454 0
				13455 1038007
				13456 0
				13457 0
				13458 0
				13459 0
				13460 0
				13461 0
				13462 0
				13463 0
				13464 0
				13465 0
				13466 0
				13467 0
				13468 0
				13469 0
				13470 1078983
				13471 0
				13472 0
				13473 0
				13474 0
				13475 0
				13476 0
				13477 0
				13478 0
				13479 0
				13480 0
				13481 0
				13482 0
				13483 0
				13484 0
				13485 1074268
				13486 0
				13487 0
				13488 0
				13489 0
				13490 0
				13491 0
				13492 0
				13493 0
				13494 0
				13495 0
				13496 0
				13497 0
				13498 0
				13499 0
				13500 1054641
				13501 0
				13502 0
				13503 0
				13504 0
				13505 0
				13506 0
				13507 0
				13508 0
				13509 0
				13510 0
				13511 0
				13512 0
				13513 0
				13514 0
				13515 1021969
				13516 0
				13517 0
				13518 0
				13519 0
				13520 0
				13521 0
				13522 0
				13523 0
				13524 0
				13525 0
				13526 0
				13527 0
				13528 0
				13529 0
				13530 1026022
				13531 0
				13532 0
				13533 0
				13534 0
				13535 0
				13536 0
				13537 0
				13538 0
				13539 0
				13540 0
				13541 0
				13542 0
				13543 0
				13544 0
				13545 1076241
				13546 0
				13547 0
				13548 0
				13549 0
				13550 0
				13551 0
				13552 0
				13553 0
				13554 0
				13555 0
				13556 0
				13557 0
				13558 0
				13559 0
				13560 1005062
				13561 0
				13562 0
				13563 0
				13564 0
				13565 0
				13566 0
				13567 0
				13568 0
				13569 0
				13570 0
				13571 0
				13572 0
				13573 0
				13574 0
				13575 1063323
				13576 0
				13577 0
				13578 0
				13579 0
				13580 0
				13581 0
				13582 0
				13583 0
				13584 0
				13585 0
				13586 0
				13587 0
				13588 0
				13589 0
				13590 1049137
				13591 0
				13592 0
				13593 0
				13594 0
				13595 0
				13596 0
				13597 0
				13598 0
				13599 0
				13600 0
				13601 0
				13602 0
				13603 0
				13604 0
				13605 1047877
				13606 0
				13607 0
				13608 0
				13609 0
				13610 0
				13611 0
				13612 0
				13613 0
				13614 0
				13615 0
				13616 0
				13617 0
				13618 0
				13619 0
				13620 1027558
				13621 0
				13622 0
				13623 0
				13624 0
				13625 0
				13626 0
				13627 0
				13628 0
				13629 0
				13630 0
				13631 0
				13632 0
				13633 0
				13634 0
				13635 1066020
				13636 0
				13637 0
				13638 0
				13639 0
				13640 0
				13641 0
				13642 0
				13643 0
				13644 0
				13645 0
				13646 0
				13647 0
				13648 0
				13649 0
				13650 1088659
				13651 0
				13652 0
				13653 0
				13654 0
				13655 0
				13656 0
				13657 0
				13658 0
				13659 0
				13660 0
				13661 0
				13662 0
				13663 0
				13664 0
				13665 1019870
				13666 0
				13667 0
				13668 0
				13669 0
				13670 0
				13671 0
				13672 0
				13673 0
				13674 0
				13675 0
				13676 0
				13677 0
				13678 0
				13679 0
				13680 1003350
				13681 0
				13682 0
				13683 0
				13684 0
				13685 0
				13686 0
				13687 0
				13688 0
				13689 0
				13690 0
				13691 0
				13692 0
				13693 0
				13694 0
				13695 1055938
				13696 0
				13697 0
				13698 0
				13699 0
				13700 0
				13701 0
				13702 0
				13703 0
				13704 0
				13705 0
				13706 0
				13707 0
				13708 0
				13709 0
				13710 1033812
				13711 0
				13712 0
				13713 0
				13714 0
				13715 0
				13716 0
				13717 0
				13718 0
				13719 0
				13720 0
				13721 0
				13722 0
				13723 0
				13724 0
				13725 1031464
				13726 0
				13727 0
				13728 0
				13729 0
				13730 0
				13731 0
				13732 0
				13733 0
				13734 0
				13735 0
				13736 0
				13737 0
				13738 0
				13739 0
				13740 1036659
				13741 0
				13742 0
				13743 0
				13744 0
				13745 0
				13746 0
				13747 0
				13748 0
				13749 0
				13750 0
				13751 0
				13752 0
				13753 0
				13754 0
				13755 1006473
				13756 0
				13757 0
				13758 0
				13759 0
				13760 0
				13761 0
				13762 0
				13763 0
				13764 0
				13765 0
				13766 0
				13767 0
				13768 0
				13769 0
				13770 1016290
				13771 0
				13772 0
				13773 0
				13774 0
				13775 0
				13776 0
				13777 0
				13778 0
				13779 0
				13780 0
				13781 0
				13782 0
				13783 0
				13784 0
				13785 1021750
				13786 0
				13787 0
				13788 0
				13789 0
				13790 0
				13791 0
				13792 0
				13793 0
				13794 0
				13795 0
				13796 0
				13797 0
				13798 0
				13799 0
				13800 1030596
				13801 0
				13802 0
				13803 0
				13804 0
				13805 0
				13806 0
				13807 0
				13808 0
				13809 0
				13810 0
				13811 0
				13812 0
				13813 0
				13814 0
				13815 1015554
				13816 0
				13817 0
				13818 0
				13819 0
				13820 0
				13821 0
				13822 0
				13823 0
				13824 0
				13825 0
				13826 0
				13827 0
				13828 0
				13829 0
				13830 1020330
				13831 0
				13832 0
				13833 0
				13834 0
				13835 0
				13836 0
				13837 0
				13838 0
				13839 0
				13840 0
				13841 0
				13842 0
				13843 0
				13844 0
				13845 1016214
				13846 0
				13847 0
				13848 0
				13849 0
				13850 0
				13851 0
				13852 0
				13853 0
				13854 0
				13855 0
				13856 0
				13857 0
				13858 0
				13859 0
				13860 999975
				13861 0
				13862 0
				13863 0
				13864 0
				13865 0
				13866 0
				13867 0
				13868 0
				13869 0
				13870 0
				13871 0
				13872 0
				13873 0
				13874 0
				13875 1030778
				13876 0
				13877 0
				13878 0
				13879 0
				13880 0
				13881 0
				13882 0
				13883 0
				13884 0
				13885 0
				13886 0
				13887 0
				13888 0
				13889 0
				13890 1026393
				13891 0
				13892 0
				13893 0
				13894 0
				13895 0
				13896 0
				13897 0
				13898 0
				13899 0
				13900 0
				13901 0
				13902 0
				13903 0
				13904 0
				13905 1038318
				13906 0
				13907 0
				13908 0
				13909 0
				13910 0
				13911 0
				13912 0
				13913 0
				13914 0
				13915 0
				13916 0
				13917 0
				13918 0
				13919 0
				13920 1062196
				13921 0
				13922 0
				13923 0
				13924 0
				13925 0
				13926 0
				13927 0
				13928 0
				13929 0
				13930 0
				13931 0
				13932 0
				13933 0
				13934 0
				13935 1055930
				13936 0
				13937 0
				13938 0
				13939 0
				13940 0
				13941 0
				13942 0
				13943 0
				13944 0
				13945 0
				13946 0
				13947 0
				13948 0
				13949 0
				13950 1022264
				13951 0
				13952 0
				13953 0
				13954 0
				13955 0
				13956 0
				13957 0
				13958 0
				13959 0
				13960 0
				13961 0
				13962 0
				13963 0
				13964 0
				13965 1028217
				13966 0
				13967 0
				13968 0
				13969 0
				13970 0
				13971 0
				13972 0
				13973 0
				13974 0
				13975 0
				13976 0
				13977 0
				13978 0
				13979 0
				13980 1047901
				13981 0
				13982 0
				13983 0
				13984 0
				13985 0
				13986 0
				13987 0
				13988 0
				13989 0
				13990 0
				13991 0
				13992 0
				13993 0
				13994 0
				13995 1016946
				13996 0
				13997 0
				13998 0
				13999 0
				14000 0
				14001 0
				14002 0
				14003 0
				14004 0
				14005 0
				14006 0
				14007 0
				14008 0
				14009 0
				14010 1042434
				14011 0
				14012 0
				14013 0
				14014 0
				14015 0
				14016 0
				14017 0
				14018 0
				14019 0
				14020 0
				14021 0
				14022 0
				14023 0
				14024 0
				14025 1020781
				14026 0
				14027 0
				14028 0
				14029 0
				14030 0
				14031 0
				14032 0
				14033 0
				14034 0
				14035 0
				14036 0
				14037 0
				14038 0
				14039 0
				14040 1005468
				14041 0
				14042 0
				14043 0
				14044 0
				14045 0
				14046 0
				14047 0
				14048 0
				14049 0
				14050 0
				14051 0
				14052 0
				14053 0
				14054 0
				14055 1039625
				14056 0
				14057 0
				14058 0
				14059 0
				14060 0
				14061 0
				14062 0
				14063 0
				14064 0
				14065 0
				14066 0
				14067 0
				14068 0
				14069 0
				14070 1036793
				14071 0
				14072 0
				14073 0
				14074 0
				14075 0
				14076 0
				14077 0
				14078 0
				14079 0
				14080 0
				14081 0
				14082 0
				14083 0
				14084 0
				14085 1032747
				14086 0
				14087 0
				14088 0
				14089 0
				14090 0
				14091 0
				14092 0
				14093 0
				14094 0
				14095 0
				14096 0
				14097 0
				14098 0
				14099 0
				14100 1063871
				14101 0
				14102 0
				14103 0
				14104 0
				14105 0
				14106 0
				14107 0
				14108 0
				14109 0
				14110 0
				14111 0
				14112 0
				14113 0
				14114 0
				14115 1029088
				14116 0
				14117 0
				14118 0
				14119 0
				14120 0
				14121 0
				14122 0
				14123 0
				14124 0
				14125 0
				14126 0
				14127 0
				14128 0
				14129 0
				14130 1073160
				14131 0
				14132 0
				14133 0
				14134 0
				14135 0
				14136 0
				14137 0
				14138 0
				14139 0
				14140 0
				14141 0
				14142 0
				14143 0
				14144 0
				14145 1061368
				14146 0
				14147 0
				14148 0
				14149 0
				14150 0
				14151 0
				14152 0
				14153 0
				14154 0
				14155 0
				14156 0
				14157 0
				14158 0
				14159 0
				14160 1061524
				14161 0
				14162 0
				14163 0
				14164 0
				14165 0
				14166 0
				14167 0
				14168 0
				14169 0
				14170 0
				14171 0
				14172 0
				14173 0
				14174 0
				14175 1051280
				14176 0
				14177 0
				14178 0
				14179 0
				14180 0
				14181 0
				14182 0
				14183 0
				14184 0
				14185 0
				14186 0
				14187 0
				14188 0
				14189 0
				14190 1079492
				14191 0
				14192 0
				14193 0
				14194 0
				14195 0
				14196 0
				14197 0
				14198 0
				14199 0
				14200 0
				14201 0
				14202 0
				14203 0
				14204 0
				14205 1035024
				14206 0
				14207 0
				14208 0
				14209 0
				14210 0
				14211 0
				14212 0
				14213 0
				14214 0
				14215 0
				14216 0
				14217 0
				14218 0
				14219 0
				14220 1062488
				14221 0
				14222 0
				14223 0
				14224 0
				14225 0
				14226 0
				14227 0
				14228 0
				14229 0
				14230 0
				14231 0
				14232 0
				14233 0
				14234 0
				14235 1039461
				14236 0
				14237 0
				14238 0
				14239 0
				14240 0
				14241 0
				14242 0
				14243 0
				14244 0
				14245 0
				14246 0
				14247 0
				14248 0
				14249 0
				14250 1027597
				14251 0
				14252 0
				14253 0
				14254 0
				14255 0
				14256 0
				14257 0
				14258 0
				14259 0
				14260 0
				14261 0
				14262 0
				14263 0
				14264 0
				14265 1069959
				14266 0
				14267 0
				14268 0
				14269 0
				14270 0
				14271 0
				14272 0
				14273 0
				14274 0
				14275 0
				14276 0
				14277 0
				14278 0
				14279 0
				14280 1005936
				14281 0
				14282 0
				14283 0
				14284 0
				14285 0
				14286 0
				14287 0
				14288 0
				14289 0
				14290 0
				14291 0
				14292 0
				14293 0
				14294 0
				14295 1067034
				14296 0
				14297 0
				14298 0
				14299 0
				14300 0
				14301 0
				14302 0
				14303 0
				14304 0
				14305 0
				14306 0
				14307 0
				14308 0
				14309 0
				14310 1050893
				14311 0
				14312 0
				14313 0
				14314 0
				14315 0
				14316 0
				14317 0
				14318 0
				14319 0
				14320 0
				14321 0
				14322 0
				14323 0
				14324 0
				14325 1021808
				14326 0
				14327 0
				14328 0
				14329 0
				14330 0
				14331 0
				14332 0
				14333 0
				14334 0
				14335 0
				14336 0
				14337 0
				14338 0
				14339 0
				14340 1043530
				14341 0
				14342 0
				14343 0
				14344 0
				14345 0
				14346 0
				14347 0
				14348 0
				14349 0
				14350 0
				14351 0
				14352 0
				14353 0
				14354 0
				14355 1044783
				14356 0
				14357 0
				14358 0
				14359 0
				14360 0
				14361 0
				14362 0
				14363 0
				14364 0
				14365 0
				14366 0
				14367 0
				14368 0
				14369 0
				14370 1064881
				14371 0
				14372 0
				14373 0
				14374 0
				14375 0
				14376 0
				14377 0
				14378 0
				14379 0
				14380 0
				14381 0
				14382 0
				14383 0
				14384 0
				14385 1026781
				14386 0
				14387 0
				14388 0
				14389 0
				14390 0
				14391 0
				14392 0
				14393 0
				14394 0
				14395 0
				14396 0
				14397 0
				14398 0
				14399 0
				14400 1013035
				14401 0
				14402 0
				14403 0
				14404 0
				14405 0
				14406 0
				14407 0
				14408 0
				14409 0
				14410 0
				14411 0
				14412 0
				14413 0
				14414 0
				14415 1016666
				14416 0
				14417 0
				14418 0
				14419 0
				14420 0
				14421 0
				14422 0
				14423 0
				14424 0
				14425 0
				14426 0
				14427 0
				14428 0
				14429 0
				14430 1055708
				14431 0
				14432 0
				14433 0
				14434 0
				14435 0
				14436 0
				14437 0
				14438 0
				14439 0
				14440 0
				14441 0
				14442 0
				14443 0
				14444 0
				14445 1001582
				14446 0
				14447 0
				14448 0
				14449 0
				14450 0
				14451 0
				14452 0
				14453 0
				14454 0
				14455 0
				14456 0
				14457 0
				14458 0
				14459 0
				14460 1037612
				14461 0
				14462 0
				14463 0
				14464 0
				14465 0
				14466 0
				14467 0
				14468 0
				14469 0
				14470 0
				14471 0
				14472 0
				14473 0
				14474 0
				14475 1029919
				14476 0
				14477 0
				14478 0
				14479 0
				14480 0
				14481 0
				14482 0
				14483 0
				14484 0
				14485 0
				14486 0
				14487 0
				14488 0
				14489 0
				14490 1034196
				14491 0
				14492 0
				14493 0
				14494 0
				14495 0
				14496 0
				14497 0
				14498 0
				14499 0
				14500 0
				14501 0
				14502 0
				14503 0
				14504 0
				14505 1051882
				14506 0
				14507 0
				14508 0
				14509 0
				14510 0
				14511 0
				14512 0
				14513 0
				14514 0
				14515 0
				14516 0
				14517 0
				14518 0
				14519 0
				14520 1013541
				14521 0
				14522 0
				14523 0
				14524 0
				14525 0
				14526 0
				14527 0
				14528 0
				14529 0
				14530 0
				14531 0
				14532 0
				14533 0
				14534 0
				14535 1006276
				14536 0
				14537 0
				14538 0
				14539 0
				14540 0
				14541 0
				14542 0
				14543 0
				14544 0
				14545 0
				14546 0
				14547 0
				14548 0
				14549 0
				14550 995597
				14551 0
				14552 0
				14553 0
				14554 0
				14555 0
				14556 0
				14557 0
				14558 0
				14559 0
				14560 0
				14561 0
				14562 0
				14563 0
				14564 0
				14565 1037478
				14566 0
				14567 0
				14568 0
				14569 0
				14570 0
				14571 0
				14572 0
				14573 0
				14574 0
				14575 0
				14576 0
				14577 0
				14578 0
				14579 0
				14580 1030115
				14581 0
				14582 0
				14583 0
				14584 0
				14585 0
				14586 0
				14587 0
				14588 0
				14589 0
				14590 0
				14591 0
				14592 0
				14593 0
				14594 0
				14595 1041623
				14596 0
				14597 0
				14598 0
				14599 0
				14600 0
				14601 0
				14602 0
				14603 0
				14604 0
				14605 0
				14606 0
				14607 0
				14608 0
				14609 0
				14610 1049011
				14611 0
				14612 0
				14613 0
				14614 0
				14615 0
				14616 0
				14617 0
				14618 0
				14619 0
				14620 0
				14621 0
				14622 0
				14623 0
				14624 0
				14625 1029913
				14626 0
				14627 0
				14628 0
				14629 0
				14630 0
				14631 0
				14632 0
				14633 0
				14634 0
				14635 0
				14636 0
				14637 0
				14638 0
				14639 0
				14640 1020189
				14641 0
				14642 0
				14643 0
				14644 0
				14645 0
				14646 0
				14647 0
				14648 0
				14649 0
				14650 0
				14651 0
				14652 0
				14653 0
				14654 0
				14655 1009461
				14656 0
				14657 0
				14658 0
				14659 0
				14660 0
				14661 0
				14662 0
				14663 0
				14664 0
				14665 0
				14666 0
				14667 0
				14668 0
				14669 0
				14670 1015036
				14671 0
				14672 0
				14673 0
				14674 0
				14675 0
				14676 0
				14677 0
				14678 0
				14679 0
				14680 0
				14681 0
				14682 0
				14683 0
				14684 0
				14685 1066627
				14686 0
				14687 0
				14688 0
				14689 0
				14690 0
				14691 0
				14692 0
				14693 0
				14694 0
				14695 0
				14696 0
				14697 0
				14698 0
				14699 0
				14700 1062045
				14701 0
				14702 0
				14703 0
				14704 0
				14705 0
				14706 0
				14707 0
				14708 0
				14709 0
				14710 0
				14711 0
				14712 0
				14713 0
				14714 0
				14715 1043849
				14716 0
				14717 0
				14718 0
				14719 0
				14720 0
				14721 0
				14722 0
				14723 0
				14724 0
				14725 0
				14726 0
				14727 0
				14728 0
				14729 0
				14730 1057722
				14731 0
				14732 0
				14733 0
				14734 0
				14735 0
				14736 0
				14737 0
				14738 0
				14739 0
				14740 0
				14741 0
				14742 0
				14743 0
				14744 0
				14745 1046929
				14746 0
				14747 0
				14748 0
				14749 0
				14750 0
				14751 0
				14752 0
				14753 0
				14754 0
				14755 0
				14756 0
				14757 0
				14758 0
				14759 0
				14760 1043680
				14761 0
				14762 0
				14763 0
				14764 0
				14765 0
				14766 0
				14767 0
				14768 0
				14769 0
				14770 0
				14771 0
				14772 0
				14773 0
				14774 0
				14775 1027321
				14776 0
				14777 0
				14778 0
				14779 0
				14780 0
				14781 0
				14782 0
				14783 0
				14784 0
				14785 0
				14786 0
				14787 0
				14788 0
				14789 0
				14790 1010234
				14791 0
				14792 0
				14793 0
				14794 0
				14795 0
				14796 0
				14797 0
				14798 0
				14799 0
				14800 0
				14801 0
				14802 0
				14803 0
				14804 0
				14805 1013224
				14806 0
				14807 0
				14808 0
				14809 0
				14810 0
				14811 0
				14812 0
				14813 0
				14814 0
				14815 0
				14816 0
				14817 0
				14818 0
				14819 0
				14820 1026153
				14821 0
				14822 0
				14823 0
				14824 0
				14825 0
				14826 0
				14827 0
				14828 0
				14829 0
				14830 0
				14831 0
				14832 0
				14833 0
				14834 0
				14835 1031251
				14836 0
				14837 0
				14838 0
				14839 0
				14840 0
				14841 0
				14842 0
				14843 0
				14844 0
				14845 0
				14846 0
				14847 0
				14848 0
				14849 0
				14850 1030827
				14851 0
				14852 0
				14853 0
				14854 0
				14855 0
				14856 0
				14857 0
				14858 0
				14859 0
				14860 0
				14861 0
				14862 0
				14863 0
				14864 0
				14865 1018763
				14866 0
				14867 0
				14868 0
				14869 0
				14870 0
				14871 0
				14872 0
				14873 0
				14874 0
				14875 0
				14876 0
				14877 0
				14878 0
				14879 0
				14880 1054271
				14881 0
				14882 0
				14883 0
				14884 0
				14885 0
				14886 0
				14887 0
				14888 0
				14889 0
				14890 0
				14891 0
				14892 0
				14893 0
				14894 0
				14895 1027510
				14896 0
				14897 0
				14898 0
				14899 0
				14900 0
				14901 0
				14902 0
				14903 0
				14904 0
				14905 0
				14906 0
				14907 0
				14908 0
				14909 0
				14910 1013559
				14911 0
				14912 0
				14913 0
				14914 0
				14915 0
				14916 0
				14917 0
				14918 0
				14919 0
				14920 0
				14921 0
				14922 0
				14923 0
				14924 0
				14925 1034668
				14926 0
				14927 0
				14928 0
				14929 0
				14930 0
				14931 0
				14932 0
				14933 0
				14934 0
				14935 0
				14936 0
				14937 0
				14938 0
				14939 0
				14940 1057172
				14941 0
				14942 0
				14943 0
				14944 0
				14945 0
				14946 0
				14947 0
				14948 0
				14949 0
				14950 0
				14951 0
				14952 0
				14953 0
				14954 0
				14955 1037641
				14956 0
				14957 0
				14958 0
				14959 0
				14960 0
				14961 0
				14962 0
				14963 0
				14964 0
				14965 0
				14966 0
				14967 0
				14968 0
				14969 0
				14970 1017457
				14971 0
				14972 0
				14973 0
				14974 0
				14975 0
				14976 0
				14977 0
				14978 0
				14979 0
				14980 0
				14981 0
				14982 0
				14983 0
				14984 0
				14985 1037129
				14986 0
				14987 0
				14988 0
				14989 0
				14990 0
				14991 0
				14992 0
				14993 0
				14994 0
				14995 0
				14996 0
				14997 0
				14998 0
				14999 0
				15000 1009579
				15001 0
				15002 0
				15003 0
				15004 0
				15005 0
				15006 0
				15007 0
				15008 0
				15009 0
				15010 0
				15011 0
				15012 0
				15013 0
				15014 0
				15015 1027641
				15016 0
				15017 0
				15018 0
				15019 0
				15020 0
				15021 0
				15022 0
				15023 0
				15024 0
				15025 0
				15026 0
				15027 0
				15028 0
				15029 0
				15030 1013034
				15031 0
				15032 0
				15033 0
				15034 0
				15035 0
				15036 0
				15037 0
				15038 0
				15039 0
				15040 0
				15041 0
				15042 0
				15043 0
				15044 0
				15045 1021094
				15046 0
				15047 0
				15048 0
				15049 0
				15050 0
				15051 0
				15052 0
				15053 0
				15054 0
				15055 0
				15056 0
				15057 0
				15058 0
				15059 0
				15060 1024648
				15061 0
				15062 0
				15063 0
				15064 0
				15065 0
				15066 0
				15067 0
				15068 0
				15069 0
				15070 0
				15071 0
				15072 0
				15073 0
				15074 0
				15075 1032351
				15076 0
				15077 0
				15078 0
				15079 0
				15080 0
				15081 0
				15082 0
				15083 0
				15084 0
				15085 0
				15086 0
				15087 0
				15088 0
				15089 0
				15090 1030040
				15091 0
				15092 0
				15093 0
				15094 0
				15095 0
				15096 0
				15097 0
				15098 0
				15099 0
				15100 0
				15101 0
				15102 0
				15103 0
				15104 0
				15105 1007919
				15106 0
				15107 0
				15108 0
				15109 0
				15110 0
				15111 0
				15112 0
				15113 0
				15114 0
				15115 0
				15116 0
				15117 0
				15118 0
				15119 0
				15120 1003548
				15121 0
				15122 0
				15123 0
				15124 0
				15125 0
				15126 0
				15127 0
				15128 0
				15129 0
				15130 0
				15131 0
				15132 0
				15133 0
				15134 0
				15135 1049832
				15136 0
				15137 0
				15138 0
				15139 0
				15140 0
				15141 0
				15142 0
				15143 0
				15144 0
				15145 0
				15146 0
				15147 0
				15148 0
				15149 0
				15150 1043078
				15151 0
				15152 0
				15153 0
				15154 0
				15155 0
				15156 0
				15157 0
				15158 0
				15159 0
				15160 0
				15161 0
				15162 0
				15163 0
				15164 0
				15165 1031176
				15166 0
				15167 0
				15168 0
				15169 0
				15170 0
				15171 0
				15172 0
				15173 0
				15174 0
				15175 0
				15176 0
				15177 0
				15178 0
				15179 0
				15180 1009552
				15181 0
				15182 0
				15183 0
				15184 0
				15185 0
				15186 0
				15187 0
				15188 0
				15189 0
				15190 0
				15191 0
				15192 0
				15193 0
				15194 0
				15195 1013134
				15196 0
				15197 0
				15198 0
				15199 0
				15200 0
				15201 0
				15202 0
				15203 0
				15204 0
				15205 0
				15206 0
				15207 0
				15208 0
				15209 0
				15210 1002254
				15211 0
				15212 0
				15213 0
				15214 0
				15215 0
				15216 0
				15217 0
				15218 0
				15219 0
				15220 0
				15221 0
				15222 0
				15223 0
				15224 0
				15225 1008468
				15226 0
				15227 0
				15228 0
				15229 0
				15230 0
				15231 0
				15232 0
				15233 0
				15234 0
				15235 0
				15236 0
				15237 0
				15238 0
				15239 0
				15240 1049390
				15241 0
				15242 0
				15243 0
				15244 0
				15245 0
				15246 0
				15247 0
				15248 0
				15249 0
				15250 0
				15251 0
				15252 0
				15253 0
				15254 0
				15255 1017548
				15256 0
				15257 0
				15258 0
				15259 0
				15260 0
				15261 0
				15262 0
				15263 0
				15264 0
				15265 0
				15266 0
				15267 0
				15268 0
				15269 0
				15270 1021828
				15271 0
				15272 0
				15273 0
				15274 0
				15275 0
				15276 0
				15277 0
				15278 0
				15279 0
				15280 0
				15281 0
				15282 0
				15283 0
				15284 0
				15285 1007819
				15286 0
				15287 0
				15288 0
				15289 0
				15290 0
				15291 0
				15292 0
				15293 0
				15294 0
				15295 0
				15296 0
				15297 0
				15298 0
				15299 0
				15300 1019391
				15301 0
				15302 0
				15303 0
				15304 0
				15305 0
				15306 0
				15307 0
				15308 0
				15309 0
				15310 0
				15311 0
				15312 0
				15313 0
				15314 0
				15315 1017001
				15316 0
				15317 0
				15318 0
				15319 0
				15320 0
				15321 0
				15322 0
				15323 0
				15324 0
				15325 0
				15326 0
				15327 0
				15328 0
				15329 0
				15330 1059973
				15331 0
				15332 0
				15333 0
				15334 0
				15335 0
				15336 0
				15337 0
				15338 0
				15339 0
				15340 0
				15341 0
				15342 0
				15343 0
				15344 0
				15345 1013482
				15346 0
				15347 0
				15348 0
				15349 0
				15350 0
				15351 0
				15352 0
				15353 0
				15354 0
				15355 0
				15356 0
				15357 0
				15358 0
				15359 0
				15360 990447
				15361 0
				15362 0
				15363 0
				15364 0
				15365 0
				15366 0
				15367 0
				15368 0
				15369 0
				15370 0
				15371 0
				15372 0
				15373 0
				15374 0
				15375 1034187
				15376 0
				15377 0
				15378 0
				15379 0
				15380 0
				15381 0
				15382 0
				15383 0
				15384 0
				15385 0
				15386 0
				15387 0
				15388 0
				15389 0
				15390 1005132
				15391 0
				15392 0
				15393 0
				15394 0
				15395 0
				15396 0
				15397 0
				15398 0
				15399 0
				15400 0
				15401 0
				15402 0
				15403 0
				15404 0
				15405 999133
				15406 0
				15407 0
				15408 0
				15409 0
				15410 0
				15411 0
				15412 0
				15413 0
				15414 0
				15415 0
				15416 0
				15417 0
				15418 0
				15419 0
				15420 998280
				15421 0
				15422 0
				15423 0
				15424 0
				15425 0
				15426 0
				15427 0
				15428 0
				15429 0
				15430 0
				15431 0
				15432 0
				15433 0
				15434 0
				15435 1006065
				15436 0
				15437 0
				15438 0
				15439 0
				15440 0
				15441 0
				15442 0
				15443 0
				15444 0
				15445 0
				15446 0
				15447 0
				15448 0
				15449 0
				15450 1020022
				15451 0
				15452 0
				15453 0
				15454 0
				15455 0
				15456 0
				15457 0
				15458 0
				15459 0
				15460 0
				15461 0
				15462 0
				15463 0
				15464 0
				15465 1020161
				15466 0
				15467 0
				15468 0
				15469 0
				15470 0
				15471 0
				15472 0
				15473 0
				15474 0
				15475 0
				15476 0
				15477 0
				15478 0
				15479 0
				15480 1021787
				15481 0
				15482 0
				15483 0
				15484 0
				15485 0
				15486 0
				15487 0
				15488 0
				15489 0
				15490 0
				15491 0
				15492 0
				15493 0
				15494 0
				15495 1022032
				15496 0
				15497 0
				15498 0
				15499 0
				15500 0
				15501 0
				15502 0
				15503 0
				15504 0
				15505 0
				15506 0
				15507 0
				15508 0
				15509 0
				15510 1023079
				15511 0
				15512 0
				15513 0
				15514 0
				15515 0
				15516 0
				15517 0
				15518 0
				15519 0
				15520 0
				15521 0
				15522 0
				15523 0
				15524 0
				15525 1018972
				15526 0
				15527 0
				15528 0
				15529 0
				15530 0
				15531 0
				15532 0
				15533 0
				15534 0
				15535 0
				15536 0
				15537 0
				15538 0
				15539 0
				15540 1012104
				15541 0
				15542 0
				15543 0
				15544 0
				15545 0
				15546 0
				15547 0
				15548 0
				15549 0
				15550 0
				15551 0
				15552 0
				15553 0
				15554 0
				15555 1000189
				15556 0
				15557 0
				15558 0
				15559 0
				15560 0
				15561 0
				15562 0
				15563 0
				15564 0
				15565 0
				15566 0
				15567 0
				15568 0
				15569 0
				15570 997790
				15571 0
				15572 0
				15573 0
				15574 0
				15575 0
				15576 0
				15577 0
				15578 0
				15579 0
				15580 0
				15581 0
				15582 0
				15583 0
				15584 0
				15585 1060571
				15586 0
				15587 0
				15588 0
				15589 0
				15590 0
				15591 0
				15592 0
				15593 0
				15594 0
				15595 0
				15596 0
				15597 0
				15598 0
				15599 0
				15600 988103
				15601 0
				15602 0
				15603 0
				15604 0
				15605 0
				15606 0
				15607 0
				15608 0
				15609 0
				15610 0
				15611 0
				15612 0
				15613 0
				15614 0
				15615 1055351
				15616 0
				15617 0
				15618 0
				15619 0
				15620 0
				15621 0
				15622 0
				15623 0
				15624 0
				15625 0
				15626 0
				15627 0
				15628 0
				15629 0
				15630 1066063
				15631 0
				15632 0
				15633 0
				15634 0
				15635 0
				15636 0
				15637 0
				15638 0
				15639 0
				15640 0
				15641 0
				15642 0
				15643 0
				15644 0
				15645 999506
				15646 0
				15647 0
				15648 0
				15649 0
				15650 0
				15651 0
				15652 0
				15653 0
				15654 0
				15655 0
				15656 0
				15657 0
				15658 0
				15659 0
				15660 1048278
				15661 0
				15662 0
				15663 0
				15664 0
				15665 0
				15666 0
				15667 0
				15668 0
				15669 0
				15670 0
				15671 0
				15672 0
				15673 0
				15674 0
				15675 1000523
				15676 0
				15677 0
				15678 0
				15679 0
				15680 0
				15681 0
				15682 0
				15683 0
				15684 0
				15685 0
				15686 0
				15687 0
				15688 0
				15689 0
				15690 1008133
				15691 0
				15692 0
				15693 0
				15694 0
				15695 0
				15696 0
				15697 0
				15698 0
				15699 0
				15700 0
				15701 0
				15702 0
				15703 0
				15704 0
				15705 1012496
				15706 0
				15707 0
				15708 0
				15709 0
				15710 0
				15711 0
				15712 0
				15713 0
				15714 0
				15715 0
				15716 0
				15717 0
				15718 0
				15719 0
				15720 997567
				15721 0
				15722 0
				15723 0
				15724 0
				15725 0
				15726 0
				15727 0
				15728 0
				15729 0
				15730 0
				15731 0
				15732 0
				15733 0
				15734 0
				15735 1009694
				15736 0
				15737 0
				15738 0
				15739 0
				15740 0
				15741 0
				15742 0
				15743 0
				15744 0
				15745 0
				15746 0
				15747 0
				15748 0
				15749 0
				15750 1025424
				15751 0
				15752 0
				15753 0
				15754 0
				15755 0
				15756 0
				15757 0
				15758 0
				15759 0
				15760 0
				15761 0
				15762 0
				15763 0
				15764 0
				15765 1008166
				15766 0
				15767 0
				15768 0
				15769 0
				15770 0
				15771 0
				15772 0
				15773 0
				15774 0
				15775 0
				15776 0
				15777 0
				15778 0
				15779 0
				15780 1024058
				15781 0
				15782 0
				15783 0
				15784 0
				15785 0
				15786 0
				15787 0
				15788 0
				15789 0
				15790 0
				15791 0
				15792 0
				15793 0
				15794 0
				15795 995044
				15796 0
				15797 0
				15798 0
				15799 0
				15800 0
				15801 0
				15802 0
				15803 0
				15804 0
				15805 0
				15806 0
				15807 0
				15808 0
				15809 0
				15810 1005164
				15811 0
				15812 0
				15813 0
				15814 0
				15815 0
				15816 0
				15817 0
				15818 0
				15819 0
				15820 0
				15821 0
				15822 0
				15823 0
				15824 0
				15825 1041454
				15826 0
				15827 0
				15828 0
				15829 0
				15830 0
				15831 0
				15832 0
				15833 0
				15834 0
				15835 0
				15836 0
				15837 0
				15838 0
				15839 0
				15840 1003533
				15841 0
				15842 0
				15843 0
				15844 0
				15845 0
				15846 0
				15847 0
				15848 0
				15849 0
				15850 0
				15851 0
				15852 0
				15853 0
				15854 0
				15855 994135
				15856 0
				15857 0
				15858 0
				15859 0
				15860 0
				15861 0
				15862 0
				15863 0
				15864 0
				15865 0
				15866 0
				15867 0
				15868 0
				15869 0
				15870 1002524
				15871 0
				15872 0
				15873 0
				15874 0
				15875 0
				15876 0
				15877 0
				15878 0
				15879 0
				15880 0
				15881 0
				15882 0
				15883 0
				15884 0
				15885 1021082
				15886 0
				15887 0
				15888 0
				15889 0
				15890 0
				15891 0
				15892 0
				15893 0
				15894 0
				15895 0
				15896 0
				15897 0
				15898 0
				15899 0
				15900 990794
				15901 0
				15902 0
				15903 0
				15904 0
				15905 0
				15906 0
				15907 0
				15908 0
				15909 0
				15910 0
				15911 0
				15912 0
				15913 0
				15914 0
				15915 1040337
				15916 0
				15917 0
				15918 0
				15919 0
				15920 0
				15921 0
				15922 0
				15923 0
				15924 0
				15925 0
				15926 0
				15927 0
				15928 0
				15929 0
				15930 1019577
				15931 0
				15932 0
				15933 0
				15934 0
				15935 0
				15936 0
				15937 0
				15938 0
				15939 0
				15940 0
				15941 0
				15942 0
				15943 0
				15944 0
				15945 1018438
				15946 0
				15947 0
				15948 0
				15949 0
				15950 0
				15951 0
				15952 0
				15953 0
				15954 0
				15955 0
				15956 0
				15957 0
				15958 0
				15959 0
				15960 1035342
				15961 0
				15962 0
				15963 0
				15964 0
				15965 0
				15966 0
				15967 0
				15968 0
				15969 0
				15970 0
				15971 0
				15972 0
				15973 0
				15974 0
				15975 1009282
				15976 0
				15977 0
				15978 0
				15979 0
				15980 0
				15981 0
				15982 0
				15983 0
				15984 0
				15985 0
				15986 0
				15987 0
				15988 0
				15989 0
				15990 1005665
				15991 0
				15992 0
				15993 0
				15994 0
				15995 0
				15996 0
				15997 0
				15998 0
				15999 0
				16000 0
				16001 0
				16002 0
				16003 0
				16004 0
				16005 1014225
				16006 0
				16007 0
				16008 0
				16009 0
				16010 0
				16011 0
				16012 0
				16013 0
				16014 0
				16015 0
				16016 0
				16017 0
				16018 0
				16019 0
				16020 1023372
				16021 0
				16022 0
				16023 0
				16024 0
				16025 0
				16026 0
				16027 0
				16028 0
				16029 0
				16030 0
				16031 0
				16032 0
				16033 0
				16034 0
				16035 992995
				16036 0
				16037 0
				16038 0
				16039 0
				16040 0
				16041 0
				16042 0
				16043 0
				16044 0
				16045 0
				16046 0
				16047 0
				16048 0
				16049 0
				16050 999677
				16051 0
				16052 0
				16053 0
				16054 0
				16055 0
				16056 0
				16057 0
				16058 0
				16059 0
				16060 0
				16061 0
				16062 0
				16063 0
				16064 0
				16065 1037517
				16066 0
				16067 0
				16068 0
				16069 0
				16070 0
				16071 0
				16072 0
				16073 0
				16074 0
				16075 0
				16076 0
				16077 0
				16078 0
				16079 0
				16080 1015128
				16081 0
				16082 0
				16083 0
				16084 0
				16085 0
				16086 0
				16087 0
				16088 0
				16089 0
				16090 0
				16091 0
				16092 0
				16093 0
				16094 0
				16095 1020258
				16096 0
				16097 0
				16098 0
				16099 0
				16100 0
				16101 0
				16102 0
				16103 0
				16104 0
				16105 0
				16106 0
				16107 0
				16108 0
				16109 0
				16110 1004724
				16111 0
				16112 0
				16113 0
				16114 0
				16115 0
				16116 0
				16117 0
				16118 0
				16119 0
				16120 0
				16121 0
				16122 0
				16123 0
				16124 0
				16125 1015361
				16126 0
				16127 0
				16128 0
				16129 0
				16130 0
				16131 0
				16132 0
				16133 0
				16134 0
				16135 0
				16136 0
				16137 0
				16138 0
				16139 0
				16140 1011624
				16141 0
				16142 0
				16143 0
				16144 0
				16145 0
				16146 0
				16147 0
				16148 0
				16149 0
				16150 0
				16151 0
				16152 0
				16153 0
				16154 0
				16155 1001696
				16156 0
				16157 0
				16158 0
				16159 0
				16160 0
				16161 0
				16162 0
				16163 0
				16164 0
				16165 0
				16166 0
				16167 0
				16168 0
				16169 0
				16170 1010307
				16171 0
				16172 0
				16173 0
				16174 0
				16175 0
				16176 0
				16177 0
				16178 0
				16179 0
				16180 0
				16181 0
				16182 0
				16183 0
				16184 0
				16185 1004303
				16186 0
				16187 0
				16188 0
				16189 0
				16190 0
				16191 0
				16192 0
				16193 0
				16194 0
				16195 0
				16196 0
				16197 0
				16198 0
				16199 0
				16200 1005008
				16201 0
				16202 0
				16203 0
				16204 0
				16205 0
				16206 0
				16207 0
				16208 0
				16209 0
				16210 0
				16211 0
				16212 0
				16213 0
				16214 0
				16215 1030873
				16216 0
				16217 0
				16218 0
				16219 0
				16220 0
				16221 0
				16222 0
				16223 0
				16224 0
				16225 0
				16226 0
				16227 0
				16228 0
				16229 0
				16230 1067433
				16231 0
				16232 0
				16233 0
				16234 0
				16235 0
				16236 0
				16237 0
				16238 0
				16239 0
				16240 0
				16241 0
				16242 0
				16243 0
				16244 0
				16245 1015062
				16246 0
				16247 0
				16248 0
				16249 0
				16250 0
				16251 0
				16252 0
				16253 0
				16254 0
				16255 0
				16256 0
				16257 0
				16258 0
				16259 0
				16260 1024166
				16261 0
				16262 0
				16263 0
				16264 0
				16265 0
				16266 0
				16267 0
				16268 0
				16269 0
				16270 0
				16271 0
				16272 0
				16273 0
				16274 0
				16275 1022486
				16276 0
				16277 0
				16278 0
				16279 0
				16280 0
				16281 0
				16282 0
				16283 0
				16284 0
				16285 0
				16286 0
				16287 0
				16288 0
				16289 0
				16290 1062238
				16291 0
				16292 0
				16293 0
				16294 0
				16295 0
				16296 0
				16297 0
				16298 0
				16299 0
				16300 0
				16301 0
				16302 0
				16303 0
				16304 0
				16305 1023513
				16306 0
				16307 0
				16308 0
				16309 0
				16310 0
				16311 0
				16312 0
				16313 0
				16314 0
				16315 0
				16316 0
				16317 0
				16318 0
				16319 0
				16320 1001408
				16321 0
				16322 0
				16323 0
				16324 0
				16325 0
				16326 0
				16327 0
				16328 0
				16329 0
				16330 0
				16331 0
				16332 0
				16333 0
				16334 0
				16335 1043505
				16336 0
				16337 0
				16338 0
				16339 0
				16340 0
				16341 0
				16342 0
				16343 0
				16344 0
				16345 0
				16346 0
				16347 0
				16348 0
				16349 0
				16350 1017300
				16351 0
				16352 0
				16353 0
				16354 0
				16355 0
				16356 0
				16357 0
				16358 0
				16359 0
				16360 0
				16361 0
				16362 0
				16363 0
				16364 0
				16365 1019024
				16366 0
				16367 0
				16368 0
				16369 0
				16370 0
				16371 0
				16372 0
				16373 0
				16374 0
				16375 0
				16376 0
				16377 0
				16378 0
				16379 0
				16380 1017600
				16381 0
				16382 0
				16383 0
				16384 0
				16385 0
				16386 0
				16387 0
				16388 0
				16389 0
				16390 0
				16391 0
				16392 0
				16393 0
				16394 0
				16395 1007534
				16396 0
				16397 0
				16398 0
				16399 0
				16400 0
				16401 0
				16402 0
				16403 0
				16404 0
				16405 0
				16406 0
				16407 0
				16408 0
				16409 0
				16410 1006168
				16411 0
				16412 0
				16413 0
				16414 0
				16415 0
				16416 0
				16417 0
				16418 0
				16419 0
				16420 0
				16421 0
				16422 0
				16423 0
				16424 0
				16425 1000194
				16426 0
				16427 0
				16428 0
				16429 0
				16430 0
				16431 0
				16432 0
				16433 0
				16434 0
				16435 0
				16436 0
				16437 0
				16438 0
				16439 0
				16440 1054009
				16441 0
				16442 0
				16443 0
				16444 0
				16445 0
				16446 0
				16447 0
				16448 0
				16449 0
				16450 0
				16451 0
				16452 0
				16453 0
				16454 0
				16455 1003062
				16456 0
				16457 0
				16458 0
				16459 0
				16460 0
				16461 0
				16462 0
				16463 0
				16464 0
				16465 0
				16466 0
				16467 0
				16468 0
				16469 0
				16470 999540
				16471 0
				16472 0
				16473 0
				16474 0
				16475 0
				16476 0
				16477 0
				16478 0
				16479 0
				16480 0
				16481 0
				16482 0
				16483 0
				16484 0
				16485 1003222
				16486 0
				16487 0
				16488 0
				16489 0
				16490 0
				16491 0
				16492 0
				16493 0
				16494 0
				16495 0
				16496 0
				16497 0
				16498 0
				16499 0
				16500 1028379
				16501 0
				16502 0
				16503 0
				16504 0
				16505 0
				16506 0
				16507 0
				16508 0
				16509 0
				16510 0
				16511 0
				16512 0
				16513 0
				16514 0
				16515 1020075
				16516 0
				16517 0
				16518 0
				16519 0
				16520 0
				16521 0
				16522 0
				16523 0
				16524 0
				16525 0
				16526 0
				16527 0
				16528 0
				16529 0
				16530 1006823
				16531 0
				16532 0
				16533 0
				16534 0
				16535 0
				16536 0
				16537 0
				16538 0
				16539 0
				16540 0
				16541 0
				16542 0
				16543 0
				16544 0
				16545 1018210
				16546 0
				16547 0
				16548 0
				16549 0
				16550 0
				16551 0
				16552 0
				16553 0
				16554 0
				16555 0
				16556 0
				16557 0
				16558 0
				16559 0
				16560 986022
				16561 0
				16562 0
				16563 0
				16564 0
				16565 0
				16566 0
				16567 0
				16568 0
				16569 0
				16570 0
				16571 0
				16572 0
				16573 0
				16574 0
				16575 1014526
				16576 0
				16577 0
				16578 0
				16579 0
				16580 0
				16581 0
				16582 0
				16583 0
				16584 0
				16585 0
				16586 0
				16587 0
				16588 0
				16589 0
				16590 1015726
				16591 0
				16592 0
				16593 0
				16594 0
				16595 0
				16596 0
				16597 0
				16598 0
				16599 0
				16600 0
				16601 0
				16602 0
				16603 0
				16604 0
				16605 1006073
				16606 0
				16607 0
				16608 0
				16609 0
				16610 0
				16611 0
				16612 0
				16613 0
				16614 0
				16615 0
				16616 0
				16617 0
				16618 0
				16619 0
				16620 1012351
				16621 0
				16622 0
				16623 0
				16624 0
				16625 0
				16626 0
				16627 0
				16628 0
				16629 0
				16630 0
				16631 0
				16632 0
				16633 0
				16634 0
				16635 995698
				16636 0
				16637 0
				16638 0
				16639 0
				16640 0
				16641 0
				16642 0
				16643 0
				16644 0
				16645 0
				16646 0
				16647 0
				16648 0
				16649 0
				16650 990316
				16651 0
				16652 0
				16653 0
				16654 0
				16655 0
				16656 0
				16657 0
				16658 0
				16659 0
				16660 0
				16661 0
				16662 0
				16663 0
				16664 0
				16665 1018315
				16666 0
				16667 0
				16668 0
				16669 0
				16670 0
				16671 0
				16672 0
				16673 0
				16674 0
				16675 0
				16676 0
				16677 0
				16678 0
				16679 0
				16680 1000856
				16681 0
				16682 0
				16683 0
				16684 0
				16685 0
				16686 0
				16687 0
				16688 0
				16689 0
				16690 0
				16691 0
				16692 0
				16693 0
				16694 0
				16695 1012707
				16696 0
				16697 0
				16698 0
				16699 0
				16700 0
				16701 0
				16702 0
				16703 0
				16704 0
				16705 0
				16706 0
				16707 0
				16708 0
				16709 0
				16710 1013608
				16711 0
				16712 0
				16713 0
				16714 0
				16715 0
				16716 0
				16717 0
				16718 0
				16719 0
				16720 0
				16721 0
				16722 0
				16723 0
				16724 0
				16725 1030891
				16726 0
				16727 0
				16728 0
				16729 0
				16730 0
				16731 0
				16732 0
				16733 0
				16734 0
				16735 0
				16736 0
				16737 0
				16738 0
				16739 0
				16740 1027549
				16741 0
				16742 0
				16743 0
				16744 0
				16745 0
				16746 0
				16747 0
				16748 0
				16749 0
				16750 0
				16751 0
				16752 0
				16753 0
				16754 0
				16755 1059920
				16756 0
				16757 0
				16758 0
				16759 0
				16760 0
				16761 0
				16762 0
				16763 0
				16764 0
				16765 0
				16766 0
				16767 0
				16768 0
				16769 0
				16770 1011945
				16771 0
				16772 0
				16773 0
				16774 0
				16775 0
				16776 0
				16777 0
				16778 0
				16779 0
				16780 0
				16781 0
				16782 0
				16783 0
				16784 0
				16785 1010805
				16786 0
				16787 0
				16788 0
				16789 0
				16790 0
				16791 0
				16792 0
				16793 0
				16794 0
				16795 0
				16796 0
				16797 0
				16798 0
				16799 0
				16800 1021182
				16801 0
				16802 0
				16803 0
				16804 0
				16805 0
				16806 0
				16807 0
				16808 0
				16809 0
				16810 0
				16811 0
				16812 0
				16813 0
				16814 0
				16815 1003302
				16816 0
				16817 0
				16818 0
				16819 0
				16820 0
				16821 0
				16822 0
				16823 0
				16824 0
				16825 0
				16826 0
				16827 0
				16828 0
				16829 0
				16830 1026817
				16831 0
				16832 0
				16833 0
				16834 0
				16835 0
				16836 0
				16837 0
				16838 0
				16839 0
				16840 0
				16841 0
				16842 0
				16843 0
				16844 0
				16845 1006945
				16846 0
				16847 0
				16848 0
				16849 0
				16850 0
				16851 0
				16852 0
				16853 0
				16854 0
				16855 0
				16856 0
				16857 0
				16858 0
				16859 0
				16860 1052925
				16861 0
				16862 0
				16863 0
				16864 0
				16865 0
				16866 0
				16867 0
				16868 0
				16869 0
				16870 0
				16871 0
				16872 0
				16873 0
				16874 0
				16875 1030924
				16876 0
				16877 0
				16878 0
				16879 0
				16880 0
				16881 0
				16882 0
				16883 0
				16884 0
				16885 0
				16886 0
				16887 0
				16888 0
				16889 0
				16890 1058169
				16891 0
				16892 0
				16893 0
				16894 0
				16895 0
				16896 0
				16897 0
				16898 0
				16899 0
				16900 0
				16901 0
				16902 0
				16903 0
				16904 0
				16905 1041435
				16906 0
				16907 0
				16908 0
				16909 0
				16910 0
				16911 0
				16912 0
				16913 0
				16914 0
				16915 0
				16916 0
				16917 0
				16918 0
				16919 0
				16920 1011572
				16921 0
				16922 0
				16923 0
				16924 0
				16925 0
				16926 0
				16927 0
				16928 0
				16929 0
				16930 0
				16931 0
				16932 0
				16933 0
				16934 0
				16935 1016033
				16936 0
				16937 0
				16938 0
				16939 0
				16940 0
				16941 0
				16942 0
				16943 0
				16944 0
				16945 0
				16946 0
				16947 0
				16948 0
				16949 0
				16950 999938
				16951 0
				16952 0
				16953 0
				16954 0
				16955 0
				16956 0
				16957 0
				16958 0
				16959 0
				16960 0
				16961 0
				16962 0
				16963 0
				16964 0
				16965 1010365
				16966 0
				16967 0
				16968 0
				16969 0
				16970 0
				16971 0
				16972 0
				16973 0
				16974 0
				16975 0
				16976 0
				16977 0
				16978 0
				16979 0
				16980 996645
				16981 0
				16982 0
				16983 0
				16984 0
				16985 0
				16986 0
				16987 0
				16988 0
				16989 0
				16990 0
				16991 0
				16992 0
				16993 0
				16994 0
				16995 1006546
				16996 0
				16997 0
				16998 0
				16999 0
				17000 0
				17001 0
				17002 0
				17003 0
				17004 0
				17005 0
				17006 0
				17007 0
				17008 0
				17009 0
				17010 1040413
				17011 0
				17012 0
				17013 0
				17014 0
				17015 0
				17016 0
				17017 0
				17018 0
				17019 0
				17020 0
				17021 0
				17022 0
				17023 0
				17024 0
				17025 1016013
				17026 0
				17027 0
				17028 0
				17029 0
				17030 0
				17031 0
				17032 0
				17033 0
				17034 0
				17035 0
				17036 0
				17037 0
				17038 0
				17039 0
				17040 1052818
				17041 0
				17042 0
				17043 0
				17044 0
				17045 0
				17046 0
				17047 0
				17048 0
				17049 0
				17050 0
				17051 0
				17052 0
				17053 0
				17054 0
				17055 1012834
				17056 0
				17057 0
				17058 0
				17059 0
				17060 0
				17061 0
				17062 0
				17063 0
				17064 0
				17065 0
				17066 0
				17067 0
				17068 0
				17069 0
				17070 1015494
				17071 0
				17072 0
				17073 0
				17074 0
				17075 0
				17076 0
				17077 0
				17078 0
				17079 0
				17080 0
				17081 0
				17082 0
				17083 0
				17084 0
				17085 1008924
				17086 0
				17087 0
				17088 0
				17089 0
				17090 0
				17091 0
				17092 0
				17093 0
				17094 0
				17095 0
				17096 0
				17097 0
				17098 0
				17099 0
				17100 1004634
				17101 0
				17102 0
				17103 0
				17104 0
				17105 0
				17106 0
				17107 0
				17108 0
				17109 0
				17110 0
				17111 0
				17112 0
				17113 0
				17114 0
				17115 1023662
				17116 0
				17117 0
				17118 0
				17119 0
				17120 0
				17121 0
				17122 0
				17123 0
				17124 0
				17125 0
				17126 0
				17127 0
				17128 0
				17129 0
				17130 1000957
				17131 0
				17132 0
				17133 0
				17134 0
				17135 0
				17136 0
				17137 0
				17138 0
				17139 0
				17140 0
				17141 0
				17142 0
				17143 0
				17144 0
				17145 1024245
				17146 0
				17147 0
				17148 0
				17149 0
				17150 0
				17151 0
				17152 0
				17153 0
				17154 0
				17155 0
				17156 0
				17157 0
				17158 0
				17159 0
				17160 995671
				17161 0
				17162 0
				17163 0
				17164 0
				17165 0
				17166 0
				17167 0
				17168 0
				17169 0
				17170 0
				17171 0
				17172 0
				17173 0
				17174 0
				17175 1008206
				17176 0
				17177 0
				17178 0
				17179 0
				17180 0
				17181 0
				17182 0
				17183 0
				17184 0
				17185 0
				17186 0
				17187 0
				17188 0
				17189 0
				17190 990769
				17191 0
				17192 0
				17193 0
				17194 0
				17195 0
				17196 0
				17197 0
				17198 0
				17199 0
				17200 0
				17201 0
				17202 0
				17203 0
				17204 0
				17205 1016703
				17206 0
				17207 0
				17208 0
				17209 0
				17210 0
				17211 0
				17212 0
				17213 0
				17214 0
				17215 0
				17216 0
				17217 0
				17218 0
				17219 0
				17220 1005283
				17221 0
				17222 0
				17223 0
				17224 0
				17225 0
				17226 0
				17227 0
				17228 0
				17229 0
				17230 0
				17231 0
				17232 0
				17233 0
				17234 0
				17235 1013632
				17236 0
				17237 0
				17238 0
				17239 0
				17240 0
				17241 0
				17242 0
				17243 0
				17244 0
				17245 0
				17246 0
				17247 0
				17248 0
				17249 0
				17250 1038821
				17251 0
				17252 0
				17253 0
				17254 0
				17255 0
				17256 0
				17257 0
				17258 0
				17259 0
				17260 0
				17261 0
				17262 0
				17263 0
				17264 0
				17265 1003741
				17266 0
				17267 0
				17268 0
				17269 0
				17270 0
				17271 0
				17272 0
				17273 0
				17274 0
				17275 0
				17276 0
				17277 0
				17278 0
				17279 0
				17280 998619
				17281 0
				17282 0
				17283 0
				17284 0
				17285 0
				17286 0
				17287 0
				17288 0
				17289 0
				17290 0
				17291 0
				17292 0
				17293 0
				17294 0
				17295 1017576
				17296 0
				17297 0
				17298 0
				17299 0
				17300 0
				17301 0
				17302 0
				17303 0
				17304 0
				17305 0
				17306 0
				17307 0
				17308 0
				17309 0
				17310 1006264
				17311 0
				17312 0
				17313 0
				17314 0
				17315 0
				17316 0
				17317 0
				17318 0
				17319 0
				17320 0
				17321 0
				17322 0
				17323 0
				17324 0
				17325 1040987
				17326 0
				17327 0
				17328 0
				17329 0
				17330 0
				17331 0
				17332 0
				17333 0
				17334 0
				17335 0
				17336 0
				17337 0
				17338 0
				17339 0
				17340 1049050
				17341 0
				17342 0
				17343 0
				17344 0
				17345 0
				17346 0
				17347 0
				17348 0
				17349 0
				17350 0
				17351 0
				17352 0
				17353 0
				17354 0
				17355 1059280
				17356 0
				17357 0
				17358 0
				17359 0
				17360 0
				17361 0
				17362 0
				17363 0
				17364 0
				17365 0
				17366 0
				17367 0
				17368 0
				17369 0
				17370 1035685
				17371 0
				17372 0
				17373 0
				17374 0
				17375 0
				17376 0
				17377 0
				17378 0
				17379 0
				17380 0
				17381 0
				17382 0
				17383 0
				17384 0
				17385 1031740
				17386 0
				17387 0
				17388 0
				17389 0
				17390 0
				17391 0
				17392 0
				17393 0
				17394 0
				17395 0
				17396 0
				17397 0
				17398 0
				17399 0
				17400 1033610
				17401 0
				17402 0
				17403 0
				17404 0
				17405 0
				17406 0
				17407 0
				17408 0
				17409 0
				17410 0
				17411 0
				17412 0
				17413 0
				17414 0
				17415 1012001
				17416 0
				17417 0
				17418 0
				17419 0
				17420 0
				17421 0
				17422 0
				17423 0
				17424 0
				17425 0
				17426 0
				17427 0
				17428 0
				17429 0
				17430 999582
				17431 0
				17432 0
				17433 0
				17434 0
				17435 0
				17436 0
				17437 0
				17438 0
				17439 0
				17440 0
				17441 0
				17442 0
				17443 0
				17444 0
				17445 1005385
				17446 0
				17447 0
				17448 0
				17449 0
				17450 0
				17451 0
				17452 0
				17453 0
				17454 0
				17455 0
				17456 0
				17457 0
				17458 0
				17459 0
				17460 1007501
				17461 0
				17462 0
				17463 0
				17464 0
				17465 0
				17466 0
				17467 0
				17468 0
				17469 0
				17470 0
				17471 0
				17472 0
				17473 0
				17474 0
				17475 1007482
				17476 0
				17477 0
				17478 0
				17479 0
				17480 0
				17481 0
				17482 0
				17483 0
				17484 0
				17485 0
				17486 0
				17487 0
				17488 0
				17489 0
				17490 995060
				17491 0
				17492 0
				17493 0
				17494 0
				17495 0
				17496 0
				17497 0
				17498 0
				17499 0
				17500 0
				17501 0
				17502 0
				17503 0
				17504 0
				17505 1001922
				17506 0
				17507 0
				17508 0
				17509 0
				17510 0
				17511 0
				17512 0
				17513 0
				17514 0
				17515 0
				17516 0
				17517 0
				17518 0
				17519 0
				17520 1018814
				17521 0
				17522 0
				17523 0
				17524 0
				17525 0
				17526 0
				17527 0
				17528 0
				17529 0
				17530 0
				17531 0
				17532 0
				17533 0
				17534 0
				17535 1000357
				17536 0
				17537 0
				17538 0
				17539 0
				17540 0
				17541 0
				17542 0
				17543 0
				17544 0
				17545 0
				17546 0
				17547 0
				17548 0
				17549 0
				17550 1049180
				17551 0
				17552 0
				17553 0
				17554 0
				17555 0
				17556 0
				17557 0
				17558 0
				17559 0
				17560 0
				17561 0
				17562 0
				17563 0
				17564 0
				17565 1012586
				17566 0
				17567 0
				17568 0
				17569 0
				17570 0
				17571 0
				17572 0
				17573 0
				17574 0
				17575 0
				17576 0
				17577 0
				17578 0
				17579 0
				17580 1011453
				17581 0
				17582 0
				17583 0
				17584 0
				17585 0
				17586 0
				17587 0
				17588 0
				17589 0
				17590 0
				17591 0
				17592 0
				17593 0
				17594 0
				17595 1029760
				17596 0
				17597 0
				17598 0
				17599 0
				17600 0
				17601 0
				17602 0
				17603 0
				17604 0
				17605 0
				17606 0
				17607 0
				17608 0
				17609 0
				17610 1055162
				17611 0
				17612 0
				17613 0
				17614 0
				17615 0
				17616 0
				17617 0
				17618 0
				17619 0
				17620 0
				17621 0
				17622 0
				17623 0
				17624 0
				17625 1007836
				17626 0
				17627 0
				17628 0
				17629 0
				17630 0
				17631 0
				17632 0
				17633 0
				17634 0
				17635 0
				17636 0
				17637 0
				17638 0
				17639 0
				17640 1008974
				17641 0
				17642 0
				17643 0
				17644 0
				17645 0
				17646 0
				17647 0
				17648 0
				17649 0
				17650 0
				17651 0
				17652 0
				17653 0
				17654 0
				17655 1048150
				17656 0
				17657 0
				17658 0
				17659 0
				17660 0
				17661 0
				17662 0
				17663 0
				17664 0
				17665 0
				17666 0
				17667 0
				17668 0
				17669 0
				17670 1008858
				17671 0
				17672 0
				17673 0
				17674 0
				17675 0
				17676 0
				17677 0
				17678 0
				17679 0
				17680 0
				17681 0
				17682 0
				17683 0
				17684 0
				17685 1020028
				17686 0
				17687 0
				17688 0
				17689 0
				17690 0
				17691 0
				17692 0
				17693 0
				17694 0
				17695 0
				17696 0
				17697 0
				17698 0
				17699 0
				17700 1023536
				17701 0
				17702 0
				17703 0
				17704 0
				17705 0
				17706 0
				17707 0
				17708 0
				17709 0
				17710 0
				17711 0
				17712 0
				17713 0
				17714 0
				17715 1020088
				17716 0
				17717 0
				17718 0
				17719 0
				17720 0
				17721 0
				17722 0
				17723 0
				17724 0
				17725 0
				17726 0
				17727 0
				17728 0
				17729 0
				17730 1038012
				17731 0
				17732 0
				17733 0
				17734 0
				17735 0
				17736 0
				17737 0
				17738 0
				17739 0
				17740 0
				17741 0
				17742 0
				17743 0
				17744 0
				17745 1065713
				17746 0
				17747 0
				17748 0
				17749 0
				17750 0
				17751 0
				17752 0
				17753 0
				17754 0
				17755 0
				17756 0
				17757 0
				17758 0
				17759 0
				17760 1014870
				17761 0
				17762 0
				17763 0
				17764 0
				17765 0
				17766 0
				17767 0
				17768 0
				17769 0
				17770 0
				17771 0
				17772 0
				17773 0
				17774 0
				17775 1011709
				17776 0
				17777 0
				17778 0
				17779 0
				17780 0
				17781 0
				17782 0
				17783 0
				17784 0
				17785 0
				17786 0
				17787 0
				17788 0
				17789 0
				17790 1022793
				17791 0
				17792 0
				17793 0
				17794 0
				17795 0
				17796 0
				17797 0
				17798 0
				17799 0
				17800 0
				17801 0
				17802 0
				17803 0
				17804 0
				17805 1024693
				17806 0
				17807 0
				17808 0
				17809 0
				17810 0
				17811 0
				17812 0
				17813 0
				17814 0
				17815 0
				17816 0
				17817 0
				17818 0
				17819 0
				17820 1018353
				17821 0
				17822 0
				17823 0
				17824 0
				17825 0
				17826 0
				17827 0
				17828 0
				17829 0
				17830 0
				17831 0
				17832 0
				17833 0
				17834 0
				17835 1039502
				17836 0
				17837 0
				17838 0
				17839 0
				17840 0
				17841 0
				17842 0
				17843 0
				17844 0
				17845 0
				17846 0
				17847 0
				17848 0
				17849 0
				17850 1005774
				17851 0
				17852 0
				17853 0
				17854 0
				17855 0
				17856 0
				17857 0
				17858 0
				17859 0
				17860 0
				17861 0
				17862 0
				17863 0
				17864 0
				17865 1037347
				17866 0
				17867 0
				17868 0
				17869 0
				17870 0
				17871 0
				17872 0
				17873 0
				17874 0
				17875 0
				17876 0
				17877 0
				17878 0
				17879 0
				17880 1019109
				17881 0
				17882 0
				17883 0
				17884 0
				17885 0
				17886 0
				17887 0
				17888 0
				17889 0
				17890 0
				17891 0
				17892 0
				17893 0
				17894 0
				17895 1026017
				17896 0
				17897 0
				17898 0
				17899 0
				17900 0
				17901 0
				17902 0
				17903 0
				17904 0
				17905 0
				17906 0
				17907 0
				17908 0
				17909 0
				17910 1001518
				17911 0
				17912 0
				17913 0
				17914 0
				17915 0
				17916 0
				17917 0
				17918 0
				17919 0
				17920 0
				17921 0
				17922 0
				17923 0
				17924 0
				17925 1073910
				17926 0
				17927 0
				17928 0
				17929 0
				17930 0
				17931 0
				17932 0
				17933 0
				17934 0
				17935 0
				17936 0
				17937 0
				17938 0
				17939 0
				17940 1022013
				17941 0
				17942 0
				17943 0
				17944 0
				17945 0
				17946 0
				17947 0
				17948 0
				17949 0
				17950 0
				17951 0
				17952 0
				17953 0
				17954 0
				17955 1072232
				17956 0
				17957 0
				17958 0
				17959 0
				17960 0
				17961 0
				17962 0
				17963 0
				17964 0
				17965 0
				17966 0
				17967 0
				17968 0
				17969 0
				17970 1018086
				17971 0
				17972 0
				17973 0
				17974 0
				17975 0
				17976 0
				17977 0
				17978 0
				17979 0
				17980 0
				17981 0
				17982 0
				17983 0
				17984 0
				17985 1061937
				17986 0
				17987 0
				17988 0
				17989 0
				17990 0
				17991 0
				17992 0
				17993 0
				17994 0
				17995 0
				17996 0
				17997 0
				17998 0
				17999 0
				18000 1022817
				18001 0
				18002 0
				18003 0
				18004 0
				18005 0
				18006 0
				18007 0
				18008 0
				18009 0
				18010 0
				18011 0
				18012 0
				18013 0
				18014 0
				18015 1040446
				18016 0
				18017 0
				18018 0
				18019 0
				18020 0
				18021 0
				18022 0
				18023 0
				18024 0
				18025 0
				18026 0
				18027 0
				18028 0
				18029 0
				18030 1033232
				18031 0
				18032 0
				18033 0
				18034 0
				18035 0
				18036 0
				18037 0
				18038 0
				18039 0
				18040 0
				18041 0
				18042 0
				18043 0
				18044 0
				18045 1056891
				18046 0
				18047 0
				18048 0
				18049 0
				18050 0
				18051 0
				18052 0
				18053 0
				18054 0
				18055 0
				18056 0
				18057 0
				18058 0
				18059 0
				18060 1009039
				18061 0
				18062 0
				18063 0
				18064 0
				18065 0
				18066 0
				18067 0
				18068 0
				18069 0
				18070 0
				18071 0
				18072 0
				18073 0
				18074 0
				18075 1040063
				18076 0
				18077 0
				18078 0
				18079 0
				18080 0
				18081 0
				18082 0
				18083 0
				18084 0
				18085 0
				18086 0
				18087 0
				18088 0
				18089 0
				18090 1031255
				18091 0
				18092 0
				18093 0
				18094 0
				18095 0
				18096 0
				18097 0
				18098 0
				18099 0
				18100 0
				18101 0
				18102 0
				18103 0
				18104 0
				18105 1006559
				18106 0
				18107 0
				18108 0
				18109 0
				18110 0
				18111 0
				18112 0
				18113 0
				18114 0
				18115 0
				18116 0
				18117 0
				18118 0
				18119 0
				18120 1033486
				18121 0
				18122 0
				18123 0
				18124 0
				18125 0
				18126 0
				18127 0
				18128 0
				18129 0
				18130 0
				18131 0
				18132 0
				18133 0
				18134 0
				18135 1013012
				18136 0
				18137 0
				18138 0
				18139 0
				18140 0
				18141 0
				18142 0
				18143 0
				18144 0
				18145 0
				18146 0
				18147 0
				18148 0
				18149 0
				18150 1008207
				18151 0
				18152 0
				18153 0
				18154 0
				18155 0
				18156 0
				18157 0
				18158 0
				18159 0
				18160 0
				18161 0
				18162 0
				18163 0
				18164 0
				18165 1014408
				18166 0
				18167 0
				18168 0
				18169 0
				18170 0
				18171 0
				18172 0
				18173 0
				18174 0
				18175 0
				18176 0
				18177 0
				18178 0
				18179 0
				18180 1005559
				18181 0
				18182 0
				18183 0
				18184 0
				18185 0
				18186 0
				18187 0
				18188 0
				18189 0
				18190 0
				18191 0
				18192 0
				18193 0
				18194 0
				18195 1049002
				18196 0
				18197 0
				18198 0
				18199 0
				18200 0
				18201 0
				18202 0
				18203 0
				18204 0
				18205 0
				18206 0
				18207 0
				18208 0
				18209 0
				18210 1046604
				18211 0
				18212 0
				18213 0
				18214 0
				18215 0
				18216 0
				18217 0
				18218 0
				18219 0
				18220 0
				18221 0
				18222 0
				18223 0
				18224 0
				18225 1008166
				18226 0
				18227 0
				18228 0
				18229 0
				18230 0
				18231 0
				18232 0
				18233 0
				18234 0
				18235 0
				18236 0
				18237 0
				18238 0
				18239 0
				18240 1023338
				18241 0
				18242 0
				18243 0
				18244 0
				18245 0
				18246 0
				18247 0
				18248 0
				18249 0
				18250 0
				18251 0
				18252 0
				18253 0
				18254 0
				18255 1000293
				18256 0
				18257 0
				18258 0
				18259 0
				18260 0
				18261 0
				18262 0
				18263 0
				18264 0
				18265 0
				18266 0
				18267 0
				18268 0
				18269 0
				18270 1046136
				18271 0
				18272 0
				18273 0
				18274 0
				18275 0
				18276 0
				18277 0
				18278 0
				18279 0
				18280 0
				18281 0
				18282 0
				18283 0
				18284 0
				18285 984713
				18286 0
				18287 0
				18288 0
				18289 0
				18290 0
				18291 0
				18292 0
				18293 0
				18294 0
				18295 0
				18296 0
				18297 0
				18298 0
				18299 0
				18300 1025620
				18301 0
				18302 0
				18303 0
				18304 0
				18305 0
				18306 0
				18307 0
				18308 0
				18309 0
				18310 0
				18311 0
				18312 0
				18313 0
				18314 0
				18315 1009174
				18316 0
				18317 0
				18318 0
				18319 0
				18320 0
				18321 0
				18322 0
				18323 0
				18324 0
				18325 0
				18326 0
				18327 0
				18328 0
				18329 0
				18330 990802
				18331 0
				18332 0
				18333 0
				18334 0
				18335 0
				18336 0
				18337 0
				18338 0
				18339 0
				18340 0
				18341 0
				18342 0
				18343 0
				18344 0
				18345 994696
				18346 0
				18347 0
				18348 0
				18349 0
				18350 0
				18351 0
				18352 0
				18353 0
				18354 0
				18355 0
				18356 0
				18357 0
				18358 0
				18359 0
				18360 993008
				18361 0
				18362 0
				18363 0
				18364 0
				18365 0
				18366 0
				18367 0
				18368 0
				18369 0
				18370 0
				18371 0
				18372 0
				18373 0
				18374 0
				18375 1038806
				18376 0
				18377 0
				18378 0
				18379 0
				18380 0
				18381 0
				18382 0
				18383 0
				18384 0
				18385 0
				18386 0
				18387 0
				18388 0
				18389 0
				18390 1057416
				18391 0
				18392 0
				18393 0
				18394 0
				18395 0
				18396 0
				18397 0
				18398 0
				18399 0
				18400 0
				18401 0
				18402 0
				18403 0
				18404 0
				18405 992658
				18406 0
				18407 0
				18408 0
				18409 0
				18410 0
				18411 0
				18412 0
				18413 0
				18414 0
				18415 0
				18416 0
				18417 0
				18418 0
				18419 0
				18420 1048642
				18421 0
				18422 0
				18423 0
				18424 0
				18425 0
				18426 0
				18427 0
				18428 0
				18429 0
				18430 0
				18431 0
				18432 0
				18433 0
				18434 0
				18435 1037293
				18436 0
				18437 0
				18438 0
				18439 0
				18440 0
				18441 0
				18442 0
				18443 0
				18444 0
				18445 0
				18446 0
				18447 0
				18448 0
				18449 0
				18450 988499
				18451 0
				18452 0
				18453 0
				18454 0
				18455 0
				18456 0
				18457 0
				18458 0
				18459 0
				18460 0
				18461 0
				18462 0
				18463 0
				18464 0
				18465 998232
				18466 0
				18467 0
				18468 0
				18469 0
				18470 0
				18471 0
				18472 0
				18473 0
				18474 0
				18475 0
				18476 0
				18477 0
				18478 0
				18479 0
				18480 997316
				18481 0
				18482 0
				18483 0
				18484 0
				18485 0
				18486 0
				18487 0
				18488 0
				18489 0
				18490 0
				18491 0
				18492 0
				18493 0
				18494 0
				18495 1017853
				18496 0
				18497 0
				18498 0
				18499 0
				18500 0
				18501 0
				18502 0
				18503 0
				18504 0
				18505 0
				18506 0
				18507 0
				18508 0
				18509 0
				18510 989714
				18511 0
				18512 0
				18513 0
				18514 0
				18515 0
				18516 0
				18517 0
				18518 0
				18519 0
				18520 0
				18521 0
				18522 0
				18523 0
				18524 0
				18525 993070
				18526 0
				18527 0
				18528 0
				18529 0
				18530 0
				18531 0
				18532 0
				18533 0
				18534 0
				18535 0
				18536 0
				18537 0
				18538 0
				18539 0
				18540 1013690
				18541 0
				18542 0
				18543 0
				18544 0
				18545 0
				18546 0
				18547 0
				18548 0
				18549 0
				18550 0
				18551 0
				18552 0
				18553 0
				18554 0
				18555 1021348
				18556 0
				18557 0
				18558 0
				18559 0
				18560 0
				18561 0
				18562 0
				18563 0
				18564 0
				18565 0
				18566 0
				18567 0
				18568 0
				18569 0
				18570 1013695
				18571 0
				18572 0
				18573 0
				18574 0
				18575 0
				18576 0
				18577 0
				18578 0
				18579 0
				18580 0
				18581 0
				18582 0
				18583 0
				18584 0
				18585 1000204
				18586 0
				18587 0
				18588 0
				18589 0
				18590 0
				18591 0
				18592 0
				18593 0
				18594 0
				18595 0
				18596 0
				18597 0
				18598 0
				18599 0
				18600 987390
				18601 0
				18602 0
				18603 0
				18604 0
				18605 0
				18606 0
				18607 0
				18608 0
				18609 0
				18610 0
				18611 0
				18612 0
				18613 0
				18614 0
				18615 1018628
				18616 0
				18617 0
				18618 0
				18619 0
				18620 0
				18621 0
				18622 0
				18623 0
				18624 0
				18625 0
				18626 0
				18627 0
				18628 0
				18629 0
				18630 1009962
				18631 0
				18632 0
				18633 0
				18634 0
				18635 0
				18636 0
				18637 0
				18638 0
				18639 0
				18640 0
				18641 0
				18642 0
				18643 0
				18644 0
				18645 1007963
				18646 0
				18647 0
				18648 0
				18649 0
				18650 0
				18651 0
				18652 0
				18653 0
				18654 0
				18655 0
				18656 0
				18657 0
				18658 0
				18659 0
				18660 987507
				18661 0
				18662 0
				18663 0
				18664 0
				18665 0
				18666 0
				18667 0
				18668 0
				18669 0
				18670 0
				18671 0
				18672 0
				18673 0
				18674 0
				18675 998472
				18676 0
				18677 0
				18678 0
				18679 0
				18680 0
				18681 0
				18682 0
				18683 0
				18684 0
				18685 0
				18686 0
				18687 0
				18688 0
				18689 0
				18690 996782
				18691 0
				18692 0
				18693 0
				18694 0
				18695 0
				18696 0
				18697 0
				18698 0
				18699 0
				18700 0
				18701 0
				18702 0
				18703 0
				18704 0
				18705 1000011
				18706 0
				18707 0
				18708 0
				18709 0
				18710 0
				18711 0
				18712 0
				18713 0
				18714 0
				18715 0
				18716 0
				18717 0
				18718 0
				18719 0
				18720 987138
				18721 0
				18722 0
				18723 0
				18724 0
				18725 0
				18726 0
				18727 0
				18728 0
				18729 0
				18730 0
				18731 0
				18732 0
				18733 0
				18734 0
				18735 1017310
				18736 0
				18737 0
				18738 0
				18739 0
				18740 0
				18741 0
				18742 0
				18743 0
				18744 0
				18745 0
				18746 0
				18747 0
				18748 0
				18749 0
				18750 1020192
				18751 0
				18752 0
				18753 0
				18754 0
				18755 0
				18756 0
				18757 0
				18758 0
				18759 0
				18760 0
				18761 0
				18762 0
				18763 0
				18764 0
				18765 986556
				18766 0
				18767 0
				18768 0
				18769 0
				18770 0
				18771 0
				18772 0
				18773 0
				18774 0
				18775 0
				18776 0
				18777 0
				18778 0
				18779 0
				18780 1011832
				18781 0
				18782 0
				18783 0
				18784 0
				18785 0
				18786 0
				18787 0
				18788 0
				18789 0
				18790 0
				18791 0
				18792 0
				18793 0
				18794 0
				18795 1011435
				18796 0
				18797 0
				18798 0
				18799 0
				18800 0
				18801 0
				18802 0
				18803 0
				18804 0
				18805 0
				18806 0
				18807 0
				18808 0
				18809 0
				18810 1004938
				18811 0
				18812 0
				18813 0
				18814 0
				18815 0
				18816 0
				18817 0
				18818 0
				18819 0
				18820 0
				18821 0
				18822 0
				18823 0
				18824 0
				18825 1012786
				18826 0
				18827 0
				18828 0
				18829 0
				18830 0
				18831 0
				18832 0
				18833 0
				18834 0
				18835 0
				18836 0
				18837 0
				18838 0
				18839 0
				18840 1015814
				18841 0
				18842 0
				18843 0
				18844 0
				18845 0
				18846 0
				18847 0
				18848 0
				18849 0
				18850 0
				18851 0
				18852 0
				18853 0
				18854 0
				18855 1031313
				18856 0
				18857 0
				18858 0
				18859 0
				18860 0
				18861 0
				18862 0
				18863 0
				18864 0
				18865 0
				18866 0
				18867 0
				18868 0
				18869 0
				18870 1014796
				18871 0
				18872 0
				18873 0
				18874 0
				18875 0
				18876 0
				18877 0
				18878 0
				18879 0
				18880 0
				18881 0
				18882 0
				18883 0
				18884 0
				18885 1019748
				18886 0
				18887 0
				18888 0
				18889 0
				18890 0
				18891 0
				18892 0
				18893 0
				18894 0
				18895 0
				18896 0
				18897 0
				18898 0
				18899 0
				18900 1007108
				18901 0
				18902 0
				18903 0
				18904 0
				18905 0
				18906 0
				18907 0
				18908 0
				18909 0
				18910 0
				18911 0
				18912 0
				18913 0
				18914 0
				18915 1008392
				18916 0
				18917 0
				18918 0
				18919 0
				18920 0
				18921 0
				18922 0
				18923 0
				18924 0
				18925 0
				18926 0
				18927 0
				18928 0
				18929 0
				18930 1052902
				18931 0
				18932 0
				18933 0
				18934 0
				18935 0
				18936 0
				18937 0
				18938 0
				18939 0
				18940 0
				18941 0
				18942 0
				18943 0
				18944 0
				18945 1035550
				18946 0
				18947 0
				18948 0
				18949 0
				18950 0
				18951 0
				18952 0
				18953 0
				18954 0
				18955 0
				18956 0
				18957 0
				18958 0
				18959 0
				18960 1005842
				18961 0
				18962 0
				18963 0
				18964 0
				18965 0
				18966 0
				18967 0
				18968 0
				18969 0
				18970 0
				18971 0
				18972 0
				18973 0
				18974 0
				18975 1007347
				18976 0
				18977 0
				18978 0
				18979 0
				18980 0
				18981 0
				18982 0
				18983 0
				18984 0
				18985 0
				18986 0
				18987 0
				18988 0
				18989 0
				18990 1000471
				18991 0
				18992 0
				18993 0
				18994 0
				18995 0
				18996 0
				18997 0
				18998 0
				18999 0
				19000 0
				19001 0
				19002 0
				19003 0
				19004 0
				19005 1065177
				19006 0
				19007 0
				19008 0
				19009 0
				19010 0
				19011 0
				19012 0
				19013 0
				19014 0
				19015 0
				19016 0
				19017 0
				19018 0
				19019 0
				19020 995026
				19021 0
				19022 0
				19023 0
				19024 0
				19025 0
				19026 0
				19027 0
				19028 0
				19029 0
				19030 0
				19031 0
				19032 0
				19033 0
				19034 0
				19035 1001125
				19036 0
				19037 0
				19038 0
				19039 0
				19040 0
				19041 0
				19042 0
				19043 0
				19044 0
				19045 0
				19046 0
				19047 0
				19048 0
				19049 0
				19050 1022680
				19051 0
				19052 0
				19053 0
				19054 0
				19055 0
				19056 0
				19057 0
				19058 0
				19059 0
				19060 0
				19061 0
				19062 0
				19063 0
				19064 0
				19065 994703
				19066 0
				19067 0
				19068 0
				19069 0
				19070 0
				19071 0
				19072 0
				19073 0
				19074 0
				19075 0
				19076 0
				19077 0
				19078 0
				19079 0
				19080 1002366
				19081 0
				19082 0
				19083 0
				19084 0
				19085 0
				19086 0
				19087 0
				19088 0
				19089 0
				19090 0
				19091 0
				19092 0
				19093 0
				19094 0
				19095 998546
				19096 0
				19097 0
				19098 0
				19099 0
				19100 0
				19101 0
				19102 0
				19103 0
				19104 0
				19105 0
				19106 0
				19107 0
				19108 0
				19109 0
				19110 996679
				19111 0
				19112 0
				19113 0
				19114 0
				19115 0
				19116 0
				19117 0
				19118 0
				19119 0
				19120 0
				19121 0
				19122 0
				19123 0
				19124 0
				19125 1013733
				19126 0
				19127 0
				19128 0
				19129 0
				19130 0
				19131 0
				19132 0
				19133 0
				19134 0
				19135 0
				19136 0
				19137 0
				19138 0
				19139 0
				19140 998528
				19141 0
				19142 0
				19143 0
				19144 0
				19145 0
				19146 0
				19147 0
				19148 0
				19149 0
				19150 0
				19151 0
				19152 0
				19153 0
				19154 0
				19155 1013125
				19156 0
				19157 0
				19158 0
				19159 0
				19160 0
				19161 0
				19162 0
				19163 0
				19164 0
				19165 0
				19166 0
				19167 0
				19168 0
				19169 0
				19170 1003360
				19171 0
				19172 0
				19173 0
				19174 0
				19175 0
				19176 0
				19177 0
				19178 0
				19179 0
				19180 0
				19181 0
				19182 0
				19183 0
				19184 0
				19185 994160
				19186 0
				19187 0
				19188 0
				19189 0
				19190 0
				19191 0
				19192 0
				19193 0
				19194 0
				19195 0
				19196 0
				19197 0
				19198 0
				19199 0
				19200 1016415
				19201 0
				19202 0
				19203 0
				19204 0
				19205 0
				19206 0
				19207 0
				19208 0
				19209 0
				19210 0
				19211 0
				19212 0
				19213 0
				19214 0
				19215 993423
				19216 0
				19217 0
				19218 0
				19219 0
				19220 0
				19221 0
				19222 0
				19223 0
				19224 0
				19225 0
				19226 0
				19227 0
				19228 0
				19229 0
				19230 1002816
				19231 0
				19232 0
				19233 0
				19234 0
				19235 0
				19236 0
				19237 0
				19238 0
				19239 0
				19240 0
				19241 0
				19242 0
				19243 0
				19244 0
				19245 1008509
				19246 0
				19247 0
				19248 0
				19249 0
				19250 0
				19251 0
				19252 0
				19253 0
				19254 0
				19255 0
				19256 0
				19257 0
				19258 0
				19259 0
				19260 1028654
				19261 0
				19262 0
				19263 0
				19264 0
				19265 0
				19266 0
				19267 0
				19268 0
				19269 0
				19270 0
				19271 0
				19272 0
				19273 0
				19274 0
				19275 1010408
				19276 0
				19277 0
				19278 0
				19279 0
				19280 0
				19281 0
				19282 0
				19283 0
				19284 0
				19285 0
				19286 0
				19287 0
				19288 0
				19289 0
				19290 1049129
				19291 0
				19292 0
				19293 0
				19294 0
				19295 0
				19296 0
				19297 0
				19298 0
				19299 0
				19300 0
				19301 0
				19302 0
				19303 0
				19304 0
				19305 1050380
				19306 0
				19307 0
				19308 0
				19309 0
				19310 0
				19311 0
				19312 0
				19313 0
				19314 0
				19315 0
				19316 0
				19317 0
				19318 0
				19319 0
				19320 1008462
				19321 0
				19322 0
				19323 0
				19324 0
				19325 0
				19326 0
				19327 0
				19328 0
				19329 0
				19330 0
				19331 0
				19332 0
				19333 0
				19334 0
				19335 1005796
				19336 0
				19337 0
				19338 0
				19339 0
				19340 0
				19341 0
				19342 0
				19343 0
				19344 0
				19345 0
				19346 0
				19347 0
				19348 0
				19349 0
				19350 990180
				19351 0
				19352 0
				19353 0
				19354 0
				19355 0
				19356 0
				19357 0
				19358 0
				19359 0
				19360 0
				19361 0
				19362 0
				19363 0
				19364 0
				19365 1007718
				19366 0
				19367 0
				19368 0
				19369 0
				19370 0
				19371 0
				19372 0
				19373 0
				19374 0
				19375 0
				19376 0
				19377 0
				19378 0
				19379 0
				19380 985598
				19381 0
				19382 0
				19383 0
				19384 0
				19385 0
				19386 0
				19387 0
				19388 0
				19389 0
				19390 0
				19391 0
				19392 0
				19393 0
				19394 0
				19395 978265
				19396 0
				19397 0
				19398 0
				19399 0
				19400 0
				19401 0
				19402 0
				19403 0
				19404 0
				19405 0
				19406 0
				19407 0
				19408 0
				19409 0
				19410 1005104
				19411 0
				19412 0
				19413 0
				19414 0
				19415 0
				19416 0
				19417 0
				19418 0
				19419 0
				19420 0
				19421 0
				19422 0
				19423 0
				19424 0
				19425 992488
				19426 0
				19427 0
				19428 0
				19429 0
				19430 0
				19431 0
				19432 0
				19433 0
				19434 0
				19435 0
				19436 0
				19437 0
				19438 0
				19439 0
				19440 1000668
				19441 0
				19442 0
				19443 0
				19444 0
				19445 0
				19446 0
				19447 0
				19448 0
				19449 0
				19450 0
				19451 0
				19452 0
				19453 0
				19454 0
				19455 1025480
				19456 0
				19457 0
				19458 0
				19459 0
				19460 0
				19461 0
				19462 0
				19463 0
				19464 0
				19465 0
				19466 0
				19467 0
				19468 0
				19469 0
				19470 1007453
				19471 0
				19472 0
				19473 0
				19474 0
				19475 0
				19476 0
				19477 0
				19478 0
				19479 0
				19480 0
				19481 0
				19482 0
				19483 0
				19484 0
				19485 1017230
				19486 0
				19487 0
				19488 0
				19489 0
				19490 0
				19491 0
				19492 0
				19493 0
				19494 0
				19495 0
				19496 0
				19497 0
				19498 0
				19499 0
				19500 1000914
				19501 0
				19502 0
				19503 0
				19504 0
				19505 0
				19506 0
				19507 0
				19508 0
				19509 0
				19510 0
				19511 0
				19512 0
				19513 0
				19514 0
				19515 996545
				19516 0
				19517 0
				19518 0
				19519 0
				19520 0
				19521 0
				19522 0
				19523 0
				19524 0
				19525 0
				19526 0
				19527 0
				19528 0
				19529 0
				19530 1058076
				19531 0
				19532 0
				19533 0
				19534 0
				19535 0
				19536 0
				19537 0
				19538 0
				19539 0
				19540 0
				19541 0
				19542 0
				19543 0
				19544 0
				19545 1011845
				19546 0
				19547 0
				19548 0
				19549 0
				19550 0
				19551 0
				19552 0
				19553 0
				19554 0
				19555 0
				19556 0
				19557 0
				19558 0
				19559 0
				19560 997454
				19561 0
				19562 0
				19563 0
				19564 0
				19565 0
				19566 0
				19567 0
				19568 0
				19569 0
				19570 0
				19571 0
				19572 0
				19573 0
				19574 0
				19575 1007403
				19576 0
				19577 0
				19578 0
				19579 0
				19580 0
				19581 0
				19582 0
				19583 0
				19584 0
				19585 0
				19586 0
				19587 0
				19588 0
				19589 0
				19590 984032
				19591 0
				19592 0
				19593 0
				19594 0
				19595 0
				19596 0
				19597 0
				19598 0
				19599 0
				19600 0
				19601 0
				19602 0
				19603 0
				19604 0
				19605 1050294
				19606 0
				19607 0
				19608 0
				19609 0
				19610 0
				19611 0
				19612 0
				19613 0
				19614 0
				19615 0
				19616 0
				19617 0
				19618 0
				19619 0
				19620 1047275
				19621 0
				19622 0
				19623 0
				19624 0
				19625 0
				19626 0
				19627 0
				19628 0
				19629 0
				19630 0
				19631 0
				19632 0
				19633 0
				19634 0
				19635 1003812
				19636 0
				19637 0
				19638 0
				19639 0
				19640 0
				19641 0
				19642 0
				19643 0
				19644 0
				19645 0
				19646 0
				19647 0
				19648 0
				19649 0
				19650 995833
				19651 0
				19652 0
				19653 0
				19654 0
				19655 0
				19656 0
				19657 0
				19658 0
				19659 0
				19660 0
				19661 0
				19662 0
				19663 0
				19664 0
				19665 1051182
				19666 0
				19667 0
				19668 0
				19669 0
				19670 0
				19671 0
				19672 0
				19673 0
				19674 0
				19675 0
				19676 0
				19677 0
				19678 0
				19679 0
				19680 992318
				19681 0
				19682 0
				19683 0
				19684 0
				19685 0
				19686 0
				19687 0
				19688 0
				19689 0
				19690 0
				19691 0
				19692 0
				19693 0
				19694 0
				19695 1034288
				19696 0
				19697 0
				19698 0
				19699 0
				19700 0
				19701 0
				19702 0
				19703 0
				19704 0
				19705 0
				19706 0
				19707 0
				19708 0
				19709 0
				19710 1004673
				19711 0
				19712 0
				19713 0
				19714 0
				19715 0
				19716 0
				19717 0
				19718 0
				19719 0
				19720 0
				19721 0
				19722 0
				19723 0
				19724 0
				19725 992152
				19726 0
				19727 0
				19728 0
				19729 0
				19730 0
				19731 0
				19732 0
				19733 0
				19734 0
				19735 0
				19736 0
				19737 0
				19738 0
				19739 0
				19740 981775
				19741 0
				19742 0
				19743 0
				19744 0
				19745 0
				19746 0
				19747 0
				19748 0
				19749 0
				19750 0
				19751 0
				19752 0
				19753 0
				19754 0
				19755 1046561
				19756 0
				19757 0
				19758 0
				19759 0
				19760 0
				19761 0
				19762 0
				19763 0
				19764 0
				19765 0
				19766 0
				19767 0
				19768 0
				19769 0
				19770 1045256
				19771 0
				19772 0
				19773 0
				19774 0
				19775 0
				19776 0
				19777 0
				19778 0
				19779 0
				19780 0
				19781 0
				19782 0
				19783 0
				19784 0
				19785 1013885
				19786 0
				19787 0
				19788 0
				19789 0
				19790 0
				19791 0
				19792 0
				19793 0
				19794 0
				19795 0
				19796 0
				19797 0
				19798 0
				19799 0
				19800 1009555
				19801 0
				19802 0
				19803 0
				19804 0
				19805 0
				19806 0
				19807 0
				19808 0
				19809 0
				19810 0
				19811 0
				19812 0
				19813 0
				19814 0
				19815 997696
				19816 0
				19817 0
				19818 0
				19819 0
				19820 0
				19821 0
				19822 0
				19823 0
				19824 0
				19825 0
				19826 0
				19827 0
				19828 0
				19829 0
				19830 993065
				19831 0
				19832 0
				19833 0
				19834 0
				19835 0
				19836 0
				19837 0
				19838 0
				19839 0
				19840 0
				19841 0
				19842 0
				19843 0
				19844 0
				19845 1015706
				19846 0
				19847 0
				19848 0
				19849 0
				19850 0
				19851 0
				19852 0
				19853 0
				19854 0
				19855 0
				19856 0
				19857 0
				19858 0
				19859 0
				19860 1058267
				19861 0
				19862 0
				19863 0
				19864 0
				19865 0
				19866 0
				19867 0
				19868 0
				19869 0
				19870 0
				19871 0
				19872 0
				19873 0
				19874 0
				19875 1007929
				19876 0
				19877 0
				19878 0
				19879 0
				19880 0
				19881 0
				19882 0
				19883 0
				19884 0
				19885 0
				19886 0
				19887 0
				19888 0
				19889 0
				19890 996995
				19891 0
				19892 0
				19893 0
				19894 0
				19895 0
				19896 0
				19897 0
				19898 0
				19899 0
				19900 0
				19901 0
				19902 0
				19903 0
				19904 0
				19905 999473
				19906 0
				19907 0
				19908 0
				19909 0
				19910 0
				19911 0
				19912 0
				19913 0
				19914 0
				19915 0
				19916 0
				19917 0
				19918 0
				19919 0
				19920 1029512
				19921 0
				19922 0
				19923 0
				19924 0
				19925 0
				19926 0
				19927 0
				19928 0
				19929 0
				19930 0
				19931 0
				19932 0
				19933 0
				19934 0
				19935 995542
				19936 0
				19937 0
				19938 0
				19939 0
				19940 0
				19941 0
				19942 0
				19943 0
				19944 0
				19945 0
				19946 0
				19947 0
				19948 0
				19949 0
				19950 1034927
				19951 0
				19952 0
				19953 0
				19954 0
				19955 0
				19956 0
				19957 0
				19958 0
				19959 0
				19960 0
				19961 0
				19962 0
				19963 0
				19964 0
				19965 1011991
				19966 0
				19967 0
				19968 0
				19969 0
				19970 0
				19971 0
				19972 0
				19973 0
				19974 0
				19975 0
				19976 0
				19977 0
				19978 0
				19979 0
				19980 1013658
				19981 0
				19982 0
				19983 0
				19984 0
				19985 0
				19986 0
				19987 0
				19988 0
				19989 0
				19990 0
				19991 0
				19992 0
				19993 0
				19994 0
				19995 1000809
				19996 0
				19997 0
				19998 0
				19999 0
			};
		\addplot [draw=blue, fill=blue, mark=*, only marks]
		table{%
				x  y
				2160 336821
				2161 321784
				2162 318759
				2163 318063
				2164 321325
				2165 318865
				2166 319969
				2167 318757
				2168 318258
				2169 321353
				2170 316453
				2171 311135
				2172 314078
				2173 314123
				2174 318278
				2175 329518
				2176 321934
				2177 327940
				2178 323699
				2179 319461
				2180 319891
				2181 315099
				2182 317675
				2183 315253
				2184 310484
				2185 320195
				2186 318758
				2187 315897
				2188 317201
				2189 323490
				2190 325202
				2191 316324
				2192 318964
				2193 313034
				2194 317216
				2195 314400
				2196 314360
				2197 309254
				2198 310188
				2199 317000
				2200 318398
				2201 316316
				2202 317430
				2203 315232
				2204 322427
				2205 330236
				2206 318031
				2207 312501
				2208 317054
				2209 323461
				2210 317325
				2211 313512
				2212 314984
				2213 317146
				2214 314569
				2215 316528
				2216 308388
				2217 316908
				2218 312255
				2219 320919
				2220 328479
				2221 320664
				2222 316236
				2223 311231
				2224 313943
				2225 312977
				2226 313423
				2227 317266
				2228 311269
				2229 313327
				2230 313068
				2231 310305
				2232 316639
				2233 312015
				2234 315377
				2235 334806
				2236 320111
				2237 314852
				2238 314033
				2239 322352
				2240 312275
				2241 313921
				2242 320771
				2243 322536
				2244 319948
				2245 316513
				2246 313589
				2247 317364
				2248 315660
				2249 313102
				2250 324040
				2251 317509
				2252 335540
				2253 315645
				2254 314285
				2255 318400
				2256 311411
				2257 313627
				2258 310520
				2259 314949
				2260 328795
				2261 315802
				2262 317455
				2263 307387
				2264 317772
				2265 330769
				2266 316756
				2267 313961
				2268 319826
				2269 315598
				2270 319150
				2271 310818
				2272 312212
				2273 318138
				2274 319336
				2275 313666
				2276 313836
				2277 312972
				2278 319989
				2279 316539
				2280 336167
				2281 319398
				2282 319743
				2283 314116
				2284 316499
				2285 313189
				2286 320127
				2287 316102
				2288 318436
				2289 308981
				2290 314252
				2291 314224
				2292 314186
				2293 308320
				2294 319186
				2295 334386
				2296 316179
				2297 330556
				2298 320511
				2299 316072
				2300 312044
				2301 313199
				2302 311689
				2303 314935
				2304 313875
				2305 311547
				2306 320443
				2307 318927
				2308 314945
				2309 311916
				2310 325471
				2311 324716
				2312 316065
				2313 315575
				2314 314370
				2315 311789
				2316 315949
				2317 309621
				2318 312971
				2319 320491
				2320 319734
				2321 313099
				2322 314465
				2323 309401
				2324 313040
				2325 331159
				2326 317047
				2327 316824
				2328 321861
				2329 310355
				2330 312531
				2331 323284
				2332 313311
				2333 320043
				2334 317021
				2335 319319
				2336 313973
				2337 308572
				2338 315075
				2339 313044
				2340 329940
				2341 320972
				2342 315499
				2343 326697
				2344 320900
				2345 318883
				2346 316889
				2347 320657
				2348 313085
				2349 312772
				2350 312971
				2351 322847
				2352 315894
				2353 311643
				2354 315188
				2355 323030
				2356 323833
				2357 313308
				2358 312186
				2359 318119
				2360 312806
				2361 314998
				2362 316236
				2363 315624
				2364 313626
				2365 310026
				2366 317908
				2367 313809
				2368 314014
				2369 311930
				2370 328802
				2371 317898
				2372 320001
				2373 314870
				2374 314897
				2375 314897
				2376 315040
				2377 317512
				2378 316890
				2379 321390
				2380 317668
				2381 317831
				2382 323672
				2383 316291
				2384 319037
				2385 324237
				2386 324321
				2387 315402
				2388 317753
				2389 325604
				2390 322970
				2391 321228
				2392 314957
				2393 317112
				2394 313142
				2395 317507
				2396 322408
				2397 321970
				2398 320901
				2399 322979
				2400 329344
				2401 315246
				2402 320081
				2403 316916
				2404 317862
				2405 315033
				2406 319129
				2407 320692
				2408 324262
				2409 314871
				2410 320635
				2411 317820
				2412 322717
				2413 325137
				2414 314524
				2415 330891
				2416 319104
				2417 322546
				2418 324968
				2419 319300
				2420 323344
				2421 315656
				2422 317612
				2423 314922
				2424 322259
				2425 327371
				2426 315643
				2427 328004
				2428 317617
				2429 319559
				2430 332021
				2431 327473
				2432 322569
				2433 321536
				2434 321281
				2435 313576
				2436 318769
				2437 330882
				2438 320622
				2439 315391
				2440 318814
				2441 320833
				2442 320533
				2443 327022
				2444 320053
				2445 332093
				2446 326082
				2447 319728
				2448 319677
				2449 326379
				2450 312331
				2451 320173
				2452 326003
				2453 320893
				2454 320631
				2455 319342
				2456 316930
				2457 315764
				2458 321378
				2459 313398
				2460 332231
				2461 326284
				2462 316493
				2463 318140
				2464 323649
				2465 319427
				2466 315322
				2467 319028
				2468 317894
				2469 326015
				2470 316875
				2471 315327
				2472 315291
				2473 323756
				2474 321677
				2475 327413
				2476 319623
				2477 336630
				2478 321499
				2479 324090
				2480 320838
				2481 325927
				2482 315668
				2483 313008
				2484 320387
				2485 328532
				2486 318264
				2487 318211
				2488 318347
				2489 322848
				2490 325183
				2491 327488
				2492 313833
				2493 315059
				2494 318565
				2495 323371
				2496 332893
				2497 312585
				2498 316271
				2499 324132
				2500 320442
				2501 315616
				2502 317951
				2503 321658
				2504 328694
				2505 331051
				2506 335104
				2507 321493
				2508 312657
				2509 325884
				2510 317172
				2511 318604
				2512 321137
				2513 319005
				2514 315638
				2515 324447
				2516 323465
				2517 321837
				2518 314874
				2519 317460
				2520 323504
				2521 320782
				2522 321445
				2523 315565
				2524 318032
				2525 328574
				2526 309041
				2527 324053
				2528 317698
				2529 317085
				2530 317293
				2531 315481
				2532 317127
				2533 326059
				2534 320306
				2535 330541
				2536 327989
				2537 324835
				2538 315511
				2539 317146
				2540 319889
				2541 323135
				2542 315397
				2543 321994
				2544 320993
				2545 320739
				2546 320208
				2547 319822
				2548 320609
				2549 316033
				2550 329243
				2551 323677
				2552 322814
				2553 325033
				2554 319161
				2555 322505
				2556 320027
				2557 318770
				2558 322666
				2559 320137
				2560 317632
				2561 335473
				2562 316469
				2563 323479
				2564 323906
				2565 333341
				2566 319480
				2567 316656
				2568 321490
				2569 312172
				2570 319252
				2571 312492
				2572 319679
				2573 319015
				2574 316119
				2575 327546
				2576 320379
				2577 312771
				2578 321881
				2579 319676
				2580 330541
				2581 322253
				2582 317427
				2583 318509
				2584 320586
				2585 321204
				2586 318293
				2587 316389
				2588 321776
				2589 316074
				2590 322969
				2591 322933
				2592 324654
				2593 316201
				2594 313498
				2595 334934
				2596 323744
				2597 320113
				2598 315167
				2599 318525
				2600 322940
				2601 315040
				2602 320424
				2603 327462
				2604 319552
				2605 315130
				2606 321902
				2607 319725
				2608 318624
				2609 314655
				2610 327478
				2611 320653
				2612 321789
				2613 329508
				2614 325056
				2615 319568
				2616 317553
				2617 318240
				2618 320006
				2619 314375
				2620 318514
				2621 320010
				2622 317291
				2623 317440
				2624 318501
				2625 324347
				2626 324679
				2627 316321
				2628 317971
				2629 314189
				2630 325554
				2631 325661
				2632 317394
				2633 315058
				2634 312395
				2635 320894
				2636 319486
				2637 315774
				2638 317441
				2639 319059
				2640 326500
				2641 325890
				2642 323587
				2643 317093
				2644 315473
				2645 329013
				2646 314375
				2647 315198
				2648 314442
				2649 323116
				2650 321477
				2651 318563
				2652 326917
				2653 320213
				2654 318319
				2655 321867
				2656 322270
				2657 314392
				2658 320988
				2659 320966
				2660 317276
				2661 331754
				2662 314368
				2663 322116
				2664 319581
				2665 319022
				2666 317144
				2667 318960
				2668 314869
				2669 319629
				2670 324895
				2671 318450
				2672 323843
				2673 315071
				2674 316230
				2675 320414
				2676 314906
				2677 319428
				2678 318184
				2679 312543
				2680 321135
				2681 317976
				2682 320521
				2683 313804
				2684 318113
				2685 326311
				2686 320738
				2687 315574
				2688 323580
				2689 317966
				2690 322879
				2691 318409
				2692 316387
				2693 320187
				2694 316820
				2695 316919
				2696 319640
				2697 315428
				2698 319460
				2699 329172
				2700 328775
				2701 349370
				2702 318569
				2703 321336
				2704 318526
				2705 318542
				2706 322825
				2707 320408
				2708 320189
				2709 317298
				2710 330490
				2711 317596
				2712 321044
				2713 317960
				2714 319339
				2715 328304
				2716 321338
				2717 318585
				2718 321812
				2719 318915
				2720 313011
				2721 314859
				2722 316578
				2723 320487
				2724 322020
				2725 315687
				2726 317399
				2727 320275
				2728 317617
				2729 327744
				2730 327323
				2731 329322
				2732 314589
				2733 321016
				2734 317443
				2735 314025
				2736 311480
				2737 312335
				2738 314712
				2739 323187
				2740 317449
				2741 317535
				2742 316294
				2743 317906
				2744 320844
				2745 332139
				2746 321144
				2747 326152
				2748 314587
				2749 316509
				2750 321308
				2751 322859
				2752 320109
				2753 319979
				2754 324758
				2755 317015
				2756 314801
				2757 314858
				2758 320121
				2759 324225
				2760 342191
				2761 327116
				2762 318817
				2763 317494
				2764 318050
				2765 314124
				2766 318335
				2767 314890
				2768 317602
				2769 313058
				2770 310893
				2771 314197
				2772 313522
				2773 314661
				2774 314335
				2775 324953
				2776 315928
				2777 318980
				2778 319192
				2779 308914
				2780 318149
				2781 315897
				2782 315916
				2783 316230
				2784 317182
				2785 319496
				2786 312049
				2787 316457
				2788 317630
				2789 313847
				2790 324505
				2791 315362
				2792 316767
				2793 317811
				2794 316140
				2795 315015
				2796 315985
				2797 322029
				2798 310760
				2799 318847
				2800 313156
				2801 316093
				2802 312761
				2803 319565
				2804 316267
				2805 328828
				2806 323478
				2807 318410
				2808 316569
				2809 315666
				2810 319840
				2811 312820
				2812 313743
				2813 320580
				2814 309895
				2815 317629
				2816 312836
				2817 313976
				2818 312687
				2819 317435
				2820 326627
				2821 319898
				2822 318367
				2823 321091
				2824 318439
				2825 324763
				2826 318706
				2827 313582
				2828 314276
				2829 315128
				2830 314745
				2831 314182
				2832 319517
				2833 318353
				2834 321962
				2835 326933
				2836 328240
				2837 316582
				2838 317826
				2839 320542
				2840 310672
				2841 313838
				2842 318229
				2843 314619
				2844 333422
				2845 311787
				2846 315123
				2847 316634
				2848 311565
				2849 322222
				2850 321977
				2851 324444
				2852 316050
				2853 316124
				2854 318887
				2855 318129
				2856 311346
				2857 315260
				2858 318090
				2859 315653
				2860 317037
				2861 316848
				2862 323274
				2863 317186
				2864 313825
				2865 322521
				2866 325878
				2867 320501
				2868 324883
				2869 318058
				2870 309028
				2871 310652
				2872 322476
				2873 329248
				2874 321336
				2875 314475
				2876 319907
				2877 316508
				2878 315883
				2879 321949
				2880 327041
				2881 324092
				2882 319778
				2883 320455
				2884 312143
				2885 316188
				2886 315277
				2887 316790
				2888 317309
				2889 319197
				2890 312808
				2891 317620
				2892 313679
				2893 313199
				2894 323290
				2895 324863
				2896 318046
				2897 318877
				2898 315752
				2899 316725
				2900 321567
				2901 316432
				2902 317019
				2903 314694
				2904 318509
				2905 320948
				2906 317705
				2907 313323
				2908 316094
				2909 318996
				2910 324754
				2911 320106
				2912 325193
				2913 316646
				2914 314254
				2915 318086
				2916 312568
				2917 315199
				2918 312248
				2919 313307
				2920 314227
				2921 315207
				2922 314069
				2923 323782
				2924 314308
				2925 326101
				2926 321316
				2927 323459
				2928 318307
				2929 323842
				2930 321446
				2931 316565
				2932 320551
				2933 316486
				2934 312725
				2935 322492
				2936 317108
				2937 316050
				2938 324470
				2939 313957
				2940 336272
				2941 319664
				2942 318516
				2943 316779
				2944 320833
				2945 315986
				2946 318795
				2947 315088
				2948 323561
				2949 318951
				2950 315924
				2951 318121
				2952 323220
				2953 317138
				2954 315342
				2955 322240
				2956 320537
				2957 320007
				2958 317435
				2959 320356
				2960 316684
				2961 318499
				2962 312727
				2963 315231
				2964 319919
				2965 316072
				2966 314302
				2967 324896
				2968 318068
				2969 317554
				2970 322583
				2971 320433
				2972 316824
				2973 316958
				2974 317263
				2975 315568
				2976 315473
				2977 317950
				2978 316855
				2979 316050
				2980 314001
				2981 316071
				2982 317118
				2983 314517
				2984 316613
				2985 330727
				2986 323367
				2987 321586
				2988 319766
				2989 316327
				2990 313166
				2991 316529
				2992 320634
				2993 312259
				2994 321358
				2995 312414
				2996 323725
				2997 318474
				2998 315158
				2999 315956
				3000 326306
				3001 317888
				3002 319172
				3003 315007
				3004 317421
				3005 314227
				3006 321989
				3007 319648
				3008 321116
				3009 311921
				3010 314722
				3011 316759
				3012 315833
				3013 323337
				3014 324946
				3015 325031
				3016 332671
				3017 317536
				3018 313916
				3019 313695
				3020 313181
				3021 315816
				3022 314572
				3023 310549
				3024 321538
				3025 314591
				3026 315124
				3027 324232
				3028 315419
				3029 316728
				3030 330702
				3031 318203
				3032 317535
				3033 322382
				3034 314348
				3035 316008
				3036 315317
				3037 318283
				3038 314439
				3039 314128
				3040 315694
				3041 310924
				3042 316579
				3043 313916
				3044 317365
				3045 329203
				3046 316989
				3047 322298
				3048 315495
				3049 315393
				3050 314593
				3051 312720
				3052 316140
				3053 320002
				3054 313363
				3055 309594
				3056 311740
				3057 315009
				3058 313766
				3059 312441
				3060 333511
				3061 319325
				3062 317492
				3063 325402
				3064 313336
				3065 319659
				3066 314665
				3067 312059
				3068 320709
				3069 316214
				3070 313283
				3071 321654
				3072 315532
				3073 312484
				3074 311779
				3075 328327
				3076 326529
				3077 319144
				3078 315676
				3079 321075
				3080 315233
				3081 319143
				3082 317771
				3083 312356
				3084 314374
				3085 315796
				3086 315927
				3087 319551
				3088 310108
				3089 326554
				3090 328823
				3091 319876
				3092 320622
				3093 309552
				3094 318137
				3095 318785
				3096 308371
				3097 318558
				3098 316566
				3099 318113
				3100 312847
				3101 317914
				3102 320633
				3103 318405
				3104 323345
				3105 323720
				3106 318251
				3107 319292
				3108 310952
				3109 319835
				3110 312048
				3111 308998
				3112 315132
				3113 321580
				3114 310147
				3115 319679
				3116 322216
				3117 315700
				3118 319994
				3119 315206
				3120 324270
				3121 319730
				3122 312327
				3123 312830
				3124 313120
				3125 318914
				3126 314878
				3127 315913
				3128 321250
				3129 317864
				3130 316520
				3131 315078
				3132 315795
				3133 311786
				3134 314337
				3135 330729
				3136 322401
				3137 315133
				3138 323127
				3139 315407
				3140 313612
				3141 320777
				3142 318254
				3143 315063
				3144 312591
				3145 314507
				3146 322395
				3147 315037
				3148 310982
				3149 311015
				3150 325742
				3151 319845
				3152 321681
				3153 314277
				3154 312746
				3155 314544
				3156 313741
				3157 313891
				3158 319155
				3159 315517
				3160 312892
				3161 313470
				3162 311868
				3163 310796
				3164 320514
				3165 322345
				3166 320945
				3167 319483
				3168 316075
				3169 315616
				3170 319383
				3171 320962
				3172 313481
				3173 318050
				3174 314520
				3175 318145
				3176 309820
				3177 314080
				3178 311696
				3179 312479
				3180 327023
				3181 319399
				3182 320735
				3183 319545
				3184 313064
				3185 322275
				3186 314392
				3187 312709
				3188 308503
				3189 315070
				3190 315960
				3191 318732
				3192 312923
				3193 319381
				3194 318130
				3195 328519
				3196 316189
				3197 316666
				3198 315863
				3199 309148
				3200 319276
				3201 315493
				3202 318967
				3203 315633
				3204 314391
				3205 314977
				3206 314376
				3207 312926
				3208 311027
				3209 316975
				3210 322848
				3211 324758
				3212 320327
				3213 313853
				3214 313157
				3215 314284
				3216 316189
				3217 314278
				3218 315484
				3219 311791
				3220 311312
				3221 313508
				3222 311665
				3223 314067
				3224 314947
				3225 323855
				3226 320996
				3227 319102
				3228 314313
				3229 314632
				3230 315795
				3231 318465
				3232 317641
				3233 308776
				3234 311513
				3235 314655
				3236 316865
				3237 309755
				3238 318097
				3239 317817
				3240 327223
				3241 316892
				3242 314860
				3243 307801
				3244 317513
				3245 317267
				3246 313272
				3247 316214
				3248 314362
				3249 311801
				3250 312421
				3251 313959
				3252 308594
				3253 316444
				3254 316509
				3255 323905
				3256 321738
				3257 317066
				3258 314331
				3259 317170
				3260 312128
				3261 311164
				3262 312814
				3263 316472
				3264 310653
				3265 312600
				3266 309667
				3267 324810
				3268 310998
				3269 308882
				3270 324187
				3271 317379
				3272 319341
				3273 311681
				3274 316091
				3275 314608
				3276 313589
				3277 316510
				3278 305683
				3279 310987
				3280 316337
				3281 307839
				3282 314813
				3283 316889
				3284 316309
				3285 328232
				3286 315629
				3287 323987
				3288 315492
				3289 312770
				3290 312621
				3291 315809
				3292 310320
				3293 314558
				3294 310798
				3295 315913
				3296 316444
				3297 313100
				3298 313731
				3299 311135
				3300 321305
				3301 328761
				3302 313043
				3303 316886
				3304 314087
				3305 314539
				3306 308999
				3307 310684
				3308 309334
				3309 313129
				3310 312283
				3311 316057
				3312 311147
				3313 315025
				3314 316990
				3315 333848
				3316 320715
				3317 323004
				3318 316121
				3319 322318
				3320 315587
				3321 314862
				3322 322861
				3323 318372
				3324 326179
				3325 315162
				3326 322018
				3327 320941
				3328 318790
				3329 317409
				3330 325982
				3331 326120
				3332 315917
				3333 318428
				3334 324057
				3335 324750
				3336 314035
				3337 316795
				3338 318162
				3339 322306
				3340 320674
				3341 317242
				3342 325718
				3343 320544
				3344 316704
				3345 332188
				3346 320774
				3347 317381
				3348 321272
				3349 319392
				3350 317146
				3351 316790
				3352 317309
				3353 340712
				3354 322318
				3355 321727
				3356 321174
				3357 314285
				3358 311889
				3359 320041
				3360 326104
				3361 327213
				3362 319199
				3363 323508
				3364 330947
				3365 320314
				3366 319785
				3367 314047
				3368 324055
				3369 315512
				3370 313273
				3371 316626
				3372 317877
				3373 326322
				3374 317134
				3375 328338
				3376 322687
				3377 324594
				3378 321112
				3379 321024
				3380 320699
				3381 321268
				3382 321171
				3383 323663
				3384 315460
				3385 321058
				3386 314126
				3387 317207
				3388 323100
				3389 317768
				3390 327860
				3391 320811
				3392 316101
				3393 325925
				3394 325509
				3395 320213
				3396 321684
				3397 318210
				3398 312283
				3399 314888
				3400 318784
				3401 321627
				3402 326918
				3403 324647
				3404 321495
				3405 333017
				3406 317035
				3407 319354
				3408 318078
				3409 320252
				3410 317636
				3411 315481
				3412 321599
				3413 314246
				3414 316543
				3415 318142
				3416 320477
				3417 317944
				3418 321767
				3419 318588
				3420 332422
				3421 326688
				3422 322161
				3423 322548
				3424 315154
				3425 317674
				3426 321141
				3427 311765
				3428 314067
				3429 325789
				3430 314205
				3431 320828
				3432 318206
				3433 316675
				3434 317689
				3435 331443
				3436 317094
				3437 321942
				3438 313835
				3439 320279
				3440 320156
				3441 314561
				3442 321500
				3443 321156
				3444 316044
				3445 320877
				3446 321490
				3447 320562
				3448 316231
				3449 319172
				3450 329168
				3451 325918
				3452 321122
				3453 325926
				3454 317636
				3455 317314
				3456 321967
				3457 323123
				3458 315394
				3459 321430
				3460 319414
				3461 328753
				3462 319250
				3463 318962
				3464 316663
				3465 328712
				3466 324027
				3467 319958
				3468 321414
				3469 323117
				3470 318425
				3471 317297
				3472 331883
				3473 316952
				3474 317801
				3475 323229
				3476 319293
				3477 326956
				3478 321120
				3479 317159
				3480 326528
				3481 327311
				3482 321248
				3483 321010
				3484 321937
				3485 317547
				3486 328801
				3487 315328
				3488 318290
				3489 314048
				3490 323526
				3491 321934
				3492 321855
				3493 315999
				3494 316781
				3495 329768
				3496 328534
				3497 317561
				3498 316213
				3499 323969
				3500 316907
				3501 316892
				3502 330272
				3503 322697
				3504 320648
				3505 317278
				3506 310239
				3507 317527
				3508 315449
				3509 314339
				3510 330956
				3511 323642
				3512 321728
				3513 321851
				3514 317136
				3515 321159
				3516 324610
				3517 319222
				3518 321531
				3519 320148
				3520 322267
				3521 334474
				3522 320574
				3523 322784
				3524 321966
				3525 326681
				3526 326532
				3527 321901
				3528 317823
				3529 314215
				3530 320108
				3531 326214
				3532 312501
				3533 317484
				3534 325162
				3535 316846
				3536 320983
				3537 325372
				3538 318196
				3539 314707
				3540 329488
				3541 319644
				3542 331354
				3543 312170
				3544 321418
				3545 323006
				3546 321444
				3547 316610
				3548 319283
				3549 315499
				3550 322471
				3551 321037
				3552 321096
				3553 317426
				3554 324600
				3555 324513
				3556 325043
				3557 314918
				3558 319537
				3559 317950
				3560 313919
				3561 319941
				3562 318456
				3563 311642
				3564 318313
				3565 322452
				3566 315729
				3567 317243
				3568 316917
				3569 319464
				3570 329107
				3571 326892
				3572 323664
				3573 322261
				3574 321321
				3575 315282
				3576 317840
				3577 315177
				3578 324143
				3579 322769
				3580 324072
				3581 322458
				3582 320551
				3583 316173
				3584 314396
				3585 328713
				3586 321322
				3587 320473
				3588 319456
				3589 322703
				3590 317071
				3591 325348
				3592 318815
				3593 321755
				3594 322848
				3595 322166
				3596 319955
				3597 314504
				3598 318426
				3599 321121
				3600 330979
				3601 322790
				3602 317912
				3603 324894
				3604 322972
				3605 316149
				3606 318194
				3607 315702
				3608 320223
				3609 319553
				3610 328958
				3611 322517
				3612 312690
				3613 316776
				3614 322032
				3615 329411
				3616 324203
				3617 332997
				3618 320602
				3619 324388
				3620 315410
				3621 324494
				3622 316792
				3623 314511
				3624 319453
				3625 318120
				3626 319608
				3627 319569
				3628 323544
				3629 313357
				3630 329638
				3631 325295
				3632 317556
				3633 316700
				3634 324517
				3635 318586
				3636 318833
				3637 314676
				3638 315609
				3639 314496
				3640 331432
				3641 316374
				3642 322932
				3643 316762
				3644 318303
				3645 322557
				3646 317437
				3647 325026
				3648 326059
				3649 325104
				3650 323420
				3651 331718
				3652 316015
				3653 321240
				3654 321049
				3655 315780
				3656 316032
				3657 320650
				3658 320401
				3659 326499
				3660 325059
				3661 331625
				3662 319157
				3663 324076
				3664 316366
				3665 325023
				3666 318583
				3667 317583
				3668 315224
				3669 319781
				3670 321326
				3671 316602
				3672 323423
				3673 315314
				3674 313737
				3675 327307
				3676 322671
				3677 324165
				3678 321333
				3679 318928
				3680 325901
				3681 312775
				3682 316833
				3683 328490
				3684 320657
				3685 320740
				3686 320959
				3687 317817
				3688 326136
				3689 314194
				3690 326528
				3691 327379
				3692 320324
				3693 315781
				3694 321354
				3695 316306
				3696 318303
				3697 318437
				3698 322048
				3699 328727
				3700 310252
				3701 317590
				3702 316396
				3703 316291
				3704 322252
				3705 326060
				3706 323313
				3707 317067
				3708 314400
				3709 315808
				3710 315990
				3711 320777
				3712 316428
				3713 318823
				3714 321670
				3715 318235
				3716 313443
				3717 318342
				3718 316960
				3719 315704
				3720 329682
				3721 316677
				3722 319674
				3723 316178
				3724 324728
				3725 326365
				3726 320465
				3727 311644
				3728 324859
				3729 315383
				3730 319191
				3731 317484
				3732 314317
				3733 313166
				3734 310510
				3735 330642
				3736 324319
				3737 311329
				3738 325500
				3739 318498
				3740 318279
				3741 318170
				3742 321297
				3743 316057
				3744 315830
				3745 315774
				3746 320838
				3747 314478
				3748 318638
				3749 321419
				3750 329801
				3751 318940
				3752 318073
				3753 312859
				3754 311881
				3755 316123
				3756 317220
				3757 313547
				3758 316287
				3759 320784
				3760 319108
				3761 319553
				3762 311271
				3763 311417
				3764 321163
				3765 325957
				3766 317944
				3767 312292
				3768 314831
				3769 311520
				3770 319205
				3771 317546
				3772 317467
				3773 313589
				3774 316004
				3775 319865
				3776 316578
				3777 314588
				3778 318490
				3779 311367
				3780 329668
				3781 318861
				3782 312003
				3783 321920
				3784 317132
				3785 317905
				3786 315864
				3787 316183
				3788 313645
				3789 315727
				3790 322057
				3791 314079
				3792 320371
				3793 317471
				3794 319647
				3795 327498
				3796 327729
				3797 325101
				3798 314667
				3799 319164
				3800 315237
				3801 319158
				3802 319820
				3803 320443
				3804 318436
				3805 315518
				3806 310671
				3807 317387
				3808 315570
				3809 323797
				3810 330168
				3811 325916
				3812 319931
				3813 326476
				3814 315519
				3815 312896
				3816 314881
				3817 323549
				3818 314612
				3819 312804
				3820 312199
				3821 318539
				3822 311432
				3823 312562
				3824 312586
				3825 332565
				3826 321249
				3827 316471
				3828 314857
				3829 310340
				3830 315931
				3831 315804
				3832 313666
				3833 314804
				3834 323198
				3835 315221
				3836 312609
				3837 316247
				3838 312095
				3839 312984
				3840 320544
				3841 322266
				3842 320391
				3843 318645
				3844 319044
				3845 316232
				3846 318312
				3847 319734
				3848 316617
				3849 317027
				3850 319187
				3851 320637
				3852 314896
				3853 313119
				3854 325437
				3855 333161
				3856 322041
				3857 322284
				3858 315857
				3859 316316
				3860 322078
				3861 318716
				3862 320072
				3863 322594
				3864 312259
				3865 312841
				3866 315319
				3867 314384
				3868 324314
				3869 314790
				3870 339359
				3871 317910
				3872 318968
				3873 318787
				3874 313580
				3875 313709
				3876 313305
				3877 318254
				3878 321010
				3879 315426
				3880 315951
				3881 323006
				3882 308789
				3883 313944
				3884 315903
				3885 327183
				3886 319531
				3887 314429
				3888 313964
				3889 322346
				3890 319414
				3891 316675
				3892 309967
				3893 313494
				3894 323146
				3895 318106
				3896 312881
				3897 324416
				3898 314657
				3899 317010
				3900 322268
				3901 319515
				3902 316948
				3903 318246
				3904 320915
				3905 312561
				3906 315988
				3907 320190
				3908 314270
				3909 311633
				3910 316373
				3911 320387
				3912 312947
				3913 320090
				3914 320030
				3915 330230
				3916 319589
				3917 331986
				3918 322410
				3919 317803
				3920 314221
				3921 314271
				3922 318194
				3923 312635
				3924 317427
				3925 313705
				3926 317899
				3927 312425
				3928 312125
				3929 318724
				3930 325413
				3931 319429
				3932 316654
				3933 313441
				3934 309791
				3935 312635
				3936 322537
				3937 317203
				3938 311328
				3939 315406
				3940 314545
				3941 312701
				3942 312567
				3943 313935
				3944 320311
				3945 325864
				3946 338506
				3947 318833
				3948 315605
				3949 315104
				3950 316501
				3951 313991
				3952 324283
				3953 315322
				3954 334706
				3955 315750
				3956 318722
				3957 317448
				3958 317209
				3959 318587
				3960 328430
				3961 320026
				3962 316507
				3963 320329
				3964 312361
				3965 311661
				3966 320647
				3967 321436
				3968 318033
				3969 318155
				3970 313952
				3971 314029
				3972 314465
				3973 322555
				3974 310182
				3975 322883
				3976 319626
				3977 314431
				3978 316723
				3979 316824
				3980 312591
				3981 314060
				3982 316894
				3983 323043
				3984 315205
				3985 317918
				3986 311656
				3987 317709
				3988 314656
				3989 312134
				3990 326107
				3991 320987
				3992 321310
				3993 323933
				3994 320618
				3995 320528
				3996 314164
				3997 313566
				3998 316340
				3999 319189
				4000 312853
				4001 316897
				4002 310328
				4003 316708
				4004 316810
				4005 327724
				4006 323212
				4007 318532
				4008 313198
				4009 318170
				4010 314120
				4011 319791
				4012 314376
				4013 314198
				4014 310641
				4015 314240
				4016 312274
				4017 317042
				4018 312093
				4019 320235
				4020 328549
				4021 328803
				4022 317278
				4023 313549
				4024 316303
				4025 320330
				4026 322771
				4027 314705
				4028 313614
				4029 323677
				4030 315537
				4031 313903
				4032 311799
				4033 315296
				4034 311356
				4035 325462
				4036 317372
				4037 316388
				4038 318183
				4039 324841
				4040 317109
				4041 313319
				4042 313227
				4043 312236
				4044 309618
				4045 310388
				4046 317086
				4047 318046
				4048 312870
				4049 314096
				4050 328297
				4051 320114
				4052 312293
				4053 319614
				4054 316014
				4055 315551
				4056 315141
				4057 316637
				4058 320646
				4059 315872
				4060 317772
				4061 318352
				4062 312686
				4063 320194
				4064 313730
				4065 325302
				4066 318185
				4067 314172
				4068 321485
				4069 316172
				4070 316979
				4071 319022
				4072 310212
				4073 311996
				4074 319021
				4075 317275
				4076 328177
				4077 316542
				4078 309095
				4079 315999
				4080 322400
				4081 319614
				4082 319308
				4083 313893
				4084 313236
				4085 316939
				4086 311509
				4087 313660
				4088 315036
				4089 310102
				4090 309010
				4091 316682
				4092 320899
				4093 311700
				4094 313780
				4095 324494
				4096 317792
				4097 322670
				4098 310100
				4099 319476
				4100 317706
				4101 317293
				4102 313990
				4103 317158
				4104 315941
				4105 326537
				4106 318255
				4107 312644
				4108 315684
				4109 319177
				4110 321547
				4111 327713
				4112 320253
				4113 313478
				4114 313845
				4115 321072
				4116 320831
				4117 313503
				4118 320311
				4119 316685
				4120 316563
				4121 317309
				4122 316343
				4123 315746
				4124 316691
				4125 322207
				4126 317562
				4127 318894
				4128 308430
				4129 315792
				4130 315400
				4131 310188
				4132 314197
				4133 313963
				4134 311428
				4135 320415
				4136 318283
				4137 314253
				4138 313205
				4139 317424
				4140 327351
				4141 321915
				4142 318628
				4143 314438
				4144 317484
				4145 323190
				4146 310852
				4147 310404
				4148 314717
				4149 324395
				4150 315242
				4151 316222
				4152 343034
				4153 321347
				4154 309546
				4155 328241
				4156 322336
				4157 318375
				4158 309991
				4159 317975
				4160 316425
				4161 313973
				4162 314094
				4163 310805
				4164 313995
				4165 310614
				4166 320599
				4167 311129
				4168 318772
				4169 312882
				4170 326346
				4171 319890
				4172 320190
				4173 318011
				4174 321191
				4175 315513
				4176 311787
				4177 312057
				4178 316456
				4179 321483
				4180 314156
				4181 317772
				4182 315375
				4183 318570
				4184 316377
				4185 324529
				4186 319686
				4187 318182
				4188 315586
				4189 313736
				4190 318163
				4191 320953
				4192 319495
				4193 309976
				4194 314314
				4195 311057
				4196 318988
				4197 314648
				4198 315371
				4199 315611
				4200 330486
				4201 321406
				4202 316122
				4203 320732
				4204 313622
				4205 310439
				4206 313768
				4207 316974
				4208 316199
				4209 314188
				4210 311817
				4211 316355
				4212 309836
				4213 312936
				4214 313606
				4215 329324
				4216 319238
				4217 319414
				4218 310927
				4219 316188
				4220 318778
				4221 313132
				4222 313502
				4223 313946
				4224 319611
				4225 315913
				4226 318041
				4227 314858
				4228 311599
				4229 314112
				4230 321612
				4231 326673
				4232 319674
				4233 321562
				4234 315390
				4235 319885
				4236 316035
				4237 319619
				4238 312218
				4239 321621
				4240 320660
				4241 311725
				4242 315368
				4243 319129
				4244 309498
				4245 338702
				4246 316267
				4247 317088
				4248 319039
				4249 315212
				4250 316846
				4251 310937
				4252 311026
				4253 323170
				4254 317069
				4255 312822
				4256 316369
				4257 309140
				4258 313602
				4259 326804
				4260 326503
				4261 322724
				4262 316785
				4263 315723
				4264 320169
				4265 322360
				4266 313006
				4267 321514
				4268 318232
				4269 313428
				4270 317490
				4271 324740
				4272 322965
				4273 317804
				4274 322706
				4275 320941
				4276 321606
				4277 326803
				4278 318840
				4279 319426
				4280 320815
				4281 313025
				4282 313047
				4283 315531
				4284 314434
				4285 321592
				4286 318810
				4287 319492
				4288 325166
				4289 313021
				4290 328495
				4291 316642
				4292 317413
				4293 331835
				4294 319154
				4295 331107
				4296 315674
				4297 327597
				4298 318648
				4299 319955
				4300 314146
				4301 324230
				4302 327915
				4303 322180
				4304 318482
				4305 322425
				4306 318789
				4307 319363
				4308 320104
				4309 317511
				4310 315718
				4311 312498
				4312 317049
				4313 313000
				4314 316978
				4315 315604
				4316 321791
				4317 315739
				4318 322546
				4319 320286
				4320 325199
				4321 326055
				4322 318130
				4323 327308
				4324 316915
				4325 317176
				4326 316892
				4327 324462
				4328 319102
				4329 322384
				4330 316506
				4331 317615
				4332 317043
				4333 315729
				4334 326948
				4335 330093
				4336 328190
				4337 318866
				4338 324083
				4339 311279
				4340 322951
				4341 319500
				4342 325966
				4343 312706
				4344 323404
				4345 315521
				4346 319890
				4347 317947
				4348 319477
				4349 326053
				4350 320384
				4351 324010
				4352 319930
				4353 317615
				4354 321537
				4355 320431
				4356 321249
				4357 318041
				4358 319689
				4359 324488
				4360 312237
				4361 322482
				4362 316731
				4363 321093
				4364 318082
				4365 328773
				4366 322240
				4367 330630
				4368 322556
				4369 319387
				4370 318731
				4371 322946
				4372 330168
				4373 322956
				4374 318956
				4375 317538
				4376 315315
				4377 315987
				4378 316153
				4379 325324
				4380 330790
				4381 320189
				4382 319959
				4383 317504
				4384 326478
				4385 320553
				4386 318527
				4387 319874
				4388 318284
				4389 318537
				4390 320700
				4391 327963
				4392 316557
				4393 314702
				4394 321783
				4395 327542
				4396 322326
				4397 318846
				4398 322645
				4399 312998
				4400 322520
				4401 316826
				4402 322824
				4403 322286
				4404 312676
				4405 320771
				4406 318381
				4407 321270
				4408 322783
				4409 322297
				4410 328170
				4411 322274
				4412 319936
				4413 321680
				4414 322836
				4415 316341
				4416 318297
				4417 321446
				4418 335079
				4419 321748
				4420 318771
				4421 322612
				4422 312308
				4423 317903
				4424 324777
				4425 330177
				4426 322674
				4427 319381
				4428 315793
				4429 316773
				4430 325411
				4431 315223
				4432 317910
				4433 320194
				4434 321068
				4435 324756
				4436 314815
				4437 321373
				4438 315447
				4439 321917
				4440 328378
				4441 321162
				4442 343776
				4443 314406
				4444 318867
				4445 320066
				4446 312470
				4447 323169
				4448 316868
				4449 315940
				4450 313686
				4451 316407
				4452 324191
				4453 320944
				4454 320344
				4455 329234
				4456 326408
				4457 314845
				4458 324029
				4459 322684
				4460 317756
				4461 331688
				4462 321529
				4463 325194
				4464 316256
				4465 319375
				4466 319416
				4467 313579
				4468 315428
				4469 326324
				4470 333975
				4471 316625
				4472 317129
				4473 319989
				4474 319067
				4475 326023
				4476 320700
				4477 319100
				4478 320338
				4479 314915
				4480 324484
				4481 318919
				4482 316869
				4483 319255
				4484 313658
				4485 331886
				4486 321925
				4487 322978
				4488 320652
				4489 322074
				4490 320113
				4491 320221
				4492 321914
				4493 319799
				4494 321520
				4495 316721
				4496 318247
				4497 318891
				4498 318959
				4499 329487
				4500 327630
				4501 329957
				4502 319566
				4503 325702
				4504 321632
				4505 319087
				4506 318191
				4507 318706
				4508 319542
				4509 324507
				4510 330726
				4511 318209
				4512 314211
				4513 315721
				4514 319474
				4515 325601
				4516 324292
				4517 319895
				4518 319330
				4519 318255
				4520 321575
				4521 316636
				4522 313378
				4523 318069
				4524 317649
				4525 317118
				4526 315698
				4527 322687
				4528 319161
				4529 325921
				4530 329331
				4531 329640
				4532 317219
				4533 320715
				4534 312248
				4535 316942
				4536 317913
				4537 320048
				4538 312085
				4539 318648
				4540 320024
				4541 321019
				4542 317607
				4543 319522
				4544 318970
				4545 327837
				4546 322395
				4547 319526
				4548 322135
				4549 313725
				4550 331102
				4551 311984
				4552 316905
				4553 323599
				4554 314336
				4555 316189
				4556 323636
				4557 316638
				4558 315224
				4559 328240
				4560 334089
				4561 318895
				4562 320972
				4563 318620
				4564 317576
				4565 320800
				4566 313497
				4567 319500
				4568 315545
				4569 322861
				4570 328944
				4571 326203
				4572 317786
				4573 318019
				4574 315590
				4575 327549
				4576 323558
				4577 315051
				4578 313491
				4579 319561
				4580 312797
				4581 319380
				4582 318402
				4583 315186
				4584 314991
				4585 320121
				4586 325043
				4587 319255
				4588 329384
				4589 318040
				4590 333431
				4591 323077
				4592 318618
				4593 318115
				4594 317818
				4595 318733
				4596 315917
				4597 313240
				4598 319063
				4599 324176
				4600 317907
				4601 313721
				4602 312562
				4603 324366
				4604 321405
				4605 330242
				4606 324775
				4607 318504
				4608 316789
				4609 327582
				4610 318835
				4611 319552
				4612 320307
				4613 317740
				4614 314054
				4615 323738
				4616 314596
				4617 329526
				4618 316702
				4619 323606
				4620 324327
				4621 322770
				4622 317354
				4623 319316
				4624 317595
				4625 321651
				4626 318442
				4627 310253
				4628 321109
				4629 316641
				4630 316683
				4631 317494
				4632 314654
				4633 322167
				4634 319100
				4635 329101
				4636 324081
				4637 318404
				4638 323510
				4639 314444
				4640 315454
				4641 322661
				4642 324101
				4643 315532
				4644 315871
				4645 313125
				4646 315900
				4647 317261
				4648 315155
				4649 319078
				4650 327258
				4651 324534
				4652 321691
				4653 318464
				4654 319738
				4655 316990
				4656 317667
				4657 311174
				4658 315969
				4659 314864
				4660 314723
				4661 317234
				4662 312005
				4663 312778
				4664 316547
				4665 324109
				4666 319289
				4667 327435
				4668 321800
				4669 320609
				4670 314586
				4671 318693
				4672 317185
				4673 313114
				4674 318917
				4675 320764
				4676 316721
				4677 314874
				4678 318581
				4679 314090
				4680 324627
				4681 321689
				4682 316111
				4683 318129
				4684 317830
				4685 322835
				4686 315223
				4687 317370
				4688 316480
				4689 314502
				4690 321636
				4691 315191
				4692 322428
				4693 317025
				4694 315297
				4695 347202
				4696 318941
				4697 336891
				4698 318177
				4699 314467
				4700 314558
				4701 311803
				4702 322809
				4703 319747
				4704 321986
				4705 314428
				4706 320146
				4707 321613
				4708 316341
				4709 315759
				4710 325588
				4711 319815
				4712 318624
				4713 317599
				4714 324707
				4715 315398
				4716 313736
				4717 315427
				4718 314025
				4719 318437
				4720 317539
				4721 316786
				4722 316673
				4723 311519
				4724 314454
				4725 322795
				4726 319634
				4727 315853
				4728 313357
				4729 318188
				4730 313675
				4731 318474
				4732 311487
				4733 322990
				4734 322095
				4735 318349
				4736 314572
				4737 311292
				4738 319688
				4739 314309
				4740 326242
				4741 323481
				4742 319709
				4743 312914
				4744 320102
				4745 312442
				4746 316466
				4747 315224
				4748 319486
				4749 314351
				4750 314774
				4751 321823
				4752 315621
				4753 314515
				4754 311469
				4755 327832
				4756 323024
				4757 324938
				4758 323861
				4759 311218
				4760 317952
				4761 322959
				4762 315480
				4763 317602
				4764 314796
				4765 314956
				4766 317439
				4767 323040
				4768 315043
				4769 323788
				4770 329227
				4771 329892
				4772 319683
				4773 319339
				4774 322849
				4775 311131
				4776 316927
				4777 324995
				4778 312273
				4779 316349
				4780 314315
				4781 315192
				4782 317299
				4783 317883
				4784 317498
				4785 325748
				4786 319307
				4787 315861
				4788 313118
				4789 320727
				4790 315752
				4791 316522
				4792 309583
				4793 319974
				4794 316616
				4795 313803
				4796 310202
				4797 315896
				4798 343305
				4799 314548
				4800 325198
				4801 320327
				4802 313487
				4803 317496
				4804 318058
				4805 313723
				4806 311177
				4807 317982
				4808 322552
				4809 313327
				4810 317827
				4811 320268
				4812 315052
				4813 316013
				4814 316502
				4815 328013
				4816 322373
				4817 318522
				4818 320613
				4819 314993
				4820 315608
				4821 314742
				4822 315289
				4823 322511
				4824 314116
				4825 316986
				4826 324435
				4827 314404
				4828 310538
				4829 313498
				4830 329282
				4831 325239
				4832 315116
				4833 315874
				4834 317614
				4835 315638
				4836 313506
				4837 316989
				4838 309883
				4839 314489
				4840 309975
				4841 317883
				4842 311807
				4843 316204
				4844 314108
				4845 329825
				4846 317198
				4847 316833
				4848 313007
				4849 315930
				4850 324054
				4851 312578
				4852 311311
				4853 317637
				4854 309523
				4855 320316
				4856 313028
				4857 315949
				4858 314565
				4859 315471
				4860 331260
				4861 316843
				4862 313133
				4863 320089
				4864 318045
				4865 321189
				4866 315723
				4867 317944
				4868 323727
				4869 316049
				4870 315345
				4871 318934
				4872 312549
				4873 321485
				4874 313688
				4875 336743
				4876 322133
				4877 317124
				4878 314463
				4879 315108
				4880 317413
				4881 316351
				4882 313443
				4883 328787
				4884 313745
				4885 317615
				4886 311118
				4887 320219
				4888 312443
				4889 307089
				4890 324450
				4891 320284
				4892 319252
				4893 324997
				4894 320603
				4895 317862
				4896 317756
				4897 317051
				4898 311245
				4899 319100
				4900 315869
				4901 320970
				4902 320688
				4903 311514
				4904 315444
				4905 328968
				4906 320717
				4907 312268
				4908 318154
				4909 315399
				4910 313026
				4911 316699
				4912 319903
				4913 312542
				4914 319917
				4915 317965
				4916 312843
				4917 313987
				4918 314064
				4919 316402
				4920 329678
				4921 320656
				4922 325821
				4923 322637
				4924 316135
				4925 325921
				4926 309843
				4927 313365
				4928 314706
				4929 314329
				4930 333249
				4931 314536
				4932 312205
				4933 322972
				4934 311306
				4935 324960
				4936 320238
				4937 317592
				4938 313963
				4939 311980
				4940 322908
				4941 314861
				4942 314898
				4943 310343
				4944 317265
				4945 317739
				4946 318307
				4947 317338
				4948 317052
				4949 316307
				4950 335339
				4951 320759
				4952 321873
				4953 314841
				4954 321137
				4955 313101
				4956 323019
				4957 318007
				4958 314667
				4959 314536
				4960 311363
				4961 312514
				4962 321088
				4963 315509
				4964 317743
				4965 321909
				4966 329925
				4967 315311
				4968 314474
				4969 312688
				4970 312645
				4971 310135
				4972 315300
				4973 315072
				4974 323112
				4975 309643
				4976 314943
				4977 326884
				4978 310304
				4979 315229
				4980 325394
				4981 326333
				4982 318422
				4983 321142
				4984 315732
				4985 314572
				4986 313451
				4987 320652
				4988 314154
				4989 315192
				4990 316296
				4991 313447
				4992 318558
				4993 316065
				4994 312717
				4995 331452
				4996 319354
				4997 335784
				4998 313467
				4999 317912
				5000 316447
				5001 315284
				5002 316199
				5003 317226
				5004 315701
				5005 320664
				5006 314471
				5007 317580
				5008 313619
				5009 309370
				5010 329343
				5011 321922
				5012 323968
				5013 313884
				5014 319064
				5015 314100
				5016 316648
				5017 318482
				5018 318682
				5019 317020
				5020 317213
				5021 312725
				5022 317684
				5023 315422
				5024 310637
				5025 327228
				5026 321951
				5027 318125
				5028 317870
				5029 314881
				5030 318400
				5031 314840
				5032 312182
				5033 316405
				5034 315180
				5035 318495
				5036 316788
				5037 314434
				5038 314836
				5039 314238
				5040 322262
				5041 319585
				5042 314371
				5043 319451
				5044 324502
				5045 308815
				5046 315938
				5047 320547
				5048 314692
				5049 308803
				5050 323121
				5051 320184
				5052 327695
				5053 317469
				5054 320423
				5055 325158
				5056 317150
				5057 317059
				5058 315385
				5059 315002
				5060 315539
				5061 313761
				5062 331192
				5063 311726
				5064 313037
				5065 313126
				5066 313989
				5067 314246
				5068 312337
				5069 311491
				5070 324896
				5071 321199
				5072 322488
				5073 314267
				5074 312191
				5075 320055
				5076 314233
				5077 317776
				5078 321597
				5079 311615
				5080 321951
				5081 318766
				5082 310087
				5083 314608
				5084 318590
				5085 324103
				5086 321973
				5087 319661
				5088 312778
				5089 309817
				5090 313208
				5091 311422
				5092 313020
				5093 310524
				5094 319052
				5095 312366
				5096 319857
				5097 311091
				5098 326113
				5099 309958
				5100 328512
				5101 320131
				5102 317513
				5103 313090
				5104 315629
				5105 316915
				5106 308579
				5107 311213
				5108 320461
				5109 317206
				5110 311658
				5111 310861
				5112 313055
				5113 315395
				5114 314848
				5115 325810
				5116 318920
				5117 310866
				5118 320661
				5119 316537
				5120 312366
				5121 309088
				5122 310356
				5123 313555
				5124 311763
				5125 312115
				5126 322510
				5127 310942
				5128 321212
				5129 310808
				5130 326130
				5131 315894
				5132 311734
				5133 315923
				5134 312670
				5135 312300
				5136 321415
				5137 317511
				5138 314118
				5139 318309
				5140 315582
				5141 310399
				5142 309874
				5143 312542
				5144 318567
				5145 324690
				5146 327598
				5147 311614
				5148 315634
				5149 313181
				5150 314451
				5151 317363
				5152 313063
				5153 313096
				5154 324599
				5155 311539
				5156 322352
				5157 315661
				5158 315560
				5159 317876
				5160 324040
				5161 318388
				5162 314206
				5163 317397
				5164 319502
				5165 315125
				5166 321908
				5167 312025
				5168 311884
				5169 320738
				5170 318817
				5171 307850
				5172 317116
				5173 318704
				5174 309931
				5175 324080
				5176 322517
				5177 322554
				5178 319791
				5179 314574
				5180 314535
				5181 320993
				5182 323639
				5183 315675
				5184 321478
				5185 319272
				5186 319089
				5187 317579
				5188 318726
				5189 316718
				5190 328813
				5191 317614
				5192 321797
				5193 323160
				5194 325275
				5195 313035
				5196 322262
				5197 314705
				5198 319889
				5199 317819
				5200 319105
				5201 326456
				5202 321362
				5203 321868
				5204 320115
				5205 329087
				5206 329713
				5207 317186
				5208 319075
				5209 319525
				5210 322804
				5211 320824
				5212 318007
				5213 319247
				5214 321692
				5215 329779
				5216 319309
				5217 320529
				5218 320796
				5219 321961
				5220 325891
				5221 322316
				5222 322585
				5223 320197
				5224 317312
				5225 317102
				5226 318848
				5227 323114
				5228 322095
				5229 317890
				5230 319581
				5231 321039
				5232 319810
				5233 321236
				5234 319285
				5235 333135
				5236 323586
				5237 325353
				5238 321105
				5239 316691
				5240 321333
				5241 320871
				5242 318282
				5243 320808
				5244 313887
				5245 321580
				5246 318134
				5247 314503
				5248 316516
				5249 320964
				5250 326222
				5251 336706
				5252 322505
				5253 323360
				5254 319497
				5255 315308
				5256 322518
				5257 319374
				5258 321604
				5259 316375
				5260 321429
				5261 322541
				5262 320549
				5263 316510
				5264 317876
				5265 325095
				5266 327215
				5267 320914
				5268 318244
				5269 326170
				5270 324971
				5271 330273
				5272 317378
				5273 318386
				5274 316647
				5275 320015
				5276 315334
				5277 314279
				5278 322944
				5279 321598
				5280 328268
				5281 319992
				5282 325735
				5283 316922
				5284 322745
				5285 322637
				5286 321348
				5287 319677
				5288 318999
				5289 321555
				5290 321510
				5291 321762
				5292 323235
				5293 318031
				5294 318981
				5295 329560
				5296 322849
				5297 318238
				5298 316619
				5299 318136
				5300 315633
				5301 319830
				5302 317631
				5303 317541
				5304 316280
				5305 321318
				5306 318520
				5307 323096
				5308 322802
				5309 320311
				5310 319905
				5311 324084
				5312 318574
				5313 324400
				5314 316760
				5315 311336
				5316 317512
				5317 316530
				5318 318709
				5319 321983
				5320 330332
				5321 317486
				5322 319139
				5323 315655
				5324 318987
				5325 325730
				5326 326832
				5327 317255
				5328 317771
				5329 318157
				5330 312775
				5331 319676
				5332 317046
				5333 322634
				5334 322520
				5335 321756
				5336 320825
				5337 315913
				5338 321397
				5339 342692
				5340 329601
				5341 324964
				5342 322609
				5343 319903
				5344 317766
				5345 321188
				5346 324497
				5347 319620
				5348 320093
				5349 322168
				5350 322723
				5351 326329
				5352 320307
				5353 318423
				5354 318657
				5355 326696
				5356 326085
				5357 318005
				5358 322013
				5359 323619
				5360 327888
				5361 320138
				5362 323200
				5363 320385
				5364 314164
				5365 326207
				5366 315207
				5367 318346
				5368 311200
				5369 324165
				5370 325532
				5371 325535
				5372 323734
				5373 317908
				5374 318703
				5375 317337
				5376 314109
				5377 324500
				5378 323270
				5379 326883
				5380 317655
				5381 321736
				5382 322630
				5383 319249
				5384 315515
				5385 332122
				5386 328027
				5387 316607
				5388 320064
				5389 318493
				5390 324116
				5391 319099
				5392 321548
				5393 321078
				5394 317951
				5395 317317
				5396 317054
				5397 316237
				5398 316759
				5399 316211
				5400 330352
				5401 319939
				5402 317459
				5403 312908
				5404 315822
				5405 316872
				5406 313374
				5407 314772
				5408 319271
				5409 329692
				5410 317608
				5411 317889
				5412 321065
				5413 321105
				5414 312125
				5415 329209
				5416 323694
				5417 317099
				5418 320168
				5419 321144
				5420 317048
				5421 319748
				5422 320393
				5423 322307
				5424 325329
				5425 321215
				5426 321933
				5427 309489
				5428 321887
				5429 311882
				5430 337441
				5431 320148
				5432 322760
				5433 317047
				5434 317920
				5435 316641
				5436 324041
				5437 315594
				5438 320673
				5439 326504
				5440 321039
				5441 320103
				5442 321300
				5443 317624
				5444 317015
				5445 330759
				5446 326425
				5447 322127
				5448 325126
				5449 325304
				5450 320357
				5451 318146
				5452 324038
				5453 320261
				5454 317791
				5455 321128
				5456 327835
				5457 322885
				5458 330314
				5459 316725
				5460 333825
				5461 326330
				5462 317450
				5463 321476
				5464 322733
				5465 315532
				5466 314380
				5467 320684
				5468 313003
				5469 314872
				5470 323855
				5471 316341
				5472 314843
				5473 315461
				5474 316565
				5475 329204
				5476 317920
				5477 323870
				5478 318685
				5479 323152
				5480 323739
				5481 323888
				5482 312652
				5483 317207
				5484 315607
				5485 315382
				5486 317617
				5487 326031
				5488 312930
				5489 324011
				5490 325307
				5491 326298
				5492 317173
				5493 318674
				5494 314670
				5495 318630
				5496 315480
				5497 323636
				5498 320289
				5499 314989
				5500 321675
				5501 317069
				5502 318819
				5503 324502
				5504 317298
				5505 331293
				5506 320914
				5507 319254
				5508 329603
				5509 322775
				5510 315349
				5511 319575
				5512 325524
				5513 316988
				5514 314644
				5515 316810
				5516 316444
				5517 315356
				5518 319621
				5519 312455
				5520 331178
				5521 321043
				5522 316433
				5523 318231
				5524 315374
				5525 312962
				5526 316380
				5527 330391
				5528 316483
				5529 317173
				5530 317949
				5531 324042
				5532 324471
				5533 326090
				5534 314032
				5535 327213
				5536 323224
				5537 321926
				5538 320503
				5539 318312
				5540 313454
				5541 322454
				5542 318124
				5543 316442
				5544 320724
				5545 319073
				5546 321575
				5547 315596
				5548 326399
				5549 315122
				5550 329162
				5551 317690
				5552 319933
				5553 319049
				5554 315114
				5555 319178
				5556 313963
				5557 320127
				5558 318332
				5559 319146
				5560 323005
				5561 319322
				5562 318350
				5563 316614
				5564 317924
				5565 324848
				5566 318682
				5567 327711
				5568 319344
				5569 315228
				5570 318605
				5571 320550
				5572 316600
				5573 312898
				5574 318885
				5575 320759
				5576 314599
				5577 309168
				5578 321253
				5579 316964
				5580 331916
				5581 317140
				5582 313455
				5583 313121
				5584 321196
				5585 316027
				5586 311282
				5587 323849
				5588 318802
				5589 314162
				5590 318885
				5591 314670
				5592 310374
				5593 321338
				5594 320982
				5595 322719
				5596 324706
				5597 312640
				5598 315486
				5599 319488
				5600 313686
				5601 319563
				5602 315539
				5603 317545
				5604 327945
				5605 319854
				5606 317501
				5607 320323
				5608 317592
				5609 318912
				5610 329250
				5611 318988
				5612 313868
				5613 316587
				5614 330120
				5615 315635
				5616 313916
				5617 316427
				5618 317868
				5619 315649
				5620 311047
				5621 314791
				5622 313561
				5623 316573
				5624 317887
				5625 330354
				5626 318607
				5627 318229
				5628 311139
				5629 315864
				5630 316568
				5631 314705
				5632 315431
				5633 316605
				5634 313323
				5635 311176
				5636 311517
				5637 320760
				5638 321001
				5639 314751
				5640 323379
				5641 323926
				5642 316501
				5643 320956
				5644 316940
				5645 317649
				5646 319424
				5647 313087
				5648 312962
				5649 315518
				5650 317657
				5651 318274
				5652 315113
				5653 315438
				5654 317617
				5655 337207
				5656 319570
				5657 312318
				5658 317701
				5659 319004
				5660 314239
				5661 322022
				5662 315464
				5663 316676
				5664 314947
				5665 315900
				5666 312290
				5667 316044
				5668 311526
				5669 314996
				5670 326418
				5671 328039
				5672 313365
				5673 313604
				5674 318761
				5675 320854
				5676 311040
				5677 313805
				5678 315804
				5679 317418
				5680 316636
				5681 313576
				5682 310525
				5683 314473
				5684 313569
				5685 333703
				5686 319268
				5687 325678
				5688 316516
				5689 317896
				5690 318767
				5691 314310
				5692 316174
				5693 319489
				5694 316093
				5695 317629
				5696 317449
				5697 309808
				5698 321873
				5699 315893
				5700 327033
				5701 319646
				5702 319938
				5703 319336
				5704 310227
				5705 316695
				5706 326202
				5707 312204
				5708 316168
				5709 314077
				5710 316731
				5711 311909
				5712 318021
				5713 314546
				5714 314302
				5715 332678
				5716 318323
				5717 313330
				5718 317847
				5719 316869
				5720 312231
				5721 326411
				5722 314373
				5723 315470
				5724 316622
				5725 321296
				5726 317010
				5727 318225
				5728 314167
				5729 316898
				5730 330941
				5731 319308
				5732 322103
				5733 314886
				5734 316598
				5735 318137
				5736 321297
				5737 325764
				5738 316960
				5739 317073
				5740 318489
				5741 318187
				5742 316267
				5743 307055
				5744 318959
				5745 323192
				5746 320601
				5747 321291
				5748 316855
				5749 316313
				5750 310286
				5751 317159
				5752 318503
				5753 313825
				5754 313017
				5755 320119
				5756 310318
				5757 315563
				5758 312155
				5759 314824
				5760 327912
				5761 325591
				5762 319656
				5763 312686
				5764 314125
				5765 314149
				5766 316414
				5767 321686
				5768 316518
				5769 319202
				5770 313899
				5771 316108
				5772 316189
				5773 321239
				5774 336850
				5775 327050
				5776 334278
				5777 315447
				5778 316836
				5779 318940
				5780 316108
				5781 314107
				5782 315457
				5783 310361
				5784 322878
				5785 317776
				5786 311996
				5787 311512
				5788 316413
				5789 314050
				5790 327210
				5791 322112
				5792 322215
				5793 314606
				5794 319047
				5795 317039
				5796 313221
				5797 315642
				5798 314774
				5799 313042
				5800 313737
				5801 317306
				5802 312582
				5803 313249
				5804 316587
				5805 328537
				5806 319040
				5807 314222
				5808 315746
				5809 313314
				5810 313512
				5811 320644
				5812 311694
				5813 318165
				5814 318945
				5815 317109
				5816 314260
				5817 308668
				5818 318978
				5819 309336
				5820 330052
				5821 322906
				5822 315569
				5823 321596
				5824 317637
				5825 313485
				5826 318658
				5827 314648
				5828 318494
				5829 310059
				5830 316078
				5831 330995
				5832 319924
				5833 313365
				5834 316464
				5835 326765
				5836 319444
				5837 318278
				5838 314576
				5839 318609
				5840 315608
				5841 318175
				5842 322245
				5843 316110
				5844 321301
				5845 315104
				5846 315784
				5847 318893
				5848 319185
				5849 315797
				5850 323851
				5851 320655
				5852 313366
				5853 311077
				5854 313167
				5855 313468
				5856 316920
				5857 311540
				5858 315043
				5859 313151
				5860 316539
				5861 314072
				5862 311530
				5863 312231
				5864 314758
				5865 330362
				5866 322415
				5867 323145
				5868 311282
				5869 318697
				5870 315825
				5871 318717
				5872 315748
				5873 315587
				5874 313356
				5875 316681
				5876 313161
				5877 315653
				5878 316676
				5879 312782
				5880 329111
				5881 318474
				5882 316845
				5883 309672
				5884 310014
				5885 317436
				5886 318874
				5887 316588
				5888 324833
				5889 318798
				5890 313547
				5891 313110
				5892 307867
				5893 323562
				5894 313070
				5895 330016
				5896 323239
				5897 313558
				5898 325594
				5899 313499
				5900 322210
				5901 315266
				5902 313130
				5903 312440
				5904 317663
				5905 316419
				5906 325627
				5907 315019
				5908 314427
				5909 310406
				5910 324227
				5911 321008
				5912 314920
				5913 321532
				5914 312568
				5915 315238
				5916 321767
				5917 319463
				5918 320146
				5919 313154
				5920 315600
				5921 315881
				5922 322274
				5923 320185
				5924 323495
				5925 320229
				5926 327136
				5927 318308
				5928 311957
				5929 316071
				5930 315006
				5931 313838
				5932 317838
				5933 314731
				5934 322059
				5935 313983
				5936 314846
				5937 316472
				5938 321473
				5939 314406
				5940 329400
				5941 319048
				5942 316332
				5943 319448
				5944 317932
				5945 318436
				5946 316547
				5947 321156
				5948 315812
				5949 308777
				5950 318582
				5951 317428
				5952 318199
				5953 317014
				5954 319680
				5955 322951
				5956 319779
				5957 316103
				5958 311104
				5959 313435
				5960 315956
				5961 309349
				5962 322700
				5963 313906
				5964 311828
				5965 314737
				5966 312892
				5967 318652
				5968 316029
				5969 313446
				5970 324845
				5971 320732
				5972 314361
				5973 328273
				5974 321018
				5975 312972
				5976 316588
				5977 316314
				5978 311122
				5979 318925
				5980 318182
				5981 321554
				5982 317401
				5983 320120
				5984 307472
				5985 327466
				5986 325885
				5987 316509
				5988 313220
				5989 313678
				5990 313073
				5991 320056
				5992 315837
				5993 316530
				5994 313466
				5995 318647
				5996 319518
				5997 318347
				5998 320958
				5999 326566
				6000 331177
				6001 328815
				6002 314671
				6003 319585
				6004 317858
				6005 312189
				6006 313510
				6007 310262
				6008 316830
				6009 326586
				6010 316063
				6011 317339
				6012 314338
				6013 309467
				6014 319645
				6015 326353
				6016 317898
				6017 316208
				6018 315556
				6019 316719
				6020 320812
				6021 316001
				6022 319122
				6023 308473
				6024 313227
				6025 309716
				6026 315389
				6027 318791
				6028 312436
				6029 312222
				6030 323783
				6031 315899
				6032 317248
				6033 315212
				6034 316598
				6035 312816
				6036 317307
				6037 319475
				6038 314085
				6039 310545
				6040 311776
				6041 315314
				6042 313576
				6043 312563
				6044 317513
				6045 326946
				6046 317757
				6047 318523
				6048 314468
				6049 312957
				6050 315512
				6051 315105
				6052 314898
				6053 312611
				6054 315037
				6055 321416
				6056 314139
				6057 313663
				6058 311654
				6059 311935
				6060 333419
				6061 322833
				6062 314154
				6063 315740
				6064 320504
				6065 322666
				6066 320369
				6067 309184
				6068 313727
				6069 312945
				6070 317069
				6071 314556
				6072 314477
				6073 315966
				6074 313880
				6075 330371
				6076 314817
				6077 319676
				6078 314593
				6079 321268
				6080 320229
				6081 315649
				6082 316285
				6083 323759
				6084 316008
				6085 313743
				6086 310995
				6087 312252
				6088 317050
				6089 312775
				6090 326473
				6091 318526
				6092 320249
				6093 318863
				6094 314869
				6095 319207
				6096 316465
				6097 311667
				6098 324640
				6099 317076
				6100 315424
				6101 325889
				6102 318606
				6103 324387
				6104 317603
				6105 327524
				6106 323486
				6107 320840
				6108 318598
				6109 317074
				6110 320691
				6111 316118
				6112 325496
				6113 325287
				6114 324559
				6115 324378
				6116 316061
				6117 325471
				6118 316489
				6119 321112
				6120 323920
				6121 317990
				6122 321368
				6123 324346
				6124 322608
				6125 323493
				6126 315020
				6127 314048
				6128 320221
				6129 319344
				6130 321320
				6131 323682
				6132 322036
				6133 313690
				6134 321724
				6135 329105
				6136 322485
				6137 325545
				6138 324205
				6139 319843
				6140 314712
				6141 323964
				6142 311783
				6143 320099
				6144 318092
				6145 317918
				6146 322753
				6147 317844
				6148 319006
				6149 316542
				6150 330344
				6151 321846
				6152 325059
				6153 314506
				6154 318332
				6155 322881
				6156 321384
				6157 316203
				6158 314617
				6159 315026
				6160 325023
				6161 319731
				6162 316858
				6163 325578
				6164 324872
				6165 323784
				6166 325412
				6167 324462
				6168 313201
				6169 318705
				6170 320333
				6171 321335
				6172 320102
				6173 322208
				6174 314764
				6175 321535
				6176 311167
				6177 317946
				6178 322622
				6179 340772
				6180 332257
				6181 332772
				6182 320634
				6183 320778
				6184 316732
				6185 322844
				6186 326183
				6187 324683
				6188 315871
				6189 319507
				6190 321733
				6191 321238
				6192 315909
				6193 316202
				6194 321952
				6195 328886
				6196 324469
				6197 315959
				6198 314327
				6199 317703
				6200 330506
				6201 317958
				6202 320956
				6203 313631
				6204 322501
				6205 320443
				6206 323363
				6207 320627
				6208 317185
				6209 315435
				6210 326017
				6211 327572
				6212 320271
				6213 314001
				6214 322805
				6215 319349
				6216 317374
				6217 322483
				6218 320453
				6219 325084
				6220 322083
				6221 316872
				6222 317909
				6223 318376
				6224 318506
				6225 327255
				6226 321014
				6227 319224
				6228 317657
				6229 318316
				6230 314326
				6231 315193
				6232 321060
				6233 323716
				6234 320651
				6235 315770
				6236 318478
				6237 313399
				6238 323189
				6239 315107
				6240 333811
				6241 323691
				6242 315546
				6243 317988
				6244 315356
				6245 318125
				6246 319118
				6247 317412
				6248 318744
				6249 318002
				6250 314576
				6251 314996
				6252 324093
				6253 320170
				6254 321671
				6255 329198
				6256 324274
				6257 320894
				6258 317756
				6259 324398
				6260 320069
				6261 316474
				6262 318467
				6263 320742
				6264 316951
				6265 329384
				6266 318512
				6267 326507
				6268 320191
				6269 317496
				6270 326827
				6271 321568
				6272 319158
				6273 323895
				6274 314232
				6275 318296
				6276 316921
				6277 320471
				6278 321659
				6279 312689
				6280 319607
				6281 325922
				6282 321512
				6283 315117
				6284 317153
				6285 324543
				6286 326296
				6287 320412
				6288 320399
				6289 313544
				6290 320782
				6291 316650
				6292 313297
				6293 320529
				6294 321723
				6295 322393
				6296 321968
				6297 321393
				6298 330752
				6299 323333
				6300 328518
				6301 328367
				6302 320002
				6303 315768
				6304 324463
				6305 319683
				6306 319196
				6307 319740
				6308 328617
				6309 319334
				6310 319231
				6311 317046
				6312 317129
				6313 313529
				6314 315710
				6315 327107
				6316 331741
				6317 324047
				6318 323691
				6319 324225
				6320 319767
				6321 319164
				6322 319012
				6323 319432
				6324 315375
				6325 321700
				6326 320797
				6327 323895
				6328 316315
				6329 317704
				6330 321873
				6331 328546
				6332 318998
				6333 330084
				6334 321677
				6335 319560
				6336 311177
				6337 313992
				6338 318788
				6339 316728
				6340 319867
				6341 316252
				6342 317203
				6343 321997
				6344 317737
				6345 328064
				6346 321727
				6347 322529
				6348 325255
				6349 315667
				6350 315952
				6351 322741
				6352 316382
				6353 316986
				6354 321751
				6355 313796
				6356 316053
				6357 322933
				6358 320634
				6359 317414
				6360 328969
				6361 322551
				6362 315750
				6363 315270
				6364 316575
				6365 319748
				6366 319475
				6367 325140
				6368 321763
				6369 313225
				6370 315525
				6371 321692
				6372 316276
				6373 322014
				6374 320033
				6375 328353
				6376 328535
				6377 324604
				6378 323342
				6379 314124
				6380 319062
				6381 319503
				6382 321151
				6383 322758
				6384 314179
				6385 324541
				6386 321431
				6387 318204
				6388 318149
				6389 315283
				6390 324044
				6391 322918
				6392 322019
				6393 318286
				6394 317522
				6395 324053
				6396 322638
				6397 326680
				6398 323627
				6399 320310
				6400 313335
				6401 318898
				6402 316016
				6403 312752
				6404 325801
				6405 329592
				6406 320515
				6407 321175
				6408 325107
				6409 320260
				6410 314000
				6411 322181
				6412 312840
				6413 321654
				6414 313389
				6415 316546
				6416 325710
				6417 319616
				6418 319283
				6419 320794
				6420 327021
				6421 321804
				6422 320851
				6423 312832
				6424 314135
				6425 320432
				6426 318544
				6427 320160
				6428 316332
				6429 319012
				6430 325859
				6431 320850
				6432 319753
				6433 316702
				6434 320717
				6435 325846
				6436 323207
				6437 323550
				6438 314990
				6439 324762
				6440 319409
				6441 321823
				6442 318748
				6443 313008
				6444 318965
				6445 317559
				6446 316653
				6447 320666
				6448 318913
				6449 319937
				6450 328129
				6451 326788
				6452 325090
				6453 321693
				6454 319251
				6455 321066
				6456 331434
				6457 319542
				6458 313502
				6459 318302
				6460 312794
				6461 316496
				6462 320110
				6463 319051
				6464 312923
				6465 329195
				6466 324809
				6467 318412
				6468 321263
				6469 318644
				6470 321377
				6471 321807
				6472 318717
				6473 319138
				6474 323177
				6475 323510
				6476 323739
				6477 321201
				6478 319340
				6479 323451
				6480 331206
				6481 324601
				6482 317313
				6483 316779
				6484 318500
				6485 323777
				6486 357469
				6487 317679
				6488 317515
				6489 312999
				6490 318432
				6491 311239
				6492 326645
				6493 320306
				6494 324750
				6495 322615
				6496 330927
				6497 319019
				6498 313371
				6499 314129
				6500 313355
				6501 314611
				6502 316039
				6503 312205
				6504 322724
				6505 317634
				6506 319595
				6507 315360
				6508 313783
				6509 313741
				6510 328157
				6511 321345
				6512 315439
				6513 319718
				6514 323162
				6515 318337
				6516 316689
				6517 313129
				6518 318067
				6519 311876
				6520 318128
				6521 317552
				6522 315243
				6523 320337
				6524 317989
				6525 327374
				6526 322414
				6527 320459
				6528 321685
				6529 317338
				6530 320509
				6531 310658
				6532 313871
				6533 323082
				6534 318660
				6535 317995
				6536 317172
				6537 314850
				6538 320248
				6539 317730
				6540 333182
				6541 320229
				6542 318486
				6543 326091
				6544 319114
				6545 315362
				6546 318359
				6547 308715
				6548 318967
				6549 315388
				6550 312630
				6551 325034
				6552 312774
				6553 315330
				6554 319104
				6555 329000
				6556 326749
				6557 316045
				6558 319027
				6559 319722
				6560 317474
				6561 311796
				6562 317313
				6563 312978
				6564 319602
				6565 320408
				6566 318146
				6567 317903
				6568 315348
				6569 308405
				6570 323205
				6571 319468
				6572 314272
				6573 316379
				6574 316741
				6575 313992
				6576 313103
				6577 314889
				6578 313256
				6579 318411
				6580 312193
				6581 308760
				6582 315939
				6583 308411
				6584 315985
				6585 324854
				6586 322100
				6587 322195
				6588 319372
				6589 317632
				6590 312429
				6591 315955
				6592 318811
				6593 311432
				6594 317344
				6595 315788
				6596 312779
				6597 309356
				6598 322245
				6599 317728
				6600 336279
				6601 324461
				6602 323155
				6603 319406
				6604 316702
				6605 315832
				6606 310986
				6607 314050
				6608 321968
				6609 317663
				6610 321691
				6611 321823
				6612 316652
				6613 315290
				6614 313365
				6615 335204
				6616 316203
				6617 314361
				6618 317561
				6619 314473
				6620 315019
				6621 315045
				6622 313393
				6623 321287
				6624 317560
				6625 314692
				6626 324110
				6627 326191
				6628 314326
				6629 313175
				6630 327831
				6631 318402
				6632 317316
				6633 318972
				6634 317346
				6635 320531
				6636 321598
				6637 324304
				6638 317205
				6639 317279
				6640 316586
				6641 317642
				6642 317306
				6643 318007
				6644 314284
				6645 323116
				6646 317997
				6647 320938
				6648 315647
				6649 313744
				6650 322985
				6651 311336
				6652 315769
				6653 315793
				6654 315290
				6655 324087
				6656 311737
				6657 320012
				6658 316560
				6659 318262
				6660 327097
				6661 323793
				6662 317411
				6663 314826
				6664 313422
				6665 327311
				6666 312331
				6667 318086
				6668 319778
				6669 317275
				6670 322123
				6671 316167
				6672 313207
				6673 320927
				6674 315942
				6675 330642
				6676 317811
				6677 315241
				6678 313239
				6679 311133
				6680 313273
				6681 311995
				6682 315236
				6683 319441
				6684 318230
				6685 317765
				6686 310440
				6687 322839
				6688 315077
				6689 316020
				6690 330142
				6691 322762
				6692 311731
				6693 314340
				6694 313152
				6695 314811
				6696 314960
				6697 318334
				6698 320623
				6699 315413
				6700 315616
				6701 317183
				6702 320715
				6703 320629
				6704 317078
				6705 325444
				6706 316902
				6707 321466
				6708 318297
				6709 317908
				6710 316075
				6711 322435
				6712 319756
				6713 319525
				6714 315360
				6715 316399
				6716 310511
				6717 312568
				6718 310947
				6719 309309
				6720 326261
				6721 325174
				6722 323072
				6723 319722
				6724 316166
				6725 311819
				6726 315913
				6727 315022
				6728 311807
				6729 314079
				6730 330204
				6731 316838
				6732 320242
				6733 317007
				6734 321021
				6735 325481
				6736 324876
				6737 319002
				6738 309759
				6739 318058
				6740 321614
				6741 314348
				6742 310735
				6743 319791
				6744 316348
				6745 313971
				6746 315690
				6747 310900
				6748 323178
				6749 310577
				6750 349543
				6751 323751
				6752 319500
				6753 318521
				6754 317923
				6755 314919
				6756 312458
				6757 317578
				6758 323932
				6759 313666
				6760 319107
				6761 311403
				6762 314382
				6763 312345
				6764 314144
				6765 323156
				6766 321963
				6767 318656
				6768 327780
				6769 314454
				6770 314805
				6771 321959
				6772 313954
				6773 318836
				6774 319258
				6775 309088
				6776 319700
				6777 319715
				6778 314063
				6779 313338
				6780 325855
				6781 322627
				6782 312991
				6783 312890
				6784 323566
				6785 311879
				6786 314434
				6787 322102
				6788 322347
				6789 316623
				6790 312921
				6791 317260
				6792 313451
				6793 314811
				6794 323736
				6795 333193
				6796 324269
				6797 312603
				6798 316918
				6799 314798
				6800 323378
				6801 316943
				6802 321495
				6803 309046
				6804 323103
				6805 317994
				6806 320403
				6807 312469
				6808 311908
				6809 311454
				6810 330626
				6811 326510
				6812 317282
				6813 317229
				6814 317127
				6815 320637
				6816 318143
				6817 313174
				6818 314700
				6819 316447
				6820 305584
				6821 319371
				6822 318996
				6823 322096
				6824 313864
				6825 344191
				6826 317701
				6827 311884
				6828 316057
				6829 316492
				6830 316792
				6831 313620
				6832 313341
				6833 322087
				6834 317302
				6835 318447
				6836 324907
				6837 316020
				6838 311491
				6839 315724
				6840 327814
				6841 320702
				6842 320428
				6843 314521
				6844 314220
				6845 315613
				6846 322533
				6847 315117
				6848 314135
				6849 316116
				6850 312487
				6851 315040
				6852 312762
				6853 314327
				6854 315692
				6855 324998
				6856 318557
				6857 312632
				6858 317976
				6859 316075
				6860 316324
				6861 320708
				6862 322505
				6863 313981
				6864 317717
				6865 324276
				6866 310126
				6867 314073
				6868 315579
				6869 317999
				6870 329090
				6871 317693
				6872 332811
				6873 321639
				6874 316425
				6875 313445
				6876 316852
				6877 320024
				6878 316288
				6879 318749
				6880 319376
				6881 314515
				6882 312348
				6883 311044
				6884 320781
				6885 334303
				6886 321247
				6887 316917
				6888 315180
				6889 317842
				6890 321812
				6891 313673
				6892 317273
				6893 315215
				6894 313608
				6895 322077
				6896 309925
				6897 309980
				6898 328446
				6899 318349
				6900 329851
				6901 320901
				6902 313554
				6903 321409
				6904 323794
				6905 319376
				6906 312749
				6907 315790
				6908 327271
				6909 314450
				6910 315517
				6911 315568
				6912 313931
				6913 314698
				6914 309998
				6915 323609
				6916 320098
				6917 325753
				6918 320575
				6919 321215
				6920 319942
				6921 310881
				6922 319065
				6923 318206
				6924 310375
				6925 309144
				6926 320944
				6927 313280
				6928 312324
				6929 316952
				6930 324285
				6931 318226
				6932 317505
				6933 311959
				6934 315205
				6935 320955
				6936 328909
				6937 312122
				6938 314306
				6939 309019
				6940 313746
				6941 317613
				6942 315735
				6943 315169
				6944 320768
				6945 322874
				6946 322522
				6947 316015
				6948 310909
				6949 314040
				6950 311273
				6951 311508
				6952 318327
				6953 315481
				6954 318990
				6955 317146
				6956 317919
				6957 315071
				6958 313292
				6959 311575
				6960 321035
				6961 319534
				6962 314332
				6963 315240
				6964 320281
				6965 320750
				6966 312627
				6967 318360
				6968 313812
				6969 315139
				6970 313249
				6971 312084
				6972 323120
				6973 317921
				6974 310879
				6975 330151
				6976 321424
				6977 319754
				6978 312192
				6979 318355
				6980 318227
				6981 313159
				6982 321134
				6983 315107
				6984 312875
				6985 315395
				6986 317389
				6987 308735
				6988 313379
				6989 316396
				6990 327541
				6991 321156
				6992 329476
				6993 322072
				6994 321122
				6995 314268
				6996 311302
				6997 315301
				6998 315041
				6999 311833
				7000 319745
				7001 312687
				7002 314698
				7003 313117
				7004 313496
				7005 329262
				7006 320892
				7007 320732
				7008 314622
				7009 315313
				7010 316214
				7011 325670
				7012 319327
				7013 316363
				7014 318899
				7015 316480
				7016 318131
				7017 320044
				7018 318675
				7019 328675
				7020 325336
				7021 327732
				7022 317981
				7023 318075
				7024 323206
				7025 318894
				7026 317934
				7027 322578
				7028 321388
				7029 320580
				7030 316727
				7031 318014
				7032 324119
				7033 322607
				7034 315689
				7035 331405
				7036 329068
				7037 333892
				7038 317940
				7039 325917
				7040 324901
				7041 321955
				7042 315073
				7043 321975
				7044 315807
				7045 317864
				7046 315721
				7047 317874
				7048 317664
				7049 314685
				7050 329167
				7051 325296
				7052 319240
				7053 320003
				7054 324833
				7055 313387
				7056 322929
				7057 321127
				7058 317692
				7059 327818
				7060 323149
				7061 319009
				7062 318433
				7063 316024
				7064 320798
				7065 330147
				7066 320162
				7067 321275
				7068 317576
				7069 320948
				7070 323200
				7071 315828
				7072 317836
				7073 318992
				7074 318207
				7075 322018
				7076 317838
				7077 318134
				7078 325186
				7079 317390
				7080 359303
				7081 319296
				7082 321918
				7083 324795
				7084 322792
				7085 317563
				7086 321048
				7087 316719
				7088 322133
				7089 323415
				7090 315748
				7091 320382
				7092 319243
				7093 318215
				7094 316064
				7095 323248
				7096 324262
				7097 319561
				7098 316216
				7099 328702
				7100 318889
				7101 325932
				7102 315255
				7103 319113
				7104 320437
				7105 316307
				7106 321182
				7107 313571
				7108 328619
				7109 321841
				7110 329169
				7111 321698
				7112 321408
				7113 315571
				7114 318461
				7115 316310
				7116 314271
				7117 327529
				7118 318396
				7119 320394
				7120 324341
				7121 319519
				7122 318379
				7123 317849
				7124 320096
				7125 326411
				7126 324991
				7127 321648
				7128 312737
				7129 329093
				7130 321165
				7131 324915
				7132 319082
				7133 318417
				7134 316769
				7135 321114
				7136 322072
				7137 323954
				7138 323858
				7139 321874
				7140 325281
				7141 321126
				7142 319595
				7143 314577
				7144 318355
				7145 319704
				7146 318609
				7147 320070
				7148 325244
				7149 311221
				7150 323747
				7151 319328
				7152 323010
				7153 326852
				7154 317332
				7155 329578
				7156 327610
				7157 319127
				7158 316444
				7159 341119
				7160 315437
				7161 317451
				7162 323407
				7163 316776
				7164 324266
				7165 316478
				7166 314770
				7167 334503
				7168 319333
				7169 314357
				7170 328398
				7171 319764
				7172 321625
				7173 314878
				7174 316095
				7175 318556
				7176 319381
				7177 320008
				7178 312444
				7179 315084
				7180 313721
				7181 315203
				7182 311999
				7183 322164
				7184 314449
				7185 330260
				7186 325713
				7187 313500
				7188 321318
				7189 321258
				7190 319269
				7191 317545
				7192 312881
				7193 311871
				7194 318117
				7195 316810
				7196 322636
				7197 315994
				7198 318459
				7199 317912
				7200 331138
				7201 319291
				7202 322498
				7203 315652
				7204 322720
				7205 321300
				7206 317990
				7207 330610
				7208 312664
				7209 323793
				7210 317796
				7211 317444
				7212 323673
				7213 321453
				7214 323038
				7215 329636
				7216 326167
				7217 325988
				7218 317468
				7219 321115
				7220 320822
				7221 318399
				7222 319033
				7223 318696
				7224 321037
				7225 311107
				7226 326049
				7227 325704
				7228 324030
				7229 318134
				7230 326352
				7231 322582
				7232 314120
				7233 319142
				7234 321431
				7235 321541
				7236 325620
				7237 319227
				7238 318306
				7239 320264
				7240 316858
				7241 319090
				7242 318007
				7243 322826
				7244 317719
				7245 321294
				7246 319694
				7247 325432
				7248 319954
				7249 318819
				7250 316044
				7251 321980
				7252 314548
				7253 315648
				7254 322492
				7255 328321
				7256 320434
				7257 321007
				7258 322953
				7259 321921
				7260 327107
				7261 323560
				7262 322891
				7263 315643
				7264 318020
				7265 318380
				7266 320415
				7267 322416
				7268 319722
				7269 316828
				7270 321803
				7271 319482
				7272 323513
				7273 313934
				7274 325083
				7275 325370
				7276 317064
				7277 320773
				7278 319807
				7279 311648
				7280 312505
				7281 319790
				7282 316994
				7283 315056
				7284 327851
				7285 315384
				7286 318255
				7287 314923
				7288 317135
				7289 319756
				7290 326986
				7291 320315
				7292 316395
				7293 327996
				7294 319307
				7295 318742
				7296 315711
				7297 322700
				7298 317470
				7299 318463
				7300 323340
				7301 322725
				7302 321718
				7303 324859
				7304 321665
				7305 331403
				7306 322412
				7307 318286
				7308 318555
				7309 319379
				7310 315178
				7311 318016
				7312 319030
				7313 315585
				7314 317757
				7315 316775
				7316 320896
				7317 317513
				7318 315988
				7319 318443
				7320 323523
				7321 327880
				7322 320418
				7323 317686
				7324 325205
				7325 320292
				7326 317768
				7327 321893
				7328 323727
				7329 325926
				7330 313244
				7331 319855
				7332 320406
				7333 322577
				7334 315506
				7335 333088
				7336 322519
				7337 322414
				7338 319074
				7339 317865
				7340 317979
				7341 314252
				7342 330240
				7343 319633
				7344 320377
				7345 315795
				7346 315599
				7347 317453
				7348 323535
				7349 319197
				7350 329574
				7351 323231
				7352 313606
				7353 316946
				7354 323532
				7355 319497
				7356 318179
				7357 315390
				7358 319810
				7359 321787
				7360 317474
				7361 319074
				7362 323605
				7363 320257
				7364 320701
				7365 330734
				7366 324399
				7367 321113
				7368 316308
				7369 326707
				7370 317695
				7371 315425
				7372 314125
				7373 323712
				7374 320722
				7375 317250
				7376 316681
				7377 318343
				7378 315573
				7379 324035
				7380 325053
				7381 316683
				7382 319749
				7383 322252
				7384 319738
				7385 322270
				7386 319382
				7387 315935
				7388 319641
				7389 318518
				7390 317451
				7391 316748
				7392 329162
				7393 320906
				7394 317921
				7395 326252
				7396 322149
				7397 319295
				7398 316578
				7399 314955
				7400 309880
				7401 313772
				7402 314185
				7403 316005
				7404 322321
				7405 316671
				7406 321266
				7407 310931
				7408 317797
				7409 318594
				7410 330619
				7411 319377
				7412 316550
				7413 315380
				7414 320869
				7415 317013
				7416 312175
				7417 318227
				7418 313012
				7419 315879
				7420 317258
				7421 321122
				7422 312414
				7423 323470
				7424 318799
				7425 327597
				7426 317765
				7427 321434
				7428 319958
				7429 317148
				7430 313329
				7431 328629
				7432 314747
				7433 317501
				7434 316048
				7435 319701
				7436 315039
				7437 316668
				7438 313666
				7439 314669
				7440 324807
				7441 323758
				7442 309483
				7443 315721
				7444 316936
				7445 315709
				7446 319719
				7447 316261
				7448 319723
				7449 316581
				7450 313235
				7451 309367
				7452 317357
				7453 316836
				7454 319103
				7455 332790
				7456 319789
				7457 322200
				7458 314702
				7459 319287
				7460 321268
				7461 313843
				7462 317392
				7463 316639
				7464 308959
				7465 317986
				7466 313631
				7467 316547
				7468 320690
				7469 318890
				7470 329051
				7471 320687
				7472 321206
				7473 314771
				7474 316704
				7475 313608
				7476 313343
				7477 317790
				7478 314971
				7479 315386
				7480 315117
				7481 312904
				7482 316236
				7483 319475
				7484 311198
				7485 325288
				7486 324129
				7487 312917
				7488 311534
				7489 319076
				7490 320807
				7491 312180
				7492 316574
				7493 314506
				7494 316383
				7495 318430
				7496 319967
				7497 319219
				7498 319401
				7499 313785
				7500 323696
				7501 324810
				7502 321948
				7503 312500
				7504 324919
				7505 316735
				7506 314342
				7507 324856
				7508 315527
				7509 317810
				7510 318927
				7511 320040
				7512 314708
				7513 318085
				7514 317966
				7515 326652
				7516 323005
				7517 323500
				7518 319417
				7519 316059
				7520 316531
				7521 318675
				7522 313302
				7523 318724
				7524 321400
				7525 319020
				7526 311828
				7527 316737
				7528 316486
				7529 318833
				7530 323557
				7531 318770
				7532 319748
				7533 320795
				7534 317876
				7535 310514
				7536 319917
				7537 314141
				7538 315787
				7539 320184
				7540 312674
				7541 317551
				7542 316206
				7543 312661
				7544 318258
				7545 330479
				7546 324799
				7547 315661
				7548 315481
				7549 311243
				7550 316042
				7551 313262
				7552 317459
				7553 316882
				7554 323332
				7555 311784
				7556 319808
				7557 317419
				7558 317337
				7559 317364
				7560 329712
				7561 322781
				7562 314821
				7563 317335
				7564 317929
				7565 318214
				7566 320209
				7567 319484
				7568 315450
				7569 321363
				7570 313737
				7571 310836
				7572 316656
				7573 315737
				7574 316547
				7575 325715
				7576 321900
				7577 319422
				7578 313900
				7579 315226
				7580 311804
				7581 311918
				7582 316322
				7583 325705
				7584 320321
				7585 316482
				7586 309579
				7587 316634
				7588 319482
				7589 312143
				7590 329212
				7591 318489
				7592 324833
				7593 313590
				7594 320706
				7595 320443
				7596 315920
				7597 312851
				7598 314283
				7599 311516
				7600 313091
				7601 315888
				7602 316933
				7603 318337
				7604 312414
				7605 326763
				7606 315885
				7607 319337
				7608 318086
				7609 315155
				7610 309162
				7611 322907
				7612 320490
				7613 314376
				7614 312095
				7615 315246
				7616 318187
				7617 317934
				7618 319674
				7619 315683
				7620 322818
				7621 322351
				7622 328410
				7623 321133
				7624 317650
				7625 313693
				7626 315968
				7627 313296
				7628 313300
				7629 315725
				7630 314743
				7631 315901
				7632 310333
				7633 314508
				7634 311595
				7635 328352
				7636 322386
				7637 309590
				7638 314200
				7639 312533
				7640 320722
				7641 316718
				7642 310464
				7643 314312
				7644 317301
				7645 318898
				7646 314187
				7647 316052
				7648 316519
				7649 320535
				7650 333574
				7651 322759
				7652 313842
				7653 315820
				7654 310345
				7655 315304
				7656 313190
				7657 322378
				7658 312182
				7659 323142
				7660 313112
				7661 322249
				7662 316212
				7663 313081
				7664 316613
				7665 324087
				7666 328386
				7667 312666
				7668 317704
				7669 320758
				7670 317866
				7671 313967
				7672 314197
				7673 316584
				7674 316226
				7675 319376
				7676 315213
				7677 314185
				7678 317683
				7679 314759
				7680 323901
				7681 326338
				7682 321976
				7683 319161
				7684 318581
				7685 318573
				7686 317654
				7687 323208
				7688 321273
				7689 314745
				7690 319926
				7691 317309
				7692 323446
				7693 315668
				7694 314152
				7695 322422
				7696 318107
				7697 317125
				7698 311998
				7699 320018
				7700 314355
				7701 310704
				7702 314773
				7703 319730
				7704 314877
				7705 312598
				7706 326532
				7707 314044
				7708 313084
				7709 315842
				7710 330477
				7711 319809
				7712 318387
				7713 320838
				7714 315145
				7715 320657
				7716 330019
				7717 311075
				7718 319146
				7719 310790
				7720 313999
				7721 312897
				7722 312492
				7723 315925
				7724 313045
				7725 324906
				7726 320729
				7727 317314
				7728 313418
				7729 321875
				7730 322136
				7731 314084
				7732 310200
				7733 318869
				7734 311683
				7735 314116
				7736 320516
				7737 315845
				7738 308881
				7739 314175
				7740 328571
				7741 314670
				7742 311527
				7743 318569
				7744 314633
				7745 315409
				7746 317346
				7747 315647
				7748 314168
				7749 315843
				7750 316673
				7751 314674
				7752 319754
				7753 326005
				7754 315489
				7755 325266
				7756 326399
				7757 323695
				7758 316095
				7759 319916
				7760 312121
				7761 316464
				7762 316303
				7763 322131
				7764 312476
				7765 313246
				7766 310166
				7767 315060
				7768 324871
				7769 315699
				7770 316769
				7771 327616
				7772 312889
				7773 327637
				7774 316788
				7775 317940
				7776 310479
				7777 315118
				7778 320661
				7779 310821
				7780 311737
				7781 329073
				7782 314711
				7783 314108
				7784 311609
				7785 327568
				7786 317799
				7787 318959
				7788 318279
				7789 315099
				7790 313979
				7791 316571
				7792 311851
				7793 312737
				7794 315833
				7795 312139
				7796 315172
				7797 317625
				7798 314128
				7799 322068
				7800 328605
				7801 322630
				7802 321901
				7803 316963
				7804 314603
				7805 312191
				7806 317485
				7807 314833
				7808 313679
				7809 308289
				7810 327236
				7811 317324
				7812 318787
				7813 311723
				7814 309751
				7815 328574
				7816 318176
				7817 317743
				7818 323260
				7819 308578
				7820 320737
				7821 318047
				7822 315258
				7823 313628
				7824 319424
				7825 311562
				7826 312510
				7827 316046
				7828 322261
				7829 313012
				7830 334001
				7831 318312
				7832 334923
				7833 314229
				7834 316772
				7835 316835
				7836 320367
				7837 320175
				7838 316724
				7839 310629
				7840 314898
				7841 321798
				7842 316293
				7843 321373
				7844 311403
				7845 329969
				7846 321887
				7847 320411
				7848 312987
				7849 317413
				7850 312509
				7851 318457
				7852 322040
				7853 319612
				7854 314842
				7855 316120
				7856 314381
				7857 318483
				7858 316642
				7859 316677
				7860 322605
				7861 315836
				7862 313127
				7863 317225
				7864 320118
				7865 315171
				7866 320023
				7867 322333
				7868 313754
				7869 315196
				7870 314056
				7871 311999
				7872 311753
				7873 311739
				7874 314557
				7875 329149
				7876 318916
				7877 321876
				7878 317554
				7879 311341
				7880 317370
				7881 308518
				7882 317084
				7883 316454
				7884 319014
				7885 324426
				7886 323644
				7887 312443
				7888 316028
				7889 317630
				7890 326289
				7891 318667
				7892 323230
				7893 316744
				7894 323478
				7895 330604
				7896 315120
				7897 319394
				7898 313140
				7899 314080
				7900 317210
				7901 316595
				7902 314398
				7903 321045
				7904 316128
				7905 323427
				7906 320012
				7907 315878
				7908 320326
				7909 316459
				7910 312633
				7911 315373
				7912 308459
				7913 317075
				7914 311601
				7915 314570
				7916 317115
				7917 314421
				7918 320561
				7919 310883
				7920 324214
				7921 326990
				7922 321123
				7923 325005
				7924 310670
				7925 313516
				7926 316355
				7927 312652
				7928 310102
				7929 317552
				7930 313638
				7931 327630
				7932 318459
				7933 316733
				7934 312535
				7935 325938
				7936 316357
				7937 316328
				7938 320422
				7939 315664
				7940 317485
				7941 312623
				7942 314207
				7943 308409
				7944 311067
				7945 315874
				7946 316908
				7947 309310
				7948 314860
				7949 310783
				7950 319349
				7951 316371
				7952 317372
				7953 318530
				7954 323225
				7955 314152
				7956 317681
				7957 325648
				7958 311701
				7959 316896
				7960 318505
				7961 310403
				7962 314339
				7963 306210
				7964 312011
				7965 322249
				7966 321997
				7967 322614
				7968 322063
				7969 310532
				7970 314880
				7971 309621
				7972 312309
				7973 313257
				7974 315137
				7975 315479
				7976 315732
				7977 322936
				7978 310135
				7979 312361
				7980 326626
				7981 319114
				7982 321127
				7983 316711
				7984 319456
				7985 314775
				7986 322123
				7987 315186
				7988 312422
				7989 308796
				7990 311645
				7991 310914
				7992 309950
				7993 314095
				7994 311151
				7995 324179
				7996 324368
				7997 313422
				7998 317549
				7999 317714
				8000 314094
				8001 312806
				8002 314518
				8003 317794
				8004 320531
				8005 321030
				8006 307532
				8007 313174
				8008 317441
				8009 311387
				8010 319945
				8011 318879
				8012 316830
				8013 324243
				8014 322156
				8015 316693
				8016 316740
				8017 310692
				8018 316042
				8019 313235
				8020 310949
				8021 314463
				8022 311175
				8023 317936
				8024 314796
				8025 321988
				8026 320303
				8027 317783
				8028 309491
				8029 317100
				8030 317753
				8031 312071
				8032 321516
				8033 315483
				8034 311152
				8035 310413
				8036 313520
				8037 311746
				8038 315550
				8039 317166
				8040 324670
				8041 316814
				8042 325023
				8043 314877
				8044 324843
				8045 312548
				8046 314937
				8047 321079
				8048 319160
				8049 315751
				8050 319289
				8051 317622
				8052 323865
				8053 313086
				8054 314354
				8055 326450
				8056 320025
				8057 316494
				8058 312221
				8059 315600
				8060 319236
				8061 314127
				8062 313153
				8063 319680
				8064 313027
				8065 311468
				8066 315936
				8067 311585
				8068 319256
				8069 312421
				8070 318666
				8071 319158
				8072 319536
				8073 314238
				8074 308191
				8075 317083
				8076 315660
				8077 316219
				8078 316019
				8079 314684
				8080 316369
				8081 311326
				8082 317537
				8083 313947
				8084 318909
				8085 322763
				8086 320601
				8087 316079
				8088 310866
				8089 311725
				8090 310034
				8091 317618
				8092 311095
				8093 310243
				8094 316000
				8095 327140
				8096 317456
				8097 319835
				8098 309999
				8099 315700
				8100 327113
				8101 317514
				8102 326123
				8103 316001
				8104 316985
				8105 314022
				8106 310843
				8107 312534
				8108 318266
				8109 317261
				8110 313989
				8111 318332
				8112 316080
				8113 316562
				8114 318888
				8115 333452
				8116 318605
				8117 312360
				8118 308659
				8119 312454
				8120 313089
				8121 315614
				8122 316891
				8123 322418
				8124 322678
				8125 312386
				8126 314078
				8127 312038
				8128 312101
				8129 317232
				8130 325880
				8131 317218
				8132 312091
				8133 326781
				8134 316623
				8135 312007
				8136 315836
				8137 313024
				8138 309784
				8139 310627
				8140 313427
				8141 324157
				8142 313175
				8143 313719
				8144 307535
				8145 326256
				8146 320030
				8147 317676
				8148 311654
				8149 312740
				8150 312813
				8151 314512
				8152 318587
				8153 314043
				8154 312447
				8155 308852
				8156 315138
				8157 310037
				8158 314845
				8159 320252
				8160 327418
				8161 315078
				8162 314543
				8163 310052
				8164 317821
				8165 317465
				8166 309698
				8167 307136
				8168 313814
				8169 318972
				8170 310933
				8171 315384
				8172 311643
				8173 314752
				8174 311662
				8175 332548
				8176 321646
				8177 317470
				8178 317832
				8179 314663
				8180 314298
				8181 316737
				8182 315326
				8183 311515
				8184 315938
				8185 313324
				8186 304474
				8187 316547
				8188 313772
				8189 317548
				8190 322945
				8191 321460
				8192 318281
				8193 310196
				8194 312462
				8195 315119
				8196 311664
				8197 317657
				8198 314559
				8199 315788
				8200 316607
				8201 311480
				8202 312557
				8203 313543
				8204 310416
				8205 328951
				8206 330343
				8207 316716
				8208 313057
				8209 312643
				8210 308238
				8211 317659
				8212 313028
				8213 313152
				8214 318860
				8215 314455
				8216 317438
				8217 311765
				8218 314722
				8219 313529
				8220 323121
				8221 314877
				8222 315591
				8223 314868
				8224 316979
				8225 313033
				8226 310527
				8227 310521
				8228 310192
				8229 314494
				8230 313662
				8231 312555
				8232 319120
				8233 313712
				8234 314949
				8235 325420
				8236 319370
				8237 310359
				8238 314388
				8239 313452
				8240 317222
				8241 316689
				8242 325764
				8243 313788
				8244 320924
				8245 316013
				8246 314398
				8247 313703
				8248 318131
				8249 310629
				8250 322721
				8251 314847
				8252 314156
				8253 314054
				8254 314506
				8255 311750
				8256 314770
				8257 317104
				8258 312501
				8259 313531
				8260 313435
				8261 308200
				8262 311680
				8263 310653
				8264 310331
				8265 324880
				8266 318888
				8267 318548
				8268 315742
				8269 308011
				8270 320499
				8271 303040
				8272 317529
				8273 314123
				8274 316265
				8275 307907
				8276 309898
				8277 314527
				8278 305891
				8279 313366
				8280 326044
				8281 318812
				8282 316144
				8283 313147
				8284 311760
				8285 310401
				8286 315144
				8287 328560
				8288 311796
				8289 310455
				8290 313536
				8291 315962
				8292 311097
				8293 317578
				8294 311395
				8295 327695
				8296 322382
				8297 316875
				8298 321185
				8299 311556
				8300 315503
				8301 313825
				8302 311876
				8303 313998
				8304 313697
				8305 312723
				8306 314260
				8307 317368
				8308 314025
				8309 311462
				8310 325021
				8311 311088
				8312 316188
				8313 310961
				8314 310323
				8315 315195
				8316 317242
				8317 306641
				8318 312511
				8319 315092
				8320 310104
				8321 309235
				8322 317600
				8323 314649
				8324 312139
				8325 324690
				8326 325903
				8327 317663
				8328 318432
				8329 318509
				8330 314566
				8331 312413
				8332 312549
				8333 312161
				8334 308711
				8335 310116
				8336 312862
				8337 318509
				8338 309493
				8339 308642
				8340 325431
				8341 319522
				8342 322468
				8343 317605
				8344 318314
				8345 317705
				8346 314297
				8347 312947
				8348 319855
				8349 312700
				8350 313714
				8351 310740
				8352 311622
				8353 311087
				8354 313249
				8355 325492
				8356 319723
				8357 316355
				8358 315268
				8359 308151
				8360 319261
				8361 310901
				8362 312769
				8363 312786
				8364 320111
				8365 316384
				8366 325852
				8367 321248
				8368 320984
				8369 311121
				8370 330879
				8371 317229
				8372 309754
				8373 313377
				8374 312463
				8375 315357
				8376 313583
				8377 316645
				8378 313524
				8379 310662
				8380 312692
				8381 314848
				8382 314512
				8383 309349
				8384 311079
				8385 319714
				8386 314603
				8387 315699
				8388 316733
				8389 312619
				8390 313621
				8391 311345
				8392 313025
				8393 310999
				8394 318209
				8395 317192
				8396 316286
				8397 310995
				8398 314351
				8399 318096
				8400 324046
				8401 314115
				8402 309872
				8403 311550
				8404 312613
				8405 315045
				8406 320274
				8407 307135
				8408 309744
				8409 311613
				8410 312522
				8411 313404
				8412 305085
				8413 307659
				8414 319960
				8415 322229
				8416 315099
				8417 310452
				8418 311494
				8419 312753
				8420 314812
				8421 311498
				8422 305540
				8423 308180
				8424 311736
				8425 307777
				8426 308542
				8427 317584
				8428 308459
				8429 319581
				8430 322937
				8431 320507
				8432 311544
				8433 324975
				8434 315903
				8435 313589
				8436 313174
				8437 305991
				8438 311399
				8439 310561
				8440 314844
				8441 317921
				8442 311721
				8443 311640
				8444 305101
				8445 338327
				8446 323656
				8447 319676
				8448 312754
				8449 315270
				8450 311810
				8451 316474
				8452 315866
				8453 313940
				8454 312938
				8455 315568
				8456 315375
				8457 320304
				8458 311844
				8459 324064
				8460 331250
				8461 328794
				8462 328034
				8463 324491
				8464 320910
				8465 315046
				8466 322514
				8467 317397
				8468 324251
				8469 317377
				8470 319094
				8471 317088
				8472 319147
				8473 320548
				8474 317985
				8475 340384
				8476 326538
				8477 328444
				8478 323097
				8479 322270
				8480 328718
				8481 320950
				8482 318096
				8483 326319
				8484 318954
				8485 323614
				8486 323474
				8487 317871
				8488 318765
				8489 321449
				8490 339443
				8491 335414
				8492 330110
				8493 330248
				8494 324471
				8495 323456
				8496 326975
				8497 327892
				8498 333078
				8499 322991
				8500 327234
				8501 332339
				8502 330719
				8503 325754
				8504 326260
				8505 336707
				8506 330213
				8507 331957
				8508 334455
				8509 331450
				8510 331288
				8511 323460
				8512 327477
				8513 321939
				8514 321632
				8515 329932
				8516 333759
				8517 320346
				8518 326694
				8519 328300
				8520 337871
				8521 331853
				8522 320332
				8523 324257
				8524 327091
				8525 332228
				8526 329908
				8527 335726
				8528 331651
				8529 327060
				8530 327681
				8531 327798
				8532 328399
				8533 322264
				8534 331641
				8535 334305
				8536 341638
				8537 325224
				8538 328923
				8539 333949
				8540 327873
				8541 329382
				8542 329144
				8543 332944
				8544 328095
				8545 322914
				8546 333596
				8547 322802
				8548 323532
				8549 330836
				8550 340813
				8551 339284
				8552 327236
				8553 327608
				8554 332159
				8555 326671
				8556 331133
				8557 323016
				8558 324609
				8559 337526
				8560 323499
				8561 328994
				8562 328001
				8563 323946
				8564 332059
				8565 338419
				8566 328074
				8567 327505
				8568 328279
				8569 335096
				8570 333649
				8571 332268
				8572 328278
				8573 334222
				8574 330829
				8575 325027
				8576 330263
				8577 323749
				8578 327394
				8579 326423
				8580 339761
				8581 326158
				8582 326636
				8583 329558
				8584 325463
				8585 326774
				8586 329280
				8587 324134
				8588 328136
				8589 328159
				8590 331807
				8591 332474
				8592 331284
				8593 328587
				8594 327894
				8595 346761
				8596 329379
				8597 331011
				8598 326673
				8599 323484
				8600 328375
				8601 328139
				8602 323714
				8603 331419
				8604 324280
				8605 324701
				8606 327431
				8607 325474
				8608 332246
				8609 326644
				8610 328811
				8611 328765
				8612 333863
				8613 326724
				8614 335512
				8615 325969
				8616 329993
				8617 328713
				8618 322988
				8619 319658
				8620 333001
				8621 323383
				8622 321759
				8623 322650
				8624 328611
				8625 329571
				8626 329632
				8627 326730
				8628 330428
				8629 325102
				8630 340679
				8631 331083
				8632 327464
				8633 322641
				8634 319130
				8635 321623
				8636 319596
				8637 318606
				8638 326663
				8639 321815
				8640 339009
				8641 327810
				8642 321128
				8643 323078
				8644 317767
				8645 321753
				8646 322891
				8647 324483
				8648 328026
				8649 321558
				8650 322171
				8651 319827
				8652 321134
				8653 324018
				8654 318020
				8655 333047
				8656 323796
				8657 325657
				8658 330875
				8659 327247
				8660 318450
				8661 322929
				8662 322547
				8663 324278
				8664 328225
				8665 322250
				8666 324546
				8667 322848
				8668 328384
				8669 322488
				8670 336064
				8671 322748
				8672 330788
				8673 330753
				8674 325463
				8675 323069
				8676 329652
				8677 319252
				8678 319076
				8679 323485
				8680 326367
				8681 327863
				8682 321430
				8683 328308
				8684 325680
				8685 336148
				8686 333460
				8687 322515
				8688 322369
				8689 324772
				8690 323749
				8691 323740
				8692 322742
				8693 325503
				8694 321806
				8695 319605
				8696 322670
				8697 326814
				8698 327095
				8699 324009
				8700 338628
				8701 327497
				8702 321216
				8703 327488
				8704 320583
				8705 328202
				8706 326262
				8707 319024
				8708 321668
				8709 323053
				8710 325733
				8711 321799
				8712 325547
				8713 324056
				8714 321990
				8715 336596
				8716 326094
				8717 325647
				8718 324506
				8719 322484
				8720 321812
				8721 319830
				8722 324626
				8723 323060
				8724 331311
				8725 322733
				8726 324631
				8727 323411
				8728 321080
				8729 321953
				8730 337530
				8731 328511
				8732 324525
				8733 334138
				8734 321558
				8735 324903
				8736 320190
				8737 322483
				8738 326472
				8739 324997
				8740 317066
				8741 331060
				8742 323156
				8743 321948
				8744 322508
				8745 332407
				8746 338331
				8747 329144
				8748 323610
				8749 327364
				8750 326164
				8751 327943
				8752 322074
				8753 320230
				8754 320496
				8755 323044
				8756 317825
				8757 321906
				8758 319627
				8759 336878
				8760 334699
				8761 343262
				8762 321743
				8763 324649
				8764 321792
				8765 324044
				8766 317060
				8767 324832
				8768 322803
				8769 318010
				8770 324103
				8771 317651
				8772 322834
				8773 319285
				8774 322038
				8775 335992
				8776 330498
				8777 323265
				8778 321181
				8779 324684
				8780 324069
				8781 321067
				8782 329095
				8783 328616
				8784 317656
				8785 325455
				8786 317076
				8787 321681
				8788 327404
				8789 325892
				8790 327881
				8791 328664
				8792 320468
				8793 326353
				8794 320148
				8795 330788
				8796 329163
				8797 319348
				8798 326308
				8799 325467
				8800 323022
				8801 325435
				8802 324988
				8803 327244
				8804 325437
				8805 335643
				8806 331972
				8807 325987
				8808 326046
				8809 322682
				8810 324007
				8811 318122
				8812 317370
				8813 323357
				8814 317807
				8815 325129
				8816 332762
				8817 325303
				8818 328569
				8819 317812
				8820 331999
				8821 324990
				8822 331260
				8823 323124
				8824 324771
				8825 327299
				8826 326994
				8827 322376
				8828 326858
				8829 326804
				8830 324937
				8831 317242
				8832 321166
				8833 327353
				8834 328732
				8835 330219
				8836 328090
				8837 327484
				8838 320798
				8839 321254
				8840 321456
				8841 319992
				8842 325586
				8843 319125
				8844 328634
				8845 325772
				8846 321016
				8847 317743
				8848 321695
				8849 322816
				8850 328554
				8851 329655
				8852 323400
				8853 328208
				8854 325144
				8855 325324
				8856 324982
				8857 327717
				8858 321003
				8859 322822
				8860 325625
				8861 323114
				8862 323452
				8863 324036
				8864 313915
				8865 335804
				8866 331513
				8867 322701
				8868 330099
				8869 320748
				8870 324391
				8871 322369
				8872 332274
				8873 320514
				8874 320709
				8875 322393
				8876 320827
				8877 325253
				8878 321542
				8879 322241
				8880 339293
				8881 330467
				8882 331356
				8883 321949
				8884 325970
				8885 317223
				8886 321102
				8887 319894
				8888 316527
				8889 325593
				8890 331704
				8891 318317
				8892 325258
				8893 323384
				8894 320513
				8895 338912
				8896 330189
				8897 327381
				8898 319155
				8899 321424
				8900 335639
				8901 326292
				8902 325966
				8903 321791
				8904 328006
				8905 323657
				8906 324956
				8907 324678
				8908 330527
				8909 326051
				8910 344192
				8911 332740
				8912 324045
				8913 331285
				8914 322655
				8915 332541
				8916 324687
				8917 325131
				8918 340057
				8919 323092
				8920 323457
				8921 325027
				8922 320693
				8923 326217
				8924 318710
				8925 334730
				8926 331776
				8927 324916
				8928 329557
				8929 323552
				8930 324190
				8931 325581
				8932 322030
				8933 323760
				8934 326454
				8935 323846
				8936 323836
				8937 319096
				8938 324519
				8939 319933
				8940 336770
				8941 331276
				8942 326220
				8943 320962
				8944 329816
				8945 320101
				8946 328417
				8947 333154
				8948 324576
				8949 319880
				8950 330993
				8951 327581
				8952 330609
				8953 323359
				8954 330869
				8955 332666
				8956 338036
				8957 316019
				8958 330295
				8959 325266
				8960 329312
				8961 328295
				8962 324405
				8963 323001
				8964 332683
				8965 322576
				8966 326334
				8967 319916
				8968 319836
				8969 325657
				8970 332005
				8971 332855
				8972 324687
				8973 320144
				8974 332033
				8975 328656
				8976 329413
				8977 319954
				8978 327525
				8979 317574
				8980 326215
				8981 321824
				8982 327385
				8983 326259
				8984 325105
				8985 331898
				8986 330858
				8987 331686
				8988 323852
				8989 325455
				8990 329796
				8991 325026
				8992 324209
				8993 324717
				8994 325415
				8995 325797
				8996 327414
				8997 321886
				8998 326360
				8999 325139
				9000 336992
				9001 321308
				9002 326839
				9003 319214
				9004 327674
				9005 321514
				9006 321169
				9007 323062
				9008 326174
				9009 327197
				9010 326368
				9011 328024
				9012 320519
				9013 326062
				9014 317939
				9015 332015
				9016 334078
				9017 333006
				9018 329216
				9019 319474
				9020 331186
				9021 340446
				9022 331715
				9023 326669
				9024 320445
				9025 326275
				9026 324388
				9027 318899
				9028 323005
				9029 331558
				9030 337609
				9031 335280
				9032 329217
				9033 326881
				9034 327108
				9035 334137
				9036 322183
				9037 328516
				9038 321852
				9039 337019
				9040 327938
				9041 329214
				9042 323063
				9043 322350
				9044 325263
				9045 332292
				9046 335618
				9047 329362
				9048 327389
				9049 327544
				9050 326224
				9051 321987
				9052 321430
				9053 323420
				9054 331022
				9055 331555
				9056 322053
				9057 326910
				9058 328804
				9059 323995
				9060 326561
				9061 329411
				9062 324645
				9063 326176
				9064 324856
				9065 321924
				9066 330587
				9067 320285
				9068 325614
				9069 331502
				9070 323194
				9071 320796
				9072 321596
				9073 323990
				9074 319076
				9075 332378
				9076 331576
				9077 324282
				9078 329363
				9079 329847
				9080 329914
				9081 325491
				9082 325335
				9083 324755
				9084 322140
				9085 320909
				9086 338279
				9087 322518
				9088 325013
				9089 322938
				9090 341702
				9091 327614
				9092 332687
				9093 324670
				9094 330646
				9095 326837
				9096 332501
				9097 324880
				9098 320167
				9099 328439
				9100 324398
				9101 324988
				9102 323633
				9103 324273
				9104 327529
				9105 334876
				9106 336289
				9107 331520
				9108 332031
				9109 328435
				9110 325214
				9111 328484
				9112 332765
				9113 326467
				9114 333387
				9115 322927
				9116 327117
				9117 325880
				9118 330450
				9119 321591
				9120 339257
				9121 330153
				9122 328552
				9123 323542
				9124 328316
				9125 327819
				9126 326664
				9127 327483
				9128 336003
				9129 330766
				9130 323796
				9131 332971
				9132 326111
				9133 323531
				9134 331018
				9135 341255
				9136 332170
				9137 329115
				9138 323646
				9139 329754
				9140 326393
				9141 328268
				9142 325369
				9143 327777
				9144 323727
				9145 327860
				9146 318372
				9147 326686
				9148 324699
				9149 325724
				9150 339898
				9151 343827
				9152 333732
				9153 326684
				9154 330572
				9155 328701
				9156 326972
				9157 327477
				9158 324347
				9159 336966
				9160 323861
				9161 334698
				9162 328202
				9163 327972
				9164 321694
				9165 336347
				9166 331415
				9167 324629
				9168 325621
				9169 328477
				9170 327504
				9171 335558
				9172 334973
				9173 332997
				9174 331744
				9175 331035
				9176 324086
				9177 324250
				9178 326819
				9179 326232
				9180 336748
				9181 336068
				9182 328559
				9183 329618
				9184 329207
				9185 330559
				9186 334987
				9187 333713
				9188 334505
				9189 337724
				9190 328392
				9191 328456
				9192 333480
				9193 334441
				9194 326301
				9195 341854
				9196 336290
				9197 328399
				9198 326112
				9199 338220
				9200 333854
				9201 333831
				9202 325495
				9203 322591
				9204 331813
				9205 326071
				9206 329740
				9207 333089
				9208 339834
				9209 332267
				9210 345095
				9211 328355
				9212 334832
				9213 333337
				9214 329091
				9215 326222
				9216 327371
				9217 333260
				9218 328975
				9219 325223
				9220 330161
				9221 328518
				9222 324992
				9223 332741
				9224 328143
				9225 344453
				9226 332754
				9227 333112
				9228 331326
				9229 335073
				9230 328919
				9231 335354
				9232 333172
				9233 329608
				9234 327048
				9235 331315
				9236 329370
				9237 327503
				9238 325393
				9239 325507
				9240 342120
				9241 336186
				9242 336799
				9243 329207
				9244 332928
				9245 333195
				9246 332814
				9247 329139
				9248 333033
				9249 326942
				9250 331757
				9251 330449
				9252 329766
				9253 328243
				9254 331351
				9255 342572
				9256 331355
				9257 334432
				9258 330466
				9259 331386
				9260 335650
				9261 325488
				9262 329383
				9263 336768
				9264 331989
				9265 346596
				9266 328002
				9267 327148
				9268 328890
				9269 324918
				9270 346282
				9271 333335
				9272 336974
				9273 335691
				9274 337138
				9275 341260
				9276 330349
				9277 330984
				9278 332640
				9279 325873
				9280 332112
				9281 328236
				9282 329246
				9283 331953
				9284 337142
				9285 347761
				9286 339954
				9287 328005
				9288 330991
				9289 332291
				9290 326917
				9291 330805
				9292 333763
				9293 335486
				9294 340481
				9295 331722
				9296 327394
				9297 327513
				9298 330783
				9299 330503
				9300 339497
				9301 340207
				9302 335669
				9303 334110
				9304 339309
				9305 329556
				9306 330420
				9307 341671
				9308 331603
				9309 328095
				9310 328290
				9311 335773
				9312 337181
				9313 336402
				9314 326385
				9315 336504
				9316 338234
				9317 336716
				9318 329531
				9319 330955
				9320 325395
				9321 329746
				9322 339646
				9323 331036
				9324 332263
				9325 329048
				9326 326152
				9327 330515
				9328 332903
				9329 329705
				9330 343346
				9331 334326
				9332 335341
				9333 331311
				9334 333746
				9335 331137
				9336 331156
				9337 336880
				9338 328106
				9339 330874
				9340 331592
				9341 328551
				9342 329477
				9343 331503
				9344 333352
				9345 343179
				9346 332197
				9347 336232
				9348 331889
				9349 330490
				9350 331296
				9351 333838
				9352 334013
				9353 334945
				9354 331622
				9355 334773
				9356 333205
				9357 330699
				9358 327081
				9359 338898
				9360 342159
				9361 346997
				9362 336862
				9363 325585
				9364 329587
				9365 330933
				9366 328138
				9367 334846
				9368 329254
				9369 338148
				9370 330002
				9371 335461
				9372 333525
				9373 328869
				9374 334346
				9375 339651
				9376 338046
				9377 332978
				9378 336538
				9379 334940
				9380 338710
				9381 333933
				9382 330910
				9383 335587
				9384 329674
				9385 330649
				9386 334292
				9387 331656
				9388 338668
				9389 332266
				9390 359093
				9391 336706
				9392 333401
				9393 332596
				9394 329878
				9395 328094
				9396 334486
				9397 331989
				9398 331223
				9399 327584
				9400 337340
				9401 335950
				9402 339695
				9403 333603
				9404 332937
				9405 356709
				9406 335348
				9407 332657
				9408 332043
				9409 339536
				9410 338002
				9411 333290
				9412 328368
				9413 330961
				9414 335879
				9415 330552
				9416 333557
				9417 336284
				9418 329104
				9419 330718
				9420 349670
				9421 337813
				9422 334490
				9423 332475
				9424 330495
				9425 328912
				9426 343188
				9427 337640
				9428 344307
				9429 331821
				9430 345511
				9431 329246
				9432 342065
				9433 334288
				9434 330021
				9435 342784
				9436 346664
				9437 339180
				9438 339470
				9439 333224
				9440 334209
				9441 331108
				9442 334665
				9443 334862
				9444 335555
				9445 332387
				9446 336448
				9447 336393
				9448 337911
				9449 341015
				9450 346560
				9451 339313
				9452 334113
				9453 328013
				9454 338414
				9455 334326
				9456 338543
				9457 340569
				9458 328115
				9459 331783
				9460 337250
				9461 335508
				9462 335716
				9463 337268
				9464 328171
				9465 339547
				9466 340069
				9467 346566
				9468 337578
				9469 335014
				9470 330042
				9471 330391
				9472 326094
				9473 338820
				9474 329783
				9475 337796
				9476 333916
				9477 333346
				9478 331164
				9479 335507
				9480 343836
				9481 340840
				9482 335686
				9483 332057
				9484 339294
				9485 333485
				9486 342073
				9487 332710
				9488 330882
				9489 335031
				9490 336715
				9491 334078
				9492 327274
				9493 333209
				9494 337278
				9495 347509
				9496 341193
				9497 335495
				9498 334494
				9499 332989
				9500 334188
				9501 335217
				9502 340984
				9503 337005
				9504 331116
				9505 341939
				9506 335035
				9507 331501
				9508 328881
				9509 335028
				9510 349739
				9511 346584
				9512 335535
				9513 333856
				9514 329744
				9515 343282
				9516 339139
				9517 331954
				9518 334952
				9519 338890
				9520 335766
				9521 335900
				9522 337194
				9523 329798
				9524 329950
				9525 351094
				9526 333398
				9527 333215
				9528 335872
				9529 330655
				9530 333347
				9531 333151
				9532 332538
				9533 328128
				9534 347222
				9535 332423
				9536 332490
				9537 336835
				9538 334837
				9539 334697
				9540 343076
				9541 335025
				9542 334794
				9543 327612
				9544 334330
				9545 331299
				9546 338613
				9547 331011
				9548 329262
				9549 329238
				9550 332109
				9551 331111
				9552 333972
				9553 330402
				9554 332776
				9555 347745
				9556 338195
				9557 332315
				9558 338902
				9559 333883
				9560 332789
				9561 335681
				9562 342751
				9563 338687
				9564 329204
				9565 338345
				9566 332748
				9567 335389
				9568 332917
				9569 332862
				9570 339858
				9571 344750
				9572 333766
				9573 332277
				9574 336877
				9575 333204
				9576 333063
				9577 335737
				9578 330108
				9579 333441
				9580 336650
				9581 332044
				9582 331250
				9583 330628
				9584 329654
				9585 345884
				9586 338103
				9587 325235
				9588 338914
				9589 337627
				9590 332531
				9591 328345
				9592 334673
				9593 334810
				9594 330742
				9595 333352
				9596 335099
				9597 333660
				9598 331881
				9599 322955
				9600 342447
				9601 332879
				9602 353853
				9603 330135
				9604 332139
				9605 334073
				9606 333886
				9607 328848
				9608 325747
				9609 331600
				9610 343558
				9611 334664
				9612 336909
				9613 327465
				9614 332387
				9615 352777
				9616 346130
				9617 331582
				9618 334241
				9619 334230
				9620 333612
				9621 330005
				9622 328234
				9623 340185
				9624 336679
				9625 335284
				9626 340279
				9627 331792
				9628 334330
				9629 341854
				9630 348409
				9631 343098
				9632 341149
				9633 337466
				9634 334410
				9635 335720
				9636 329246
				9637 341959
				9638 331182
				9639 340807
				9640 331602
				9641 331598
				9642 340095
				9643 332425
				9644 334164
				9645 350723
				9646 341482
				9647 335413
				9648 336904
				9649 338443
				9650 335402
				9651 331104
				9652 332490
				9653 340317
				9654 331573
				9655 339874
				9656 332776
				9657 335951
				9658 337469
				9659 340832
				9660 346655
				9661 341571
				9662 333550
				9663 335979
				9664 330607
				9665 327986
				9666 330576
				9667 336072
				9668 332420
				9669 329049
				9670 334497
				9671 331282
				9672 334090
				9673 339976
				9674 332292
				9675 345428
				9676 343587
				9677 337198
				9678 344008
				9679 339747
				9680 329451
				9681 327678
				9682 330353
				9683 338342
				9684 334188
				9685 329741
				9686 331488
				9687 339377
				9688 331255
				9689 336161
				9690 347613
				9691 337446
				9692 334520
				9693 332905
				9694 340475
				9695 342931
				9696 339103
				9697 331247
				9698 336696
				9699 338996
				9700 341096
				9701 337329
				9702 336299
				9703 334412
				9704 339797
				9705 352872
				9706 344237
				9707 338562
				9708 346543
				9709 342071
				9710 334585
				9711 338691
				9712 338481
				9713 328561
				9714 331984
				9715 336628
				9716 352035
				9717 333584
				9718 334312
				9719 333669
				9720 342735
				9721 340471
				9722 338609
				9723 344934
				9724 336765
				9725 336310
				9726 333152
				9727 335632
				9728 335281
				9729 336494
				9730 329064
				9731 333067
				9732 340403
				9733 337124
				9734 337540
				9735 350349
				9736 337602
				9737 342834
				9738 337615
				9739 337057
				9740 340899
				9741 331609
				9742 331374
				9743 334546
				9744 344388
				9745 333809
				9746 339748
				9747 333900
				9748 340677
				9749 332725
				9750 348269
				9751 344262
				9752 335859
				9753 336002
				9754 344670
				9755 339923
				9756 337137
				9757 335490
				9758 337416
				9759 331069
				9760 329197
				9761 332651
				9762 337982
				9763 331251
				9764 353605
				9765 352749
				9766 344624
				9767 338737
				9768 333156
				9769 335946
				9770 336744
				9771 335084
				9772 330887
				9773 344038
				9774 337657
				9775 339574
				9776 338643
				9777 333501
				9778 345025
				9779 327803
				9780 348534
				9781 341344
				9782 338664
				9783 338151
				9784 332810
				9785 338298
				9786 337287
				9787 346373
				9788 336193
				9789 335584
				9790 331903
				9791 332724
				9792 332323
				9793 338496
				9794 336955
				9795 357040
				9796 346093
				9797 340822
				9798 333640
				9799 334669
				9800 331825
				9801 336354
				9802 333912
				9803 335791
				9804 334227
				9805 334938
				9806 345297
				9807 337944
				9808 337090
				9809 335063
				9810 352442
				9811 343785
				9812 337630
				9813 336376
				9814 341456
				9815 336046
				9816 339212
				9817 336321
				9818 336510
				9819 342961
				9820 333738
				9821 408359
				9822 371118
				9823 382568
				9824 394461
				9825 369133
				9826 345973
				9827 337442
				9828 341968
				9829 338072
				9830 338859
				9831 339844
				9832 335630
				9833 337188
				9834 336866
				9835 338713
				9836 333848
				9837 335903
				9838 343587
				9839 341496
				9840 350038
				9841 338156
				9842 336228
				9843 343855
				9844 336294
				9845 337026
				9846 341755
				9847 336926
				9848 342606
				9849 338179
				9850 340589
				9851 338002
				9852 341094
				9853 345447
				9854 333006
				9855 359098
				9856 351223
				9857 341794
				9858 337959
				9859 343965
				9860 336810
				9861 341089
				9862 342102
				9863 348440
				9864 342528
				9865 345748
				9866 347852
				9867 344049
				9868 345452
				9869 344268
				9870 352196
				9871 345921
				9872 348374
				9873 343034
				9874 357897
				9875 341710
				9876 343771
				9877 342776
				9878 353884
				9879 338549
				9880 341922
				9881 342567
				9882 387128
				9883 399942
				9884 384038
				9885 368897
				9886 352972
				9887 342674
				9888 341948
				9889 340574
				9890 341299
				9891 340903
				9892 343883
				9893 344352
				9894 355331
				9895 346603
				9896 340195
				9897 386534
				9898 381126
				9899 389129
				9900 354543
				9901 349408
				9902 349553
				9903 343418
				9904 348879
				9905 349377
				9906 339404
				9907 345268
				9908 348769
				9909 338240
				9910 341253
				9911 340520
				9912 338919
				9913 344413
				9914 342634
				9915 361514
				9916 347000
				9917 350315
				9918 351215
				9919 346595
				9920 340922
				9921 349662
				9922 344424
				9923 340752
				9924 350837
				9925 343525
				9926 342620
				9927 391615
				9928 387484
				9929 383135
				9930 361648
				9931 353026
				9932 359830
				9933 353853
				9934 339086
				9935 352962
				9936 342623
				9937 347639
				9938 345546
				9939 378153
				9940 380105
				9941 377785
				9942 380291
				9943 377206
				9944 401761
				9945 362978
				9946 360586
				9947 351736
				9948 345109
				9949 337541
				9950 342419
				9951 343851
				9952 349280
				9953 349706
				9954 337299
				9955 340441
				9956 361115
				9957 339924
				9958 348901
				9959 344787
				9960 360935
				9961 350995
				9962 350835
				9963 349694
				9964 341068
				9965 343394
				9966 346915
				9967 347980
				9968 342013
				9969 347075
				9970 352648
				9971 348395
				9972 341669
				9973 350102
				9974 356111
				9975 359095
				9976 351651
				9977 345185
				9978 349607
				9979 347517
				9980 350029
				9981 345496
				9982 392946
				9983 388310
				9984 387853
				9985 390106
				9986 378957
				9987 386314
				9988 385465
				9989 384875
				9990 359294
				9991 353535
				9992 352877
				9993 344946
				9994 344640
				9995 345412
				9996 340109
				9997 345625
				9998 351419
				9999 347112
				10000 348904
				10001 348680
				10002 353142
				10003 339836
				10004 362786
				10005 360872
				10006 358249
				10007 354428
				10008 358263
				10009 348127
				10010 344725
				10011 347870
				10012 354088
				10013 348403
				10014 350775
				10015 348991
				10016 350224
				10017 348952
				10018 346603
				10019 343872
				10020 366130
				10021 357327
				10022 349055
				10023 347548
				10024 359188
				10025 353755
				10026 347994
				10027 342794
				10028 390797
				10029 393494
				10030 394683
				10031 395555
				10032 382199
				10033 397244
				10034 391249
				10035 371834
				10036 353345
				10037 348862
				10038 349474
				10039 344034
				10040 354670
				10041 342661
				10042 346898
				10043 355037
				10044 344134
				10045 344459
				10046 345478
				10047 350938
				10048 352953
				10049 356596
				10050 362435
				10051 350181
				10052 348053
				10053 351382
				10054 354899
				10055 350478
				10056 348994
				10057 341097
				10058 339639
				10059 354248
				10060 375848
				10061 378875
				10062 385637
				10063 379321
				10064 377920
				10065 365334
				10066 366066
				10067 350550
				10068 344949
				10069 343859
				10070 347752
				10071 352122
				10072 350595
				10073 348309
				10074 348415
				10075 346572
				10076 345043
				10077 348198
				10078 351097
				10079 347222
				10080 365394
				10081 358719
				10082 349264
				10083 348566
				10084 344656
				10085 348570
				10086 339373
				10087 350192
				10088 341628
				10089 344740
				10090 360460
				10091 351643
				10092 343422
				10093 402324
				10094 392385
				10095 363135
				10096 343199
				10097 345798
				10098 352078
				10099 348235
				10100 406183
				10101 393187
				10102 396764
				10103 387729
				10104 384287
				10105 397422
				10106 384825
				10107 396687
				10108 385392
				10109 389098
				10110 366725
				10111 355778
				10112 356071
				10113 348478
				10114 353076
				10115 346784
				10116 354336
				10117 350398
				10118 354419
				10119 354324
				10120 350767
				10121 352496
				10122 355757
				10123 351944
				10124 347577
				10125 362387
				10126 364492
				10127 355175
				10128 352070
				10129 357941
				10130 358106
				10131 357614
				10132 351880
				10133 345320
				10134 356620
				10135 356464
				10136 351035
				10137 358368
				10138 350402
				10139 362956
				10140 367504
				10141 356965
				10142 356678
				10143 358284
				10144 348554
				10145 357011
				10146 358087
				10147 403108
				10148 400313
				10149 405096
				10150 404709
				10151 399594
				10152 398512
				10153 397585
				10154 401522
				10155 367572
				10156 362953
				10157 357566
				10158 360112
				10159 358154
				10160 357200
				10161 362550
				10162 357555
				10163 356365
				10164 351664
				10165 366976
				10166 364264
				10167 356652
				10168 354786
				10169 363085
				10170 370405
				10171 361475
				10172 357878
				10173 353630
				10174 355204
				10175 352457
				10176 358804
				10177 360855
				10178 351633
				10179 364332
				10180 359749
				10181 357767
				10182 353929
				10183 349896
				10184 356106
				10185 371776
				10186 365804
				10187 367802
				10188 371792
				10189 367478
				10190 372865
				10191 368837
				10192 365229
				10193 376185
				10194 371313
				10195 372330
				10196 369554
				10197 373933
				10198 365753
				10199 368946
				10200 372268
				10201 357047
				10202 368928
				10203 362279
				10204 359462
				10205 360647
				10206 364437
				10207 354948
				10208 349797
				10209 355372
				10210 360600
				10211 356545
				10212 354983
				10213 359804
				10214 360627
				10215 369623
				10216 361448
				10217 363631
				10218 356418
				10219 357097
				10220 359023
				10221 359179
				10222 361320
				10223 367996
				10224 357216
				10225 356679
				10226 358096
				10227 362277
				10228 353211
				10229 362168
				10230 371465
				10231 363912
				10232 365368
				10233 369763
				10234 359032
				10235 354539
				10236 359182
				10237 359386
				10238 370141
				10239 376577
				10240 371942
				10241 383749
				10242 373690
				10243 376820
				10244 374225
				10245 369164
				10246 425702
				10247 410601
				10248 412103
				10249 404241
				10250 401948
				10251 398165
				10252 420352
				10253 410095
				10254 420018
				10255 417649
				10256 420773
				10257 414031
				10258 411927
				10259 422563
				10260 368554
				10261 364400
				10262 405246
				10263 403393
				10264 402954
				10265 404608
				10266 414148
				10267 405549
				10268 401562
				10269 410738
				10270 414946
				10271 414572
				10272 418837
				10273 405742
				10274 413005
				10275 378471
				10276 371828
				10277 372394
				10278 363853
				10279 361634
				10280 363168
				10281 361562
				10282 379072
				10283 379134
				10284 372582
				10285 370073
				10286 361173
				10287 368267
				10288 378702
				10289 376983
				10290 376088
				10291 371259
				10292 368962
				10293 372042
				10294 368515
				10295 362675
				10296 358858
				10297 373547
				10298 361431
				10299 360240
				10300 367922
				10301 368938
				10302 410519
				10303 402112
				10304 396256
				10305 370501
				10306 374579
				10307 364797
				10308 370541
				10309 365379
				10310 363358
				10311 360038
				10312 352281
				10313 362551
				10314 367535
				10315 365090
				10316 365504
				10317 431626
				10318 404416
				10319 399135
				10320 368222
				10321 378366
				10322 367524
				10323 364839
				10324 364766
				10325 370721
				10326 363321
				10327 366159
				10328 370567
				10329 366216
				10330 386011
				10331 375004
				10332 369231
				10333 375514
				10334 366823
				10335 376244
				10336 372690
				10337 365091
				10338 362438
				10339 371000
				10340 368115
				10341 363037
				10342 379434
				10343 366159
				10344 360415
				10345 416786
				10346 402483
				10347 415957
				10348 400189
				10349 402505
				10350 368647
				10351 376679
				10352 359311
				10353 361026
				10354 362511
				10355 356659
				10356 360069
				10357 369340
				10358 362724
				10359 355935
				10360 369252
				10361 369113
				10362 373958
				10363 372847
				10364 377339
				10365 373443
				10366 375074
				10367 366705
				10368 364357
				10369 370512
				10370 363750
				10371 362320
				10372 408867
				10373 406368
				10374 409848
				10375 399056
				10376 418123
				10377 406080
				10378 415299
				10379 407479
				10380 379658
				10381 370879
				10382 369777
				10383 369487
				10384 369042
				10385 362264
				10386 365955
				10387 362613
				10388 363552
				10389 367106
				10390 366286
				10391 358983
				10392 423288
				10393 420488
				10394 410571
				10395 383440
				10396 376091
				10397 373703
				10398 373577
				10399 377621
				10400 374733
				10401 362977
				10402 411757
				10403 414603
				10404 400250
				10405 409122
				10406 399329
				10407 395604
				10408 403772
				10409 416381
				10410 382770
				10411 372750
				10412 369817
				10413 360887
				10414 373854
				10415 381420
				10416 377518
				10417 378292
				10418 382719
				10419 376942
				10420 383056
				10421 373849
				10422 380628
				10423 372197
				10424 375923
				10425 381696
				10426 371091
				10427 375393
				10428 365393
				10429 385501
				10430 373626
				10431 412111
				10432 404835
				10433 404078
				10434 407779
				10435 424473
				10436 407441
				10437 409134
				10438 402318
				10439 402383
				10440 372242
				10441 377077
				10442 373014
				10443 372754
				10444 387895
				10445 387890
				10446 404160
				10447 394696
				10448 424912
				10449 417965
				10450 427010
				10451 417426
				10452 414691
				10453 433573
				10454 417899
				10455 397320
				10456 385475
				10457 367826
				10458 372218
				10459 371202
				10460 372150
				10461 373340
				10462 373011
				10463 377046
				10464 364829
				10465 378886
				10466 371005
				10467 366868
				10468 365629
				10469 372152
				10470 386994
				10471 427120
				10472 407922
				10473 427186
				10474 415040
				10475 416887
				10476 409487
				10477 406980
				10478 416691
				10479 412779
				10480 405661
				10481 414911
				10482 427102
				10483 405083
				10484 423298
				10485 389801
				10486 375532
				10487 377803
				10488 373034
				10489 368830
				10490 367170
				10491 373345
				10492 381252
				10493 370394
				10494 368264
				10495 371710
				10496 369122
				10497 373063
				10498 370340
				10499 372310
				10500 393413
				10501 379115
				10502 383105
				10503 372704
				10504 364995
				10505 365184
				10506 373312
				10507 371112
				10508 365879
				10509 374830
				10510 379083
				10511 366038
				10512 368106
				10513 368996
				10514 365125
				10515 376633
				10516 383690
				10517 373181
				10518 368547
				10519 424796
				10520 421785
				10521 411762
				10522 410658
				10523 417459
				10524 415410
				10525 413411
				10526 418107
				10527 425717
				10528 420507
				10529 410416
				10530 384754
				10531 374001
				10532 377857
				10533 375829
				10534 368078
				10535 384278
				10536 371457
				10537 366539
				10538 371616
				10539 366286
				10540 368462
				10541 368712
				10542 369922
				10543 367728
				10544 370619
				10545 382658
				10546 379593
				10547 375353
				10548 366414
				10549 383197
				10550 374072
				10551 380005
				10552 369369
				10553 370580
				10554 368905
				10555 368541
				10556 369799
				10557 378180
				10558 374719
				10559 378451
				10560 390345
				10561 419028
				10562 411805
				10563 406299
				10564 409241
				10565 404137
				10566 403671
				10567 416665
				10568 417400
				10569 423881
				10570 430443
				10571 434949
				10572 421490
				10573 429367
				10574 430244
				10575 378919
				10576 386741
				10577 378729
				10578 368101
				10579 364833
				10580 374469
				10581 375712
				10582 377008
				10583 375725
				10584 368495
				10585 372360
				10586 372211
				10587 369190
				10588 383170
				10589 381352
				10590 383321
				10591 378157
				10592 379884
				10593 385400
				10594 365435
				10595 373107
				10596 371141
				10597 363086
				10598 371431
				10599 369763
				10600 378277
				10601 377449
				10602 362498
				10603 381852
				10604 370463
				10605 388088
				10606 376404
				10607 374086
				10608 377493
				10609 370162
				10610 397734
				10611 386543
				10612 384689
				10613 384795
				10614 383835
				10615 415968
				10616 422374
				10617 413385
				10618 421472
				10619 399911
				10620 387476
				10621 381874
				10622 378894
				10623 368315
				10624 378479
				10625 375435
				10626 368845
				10627 370763
				10628 374742
				10629 375459
				10630 377766
				10631 369572
				10632 373099
				10633 368175
				10634 373249
				10635 383676
				10636 384426
				10637 439906
				10638 421502
				10639 423052
				10640 420550
				10641 416818
				10642 412246
				10643 414053
				10644 428039
				10645 434163
				10646 434977
				10647 422372
				10648 427355
				10649 432387
				10650 383453
				10651 386777
				10652 372867
				10653 417795
				10654 415923
				10655 417920
				10656 409250
				10657 409079
				10658 413031
				10659 419178
				10660 408945
				10661 417076
				10662 412037
				10663 407134
				10664 421521
				10665 385958
				10666 377974
				10667 377635
				10668 387054
				10669 386772
				10670 386729
				10671 392088
				10672 391774
				10673 381189
				10674 382346
				10675 384503
				10676 380012
				10677 422891
				10678 412926
				10679 409919
				10680 391500
				10681 385190
				10682 389193
				10683 378275
				10684 365035
				10685 372825
				10686 371114
				10687 372789
				10688 377254
				10689 379591
				10690 371272
				10691 376940
				10692 383089
				10693 366782
				10694 372280
				10695 385335
				10696 375528
				10697 368471
				10698 371907
				10699 369153
				10700 428994
				10701 430288
				10702 440146
				10703 429880
				10704 436622
				10705 430998
				10706 424130
				10707 437808
				10708 426076
				10709 431991
				10710 385438
				10711 386037
				10712 382066
				10713 376241
				10714 370516
				10715 379204
				10716 372220
				10717 369472
				10718 375978
				10719 414849
				10720 410023
				10721 428555
				10722 425188
				10723 425992
				10724 419046
				10725 386910
				10726 394148
				10727 427554
				10728 418390
				10729 440417
				10730 426562
				10731 435566
				10732 431409
				10733 423515
				10734 438231
				10735 439821
				10736 429370
				10737 432302
				10738 431614
				10739 425873
				10740 383697
				10741 375155
				10742 380762
				10743 414242
				10744 406839
				10745 410718
				10746 423212
				10747 415463
				10748 408342
				10749 411432
				10750 415521
				10751 400244
				10752 418196
				10753 415695
				10754 420320
				10755 385976
				10756 380343
				10757 373961
				10758 377660
				10759 380258
				10760 389715
				10761 375499
				10762 387654
				10763 383930
				10764 399702
				10765 413989
				10766 414694
				10767 426679
				10768 422291
				10769 415643
				10770 394942
				10771 386033
				10772 379737
				10773 377442
				10774 385625
				10775 376196
				10776 378067
				10777 375248
				10778 370243
				10779 371691
				10780 380718
				10781 362461
				10782 370256
				10783 383902
				10784 389785
				10785 386335
				10786 387340
				10787 377967
				10788 381047
				10789 385920
				10790 381260
				10791 373946
				10792 369203
				10793 376261
				10794 375377
				10795 386004
				10796 379519
				10797 378690
				10798 425228
				10799 409271
				10800 384268
				10801 384743
				10802 375780
				10803 377857
				10804 379921
				10805 376222
				10806 385081
				10807 375378
				10808 377421
				10809 413384
				10810 408498
				10811 414599
				10812 409153
				10813 410147
				10814 419137
				10815 396902
				10816 381279
				10817 379929
				10818 378874
				10819 370941
				10820 382043
				10821 373605
				10822 375878
				10823 380907
				10824 381779
				10825 374963
				10826 384282
				10827 389631
				10828 366676
				10829 381952
				10830 384079
				10831 385031
				10832 376778
				10833 377483
				10834 374622
				10835 375852
				10836 419817
				10837 421396
				10838 409466
				10839 416738
				10840 419704
				10841 426394
				10842 440233
				10843 431904
				10844 435971
				10845 382161
				10846 382037
				10847 389348
				10848 374719
				10849 374973
				10850 390748
				10851 380160
				10852 382430
				10853 369351
				10854 375265
				10855 376137
				10856 376355
				10857 373725
				10858 372054
				10859 391305
				10860 388032
				10861 378271
				10862 385326
				10863 370359
				10864 378718
				10865 376573
				10866 393051
				10867 378430
				10868 376949
				10869 378032
				10870 392331
				10871 393194
				10872 423667
				10873 431649
				10874 437032
				10875 386260
				10876 381897
				10877 373629
				10878 367759
				10879 386541
				10880 378892
				10881 378971
				10882 374120
				10883 366105
				10884 378257
				10885 385286
				10886 380276
				10887 380339
				10888 378457
				10889 372147
				10890 396987
				10891 377655
				10892 377962
				10893 378211
				10894 384093
				10895 379395
				10896 377257
				10897 397924
				10898 398245
				10899 395883
				10900 386378
				10901 429013
				10902 422477
				10903 416248
				10904 420478
				10905 395156
				10906 390842
				10907 381799
				10908 381488
				10909 381432
				10910 378950
				10911 387240
				10912 380868
				10913 385186
				10914 389772
				10915 385191
				10916 382282
				10917 383407
				10918 381920
				10919 385415
				10920 391240
				10921 388277
				10922 384129
				10923 382185
				10924 389131
				10925 381410
				10926 384222
				10927 380739
				10928 421651
				10929 419286
				10930 414124
				10931 439011
				10932 422358
				10933 420816
				10934 430220
				10935 400615
				10936 389523
				10937 391514
				10938 381432
				10939 393324
				10940 404941
				10941 400080
				10942 397059
				10943 383071
				10944 382487
				10945 394258
				10946 408529
				10947 388089
				10948 390012
				10949 393839
				10950 391903
				10951 391818
				10952 370641
				10953 381132
				10954 394931
				10955 385156
				10956 384814
				10957 382026
				10958 380273
				10959 382757
				10960 390246
				10961 385976
				10962 375270
				10963 384646
				10964 387724
				10965 399382
				10966 402190
				10967 392827
				10968 397480
				10969 404730
				10970 393019
				10971 386595
				10972 390804
				10973 402244
				10974 397476
				10975 390898
				10976 426350
				10977 424601
				10978 420565
				10979 425558
				10980 392810
				10981 390823
				10982 386881
				10983 386260
				10984 396301
				10985 393220
				10986 386038
				10987 386026
				10988 391038
				10989 394064
				10990 390144
				10991 381176
				10992 392549
				10993 388163
				10994 391937
				10995 391345
				10996 405156
				10997 376743
				10998 368776
				10999 384498
				11000 378948
				11001 384059
				11002 385728
				11003 402524
				11004 384530
				11005 395379
				11006 396937
				11007 385760
				11008 406128
				11009 391526
				11010 400423
				11011 391460
				11012 392988
				11013 383321
				11014 422006
				11015 416691
				11016 417017
				11017 411269
				11018 415932
				11019 413036
				11020 417546
				11021 417069
				11022 411962
				11023 418914
				11024 417247
				11025 393363
				11026 433521
				11027 424526
				11028 426019
				11029 424476
				11030 414526
				11031 426743
				11032 427246
				11033 426954
				11034 430995
				11035 428272
				11036 439707
				11037 434862
				11038 432870
				11039 440124
				11040 383894
				11041 395884
				11042 377985
				11043 382110
				11044 386299
				11045 384795
				11046 376639
				11047 420649
				11048 418664
				11049 414479
				11050 414121
				11051 438190
				11052 435423
				11053 426034
				11054 426985
				11055 396938
				11056 380416
				11057 386175
				11058 377146
				11059 377653
				11060 381127
				11061 388693
				11062 382017
				11063 388065
				11064 389713
				11065 397399
				11066 394350
				11067 416700
				11068 402036
				11069 398878
				11070 387754
				11071 442103
				11072 437621
				11073 432690
				11074 431114
				11075 439354
				11076 437891
				11077 424491
				11078 414403
				11079 425401
				11080 427095
				11081 426962
				11082 417312
				11083 429413
				11084 440750
				11085 396016
				11086 400795
				11087 383192
				11088 381175
				11089 392455
				11090 381605
				11091 375405
				11092 389320
				11093 383965
				11094 378943
				11095 380163
				11096 433205
				11097 419394
				11098 424580
				11099 432826
				11100 392376
				11101 389144
				11102 406107
				11103 381569
				11104 382417
				11105 377163
				11106 388178
				11107 374807
				11108 390809
				11109 390007
				11110 397007
				11111 385744
				11112 391780
				11113 401199
				11114 393750
				11115 393996
				11116 395209
				11117 387634
				11118 390398
				11119 384457
				11120 381583
				11121 388247
				11122 380643
				11123 373513
				11124 382731
				11125 377041
				11126 384722
				11127 381145
				11128 385901
				11129 383675
				11130 402489
				11131 386538
				11132 381340
				11133 386348
				11134 388277
				11135 385949
				11136 390705
				11137 384856
				11138 379563
				11139 380950
				11140 394594
				11141 384083
				11142 408773
				11143 397244
				11144 395402
				11145 396231
				11146 391540
				11147 381430
				11148 401192
				11149 398197
				11150 397142
				11151 400113
				11152 390721
				11153 396669
				11154 393018
				11155 391952
				11156 421359
				11157 421240
				11158 410599
				11159 413549
				11160 401630
				11161 395134
				11162 396155
				11163 390784
				11164 388966
				11165 392983
				11166 382183
				11167 420352
				11168 402442
				11169 402616
				11170 393646
				11171 414303
				11172 413746
				11173 413946
				11174 405043
				11175 401981
				11176 391491
				11177 397129
				11178 384377
				11179 386120
				11180 392603
				11181 384152
				11182 390113
				11183 395734
				11184 394895
				11185 388426
				11186 395801
				11187 384133
				11188 405320
				11189 392512
				11190 399745
				11191 436408
				11192 423810
				11193 420176
				11194 426433
				11195 425721
				11196 418259
				11197 440606
				11198 421384
				11199 417869
				11200 424230
				11201 443123
				11202 431208
				11203 440198
				11204 441438
				11205 401450
				11206 393155
				11207 397011
				11208 394378
				11209 391930
				11210 387748
				11211 393211
				11212 387351
				11213 393447
				11214 405219
				11215 403071
				11216 408615
				11217 399454
				11218 404029
				11219 407295
				11220 410872
				11221 412252
				11222 413121
				11223 404378
				11224 410122
				11225 407751
				11226 403108
				11227 410471
				11228 399677
				11229 400882
				11230 397508
				11231 402008
				11232 402683
				11233 401284
				11234 439188
				11235 414985
				11236 414193
				11237 393029
				11238 397789
				11239 394685
				11240 394198
				11241 449291
				11242 437893
				11243 447468
				11244 445054
				11245 437248
				11246 440542
				11247 444133
				11248 429288
				11249 440077
				11250 410955
				11251 412870
				11252 390433
				11253 400395
				11254 402533
				11255 457637
				11256 445922
				11257 436081
				11258 444522
				11259 434681
				11260 458540
				11261 445324
				11262 441188
				11263 445515
				11264 441045
				11265 401364
				11266 411189
				11267 392054
				11268 394704
				11269 396958
				11270 394428
				11271 389658
				11272 394061
				11273 394760
				11274 397480
				11275 393595
				11276 404922
				11277 384211
				11278 397226
				11279 389135
				11280 411987
				11281 400522
				11282 414791
				11283 413610
				11284 410912
				11285 398858
				11286 400562
				11287 414197
				11288 404021
				11289 393717
				11290 399790
				11291 392166
				11292 404116
				11293 406027
				11294 407436
				11295 417214
				11296 418827
				11297 414841
				11298 411530
				11299 406170
				11300 403427
				11301 404483
				11302 409949
				11303 401891
				11304 407745
				11305 410585
				11306 413295
				11307 411016
				11308 416014
				11309 408959
				11310 401786
				11311 400818
				11312 398295
				11313 402536
				11314 406425
				11315 408863
				11316 402975
				11317 396070
				11318 410444
				11319 406263
				11320 406958
				11321 407560
				11322 404660
				11323 448290
				11324 438902
				11325 411794
				11326 403742
				11327 404119
				11328 392876
				11329 395934
				11330 399010
				11331 387523
				11332 418942
				11333 394992
				11334 395680
				11335 410274
				11336 437656
				11337 440176
				11338 433977
				11339 450215
				11340 409875
				11341 412064
				11342 394210
				11343 388218
				11344 454893
				11345 448070
				11346 441767
				11347 450864
				11348 448766
				11349 450548
				11350 453369
				11351 442408
				11352 452599
				11353 443735
				11354 450184
				11355 404403
				11356 411724
				11357 399760
				11358 406621
				11359 415165
				11360 401236
				11361 411886
				11362 399276
				11363 403628
				11364 414007
				11365 411769
				11366 410377
				11367 414735
				11368 408048
				11369 408252
				11370 408520
				11371 406967
				11372 395556
				11373 404304
				11374 406603
				11375 401804
				11376 400938
				11377 392616
				11378 398137
				11379 410166
				11380 421546
				11381 417272
				11382 402750
				11383 408704
				11384 427648
				11385 418904
				11386 395687
				11387 400504
				11388 395577
				11389 403447
				11390 395181
				11391 395663
				11392 394393
				11393 394196
				11394 445793
				11395 435878
				11396 426943
				11397 423431
				11398 433743
				11399 431172
				11400 403169
				11401 396681
				11402 388595
				11403 386987
				11404 396168
				11405 407663
				11406 389625
				11407 400530
				11408 454013
				11409 446318
				11410 447786
				11411 432993
				11412 440369
				11413 438690
				11414 435890
				11415 413015
				11416 411504
				11417 396872
				11418 393849
				11419 399501
				11420 398754
				11421 420225
				11422 402182
				11423 404070
				11424 407519
				11425 413331
				11426 440808
				11427 429554
				11428 444482
				11429 439489
				11430 413328
				11431 406330
				11432 397877
				11433 401989
				11434 396599
				11435 395524
				11436 446755
				11437 442332
				11438 433324
				11439 437717
				11440 445080
				11441 455745
				11442 449576
				11443 450756
				11444 458292
				11445 403938
				11446 417705
				11447 397190
				11448 402441
				11449 406289
				11450 409921
				11451 405370
				11452 448654
				11453 458200
				11454 453197
				11455 452060
				11456 467979
				11457 457486
				11458 463561
				11459 449076
				11460 416040
				11461 396264
				11462 402955
				11463 392408
				11464 392235
				11465 405830
				11466 397015
				11467 446698
				11468 443755
				11469 440016
				11470 433453
				11471 441172
				11472 444598
				11473 455289
				11474 450512
				11475 410050
				11476 414022
				11477 410070
				11478 436515
				11479 433618
				11480 433216
				11481 439967
				11482 437009
				11483 430659
				11484 436120
				11485 437206
				11486 456111
				11487 444049
				11488 435952
				11489 446965
				11490 421359
				11491 399881
				11492 409167
				11493 405223
				11494 391602
				11495 404045
				11496 397381
				11497 399177
				11498 396157
				11499 406583
				11500 404370
				11501 401072
				11502 457151
				11503 434729
				11504 440179
				11505 411020
				11506 413290
				11507 422003
				11508 414379
				11509 407084
				11510 457183
				11511 449101
				11512 454918
				11513 450528
				11514 463863
				11515 463522
				11516 450877
				11517 437268
				11518 461110
				11519 453799
				11520 413048
				11521 396043
				11522 412426
				11523 410066
				11524 419040
				11525 412063
				11526 407800
				11527 408738
				11528 399686
				11529 403088
				11530 402012
				11531 414788
				11532 406771
				11533 410272
				11534 403039
				11535 420200
				11536 409409
				11537 437766
				11538 430219
				11539 443075
				11540 442327
				11541 444756
				11542 436545
				11543 439019
				11544 442749
				11545 439387
				11546 430700
				11547 435554
				11548 440221
				11549 443526
				11550 406119
				11551 397529
				11552 396090
				11553 416357
				11554 405722
				11555 399995
				11556 405534
				11557 407905
				11558 410434
				11559 406134
				11560 415574
				11561 411159
				11562 399215
				11563 406734
				11564 395860
				11565 421719
				11566 456202
				11567 444106
				11568 445581
				11569 453839
				11570 451544
				11571 459708
				11572 451423
				11573 458693
				11574 453543
				11575 440116
				11576 441066
				11577 456683
				11578 451127
				11579 444667
				11580 413899
				11581 452907
				11582 459721
				11583 460282
				11584 453324
				11585 456540
				11586 459135
				11587 458051
				11588 442829
				11589 475963
				11590 455200
				11591 457194
				11592 456970
				11593 459449
				11594 450785
				11595 416988
				11596 454400
				11597 457775
				11598 468739
				11599 461904
				11600 459187
				11601 451714
				11602 463441
				11603 457330
				11604 469619
				11605 479803
				11606 484114
				11607 483909
				11608 476320
				11609 485692
				11610 426135
				11611 413695
				11612 410993
				11613 407261
				11614 413054
				11615 407611
				11616 418482
				11617 424091
				11618 416112
				11619 420475
				11620 412466
				11621 406559
				11622 414589
				11623 418421
				11624 414643
				11625 432696
				11626 416626
				11627 422383
				11628 409848
				11629 413358
				11630 414475
				11631 409053
				11632 435373
				11633 412845
				11634 421201
				11635 424490
				11636 419165
				11637 425951
				11638 410488
				11639 409507
				11640 418667
				11641 415326
				11642 421471
				11643 425723
				11644 434431
				11645 417082
				11646 426515
				11647 413235
				11648 426903
				11649 435412
				11650 449380
				11651 454978
				11652 468768
				11653 477102
				11654 488824
				11655 427857
				11656 466344
				11657 453562
				11658 442994
				11659 469925
				11660 463872
				11661 460238
				11662 459580
				11663 454924
				11664 454027
				11665 456790
				11666 449770
				11667 466827
				11668 448669
				11669 459265
				11670 430816
				11671 426794
				11672 416217
				11673 416152
				11674 462615
				11675 462536
				11676 469954
				11677 467569
				11678 466415
				11679 467986
				11680 472700
				11681 471014
				11682 459899
				11683 474741
				11684 471741
				11685 428858
				11686 447633
				11687 423110
				11688 427858
				11689 439116
				11690 428296
				11691 432189
				11692 432242
				11693 426699
				11694 441967
				11695 417909
				11696 419891
				11697 453167
				11698 455600
				11699 466093
				11700 433202
				11701 436502
				11702 438128
				11703 429489
				11704 432290
				11705 427953
				11706 436560
				11707 423294
				11708 427234
				11709 420080
				11710 435126
				11711 440165
				11712 413947
				11713 432517
				11714 427650
				11715 437157
				11716 430439
				11717 423664
				11718 416974
				11719 423215
				11720 466286
				11721 452004
				11722 455090
				11723 454262
				11724 456967
				11725 459077
				11726 453217
				11727 459128
				11728 454628
				11729 474895
				11730 426515
				11731 430273
				11732 418254
				11733 412525
				11734 424247
				11735 427418
				11736 431082
				11737 418950
				11738 426358
				11739 456422
				11740 466044
				11741 477281
				11742 474949
				11743 487834
				11744 472696
				11745 413722
				11746 421937
				11747 426861
				11748 416251
				11749 417474
				11750 415409
				11751 423573
				11752 419946
				11753 421950
				11754 418207
				11755 419435
				11756 409918
				11757 414574
				11758 437390
				11759 428444
				11760 428918
				11761 422676
				11762 468400
				11763 468398
				11764 472598
				11765 467657
				11766 476629
				11767 456999
				11768 464329
				11769 472147
				11770 467777
				11771 470155
				11772 473477
				11773 465276
				11774 474246
				11775 434303
				11776 427593
				11777 418115
				11778 419667
				11779 420892
				11780 418534
				11781 434886
				11782 441626
				11783 423200
				11784 431148
				11785 413088
				11786 425838
				11787 427044
				11788 432187
				11789 421420
				11790 435333
				11791 426072
				11792 420032
				11793 427169
				11794 433148
				11795 425502
				11796 421955
				11797 434742
				11798 429654
				11799 458245
				11800 444008
				11801 445931
				11802 442166
				11803 468345
				11804 453922
				11805 438271
				11806 425942
				11807 414052
				11808 403277
				11809 428231
				11810 416780
				11811 421578
				11812 410720
				11813 421062
				11814 419438
				11815 420694
				11816 408217
				11817 401767
				11818 478545
				11819 470336
				11820 438063
				11821 420423
				11822 423603
				11823 423341
				11824 411710
				11825 423155
				11826 412894
				11827 404022
				11828 412212
				11829 409431
				11830 460520
				11831 458628
				11832 473818
				11833 460523
				11834 478591
				11835 443889
				11836 472912
				11837 458666
				11838 474517
				11839 465534
				11840 460530
				11841 461663
				11842 466118
				11843 479176
				11844 463780
				11845 478919
				11846 468576
				11847 459496
				11848 457808
				11849 472427
				11850 436786
				11851 424915
				11852 439624
				11853 420867
				11854 421973
				11855 413802
				11856 467199
				11857 449119
				11858 471541
				11859 468367
				11860 453082
				11861 467191
				11862 470554
				11863 454623
				11864 458912
				11865 432721
				11866 433558
				11867 459348
				11868 451389
				11869 451768
				11870 457507
				11871 450218
				11872 451758
				11873 455813
				11874 457469
				11875 451985
				11876 446863
				11877 473015
				11878 495985
				11879 496467
				11880 445071
				11881 432976
				11882 415977
				11883 473627
				11884 471426
				11885 471618
				11886 461268
				11887 458903
				11888 451581
				11889 456485
				11890 455417
				11891 454791
				11892 473403
				11893 446289
				11894 453342
				11895 439319
				11896 450149
				11897 445260
				11898 431037
				11899 431488
				11900 426533
				11901 464133
				11902 461837
				11903 446171
				11904 461897
				11905 460676
				11906 474705
				11907 464522
				11908 479443
				11909 495443
				11910 435410
				11911 489672
				11912 470365
				11913 465712
				11914 456396
				11915 442140
				11916 455631
				11917 467223
				11918 480797
				11919 485030
				11920 461572
				11921 478625
				11922 469115
				11923 491789
				11924 481676
				11925 435637
				11926 419932
				11927 409986
				11928 437360
				11929 427653
				11930 433205
				11931 424614
				11932 428027
				11933 425202
				11934 435345
				11935 428939
				11936 428161
				11937 427611
				11938 436972
				11939 440473
				11940 431428
				11941 430043
				11942 430408
				11943 419159
				11944 453344
				11945 458932
				11946 458682
				11947 460511
				11948 456985
				11949 456914
				11950 459265
				11951 452368
				11952 463664
				11953 464433
				11954 465248
				11955 425406
				11956 419367
				11957 408503
				11958 458347
				11959 449825
				11960 446710
				11961 444016
				11962 442692
				11963 452945
				11964 451772
				11965 445880
				11966 442486
				11967 448942
				11968 442378
				11969 469828
				11970 420845
				11971 430902
				11972 410243
				11973 415381
				11974 417683
				11975 423112
				11976 417730
				11977 415517
				11978 419240
				11979 429942
				11980 425086
				11981 459650
				11982 448897
				11983 462619
				11984 458658
				11985 425929
				11986 411340
				11987 403952
				11988 428777
				11989 413174
				11990 418980
				11991 417487
				11992 415558
				11993 447050
				11994 466507
				11995 458084
				11996 457715
				11997 453816
				11998 456234
				11999 449499
				12000 415885
				12001 408838
				12002 399866
				12003 404593
				12004 409248
				12005 408174
				12006 410866
				12007 415783
				12008 423047
				12009 451077
				12010 434193
				12011 446213
				12012 451268
				12013 437218
				12014 441146
				12015 413087
				12016 415417
				12017 400678
				12018 412923
				12019 406684
				12020 402864
				12021 395087
				12022 435794
				12023 431129
				12024 427035
				12025 429959
				12026 432219
				12027 429769
				12028 422878
				12029 425953
				12030 418988
				12031 409306
				12032 407232
				12033 420386
				12034 418818
				12035 420271
				12036 415474
				12037 418152
				12038 416380
				12039 420217
				12040 452513
				12041 451502
				12042 450925
				12043 446142
				12044 457528
				12045 430764
				12046 423625
				12047 455487
				12048 452084
				12049 452480
				12050 460834
				12051 451751
				12052 456434
				12053 453105
				12054 454740
				12055 449507
				12056 455530
				12057 452391
				12058 462392
				12059 460730
				12060 425396
				12061 419861
				12062 463432
				12063 456719
				12064 452414
				12065 443737
				12066 462549
				12067 452521
				12068 441484
				12069 455682
				12070 456085
				12071 453838
				12072 467091
				12073 467907
				12074 472197
				12075 432442
				12076 419570
				12077 424686
				12078 411700
				12079 406664
				12080 418519
				12081 406441
				12082 413920
				12083 414105
				12084 467267
				12085 456488
				12086 453445
				12087 469507
				12088 454915
				12089 459920
				12090 426653
				12091 419953
				12092 409907
				12093 456401
				12094 451319
				12095 452076
				12096 451915
				12097 455759
				12098 456349
				12099 454071
				12100 439662
				12101 448353
				12102 456647
				12103 450177
				12104 451697
				12105 424801
				12106 422320
				12107 418254
				12108 406683
				12109 407258
				12110 423165
				12111 416072
				12112 415524
				12113 406641
				12114 405189
				12115 426081
				12116 408881
				12117 413824
				12118 413708
				12119 415852
				12120 425299
				12121 423655
				12122 413844
				12123 406970
				12124 437775
				12125 420963
				12126 413991
				12127 433668
				12128 423405
				12129 417539
				12130 418233
				12131 437767
				12132 427860
				12133 464001
				12134 483253
				12135 438147
				12136 423547
				12137 415134
				12138 422300
				12139 409355
				12140 422206
				12141 423078
				12142 465432
				12143 468545
				12144 461623
				12145 474357
				12146 465818
				12147 493493
				12148 480968
				12149 486340
				12150 429822
				12151 424469
				12152 476915
				12153 472765
				12154 469935
				12155 478124
				12156 471870
				12157 493026
				12158 475157
				12159 473072
				12160 485521
				12161 504138
				12162 494488
				12163 487827
				12164 483604
				12165 434667
				12166 422481
				12167 426054
				12168 420049
				12169 414830
				12170 407394
				12171 421515
				12172 419764
				12173 416326
				12174 477048
				12175 471302
				12176 456823
				12177 465462
				12178 477434
				12179 471241
				12180 432259
				12181 422429
				12182 411893
				12183 413964
				12184 410008
				12185 409335
				12186 416576
				12187 437510
				12188 450154
				12189 458707
				12190 471552
				12191 467131
				12192 456181
				12193 468484
				12194 477978
				12195 431599
				12196 481419
				12197 475101
				12198 473623
				12199 471876
				12200 473247
				12201 478121
				12202 462811
				12203 471181
				12204 479476
				12205 473173
				12206 466637
				12207 486454
				12208 467133
				12209 477564
				12210 435763
				12211 420562
				12212 420068
				12213 412705
				12214 399673
				12215 472211
				12216 457084
				12217 466496
				12218 450649
				12219 451535
				12220 450757
				12221 447400
				12222 460039
				12223 484897
				12224 479491
				12225 437225
				12226 432567
				12227 425601
				12228 401001
				12229 426276
				12230 417194
				12231 417995
				12232 420561
				12233 470617
				12234 465495
				12235 471854
				12236 460627
				12237 467393
				12238 464161
				12239 478276
				12240 428881
				12241 434896
				12242 418376
				12243 407970
				12244 418750
				12245 419935
				12246 468508
				12247 462046
				12248 460306
				12249 456340
				12250 458920
				12251 453021
				12252 457533
				12253 462727
				12254 452174
				12255 431638
				12256 418452
				12257 441850
				12258 418313
				12259 422483
				12260 433219
				12261 424295
				12262 414572
				12263 455248
				12264 469365
				12265 461159
				12266 468731
				12267 464509
				12268 464947
				12269 459061
				12270 428144
				12271 415447
				12272 417962
				12273 420749
				12274 423106
				12275 425474
				12276 423322
				12277 426004
				12278 432809
				12279 430320
				12280 423017
				12281 432674
				12282 424165
				12283 416993
				12284 419512
				12285 430099
				12286 420579
				12287 472035
				12288 479941
				12289 451198
				12290 463859
				12291 466807
				12292 458657
				12293 456039
				12294 466298
				12295 464455
				12296 492446
				12297 462479
				12298 469990
				12299 469936
				12300 429807
				12301 420170
				12302 416551
				12303 458632
				12304 456826
				12305 452682
				12306 468436
				12307 456557
				12308 454200
				12309 448400
				12310 455929
				12311 476228
				12312 492301
				12313 473205
				12314 473031
				12315 429261
				12316 427285
				12317 415161
				12318 421604
				12319 418891
				12320 410846
				12321 414185
				12322 409162
				12323 420334
				12324 416866
				12325 413659
				12326 418267
				12327 418680
				12328 413825
				12329 411861
				12330 424572
				12331 419678
				12332 432214
				12333 411374
				12334 437291
				12335 436650
				12336 428328
				12337 421771
				12338 419782
				12339 431807
				12340 431311
				12341 467104
				12342 453877
				12343 462335
				12344 462523
				12345 432052
				12346 427443
				12347 422809
				12348 415741
				12349 415554
				12350 426034
				12351 427736
				12352 428366
				12353 427920
				12354 426075
				12355 421127
				12356 428081
				12357 430782
				12358 488462
				12359 465987
				12360 409660
				12361 416754
				12362 407144
				12363 470233
				12364 453677
				12365 449935
				12366 456541
				12367 475034
				12368 469128
				12369 474971
				12370 480832
				12371 485801
				12372 464021
				12373 478034
				12374 473361
				12375 430417
				12376 408367
				12377 407297
				12378 406734
				12379 411303
				12380 400438
				12381 463288
				12382 460479
				12383 462879
				12384 468244
				12385 455258
				12386 460096
				12387 472142
				12388 462237
				12389 472099
				12390 421709
				12391 430892
				12392 411840
				12393 418639
				12394 429792
				12395 421434
				12396 432293
				12397 429959
				12398 442355
				12399 435637
				12400 436891
				12401 448848
				12402 470206
				12403 456436
				12404 466960
				12405 436347
				12406 435638
				12407 420942
				12408 412683
				12409 426421
				12410 411724
				12411 423011
				12412 416345
				12413 421694
				12414 419544
				12415 419491
				12416 428000
				12417 415304
				12418 417163
				12419 424244
				12420 425587
				12421 414181
				12422 408786
				12423 410711
				12424 465807
				12425 449951
				12426 454026
				12427 451124
				12428 442106
				12429 471024
				12430 459032
				12431 464190
				12432 465920
				12433 463867
				12434 465592
				12435 434383
				12436 415066
				12437 419732
				12438 474511
				12439 473079
				12440 469194
				12441 457990
				12442 479008
				12443 467705
				12444 465049
				12445 473186
				12446 461751
				12447 469430
				12448 464519
				12449 469911
				12450 421332
				12451 409365
				12452 407065
				12453 411896
				12454 395054
				12455 415802
				12456 430536
				12457 400076
				12458 398955
				12459 409529
				12460 405649
				12461 396786
				12462 414494
				12463 412927
				12464 433478
				12465 423835
				12466 414960
				12467 423529
				12468 421635
				12469 404183
				12470 418984
				12471 422449
				12472 414341
				12473 412757
				12474 416271
				12475 431968
				12476 427632
				12477 416383
				12478 462861
				12479 456818
				12480 427278
				12481 407676
				12482 403646
				12483 402474
				12484 410080
				12485 406574
				12486 402353
				12487 401263
				12488 396923
				12489 409948
				12490 412198
				12491 421524
				12492 441886
				12493 437477
				12494 439188
				12495 422524
				12496 422158
				12497 424510
				12498 426007
				12499 424802
				12500 443849
				12501 473713
				12502 455264
				12503 469496
				12504 480112
				12505 474541
				12506 473578
				12507 464648
				12508 468575
				12509 466951
				12510 423007
				12511 417154
				12512 406773
				12513 409590
				12514 429459
				12515 424680
				12516 419388
				12517 415865
				12518 408486
				12519 417244
				12520 416665
				12521 412664
				12522 409306
				12523 418452
				12524 414160
				12525 427934
				12526 415926
				12527 406475
				12528 406087
				12529 416894
				12530 416724
				12531 402343
				12532 403443
				12533 405228
				12534 400740
				12535 433900
				12536 470132
				12537 465086
				12538 466205
				12539 478181
				12540 427320
				12541 469371
				12542 441099
				12543 450475
				12544 473749
				12545 445883
				12546 439890
				12547 451657
				12548 440714
				12549 429498
				12550 462235
				12551 449661
				12552 458772
				12553 459593
				12554 451682
				12555 432889
				12556 413740
				12557 416225
				12558 409612
				12559 408543
				12560 402552
				12561 410225
				12562 406161
				12563 410322
				12564 424523
				12565 417833
				12566 410959
				12567 412883
				12568 406565
				12569 413700
				12570 417538
				12571 409077
				12572 403887
				12573 461045
				12574 441993
				12575 446061
				12576 464231
				12577 468597
				12578 458407
				12579 469274
				12580 462493
				12581 467141
				12582 452937
				12583 474512
				12584 465202
				12585 431451
				12586 418643
				12587 416011
				12588 405900
				12589 400626
				12590 398366
				12591 411724
				12592 415783
				12593 403166
				12594 405834
				12595 430822
				12596 411669
				12597 420974
				12598 447762
				12599 445916
				12600 420994
				12601 415923
				12602 403656
				12603 399348
				12604 413177
				12605 421894
				12606 425282
				12607 463229
				12608 455125
				12609 474525
				12610 469065
				12611 464114
				12612 470360
				12613 450318
				12614 460113
				12615 424038
				12616 412655
				12617 448237
				12618 418065
				12619 407583
				12620 463875
				12621 464521
				12622 463917
				12623 465660
				12624 469687
				12625 458932
				12626 455927
				12627 459028
				12628 461514
				12629 472300
				12630 418520
				12631 414518
				12632 461917
				12633 479830
				12634 480054
				12635 479360
				12636 471588
				12637 461256
				12638 477878
				12639 472277
				12640 467616
				12641 471799
				12642 487053
				12643 461029
				12644 477894
				12645 420711
				12646 415930
				12647 415140
				12648 420917
				12649 420243
				12650 429938
				12651 420692
				12652 419338
				12653 417613
				12654 414257
				12655 412836
				12656 428291
				12657 436639
				12658 427872
				12659 434472
				12660 419461
				12661 421430
				12662 421155
				12663 413042
				12664 472687
				12665 469008
				12666 471915
				12667 468457
				12668 477275
				12669 481737
				12670 486613
				12671 479422
				12672 476837
				12673 487520
				12674 479306
				12675 421015
				12676 423856
				12677 419307
				12678 454847
				12679 475547
				12680 467793
				12681 465228
				12682 470390
				12683 462705
				12684 491801
				12685 469732
				12686 475783
				12687 474472
				12688 471638
				12689 461918
				12690 430928
				12691 429477
				12692 425012
				12693 431025
				12694 412269
				12695 421875
				12696 427604
				12697 419854
				12698 425024
				12699 429965
				12700 421603
				12701 411430
				12702 460434
				12703 445734
				12704 459997
				12705 415994
				12706 415305
				12707 420430
				12708 427453
				12709 416635
				12710 419805
				12711 424891
				12712 407011
				12713 419371
				12714 422009
				12715 415265
				12716 417023
				12717 419228
				12718 420770
				12719 421803
				12720 427878
				12721 411725
				12722 408902
				12723 415187
				12724 410492
				12725 415686
				12726 402418
				12727 404345
				12728 408954
				12729 413722
				12730 396144
				12731 423716
				12732 406314
				12733 410588
				12734 409679
				12735 413586
				12736 412195
				12737 410525
				12738 402630
				12739 446247
				12740 450708
				12741 430346
				12742 438641
				12743 427814
				12744 441628
				12745 485445
				12746 473962
				12747 468825
				12748 461627
				12749 468968
				12750 432322
				12751 414339
				12752 416448
				12753 452691
				12754 459228
				12755 451265
				12756 444488
				12757 458065
				12758 453129
				12759 443431
				12760 444912
				12761 473949
				12762 470344
				12763 454066
				12764 468633
				12765 431396
				12766 484787
				12767 476662
				12768 464734
				12769 479026
				12770 493078
				12771 478656
				12772 475391
				12773 477409
				12774 491112
				12775 474746
				12776 484938
				12777 476765
				12778 486712
				12779 483732
				12780 419131
				12781 414288
				12782 412409
				12783 448221
				12784 454308
				12785 460975
				12786 447804
				12787 443500
				12788 457611
				12789 452783
				12790 454473
				12791 472545
				12792 450189
				12793 462532
				12794 458157
				12795 416414
				12796 465849
				12797 456248
				12798 476031
				12799 461446
				12800 497604
				12801 473108
				12802 477573
				12803 485019
				12804 484290
				12805 479533
				12806 503842
				12807 479439
				12808 477900
				12809 482862
				12810 423599
				12811 404354
				12812 407250
				12813 402668
				12814 401667
				12815 400841
				12816 400592
				12817 456943
				12818 452617
				12819 457217
				12820 452516
				12821 454240
				12822 465904
				12823 455855
				12824 456052
				12825 423003
				12826 412899
				12827 412312
				12828 408720
				12829 454284
				12830 455891
				12831 455746
				12832 452967
				12833 449721
				12834 457398
				12835 444793
				12836 447133
				12837 454491
				12838 455793
				12839 454457
				12840 426252
				12841 418440
				12842 405156
				12843 398007
				12844 400144
				12845 400980
				12846 422203
				12847 421508
				12848 407056
				12849 414908
				12850 407358
				12851 413689
				12852 422181
				12853 422388
				12854 414929
				12855 417406
				12856 411769
				12857 454067
				12858 452407
				12859 447172
				12860 445024
				12861 437679
				12862 448666
				12863 444000
				12864 440334
				12865 442419
				12866 437296
				12867 440204
				12868 448339
				12869 448876
				12870 431475
				12871 421540
				12872 448673
				12873 448169
				12874 443510
				12875 445748
				12876 439700
				12877 443672
				12878 448677
				12879 438594
				12880 447468
				12881 454143
				12882 449404
				12883 449471
				12884 443576
				12885 431411
				12886 415356
				12887 407054
				12888 395862
				12889 406685
				12890 413058
				12891 427079
				12892 415994
				12893 422765
				12894 408017
				12895 411510
				12896 408024
				12897 415122
				12898 418300
				12899 414876
				12900 420269
				12901 474534
				12902 456052
				12903 461370
				12904 460977
				12905 452189
				12906 460673
				12907 449532
				12908 454347
				12909 438191
				12910 433090
				12911 434009
				12912 452061
				12913 433375
				12914 450771
				12915 418734
				12916 416530
				12917 394955
				12918 445327
				12919 434509
				12920 450607
				12921 447228
				12922 450423
				12923 442720
				12924 440775
				12925 447302
				12926 470222
				12927 463199
				12928 466409
				12929 460821
				12930 423974
				12931 453013
				12932 436354
				12933 434920
				12934 435181
				12935 449266
				12936 446837
				12937 472221
				12938 454203
				12939 454139
				12940 467934
				12941 463734
				12942 462701
				12943 473543
				12944 470349
				12945 427208
				12946 463962
				12947 458264
				12948 448489
				12949 468562
				12950 467939
				12951 469063
				12952 481857
				12953 478172
				12954 473583
				12955 474146
				12956 470241
				12957 470045
				12958 475870
				12959 474106
				12960 419638
				12961 422335
				12962 406876
				12963 425423
				12964 414896
				12965 420232
				12966 415828
				12967 415662
				12968 423013
				12969 428687
				12970 417491
				12971 420928
				12972 428037
				12973 423777
				12974 426710
				12975 425863
				12976 419048
				12977 468952
				12978 465509
				12979 464072
				12980 465608
				12981 469108
				12982 472684
				12983 473730
				12984 476238
				12985 485242
				12986 485366
				12987 485259
				12988 476459
				12989 480774
				12990 417332
				12991 418096
				12992 410305
				12993 404751
				12994 418875
				12995 402067
				12996 411697
				12997 450886
				12998 444626
				12999 448405
				13000 454935
				13001 451827
				13002 440922
				13003 443116
				13004 447389
				13005 434885
				13006 420652
				13007 421645
				13008 424761
				13009 461009
				13010 455482
				13011 453732
				13012 467602
				13013 472723
				13014 463213
				13015 482651
				13016 461569
				13017 465622
				13018 467659
				13019 464118
				13020 428238
				13021 414398
				13022 413872
				13023 407596
				13024 420274
				13025 407020
				13026 425758
				13027 415807
				13028 415257
				13029 413754
				13030 421516
				13031 426562
				13032 432653
				13033 420097
				13034 422005
				13035 450081
				13036 419304
				13037 419198
				13038 420775
				13039 413189
				13040 411378
				13041 417515
				13042 420622
				13043 416957
				13044 478117
				13045 468293
				13046 470272
				13047 474463
				13048 463857
				13049 465402
				13050 426254
				13051 426531
				13052 416297
				13053 420281
				13054 443598
				13055 445885
				13056 437175
				13057 470633
				13058 472494
				13059 481202
				13060 476553
				13061 463709
				13062 468768
				13063 461519
				13064 486909
				13065 431735
				13066 419064
				13067 427437
				13068 427247
				13069 461037
				13070 469523
				13071 463400
				13072 461925
				13073 457522
				13074 458383
				13075 480562
				13076 476739
				13077 478421
				13078 476670
				13079 476836
				13080 437601
				13081 427504
				13082 423338
				13083 409687
				13084 418853
				13085 472428
				13086 456562
				13087 473771
				13088 471754
				13089 495659
				13090 501343
				13091 491571
				13092 485655
				13093 479034
				13094 492993
				13095 432345
				13096 423115
				13097 417752
				13098 432158
				13099 443877
				13100 420860
				13101 420697
				13102 428474
				13103 432061
				13104 423672
				13105 422227
				13106 434708
				13107 433777
				13108 463430
				13109 452762
				13110 438151
				13111 423453
				13112 429671
				13113 436166
				13114 455637
				13115 458171
				13116 457040
				13117 474215
				13118 467765
				13119 468815
				13120 468286
				13121 463376
				13122 464366
				13123 464561
				13124 465970
				13125 429818
				13126 430543
				13127 418930
				13128 422565
				13129 419154
				13130 426585
				13131 426097
				13132 425862
				13133 433875
				13134 426071
				13135 423058
				13136 422775
				13137 444192
				13138 433644
				13139 446214
				13140 433497
				13141 429281
				13142 421644
				13143 477795
				13144 479830
				13145 468126
				13146 469157
				13147 477165
				13148 476981
				13149 476108
				13150 465887
				13151 490956
				13152 478543
				13153 487675
				13154 475196
				13155 427687
				13156 424561
				13157 430985
				13158 432224
				13159 429432
				13160 428056
				13161 419498
				13162 440085
				13163 443740
				13164 444052
				13165 445546
				13166 448216
				13167 434278
				13168 436169
				13169 439974
				13170 443570
				13171 419093
				13172 407899
				13173 431028
				13174 427440
				13175 426510
				13176 437043
				13177 435620
				13178 462544
				13179 444472
				13180 446838
				13181 456047
				13182 451671
				13183 447049
				13184 452082
				13185 426023
				13186 425399
				13187 433893
				13188 429772
				13189 424289
				13190 413342
				13191 414392
				13192 424491
				13193 428959
				13194 427536
				13195 427681
				13196 434161
				13197 431748
				13198 433315
				13199 432979
				13200 432968
				13201 419013
				13202 419955
				13203 410333
				13204 409486
				13205 460155
				13206 456233
				13207 457190
				13208 458726
				13209 462577
				13210 455450
				13211 481220
				13212 479379
				13213 470412
				13214 474450
				13215 431060
				13216 421788
				13217 471501
				13218 461403
				13219 476268
				13220 472421
				13221 469602
				13222 472932
				13223 476903
				13224 484113
				13225 471641
				13226 491670
				13227 490414
				13228 487303
				13229 482895
				13230 432929
				13231 439748
				13232 422101
				13233 430309
				13234 422365
				13235 421146
				13236 425496
				13237 425003
				13238 421772
				13239 424048
				13240 416152
				13241 432678
				13242 428151
				13243 432936
				13244 422286
				13245 426445
				13246 434006
				13247 410662
				13248 430751
				13249 412710
				13250 414573
				13251 422926
				13252 417027
				13253 408941
				13254 420948
				13255 423486
				13256 469257
				13257 470373
				13258 463369
				13259 459019
				13260 427485
				13261 426890
				13262 406487
				13263 417994
				13264 431119
				13265 430323
				13266 424462
				13267 429652
				13268 424533
				13269 430484
				13270 415299
				13271 425185
				13272 419894
				13273 435993
				13274 429382
				13275 428056
				13276 418089
				13277 429661
				13278 412180
				13279 466493
				13280 471934
				13281 483877
				13282 474305
				13283 491851
				13284 489518
				13285 480126
				13286 487956
				13287 486894
				13288 493650
				13289 492960
				13290 433375
				13291 429647
				13292 416519
				13293 404903
				13294 470485
				13295 463473
				13296 456281
				13297 460338
				13298 460461
				13299 455653
				13300 465783
				13301 459307
				13302 443487
				13303 455887
				13304 466210
				13305 432872
				13306 429611
				13307 421158
				13308 438298
				13309 439164
				13310 433747
				13311 457607
				13312 441162
				13313 440518
				13314 471105
				13315 458426
				13316 451554
				13317 460011
				13318 463338
				13319 472615
				13320 434822
				13321 437051
				13322 438205
				13323 434885
				13324 430698
				13325 419889
				13326 459238
				13327 457243
				13328 447925
				13329 453935
				13330 478910
				13331 476777
				13332 473282
				13333 469431
				13334 479795
				13335 422639
				13336 430912
				13337 424118
				13338 421681
				13339 420728
				13340 422175
				13341 416898
				13342 417633
				13343 424728
				13344 424343
				13345 418560
				13346 411650
				13347 479026
				13348 479653
				13349 481589
				13350 432930
				13351 428088
				13352 424911
				13353 422246
				13354 411516
				13355 408035
				13356 418741
				13357 480721
				13358 474071
				13359 459106
				13360 473355
				13361 477889
				13362 473682
				13363 467367
				13364 478505
				13365 430500
				13366 417382
				13367 456711
				13368 438623
				13369 448518
				13370 451349
				13371 448925
				13372 442034
				13373 444390
				13374 443620
				13375 440050
				13376 450034
				13377 447127
				13378 434131
				13379 445747
				13380 424513
				13381 438299
				13382 426935
				13383 421671
				13384 422454
				13385 426018
				13386 423063
				13387 422339
				13388 428809
				13389 423262
				13390 422890
				13391 424248
				13392 417680
				13393 427952
				13394 425869
				13395 419870
				13396 425363
				13397 426236
				13398 427459
				13399 426166
				13400 422559
				13401 415032
				13402 415552
				13403 405261
				13404 429543
				13405 421447
				13406 430520
				13407 424764
				13408 426326
				13409 427311
				13410 421649
				13411 420396
				13412 415220
				13413 414104
				13414 461231
				13415 451873
				13416 448793
				13417 449142
				13418 450080
				13419 464427
				13420 466053
				13421 461932
				13422 475372
				13423 467638
				13424 452281
				13425 413164
				13426 425074
				13427 410578
				13428 455311
				13429 450056
				13430 452645
				13431 461098
				13432 464774
				13433 469587
				13434 473075
				13435 459906
				13436 461007
				13437 464596
				13438 465803
				13439 461979
				13440 417269
				13441 461025
				13442 464174
				13443 455318
				13444 462537
				13445 495721
				13446 464230
				13447 470980
				13448 460642
				13449 472945
				13450 459906
				13451 451344
				13452 456971
				13453 469864
				13454 464178
				13455 422599
				13456 418402
				13457 415186
				13458 405923
				13459 409350
				13460 412571
				13461 428768
				13462 406600
				13463 409882
				13464 401980
				13465 412610
				13466 450923
				13467 457908
				13468 453615
				13469 450880
				13470 425619
				13471 420234
				13472 408575
				13473 408113
				13474 413396
				13475 410123
				13476 406073
				13477 405422
				13478 404389
				13479 416022
				13480 407199
				13481 416440
				13482 470452
				13483 440197
				13484 448045
				13485 421982
				13486 422250
				13487 402402
				13488 406350
				13489 402677
				13490 427449
				13491 396553
				13492 400832
				13493 482230
				13494 467781
				13495 462623
				13496 476610
				13497 483114
				13498 469239
				13499 458002
				13500 422878
				13501 429579
				13502 408248
				13503 421833
				13504 415754
				13505 430463
				13506 420451
				13507 429063
				13508 420984
				13509 418284
				13510 419267
				13511 416117
				13512 421422
				13513 422747
				13514 421167
				13515 413593
				13516 412137
				13517 411926
				13518 409960
				13519 439644
				13520 439830
				13521 439524
				13522 434809
				13523 431461
				13524 476807
				13525 467513
				13526 470378
				13527 477283
				13528 471965
				13529 473105
				13530 420946
				13531 414301
				13532 420472
				13533 402026
				13534 400313
				13535 415622
				13536 456699
				13537 450474
				13538 442508
				13539 449723
				13540 454206
				13541 453861
				13542 454579
				13543 442428
				13544 464927
				13545 431721
				13546 419147
				13547 419109
				13548 408877
				13549 401507
				13550 451439
				13551 443339
				13552 455836
				13553 460717
				13554 453071
				13555 458847
				13556 477132
				13557 477945
				13558 477684
				13559 469069
				13560 429489
				13561 413120
				13562 417133
				13563 413654
				13564 424913
				13565 409774
				13566 414558
				13567 456015
				13568 457761
				13569 452029
				13570 455506
				13571 463586
				13572 468526
				13573 475710
				13574 471604
				13575 424126
				13576 410762
				13577 414981
				13578 408555
				13579 412616
				13580 471309
				13581 463724
				13582 452243
				13583 459794
				13584 465732
				13585 474573
				13586 472758
				13587 481670
				13588 461825
				13589 467312
				13590 418234
				13591 427427
				13592 414094
				13593 414212
				13594 422826
				13595 428304
				13596 413086
				13597 433204
				13598 422450
				13599 432554
				13600 424458
				13601 467717
				13602 464707
				13603 459619
				13604 471956
				13605 430523
				13606 449420
				13607 462937
				13608 467810
				13609 459226
				13610 468399
				13611 465096
				13612 452110
				13613 449207
				13614 450302
				13615 458079
				13616 463305
				13617 457446
				13618 461233
				13619 457501
				13620 445479
				13621 427084
				13622 432968
				13623 429437
				13624 421725
				13625 420026
				13626 464747
				13627 446327
				13628 454557
				13629 433522
				13630 448484
				13631 465672
				13632 463298
				13633 473231
				13634 484942
				13635 431289
				13636 424937
				13637 455896
				13638 485981
				13639 463525
				13640 471940
				13641 481808
				13642 462594
				13643 473015
				13644 460828
				13645 468982
				13646 468658
				13647 487207
				13648 476636
				13649 479311
				13650 432719
				13651 428356
				13652 402896
				13653 405256
				13654 464083
				13655 479832
				13656 460372
				13657 458832
				13658 454761
				13659 457805
				13660 455552
				13661 455199
				13662 454177
				13663 453890
				13664 459341
				13665 431763
				13666 418009
				13667 425577
				13668 406716
				13669 409465
				13670 422244
				13671 405757
				13672 404882
				13673 432774
				13674 419486
				13675 450157
				13676 478878
				13677 467435
				13678 475634
				13679 472161
				13680 423572
				13681 464397
				13682 444295
				13683 457957
				13684 466124
				13685 452404
				13686 448034
				13687 450502
				13688 454386
				13689 460155
				13690 452014
				13691 461178
				13692 457583
				13693 456684
				13694 466088
				13695 433102
				13696 423132
				13697 417352
				13698 429170
				13699 427731
				13700 427510
				13701 429506
				13702 432756
				13703 436792
				13704 475040
				13705 465313
				13706 461473
				13707 454973
				13708 458864
				13709 468242
				13710 429285
				13711 438636
				13712 430705
				13713 422108
				13714 425315
				13715 415933
				13716 412823
				13717 419084
				13718 427122
				13719 444467
				13720 464731
				13721 446361
				13722 451582
				13723 453928
				13724 455335
				13725 419685
				13726 425753
				13727 419932
				13728 425729
				13729 412845
				13730 416132
				13731 419294
				13732 423002
				13733 428975
				13734 426937
				13735 412694
				13736 419280
				13737 457125
				13738 457033
				13739 464264
				13740 422047
				13741 407877
				13742 404227
				13743 405760
				13744 405394
				13745 402211
				13746 431728
				13747 417116
				13748 425110
				13749 422318
				13750 417217
				13751 425261
				13752 426818
				13753 422753
				13754 412024
				13755 414625
				13756 415414
				13757 405355
				13758 408516
				13759 415826
				13760 413762
				13761 407622
				13762 418906
				13763 410807
				13764 412440
				13765 422735
				13766 420938
				13767 414270
				13768 421501
				13769 417734
				13770 423372
				13771 417653
				13772 427818
				13773 462588
				13774 462317
				13775 456510
				13776 456859
				13777 446050
				13778 466991
				13779 465162
				13780 456599
				13781 441709
				13782 452428
				13783 454917
				13784 446359
				13785 424878
				13786 414117
				13787 426253
				13788 415281
				13789 425583
				13790 421303
				13791 422248
				13792 417493
				13793 427462
				13794 418712
				13795 423657
				13796 415390
				13797 425192
				13798 426857
				13799 452515
				13800 425886
				13801 484965
				13802 473179
				13803 482454
				13804 472634
				13805 469315
				13806 468479
				13807 477062
				13808 476769
				13809 471806
				13810 477051
				13811 485974
				13812 469567
				13813 484441
				13814 469468
				13815 437920
				13816 415381
				13817 417110
				13818 438724
				13819 420082
				13820 416723
				13821 418041
				13822 436809
				13823 435936
				13824 464924
				13825 455533
				13826 451073
				13827 460921
				13828 454772
				13829 474192
				13830 415830
				13831 426722
				13832 410678
				13833 411635
				13834 413250
				13835 407499
				13836 401610
				13837 409782
				13838 403438
				13839 403681
				13840 397263
				13841 452561
				13842 446287
				13843 459603
				13844 456214
				13845 429194
				13846 418953
				13847 433642
				13848 412697
				13849 416893
				13850 411210
				13851 442039
				13852 432287
				13853 446107
				13854 438816
				13855 437413
				13856 434926
				13857 447069
				13858 431148
				13859 439383
				13860 434595
				13861 415341
				13862 414145
				13863 416682
				13864 417660
				13865 450431
				13866 460768
				13867 453234
				13868 451159
				13869 469515
				13870 462440
				13871 455137
				13872 473571
				13873 480491
				13874 473519
				13875 435274
				13876 420780
				13877 411226
				13878 409077
				13879 459891
				13880 454718
				13881 448748
				13882 457332
				13883 448591
				13884 451962
				13885 468012
				13886 469992
				13887 464143
				13888 452523
				13889 470475
				13890 420580
				13891 422642
				13892 416781
				13893 417518
				13894 419419
				13895 418298
				13896 475558
				13897 470839
				13898 471941
				13899 469136
				13900 477970
				13901 479381
				13902 478881
				13903 476753
				13904 472515
				13905 436701
				13906 442876
				13907 427367
				13908 483726
				13909 478223
				13910 480931
				13911 473297
				13912 473125
				13913 467743
				13914 469624
				13915 481924
				13916 491028
				13917 466668
				13918 491137
				13919 483919
				13920 431521
				13921 425790
				13922 419346
				13923 409778
				13924 434073
				13925 439708
				13926 438056
				13927 436996
				13928 429911
				13929 426099
				13930 431104
				13931 454811
				13932 461612
				13933 475458
				13934 461605
				13935 437027
				13936 437493
				13937 426877
				13938 416689
				13939 426294
				13940 419229
				13941 414984
				13942 464769
				13943 461763
				13944 457583
				13945 449115
				13946 463744
				13947 476156
				13948 472514
				13949 481484
				13950 435151
				13951 494124
				13952 485369
				13953 474128
				13954 487437
				13955 478402
				13956 478697
				13957 485605
				13958 468118
				13959 476962
				13960 481452
				13961 486658
				13962 472358
				13963 482843
				13964 497740
				13965 428911
				13966 436888
				13967 418050
				13968 426800
				13969 426519
				13970 423532
				13971 457195
				13972 456663
				13973 464938
				13974 473485
				13975 460398
				13976 490994
				13977 468989
				13978 472764
				13979 482710
				13980 437119
				13981 439781
				13982 441111
				13983 443643
				13984 473715
				13985 459597
				13986 460436
				13987 458923
				13988 474020
				13989 463375
				13990 462033
				13991 465349
				13992 468213
				13993 469282
				13994 470901
				13995 442991
				13996 433668
				13997 445282
				13998 441764
				13999 436947
				14000 440826
				14001 438987
				14002 439816
				14003 432034
				14004 435071
				14005 435504
				14006 432368
				14007 480871
				14008 454257
				14009 460115
				14010 442637
				14011 444634
				14012 473889
				14013 485507
				14014 482984
				14015 496827
				14016 493469
				14017 510991
				14018 493570
				14019 499950
				14020 492626
				14021 482352
				14022 498602
				14023 502494
				14024 486798
				14025 432978
				14026 471808
				14027 466527
				14028 464285
				14029 464277
				14030 454691
				14031 464962
				14032 463023
				14033 455776
				14034 463941
				14035 476082
				14036 453367
				14037 456261
				14038 454944
				14039 449063
				14040 435083
				14041 438550
				14042 416688
				14043 423759
				14044 421432
				14045 421225
				14046 425306
				14047 442295
				14048 425494
				14049 439111
				14050 429251
				14051 438856
				14052 431947
				14053 428723
				14054 428596
				14055 436531
				14056 479216
				14057 475676
				14058 465318
				14059 469095
				14060 481187
				14061 494448
				14062 476983
				14063 483459
				14064 490923
				14065 484401
				14066 476142
				14067 468977
				14068 475899
				14069 462930
				14070 435784
				14071 433111
				14072 425265
				14073 428143
				14074 470651
				14075 479967
				14076 481539
				14077 479056
				14078 480354
				14079 464742
				14080 470788
				14081 473863
				14082 472515
				14083 471861
				14084 475706
				14085 434382
				14086 421486
				14087 430616
				14088 422384
				14089 420714
				14090 414383
				14091 430762
				14092 427466
				14093 431426
				14094 430669
				14095 429760
				14096 423284
				14097 425591
				14098 424954
				14099 429663
				14100 440734
				14101 438977
				14102 426032
				14103 416979
				14104 421970
				14105 428858
				14106 434447
				14107 439651
				14108 427878
				14109 436978
				14110 441537
				14111 442525
				14112 428486
				14113 423384
				14114 475297
				14115 440371
				14116 432767
				14117 430397
				14118 476992
				14119 464994
				14120 478410
				14121 499794
				14122 493759
				14123 475627
				14124 484484
				14125 478552
				14126 496968
				14127 481117
				14128 484247
				14129 482498
				14130 441232
				14131 421645
				14132 419188
				14133 445952
				14134 427433
				14135 460802
				14136 463062
				14137 483761
				14138 455952
				14139 462454
				14140 471839
				14141 484053
				14142 469777
				14143 473164
				14144 472821
				14145 438868
				14146 431408
				14147 415360
				14148 431939
				14149 429879
				14150 417704
				14151 431835
				14152 418371
				14153 419415
				14154 421619
				14155 429227
				14156 420605
				14157 447419
				14158 436162
				14159 421507
				14160 428051
				14161 428255
				14162 437346
				14163 426523
				14164 420980
				14165 428621
				14166 424734
				14167 418330
				14168 427104
				14169 422602
				14170 441084
				14171 454145
				14172 429356
				14173 436600
				14174 443401
				14175 443492
				14176 431110
				14177 441236
				14178 436845
				14179 444570
				14180 435268
				14181 487189
				14182 483816
				14183 482830
				14184 479675
				14185 469021
				14186 468346
				14187 469392
				14188 487205
				14189 477317
				14190 444476
				14191 430031
				14192 425631
				14193 490841
				14194 468121
				14195 468911
				14196 470120
				14197 466865
				14198 496528
				14199 487115
				14200 492152
				14201 487693
				14202 484647
				14203 494497
				14204 482886
				14205 439981
				14206 447376
				14207 436644
				14208 482556
				14209 499142
				14210 485940
				14211 484716
				14212 489419
				14213 487790
				14214 492387
				14215 488960
				14216 483003
				14217 488144
				14218 488689
				14219 476127
				14220 431070
				14221 430663
				14222 430992
				14223 425600
				14224 486990
				14225 507630
				14226 497206
				14227 487629
				14228 476905
				14229 494788
				14230 489085
				14231 493033
				14232 489402
				14233 491607
				14234 486579
				14235 439623
				14236 442980
				14237 429215
				14238 424261
				14239 423195
				14240 434440
				14241 476678
				14242 483980
				14243 474120
				14244 464119
				14245 465146
				14246 486006
				14247 489546
				14248 487488
				14249 502382
				14250 449632
				14251 433073
				14252 434088
				14253 433006
				14254 469776
				14255 455091
				14256 480305
				14257 484616
				14258 476459
				14259 483951
				14260 476017
				14261 475667
				14262 474163
				14263 493326
				14264 477305
				14265 428523
				14266 477047
				14267 468518
				14268 461800
				14269 464446
				14270 474630
				14271 485860
				14272 479834
				14273 487726
				14274 480953
				14275 480812
				14276 477745
				14277 469713
				14278 499481
				14279 494523
				14280 452788
				14281 431748
				14282 420770
				14283 430809
				14284 429963
				14285 429487
				14286 428889
				14287 443329
				14288 445969
				14289 441668
				14290 447029
				14291 440488
				14292 457611
				14293 447454
				14294 436647
				14295 442118
				14296 503384
				14297 491445
				14298 477047
				14299 495359
				14300 487555
				14301 477580
				14302 485809
				14303 482853
				14304 497894
				14305 475664
				14306 491251
				14307 491938
				14308 488190
				14309 492421
				14310 426780
				14311 434394
				14312 430569
				14313 440524
				14314 454203
				14315 443319
				14316 455436
				14317 497110
				14318 489712
				14319 473932
				14320 468189
				14321 480445
				14322 474188
				14323 480207
				14324 475281
				14325 437347
				14326 438491
				14327 457520
				14328 466145
				14329 469677
				14330 462224
				14331 463833
				14332 453056
				14333 453349
				14334 444471
				14335 465618
				14336 454932
				14337 454488
				14338 455750
				14339 461560
				14340 446617
				14341 444081
				14342 472330
				14343 483905
				14344 484697
				14345 480739
				14346 472092
				14347 476992
				14348 474650
				14349 477197
				14350 470516
				14351 475169
				14352 479312
				14353 472328
				14354 479967
				14355 452938
				14356 487146
				14357 473318
				14358 484656
				14359 471419
				14360 477119
				14361 469261
				14362 461646
				14363 474083
				14364 468055
				14365 468383
				14366 465640
				14367 460328
				14368 466656
				14369 483656
				14370 441373
				14371 432783
				14372 429173
				14373 417416
				14374 433923
				14375 427674
				14376 425724
				14377 416128
				14378 495233
				14379 486347
				14380 484595
				14381 484675
				14382 494789
				14383 480206
				14384 481203
				14385 437865
				14386 436503
				14387 429933
				14388 424551
				14389 425898
				14390 411700
				14391 423453
				14392 417339
				14393 428570
				14394 427901
				14395 421185
				14396 471505
				14397 464605
				14398 459442
				14399 474839
				14400 428774
				14401 430863
				14402 423501
				14403 424343
				14404 421388
				14405 423168
				14406 413217
				14407 422456
				14408 419136
				14409 430272
				14410 420420
				14411 417513
				14412 419453
				14413 424782
				14414 427176
				14415 436350
				14416 432957
				14417 430862
				14418 425077
				14419 434791
				14420 412395
				14421 417085
				14422 436669
				14423 460251
				14424 492829
				14425 481649
				14426 497204
				14427 487359
				14428 487439
				14429 493189
				14430 438184
				14431 433299
				14432 427144
				14433 466692
				14434 477116
				14435 493687
				14436 479897
				14437 477704
				14438 476801
				14439 480453
				14440 478719
				14441 472201
				14442 477191
				14443 481707
				14444 479166
				14445 436458
				14446 431606
				14447 414289
				14448 430204
				14449 433946
				14450 426419
				14451 429018
				14452 433072
				14453 428029
				14454 432638
				14455 424859
				14456 435803
				14457 432400
				14458 466353
				14459 468907
				14460 434180
				14461 430184
				14462 425888
				14463 418595
				14464 420004
				14465 418312
				14466 425064
				14467 420159
				14468 423864
				14469 410437
				14470 428495
				14471 467086
				14472 454872
				14473 465385
				14474 466191
				14475 433328
				14476 433289
				14477 424259
				14478 409865
				14479 422626
				14480 473509
				14481 461879
				14482 462059
				14483 468571
				14484 471221
				14485 462999
				14486 458176
				14487 480339
				14488 477014
				14489 479569
				14490 429811
				14491 434191
				14492 414961
				14493 420207
				14494 417579
				14495 421168
				14496 410385
				14497 417750
				14498 416636
				14499 422454
				14500 407732
				14501 458612
				14502 475641
				14503 476755
				14504 481559
				14505 427584
				14506 471847
				14507 472375
				14508 480570
				14509 481125
				14510 464432
				14511 465394
				14512 473622
				14513 472549
				14514 463700
				14515 475915
				14516 472523
				14517 470183
				14518 479225
				14519 476857
				14520 426369
				14521 422821
				14522 419386
				14523 424792
				14524 411959
				14525 411920
				14526 410152
				14527 426077
				14528 422950
				14529 463613
				14530 464700
				14531 471740
				14532 464493
				14533 476167
				14534 461524
				14535 424468
				14536 437916
				14537 416744
				14538 412614
				14539 421022
				14540 424687
				14541 424418
				14542 413773
				14543 460848
				14544 464812
				14545 458627
				14546 462306
				14547 471985
				14548 467403
				14549 465076
				14550 444837
				14551 437779
				14552 437681
				14553 434823
				14554 425057
				14555 435834
				14556 427565
				14557 452617
				14558 459885
				14559 455058
				14560 457174
				14561 460764
				14562 458696
				14563 463428
				14564 463962
				14565 433501
				14566 424971
				14567 427118
				14568 427150
				14569 424205
				14570 418335
				14571 417506
				14572 424563
				14573 422327
				14574 469697
				14575 474065
				14576 486810
				14577 465106
				14578 480698
				14579 478402
				14580 440358
				14581 429775
				14582 424462
				14583 422522
				14584 417647
				14585 412469
				14586 422535
				14587 430349
				14588 445429
				14589 458944
				14590 458370
				14591 448719
				14592 457857
				14593 460869
				14594 476182
				14595 429381
				14596 480657
				14597 474206
				14598 471647
				14599 464357
				14600 459920
				14601 467898
				14602 478727
				14603 484138
				14604 472614
				14605 482371
				14606 479560
				14607 473900
				14608 471430
				14609 482009
				14610 430984
				14611 423546
				14612 431329
				14613 424816
				14614 466626
				14615 466167
				14616 458381
				14617 450021
				14618 466802
				14619 463142
				14620 455433
				14621 474096
				14622 470940
				14623 486627
				14624 474395
				14625 417985
				14626 428574
				14627 451291
				14628 422059
				14629 423381
				14630 418902
				14631 420729
				14632 429534
				14633 434990
				14634 429033
				14635 416755
				14636 416709
				14637 430988
				14638 429464
				14639 418100
				14640 435082
				14641 421353
				14642 421003
				14643 413430
				14644 415055
				14645 419572
				14646 413641
				14647 425887
				14648 416059
				14649 423365
				14650 407187
				14651 414382
				14652 431956
				14653 430682
				14654 426968
				14655 446711
				14656 420644
				14657 412050
				14658 409982
				14659 422492
				14660 422910
				14661 426564
				14662 414805
				14663 410758
				14664 416672
				14665 417556
				14666 412146
				14667 411427
				14668 414729
				14669 412015
				14670 426616
				14671 422102
				14672 423868
				14673 408659
				14674 423579
				14675 418964
				14676 429421
				14677 418535
				14678 412562
				14679 423291
				14680 416469
				14681 423108
				14682 413620
				14683 417238
				14684 447987
				14685 431228
				14686 425397
				14687 422670
				14688 425041
				14689 422005
				14690 411020
				14691 463275
				14692 471933
				14693 475316
				14694 477108
				14695 471576
				14696 474942
				14697 481877
				14698 486091
				14699 476793
				14700 432149
				14701 426105
				14702 482178
				14703 460802
				14704 480323
				14705 471368
				14706 470611
				14707 476841
				14708 492257
				14709 480333
				14710 468354
				14711 482907
				14712 453653
				14713 462505
				14714 461077
				14715 429499
				14716 441146
				14717 415269
				14718 421700
				14719 413568
				14720 412062
				14721 409522
				14722 425399
				14723 405457
				14724 420584
				14725 415016
				14726 419254
				14727 422957
				14728 433786
				14729 419330
				14730 432663
				14731 433677
				14732 423795
				14733 457352
				14734 453889
				14735 451209
				14736 456350
				14737 443957
				14738 450057
				14739 442289
				14740 461957
				14741 463437
				14742 478265
				14743 477567
				14744 466321
				14745 435322
				14746 427544
				14747 413382
				14748 431204
				14749 418874
				14750 419771
				14751 403235
				14752 414720
				14753 418913
				14754 448599
				14755 422408
				14756 421147
				14757 423165
				14758 430971
				14759 417080
				14760 437161
				14761 431987
				14762 426632
				14763 413589
				14764 419738
				14765 414596
				14766 412811
				14767 468700
				14768 480482
				14769 465893
				14770 469266
				14771 462410
				14772 475417
				14773 477428
				14774 473742
				14775 425302
				14776 433350
				14777 468589
				14778 455228
				14779 466047
				14780 463173
				14781 461620
				14782 464247
				14783 458958
				14784 461256
				14785 464203
				14786 465805
				14787 464779
				14788 468386
				14789 455704
				14790 442361
				14791 438654
				14792 427925
				14793 426408
				14794 418174
				14795 419361
				14796 437146
				14797 431171
				14798 460274
				14799 465002
				14800 461067
				14801 461927
				14802 455992
				14803 455133
				14804 451540
				14805 429805
				14806 433364
				14807 414365
				14808 429425
				14809 465101
				14810 455866
				14811 471384
				14812 468411
				14813 463612
				14814 456455
				14815 459170
				14816 460153
				14817 500856
				14818 486139
				14819 480267
				14820 436539
				14821 449923
				14822 438211
				14823 430721
				14824 438563
				14825 437826
				14826 439265
				14827 473091
				14828 455514
				14829 473848
				14830 461667
				14831 458021
				14832 458525
				14833 467356
				14834 466592
				14835 439429
				14836 435777
				14837 462746
				14838 491215
				14839 476173
				14840 466132
				14841 466685
				14842 462640
				14843 465891
				14844 473250
				14845 465760
				14846 479591
				14847 474887
				14848 475244
				14849 477403
				14850 426824
				14851 449690
				14852 440130
				14853 413838
				14854 421662
				14855 422509
				14856 432931
				14857 436390
				14858 427992
				14859 435364
				14860 426159
				14861 419788
				14862 434995
				14863 436842
				14864 441707
				14865 435729
				14866 447795
				14867 437372
				14868 436721
				14869 437507
				14870 425997
				14871 436538
				14872 436670
				14873 431705
				14874 433501
				14875 451956
				14876 456667
				14877 440531
				14878 489447
				14879 479964
				14880 444234
				14881 456249
				14882 451592
				14883 481967
				14884 469002
				14885 477301
				14886 476093
				14887 481961
				14888 481341
				14889 487460
				14890 481752
				14891 474722
				14892 467971
				14893 476204
				14894 466832
				14895 445414
				14896 476387
				14897 485448
				14898 473772
				14899 482906
				14900 486792
				14901 476334
				14902 488453
				14903 484721
				14904 484553
				14905 475695
				14906 479597
				14907 473801
				14908 505173
				14909 501430
				14910 441604
				14911 423190
				14912 430566
				14913 418099
				14914 423718
				14915 419982
				14916 414078
				14917 436221
				14918 428083
				14919 430082
				14920 470741
				14921 474896
				14922 471343
				14923 468905
				14924 480465
				14925 445655
				14926 433211
				14927 472515
				14928 482773
				14929 478395
				14930 480456
				14931 468847
				14932 472872
				14933 473388
				14934 477954
				14935 476414
				14936 483178
				14937 466891
				14938 479036
				14939 464438
				14940 433110
				14941 435873
				14942 482057
				14943 473300
				14944 483526
				14945 479891
				14946 472073
				14947 473579
				14948 477473
				14949 488999
				14950 483012
				14951 479560
				14952 480272
				14953 485944
				14954 494071
				14955 443236
				14956 427101
				14957 427244
				14958 417808
				14959 424655
				14960 429013
				14961 423393
				14962 425825
				14963 414011
				14964 410091
				14965 423792
				14966 418594
				14967 413071
				14968 422592
				14969 414643
				14970 446762
				14971 435635
				14972 413980
				14973 478455
				14974 471811
				14975 470337
				14976 464712
				14977 475171
				14978 474873
				14979 476391
				14980 481248
				14981 473474
				14982 468014
				14983 470889
				14984 480750
				14985 436683
				14986 435476
				14987 414717
				14988 428997
				14989 414879
				14990 426242
				14991 416887
				14992 487799
				14993 491796
				14994 478264
				14995 468117
				14996 481980
				14997 472244
				14998 486619
				14999 470482
				15000 453342
				15001 421634
				15002 476177
				15003 473570
				15004 459207
				15005 467623
				15006 474615
				15007 466418
				15008 467997
				15009 468941
				15010 486873
				15011 475498
				15012 492865
				15013 473841
				15014 479509
				15015 445844
				15016 433384
				15017 426261
				15018 418343
				15019 430638
				15020 425573
				15021 430212
				15022 421596
				15023 426538
				15024 430675
				15025 425582
				15026 437119
				15027 433240
				15028 432797
				15029 424626
				15030 434437
				15031 427030
				15032 432432
				15033 428723
				15034 426743
				15035 406012
				15036 426921
				15037 423295
				15038 478407
				15039 462264
				15040 470236
				15041 467218
				15042 468750
				15043 481103
				15044 480719
				15045 442857
				15046 436964
				15047 431737
				15048 426651
				15049 420944
				15050 426288
				15051 419655
				15052 426017
				15053 433933
				15054 418095
				15055 420860
				15056 418137
				15057 431373
				15058 408861
				15059 429436
				15060 445257
				15061 430422
				15062 421727
				15063 422363
				15064 418951
				15065 414632
				15066 424158
				15067 423770
				15068 486914
				15069 476912
				15070 477764
				15071 469233
				15072 462180
				15073 466707
				15074 490246
				15075 443218
				15076 425236
				15077 430898
				15078 430645
				15079 419158
				15080 417137
				15081 484249
				15082 475818
				15083 476516
				15084 463329
				15085 468517
				15086 467433
				15087 459545
				15088 463635
				15089 464181
				15090 429024
				15091 439012
				15092 434653
				15093 427541
				15094 437246
				15095 438022
				15096 433355
				15097 434503
				15098 425300
				15099 438537
				15100 439756
				15101 428040
				15102 442838
				15103 435622
				15104 437144
				15105 444478
				15106 435509
				15107 432449
				15108 431614
				15109 433234
				15110 440710
				15111 435576
				15112 439403
				15113 434679
				15114 442473
				15115 439817
				15116 428429
				15117 440751
				15118 437321
				15119 440921
				15120 446326
				15121 445642
				15122 445907
				15123 466023
				15124 480846
				15125 481268
				15126 498755
				15127 484377
				15128 480963
				15129 484185
				15130 479336
				15131 476287
				15132 485986
				15133 477601
				15134 478002
				15135 440193
				15136 489791
				15137 480893
				15138 476516
				15139 474069
				15140 494182
				15141 485257
				15142 487394
				15143 498444
				15144 481323
				15145 482366
				15146 496500
				15147 485944
				15148 493498
				15149 495426
				15150 444578
				15151 436617
				15152 423882
				15153 431292
				15154 431620
				15155 423316
				15156 416446
				15157 423672
				15158 418282
				15159 473380
				15160 463113
				15161 469578
				15162 477428
				15163 464416
				15164 465917
				15165 439261
				15166 433967
				15167 430057
				15168 422178
				15169 436682
				15170 424241
				15171 419008
				15172 428276
				15173 428655
				15174 418459
				15175 436341
				15176 430015
				15177 463475
				15178 475633
				15179 474821
				15180 433472
				15181 436056
				15182 418964
				15183 419544
				15184 443589
				15185 425821
				15186 426703
				15187 433140
				15188 440687
				15189 465871
				15190 477243
				15191 489006
				15192 488992
				15193 495939
				15194 503256
				15195 435116
				15196 446465
				15197 428111
				15198 423608
				15199 432491
				15200 430078
				15201 428195
				15202 439845
				15203 423342
				15204 419789
				15205 469813
				15206 459301
				15207 456969
				15208 461755
				15209 457115
				15210 432122
				15211 432729
				15212 427740
				15213 422368
				15214 419261
				15215 474738
				15216 468819
				15217 473465
				15218 466277
				15219 461563
				15220 473709
				15221 470143
				15222 471662
				15223 477774
				15224 471642
				15225 427072
				15226 473802
				15227 458571
				15228 460617
				15229 471233
				15230 453777
				15231 478207
				15232 477092
				15233 469171
				15234 481681
				15235 474256
				15236 477671
				15237 465561
				15238 470529
				15239 461321
				15240 437540
				15241 440833
				15242 421803
				15243 434737
				15244 430583
				15245 431902
				15246 424128
				15247 476293
				15248 467320
				15249 472470
				15250 466876
				15251 470626
				15252 479029
				15253 487602
				15254 482547
				15255 435130
				15256 435991
				15257 427542
				15258 424622
				15259 423193
				15260 485193
				15261 486088
				15262 483232
				15263 462863
				15264 490771
				15265 479053
				15266 476536
				15267 485246
				15268 470530
				15269 473297
				15270 444336
				15271 438856
				15272 433508
				15273 439054
				15274 434075
				15275 428346
				15276 429729
				15277 431306
				15278 428394
				15279 455784
				15280 452021
				15281 464430
				15282 450143
				15283 460859
				15284 462255
				15285 448686
				15286 446820
				15287 483463
				15288 476749
				15289 484241
				15290 487889
				15291 483589
				15292 480320
				15293 489026
				15294 482585
				15295 488941
				15296 483083
				15297 488477
				15298 485634
				15299 497688
				15300 439051
				15301 436235
				15302 438288
				15303 434460
				15304 441198
				15305 436224
				15306 438927
				15307 422296
				15308 436083
				15309 442638
				15310 434820
				15311 434036
				15312 437759
				15313 428217
				15314 471954
				15315 449684
				15316 449121
				15317 434623
				15318 424200
				15319 432071
				15320 437210
				15321 444843
				15322 435373
				15323 432547
				15324 426687
				15325 437542
				15326 431283
				15327 469282
				15328 471007
				15329 460503
				15330 441651
				15331 478245
				15332 470042
				15333 473288
				15334 472103
				15335 455138
				15336 467840
				15337 456583
				15338 459393
				15339 475860
				15340 487137
				15341 471686
				15342 488623
				15343 485818
				15344 483248
				15345 444156
				15346 422726
				15347 420725
				15348 410117
				15349 423360
				15350 416937
				15351 461795
				15352 497636
				15353 472636
				15354 466106
				15355 469128
				15356 465252
				15357 485489
				15358 475393
				15359 468763
				15360 429253
				15361 441131
				15362 440305
				15363 436729
				15364 437659
				15365 426818
				15366 433059
				15367 425791
				15368 420512
				15369 438437
				15370 427006
				15371 431788
				15372 432838
				15373 432653
				15374 430799
				15375 435535
				15376 442321
				15377 430248
				15378 423726
				15379 433109
				15380 420894
				15381 415062
				15382 440719
				15383 430422
				15384 439468
				15385 436185
				15386 427009
				15387 419360
				15388 476550
				15389 474546
				15390 433991
				15391 432156
				15392 412021
				15393 431495
				15394 430553
				15395 440912
				15396 427982
				15397 430455
				15398 412893
				15399 431772
				15400 416626
				15401 422323
				15402 427685
				15403 423453
				15404 474913
				15405 434474
				15406 423223
				15407 435925
				15408 429909
				15409 427427
				15410 451521
				15411 466206
				15412 450904
				15413 441314
				15414 475742
				15415 478123
				15416 484814
				15417 493721
				15418 483721
				15419 484946
				15420 438220
				15421 433008
				15422 437275
				15423 426387
				15424 434750
				15425 420885
				15426 426938
				15427 414645
				15428 425947
				15429 436685
				15430 440449
				15431 428736
				15432 436070
				15433 434239
				15434 425210
				15435 438397
				15436 451523
				15437 431212
				15438 446358
				15439 440656
				15440 441382
				15441 436470
				15442 449791
				15443 440660
				15444 443836
				15445 431890
				15446 432950
				15447 433186
				15448 478145
				15449 465832
				15450 439460
				15451 437808
				15452 430709
				15453 435089
				15454 433073
				15455 437766
				15456 438632
				15457 429080
				15458 423936
				15459 431261
				15460 424908
				15461 463335
				15462 458767
				15463 479792
				15464 477061
				15465 427108
				15466 431690
				15467 425176
				15468 425366
				15469 428253
				15470 430534
				15471 415148
				15472 429713
				15473 430742
				15474 437384
				15475 430619
				15476 430449
				15477 445904
				15478 441580
				15479 446282
				15480 433605
				15481 443914
				15482 422183
				15483 429594
				15484 423696
				15485 413743
				15486 425230
				15487 422190
				15488 445419
				15489 474687
				15490 454903
				15491 457981
				15492 459184
				15493 489672
				15494 478141
				15495 438728
				15496 437695
				15497 428769
				15498 409018
				15499 420943
				15500 445355
				15501 460651
				15502 437965
				15503 446109
				15504 467368
				15505 468608
				15506 470189
				15507 475697
				15508 482092
				15509 477212
				15510 447154
				15511 489027
				15512 494397
				15513 485385
				15514 487009
				15515 474281
				15516 493088
				15517 483060
				15518 478889
				15519 479283
				15520 476993
				15521 481354
				15522 472926
				15523 483693
				15524 491235
				15525 436135
				15526 427488
				15527 424422
				15528 437330
				15529 434422
				15530 434430
				15531 430148
				15532 441319
				15533 429905
				15534 425403
				15535 438339
				15536 438184
				15537 437660
				15538 449491
				15539 450215
				15540 439017
				15541 443126
				15542 420966
				15543 414048
				15544 414534
				15545 426940
				15546 428424
				15547 436933
				15548 427220
				15549 431982
				15550 424496
				15551 442240
				15552 469186
				15553 473638
				15554 465562
				15555 445875
				15556 434883
				15557 429321
				15558 419962
				15559 424164
				15560 427243
				15561 444124
				15562 435011
				15563 425981
				15564 435912
				15565 467948
				15566 467835
				15567 468251
				15568 462067
				15569 463785
				15570 435483
				15571 434927
				15572 424545
				15573 437330
				15574 423226
				15575 422968
				15576 441814
				15577 419754
				15578 423992
				15579 428085
				15580 477899
				15581 464148
				15582 468403
				15583 470251
				15584 468669
				15585 452720
				15586 485104
				15587 481539
				15588 483548
				15589 480882
				15590 481871
				15591 478126
				15592 476402
				15593 477232
				15594 482708
				15595 473682
				15596 486364
				15597 475199
				15598 477213
				15599 473984
				15600 436974
				15601 428689
				15602 426007
				15603 429481
				15604 436427
				15605 434334
				15606 425826
				15607 434382
				15608 421125
				15609 438027
				15610 427203
				15611 441657
				15612 430430
				15613 434595
				15614 441719
				15615 444734
				15616 431106
				15617 430191
				15618 426188
				15619 451005
				15620 439565
				15621 448435
				15622 439029
				15623 438890
				15624 455643
				15625 437272
				15626 449574
				15627 449131
				15628 481065
				15629 473866
				15630 439946
				15631 506378
				15632 484051
				15633 473536
				15634 481509
				15635 489607
				15636 483759
				15637 472788
				15638 493886
				15639 488524
				15640 484795
				15641 473315
				15642 493807
				15643 477406
				15644 478105
				15645 447311
				15646 430222
				15647 424187
				15648 435459
				15649 469362
				15650 484275
				15651 489700
				15652 484816
				15653 478334
				15654 490670
				15655 482380
				15656 488121
				15657 488984
				15658 479250
				15659 483029
				15660 453293
				15661 434401
				15662 490405
				15663 481504
				15664 476230
				15665 480984
				15666 484644
				15667 478745
				15668 475360
				15669 474475
				15670 496728
				15671 483301
				15672 481866
				15673 477463
				15674 476474
				15675 445409
				15676 440931
				15677 449796
				15678 440945
				15679 441172
				15680 452991
				15681 463866
				15682 455706
				15683 463767
				15684 471417
				15685 460421
				15686 462143
				15687 457483
				15688 470824
				15689 472477
				15690 435214
				15691 447384
				15692 458541
				15693 464976
				15694 469189
				15695 470724
				15696 458774
				15697 465550
				15698 471105
				15699 465363
				15700 480090
				15701 476632
				15702 482341
				15703 475165
				15704 478304
				15705 444519
				15706 428545
				15707 433338
				15708 441491
				15709 434627
				15710 438753
				15711 435868
				15712 444053
				15713 448095
				15714 442884
				15715 436499
				15716 439128
				15717 434206
				15718 452468
				15719 436880
				15720 443413
				15721 479716
				15722 489114
				15723 476739
				15724 475423
				15725 475813
				15726 475183
				15727 465742
				15728 468386
				15729 475705
				15730 491467
				15731 476886
				15732 477041
				15733 481418
				15734 480859
				15735 434806
				15736 432715
				15737 434608
				15738 423819
				15739 437499
				15740 420856
				15741 436073
				15742 438177
				15743 460118
				15744 464187
				15745 456933
				15746 454274
				15747 460321
				15748 460821
				15749 457110
				15750 435939
				15751 433111
				15752 437144
				15753 463696
				15754 462979
				15755 470607
				15756 476328
				15757 476293
				15758 469455
				15759 464701
				15760 457632
				15761 485009
				15762 484212
				15763 470524
				15764 462090
				15765 445398
				15766 435198
				15767 420674
				15768 421854
				15769 423826
				15770 425901
				15771 421446
				15772 422874
				15773 426954
				15774 419476
				15775 423748
				15776 440153
				15777 427783
				15778 429112
				15779 428379
				15780 440787
				15781 436114
				15782 430510
				15783 424621
				15784 418130
				15785 408059
				15786 428056
				15787 415118
				15788 411905
				15789 430298
				15790 413969
				15791 425002
				15792 433254
				15793 453909
				15794 457315
				15795 442979
				15796 430481
				15797 420971
				15798 446486
				15799 435378
				15800 424929
				15801 464169
				15802 463095
				15803 462339
				15804 458229
				15805 446158
				15806 479559
				15807 475206
				15808 486176
				15809 478133
				15810 426693
				15811 429720
				15812 429375
				15813 413633
				15814 407008
				15815 417475
				15816 419391
				15817 413101
				15818 417128
				15819 460911
				15820 483090
				15821 478035
				15822 464240
				15823 453939
				15824 457461
				15825 442186
				15826 429584
				15827 415969
				15828 415761
				15829 422633
				15830 422127
				15831 425238
				15832 430075
				15833 429544
				15834 428262
				15835 426374
				15836 439666
				15837 425549
				15838 437752
				15839 431817
				15840 435858
				15841 435782
				15842 434081
				15843 430558
				15844 431378
				15845 432668
				15846 429639
				15847 439383
				15848 443911
				15849 423972
				15850 439842
				15851 421427
				15852 424232
				15853 433203
				15854 427200
				15855 435692
				15856 423539
				15857 420696
				15858 415334
				15859 418497
				15860 431404
				15861 417024
				15862 418378
				15863 418509
				15864 413207
				15865 425461
				15866 419315
				15867 421724
				15868 423062
				15869 422046
				15870 432022
				15871 420779
				15872 418622
				15873 412993
				15874 428170
				15875 461122
				15876 451429
				15877 451202
				15878 460030
				15879 453152
				15880 485916
				15881 481694
				15882 490972
				15883 486646
				15884 498603
				15885 451647
				15886 424650
				15887 417450
				15888 436581
				15889 438609
				15890 420111
				15891 425707
				15892 423075
				15893 422187
				15894 436408
				15895 430317
				15896 436296
				15897 422336
				15898 433241
				15899 431451
				15900 440822
				15901 420262
				15902 416108
				15903 426891
				15904 432702
				15905 430974
				15906 437625
				15907 430006
				15908 430225
				15909 456056
				15910 465633
				15911 470571
				15912 465644
				15913 464596
				15914 485598
				15915 422962
				15916 429638
				15917 424854
				15918 420259
				15919 419266
				15920 424446
				15921 436327
				15922 407171
				15923 422541
				15924 419897
				15925 473592
				15926 468958
				15927 488766
				15928 463959
				15929 471319
				15930 436478
				15931 467061
				15932 461674
				15933 464648
				15934 468813
				15935 469864
				15936 462329
				15937 473947
				15938 482793
				15939 469839
				15940 474196
				15941 475322
				15942 494602
				15943 489155
				15944 482897
				15945 442703
				15946 429786
				15947 426987
				15948 424835
				15949 416460
				15950 456438
				15951 468657
				15952 464372
				15953 471326
				15954 480147
				15955 475533
				15956 475885
				15957 468947
				15958 495918
				15959 467039
				15960 432228
				15961 433681
				15962 421597
				15963 427044
				15964 419935
				15965 423347
				15966 486326
				15967 474559
				15968 481648
				15969 477409
				15970 483386
				15971 476740
				15972 476018
				15973 476854
				15974 480701
				15975 427264
				15976 430657
				15977 425275
				15978 410265
				15979 402596
				15980 421651
				15981 417838
				15982 431919
				15983 421494
				15984 418343
				15985 415298
				15986 429530
				15987 434034
				15988 424697
				15989 425054
				15990 428422
				15991 422165
				15992 426625
				15993 473419
				15994 466271
				15995 460281
				15996 488690
				15997 472677
				15998 483249
				15999 476400
				16000 475174
				16001 477824
				16002 475354
				16003 474034
				16004 477230
				16005 419855
				16006 433078
				16007 422180
				16008 427386
				16009 408936
				16010 414982
				16011 419750
				16012 423973
				16013 418335
				16014 459537
				16015 451336
				16016 452522
				16017 461285
				16018 457037
				16019 458368
				16020 431193
				16021 427774
				16022 423073
				16023 425619
				16024 413837
				16025 412652
				16026 464948
				16027 452998
				16028 473993
				16029 474985
				16030 466027
				16031 464507
				16032 464056
				16033 463064
				16034 470808
				16035 434651
				16036 421542
				16037 411425
				16038 461426
				16039 475596
				16040 469109
				16041 493632
				16042 478484
				16043 476934
				16044 473841
				16045 480644
				16046 485042
				16047 481921
				16048 461405
				16049 458657
				16050 434602
				16051 423286
				16052 415642
				16053 417672
				16054 415658
				16055 410927
				16056 418279
				16057 426034
				16058 425106
				16059 424914
				16060 416920
				16061 425547
				16062 435988
				16063 461666
				16064 455369
				16065 428056
				16066 418472
				16067 423348
				16068 419879
				16069 423442
				16070 423365
				16071 421693
				16072 419081
				16073 446096
				16074 442488
				16075 447091
				16076 461712
				16077 460859
				16078 458713
				16079 460101
				16080 439670
				16081 447241
				16082 413994
				16083 462347
				16084 460819
				16085 460204
				16086 456835
				16087 458044
				16088 462620
				16089 467997
				16090 463200
				16091 479804
				16092 483373
				16093 471659
				16094 472207
				16095 434670
				16096 419637
				16097 419705
				16098 417925
				16099 418398
				16100 411705
				16101 412446
				16102 411416
				16103 415227
				16104 407263
				16105 424607
				16106 440548
				16107 433299
				16108 450781
				16109 439611
				16110 433588
				16111 423464
				16112 409076
				16113 409323
				16114 418307
				16115 423502
				16116 460325
				16117 455427
				16118 455311
				16119 451843
				16120 458437
				16121 458208
				16122 455300
				16123 451315
				16124 447380
				16125 435671
				16126 425490
				16127 420979
				16128 414316
				16129 427099
				16130 414608
				16131 418668
				16132 423651
				16133 426305
				16134 456445
				16135 472939
				16136 465765
				16137 461576
				16138 456092
				16139 459113
				16140 444265
				16141 421461
				16142 445267
				16143 442152
				16144 457305
				16145 447266
				16146 460467
				16147 455478
				16148 447072
				16149 446043
				16150 448839
				16151 454828
				16152 444021
				16153 444866
				16154 448299
				16155 424365
				16156 429869
				16157 428650
				16158 417780
				16159 450993
				16160 468301
				16161 467629
				16162 465716
				16163 478311
				16164 484655
				16165 472872
				16166 477853
				16167 463751
				16168 475715
				16169 474279
				16170 435423
				16171 428701
				16172 413426
				16173 417203
				16174 412373
				16175 412296
				16176 417840
				16177 451341
				16178 471460
				16179 472185
				16180 476563
				16181 473398
				16182 466795
				16183 474594
				16184 457255
				16185 425218
				16186 414979
				16187 413527
				16188 409660
				16189 411593
				16190 410549
				16191 472755
				16192 473157
				16193 474956
				16194 470635
				16195 465455
				16196 467077
				16197 464729
				16198 473999
				16199 475245
				16200 437170
				16201 467286
				16202 468676
				16203 468843
				16204 476005
				16205 471210
				16206 465341
				16207 458889
				16208 474116
				16209 469525
				16210 475653
				16211 465283
				16212 462019
				16213 466181
				16214 474039
				16215 428366
				16216 421679
				16217 413432
				16218 409846
				16219 465621
				16220 475512
				16221 479444
				16222 472034
				16223 468312
				16224 463649
				16225 475030
				16226 467083
				16227 489467
				16228 468688
				16229 457201
				16230 438356
				16231 422430
				16232 424762
				16233 418027
				16234 419798
				16235 430775
				16236 458338
				16237 465321
				16238 461147
				16239 458851
				16240 457926
				16241 457900
				16242 463805
				16243 465278
				16244 476125
				16245 427041
				16246 420416
				16247 423156
				16248 413204
				16249 411159
				16250 459066
				16251 454679
				16252 450342
				16253 456637
				16254 452072
				16255 454697
				16256 453189
				16257 459549
				16258 455543
				16259 456748
				16260 428482
				16261 444749
				16262 420461
				16263 465219
				16264 459360
				16265 451944
				16266 457054
				16267 455089
				16268 453555
				16269 463721
				16270 464623
				16271 455495
				16272 479589
				16273 487758
				16274 478563
				16275 433181
				16276 434285
				16277 417257
				16278 421240
				16279 421922
				16280 421837
				16281 414515
				16282 429799
				16283 419588
				16284 432095
				16285 421541
				16286 419512
				16287 454220
				16288 460977
				16289 449854
				16290 434383
				16291 419394
				16292 413217
				16293 423980
				16294 412348
				16295 464516
				16296 472472
				16297 458699
				16298 472005
				16299 470693
				16300 454993
				16301 468665
				16302 455929
				16303 455947
				16304 456585
				16305 426066
				16306 419017
				16307 420868
				16308 420917
				16309 416543
				16310 418147
				16311 423732
				16312 450440
				16313 458621
				16314 466594
				16315 462827
				16316 465544
				16317 458410
				16318 454366
				16319 461154
				16320 434911
				16321 426886
				16322 417699
				16323 420046
				16324 421989
				16325 422078
				16326 419456
				16327 413908
				16328 455596
				16329 446525
				16330 456508
				16331 452093
				16332 452731
				16333 441883
				16334 447924
				16335 442781
				16336 420991
				16337 415308
				16338 413803
				16339 414358
				16340 419187
				16341 468794
				16342 469004
				16343 482126
				16344 479058
				16345 478428
				16346 473593
				16347 468570
				16348 476885
				16349 478031
				16350 431460
				16351 426531
				16352 463707
				16353 469242
				16354 467285
				16355 466815
				16356 469698
				16357 456793
				16358 462386
				16359 475165
				16360 490544
				16361 488427
				16362 475631
				16363 473954
				16364 485140
				16365 429280
				16366 429852
				16367 472641
				16368 457856
				16369 467391
				16370 462307
				16371 469151
				16372 457068
				16373 458541
				16374 455848
				16375 463245
				16376 456885
				16377 449221
				16378 457055
				16379 469029
				16380 433784
				16381 432338
				16382 419655
				16383 428165
				16384 417857
				16385 428412
				16386 426634
				16387 430403
				16388 439142
				16389 430997
				16390 434720
				16391 423023
				16392 432265
				16393 425485
				16394 447871
				16395 442420
				16396 434724
				16397 409637
				16398 471415
				16399 470181
				16400 481134
				16401 467060
				16402 462560
				16403 455087
				16404 461147
				16405 456285
				16406 461612
				16407 451637
				16408 444745
				16409 472454
				16410 417189
				16411 416339
				16412 418613
				16413 466110
				16414 466289
				16415 478028
				16416 485597
				16417 484373
				16418 484965
				16419 482688
				16420 490265
				16421 488080
				16422 482800
				16423 482897
				16424 487798
				16425 431454
				16426 421975
				16427 421157
				16428 422901
				16429 415877
				16430 427463
				16431 428464
				16432 423152
				16433 428994
				16434 430502
				16435 424116
				16436 445064
				16437 429727
				16438 437098
				16439 464525
				16440 430304
				16441 438329
				16442 418576
				16443 425233
				16444 415983
				16445 420140
				16446 422461
				16447 416788
				16448 425158
				16449 454245
				16450 458307
				16451 459725
				16452 464125
				16453 458342
				16454 473228
				16455 445600
				16456 431193
				16457 427105
				16458 428892
				16459 428248
				16460 426398
				16461 443192
				16462 428863
				16463 427256
				16464 426767
				16465 432894
				16466 440075
				16467 422662
				16468 437489
				16469 444131
				16470 436941
				16471 414839
				16472 437327
				16473 430415
				16474 430431
				16475 433233
				16476 421391
				16477 427888
				16478 436908
				16479 457127
				16480 482299
				16481 480519
				16482 481385
				16483 485162
				16484 497102
				16485 444365
				16486 434810
				16487 425749
				16488 424777
				16489 429625
				16490 436351
				16491 461129
				16492 463361
				16493 458473
				16494 455860
				16495 465181
				16496 467131
				16497 483769
				16498 475965
				16499 475065
				16500 430066
				16501 434462
				16502 416610
				16503 429984
				16504 435170
				16505 423916
				16506 417119
				16507 463308
				16508 457721
				16509 456024
				16510 479238
				16511 481037
				16512 476156
				16513 476694
				16514 491475
				16515 427823
				16516 435306
				16517 433288
				16518 432300
				16519 441477
				16520 431928
				16521 432444
				16522 450763
				16523 436220
				16524 448638
				16525 473021
				16526 482823
				16527 473335
				16528 483680
				16529 480715
				16530 436443
				16531 435859
				16532 428210
				16533 434221
				16534 470987
				16535 474341
				16536 459039
				16537 470966
				16538 462173
				16539 466297
				16540 469108
				16541 484327
				16542 470529
				16543 469113
				16544 477033
				16545 437692
				16546 432429
				16547 420081
				16548 424329
				16549 417284
				16550 409706
				16551 408995
				16552 460326
				16553 466997
				16554 455890
				16555 459754
				16556 453277
				16557 461130
				16558 459973
				16559 452429
				16560 429406
				16561 487065
				16562 468877
				16563 468728
				16564 457224
				16565 463938
				16566 468506
				16567 464005
				16568 468634
				16569 460948
				16570 456879
				16571 466770
				16572 462111
				16573 461774
				16574 457961
				16575 440166
				16576 434777
				16577 420785
				16578 417010
				16579 468101
				16580 454981
				16581 454851
				16582 456987
				16583 451077
				16584 456504
				16585 454284
				16586 454067
				16587 456200
				16588 466124
				16589 463056
				16590 430249
				16591 429831
				16592 425614
				16593 418964
				16594 423689
				16595 420776
				16596 471404
				16597 459125
				16598 469755
				16599 460084
				16600 465585
				16601 478299
				16602 464576
				16603 474033
				16604 455611
				16605 435663
				16606 419368
				16607 424388
				16608 410754
				16609 453571
				16610 473781
				16611 476143
				16612 466430
				16613 483444
				16614 467524
				16615 473350
				16616 470511
				16617 469544
				16618 460409
				16619 474329
				16620 448088
				16621 430451
				16622 409411
				16623 416316
				16624 428125
				16625 432895
				16626 439388
				16627 429174
				16628 486285
				16629 486430
				16630 488173
				16631 481785
				16632 486909
				16633 487142
				16634 485804
				16635 429187
				16636 445176
				16637 427220
				16638 420120
				16639 422283
				16640 426870
				16641 446115
				16642 435521
				16643 429766
				16644 437227
				16645 422731
				16646 419523
				16647 427367
				16648 424726
				16649 428497
				16650 427558
				16651 437518
				16652 435507
				16653 433066
				16654 423185
				16655 434702
				16656 440261
				16657 431496
				16658 425157
				16659 436119
				16660 433697
				16661 422535
				16662 423848
				16663 428859
				16664 477972
				16665 437187
				16666 437772
				16667 463789
				16668 449216
				16669 466050
				16670 463413
				16671 466578
				16672 448414
				16673 462430
				16674 461451
				16675 469150
				16676 466913
				16677 467926
				16678 464355
				16679 470589
				16680 431287
				16681 428951
				16682 424649
				16683 423450
				16684 437597
				16685 431116
				16686 419661
				16687 431215
				16688 431188
				16689 432331
				16690 432129
				16691 428661
				16692 426788
				16693 425592
				16694 432283
				16695 439520
				16696 489608
				16697 472880
				16698 454011
				16699 462449
				16700 468066
				16701 468166
				16702 466664
				16703 467957
				16704 478372
				16705 481182
				16706 477348
				16707 461455
				16708 482214
				16709 464134
				16710 435997
				16711 429849
				16712 413783
				16713 425513
				16714 418530
				16715 421000
				16716 430036
				16717 427725
				16718 434548
				16719 418124
				16720 436661
				16721 427258
				16722 429426
				16723 423174
				16724 428316
				16725 436824
				16726 430909
				16727 424093
				16728 468099
				16729 471256
				16730 455540
				16731 465167
				16732 464737
				16733 449922
				16734 460045
				16735 464452
				16736 472211
				16737 478019
				16738 487170
				16739 480746
				16740 434242
				16741 437873
				16742 429629
				16743 435069
				16744 435835
				16745 432455
				16746 430239
				16747 433912
				16748 446649
				16749 435549
				16750 448848
				16751 430791
				16752 442954
				16753 438639
				16754 444765
				16755 429559
				16756 427801
				16757 420114
				16758 420819
				16759 462841
				16760 475582
				16761 462407
				16762 457925
				16763 472921
				16764 452099
				16765 456378
				16766 463670
				16767 468493
				16768 474415
				16769 456383
				16770 440786
				16771 425617
				16772 414122
				16773 427034
				16774 412132
				16775 415080
				16776 466380
				16777 458143
				16778 461171
				16779 458567
				16780 470997
				16781 465249
				16782 460033
				16783 460119
				16784 485842
				16785 434509
				16786 417403
				16787 424153
				16788 422146
				16789 431655
				16790 432722
				16791 427086
				16792 428573
				16793 433192
				16794 429435
				16795 435369
				16796 426930
				16797 424326
				16798 441980
				16799 471816
				16800 435490
				16801 431012
				16802 427283
				16803 423752
				16804 426520
				16805 431085
				16806 429649
				16807 424770
				16808 431872
				16809 423858
				16810 429449
				16811 429504
				16812 436222
				16813 426161
				16814 438263
				16815 437615
				16816 424445
				16817 438077
				16818 434370
				16819 438403
				16820 449880
				16821 439773
				16822 455596
				16823 445480
				16824 469468
				16825 475516
				16826 467808
				16827 485611
				16828 477471
				16829 487949
				16830 430892
				16831 434338
				16832 432917
				16833 434778
				16834 437814
				16835 431117
				16836 466806
				16837 469146
				16838 473717
				16839 467171
				16840 473693
				16841 467753
				16842 465065
				16843 476857
				16844 494674
				16845 431373
				16846 426263
				16847 433234
				16848 445595
				16849 425733
				16850 467722
				16851 466908
				16852 480911
				16853 479969
				16854 478415
				16855 476753
				16856 459642
				16857 468847
				16858 479373
				16859 477771
				16860 438677
				16861 433354
				16862 431546
				16863 424437
				16864 460385
				16865 461815
				16866 457988
				16867 445897
				16868 451626
				16869 446917
				16870 494043
				16871 477290
				16872 471525
				16873 482349
				16874 480965
				16875 433196
				16876 439851
				16877 434889
				16878 428844
				16879 435687
				16880 432317
				16881 428564
				16882 421282
				16883 442307
				16884 424602
				16885 462401
				16886 452227
				16887 461734
				16888 464148
				16889 466574
				16890 444180
				16891 479851
				16892 462922
				16893 474326
				16894 457018
				16895 455311
				16896 471647
				16897 471615
				16898 452025
				16899 476010
				16900 470244
				16901 476971
				16902 472497
				16903 484260
				16904 467835
				16905 440394
				16906 443800
				16907 421433
				16908 440355
				16909 428432
				16910 441999
				16911 437402
				16912 467457
				16913 486917
				16914 480899
				16915 480411
				16916 478570
				16917 507824
				16918 497089
				16919 478965
				16920 448113
				16921 420689
				16922 424913
				16923 417385
				16924 451154
				16925 453756
				16926 444771
				16927 444760
				16928 471624
				16929 477505
				16930 483412
				16931 469515
				16932 476980
				16933 484139
				16934 485365
				16935 450416
				16936 441817
				16937 450763
				16938 484654
				16939 477351
				16940 483535
				16941 480798
				16942 497198
				16943 487687
				16944 488461
				16945 489116
				16946 487218
				16947 496211
				16948 494460
				16949 494178
				16950 443785
				16951 441840
				16952 430686
				16953 443435
				16954 476530
				16955 480669
				16956 477355
				16957 481359
				16958 474233
				16959 472832
				16960 481587
				16961 473913
				16962 478399
				16963 479147
				16964 474065
				16965 447135
				16966 426964
				16967 424363
				16968 416297
				16969 448488
				16970 462747
				16971 477831
				16972 476386
				16973 467140
				16974 471321
				16975 456339
				16976 471682
				16977 460320
				16978 475372
				16979 465692
				16980 442697
				16981 443527
				16982 429193
				16983 427126
				16984 431200
				16985 422360
				16986 427473
				16987 432216
				16988 431601
				16989 427924
				16990 447464
				16991 467880
				16992 478158
				16993 474457
				16994 472275
				16995 440351
				16996 453500
				16997 461089
				16998 462538
				16999 469775
				17000 461117
				17001 453968
				17002 462812
				17003 462081
				17004 453545
				17005 467297
				17006 459997
				17007 466415
				17008 460410
				17009 461838
				17010 454844
				17011 440773
				17012 434425
				17013 480294
				17014 481990
				17015 486537
				17016 483375
				17017 475283
				17018 491900
				17019 490438
				17020 486560
				17021 484952
				17022 478027
				17023 488952
				17024 485354
				17025 455678
				17026 441068
				17027 431394
				17028 426521
				17029 436465
				17030 436253
				17031 430281
				17032 438166
				17033 447260
				17034 438381
				17035 425620
				17036 436011
				17037 431462
				17038 445392
				17039 432528
				17040 444700
				17041 453732
				17042 444054
				17043 431700
				17044 440869
				17045 457112
				17046 465586
				17047 454564
				17048 454346
				17049 465202
				17050 461714
				17051 471376
				17052 462138
				17053 452996
				17054 466942
				17055 450367
				17056 456990
				17057 435510
				17058 446864
				17059 433541
				17060 435416
				17061 425819
				17062 430062
				17063 435209
				17064 446785
				17065 432823
				17066 432166
				17067 440930
				17068 426459
				17069 436159
				17070 454235
				17071 452800
				17072 442964
				17073 436126
				17074 434035
				17075 440101
				17076 438441
				17077 436763
				17078 452697
				17079 432856
				17080 433005
				17081 457467
				17082 481425
				17083 472548
				17084 488905
				17085 458619
				17086 450212
				17087 440814
				17088 436297
				17089 472209
				17090 482423
				17091 485469
				17092 488204
				17093 490240
				17094 493847
				17095 497405
				17096 495693
				17097 500289
				17098 505715
				17099 509173
				17100 451806
				17101 440030
				17102 434952
				17103 439912
				17104 422101
				17105 439370
				17106 425062
				17107 434974
				17108 445578
				17109 442346
				17110 418607
				17111 430780
				17112 429992
				17113 436667
				17114 444881
				17115 468456
				17116 453370
				17117 481984
				17118 479997
				17119 478334
				17120 462713
				17121 477504
				17122 473818
				17123 469451
				17124 464049
				17125 471589
				17126 475640
				17127 467721
				17128 473012
				17129 463372
				17130 456891
				17131 436627
				17132 430132
				17133 424340
				17134 425622
				17135 427327
				17136 459829
				17137 462870
				17138 483037
				17139 467274
				17140 463121
				17141 468051
				17142 473902
				17143 462917
				17144 479900
				17145 447842
				17146 444201
				17147 429995
				17148 440510
				17149 424685
				17150 471069
				17151 465615
				17152 459218
				17153 469257
				17154 467390
				17155 466176
				17156 461985
				17157 481006
				17158 475506
				17159 476770
				17160 441600
				17161 429218
				17162 426250
				17163 475817
				17164 470602
				17165 473245
				17166 471355
				17167 467654
				17168 474236
				17169 471271
				17170 483308
				17171 465980
				17172 476315
				17173 471062
				17174 467769
				17175 453154
				17176 469691
				17177 483938
				17178 483736
				17179 474356
				17180 482999
				17181 473532
				17182 476203
				17183 476900
				17184 476691
				17185 488531
				17186 475951
				17187 478058
				17188 482239
				17189 484975
				17190 451902
				17191 445178
				17192 435616
				17193 435756
				17194 444155
				17195 443157
				17196 446135
				17197 467945
				17198 469944
				17199 463787
				17200 477205
				17201 467907
				17202 473101
				17203 457380
				17204 476657
				17205 450898
				17206 450013
				17207 436923
				17208 422253
				17209 420078
				17210 426172
				17211 436406
				17212 443223
				17213 433695
				17214 434989
				17215 436847
				17216 437685
				17217 447461
				17218 435429
				17219 474202
				17220 445911
				17221 435397
				17222 432351
				17223 431870
				17224 496338
				17225 481107
				17226 475694
				17227 483908
				17228 485202
				17229 491380
				17230 478468
				17231 477865
				17232 481075
				17233 474261
				17234 477688
				17235 444791
				17236 438577
				17237 431315
				17238 422225
				17239 440626
				17240 424580
				17241 428692
				17242 433284
				17243 439304
				17244 491924
				17245 481839
				17246 488631
				17247 483957
				17248 489950
				17249 484583
				17250 442212
				17251 438042
				17252 429436
				17253 431976
				17254 455892
				17255 451742
				17256 451304
				17257 449332
				17258 442959
				17259 453658
				17260 441571
				17261 451101
				17262 445264
				17263 449504
				17264 447498
				17265 442856
				17266 433641
				17267 426475
				17268 433707
				17269 422918
				17270 442525
				17271 433439
				17272 422292
				17273 431141
				17274 432117
				17275 443013
				17276 443488
				17277 447436
				17278 435201
				17279 469568
				17280 432291
				17281 433213
				17282 429510
				17283 420831
				17284 425396
				17285 430692
				17286 429512
				17287 425058
				17288 420245
				17289 467645
				17290 487272
				17291 466453
				17292 479550
				17293 464084
				17294 465696
				17295 447762
				17296 451295
				17297 420094
				17298 421824
				17299 426908
				17300 433839
				17301 437291
				17302 431977
				17303 428435
				17304 445762
				17305 444744
				17306 437198
				17307 441339
				17308 428128
				17309 458240
				17310 436778
				17311 485716
				17312 482704
				17313 469010
				17314 473341
				17315 472239
				17316 466445
				17317 476651
				17318 477415
				17319 514344
				17320 496907
				17321 488203
				17322 495626
				17323 505266
				17324 495529
				17325 442703
				17326 484190
				17327 470176
				17328 471952
				17329 478956
				17330 469421
				17331 473352
				17332 480620
				17333 475364
				17334 469341
				17335 476044
				17336 479943
				17337 485387
				17338 477958
				17339 463528
				17340 448393
				17341 440137
				17342 429231
				17343 437995
				17344 440371
				17345 427222
				17346 424068
				17347 434809
				17348 463383
				17349 454287
				17350 473607
				17351 450794
				17352 466059
				17353 480328
				17354 462088
				17355 441985
				17356 427078
				17357 471474
				17358 470424
				17359 464634
				17360 468875
				17361 471254
				17362 458869
				17363 463686
				17364 460227
				17365 456148
				17366 455725
				17367 458365
				17368 460245
				17369 453595
				17370 442011
				17371 480404
				17372 475991
				17373 477654
				17374 461562
				17375 465376
				17376 490582
				17377 477693
				17378 481819
				17379 475371
				17380 485712
				17381 480180
				17382 480940
				17383 484588
				17384 479339
				17385 452113
				17386 436474
				17387 415892
				17388 415550
				17389 416578
				17390 466354
				17391 481220
				17392 473392
				17393 487273
				17394 486978
				17395 479053
				17396 483793
				17397 465833
				17398 482259
				17399 485030
				17400 440109
				17401 434410
				17402 426377
				17403 463472
				17404 461309
				17405 467624
				17406 480768
				17407 484260
				17408 473349
				17409 478924
				17410 468344
				17411 491997
				17412 483068
				17413 493966
				17414 476702
				17415 442040
				17416 430717
				17417 476101
				17418 496019
				17419 465954
				17420 483418
				17421 475929
				17422 483415
				17423 479022
				17424 478484
				17425 484369
				17426 471669
				17427 483204
				17428 472377
				17429 478453
				17430 438239
				17431 443074
				17432 473632
				17433 467625
				17434 476667
				17435 467198
				17436 471938
				17437 493031
				17438 495819
				17439 505139
				17440 499190
				17441 489001
				17442 496033
				17443 487839
				17444 499328
				17445 439874
				17446 437246
				17447 436507
				17448 441510
				17449 431337
				17450 446273
				17451 440485
				17452 480077
				17453 472720
				17454 464623
				17455 473128
				17456 474450
				17457 455619
				17458 473664
				17459 468321
				17460 443363
				17461 440768
				17462 422772
				17463 416891
				17464 413894
				17465 438127
				17466 440409
				17467 438552
				17468 422338
				17469 435408
				17470 434670
				17471 429770
				17472 445114
				17473 440714
				17474 434311
				17475 433472
				17476 421399
				17477 425405
				17478 424140
				17479 424490
				17480 469160
				17481 459444
				17482 465083
				17483 457876
				17484 471974
				17485 461054
				17486 474323
				17487 457901
				17488 472350
				17489 472621
				17490 438385
				17491 442199
				17492 429736
				17493 430741
				17494 415336
				17495 424762
				17496 436279
				17497 423207
				17498 432311
				17499 431920
				17500 419138
				17501 425194
				17502 432505
				17503 428802
				17504 444797
				17505 437033
				17506 428671
				17507 415598
				17508 416309
				17509 423414
				17510 434866
				17511 416108
				17512 418984
				17513 464071
				17514 475875
				17515 471039
				17516 470676
				17517 481221
				17518 469712
				17519 463599
				17520 432516
				17521 432299
				17522 437047
				17523 426398
				17524 426686
				17525 476590
				17526 472087
				17527 462597
				17528 469709
				17529 454239
				17530 455906
				17531 457630
				17532 463888
				17533 459323
				17534 453542
				17535 422581
				17536 478790
				17537 482266
				17538 473978
				17539 470293
				17540 471244
				17541 475280
				17542 477273
				17543 458613
				17544 464264
				17545 463054
				17546 472771
				17547 458875
				17548 466792
				17549 479086
				17550 438782
				17551 434739
				17552 419135
				17553 468127
				17554 470870
				17555 470302
				17556 466261
				17557 457880
				17558 472792
				17559 468212
				17560 457126
				17561 464019
				17562 464278
				17563 468508
				17564 466072
				17565 437890
				17566 431841
				17567 425447
				17568 423345
				17569 422669
				17570 428103
				17571 423919
				17572 432632
				17573 433805
				17574 434594
				17575 425452
				17576 440336
				17577 415881
				17578 424997
				17579 435335
				17580 425658
				17581 421509
				17582 445102
				17583 444845
				17584 425704
				17585 460073
				17586 457141
				17587 464204
				17588 462367
				17589 461098
				17590 476407
				17591 477304
				17592 484743
				17593 482805
				17594 497637
				17595 436829
				17596 426267
				17597 419110
				17598 416965
				17599 425937
				17600 442806
				17601 435766
				17602 454196
				17603 458936
				17604 453426
				17605 472349
				17606 461768
				17607 469739
				17608 469939
				17609 455392
				17610 444643
				17611 414739
				17612 426511
				17613 436294
				17614 420010
				17615 422517
				17616 425058
				17617 417933
				17618 432929
				17619 428216
				17620 476693
				17621 473062
				17622 467760
				17623 470593
				17624 468400
				17625 442606
				17626 428034
				17627 418160
				17628 417066
				17629 429460
				17630 430474
				17631 436556
				17632 463726
				17633 459830
				17634 465650
				17635 462212
				17636 465523
				17637 463825
				17638 456135
				17639 463174
				17640 434968
				17641 425830
				17642 467098
				17643 468898
				17644 476258
				17645 468492
				17646 465663
				17647 457104
				17648 448368
				17649 462380
				17650 455201
				17651 459671
				17652 461443
				17653 452481
				17654 468447
				17655 436974
				17656 430387
				17657 427015
				17658 411623
				17659 416269
				17660 413966
				17661 427238
				17662 410139
				17663 421166
				17664 475029
				17665 473530
				17666 472627
				17667 469452
				17668 465940
				17669 462467
				17670 428038
				17671 422990
				17672 413624
				17673 426971
				17674 414755
				17675 408079
				17676 410718
				17677 410269
				17678 443125
				17679 431876
				17680 437042
				17681 450391
				17682 453633
				17683 464647
				17684 480650
				17685 431514
				17686 460267
				17687 456341
				17688 471127
				17689 449311
				17690 474798
				17691 471581
				17692 466095
				17693 475460
				17694 472453
				17695 459471
				17696 471676
				17697 455842
				17698 462717
				17699 459469
				17700 425034
				17701 422382
				17702 422603
				17703 429234
				17704 435546
				17705 427838
				17706 438037
				17707 434832
				17708 441000
				17709 417015
				17710 427167
				17711 424841
				17712 430739
				17713 425467
				17714 435375
				17715 428767
				17716 417166
				17717 441252
				17718 430903
				17719 432342
				17720 428670
				17721 435594
				17722 429330
				17723 425724
				17724 429331
				17725 424961
				17726 421919
				17727 424203
				17728 426197
				17729 459072
				17730 432708
				17731 431332
				17732 430989
				17733 423995
				17734 433681
				17735 427492
				17736 430049
				17737 424253
				17738 432351
				17739 431991
				17740 468212
				17741 455512
				17742 485877
				17743 476896
				17744 477484
				17745 432623
				17746 426587
				17747 413849
				17748 419407
				17749 459490
				17750 457666
				17751 448415
				17752 456979
				17753 456708
				17754 456600
				17755 457855
				17756 461875
				17757 455085
				17758 457200
				17759 461363
				17760 434915
				17761 431029
				17762 445101
				17763 437487
				17764 429583
				17765 434225
				17766 437851
				17767 434571
				17768 447631
				17769 470208
				17770 466327
				17771 475451
				17772 465379
				17773 483012
				17774 468067
				17775 435501
				17776 426741
				17777 437772
				17778 457112
				17779 453265
				17780 433802
				17781 444786
				17782 448514
				17783 437230
				17784 439081
				17785 438003
				17786 438953
				17787 441985
				17788 438817
				17789 442834
				17790 434224
				17791 426759
				17792 429377
				17793 425049
				17794 424089
				17795 411388
				17796 428990
				17797 471345
				17798 483288
				17799 482501
				17800 466187
				17801 463170
				17802 448021
				17803 474713
				17804 470460
				17805 446700
				17806 430489
				17807 418474
				17808 416908
				17809 427785
				17810 436286
				17811 469975
				17812 469476
				17813 464421
				17814 458494
				17815 470368
				17816 476916
				17817 458409
				17818 490666
				17819 480948
				17820 434356
				17821 430343
				17822 435665
				17823 419684
				17824 419733
				17825 479353
				17826 471841
				17827 468208
				17828 477768
				17829 480943
				17830 466243
				17831 467351
				17832 475263
				17833 474548
				17834 475750
				17835 436128
				17836 447519
				17837 424224
				17838 423724
				17839 412974
				17840 417192
				17841 418976
				17842 432308
				17843 420484
				17844 432454
				17845 484818
				17846 479031
				17847 467109
				17848 470998
				17849 468002
				17850 445062
				17851 462619
				17852 460233
				17853 464458
				17854 456615
				17855 449045
				17856 476163
				17857 464466
				17858 460067
				17859 469607
				17860 478157
				17861 472085
				17862 476852
				17863 474439
				17864 493522
				17865 444236
				17866 431833
				17867 428977
				17868 424541
				17869 423745
				17870 423548
				17871 421813
				17872 441177
				17873 423070
				17874 422824
				17875 423580
				17876 409013
				17877 421855
				17878 423106
				17879 419135
				17880 432664
				17881 425532
				17882 435820
				17883 418612
				17884 437171
				17885 414778
				17886 420436
				17887 424831
				17888 426927
				17889 426410
				17890 413180
				17891 420158
				17892 419015
				17893 414546
				17894 420926
				17895 434729
				17896 435281
				17897 436691
				17898 432347
				17899 443790
				17900 436209
				17901 447525
				17902 439918
				17903 438669
				17904 436796
				17905 437631
				17906 436146
				17907 437716
				17908 438546
				17909 443919
				17910 427115
				17911 436224
				17912 429587
				17913 437570
				17914 441696
				17915 441952
				17916 440171
				17917 461610
				17918 450847
				17919 445065
				17920 441681
				17921 444005
				17922 444901
				17923 442494
				17924 450375
				17925 442159
				17926 433664
				17927 423532
				17928 419682
				17929 420745
				17930 421815
				17931 423285
				17932 467576
				17933 477861
				17934 471077
				17935 484373
				17936 462814
				17937 477869
				17938 455595
				17939 454427
				17940 449077
				17941 472072
				17942 479506
				17943 469454
				17944 472728
				17945 464219
				17946 477968
				17947 479142
				17948 479319
				17949 474057
				17950 468177
				17951 473382
				17952 486086
				17953 476677
				17954 485937
				17955 437488
				17956 439354
				17957 432095
				17958 432793
				17959 429821
				17960 440156
				17961 430125
				17962 435557
				17963 437051
				17964 438894
				17965 428645
				17966 431683
				17967 434521
				17968 434475
				17969 426852
				17970 437723
				17971 444340
				17972 423709
				17973 423973
				17974 431647
				17975 418770
				17976 413722
				17977 420161
				17978 439820
				17979 429061
				17980 475805
				17981 478595
				17982 471660
				17983 484360
				17984 466897
				17985 441829
				17986 428801
				17987 424353
				17988 441739
				17989 438646
				17990 438838
				17991 430990
				17992 438448
				17993 432350
				17994 434359
				17995 445364
				17996 432410
				17997 431811
				17998 433734
				17999 442847
				18000 436154
				18001 446349
				18002 490167
				18003 485068
				18004 469679
				18005 472218
				18006 480983
				18007 474511
				18008 476673
				18009 474733
				18010 464919
				18011 483841
				18012 479862
				18013 490821
				18014 489107
				18015 454798
				18016 432545
				18017 422087
				18018 427244
				18019 435365
				18020 430385
				18021 425695
				18022 436008
				18023 421971
				18024 422741
				18025 427535
				18026 430556
				18027 421759
				18028 427015
				18029 420882
				18030 436470
				18031 433884
				18032 471451
				18033 474779
				18034 462600
				18035 463890
				18036 470533
				18037 460173
				18038 489341
				18039 477182
				18040 477871
				18041 475165
				18042 482029
				18043 482293
				18044 486332
				18045 449013
				18046 436861
				18047 439379
				18048 440160
				18049 437515
				18050 437172
				18051 445209
				18052 439584
				18053 432904
				18054 461587
				18055 473908
				18056 472272
				18057 475304
				18058 471187
				18059 479172
				18060 445928
				18061 426311
				18062 433611
				18063 481307
				18064 465820
				18065 477532
				18066 478992
				18067 484932
				18068 492708
				18069 483208
				18070 489464
				18071 495990
				18072 490915
				18073 494069
				18074 501475
				18075 446222
				18076 444403
				18077 428007
				18078 426608
				18079 438645
				18080 425343
				18081 426085
				18082 461954
				18083 460915
				18084 456006
				18085 456518
				18086 465108
				18087 463865
				18088 458477
				18089 461865
				18090 441440
				18091 441065
				18092 427438
				18093 436614
				18094 429863
				18095 428614
				18096 428823
				18097 428524
				18098 487465
				18099 470245
				18100 463445
				18101 483424
				18102 477055
				18103 484780
				18104 481737
				18105 438660
				18106 480682
				18107 477897
				18108 479108
				18109 461777
				18110 477796
				18111 464834
				18112 462963
				18113 462481
				18114 462410
				18115 475308
				18116 472429
				18117 464425
				18118 486177
				18119 497397
				18120 446885
				18121 436060
				18122 442962
				18123 446565
				18124 467790
				18125 479785
				18126 488360
				18127 491391
				18128 484030
				18129 493445
				18130 478389
				18131 490864
				18132 493904
				18133 499721
				18134 501054
				18135 442180
				18136 441586
				18137 430584
				18138 422869
				18139 445131
				18140 418938
				18141 459896
				18142 466609
				18143 462260
				18144 466327
				18145 467591
				18146 489496
				18147 486882
				18148 458661
				18149 475131
				18150 452005
				18151 442652
				18152 433794
				18153 424632
				18154 464312
				18155 495956
				18156 500902
				18157 496251
				18158 491403
				18159 488784
				18160 497969
				18161 489243
				18162 483607
				18163 503506
				18164 493353
				18165 447975
				18166 430938
				18167 474897
				18168 478336
				18169 462268
				18170 469141
				18171 461064
				18172 474631
				18173 478931
				18174 464176
				18175 482123
				18176 477007
				18177 477324
				18178 477693
				18179 482378
				18180 459246
				18181 448813
				18182 441223
				18183 446692
				18184 439653
				18185 454327
				18186 471692
				18187 472841
				18188 476151
				18189 493765
				18190 493766
				18191 487504
				18192 493847
				18193 497025
				18194 480692
				18195 454312
				18196 434453
				18197 432240
				18198 417873
				18199 423482
				18200 491335
				18201 489130
				18202 481878
				18203 486438
				18204 478094
				18205 474699
				18206 486364
				18207 481907
				18208 483593
				18209 470443
				18210 443507
				18211 438426
				18212 429458
				18213 449839
				18214 434202
				18215 442127
				18216 440114
				18217 440236
				18218 429504
				18219 433525
				18220 445262
				18221 437024
				18222 489567
				18223 470262
				18224 490422
				18225 455352
				18226 444185
				18227 435278
				18228 421509
				18229 482688
				18230 480666
				18231 477846
				18232 475464
				18233 465476
				18234 470745
				18235 477032
				18236 476842
				18237 456707
				18238 464429
				18239 472272
				18240 448760
				18241 431905
				18242 436398
				18243 414110
				18244 432832
				18245 439708
				18246 444339
				18247 436215
				18248 428422
				18249 456073
				18250 438850
				18251 440379
				18252 432775
				18253 443022
				18254 449405
				18255 446239
				18256 436014
				18257 434216
				18258 423888
				18259 437857
				18260 436073
				18261 486418
				18262 477389
				18263 469763
				18264 477644
				18265 470814
				18266 470024
				18267 458724
				18268 475716
				18269 463202
				18270 448259
				18271 445862
				18272 438113
				18273 441660
				18274 446056
				18275 445824
				18276 450508
				18277 436883
				18278 431824
				18279 431017
				18280 440663
				18281 444863
				18282 435720
				18283 437510
				18284 431996
				18285 449737
				18286 439914
				18287 434057
				18288 478539
				18289 478761
				18290 481568
				18291 474003
				18292 475298
				18293 477136
				18294 480462
				18295 486244
				18296 484051
				18297 477923
				18298 490431
				18299 473471
				18300 441400
				18301 433623
				18302 428787
				18303 446390
				18304 440806
				18305 431655
				18306 433593
				18307 438415
				18308 432183
				18309 443370
				18310 439984
				18311 431594
				18312 442966
				18313 436970
				18314 440283
				18315 451376
				18316 435589
				18317 432526
				18318 421792
				18319 434283
				18320 429325
				18321 442319
				18322 431438
				18323 458685
				18324 459149
				18325 470129
				18326 463705
				18327 455494
				18328 467295
				18329 467705
				18330 443740
				18331 445492
				18332 433956
				18333 428982
				18334 440844
				18335 431357
				18336 444924
				18337 441302
				18338 446358
				18339 445535
				18340 431252
				18341 438865
				18342 436139
				18343 453426
				18344 435196
				18345 433667
				18346 441791
				18347 444343
				18348 441738
				18349 437683
				18350 426909
				18351 436662
				18352 443100
				18353 438422
				18354 443289
				18355 447494
				18356 445200
				18357 456802
				18358 455430
				18359 444502
				18360 438573
				18361 440192
				18362 446961
				18363 439507
				18364 434780
				18365 436702
				18366 438027
				18367 438966
				18368 437275
				18369 438856
				18370 433640
				18371 427939
				18372 442429
				18373 436059
				18374 468990
				18375 444553
				18376 439744
				18377 428660
				18378 435360
				18379 478906
				18380 481892
				18381 485013
				18382 472400
				18383 475118
				18384 472899
				18385 471865
				18386 478153
				18387 477157
				18388 478192
				18389 484906
				18390 453381
				18391 428694
				18392 436858
				18393 421903
				18394 429466
				18395 444378
				18396 436588
				18397 426210
				18398 429632
				18399 442687
				18400 442037
				18401 441243
				18402 438620
				18403 441782
				18404 436928
				18405 450723
				18406 431362
				18407 426777
				18408 431728
				18409 488658
				18410 480207
				18411 479688
				18412 491430
				18413 505388
				18414 491169
				18415 484898
				18416 494967
				18417 494791
				18418 487791
				18419 499278
				18420 448510
				18421 428935
				18422 427170
				18423 426696
				18424 491848
				18425 483963
				18426 477315
				18427 484052
				18428 477331
				18429 489263
				18430 490277
				18431 482086
				18432 510334
				18433 502555
				18434 496196
				18435 443112
				18436 431214
				18437 436319
				18438 469987
				18439 469055
				18440 460391
				18441 475214
				18442 479773
				18443 493250
				18444 476493
				18445 494287
				18446 488609
				18447 492614
				18448 476490
				18449 492109
				18450 446512
				18451 447798
				18452 437150
				18453 434461
				18454 432912
				18455 442405
				18456 488863
				18457 462933
				18458 467729
				18459 478766
				18460 476528
				18461 481313
				18462 472495
				18463 471491
				18464 470425
				18465 435136
				18466 443028
				18467 422082
				18468 418501
				18469 415965
				18470 424553
				18471 434457
				18472 422093
				18473 480483
				18474 473913
				18475 469034
				18476 483525
				18477 458807
				18478 467058
				18479 463369
				18480 448759
				18481 438803
				18482 479360
				18483 466117
				18484 468267
				18485 472703
				18486 461687
				18487 489983
				18488 479702
				18489 483281
				18490 501550
				18491 502443
				18492 491553
				18493 497435
				18494 491043
				18495 448420
				18496 441366
				18497 425781
				18498 431373
				18499 433093
				18500 414004
				18501 419595
				18502 436838
				18503 429910
				18504 429485
				18505 427741
				18506 424613
				18507 436147
				18508 434886
				18509 446857
				18510 455522
				18511 434654
				18512 425532
				18513 427091
				18514 428962
				18515 420179
				18516 419031
				18517 435400
				18518 408008
				18519 430137
				18520 418303
				18521 426360
				18522 432484
				18523 424002
				18524 420841
				18525 434352
				18526 439121
				18527 429325
				18528 422150
				18529 432309
				18530 430589
				18531 440104
				18532 476657
				18533 477182
				18534 466467
				18535 462771
				18536 472843
				18537 471389
				18538 461123
				18539 456810
				18540 449389
				18541 439388
				18542 435517
				18543 437857
				18544 449367
				18545 439423
				18546 441431
				18547 449481
				18548 438839
				18549 433980
				18550 455658
				18551 454989
				18552 454080
				18553 456129
				18554 448913
				18555 451860
				18556 437146
				18557 446424
				18558 438751
				18559 440452
				18560 430855
				18561 434979
				18562 460950
				18563 458684
				18564 464078
				18565 457915
				18566 454672
				18567 459892
				18568 461540
				18569 456425
				18570 443576
				18571 446861
				18572 442555
				18573 433808
				18574 438409
				18575 436636
				18576 430014
				18577 443379
				18578 435499
				18579 432997
				18580 426425
				18581 433550
				18582 430009
				18583 427850
				18584 432598
				18585 437019
				18586 436427
				18587 428205
				18588 431904
				18589 431017
				18590 434719
				18591 428148
				18592 414341
				18593 427445
				18594 420740
				18595 419516
				18596 415668
				18597 427322
				18598 412660
				18599 433462
				18600 445014
				18601 436477
				18602 443398
				18603 438500
				18604 423133
				18605 426592
				18606 483261
				18607 489985
				18608 491333
				18609 489250
				18610 485950
				18611 481952
				18612 486866
				18613 483004
				18614 496010
				18615 440831
				18616 442386
				18617 426490
				18618 435228
				18619 436750
				18620 445518
				18621 450977
				18622 444728
				18623 436732
				18624 455641
				18625 463124
				18626 453535
				18627 448255
				18628 444657
				18629 441898
				18630 432318
				18631 444407
				18632 425282
				18633 424194
				18634 428057
				18635 428871
				18636 419211
				18637 436269
				18638 419893
				18639 419564
				18640 425771
				18641 415245
				18642 422242
				18643 417004
				18644 413198
				18645 436148
				18646 427626
				18647 426114
				18648 407976
				18649 435610
				18650 426606
				18651 418182
				18652 409573
				18653 417449
				18654 428644
				18655 426002
				18656 410305
				18657 422217
				18658 420355
				18659 427446
				18660 436149
				18661 426035
				18662 422089
				18663 424268
				18664 438166
				18665 438046
				18666 438705
				18667 427589
				18668 425356
				18669 434020
				18670 432501
				18671 444841
				18672 431008
				18673 436467
				18674 429832
				18675 431844
				18676 434527
				18677 425662
				18678 422507
				18679 422367
				18680 464564
				18681 465065
				18682 461797
				18683 455228
				18684 458354
				18685 456124
				18686 447546
				18687 465297
				18688 471489
				18689 483274
				18690 450051
				18691 424453
				18692 416543
				18693 426843
				18694 448528
				18695 435344
				18696 426861
				18697 437293
				18698 436964
				18699 441087
				18700 462602
				18701 451262
				18702 464528
				18703 463741
				18704 470739
				18705 462133
				18706 448657
				18707 418654
				18708 429491
				18709 435861
				18710 424207
				18711 426129
				18712 427477
				18713 417078
				18714 433468
				18715 427152
				18716 449267
				18717 434416
				18718 439702
				18719 423751
				18720 446949
				18721 439040
				18722 472297
				18723 477687
				18724 476251
				18725 468811
				18726 477750
				18727 472992
				18728 478867
				18729 467024
				18730 479213
				18731 468860
				18732 469926
				18733 460768
				18734 459290
				18735 451661
				18736 427273
				18737 422186
				18738 438822
				18739 444541
				18740 443763
				18741 439427
				18742 439484
				18743 434140
				18744 429673
				18745 442624
				18746 433690
				18747 437072
				18748 426630
				18749 447240
				18750 437993
				18751 444656
				18752 426701
				18753 431607
				18754 424152
				18755 410378
				18756 427313
				18757 429429
				18758 427827
				18759 429327
				18760 419486
				18761 425680
				18762 430541
				18763 462644
				18764 467262
				18765 438144
				18766 484482
				18767 476335
				18768 467703
				18769 471458
				18770 483106
				18771 487516
				18772 484754
				18773 479203
				18774 476797
				18775 502375
				18776 485983
				18777 492549
				18778 492254
				18779 505805
				18780 440803
				18781 481531
				18782 462773
				18783 470372
				18784 462888
				18785 462492
				18786 464508
				18787 468283
				18788 461899
				18789 462739
				18790 447647
				18791 452052
				18792 455702
				18793 463221
				18794 460310
				18795 432207
				18796 430314
				18797 419006
				18798 433429
				18799 424714
				18800 427466
				18801 423856
				18802 419126
				18803 421308
				18804 424932
				18805 431384
				18806 426589
				18807 416998
				18808 413596
				18809 442513
				18810 452522
				18811 424999
				18812 426875
				18813 413672
				18814 424176
				18815 427666
				18816 423980
				18817 458174
				18818 455008
				18819 459882
				18820 458918
				18821 449801
				18822 461188
				18823 464025
				18824 466453
				18825 442193
				18826 431458
				18827 415492
				18828 406389
				18829 413836
				18830 416696
				18831 418743
				18832 412542
				18833 414448
				18834 416354
				18835 417536
				18836 417206
				18837 420475
				18838 425951
				18839 414873
				18840 430694
				18841 429805
				18842 413770
				18843 426504
				18844 418771
				18845 416143
				18846 412560
				18847 461684
				18848 467651
				18849 478117
				18850 467041
				18851 484764
				18852 473135
				18853 487013
				18854 470829
				18855 439814
				18856 491330
				18857 461114
				18858 463665
				18859 470201
				18860 464287
				18861 456744
				18862 454339
				18863 466842
				18864 470664
				18865 469267
				18866 457593
				18867 486376
				18868 489670
				18869 489688
				18870 426620
				18871 424649
				18872 421782
				18873 423207
				18874 432641
				18875 430672
				18876 419647
				18877 429557
				18878 430525
				18879 417733
				18880 450027
				18881 470666
				18882 452601
				18883 461897
				18884 456004
				18885 436765
				18886 483764
				18887 469194
				18888 458316
				18889 474632
				18890 476270
				18891 480800
				18892 476537
				18893 478770
				18894 461035
				18895 461539
				18896 467540
				18897 463749
				18898 466685
				18899 470385
				18900 437739
				18901 426091
				18902 426830
				18903 467139
				18904 466441
				18905 482596
				18906 459620
				18907 457270
				18908 452166
				18909 456114
				18910 466391
				18911 448944
				18912 459103
				18913 470544
				18914 459022
				18915 434537
				18916 438148
				18917 426314
				18918 435214
				18919 427390
				18920 414062
				18921 424784
				18922 422574
				18923 424927
				18924 419379
				18925 433377
				18926 422453
				18927 420373
				18928 426556
				18929 436289
				18930 434173
				18931 431292
				18932 415684
				18933 418222
				18934 420711
				18935 416958
				18936 454442
				18937 451519
				18938 449489
				18939 458114
				18940 472972
				18941 457901
				18942 450418
				18943 456910
				18944 462933
				18945 435219
				18946 420441
				18947 409329
				18948 404177
				18949 413919
				18950 421807
				18951 421380
				18952 424982
				18953 426740
				18954 426706
				18955 409827
				18956 425555
				18957 421964
				18958 423930
				18959 423509
				18960 429582
				18961 427665
				18962 406540
				18963 416037
				18964 411296
				18965 409963
				18966 411217
				18967 403176
				18968 413758
				18969 454580
				18970 462147
				18971 455065
				18972 454471
				18973 454851
				18974 459660
				18975 433955
				18976 422319
				18977 434314
				18978 433742
				18979 422968
				18980 414922
				18981 419141
				18982 463758
				18983 450933
				18984 453141
				18985 450460
				18986 456218
				18987 454816
				18988 468749
				18989 447853
				18990 421228
				18991 415684
				18992 424256
				18993 412562
				18994 427210
				18995 430019
				18996 425831
				18997 423618
				18998 423758
				18999 420196
				19000 429794
				19001 426791
				19002 421316
				19003 439692
				19004 430868
				19005 419567
				19006 430838
				19007 475426
				19008 463756
				19009 457290
				19010 467669
				19011 460305
				19012 465093
				19013 451435
				19014 461406
				19015 469126
				19016 463969
				19017 472751
				19018 468045
				19019 465986
				19020 432989
				19021 425831
				19022 418583
				19023 419715
				19024 412132
				19025 408221
				19026 407932
				19027 418255
				19028 407774
				19029 460723
				19030 456668
				19031 467889
				19032 462278
				19033 458496
				19034 453027
				19035 428198
				19036 409141
				19037 418719
				19038 413039
				19039 418967
				19040 410549
				19041 412225
				19042 411344
				19043 419499
				19044 420223
				19045 430189
				19046 422785
				19047 424673
				19048 427233
				19049 417502
				19050 432250
				19051 426326
				19052 427663
				19053 417017
				19054 409386
				19055 410886
				19056 407936
				19057 424169
				19058 399849
				19059 401210
				19060 451006
				19061 459736
				19062 477929
				19063 458412
				19064 456189
				19065 428138
				19066 423155
				19067 421034
				19068 416082
				19069 426363
				19070 422364
				19071 427195
				19072 420754
				19073 430085
				19074 440265
				19075 434867
				19076 432067
				19077 418598
				19078 441576
				19079 432457
				19080 431771
				19081 424921
				19082 418852
				19083 420670
				19084 421145
				19085 449226
				19086 452622
				19087 454197
				19088 455152
				19089 443970
				19090 448272
				19091 456669
				19092 464756
				19093 448292
				19094 460163
				19095 425380
				19096 425954
				19097 418222
				19098 418564
				19099 410872
				19100 467726
				19101 462995
				19102 451941
				19103 453839
				19104 452847
				19105 457918
				19106 460328
				19107 463470
				19108 463604
				19109 462007
				19110 431473
				19111 423298
				19112 420307
				19113 411630
				19114 413279
				19115 424443
				19116 413256
				19117 408265
				19118 418815
				19119 424731
				19120 417640
				19121 414132
				19122 425291
				19123 417769
				19124 424232
				19125 429354
				19126 427325
				19127 426792
				19128 432078
				19129 429673
				19130 435000
				19131 438142
				19132 435486
				19133 422620
				19134 427328
				19135 423776
				19136 427850
				19137 438091
				19138 430606
				19139 427624
				19140 446677
				19141 431337
				19142 418252
				19143 433642
				19144 436143
				19145 437836
				19146 426854
				19147 436106
				19148 443106
				19149 433873
				19150 435571
				19151 430260
				19152 436390
				19153 427375
				19154 440549
				19155 438270
				19156 438579
				19157 426082
				19158 416721
				19159 420843
				19160 425543
				19161 419724
				19162 415630
				19163 422152
				19164 420418
				19165 421303
				19166 421464
				19167 421684
				19168 416714
				19169 427409
				19170 436915
				19171 422406
				19172 429470
				19173 477880
				19174 455466
				19175 470623
				19176 458967
				19177 458283
				19178 468955
				19179 471755
				19180 460560
				19181 469558
				19182 465490
				19183 458491
				19184 454692
				19185 425205
				19186 486975
				19187 483017
				19188 471470
				19189 468163
				19190 469400
				19191 470961
				19192 464029
				19193 458343
				19194 474216
				19195 478841
				19196 471161
				19197 459317
				19198 460852
				19199 457625
				19200 434319
				19201 430494
				19202 418917
				19203 464607
				19204 459290
				19205 452425
				19206 453319
				19207 479602
				19208 462592
				19209 465599
				19210 459559
				19211 468849
				19212 473704
				19213 486064
				19214 477576
				19215 437197
				19216 435848
				19217 427009
				19218 471464
				19219 470143
				19220 464374
				19221 471985
				19222 455703
				19223 461666
				19224 463840
				19225 474554
				19226 457459
				19227 466691
				19228 457885
				19229 484878
				19230 436473
				19231 435450
				19232 422960
				19233 483962
				19234 469401
				19235 482345
				19236 469427
				19237 488081
				19238 479097
				19239 470939
				19240 468709
				19241 473010
				19242 462619
				19243 493592
				19244 471795
				19245 436758
				19246 429236
				19247 469116
				19248 465679
				19249 450934
				19250 453687
				19251 466581
				19252 461904
				19253 449801
				19254 460416
				19255 455276
				19256 449970
				19257 451970
				19258 455249
				19259 445782
				19260 429327
				19261 427481
				19262 430806
				19263 422954
				19264 419540
				19265 415098
				19266 414420
				19267 421134
				19268 470718
				19269 480430
				19270 473603
				19271 475199
				19272 477695
				19273 465684
				19274 478257
				19275 430688
				19276 432222
				19277 414350
				19278 432576
				19279 427866
				19280 413792
				19281 420541
				19282 408670
				19283 422207
				19284 426104
				19285 418910
				19286 427928
				19287 418057
				19288 424849
				19289 430651
				19290 442840
				19291 425636
				19292 421778
				19293 414341
				19294 414926
				19295 426749
				19296 421845
				19297 425833
				19298 428200
				19299 427327
				19300 436431
				19301 426087
				19302 428508
				19303 428024
				19304 422335
				19305 436656
				19306 477551
				19307 475886
				19308 467041
				19309 470647
				19310 471219
				19311 477701
				19312 484098
				19313 477992
				19314 480734
				19315 485217
				19316 485344
				19317 479419
				19318 493112
				19319 483682
				19320 446756
				19321 425366
				19322 420385
				19323 418150
				19324 424374
				19325 463589
				19326 471625
				19327 471094
				19328 461872
				19329 470815
				19330 485464
				19331 474663
				19332 467718
				19333 449225
				19334 464938
				19335 434522
				19336 418284
				19337 438577
				19338 431791
				19339 467208
				19340 458943
				19341 471686
				19342 457553
				19343 463383
				19344 475565
				19345 475457
				19346 476800
				19347 475832
				19348 475042
				19349 480198
				19350 445286
				19351 491337
				19352 480637
				19353 481722
				19354 477893
				19355 473234
				19356 482058
				19357 476799
				19358 471168
				19359 473499
				19360 473321
				19361 486136
				19362 477687
				19363 472803
				19364 473507
				19365 442773
				19366 443846
				19367 420113
				19368 433848
				19369 427013
				19370 437578
				19371 442515
				19372 435592
				19373 441769
				19374 440720
				19375 437315
				19376 435300
				19377 438737
				19378 423447
				19379 445746
				19380 452664
				19381 442211
				19382 425237
				19383 433303
				19384 436675
				19385 429996
				19386 434409
				19387 432603
				19388 433362
				19389 436993
				19390 436124
				19391 431134
				19392 446382
				19393 446838
				19394 443249
				19395 439794
				19396 423016
				19397 437501
				19398 425001
				19399 432935
				19400 424446
				19401 424922
				19402 430822
				19403 436265
				19404 430601
				19405 458520
				19406 448785
				19407 456266
				19408 454708
				19409 448535
				19410 439868
				19411 431423
				19412 426620
				19413 428845
				19414 419994
				19415 427775
				19416 421223
				19417 424707
				19418 433789
				19419 417676
				19420 423288
				19421 423563
				19422 422919
				19423 418170
				19424 428035
				19425 440917
				19426 434957
				19427 431810
				19428 425283
				19429 474796
				19430 482975
				19431 467491
				19432 458435
				19433 469329
				19434 471653
				19435 465216
				19436 461012
				19437 463603
				19438 468301
				19439 456860
				19440 432437
				19441 435202
				19442 422622
				19443 425353
				19444 419421
				19445 417462
				19446 439062
				19447 432789
				19448 438094
				19449 422852
				19450 451197
				19451 433872
				19452 436138
				19453 429579
				19454 434012
				19455 446104
				19456 431408
				19457 427448
				19458 423821
				19459 415310
				19460 426260
				19461 474027
				19462 463095
				19463 493303
				19464 473497
				19465 483695
				19466 482897
				19467 493399
				19468 494904
				19469 495236
				19470 427390
				19471 432331
				19472 426239
				19473 420256
				19474 422472
				19475 428378
				19476 420433
				19477 423848
				19478 418072
				19479 436129
				19480 430201
				19481 465729
				19482 466893
				19483 478145
				19484 471836
				19485 444395
				19486 433519
				19487 417754
				19488 427062
				19489 421317
				19490 420324
				19491 428217
				19492 428224
				19493 418435
				19494 423099
				19495 415046
				19496 418628
				19497 426587
				19498 417732
				19499 409955
				19500 426656
				19501 433935
				19502 419952
				19503 429683
				19504 416902
				19505 429003
				19506 430970
				19507 430944
				19508 431637
				19509 438512
				19510 428428
				19511 429665
				19512 434195
				19513 433476
				19514 431862
				19515 439541
				19516 417970
				19517 418257
				19518 420933
				19519 429567
				19520 416266
				19521 478109
				19522 503236
				19523 491066
				19524 489215
				19525 494233
				19526 493944
				19527 490679
				19528 489574
				19529 494660
				19530 437609
				19531 469414
				19532 452277
				19533 456788
				19534 453254
				19535 478766
				19536 469669
				19537 482162
				19538 475082
				19539 475798
				19540 472181
				19541 471494
				19542 491689
				19543 487632
				19544 482351
				19545 431089
				19546 434979
				19547 426562
				19548 428760
				19549 415110
				19550 424537
				19551 412177
				19552 430151
				19553 440590
				19554 432651
				19555 438387
				19556 427384
				19557 430768
				19558 435584
				19559 425504
				19560 438391
				19561 438415
				19562 431040
				19563 432304
				19564 440633
				19565 434930
				19566 438820
				19567 432309
				19568 436793
				19569 441861
				19570 438779
				19571 446373
				19572 448004
				19573 444460
				19574 427493
				19575 453448
				19576 431816
				19577 418258
				19578 417740
				19579 406284
				19580 426804
				19581 419385
				19582 410544
				19583 420490
				19584 418093
				19585 417305
				19586 410693
				19587 424704
				19588 428246
				19589 435188
				19590 435476
				19591 422788
				19592 417810
				19593 461732
				19594 472427
				19595 463676
				19596 461357
				19597 463226
				19598 467159
				19599 461741
				19600 470575
				19601 462329
				19602 462712
				19603 452021
				19604 456970
				19605 437266
				19606 482035
				19607 464880
				19608 474123
				19609 469235
				19610 465850
				19611 468756
				19612 461410
				19613 466764
				19614 460253
				19615 486271
				19616 491215
				19617 475076
				19618 486032
				19619 487640
				19620 429886
				19621 431621
				19622 413946
				19623 407263
				19624 428215
				19625 420814
				19626 429994
				19627 431640
				19628 419423
				19629 433447
				19630 429128
				19631 420770
				19632 430767
				19633 430145
				19634 457737
				19635 437782
				19636 422000
				19637 414190
				19638 419322
				19639 426204
				19640 417585
				19641 413078
				19642 422061
				19643 412037
				19644 422286
				19645 412224
				19646 425833
				19647 421159
				19648 430043
				19649 421287
				19650 433813
				19651 428944
				19652 429040
				19653 422997
				19654 424609
				19655 436280
				19656 423962
				19657 458723
				19658 462539
				19659 458882
				19660 471247
				19661 466907
				19662 472389
				19663 467421
				19664 465083
				19665 437126
				19666 429626
				19667 412208
				19668 417341
				19669 473250
				19670 473668
				19671 467845
				19672 456807
				19673 477590
				19674 475177
				19675 468359
				19676 471976
				19677 478426
				19678 476462
				19679 470648
				19680 436192
				19681 441489
				19682 467336
				19683 472510
				19684 483222
				19685 488943
				19686 480739
				19687 484504
				19688 483930
				19689 482524
				19690 481289
				19691 481263
				19692 476207
				19693 487596
				19694 482433
				19695 444365
				19696 440533
				19697 436097
				19698 477432
				19699 467595
				19700 464771
				19701 468259
				19702 469196
				19703 502496
				19704 487147
				19705 479674
				19706 486503
				19707 484458
				19708 495702
				19709 489682
				19710 442670
				19711 420209
				19712 420754
				19713 418916
				19714 433726
				19715 426595
				19716 437002
				19717 433088
				19718 423450
				19719 444803
				19720 434624
				19721 423242
				19722 424668
				19723 439962
				19724 429071
				19725 437794
				19726 422942
				19727 466282
				19728 467928
				19729 485322
				19730 480635
				19731 492160
				19732 480385
				19733 483054
				19734 481860
				19735 483246
				19736 486643
				19737 487851
				19738 475594
				19739 486746
				19740 429930
				19741 444449
				19742 431881
				19743 483813
				19744 474758
				19745 487865
				19746 485538
				19747 476630
				19748 482090
				19749 468382
				19750 474142
				19751 475104
				19752 496805
				19753 488226
				19754 480297
				19755 424949
				19756 492871
				19757 475765
				19758 472567
				19759 466217
				19760 462714
				19761 472775
				19762 468494
				19763 468097
				19764 463470
				19765 451669
				19766 455294
				19767 463189
				19768 462531
				19769 479133
				19770 445486
				19771 443040
				19772 459544
				19773 486605
				19774 488495
				19775 491331
				19776 481770
				19777 481189
				19778 480877
				19779 485284
				19780 487972
				19781 472076
				19782 487134
				19783 483307
				19784 477563
				19785 432977
				19786 431651
				19787 430650
				19788 428980
				19789 438671
				19790 440569
				19791 434478
				19792 431641
				19793 431112
				19794 433039
				19795 429978
				19796 430469
				19797 435887
				19798 431845
				19799 431057
				19800 448622
				19801 436372
				19802 439803
				19803 443672
				19804 433200
				19805 432874
				19806 430949
				19807 431202
				19808 421642
				19809 429083
				19810 427486
				19811 428400
				19812 430418
				19813 429099
				19814 429677
				19815 434046
				19816 432864
				19817 422472
				19818 463024
				19819 459735
				19820 489931
				19821 474920
				19822 491656
				19823 491476
				19824 499453
				19825 475663
				19826 479026
				19827 490959
				19828 489367
				19829 486433
				19830 437643
				19831 426642
				19832 422797
				19833 432409
				19834 432991
				19835 433417
				19836 430277
				19837 445268
				19838 436720
				19839 425675
				19840 428395
				19841 444558
				19842 442684
				19843 435672
				19844 444975
				19845 436338
				19846 429640
				19847 425562
				19848 420147
				19849 418278
				19850 419902
				19851 418989
				19852 411096
				19853 418457
				19854 417384
				19855 424007
				19856 455178
				19857 468329
				19858 457400
				19859 474980
				19860 429964
				19861 433689
				19862 413867
				19863 418572
				19864 425410
				19865 424256
				19866 424972
				19867 430384
				19868 418708
				19869 427242
				19870 432755
				19871 426660
				19872 412903
				19873 428088
				19874 425452
				19875 439794
				19876 472599
				19877 455709
				19878 469107
				19879 449031
				19880 461341
				19881 458871
				19882 461407
				19883 454893
				19884 453389
				19885 463091
				19886 458918
				19887 453957
				19888 460957
				19889 454945
				19890 433725
				19891 476510
				19892 463097
				19893 465408
				19894 455672
				19895 475053
				19896 465398
				19897 464216
				19898 471565
				19899 481720
				19900 487942
				19901 473187
				19902 481703
				19903 477528
				19904 483948
				19905 430712
				19906 427766
				19907 423982
				19908 431113
				19909 432021
				19910 421706
				19911 434056
				19912 436349
				19913 425352
				19914 434943
				19915 451449
				19916 480175
				19917 486865
				19918 478062
				19919 483650
				19920 435064
				19921 416153
				19922 470668
				19923 481983
				19924 464491
				19925 467804
				19926 474388
				19927 464986
				19928 497347
				19929 473273
				19930 467120
				19931 467607
				19932 471940
				19933 475270
				19934 472185
				19935 434389
				19936 426369
				19937 437196
				19938 438337
				19939 463735
				19940 468188
				19941 462870
				19942 459751
				19943 460116
				19944 459036
				19945 477413
				19946 465317
				19947 472009
				19948 478493
				19949 470120
				19950 433323
				19951 422490
				19952 420404
				19953 421527
				19954 413353
				19955 418275
				19956 429148
				19957 435269
				19958 431837
				19959 424186
				19960 432656
				19961 464103
				19962 469281
				19963 467837
				19964 462793
				19965 438412
				19966 437138
				19967 422476
				19968 429079
				19969 427965
				19970 418665
				19971 464370
				19972 454603
				19973 469163
				19974 461311
				19975 461223
				19976 460405
				19977 466499
				19978 464457
				19979 457190
				19980 438836
				19981 433682
				19982 428396
				19983 419397
				19984 419805
				19985 479720
				19986 469337
				19987 482925
				19988 469215
				19989 486662
				19990 479955
				19991 473779
				19992 478331
				19993 478198
				19994 475453
				19995 440340
				19996 471050
				19997 469045
				19998 459090
				19999 472078
			};
		\addplot [semithick, red, dashed]
		table {%
				2160 291060.161469119
				2161 291070.957259584
				2162 291081.753050049
				2163 291092.548840513
				2164 291103.344630978
				2165 291114.140421442
				2166 291124.936211907
				2167 291135.732002372
				2168 291146.527792836
				2169 291157.323583301
				2170 291168.119373765
				2171 291178.91516423
				2172 291189.710954694
				2173 291200.506745159
				2174 291211.302535624
				2175 291222.098326088
				2176 291232.894116553
				2177 291243.689907017
				2178 291254.485697482
				2179 291265.281487947
				2180 291276.077278411
				2181 291286.873068876
				2182 291297.66885934
				2183 291308.464649805
				2184 291319.260440269
				2185 291330.056230734
				2186 291340.852021199
				2187 291351.647811663
				2188 291362.443602128
				2189 291373.239392592
				2190 291384.035183057
				2191 291394.830973521
				2192 291405.626763986
				2193 291416.422554451
				2194 291427.218344915
				2195 291438.01413538
				2196 291448.809925844
				2197 291459.605716309
				2198 291470.401506774
				2199 291481.197297238
				2200 291491.993087703
				2201 291502.788878167
				2202 291513.584668632
				2203 291524.380459096
				2204 291535.176249561
				2205 291545.972040026
				2206 291556.76783049
				2207 291567.563620955
				2208 291578.359411419
				2209 291589.155201884
				2210 291599.950992348
				2211 291610.746782813
				2212 291621.542573278
				2213 291632.338363742
				2214 291643.134154207
				2215 291653.929944671
				2216 291664.725735136
				2217 291675.521525601
				2218 291686.317316065
				2219 291697.11310653
				2220 291707.908896994
				2221 291718.704687459
				2222 291729.500477923
				2223 291740.296268388
				2224 291751.092058853
				2225 291761.887849317
				2226 291772.683639782
				2227 291783.479430246
				2228 291794.275220711
				2229 291805.071011176
				2230 291815.86680164
				2231 291826.662592105
				2232 291837.458382569
				2233 291848.254173034
				2234 291859.049963498
				2235 291869.845753963
				2236 291880.641544428
				2237 291891.437334892
				2238 291902.233125357
				2239 291913.028915821
				2240 291923.824706286
				2241 291934.62049675
				2242 291945.416287215
				2243 291956.21207768
				2244 291967.007868144
				2245 291977.803658609
				2246 291988.599449073
				2247 291999.395239538
				2248 292010.191030003
				2249 292020.986820467
				2250 292031.782610932
				2251 292042.578401396
				2252 292053.374191861
				2253 292064.169982325
				2254 292074.96577279
				2255 292085.761563255
				2256 292096.557353719
				2257 292107.353144184
				2258 292118.148934648
				2259 292128.944725113
				2260 292139.740515577
				2261 292150.536306042
				2262 292161.332096507
				2263 292172.127886971
				2264 292182.923677436
				2265 292193.7194679
				2266 292204.515258365
				2267 292215.31104883
				2268 292226.106839294
				2269 292236.902629759
				2270 292247.698420223
				2271 292258.494210688
				2272 292269.290001152
				2273 292280.085791617
				2274 292290.881582082
				2275 292301.677372546
				2276 292312.473163011
				2277 292323.268953475
				2278 292334.06474394
				2279 292344.860534405
				2280 292355.656324869
				2281 292366.452115334
				2282 292377.247905798
				2283 292388.043696263
				2284 292398.839486727
				2285 292409.635277192
				2286 292420.431067657
				2287 292431.226858121
				2288 292442.022648586
				2289 292452.81843905
				2290 292463.614229515
				2291 292474.410019979
				2292 292485.205810444
				2293 292496.001600909
				2294 292506.797391373
				2295 292517.593181838
				2296 292528.388972302
				2297 292539.184762767
				2298 292549.980553231
				2299 292560.776343696
				2300 292571.572134161
				2301 292582.367924625
				2302 292593.16371509
				2303 292603.959505554
				2304 292614.755296019
				2305 292625.551086484
				2306 292636.346876948
				2307 292647.142667413
				2308 292657.938457877
				2309 292668.734248342
				2310 292679.530038806
				2311 292690.325829271
				2312 292701.121619736
				2313 292711.9174102
				2314 292722.713200665
				2315 292733.508991129
				2316 292744.304781594
				2317 292755.100572059
				2318 292765.896362523
				2319 292776.692152988
				2320 292787.487943452
				2321 292798.283733917
				2322 292809.079524381
				2323 292819.875314846
				2324 292830.671105311
				2325 292841.466895775
				2326 292852.26268624
				2327 292863.058476704
				2328 292873.854267169
				2329 292884.650057633
				2330 292895.445848098
				2331 292906.241638563
				2332 292917.037429027
				2333 292927.833219492
				2334 292938.629009956
				2335 292949.424800421
				2336 292960.220590886
				2337 292971.01638135
				2338 292981.812171815
				2339 292992.607962279
				2340 293003.403752744
				2341 293014.199543208
				2342 293024.995333673
				2343 293035.791124138
				2344 293046.586914602
				2345 293057.382705067
				2346 293068.178495531
				2347 293078.974285996
				2348 293089.77007646
				2349 293100.565866925
				2350 293111.36165739
				2351 293122.157447854
				2352 293132.953238319
				2353 293143.749028783
				2354 293154.544819248
				2355 293165.340609713
				2356 293176.136400177
				2357 293186.932190642
				2358 293197.727981106
				2359 293208.523771571
				2360 293219.319562035
				2361 293230.1153525
				2362 293240.911142965
				2363 293251.706933429
				2364 293262.502723894
				2365 293273.298514358
				2366 293284.094304823
				2367 293294.890095288
				2368 293305.685885752
				2369 293316.481676217
				2370 293327.277466681
				2371 293338.073257146
				2372 293348.86904761
				2373 293359.664838075
				2374 293370.46062854
				2375 293381.256419004
				2376 293392.052209469
				2377 293402.847999933
				2378 293413.643790398
				2379 293424.439580862
				2380 293435.235371327
				2381 293446.031161792
				2382 293456.826952256
				2383 293467.622742721
				2384 293478.418533185
				2385 293489.21432365
				2386 293500.010114115
				2387 293510.805904579
				2388 293521.601695044
				2389 293532.397485508
				2390 293543.193275973
				2391 293553.989066437
				2392 293564.784856902
				2393 293575.580647367
				2394 293586.376437831
				2395 293597.172228296
				2396 293607.96801876
				2397 293618.763809225
				2398 293629.559599689
				2399 293640.355390154
				2400 293651.151180619
				2401 293661.946971083
				2402 293672.742761548
				2403 293683.538552012
				2404 293694.334342477
				2405 293705.130132942
				2406 293715.925923406
				2407 293726.721713871
				2408 293737.517504335
				2409 293748.3132948
				2410 293759.109085264
				2411 293769.904875729
				2412 293780.700666194
				2413 293791.496456658
				2414 293802.292247123
				2415 293813.088037587
				2416 293823.883828052
				2417 293834.679618517
				2418 293845.475408981
				2419 293856.271199446
				2420 293867.06698991
				2421 293877.862780375
				2422 293888.658570839
				2423 293899.454361304
				2424 293910.250151769
				2425 293921.045942233
				2426 293931.841732698
				2427 293942.637523162
				2428 293953.433313627
				2429 293964.229104091
				2430 293975.024894556
				2431 293985.820685021
				2432 293996.616475485
				2433 294007.41226595
				2434 294018.208056414
				2435 294029.003846879
				2436 294039.799637344
				2437 294050.595427808
				2438 294061.391218273
				2439 294072.187008737
				2440 294082.982799202
				2441 294093.778589666
				2442 294104.574380131
				2443 294115.370170596
				2444 294126.16596106
				2445 294136.961751525
				2446 294147.757541989
				2447 294158.553332454
				2448 294169.349122919
				2449 294180.144913383
				2450 294190.940703848
				2451 294201.736494312
				2452 294212.532284777
				2453 294223.328075241
				2454 294234.123865706
				2455 294244.919656171
				2456 294255.715446635
				2457 294266.5112371
				2458 294277.307027564
				2459 294288.102818029
				2460 294298.898608493
				2461 294309.694398958
				2462 294320.490189423
				2463 294331.285979887
				2464 294342.081770352
				2465 294352.877560816
				2466 294363.673351281
				2467 294374.469141745
				2468 294385.26493221
				2469 294396.060722675
				2470 294406.856513139
				2471 294417.652303604
				2472 294428.448094068
				2473 294439.243884533
				2474 294450.039674998
				2475 294460.835465462
				2476 294471.631255927
				2477 294482.427046391
				2478 294493.222836856
				2479 294504.01862732
				2480 294514.814417785
				2481 294525.61020825
				2482 294536.405998714
				2483 294547.201789179
				2484 294557.997579643
				2485 294568.793370108
				2486 294579.589160573
				2487 294590.384951037
				2488 294601.180741502
				2489 294611.976531966
				2490 294622.772322431
				2491 294633.568112895
				2492 294644.36390336
				2493 294655.159693825
				2494 294665.955484289
				2495 294676.751274754
				2496 294687.547065218
				2497 294698.342855683
				2498 294709.138646147
				2499 294719.934436612
				2500 294730.730227077
				2501 294741.526017541
				2502 294752.321808006
				2503 294763.11759847
				2504 294773.913388935
				2505 294784.7091794
				2506 294795.504969864
				2507 294806.300760329
				2508 294817.096550793
				2509 294827.892341258
				2510 294838.688131722
				2511 294849.483922187
				2512 294860.279712652
				2513 294871.075503116
				2514 294881.871293581
				2515 294892.667084045
				2516 294903.46287451
				2517 294914.258664974
				2518 294925.054455439
				2519 294935.850245904
				2520 294946.646036368
				2521 294957.441826833
				2522 294968.237617297
				2523 294979.033407762
				2524 294989.829198227
				2525 295000.624988691
				2526 295011.420779156
				2527 295022.21656962
				2528 295033.012360085
				2529 295043.808150549
				2530 295054.603941014
				2531 295065.399731479
				2532 295076.195521943
				2533 295086.991312408
				2534 295097.787102872
				2535 295108.582893337
				2536 295119.378683802
				2537 295130.174474266
				2538 295140.970264731
				2539 295151.766055195
				2540 295162.56184566
				2541 295173.357636124
				2542 295184.153426589
				2543 295194.949217054
				2544 295205.745007518
				2545 295216.540797983
				2546 295227.336588447
				2547 295238.132378912
				2548 295248.928169376
				2549 295259.723959841
				2550 295270.519750306
				2551 295281.31554077
				2552 295292.111331235
				2553 295302.907121699
				2554 295313.702912164
				2555 295324.498702629
				2556 295335.294493093
				2557 295346.090283558
				2558 295356.886074022
				2559 295367.681864487
				2560 295378.477654951
				2561 295389.273445416
				2562 295400.069235881
				2563 295410.865026345
				2564 295421.66081681
				2565 295432.456607274
				2566 295443.252397739
				2567 295454.048188203
				2568 295464.843978668
				2569 295475.639769133
				2570 295486.435559597
				2571 295497.231350062
				2572 295508.027140526
				2573 295518.822930991
				2574 295529.618721456
				2575 295540.41451192
				2576 295551.210302385
				2577 295562.006092849
				2578 295572.801883314
				2579 295583.597673778
				2580 295594.393464243
				2581 295605.189254708
				2582 295615.985045172
				2583 295626.780835637
				2584 295637.576626101
				2585 295648.372416566
				2586 295659.168207031
				2587 295669.963997495
				2588 295680.75978796
				2589 295691.555578424
				2590 295702.351368889
				2591 295713.147159353
				2592 295723.942949818
				2593 295734.738740283
				2594 295745.534530747
				2595 295756.330321212
				2596 295767.126111676
				2597 295777.921902141
				2598 295788.717692605
				2599 295799.51348307
				2600 295810.309273535
				2601 295821.105063999
				2602 295831.900854464
				2603 295842.696644928
				2604 295853.492435393
				2605 295864.288225858
				2606 295875.084016322
				2607 295885.879806787
				2608 295896.675597251
				2609 295907.471387716
				2610 295918.26717818
				2611 295929.062968645
				2612 295939.85875911
				2613 295950.654549574
				2614 295961.450340039
				2615 295972.246130503
				2616 295983.041920968
				2617 295993.837711432
				2618 296004.633501897
				2619 296015.429292362
				2620 296026.225082826
				2621 296037.020873291
				2622 296047.816663755
				2623 296058.61245422
				2624 296069.408244685
				2625 296080.204035149
				2626 296090.999825614
				2627 296101.795616078
				2628 296112.591406543
				2629 296123.387197007
				2630 296134.182987472
				2631 296144.978777937
				2632 296155.774568401
				2633 296166.570358866
				2634 296177.36614933
				2635 296188.161939795
				2636 296198.95773026
				2637 296209.753520724
				2638 296220.549311189
				2639 296231.345101653
				2640 296242.140892118
				2641 296252.936682582
				2642 296263.732473047
				2643 296274.528263512
				2644 296285.324053976
				2645 296296.119844441
				2646 296306.915634905
				2647 296317.71142537
				2648 296328.507215834
				2649 296339.303006299
				2650 296350.098796764
				2651 296360.894587228
				2652 296371.690377693
				2653 296382.486168157
				2654 296393.281958622
				2655 296404.077749086
				2656 296414.873539551
				2657 296425.669330016
				2658 296436.46512048
				2659 296447.260910945
				2660 296458.056701409
				2661 296468.852491874
				2662 296479.648282339
				2663 296490.444072803
				2664 296501.239863268
				2665 296512.035653732
				2666 296522.831444197
				2667 296533.627234661
				2668 296544.423025126
				2669 296555.218815591
				2670 296566.014606055
				2671 296576.81039652
				2672 296587.606186984
				2673 296598.401977449
				2674 296609.197767914
				2675 296619.993558378
				2676 296630.789348843
				2677 296641.585139307
				2678 296652.380929772
				2679 296663.176720236
				2680 296673.972510701
				2681 296684.768301166
				2682 296695.56409163
				2683 296706.359882095
				2684 296717.155672559
				2685 296727.951463024
				2686 296738.747253488
				2687 296749.543043953
				2688 296760.338834418
				2689 296771.134624882
				2690 296781.930415347
				2691 296792.726205811
				2692 296803.521996276
				2693 296814.317786741
				2694 296825.113577205
				2695 296835.90936767
				2696 296846.705158134
				2697 296857.500948599
				2698 296868.296739063
				2699 296879.092529528
				2700 296889.888319993
				2701 296900.684110457
				2702 296911.479900922
				2703 296922.275691386
				2704 296933.071481851
				2705 296943.867272315
				2706 296954.66306278
				2707 296965.458853245
				2708 296976.254643709
				2709 296987.050434174
				2710 296997.846224638
				2711 297008.642015103
				2712 297019.437805568
				2713 297030.233596032
				2714 297041.029386497
				2715 297051.825176961
				2716 297062.620967426
				2717 297073.41675789
				2718 297084.212548355
				2719 297095.00833882
				2720 297105.804129284
				2721 297116.599919749
				2722 297127.395710213
				2723 297138.191500678
				2724 297148.987291143
				2725 297159.783081607
				2726 297170.578872072
				2727 297181.374662536
				2728 297192.170453001
				2729 297202.966243465
				2730 297213.76203393
				2731 297224.557824395
				2732 297235.353614859
				2733 297246.149405324
				2734 297256.945195788
				2735 297267.740986253
				2736 297278.536776717
				2737 297289.332567182
				2738 297300.128357647
				2739 297310.924148111
				2740 297321.719938576
				2741 297332.51572904
				2742 297343.311519505
				2743 297354.10730997
				2744 297364.903100434
				2745 297375.698890899
				2746 297386.494681363
				2747 297397.290471828
				2748 297408.086262292
				2749 297418.882052757
				2750 297429.677843222
				2751 297440.473633686
				2752 297451.269424151
				2753 297462.065214615
				2754 297472.86100508
				2755 297483.656795544
				2756 297494.452586009
				2757 297505.248376474
				2758 297516.044166938
				2759 297526.839957403
				2760 297537.635747867
				2761 297548.431538332
				2762 297559.227328797
				2763 297570.023119261
				2764 297580.818909726
				2765 297591.61470019
				2766 297602.410490655
				2767 297613.206281119
				2768 297624.002071584
				2769 297634.797862049
				2770 297645.593652513
				2771 297656.389442978
				2772 297667.185233442
				2773 297677.981023907
				2774 297688.776814372
				2775 297699.572604836
				2776 297710.368395301
				2777 297721.164185765
				2778 297731.95997623
				2779 297742.755766694
				2780 297753.551557159
				2781 297764.347347624
				2782 297775.143138088
				2783 297785.938928553
				2784 297796.734719017
				2785 297807.530509482
				2786 297818.326299946
				2787 297829.122090411
				2788 297839.917880876
				2789 297850.71367134
				2790 297861.509461805
				2791 297872.305252269
				2792 297883.101042734
				2793 297893.896833198
				2794 297904.692623663
				2795 297915.488414128
				2796 297926.284204592
				2797 297937.079995057
				2798 297947.875785521
				2799 297958.671575986
				2800 297969.467366451
				2801 297980.263156915
				2802 297991.05894738
				2803 298001.854737844
				2804 298012.650528309
				2805 298023.446318773
				2806 298034.242109238
				2807 298045.037899703
				2808 298055.833690167
				2809 298066.629480632
				2810 298077.425271096
				2811 298088.221061561
				2812 298099.016852026
				2813 298109.81264249
				2814 298120.608432955
				2815 298131.404223419
				2816 298142.200013884
				2817 298152.995804348
				2818 298163.791594813
				2819 298174.587385278
				2820 298185.383175742
				2821 298196.178966207
				2822 298206.974756671
				2823 298217.770547136
				2824 298228.5663376
				2825 298239.362128065
				2826 298250.15791853
				2827 298260.953708994
				2828 298271.749499459
				2829 298282.545289923
				2830 298293.341080388
				2831 298304.136870853
				2832 298314.932661317
				2833 298325.728451782
				2834 298336.524242246
				2835 298347.320032711
				2836 298358.115823175
				2837 298368.91161364
				2838 298379.707404105
				2839 298390.503194569
				2840 298401.298985034
				2841 298412.094775498
				2842 298422.890565963
				2843 298433.686356427
				2844 298444.482146892
				2845 298455.277937357
				2846 298466.073727821
				2847 298476.869518286
				2848 298487.66530875
				2849 298498.461099215
				2850 298509.25688968
				2851 298520.052680144
				2852 298530.848470609
				2853 298541.644261073
				2854 298552.440051538
				2855 298563.235842002
				2856 298574.031632467
				2857 298584.827422932
				2858 298595.623213396
				2859 298606.419003861
				2860 298617.214794325
				2861 298628.01058479
				2862 298638.806375255
				2863 298649.602165719
				2864 298660.397956184
				2865 298671.193746648
				2866 298681.989537113
				2867 298692.785327577
				2868 298703.581118042
				2869 298714.376908507
				2870 298725.172698971
				2871 298735.968489436
				2872 298746.7642799
				2873 298757.560070365
				2874 298768.355860829
				2875 298779.151651294
				2876 298789.947441759
				2877 298800.743232223
				2878 298811.539022688
				2879 298822.334813152
				2880 298833.130603617
				2881 298843.926394082
				2882 298854.722184546
				2883 298865.517975011
				2884 298876.313765475
				2885 298887.10955594
				2886 298897.905346404
				2887 298908.701136869
				2888 298919.496927334
				2889 298930.292717798
				2890 298941.088508263
				2891 298951.884298727
				2892 298962.680089192
				2893 298973.475879657
				2894 298984.271670121
				2895 298995.067460586
				2896 299005.86325105
				2897 299016.659041515
				2898 299027.454831979
				2899 299038.250622444
				2900 299049.046412909
				2901 299059.842203373
				2902 299070.637993838
				2903 299081.433784302
				2904 299092.229574767
				2905 299103.025365231
				2906 299113.821155696
				2907 299124.616946161
				2908 299135.412736625
				2909 299146.20852709
				2910 299157.004317554
				2911 299167.800108019
				2912 299178.595898484
				2913 299189.391688948
				2914 299200.187479413
				2915 299210.983269877
				2916 299221.779060342
				2917 299232.574850806
				2918 299243.370641271
				2919 299254.166431736
				2920 299264.9622222
				2921 299275.758012665
				2922 299286.553803129
				2923 299297.349593594
				2924 299308.145384058
				2925 299318.941174523
				2926 299329.736964988
				2927 299340.532755452
				2928 299351.328545917
				2929 299362.124336381
				2930 299372.920126846
				2931 299383.715917311
				2932 299394.511707775
				2933 299405.30749824
				2934 299416.103288704
				2935 299426.899079169
				2936 299437.694869633
				2937 299448.490660098
				2938 299459.286450563
				2939 299470.082241027
				2940 299480.878031492
				2941 299491.673821956
				2942 299502.469612421
				2943 299513.265402886
				2944 299524.06119335
				2945 299534.856983815
				2946 299545.652774279
				2947 299556.448564744
				2948 299567.244355208
				2949 299578.040145673
				2950 299588.835936138
				2951 299599.631726602
				2952 299610.427517067
				2953 299621.223307531
				2954 299632.019097996
				2955 299642.81488846
				2956 299653.610678925
				2957 299664.40646939
				2958 299675.202259854
				2959 299685.998050319
				2960 299696.793840783
				2961 299707.589631248
				2962 299718.385421712
				2963 299729.181212177
				2964 299739.977002642
				2965 299750.772793106
				2966 299761.568583571
				2967 299772.364374035
				2968 299783.1601645
				2969 299793.955954965
				2970 299804.751745429
				2971 299815.547535894
				2972 299826.343326358
				2973 299837.139116823
				2974 299847.934907287
				2975 299858.730697752
				2976 299869.526488217
				2977 299880.322278681
				2978 299891.118069146
				2979 299901.91385961
				2980 299912.709650075
				2981 299923.50544054
				2982 299934.301231004
				2983 299945.097021469
				2984 299955.892811933
				2985 299966.688602398
				2986 299977.484392862
				2987 299988.280183327
				2988 299999.075973792
				2989 300009.871764256
				2990 300020.667554721
				2991 300031.463345185
				2992 300042.25913565
				2993 300053.054926114
				2994 300063.850716579
				2995 300074.646507044
				2996 300085.442297508
				2997 300096.238087973
				2998 300107.033878437
				2999 300117.829668902
				3000 300128.625459367
				3001 300139.421249831
				3002 300150.217040296
				3003 300161.01283076
				3004 300171.808621225
				3005 300182.604411689
				3006 300193.400202154
				3007 300204.195992619
				3008 300214.991783083
				3009 300225.787573548
				3010 300236.583364012
				3011 300247.379154477
				3012 300258.174944941
				3013 300268.970735406
				3014 300279.766525871
				3015 300290.562316335
				3016 300301.3581068
				3017 300312.153897264
				3018 300322.949687729
				3019 300333.745478194
				3020 300344.541268658
				3021 300355.337059123
				3022 300366.132849587
				3023 300376.928640052
				3024 300387.724430516
				3025 300398.520220981
				3026 300409.316011446
				3027 300420.11180191
				3028 300430.907592375
				3029 300441.703382839
				3030 300452.499173304
				3031 300463.294963769
				3032 300474.090754233
				3033 300484.886544698
				3034 300495.682335162
				3035 300506.478125627
				3036 300517.273916091
				3037 300528.069706556
				3038 300538.865497021
				3039 300549.661287485
				3040 300560.45707795
				3041 300571.252868414
				3042 300582.048658879
				3043 300592.844449343
				3044 300603.640239808
				3045 300614.436030273
				3046 300625.231820737
				3047 300636.027611202
				3048 300646.823401666
				3049 300657.619192131
				3050 300668.414982596
				3051 300679.21077306
				3052 300690.006563525
				3053 300700.802353989
				3054 300711.598144454
				3055 300722.393934918
				3056 300733.189725383
				3057 300743.985515848
				3058 300754.781306312
				3059 300765.577096777
				3060 300776.372887241
				3061 300787.168677706
				3062 300797.96446817
				3063 300808.760258635
				3064 300819.5560491
				3065 300830.351839564
				3066 300841.147630029
				3067 300851.943420493
				3068 300862.739210958
				3069 300873.535001423
				3070 300884.330791887
				3071 300895.126582352
				3072 300905.922372816
				3073 300916.718163281
				3074 300927.513953745
				3075 300938.30974421
				3076 300949.105534675
				3077 300959.901325139
				3078 300970.697115604
				3079 300981.492906068
				3080 300992.288696533
				3081 301003.084486998
				3082 301013.880277462
				3083 301024.676067927
				3084 301035.471858391
				3085 301046.267648856
				3086 301057.06343932
				3087 301067.859229785
				3088 301078.65502025
				3089 301089.450810714
				3090 301100.246601179
				3091 301111.042391643
				3092 301121.838182108
				3093 301132.633972572
				3094 301143.429763037
				3095 301154.225553502
				3096 301165.021343966
				3097 301175.817134431
				3098 301186.612924895
				3099 301197.40871536
				3100 301208.204505824
				3101 301219.000296289
				3102 301229.796086754
				3103 301240.591877218
				3104 301251.387667683
				3105 301262.183458147
				3106 301272.979248612
				3107 301283.775039077
				3108 301294.570829541
				3109 301305.366620006
				3110 301316.16241047
				3111 301326.958200935
				3112 301337.753991399
				3113 301348.549781864
				3114 301359.345572329
				3115 301370.141362793
				3116 301380.937153258
				3117 301391.732943722
				3118 301402.528734187
				3119 301413.324524652
				3120 301424.120315116
				3121 301434.916105581
				3122 301445.711896045
				3123 301456.50768651
				3124 301467.303476974
				3125 301478.099267439
				3126 301488.895057904
				3127 301499.690848368
				3128 301510.486638833
				3129 301521.282429297
				3130 301532.078219762
				3131 301542.874010226
				3132 301553.669800691
				3133 301564.465591156
				3134 301575.26138162
				3135 301586.057172085
				3136 301596.852962549
				3137 301607.648753014
				3138 301618.444543479
				3139 301629.240333943
				3140 301640.036124408
				3141 301650.831914872
				3142 301661.627705337
				3143 301672.423495801
				3144 301683.219286266
				3145 301694.015076731
				3146 301704.810867195
				3147 301715.60665766
				3148 301726.402448124
				3149 301737.198238589
				3150 301747.994029053
				3151 301758.789819518
				3152 301769.585609983
				3153 301780.381400447
				3154 301791.177190912
				3155 301801.972981376
				3156 301812.768771841
				3157 301823.564562306
				3158 301834.36035277
				3159 301845.156143235
				3160 301855.951933699
				3161 301866.747724164
				3162 301877.543514628
				3163 301888.339305093
				3164 301899.135095558
				3165 301909.930886022
				3166 301920.726676487
				3167 301931.522466951
				3168 301942.318257416
				3169 301953.114047881
				3170 301963.909838345
				3171 301974.70562881
				3172 301985.501419274
				3173 301996.297209739
				3174 302007.093000203
				3175 302017.888790668
				3176 302028.684581133
				3177 302039.480371597
				3178 302050.276162062
				3179 302061.071952526
				3180 302071.867742991
				3181 302082.663533455
				3182 302093.45932392
				3183 302104.255114385
				3184 302115.050904849
				3185 302125.846695314
				3186 302136.642485778
				3187 302147.438276243
				3188 302158.234066708
				3189 302169.029857172
				3190 302179.825647637
				3191 302190.621438101
				3192 302201.417228566
				3193 302212.21301903
				3194 302223.008809495
				3195 302233.80459996
				3196 302244.600390424
				3197 302255.396180889
				3198 302266.191971353
				3199 302276.987761818
				3200 302287.783552282
				3201 302298.579342747
				3202 302309.375133212
				3203 302320.170923676
				3204 302330.966714141
				3205 302341.762504605
				3206 302352.55829507
				3207 302363.354085535
				3208 302374.149875999
				3209 302384.945666464
				3210 302395.741456928
				3211 302406.537247393
				3212 302417.333037857
				3213 302428.128828322
				3214 302438.924618787
				3215 302449.720409251
				3216 302460.516199716
				3217 302471.31199018
				3218 302482.107780645
				3219 302492.90357111
				3220 302503.699361574
				3221 302514.495152039
				3222 302525.290942503
				3223 302536.086732968
				3224 302546.882523432
				3225 302557.678313897
				3226 302568.474104362
				3227 302579.269894826
				3228 302590.065685291
				3229 302600.861475755
				3230 302611.65726622
				3231 302622.453056684
				3232 302633.248847149
				3233 302644.044637614
				3234 302654.840428078
				3235 302665.636218543
				3236 302676.432009007
				3237 302687.227799472
				3238 302698.023589937
				3239 302708.819380401
				3240 302719.615170866
				3241 302730.41096133
				3242 302741.206751795
				3243 302752.002542259
				3244 302762.798332724
				3245 302773.594123189
				3246 302784.389913653
				3247 302795.185704118
				3248 302805.981494582
				3249 302816.777285047
				3250 302827.573075511
				3251 302838.368865976
				3252 302849.164656441
				3253 302859.960446905
				3254 302870.75623737
				3255 302881.552027834
				3256 302892.347818299
				3257 302903.143608764
				3258 302913.939399228
				3259 302924.735189693
				3260 302935.530980157
				3261 302946.326770622
				3262 302957.122561086
				3263 302967.918351551
				3264 302978.714142016
				3265 302989.50993248
				3266 303000.305722945
				3267 303011.101513409
				3268 303021.897303874
				3269 303032.693094339
				3270 303043.488884803
				3271 303054.284675268
				3272 303065.080465732
				3273 303075.876256197
				3274 303086.672046661
				3275 303097.467837126
				3276 303108.263627591
				3277 303119.059418055
				3278 303129.85520852
				3279 303140.650998984
				3280 303151.446789449
				3281 303162.242579913
				3282 303173.038370378
				3283 303183.834160843
				3284 303194.629951307
				3285 303205.425741772
				3286 303216.221532236
				3287 303227.017322701
				3288 303237.813113165
				3289 303248.60890363
				3290 303259.404694095
				3291 303270.200484559
				3292 303280.996275024
				3293 303291.792065488
				3294 303302.587855953
				3295 303313.383646418
				3296 303324.179436882
				3297 303334.975227347
				3298 303345.771017811
				3299 303356.566808276
				3300 303367.36259874
				3301 303378.158389205
				3302 303388.95417967
				3303 303399.749970134
				3304 303410.545760599
				3305 303421.341551063
				3306 303432.137341528
				3307 303442.933131993
				3308 303453.728922457
				3309 303464.524712922
				3310 303475.320503386
				3311 303486.116293851
				3312 303496.912084315
				3313 303507.70787478
				3314 303518.503665245
				3315 303529.299455709
				3316 303540.095246174
				3317 303550.891036638
				3318 303561.686827103
				3319 303572.482617567
				3320 303583.278408032
				3321 303594.074198497
				3322 303604.869988961
				3323 303615.665779426
				3324 303626.46156989
				3325 303637.257360355
				3326 303648.05315082
				3327 303658.848941284
				3328 303669.644731749
				3329 303680.440522213
				3330 303691.236312678
				3331 303702.032103142
				3332 303712.827893607
				3333 303723.623684072
				3334 303734.419474536
				3335 303745.215265001
				3336 303756.011055465
				3337 303766.80684593
				3338 303777.602636394
				3339 303788.398426859
				3340 303799.194217324
				3341 303809.990007788
				3342 303820.785798253
				3343 303831.581588717
				3344 303842.377379182
				3345 303853.173169647
				3346 303863.968960111
				3347 303874.764750576
				3348 303885.56054104
				3349 303896.356331505
				3350 303907.152121969
				3351 303917.947912434
				3352 303928.743702899
				3353 303939.539493363
				3354 303950.335283828
				3355 303961.131074292
				3356 303971.926864757
				3357 303982.722655222
				3358 303993.518445686
				3359 304004.314236151
				3360 304015.110026615
				3361 304025.90581708
				3362 304036.701607544
				3363 304047.497398009
				3364 304058.293188474
				3365 304069.088978938
				3366 304079.884769403
				3367 304090.680559867
				3368 304101.476350332
				3369 304112.272140796
				3370 304123.067931261
				3371 304133.863721726
				3372 304144.65951219
				3373 304155.455302655
				3374 304166.251093119
				3375 304177.046883584
				3376 304187.842674049
				3377 304198.638464513
				3378 304209.434254978
				3379 304220.230045442
				3380 304231.025835907
				3381 304241.821626371
				3382 304252.617416836
				3383 304263.413207301
				3384 304274.208997765
				3385 304285.00478823
				3386 304295.800578694
				3387 304306.596369159
				3388 304317.392159624
				3389 304328.187950088
				3390 304338.983740553
				3391 304349.779531017
				3392 304360.575321482
				3393 304371.371111946
				3394 304382.166902411
				3395 304392.962692876
				3396 304403.75848334
				3397 304414.554273805
				3398 304425.350064269
				3399 304436.145854734
				3400 304446.941645198
				3401 304457.737435663
				3402 304468.533226128
				3403 304479.329016592
				3404 304490.124807057
				3405 304500.920597521
				3406 304511.716387986
				3407 304522.512178451
				3408 304533.307968915
				3409 304544.10375938
				3410 304554.899549844
				3411 304565.695340309
				3412 304576.491130773
				3413 304587.286921238
				3414 304598.082711703
				3415 304608.878502167
				3416 304619.674292632
				3417 304630.470083096
				3418 304641.265873561
				3419 304652.061664025
				3420 304662.85745449
				3421 304673.653244955
				3422 304684.449035419
				3423 304695.244825884
				3424 304706.040616348
				3425 304716.836406813
				3426 304727.632197278
				3427 304738.427987742
				3428 304749.223778207
				3429 304760.019568671
				3430 304770.815359136
				3431 304781.6111496
				3432 304792.406940065
				3433 304803.20273053
				3434 304813.998520994
				3435 304824.794311459
				3436 304835.590101923
				3437 304846.385892388
				3438 304857.181682853
				3439 304867.977473317
				3440 304878.773263782
				3441 304889.569054246
				3442 304900.364844711
				3443 304911.160635175
				3444 304921.95642564
				3445 304932.752216105
				3446 304943.548006569
				3447 304954.343797034
				3448 304965.139587498
				3449 304975.935377963
				3450 304986.731168427
				3451 304997.526958892
				3452 305008.322749357
				3453 305019.118539821
				3454 305029.914330286
				3455 305040.71012075
				3456 305051.505911215
				3457 305062.301701679
				3458 305073.097492144
				3459 305083.893282609
				3460 305094.689073073
				3461 305105.484863538
				3462 305116.280654002
				3463 305127.076444467
				3464 305137.872234932
				3465 305148.668025396
				3466 305159.463815861
				3467 305170.259606325
				3468 305181.05539679
				3469 305191.851187254
				3470 305202.646977719
				3471 305213.442768184
				3472 305224.238558648
				3473 305235.034349113
				3474 305245.830139577
				3475 305256.625930042
				3476 305267.421720507
				3477 305278.217510971
				3478 305289.013301436
				3479 305299.8090919
				3480 305310.604882365
				3481 305321.400672829
				3482 305332.196463294
				3483 305342.992253759
				3484 305353.788044223
				3485 305364.583834688
				3486 305375.379625152
				3487 305386.175415617
				3488 305396.971206081
				3489 305407.766996546
				3490 305418.562787011
				3491 305429.358577475
				3492 305440.15436794
				3493 305450.950158404
				3494 305461.745948869
				3495 305472.541739334
				3496 305483.337529798
				3497 305494.133320263
				3498 305504.929110727
				3499 305515.724901192
				3500 305526.520691656
				3501 305537.316482121
				3502 305548.112272586
				3503 305558.90806305
				3504 305569.703853515
				3505 305580.499643979
				3506 305591.295434444
				3507 305602.091224908
				3508 305612.887015373
				3509 305623.682805838
				3510 305634.478596302
				3511 305645.274386767
				3512 305656.070177231
				3513 305666.865967696
				3514 305677.661758161
				3515 305688.457548625
				3516 305699.25333909
				3517 305710.049129554
				3518 305720.844920019
				3519 305731.640710483
				3520 305742.436500948
				3521 305753.232291413
				3522 305764.028081877
				3523 305774.823872342
				3524 305785.619662806
				3525 305796.415453271
				3526 305807.211243736
				3527 305818.0070342
				3528 305828.802824665
				3529 305839.598615129
				3530 305850.394405594
				3531 305861.190196058
				3532 305871.985986523
				3533 305882.781776988
				3534 305893.577567452
				3535 305904.373357917
				3536 305915.169148381
				3537 305925.964938846
				3538 305936.76072931
				3539 305947.556519775
				3540 305958.35231024
				3541 305969.148100704
				3542 305979.943891169
				3543 305990.739681633
				3544 306001.535472098
				3545 306012.331262563
				3546 306023.127053027
				3547 306033.922843492
				3548 306044.718633956
				3549 306055.514424421
				3550 306066.310214885
				3551 306077.10600535
				3552 306087.901795815
				3553 306098.697586279
				3554 306109.493376744
				3555 306120.289167208
				3556 306131.084957673
				3557 306141.880748137
				3558 306152.676538602
				3559 306163.472329067
				3560 306174.268119531
				3561 306185.063909996
				3562 306195.85970046
				3563 306206.655490925
				3564 306217.45128139
				3565 306228.247071854
				3566 306239.042862319
				3567 306249.838652783
				3568 306260.634443248
				3569 306271.430233712
				3570 306282.226024177
				3571 306293.021814642
				3572 306303.817605106
				3573 306314.613395571
				3574 306325.409186035
				3575 306336.2049765
				3576 306347.000766965
				3577 306357.796557429
				3578 306368.592347894
				3579 306379.388138358
				3580 306390.183928823
				3581 306400.979719287
				3582 306411.775509752
				3583 306422.571300217
				3584 306433.367090681
				3585 306444.162881146
				3586 306454.95867161
				3587 306465.754462075
				3588 306476.550252539
				3589 306487.346043004
				3590 306498.141833469
				3591 306508.937623933
				3592 306519.733414398
				3593 306530.529204862
				3594 306541.324995327
				3595 306552.120785792
				3596 306562.916576256
				3597 306573.712366721
				3598 306584.508157185
				3599 306595.30394765
				3600 306606.099738114
				3601 306616.895528579
				3602 306627.691319044
				3603 306638.487109508
				3604 306649.282899973
				3605 306660.078690437
				3606 306670.874480902
				3607 306681.670271366
				3608 306692.466061831
				3609 306703.261852296
				3610 306714.05764276
				3611 306724.853433225
				3612 306735.649223689
				3613 306746.445014154
				3614 306757.240804619
				3615 306768.036595083
				3616 306778.832385548
				3617 306789.628176012
				3618 306800.423966477
				3619 306811.219756941
				3620 306822.015547406
				3621 306832.811337871
				3622 306843.607128335
				3623 306854.4029188
				3624 306865.198709264
				3625 306875.994499729
				3626 306886.790290194
				3627 306897.586080658
				3628 306908.381871123
				3629 306919.177661587
				3630 306929.973452052
				3631 306940.769242516
				3632 306951.565032981
				3633 306962.360823446
				3634 306973.15661391
				3635 306983.952404375
				3636 306994.748194839
				3637 307005.543985304
				3638 307016.339775768
				3639 307027.135566233
				3640 307037.931356698
				3641 307048.727147162
				3642 307059.522937627
				3643 307070.318728091
				3644 307081.114518556
				3645 307091.91030902
				3646 307102.706099485
				3647 307113.50188995
				3648 307124.297680414
				3649 307135.093470879
				3650 307145.889261343
				3651 307156.685051808
				3652 307167.480842273
				3653 307178.276632737
				3654 307189.072423202
				3655 307199.868213666
				3656 307210.664004131
				3657 307221.459794595
				3658 307232.25558506
				3659 307243.051375525
				3660 307253.847165989
				3661 307264.642956454
				3662 307275.438746918
				3663 307286.234537383
				3664 307297.030327848
				3665 307307.826118312
				3666 307318.621908777
				3667 307329.417699241
				3668 307340.213489706
				3669 307351.00928017
				3670 307361.805070635
				3671 307372.6008611
				3672 307383.396651564
				3673 307394.192442029
				3674 307404.988232493
				3675 307415.784022958
				3676 307426.579813422
				3677 307437.375603887
				3678 307448.171394352
				3679 307458.967184816
				3680 307469.762975281
				3681 307480.558765745
				3682 307491.35455621
				3683 307502.150346675
				3684 307512.946137139
				3685 307523.741927604
				3686 307534.537718068
				3687 307545.333508533
				3688 307556.129298997
				3689 307566.925089462
				3690 307577.720879927
				3691 307588.516670391
				3692 307599.312460856
				3693 307610.10825132
				3694 307620.904041785
				3695 307631.699832249
				3696 307642.495622714
				3697 307653.291413179
				3698 307664.087203643
				3699 307674.882994108
				3700 307685.678784572
				3701 307696.474575037
				3702 307707.270365502
				3703 307718.066155966
				3704 307728.861946431
				3705 307739.657736895
				3706 307750.45352736
				3707 307761.249317824
				3708 307772.045108289
				3709 307782.840898754
				3710 307793.636689218
				3711 307804.432479683
				3712 307815.228270147
				3713 307826.024060612
				3714 307836.819851077
				3715 307847.615641541
				3716 307858.411432006
				3717 307869.20722247
				3718 307880.003012935
				3719 307890.798803399
				3720 307901.594593864
				3721 307912.390384329
				3722 307923.186174793
				3723 307933.981965258
				3724 307944.777755722
				3725 307955.573546187
				3726 307966.369336651
				3727 307977.165127116
				3728 307987.960917581
				3729 307998.756708045
				3730 308009.55249851
				3731 308020.348288974
				3732 308031.144079439
				3733 308041.939869904
				3734 308052.735660368
				3735 308063.531450833
				3736 308074.327241297
				3737 308085.123031762
				3738 308095.918822226
				3739 308106.714612691
				3740 308117.510403156
				3741 308128.30619362
				3742 308139.101984085
				3743 308149.897774549
				3744 308160.693565014
				3745 308171.489355478
				3746 308182.285145943
				3747 308193.080936408
				3748 308203.876726872
				3749 308214.672517337
				3750 308225.468307801
				3751 308236.264098266
				3752 308247.059888731
				3753 308257.855679195
				3754 308268.65146966
				3755 308279.447260124
				3756 308290.243050589
				3757 308301.038841053
				3758 308311.834631518
				3759 308322.630421983
				3760 308333.426212447
				3761 308344.222002912
				3762 308355.017793376
				3763 308365.813583841
				3764 308376.609374306
				3765 308387.40516477
				3766 308398.200955235
				3767 308408.996745699
				3768 308419.792536164
				3769 308430.588326628
				3770 308441.384117093
				3771 308452.179907558
				3772 308462.975698022
				3773 308473.771488487
				3774 308484.567278951
				3775 308495.363069416
				3776 308506.15885988
				3777 308516.954650345
				3778 308527.75044081
				3779 308538.546231274
				3780 308549.342021739
				3781 308560.137812203
				3782 308570.933602668
				3783 308581.729393132
				3784 308592.525183597
				3785 308603.320974062
				3786 308614.116764526
				3787 308624.912554991
				3788 308635.708345455
				3789 308646.50413592
				3790 308657.299926385
				3791 308668.095716849
				3792 308678.891507314
				3793 308689.687297778
				3794 308700.483088243
				3795 308711.278878707
				3796 308722.074669172
				3797 308732.870459637
				3798 308743.666250101
				3799 308754.462040566
				3800 308765.25783103
				3801 308776.053621495
				3802 308786.84941196
				3803 308797.645202424
				3804 308808.440992889
				3805 308819.236783353
				3806 308830.032573818
				3807 308840.828364282
				3808 308851.624154747
				3809 308862.419945212
				3810 308873.215735676
				3811 308884.011526141
				3812 308894.807316605
				3813 308905.60310707
				3814 308916.398897534
				3815 308927.194687999
				3816 308937.990478464
				3817 308948.786268928
				3818 308959.582059393
				3819 308970.377849857
				3820 308981.173640322
				3821 308991.969430787
				3822 309002.765221251
				3823 309013.561011716
				3824 309024.35680218
				3825 309035.152592645
				3826 309045.948383109
				3827 309056.744173574
				3828 309067.539964039
				3829 309078.335754503
				3830 309089.131544968
				3831 309099.927335432
				3832 309110.723125897
				3833 309121.518916361
				3834 309132.314706826
				3835 309143.110497291
				3836 309153.906287755
				3837 309164.70207822
				3838 309175.497868684
				3839 309186.293659149
				3840 309197.089449614
				3841 309207.885240078
				3842 309218.681030543
				3843 309229.476821007
				3844 309240.272611472
				3845 309251.068401936
				3846 309261.864192401
				3847 309272.659982866
				3848 309283.45577333
				3849 309294.251563795
				3850 309305.047354259
				3851 309315.843144724
				3852 309326.638935189
				3853 309337.434725653
				3854 309348.230516118
				3855 309359.026306582
				3856 309369.822097047
				3857 309380.617887511
				3858 309391.413677976
				3859 309402.209468441
				3860 309413.005258905
				3861 309423.80104937
				3862 309434.596839834
				3863 309445.392630299
				3864 309456.188420763
				3865 309466.984211228
				3866 309477.780001693
				3867 309488.575792157
				3868 309499.371582622
				3869 309510.167373086
				3870 309520.963163551
				3871 309531.758954016
				3872 309542.55474448
				3873 309553.350534945
				3874 309564.146325409
				3875 309574.942115874
				3876 309585.737906338
				3877 309596.533696803
				3878 309607.329487268
				3879 309618.125277732
				3880 309628.921068197
				3881 309639.716858661
				3882 309650.512649126
				3883 309661.308439591
				3884 309672.104230055
				3885 309682.90002052
				3886 309693.695810984
				3887 309704.491601449
				3888 309715.287391913
				3889 309726.083182378
				3890 309736.878972843
				3891 309747.674763307
				3892 309758.470553772
				3893 309769.266344236
				3894 309780.062134701
				3895 309790.857925165
				3896 309801.65371563
				3897 309812.449506095
				3898 309823.245296559
				3899 309834.041087024
				3900 309844.836877488
				3901 309855.632667953
				3902 309866.428458418
				3903 309877.224248882
				3904 309888.020039347
				3905 309898.815829811
				3906 309909.611620276
				3907 309920.40741074
				3908 309931.203201205
				3909 309941.99899167
				3910 309952.794782134
				3911 309963.590572599
				3912 309974.386363063
				3913 309985.182153528
				3914 309995.977943992
				3915 310006.773734457
				3916 310017.569524922
				3917 310028.365315386
				3918 310039.161105851
				3919 310049.956896315
				3920 310060.75268678
				3921 310071.548477245
				3922 310082.344267709
				3923 310093.140058174
				3924 310103.935848638
				3925 310114.731639103
				3926 310125.527429567
				3927 310136.323220032
				3928 310147.119010497
				3929 310157.914800961
				3930 310168.710591426
				3931 310179.50638189
				3932 310190.302172355
				3933 310201.09796282
				3934 310211.893753284
				3935 310222.689543749
				3936 310233.485334213
				3937 310244.281124678
				3938 310255.076915142
				3939 310265.872705607
				3940 310276.668496072
				3941 310287.464286536
				3942 310298.260077001
				3943 310309.055867465
				3944 310319.85165793
				3945 310330.647448394
				3946 310341.443238859
				3947 310352.239029324
				3948 310363.034819788
				3949 310373.830610253
				3950 310384.626400717
				3951 310395.422191182
				3952 310406.217981647
				3953 310417.013772111
				3954 310427.809562576
				3955 310438.60535304
				3956 310449.401143505
				3957 310460.196933969
				3958 310470.992724434
				3959 310481.788514899
				3960 310492.584305363
				3961 310503.380095828
				3962 310514.175886292
				3963 310524.971676757
				3964 310535.767467221
				3965 310546.563257686
				3966 310557.359048151
				3967 310568.154838615
				3968 310578.95062908
				3969 310589.746419544
				3970 310600.542210009
				3971 310611.338000474
				3972 310622.133790938
				3973 310632.929581403
				3974 310643.725371867
				3975 310654.521162332
				3976 310665.316952796
				3977 310676.112743261
				3978 310686.908533726
				3979 310697.70432419
				3980 310708.500114655
				3981 310719.295905119
				3982 310730.091695584
				3983 310740.887486049
				3984 310751.683276513
				3985 310762.479066978
				3986 310773.274857442
				3987 310784.070647907
				3988 310794.866438371
				3989 310805.662228836
				3990 310816.458019301
				3991 310827.253809765
				3992 310838.04960023
				3993 310848.845390694
				3994 310859.641181159
				3995 310870.436971623
				3996 310881.232762088
				3997 310892.028552553
				3998 310902.824343017
				3999 310913.620133482
				4000 310924.415923946
				4001 310935.211714411
				4002 310946.007504875
				4003 310956.80329534
				4004 310967.599085805
				4005 310978.394876269
				4006 310989.190666734
				4007 310999.986457198
				4008 311010.782247663
				4009 311021.578038128
				4010 311032.373828592
				4011 311043.169619057
				4012 311053.965409521
				4013 311064.761199986
				4014 311075.55699045
				4015 311086.352780915
				4016 311097.14857138
				4017 311107.944361844
				4018 311118.740152309
				4019 311129.535942773
				4020 311140.331733238
				4021 311151.127523703
				4022 311161.923314167
				4023 311172.719104632
				4024 311183.514895096
				4025 311194.310685561
				4026 311205.106476025
				4027 311215.90226649
				4028 311226.698056955
				4029 311237.493847419
				4030 311248.289637884
				4031 311259.085428348
				4032 311269.881218813
				4033 311280.677009277
				4034 311291.472799742
				4035 311302.268590207
				4036 311313.064380671
				4037 311323.860171136
				4038 311334.6559616
				4039 311345.451752065
				4040 311356.24754253
				4041 311367.043332994
				4042 311377.839123459
				4043 311388.634913923
				4044 311399.430704388
				4045 311410.226494852
				4046 311421.022285317
				4047 311431.818075782
				4048 311442.613866246
				4049 311453.409656711
				4050 311464.205447175
				4051 311475.00123764
				4052 311485.797028104
				4053 311496.592818569
				4054 311507.388609034
				4055 311518.184399498
				4056 311528.980189963
				4057 311539.775980427
				4058 311550.571770892
				4059 311561.367561357
				4060 311572.163351821
				4061 311582.959142286
				4062 311593.75493275
				4063 311604.550723215
				4064 311615.346513679
				4065 311626.142304144
				4066 311636.938094609
				4067 311647.733885073
				4068 311658.529675538
				4069 311669.325466002
				4070 311680.121256467
				4071 311690.917046932
				4072 311701.712837396
				4073 311712.508627861
				4074 311723.304418325
				4075 311734.10020879
				4076 311744.895999254
				4077 311755.691789719
				4078 311766.487580184
				4079 311777.283370648
				4080 311788.079161113
				4081 311798.874951577
				4082 311809.670742042
				4083 311820.466532506
				4084 311831.262322971
				4085 311842.058113436
				4086 311852.8539039
				4087 311863.649694365
				4088 311874.445484829
				4089 311885.241275294
				4090 311896.037065759
				4091 311906.832856223
				4092 311917.628646688
				4093 311928.424437152
				4094 311939.220227617
				4095 311950.016018081
				4096 311960.811808546
				4097 311971.607599011
				4098 311982.403389475
				4099 311993.19917994
				4100 312003.994970404
				4101 312014.790760869
				4102 312025.586551333
				4103 312036.382341798
				4104 312047.178132263
				4105 312057.973922727
				4106 312068.769713192
				4107 312079.565503656
				4108 312090.361294121
				4109 312101.157084586
				4110 312111.95287505
				4111 312122.748665515
				4112 312133.544455979
				4113 312144.340246444
				4114 312155.136036908
				4115 312165.931827373
				4116 312176.727617838
				4117 312187.523408302
				4118 312198.319198767
				4119 312209.114989231
				4120 312219.910779696
				4121 312230.706570161
				4122 312241.502360625
				4123 312252.29815109
				4124 312263.093941554
				4125 312273.889732019
				4126 312284.685522483
				4127 312295.481312948
				4128 312306.277103413
				4129 312317.072893877
				4130 312327.868684342
				4131 312338.664474806
				4132 312349.460265271
				4133 312360.256055735
				4134 312371.0518462
				4135 312381.847636665
				4136 312392.643427129
				4137 312403.439217594
				4138 312414.235008058
				4139 312425.030798523
				4140 312435.826588987
				4141 312446.622379452
				4142 312457.418169917
				4143 312468.213960381
				4144 312479.009750846
				4145 312489.80554131
				4146 312500.601331775
				4147 312511.39712224
				4148 312522.192912704
				4149 312532.988703169
				4150 312543.784493633
				4151 312554.580284098
				4152 312565.376074562
				4153 312576.171865027
				4154 312586.967655492
				4155 312597.763445956
				4156 312608.559236421
				4157 312619.355026885
				4158 312630.15081735
				4159 312640.946607815
				4160 312651.742398279
				4161 312662.538188744
				4162 312673.333979208
				4163 312684.129769673
				4164 312694.925560137
				4165 312705.721350602
				4166 312716.517141067
				4167 312727.312931531
				4168 312738.108721996
				4169 312748.90451246
				4170 312759.700302925
				4171 312770.496093389
				4172 312781.291883854
				4173 312792.087674319
				4174 312802.883464783
				4175 312813.679255248
				4176 312824.475045712
				4177 312835.270836177
				4178 312846.066626642
				4179 312856.862417106
				4180 312867.658207571
				4181 312878.453998035
				4182 312889.2497885
				4183 312900.045578964
				4184 312910.841369429
				4185 312921.637159894
				4186 312932.432950358
				4187 312943.228740823
				4188 312954.024531287
				4189 312964.820321752
				4190 312975.616112216
				4191 312986.411902681
				4192 312997.207693146
				4193 313008.00348361
				4194 313018.799274075
				4195 313029.595064539
				4196 313040.390855004
				4197 313051.186645469
				4198 313061.982435933
				4199 313072.778226398
				4200 313083.574016862
				4201 313094.369807327
				4202 313105.165597791
				4203 313115.961388256
				4204 313126.757178721
				4205 313137.552969185
				4206 313148.34875965
				4207 313159.144550114
				4208 313169.940340579
				4209 313180.736131044
				4210 313191.531921508
				4211 313202.327711973
				4212 313213.123502437
				4213 313223.919292902
				4214 313234.715083366
				4215 313245.510873831
				4216 313256.306664296
				4217 313267.10245476
				4218 313277.898245225
				4219 313288.694035689
				4220 313299.489826154
				4221 313310.285616618
				4222 313321.081407083
				4223 313331.877197548
				4224 313342.672988012
				4225 313353.468778477
				4226 313364.264568941
				4227 313375.060359406
				4228 313385.856149871
				4229 313396.651940335
				4230 313407.4477308
				4231 313418.243521264
				4232 313429.039311729
				4233 313439.835102193
				4234 313450.630892658
				4235 313461.426683123
				4236 313472.222473587
				4237 313483.018264052
				4238 313493.814054516
				4239 313504.609844981
				4240 313515.405635445
				4241 313526.20142591
				4242 313536.997216375
				4243 313547.793006839
				4244 313558.588797304
				4245 313569.384587768
				4246 313580.180378233
				4247 313590.976168698
				4248 313601.771959162
				4249 313612.567749627
				4250 313623.363540091
				4251 313634.159330556
				4252 313644.95512102
				4253 313655.750911485
				4254 313666.54670195
				4255 313677.342492414
				4256 313688.138282879
				4257 313698.934073343
				4258 313709.729863808
				4259 313720.525654273
				4260 313731.321444737
				4261 313742.117235202
				4262 313752.913025666
				4263 313763.708816131
				4264 313774.504606595
				4265 313785.30039706
				4266 313796.096187525
				4267 313806.891977989
				4268 313817.687768454
				4269 313828.483558918
				4270 313839.279349383
				4271 313850.075139847
				4272 313860.870930312
				4273 313871.666720777
				4274 313882.462511241
				4275 313893.258301706
				4276 313904.05409217
				4277 313914.849882635
				4278 313925.6456731
				4279 313936.441463564
				4280 313947.237254029
				4281 313958.033044493
				4282 313968.828834958
				4283 313979.624625422
				4284 313990.420415887
				4285 314001.216206352
				4286 314012.011996816
				4287 314022.807787281
				4288 314033.603577745
				4289 314044.39936821
				4290 314055.195158674
				4291 314065.990949139
				4292 314076.786739604
				4293 314087.582530068
				4294 314098.378320533
				4295 314109.174110997
				4296 314119.969901462
				4297 314130.765691927
				4298 314141.561482391
				4299 314152.357272856
				4300 314163.15306332
				4301 314173.948853785
				4302 314184.744644249
				4303 314195.540434714
				4304 314206.336225179
				4305 314217.132015643
				4306 314227.927806108
				4307 314238.723596572
				4308 314249.519387037
				4309 314260.315177502
				4310 314271.110967966
				4311 314281.906758431
				4312 314292.702548895
				4313 314303.49833936
				4314 314314.294129824
				4315 314325.089920289
				4316 314335.885710754
				4317 314346.681501218
				4318 314357.477291683
				4319 314368.273082147
				4320 314379.068872612
				4321 314389.864663076
				4322 314400.660453541
				4323 314411.456244006
				4324 314422.25203447
				4325 314433.047824935
				4326 314443.843615399
				4327 314454.639405864
				4328 314465.435196329
				4329 314476.230986793
				4330 314487.026777258
				4331 314497.822567722
				4332 314508.618358187
				4333 314519.414148651
				4334 314530.209939116
				4335 314541.005729581
				4336 314551.801520045
				4337 314562.59731051
				4338 314573.393100974
				4339 314584.188891439
				4340 314594.984681904
				4341 314605.780472368
				4342 314616.576262833
				4343 314627.372053297
				4344 314638.167843762
				4345 314648.963634226
				4346 314659.759424691
				4347 314670.555215156
				4348 314681.35100562
				4349 314692.146796085
				4350 314702.942586549
				4351 314713.738377014
				4352 314724.534167478
				4353 314735.329957943
				4354 314746.125748408
				4355 314756.921538872
				4356 314767.717329337
				4357 314778.513119801
				4358 314789.308910266
				4359 314800.10470073
				4360 314810.900491195
				4361 314821.69628166
				4362 314832.492072124
				4363 314843.287862589
				4364 314854.083653053
				4365 314864.879443518
				4366 314875.675233983
				4367 314886.471024447
				4368 314897.266814912
				4369 314908.062605376
				4370 314918.858395841
				4371 314929.654186305
				4372 314940.44997677
				4373 314951.245767235
				4374 314962.041557699
				4375 314972.837348164
				4376 314983.633138628
				4377 314994.428929093
				4378 315005.224719558
				4379 315016.020510022
				4380 315026.816300487
				4381 315037.612090951
				4382 315048.407881416
				4383 315059.20367188
				4384 315069.999462345
				4385 315080.79525281
				4386 315091.591043274
				4387 315102.386833739
				4388 315113.182624203
				4389 315123.978414668
				4390 315134.774205132
				4391 315145.569995597
				4392 315156.365786062
				4393 315167.161576526
				4394 315177.957366991
				4395 315188.753157455
				4396 315199.54894792
				4397 315210.344738385
				4398 315221.140528849
				4399 315231.936319314
				4400 315242.732109778
				4401 315253.527900243
				4402 315264.323690707
				4403 315275.119481172
				4404 315285.915271637
				4405 315296.711062101
				4406 315307.506852566
				4407 315318.30264303
				4408 315329.098433495
				4409 315339.894223959
				4410 315350.690014424
				4411 315361.485804889
				4412 315372.281595353
				4413 315383.077385818
				4414 315393.873176282
				4415 315404.668966747
				4416 315415.464757212
				4417 315426.260547676
				4418 315437.056338141
				4419 315447.852128605
				4420 315458.64791907
				4421 315469.443709534
				4422 315480.239499999
				4423 315491.035290464
				4424 315501.831080928
				4425 315512.626871393
				4426 315523.422661857
				4427 315534.218452322
				4428 315545.014242787
				4429 315555.810033251
				4430 315566.605823716
				4431 315577.40161418
				4432 315588.197404645
				4433 315598.993195109
				4434 315609.788985574
				4435 315620.584776039
				4436 315631.380566503
				4437 315642.176356968
				4438 315652.972147432
				4439 315663.767937897
				4440 315674.563728361
				4441 315685.359518826
				4442 315696.155309291
				4443 315706.951099755
				4444 315717.74689022
				4445 315728.542680684
				4446 315739.338471149
				4447 315750.134261614
				4448 315760.930052078
				4449 315771.725842543
				4450 315782.521633007
				4451 315793.317423472
				4452 315804.113213936
				4453 315814.909004401
				4454 315825.704794866
				4455 315836.50058533
				4456 315847.296375795
				4457 315858.092166259
				4458 315868.887956724
				4459 315879.683747188
				4460 315890.479537653
				4461 315901.275328118
				4462 315912.071118582
				4463 315922.866909047
				4464 315933.662699511
				4465 315944.458489976
				4466 315955.254280441
				4467 315966.050070905
				4468 315976.84586137
				4469 315987.641651834
				4470 315998.437442299
				4471 316009.233232763
				4472 316020.029023228
				4473 316030.824813693
				4474 316041.620604157
				4475 316052.416394622
				4476 316063.212185086
				4477 316074.007975551
				4478 316084.803766016
				4479 316095.59955648
				4480 316106.395346945
				4481 316117.191137409
				4482 316127.986927874
				4483 316138.782718338
				4484 316149.578508803
				4485 316160.374299268
				4486 316171.170089732
				4487 316181.965880197
				4488 316192.761670661
				4489 316203.557461126
				4490 316214.35325159
				4491 316225.149042055
				4492 316235.94483252
				4493 316246.740622984
				4494 316257.536413449
				4495 316268.332203913
				4496 316279.127994378
				4497 316289.923784842
				4498 316300.719575307
				4499 316311.515365772
				4500 316322.311156236
				4501 316333.106946701
				4502 316343.902737165
				4503 316354.69852763
				4504 316365.494318095
				4505 316376.290108559
				4506 316387.085899024
				4507 316397.881689488
				4508 316408.677479953
				4509 316419.473270417
				4510 316430.269060882
				4511 316441.064851347
				4512 316451.860641811
				4513 316462.656432276
				4514 316473.45222274
				4515 316484.248013205
				4516 316495.04380367
				4517 316505.839594134
				4518 316516.635384599
				4519 316527.431175063
				4520 316538.226965528
				4521 316549.022755992
				4522 316559.818546457
				4523 316570.614336922
				4524 316581.410127386
				4525 316592.205917851
				4526 316603.001708315
				4527 316613.79749878
				4528 316624.593289244
				4529 316635.389079709
				4530 316646.184870174
				4531 316656.980660638
				4532 316667.776451103
				4533 316678.572241567
				4534 316689.368032032
				4535 316700.163822497
				4536 316710.959612961
				4537 316721.755403426
				4538 316732.55119389
				4539 316743.346984355
				4540 316754.142774819
				4541 316764.938565284
				4542 316775.734355749
				4543 316786.530146213
				4544 316797.325936678
				4545 316808.121727142
				4546 316818.917517607
				4547 316829.713308071
				4548 316840.509098536
				4549 316851.304889001
				4550 316862.100679465
				4551 316872.89646993
				4552 316883.692260394
				4553 316894.488050859
				4554 316905.283841324
				4555 316916.079631788
				4556 316926.875422253
				4557 316937.671212717
				4558 316948.467003182
				4559 316959.262793646
				4560 316970.058584111
				4561 316980.854374576
				4562 316991.65016504
				4563 317002.445955505
				4564 317013.241745969
				4565 317024.037536434
				4566 317034.833326899
				4567 317045.629117363
				4568 317056.424907828
				4569 317067.220698292
				4570 317078.016488757
				4571 317088.812279221
				4572 317099.608069686
				4573 317110.403860151
				4574 317121.199650615
				4575 317131.99544108
				4576 317142.791231544
				4577 317153.587022009
				4578 317164.382812473
				4579 317175.178602938
				4580 317185.974393403
				4581 317196.770183867
				4582 317207.565974332
				4583 317218.361764796
				4584 317229.157555261
				4585 317239.953345726
				4586 317250.74913619
				4587 317261.544926655
				4588 317272.340717119
				4589 317283.136507584
				4590 317293.932298048
				4591 317304.728088513
				4592 317315.523878978
				4593 317326.319669442
				4594 317337.115459907
				4595 317347.911250371
				4596 317358.707040836
				4597 317369.5028313
				4598 317380.298621765
				4599 317391.09441223
				4600 317401.890202694
				4601 317412.685993159
				4602 317423.481783623
				4603 317434.277574088
				4604 317445.073364553
				4605 317455.869155017
				4606 317466.664945482
				4607 317477.460735946
				4608 317488.256526411
				4609 317499.052316875
				4610 317509.84810734
				4611 317520.643897805
				4612 317531.439688269
				4613 317542.235478734
				4614 317553.031269198
				4615 317563.827059663
				4616 317574.622850128
				4617 317585.418640592
				4618 317596.214431057
				4619 317607.010221521
				4620 317617.806011986
				4621 317628.60180245
				4622 317639.397592915
				4623 317650.19338338
				4624 317660.989173844
				4625 317671.784964309
				4626 317682.580754773
				4627 317693.376545238
				4628 317704.172335702
				4629 317714.968126167
				4630 317725.763916632
				4631 317736.559707096
				4632 317747.355497561
				4633 317758.151288025
				4634 317768.94707849
				4635 317779.742868955
				4636 317790.538659419
				4637 317801.334449884
				4638 317812.130240348
				4639 317822.926030813
				4640 317833.721821277
				4641 317844.517611742
				4642 317855.313402207
				4643 317866.109192671
				4644 317876.904983136
				4645 317887.7007736
				4646 317898.496564065
				4647 317909.292354529
				4648 317920.088144994
				4649 317930.883935459
				4650 317941.679725923
				4651 317952.475516388
				4652 317963.271306852
				4653 317974.067097317
				4654 317984.862887782
				4655 317995.658678246
				4656 318006.454468711
				4657 318017.250259175
				4658 318028.04604964
				4659 318038.841840104
				4660 318049.637630569
				4661 318060.433421034
				4662 318071.229211498
				4663 318082.025001963
				4664 318092.820792427
				4665 318103.616582892
				4666 318114.412373357
				4667 318125.208163821
				4668 318136.003954286
				4669 318146.79974475
				4670 318157.595535215
				4671 318168.391325679
				4672 318179.187116144
				4673 318189.982906609
				4674 318200.778697073
				4675 318211.574487538
				4676 318222.370278002
				4677 318233.166068467
				4678 318243.961858931
				4679 318254.757649396
				4680 318265.553439861
				4681 318276.349230325
				4682 318287.14502079
				4683 318297.940811254
				4684 318308.736601719
				4685 318319.532392183
				4686 318330.328182648
				4687 318341.123973113
				4688 318351.919763577
				4689 318362.715554042
				4690 318373.511344506
				4691 318384.307134971
				4692 318395.102925436
				4693 318405.8987159
				4694 318416.694506365
				4695 318427.490296829
				4696 318438.286087294
				4697 318449.081877758
				4698 318459.877668223
				4699 318470.673458688
				4700 318481.469249152
				4701 318492.265039617
				4702 318503.060830081
				4703 318513.856620546
				4704 318524.652411011
				4705 318535.448201475
				4706 318546.24399194
				4707 318557.039782404
				4708 318567.835572869
				4709 318578.631363333
				4710 318589.427153798
				4711 318600.222944263
				4712 318611.018734727
				4713 318621.814525192
				4714 318632.610315656
				4715 318643.406106121
				4716 318654.201896585
				4717 318664.99768705
				4718 318675.793477515
				4719 318686.589267979
				4720 318697.385058444
				4721 318708.180848908
				4722 318718.976639373
				4723 318729.772429838
				4724 318740.568220302
				4725 318751.364010767
				4726 318762.159801231
				4727 318772.955591696
				4728 318783.75138216
				4729 318794.547172625
				4730 318805.34296309
				4731 318816.138753554
				4732 318826.934544019
				4733 318837.730334483
				4734 318848.526124948
				4735 318859.321915412
				4736 318870.117705877
				4737 318880.913496342
				4738 318891.709286806
				4739 318902.505077271
				4740 318913.300867735
				4741 318924.0966582
				4742 318934.892448665
				4743 318945.688239129
				4744 318956.484029594
				4745 318967.279820058
				4746 318978.075610523
				4747 318988.871400987
				4748 318999.667191452
				4749 319010.462981917
				4750 319021.258772381
				4751 319032.054562846
				4752 319042.85035331
				4753 319053.646143775
				4754 319064.44193424
				4755 319075.237724704
				4756 319086.033515169
				4757 319096.829305633
				4758 319107.625096098
				4759 319118.420886562
				4760 319129.216677027
				4761 319140.012467492
				4762 319150.808257956
				4763 319161.604048421
				4764 319172.399838885
				4765 319183.19562935
				4766 319193.991419814
				4767 319204.787210279
				4768 319215.583000744
				4769 319226.378791208
				4770 319237.174581673
				4771 319247.970372137
				4772 319258.766162602
				4773 319269.561953066
				4774 319280.357743531
				4775 319291.153533996
				4776 319301.94932446
				4777 319312.745114925
				4778 319323.540905389
				4779 319334.336695854
				4780 319345.132486319
				4781 319355.928276783
				4782 319366.724067248
				4783 319377.519857712
				4784 319388.315648177
				4785 319399.111438641
				4786 319409.907229106
				4787 319420.703019571
				4788 319431.498810035
				4789 319442.2946005
				4790 319453.090390964
				4791 319463.886181429
				4792 319474.681971894
				4793 319485.477762358
				4794 319496.273552823
				4795 319507.069343287
				4796 319517.865133752
				4797 319528.660924216
				4798 319539.456714681
				4799 319550.252505146
				4800 319561.04829561
				4801 319571.844086075
				4802 319582.639876539
				4803 319593.435667004
				4804 319604.231457468
				4805 319615.027247933
				4806 319625.823038398
				4807 319636.618828862
				4808 319647.414619327
				4809 319658.210409791
				4810 319669.006200256
				4811 319679.801990721
				4812 319690.597781185
				4813 319701.39357165
				4814 319712.189362114
				4815 319722.985152579
				4816 319733.780943043
				4817 319744.576733508
				4818 319755.372523973
				4819 319766.168314437
				4820 319776.964104902
				4821 319787.759895366
				4822 319798.555685831
				4823 319809.351476296
				4824 319820.14726676
				4825 319830.943057225
				4826 319841.738847689
				4827 319852.534638154
				4828 319863.330428618
				4829 319874.126219083
				4830 319884.922009548
				4831 319895.717800012
				4832 319906.513590477
				4833 319917.309380941
				4834 319928.105171406
				4835 319938.900961871
				4836 319949.696752335
				4837 319960.4925428
				4838 319971.288333264
				4839 319982.084123729
				4840 319992.879914193
				4841 320003.675704658
				4842 320014.471495123
				4843 320025.267285587
				4844 320036.063076052
				4845 320046.858866516
				4846 320057.654656981
				4847 320068.450447445
				4848 320079.24623791
				4849 320090.042028375
				4850 320100.837818839
				4851 320111.633609304
				4852 320122.429399768
				4853 320133.225190233
				4854 320144.020980697
				4855 320154.816771162
				4856 320165.612561627
				4857 320176.408352091
				4858 320187.204142556
				4859 320197.99993302
				4860 320208.795723485
				4861 320219.59151395
				4862 320230.387304414
				4863 320241.183094879
				4864 320251.978885343
				4865 320262.774675808
				4866 320273.570466272
				4867 320284.366256737
				4868 320295.162047202
				4869 320305.957837666
				4870 320316.753628131
				4871 320327.549418595
				4872 320338.34520906
				4873 320349.140999525
				4874 320359.936789989
				4875 320370.732580454
				4876 320381.528370918
				4877 320392.324161383
				4878 320403.119951847
				4879 320413.915742312
				4880 320424.711532777
				4881 320435.507323241
				4882 320446.303113706
				4883 320457.09890417
				4884 320467.894694635
				4885 320478.690485099
				4886 320489.486275564
				4887 320500.282066029
				4888 320511.077856493
				4889 320521.873646958
				4890 320532.669437422
				4891 320543.465227887
				4892 320554.261018352
				4893 320565.056808816
				4894 320575.852599281
				4895 320586.648389745
				4896 320597.44418021
				4897 320608.239970674
				4898 320619.035761139
				4899 320629.831551604
				4900 320640.627342068
				4901 320651.423132533
				4902 320662.218922997
				4903 320673.014713462
				4904 320683.810503926
				4905 320694.606294391
				4906 320705.402084856
				4907 320716.19787532
				4908 320726.993665785
				4909 320737.789456249
				4910 320748.585246714
				4911 320759.381037179
				4912 320770.176827643
				4913 320780.972618108
				4914 320791.768408572
				4915 320802.564199037
				4916 320813.359989501
				4917 320824.155779966
				4918 320834.951570431
				4919 320845.747360895
				4920 320856.54315136
				4921 320867.338941824
				4922 320878.134732289
				4923 320888.930522754
				4924 320899.726313218
				4925 320910.522103683
				4926 320921.317894147
				4927 320932.113684612
				4928 320942.909475076
				4929 320953.705265541
				4930 320964.501056006
				4931 320975.29684647
				4932 320986.092636935
				4933 320996.888427399
				4934 321007.684217864
				4935 321018.480008328
				4936 321029.275798793
				4937 321040.071589258
				4938 321050.867379722
				4939 321061.663170187
				4940 321072.458960651
				4941 321083.254751116
				4942 321094.050541581
				4943 321104.846332045
				4944 321115.64212251
				4945 321126.437912974
				4946 321137.233703439
				4947 321148.029493903
				4948 321158.825284368
				4949 321169.621074833
				4950 321180.416865297
				4951 321191.212655762
				4952 321202.008446226
				4953 321212.804236691
				4954 321223.600027155
				4955 321234.39581762
				4956 321245.191608085
				4957 321255.987398549
				4958 321266.783189014
				4959 321277.578979478
				4960 321288.374769943
				4961 321299.170560408
				4962 321309.966350872
				4963 321320.762141337
				4964 321331.557931801
				4965 321342.353722266
				4966 321353.14951273
				4967 321363.945303195
				4968 321374.74109366
				4969 321385.536884124
				4970 321396.332674589
				4971 321407.128465053
				4972 321417.924255518
				4973 321428.720045983
				4974 321439.515836447
				4975 321450.311626912
				4976 321461.107417376
				4977 321471.903207841
				4978 321482.698998305
				4979 321493.49478877
				4980 321504.290579235
				4981 321515.086369699
				4982 321525.882160164
				4983 321536.677950628
				4984 321547.473741093
				4985 321558.269531557
				4986 321569.065322022
				4987 321579.861112487
				4988 321590.656902951
				4989 321601.452693416
				4990 321612.24848388
				4991 321623.044274345
				4992 321633.840064809
				4993 321644.635855274
				4994 321655.431645739
				4995 321666.227436203
				4996 321677.023226668
				4997 321687.819017132
				4998 321698.614807597
				4999 321709.410598062
				5000 321720.206388526
				5001 321731.002178991
				5002 321741.797969455
				5003 321752.59375992
				5004 321763.389550384
				5005 321774.185340849
				5006 321784.981131314
				5007 321795.776921778
				5008 321806.572712243
				5009 321817.368502707
				5010 321828.164293172
				5011 321838.960083637
				5012 321849.755874101
				5013 321860.551664566
				5014 321871.34745503
				5015 321882.143245495
				5016 321892.939035959
				5017 321903.734826424
				5018 321914.530616889
				5019 321925.326407353
				5020 321936.122197818
				5021 321946.917988282
				5022 321957.713778747
				5023 321968.509569212
				5024 321979.305359676
				5025 321990.101150141
				5026 322000.896940605
				5027 322011.69273107
				5028 322022.488521534
				5029 322033.284311999
				5030 322044.080102464
				5031 322054.875892928
				5032 322065.671683393
				5033 322076.467473857
				5034 322087.263264322
				5035 322098.059054786
				5036 322108.854845251
				5037 322119.650635716
				5038 322130.44642618
				5039 322141.242216645
				5040 322152.038007109
				5041 322162.833797574
				5042 322173.629588038
				5043 322184.425378503
				5044 322195.221168968
				5045 322206.016959432
				5046 322216.812749897
				5047 322227.608540361
				5048 322238.404330826
				5049 322249.200121291
				5050 322259.995911755
				5051 322270.79170222
				5052 322281.587492684
				5053 322292.383283149
				5054 322303.179073613
				5055 322313.974864078
				5056 322324.770654543
				5057 322335.566445007
				5058 322346.362235472
				5059 322357.158025936
				5060 322367.953816401
				5061 322378.749606866
				5062 322389.54539733
				5063 322400.341187795
				5064 322411.136978259
				5065 322421.932768724
				5066 322432.728559188
				5067 322443.524349653
				5068 322454.320140118
				5069 322465.115930582
				5070 322475.911721047
				5071 322486.707511511
				5072 322497.503301976
				5073 322508.29909244
				5074 322519.094882905
				5075 322529.89067337
				5076 322540.686463834
				5077 322551.482254299
				5078 322562.278044763
				5079 322573.073835228
				5080 322583.869625693
				5081 322594.665416157
				5082 322605.461206622
				5083 322616.256997086
				5084 322627.052787551
				5085 322637.848578015
				5086 322648.64436848
				5087 322659.440158945
				5088 322670.235949409
				5089 322681.031739874
				5090 322691.827530338
				5091 322702.623320803
				5092 322713.419111267
				5093 322724.214901732
				5094 322735.010692197
				5095 322745.806482661
				5096 322756.602273126
				5097 322767.39806359
				5098 322778.193854055
				5099 322788.98964452
				5100 322799.785434984
				5101 322810.581225449
				5102 322821.377015913
				5103 322832.172806378
				5104 322842.968596842
				5105 322853.764387307
				5106 322864.560177772
				5107 322875.355968236
				5108 322886.151758701
				5109 322896.947549165
				5110 322907.74333963
				5111 322918.539130095
				5112 322929.334920559
				5113 322940.130711024
				5114 322950.926501488
				5115 322961.722291953
				5116 322972.518082417
				5117 322983.313872882
				5118 322994.109663347
				5119 323004.905453811
				5120 323015.701244276
				5121 323026.49703474
				5122 323037.292825205
				5123 323048.088615669
				5124 323058.884406134
				5125 323069.680196599
				5126 323080.475987063
				5127 323091.271777528
				5128 323102.067567992
				5129 323112.863358457
				5130 323123.659148921
				5131 323134.454939386
				5132 323145.250729851
				5133 323156.046520315
				5134 323166.84231078
				5135 323177.638101244
				5136 323188.433891709
				5137 323199.229682174
				5138 323210.025472638
				5139 323220.821263103
				5140 323231.617053567
				5141 323242.412844032
				5142 323253.208634496
				5143 323264.004424961
				5144 323274.800215426
				5145 323285.59600589
				5146 323296.391796355
				5147 323307.187586819
				5148 323317.983377284
				5149 323328.779167749
				5150 323339.574958213
				5151 323350.370748678
				5152 323361.166539142
				5153 323371.962329607
				5154 323382.758120071
				5155 323393.553910536
				5156 323404.349701001
				5157 323415.145491465
				5158 323425.94128193
				5159 323436.737072394
				5160 323447.532862859
				5161 323458.328653323
				5162 323469.124443788
				5163 323479.920234253
				5164 323490.716024717
				5165 323501.511815182
				5166 323512.307605646
				5167 323523.103396111
				5168 323533.899186576
				5169 323544.69497704
				5170 323555.490767505
				5171 323566.286557969
				5172 323577.082348434
				5173 323587.878138898
				5174 323598.673929363
				5175 323609.469719828
				5176 323620.265510292
				5177 323631.061300757
				5178 323641.857091221
				5179 323652.652881686
				5180 323663.44867215
				5181 323674.244462615
				5182 323685.04025308
				5183 323695.836043544
				5184 323706.631834009
				5185 323717.427624473
				5186 323728.223414938
				5187 323739.019205403
				5188 323749.814995867
				5189 323760.610786332
				5190 323771.406576796
				5191 323782.202367261
				5192 323792.998157725
				5193 323803.79394819
				5194 323814.589738655
				5195 323825.385529119
				5196 323836.181319584
				5197 323846.977110048
				5198 323857.772900513
				5199 323868.568690978
				5200 323879.364481442
				5201 323890.160271907
				5202 323900.956062371
				5203 323911.751852836
				5204 323922.5476433
				5205 323933.343433765
				5206 323944.13922423
				5207 323954.935014694
				5208 323965.730805159
				5209 323976.526595623
				5210 323987.322386088
				5211 323998.118176552
				5212 324008.913967017
				5213 324019.709757482
				5214 324030.505547946
				5215 324041.301338411
				5216 324052.097128875
				5217 324062.89291934
				5218 324073.688709805
				5219 324084.484500269
				5220 324095.280290734
				5221 324106.076081198
				5222 324116.871871663
				5223 324127.667662127
				5224 324138.463452592
				5225 324149.259243057
				5226 324160.055033521
				5227 324170.850823986
				5228 324181.64661445
				5229 324192.442404915
				5230 324203.238195379
				5231 324214.033985844
				5232 324224.829776309
				5233 324235.625566773
				5234 324246.421357238
				5235 324257.217147702
				5236 324268.012938167
				5237 324278.808728632
				5238 324289.604519096
				5239 324300.400309561
				5240 324311.196100025
				5241 324321.99189049
				5242 324332.787680954
				5243 324343.583471419
				5244 324354.379261884
				5245 324365.175052348
				5246 324375.970842813
				5247 324386.766633277
				5248 324397.562423742
				5249 324408.358214207
				5250 324419.154004671
				5251 324429.949795136
				5252 324440.7455856
				5253 324451.541376065
				5254 324462.337166529
				5255 324473.132956994
				5256 324483.928747459
				5257 324494.724537923
				5258 324505.520328388
				5259 324516.316118852
				5260 324527.111909317
				5261 324537.907699781
				5262 324548.703490246
				5263 324559.499280711
				5264 324570.295071175
				5265 324581.09086164
				5266 324591.886652104
				5267 324602.682442569
				5268 324613.478233034
				5269 324624.274023498
				5270 324635.069813963
				5271 324645.865604427
				5272 324656.661394892
				5273 324667.457185356
				5274 324678.252975821
				5275 324689.048766286
				5276 324699.84455675
				5277 324710.640347215
				5278 324721.436137679
				5279 324732.231928144
				5280 324743.027718609
				5281 324753.823509073
				5282 324764.619299538
				5283 324775.415090002
				5284 324786.210880467
				5285 324797.006670931
				5286 324807.802461396
				5287 324818.598251861
				5288 324829.394042325
				5289 324840.18983279
				5290 324850.985623254
				5291 324861.781413719
				5292 324872.577204183
				5293 324883.372994648
				5294 324894.168785113
				5295 324904.964575577
				5296 324915.760366042
				5297 324926.556156506
				5298 324937.351946971
				5299 324948.147737436
				5300 324958.9435279
				5301 324969.739318365
				5302 324980.535108829
				5303 324991.330899294
				5304 325002.126689758
				5305 325012.922480223
				5306 325023.718270688
				5307 325034.514061152
				5308 325045.309851617
				5309 325056.105642081
				5310 325066.901432546
				5311 325077.69722301
				5312 325088.493013475
				5313 325099.28880394
				5314 325110.084594404
				5315 325120.880384869
				5316 325131.676175333
				5317 325142.471965798
				5318 325153.267756263
				5319 325164.063546727
				5320 325174.859337192
				5321 325185.655127656
				5322 325196.450918121
				5323 325207.246708585
				5324 325218.04249905
				5325 325228.838289515
				5326 325239.634079979
				5327 325250.429870444
				5328 325261.225660908
				5329 325272.021451373
				5330 325282.817241838
				5331 325293.613032302
				5332 325304.408822767
				5333 325315.204613231
				5334 325326.000403696
				5335 325336.79619416
				5336 325347.591984625
				5337 325358.38777509
				5338 325369.183565554
				5339 325379.979356019
				5340 325390.775146483
				5341 325401.570936948
				5342 325412.366727412
				5343 325423.162517877
				5344 325433.958308342
				5345 325444.754098806
				5346 325455.549889271
				5347 325466.345679735
				5348 325477.1414702
				5349 325487.937260664
				5350 325498.733051129
				5351 325509.528841594
				5352 325520.324632058
				5353 325531.120422523
				5354 325541.916212987
				5355 325552.712003452
				5356 325563.507793917
				5357 325574.303584381
				5358 325585.099374846
				5359 325595.89516531
				5360 325606.690955775
				5361 325617.486746239
				5362 325628.282536704
				5363 325639.078327169
				5364 325649.874117633
				5365 325660.669908098
				5366 325671.465698562
				5367 325682.261489027
				5368 325693.057279492
				5369 325703.853069956
				5370 325714.648860421
				5371 325725.444650885
				5372 325736.24044135
				5373 325747.036231814
				5374 325757.832022279
				5375 325768.627812744
				5376 325779.423603208
				5377 325790.219393673
				5378 325801.015184137
				5379 325811.810974602
				5380 325822.606765066
				5381 325833.402555531
				5382 325844.198345996
				5383 325854.99413646
				5384 325865.789926925
				5385 325876.585717389
				5386 325887.381507854
				5387 325898.177298319
				5388 325908.973088783
				5389 325919.768879248
				5390 325930.564669712
				5391 325941.360460177
				5392 325952.156250641
				5393 325962.952041106
				5394 325973.747831571
				5395 325984.543622035
				5396 325995.3394125
				5397 326006.135202964
				5398 326016.930993429
				5399 326027.726783893
				5400 326038.522574358
				5401 326049.318364823
				5402 326060.114155287
				5403 326070.909945752
				5404 326081.705736216
				5405 326092.501526681
				5406 326103.297317146
				5407 326114.09310761
				5408 326124.888898075
				5409 326135.684688539
				5410 326146.480479004
				5411 326157.276269468
				5412 326168.072059933
				5413 326178.867850398
				5414 326189.663640862
				5415 326200.459431327
				5416 326211.255221791
				5417 326222.051012256
				5418 326232.846802721
				5419 326243.642593185
				5420 326254.43838365
				5421 326265.234174114
				5422 326276.029964579
				5423 326286.825755043
				5424 326297.621545508
				5425 326308.417335973
				5426 326319.213126437
				5427 326330.008916902
				5428 326340.804707366
				5429 326351.600497831
				5430 326362.396288295
				5431 326373.19207876
				5432 326383.987869225
				5433 326394.783659689
				5434 326405.579450154
				5435 326416.375240618
				5436 326427.171031083
				5437 326437.966821548
				5438 326448.762612012
				5439 326459.558402477
				5440 326470.354192941
				5441 326481.149983406
				5442 326491.94577387
				5443 326502.741564335
				5444 326513.5373548
				5445 326524.333145264
				5446 326535.128935729
				5447 326545.924726193
				5448 326556.720516658
				5449 326567.516307122
				5450 326578.312097587
				5451 326589.107888052
				5452 326599.903678516
				5453 326610.699468981
				5454 326621.495259445
				5455 326632.29104991
				5456 326643.086840375
				5457 326653.882630839
				5458 326664.678421304
				5459 326675.474211768
				5460 326686.270002233
				5461 326697.065792697
				5462 326707.861583162
				5463 326718.657373627
				5464 326729.453164091
				5465 326740.248954556
				5466 326751.04474502
				5467 326761.840535485
				5468 326772.63632595
				5469 326783.432116414
				5470 326794.227906879
				5471 326805.023697343
				5472 326815.819487808
				5473 326826.615278272
				5474 326837.411068737
				5475 326848.206859202
				5476 326859.002649666
				5477 326869.798440131
				5478 326880.594230595
				5479 326891.39002106
				5480 326902.185811524
				5481 326912.981601989
				5482 326923.777392454
				5483 326934.573182918
				5484 326945.368973383
				5485 326956.164763847
				5486 326966.960554312
				5487 326977.756344776
				5488 326988.552135241
				5489 326999.347925706
				5490 327010.14371617
				5491 327020.939506635
				5492 327031.735297099
				5493 327042.531087564
				5494 327053.326878029
				5495 327064.122668493
				5496 327074.918458958
				5497 327085.714249422
				5498 327096.510039887
				5499 327107.305830351
				5500 327118.101620816
				5501 327128.897411281
				5502 327139.693201745
				5503 327150.48899221
				5504 327161.284782674
				5505 327172.080573139
				5506 327182.876363604
				5507 327193.672154068
				5508 327204.467944533
				5509 327215.263734997
				5510 327226.059525462
				5511 327236.855315926
				5512 327247.651106391
				5513 327258.446896856
				5514 327269.24268732
				5515 327280.038477785
				5516 327290.834268249
				5517 327301.630058714
				5518 327312.425849178
				5519 327323.221639643
				5520 327334.017430108
				5521 327344.813220572
				5522 327355.609011037
				5523 327366.404801501
				5524 327377.200591966
				5525 327387.996382431
				5526 327398.792172895
				5527 327409.58796336
				5528 327420.383753824
				5529 327431.179544289
				5530 327441.975334753
				5531 327452.771125218
				5532 327463.566915683
				5533 327474.362706147
				5534 327485.158496612
				5535 327495.954287076
				5536 327506.750077541
				5537 327517.545868005
				5538 327528.34165847
				5539 327539.137448935
				5540 327549.933239399
				5541 327560.729029864
				5542 327571.524820328
				5543 327582.320610793
				5544 327593.116401258
				5545 327603.912191722
				5546 327614.707982187
				5547 327625.503772651
				5548 327636.299563116
				5549 327647.09535358
				5550 327657.891144045
				5551 327668.68693451
				5552 327679.482724974
				5553 327690.278515439
				5554 327701.074305903
				5555 327711.870096368
				5556 327722.665886833
				5557 327733.461677297
				5558 327744.257467762
				5559 327755.053258226
				5560 327765.849048691
				5561 327776.644839155
				5562 327787.44062962
				5563 327798.236420085
				5564 327809.032210549
				5565 327819.828001014
				5566 327830.623791478
				5567 327841.419581943
				5568 327852.215372407
				5569 327863.011162872
				5570 327873.806953337
				5571 327884.602743801
				5572 327895.398534266
				5573 327906.19432473
				5574 327916.990115195
				5575 327927.78590566
				5576 327938.581696124
				5577 327949.377486589
				5578 327960.173277053
				5579 327970.969067518
				5580 327981.764857982
				5581 327992.560648447
				5582 328003.356438912
				5583 328014.152229376
				5584 328024.948019841
				5585 328035.743810305
				5586 328046.53960077
				5587 328057.335391234
				5588 328068.131181699
				5589 328078.926972164
				5590 328089.722762628
				5591 328100.518553093
				5592 328111.314343557
				5593 328122.110134022
				5594 328132.905924487
				5595 328143.701714951
				5596 328154.497505416
				5597 328165.29329588
				5598 328176.089086345
				5599 328186.884876809
				5600 328197.680667274
				5601 328208.476457739
				5602 328219.272248203
				5603 328230.068038668
				5604 328240.863829132
				5605 328251.659619597
				5606 328262.455410062
				5607 328273.251200526
				5608 328284.046990991
				5609 328294.842781455
				5610 328305.63857192
				5611 328316.434362384
				5612 328327.230152849
				5613 328338.025943314
				5614 328348.821733778
				5615 328359.617524243
				5616 328370.413314707
				5617 328381.209105172
				5618 328392.004895636
				5619 328402.800686101
				5620 328413.596476566
				5621 328424.39226703
				5622 328435.188057495
				5623 328445.983847959
				5624 328456.779638424
				5625 328467.575428889
				5626 328478.371219353
				5627 328489.167009818
				5628 328499.962800282
				5629 328510.758590747
				5630 328521.554381211
				5631 328532.350171676
				5632 328543.145962141
				5633 328553.941752605
				5634 328564.73754307
				5635 328575.533333534
				5636 328586.329123999
				5637 328597.124914463
				5638 328607.920704928
				5639 328618.716495393
				5640 328629.512285857
				5641 328640.308076322
				5642 328651.103866786
				5643 328661.899657251
				5644 328672.695447716
				5645 328683.49123818
				5646 328694.287028645
				5647 328705.082819109
				5648 328715.878609574
				5649 328726.674400038
				5650 328737.470190503
				5651 328748.265980968
				5652 328759.061771432
				5653 328769.857561897
				5654 328780.653352361
				5655 328791.449142826
				5656 328802.244933291
				5657 328813.040723755
				5658 328823.83651422
				5659 328834.632304684
				5660 328845.428095149
				5661 328856.223885613
				5662 328867.019676078
				5663 328877.815466543
				5664 328888.611257007
				5665 328899.407047472
				5666 328910.202837936
				5667 328920.998628401
				5668 328931.794418865
				5669 328942.59020933
				5670 328953.385999795
				5671 328964.181790259
				5672 328974.977580724
				5673 328985.773371188
				5674 328996.569161653
				5675 329007.364952117
				5676 329018.160742582
				5677 329028.956533047
				5678 329039.752323511
				5679 329050.548113976
				5680 329061.34390444
				5681 329072.139694905
				5682 329082.93548537
				5683 329093.731275834
				5684 329104.527066299
				5685 329115.322856763
				5686 329126.118647228
				5687 329136.914437692
				5688 329147.710228157
				5689 329158.506018622
				5690 329169.301809086
				5691 329180.097599551
				5692 329190.893390015
				5693 329201.68918048
				5694 329212.484970945
				5695 329223.280761409
				5696 329234.076551874
				5697 329244.872342338
				5698 329255.668132803
				5699 329266.463923267
				5700 329277.259713732
				5701 329288.055504197
				5702 329298.851294661
				5703 329309.647085126
				5704 329320.44287559
				5705 329331.238666055
				5706 329342.034456519
				5707 329352.830246984
				5708 329363.626037449
				5709 329374.421827913
				5710 329385.217618378
				5711 329396.013408842
				5712 329406.809199307
				5713 329417.604989772
				5714 329428.400780236
				5715 329439.196570701
				5716 329449.992361165
				5717 329460.78815163
				5718 329471.583942094
				5719 329482.379732559
				5720 329493.175523024
				5721 329503.971313488
				5722 329514.767103953
				5723 329525.562894417
				5724 329536.358684882
				5725 329547.154475346
				5726 329557.950265811
				5727 329568.746056276
				5728 329579.54184674
				5729 329590.337637205
				5730 329601.133427669
				5731 329611.929218134
				5732 329622.725008599
				5733 329633.520799063
				5734 329644.316589528
				5735 329655.112379992
				5736 329665.908170457
				5737 329676.703960921
				5738 329687.499751386
				5739 329698.295541851
				5740 329709.091332315
				5741 329719.88712278
				5742 329730.682913244
				5743 329741.478703709
				5744 329752.274494174
				5745 329763.070284638
				5746 329773.866075103
				5747 329784.661865567
				5748 329795.457656032
				5749 329806.253446496
				5750 329817.049236961
				5751 329827.845027426
				5752 329838.64081789
				5753 329849.436608355
				5754 329860.232398819
				5755 329871.028189284
				5756 329881.823979748
				5757 329892.619770213
				5758 329903.415560678
				5759 329914.211351142
				5760 329925.007141607
				5761 329935.802932071
				5762 329946.598722536
				5763 329957.394513001
				5764 329968.190303465
				5765 329978.98609393
				5766 329989.781884394
				5767 330000.577674859
				5768 330011.373465323
				5769 330022.169255788
				5770 330032.965046253
				5771 330043.760836717
				5772 330054.556627182
				5773 330065.352417646
				5774 330076.148208111
				5775 330086.943998576
				5776 330097.73978904
				5777 330108.535579505
				5778 330119.331369969
				5779 330130.127160434
				5780 330140.922950898
				5781 330151.718741363
				5782 330162.514531828
				5783 330173.310322292
				5784 330184.106112757
				5785 330194.901903221
				5786 330205.697693686
				5787 330216.49348415
				5788 330227.289274615
				5789 330238.08506508
				5790 330248.880855544
				5791 330259.676646009
				5792 330270.472436473
				5793 330281.268226938
				5794 330292.064017403
				5795 330302.859807867
				5796 330313.655598332
				5797 330324.451388796
				5798 330335.247179261
				5799 330346.042969725
				5800 330356.83876019
				5801 330367.634550655
				5802 330378.430341119
				5803 330389.226131584
				5804 330400.021922048
				5805 330410.817712513
				5806 330421.613502977
				5807 330432.409293442
				5808 330443.205083907
				5809 330454.000874371
				5810 330464.796664836
				5811 330475.5924553
				5812 330486.388245765
				5813 330497.18403623
				5814 330507.979826694
				5815 330518.775617159
				5816 330529.571407623
				5817 330540.367198088
				5818 330551.162988552
				5819 330561.958779017
				5820 330572.754569482
				5821 330583.550359946
				5822 330594.346150411
				5823 330605.141940875
				5824 330615.93773134
				5825 330626.733521805
				5826 330637.529312269
				5827 330648.325102734
				5828 330659.120893198
				5829 330669.916683663
				5830 330680.712474127
				5831 330691.508264592
				5832 330702.304055057
				5833 330713.099845521
				5834 330723.895635986
				5835 330734.69142645
				5836 330745.487216915
				5837 330756.283007379
				5838 330767.078797844
				5839 330777.874588309
				5840 330788.670378773
				5841 330799.466169238
				5842 330810.261959702
				5843 330821.057750167
				5844 330831.853540631
				5845 330842.649331096
				5846 330853.445121561
				5847 330864.240912025
				5848 330875.03670249
				5849 330885.832492954
				5850 330896.628283419
				5851 330907.424073884
				5852 330918.219864348
				5853 330929.015654813
				5854 330939.811445277
				5855 330950.607235742
				5856 330961.403026206
				5857 330972.198816671
				5858 330982.994607136
				5859 330993.7903976
				5860 331004.586188065
				5861 331015.381978529
				5862 331026.177768994
				5863 331036.973559459
				5864 331047.769349923
				5865 331058.565140388
				5866 331069.360930852
				5867 331080.156721317
				5868 331090.952511781
				5869 331101.748302246
				5870 331112.544092711
				5871 331123.339883175
				5872 331134.13567364
				5873 331144.931464104
				5874 331155.727254569
				5875 331166.523045033
				5876 331177.318835498
				5877 331188.114625963
				5878 331198.910416427
				5879 331209.706206892
				5880 331220.501997356
				5881 331231.297787821
				5882 331242.093578286
				5883 331252.88936875
				5884 331263.685159215
				5885 331274.480949679
				5886 331285.276740144
				5887 331296.072530608
				5888 331306.868321073
				5889 331317.664111538
				5890 331328.459902002
				5891 331339.255692467
				5892 331350.051482931
				5893 331360.847273396
				5894 331371.64306386
				5895 331382.438854325
				5896 331393.23464479
				5897 331404.030435254
				5898 331414.826225719
				5899 331425.622016183
				5900 331436.417806648
				5901 331447.213597113
				5902 331458.009387577
				5903 331468.805178042
				5904 331479.600968506
				5905 331490.396758971
				5906 331501.192549435
				5907 331511.9883399
				5908 331522.784130365
				5909 331533.579920829
				5910 331544.375711294
				5911 331555.171501758
				5912 331565.967292223
				5913 331576.763082688
				5914 331587.558873152
				5915 331598.354663617
				5916 331609.150454081
				5917 331619.946244546
				5918 331630.74203501
				5919 331641.537825475
				5920 331652.33361594
				5921 331663.129406404
				5922 331673.925196869
				5923 331684.720987333
				5924 331695.516777798
				5925 331706.312568262
				5926 331717.108358727
				5927 331727.904149192
				5928 331738.699939656
				5929 331749.495730121
				5930 331760.291520585
				5931 331771.08731105
				5932 331781.883101515
				5933 331792.678891979
				5934 331803.474682444
				5935 331814.270472908
				5936 331825.066263373
				5937 331835.862053837
				5938 331846.657844302
				5939 331857.453634767
				5940 331868.249425231
				5941 331879.045215696
				5942 331889.84100616
				5943 331900.636796625
				5944 331911.432587089
				5945 331922.228377554
				5946 331933.024168019
				5947 331943.819958483
				5948 331954.615748948
				5949 331965.411539412
				5950 331976.207329877
				5951 331987.003120342
				5952 331997.798910806
				5953 332008.594701271
				5954 332019.390491735
				5955 332030.1862822
				5956 332040.982072664
				5957 332051.777863129
				5958 332062.573653594
				5959 332073.369444058
				5960 332084.165234523
				5961 332094.961024987
				5962 332105.756815452
				5963 332116.552605917
				5964 332127.348396381
				5965 332138.144186846
				5966 332148.93997731
				5967 332159.735767775
				5968 332170.531558239
				5969 332181.327348704
				5970 332192.123139169
				5971 332202.918929633
				5972 332213.714720098
				5973 332224.510510562
				5974 332235.306301027
				5975 332246.102091491
				5976 332256.897881956
				5977 332267.693672421
				5978 332278.489462885
				5979 332289.28525335
				5980 332300.081043814
				5981 332310.876834279
				5982 332321.672624744
				5983 332332.468415208
				5984 332343.264205673
				5985 332354.059996137
				5986 332364.855786602
				5987 332375.651577066
				5988 332386.447367531
				5989 332397.243157996
				5990 332408.03894846
				5991 332418.834738925
				5992 332429.630529389
				5993 332440.426319854
				5994 332451.222110318
				5995 332462.017900783
				5996 332472.813691248
				5997 332483.609481712
				5998 332494.405272177
				5999 332505.201062641
				6000 332515.996853106
				6001 332526.792643571
				6002 332537.588434035
				6003 332548.3842245
				6004 332559.180014964
				6005 332569.975805429
				6006 332580.771595893
				6007 332591.567386358
				6008 332602.363176823
				6009 332613.158967287
				6010 332623.954757752
				6011 332634.750548216
				6012 332645.546338681
				6013 332656.342129146
				6014 332667.13791961
				6015 332677.933710075
				6016 332688.729500539
				6017 332699.525291004
				6018 332710.321081468
				6019 332721.116871933
				6020 332731.912662398
				6021 332742.708452862
				6022 332753.504243327
				6023 332764.300033791
				6024 332775.095824256
				6025 332785.89161472
				6026 332796.687405185
				6027 332807.48319565
				6028 332818.278986114
				6029 332829.074776579
				6030 332839.870567043
				6031 332850.666357508
				6032 332861.462147972
				6033 332872.257938437
				6034 332883.053728902
				6035 332893.849519366
				6036 332904.645309831
				6037 332915.441100295
				6038 332926.23689076
				6039 332937.032681225
				6040 332947.828471689
				6041 332958.624262154
				6042 332969.420052618
				6043 332980.215843083
				6044 332991.011633547
				6045 333001.807424012
				6046 333012.603214477
				6047 333023.399004941
				6048 333034.194795406
				6049 333044.99058587
				6050 333055.786376335
				6051 333066.5821668
				6052 333077.377957264
				6053 333088.173747729
				6054 333098.969538193
				6055 333109.765328658
				6056 333120.561119122
				6057 333131.356909587
				6058 333142.152700052
				6059 333152.948490516
				6060 333163.744280981
				6061 333174.540071445
				6062 333185.33586191
				6063 333196.131652374
				6064 333206.927442839
				6065 333217.723233304
				6066 333228.519023768
				6067 333239.314814233
				6068 333250.110604697
				6069 333260.906395162
				6070 333271.702185627
				6071 333282.497976091
				6072 333293.293766556
				6073 333304.08955702
				6074 333314.885347485
				6075 333325.681137949
				6076 333336.476928414
				6077 333347.272718879
				6078 333358.068509343
				6079 333368.864299808
				6080 333379.660090272
				6081 333390.455880737
				6082 333401.251671201
				6083 333412.047461666
				6084 333422.843252131
				6085 333433.639042595
				6086 333444.43483306
				6087 333455.230623524
				6088 333466.026413989
				6089 333476.822204454
				6090 333487.617994918
				6091 333498.413785383
				6092 333509.209575847
				6093 333520.005366312
				6094 333530.801156776
				6095 333541.596947241
				6096 333552.392737706
				6097 333563.18852817
				6098 333573.984318635
				6099 333584.780109099
				6100 333595.575899564
				6101 333606.371690029
				6102 333617.167480493
				6103 333627.963270958
				6104 333638.759061422
				6105 333649.554851887
				6106 333660.350642351
				6107 333671.146432816
				6108 333681.942223281
				6109 333692.738013745
				6110 333703.53380421
				6111 333714.329594674
				6112 333725.125385139
				6113 333735.921175603
				6114 333746.716966068
				6115 333757.512756533
				6116 333768.308546997
				6117 333779.104337462
				6118 333789.900127926
				6119 333800.695918391
				6120 333811.491708855
				6121 333822.28749932
				6122 333833.083289785
				6123 333843.879080249
				6124 333854.674870714
				6125 333865.470661178
				6126 333876.266451643
				6127 333887.062242108
				6128 333897.858032572
				6129 333908.653823037
				6130 333919.449613501
				6131 333930.245403966
				6132 333941.04119443
				6133 333951.836984895
				6134 333962.63277536
				6135 333973.428565824
				6136 333984.224356289
				6137 333995.020146753
				6138 334005.815937218
				6139 334016.611727683
				6140 334027.407518147
				6141 334038.203308612
				6142 334048.999099076
				6143 334059.794889541
				6144 334070.590680005
				6145 334081.38647047
				6146 334092.182260935
				6147 334102.978051399
				6148 334113.773841864
				6149 334124.569632328
				6150 334135.365422793
				6151 334146.161213257
				6152 334156.957003722
				6153 334167.752794187
				6154 334178.548584651
				6155 334189.344375116
				6156 334200.14016558
				6157 334210.935956045
				6158 334221.73174651
				6159 334232.527536974
				6160 334243.323327439
				6161 334254.119117903
				6162 334264.914908368
				6163 334275.710698832
				6164 334286.506489297
				6165 334297.302279762
				6166 334308.098070226
				6167 334318.893860691
				6168 334329.689651155
				6169 334340.48544162
				6170 334351.281232084
				6171 334362.077022549
				6172 334372.872813014
				6173 334383.668603478
				6174 334394.464393943
				6175 334405.260184407
				6176 334416.055974872
				6177 334426.851765337
				6178 334437.647555801
				6179 334448.443346266
				6180 334459.23913673
				6181 334470.034927195
				6182 334480.830717659
				6183 334491.626508124
				6184 334502.422298589
				6185 334513.218089053
				6186 334524.013879518
				6187 334534.809669982
				6188 334545.605460447
				6189 334556.401250912
				6190 334567.197041376
				6191 334577.992831841
				6192 334588.788622305
				6193 334599.58441277
				6194 334610.380203234
				6195 334621.175993699
				6196 334631.971784164
				6197 334642.767574628
				6198 334653.563365093
				6199 334664.359155557
				6200 334675.154946022
				6201 334685.950736486
				6202 334696.746526951
				6203 334707.542317416
				6204 334718.33810788
				6205 334729.133898345
				6206 334739.929688809
				6207 334750.725479274
				6208 334761.521269739
				6209 334772.317060203
				6210 334783.112850668
				6211 334793.908641132
				6212 334804.704431597
				6213 334815.500222061
				6214 334826.296012526
				6215 334837.091802991
				6216 334847.887593455
				6217 334858.68338392
				6218 334869.479174384
				6219 334880.274964849
				6220 334891.070755313
				6221 334901.866545778
				6222 334912.662336243
				6223 334923.458126707
				6224 334934.253917172
				6225 334945.049707636
				6226 334955.845498101
				6227 334966.641288566
				6228 334977.43707903
				6229 334988.232869495
				6230 334999.028659959
				6231 335009.824450424
				6232 335020.620240888
				6233 335031.416031353
				6234 335042.211821818
				6235 335053.007612282
				6236 335063.803402747
				6237 335074.599193211
				6238 335085.394983676
				6239 335096.190774141
				6240 335106.986564605
				6241 335117.78235507
				6242 335128.578145534
				6243 335139.373935999
				6244 335150.169726463
				6245 335160.965516928
				6246 335171.761307393
				6247 335182.557097857
				6248 335193.352888322
				6249 335204.148678786
				6250 335214.944469251
				6251 335225.740259715
				6252 335236.53605018
				6253 335247.331840645
				6254 335258.127631109
				6255 335268.923421574
				6256 335279.719212038
				6257 335290.515002503
				6258 335301.310792968
				6259 335312.106583432
				6260 335322.902373897
				6261 335333.698164361
				6262 335344.493954826
				6263 335355.28974529
				6264 335366.085535755
				6265 335376.88132622
				6266 335387.677116684
				6267 335398.472907149
				6268 335409.268697613
				6269 335420.064488078
				6270 335430.860278543
				6271 335441.656069007
				6272 335452.451859472
				6273 335463.247649936
				6274 335474.043440401
				6275 335484.839230865
				6276 335495.63502133
				6277 335506.430811795
				6278 335517.226602259
				6279 335528.022392724
				6280 335538.818183188
				6281 335549.613973653
				6282 335560.409764117
				6283 335571.205554582
				6284 335582.001345047
				6285 335592.797135511
				6286 335603.592925976
				6287 335614.38871644
				6288 335625.184506905
				6289 335635.98029737
				6290 335646.776087834
				6291 335657.571878299
				6292 335668.367668763
				6293 335679.163459228
				6294 335689.959249692
				6295 335700.755040157
				6296 335711.550830622
				6297 335722.346621086
				6298 335733.142411551
				6299 335743.938202015
				6300 335754.73399248
				6301 335765.529782944
				6302 335776.325573409
				6303 335787.121363874
				6304 335797.917154338
				6305 335808.712944803
				6306 335819.508735267
				6307 335830.304525732
				6308 335841.100316197
				6309 335851.896106661
				6310 335862.691897126
				6311 335873.48768759
				6312 335884.283478055
				6313 335895.079268519
				6314 335905.875058984
				6315 335916.670849449
				6316 335927.466639913
				6317 335938.262430378
				6318 335949.058220842
				6319 335959.854011307
				6320 335970.649801772
				6321 335981.445592236
				6322 335992.241382701
				6323 336003.037173165
				6324 336013.83296363
				6325 336024.628754094
				6326 336035.424544559
				6327 336046.220335024
				6328 336057.016125488
				6329 336067.811915953
				6330 336078.607706417
				6331 336089.403496882
				6332 336100.199287346
				6333 336110.995077811
				6334 336121.790868276
				6335 336132.58665874
				6336 336143.382449205
				6337 336154.178239669
				6338 336164.974030134
				6339 336175.769820599
				6340 336186.565611063
				6341 336197.361401528
				6342 336208.157191992
				6343 336218.952982457
				6344 336229.748772921
				6345 336240.544563386
				6346 336251.340353851
				6347 336262.136144315
				6348 336272.93193478
				6349 336283.727725244
				6350 336294.523515709
				6351 336305.319306173
				6352 336316.115096638
				6353 336326.910887103
				6354 336337.706677567
				6355 336348.502468032
				6356 336359.298258496
				6357 336370.094048961
				6358 336380.889839426
				6359 336391.68562989
				6360 336402.481420355
				6361 336413.277210819
				6362 336424.073001284
				6363 336434.868791748
				6364 336445.664582213
				6365 336456.460372678
				6366 336467.256163142
				6367 336478.051953607
				6368 336488.847744071
				6369 336499.643534536
				6370 336510.439325001
				6371 336521.235115465
				6372 336532.03090593
				6373 336542.826696394
				6374 336553.622486859
				6375 336564.418277323
				6376 336575.214067788
				6377 336586.009858253
				6378 336596.805648717
				6379 336607.601439182
				6380 336618.397229646
				6381 336629.193020111
				6382 336639.988810575
				6383 336650.78460104
				6384 336661.580391505
				6385 336672.376181969
				6386 336683.171972434
				6387 336693.967762898
				6388 336704.763553363
				6389 336715.559343827
				6390 336726.355134292
				6391 336737.150924757
				6392 336747.946715221
				6393 336758.742505686
				6394 336769.53829615
				6395 336780.334086615
				6396 336791.12987708
				6397 336801.925667544
				6398 336812.721458009
				6399 336823.517248473
				6400 336834.313038938
				6401 336845.108829402
				6402 336855.904619867
				6403 336866.700410332
				6404 336877.496200796
				6405 336888.291991261
				6406 336899.087781725
				6407 336909.88357219
				6408 336920.679362655
				6409 336931.475153119
				6410 336942.270943584
				6411 336953.066734048
				6412 336963.862524513
				6413 336974.658314977
				6414 336985.454105442
				6415 336996.249895907
				6416 337007.045686371
				6417 337017.841476836
				6418 337028.6372673
				6419 337039.433057765
				6420 337050.228848229
				6421 337061.024638694
				6422 337071.820429159
				6423 337082.616219623
				6424 337093.412010088
				6425 337104.207800552
				6426 337115.003591017
				6427 337125.799381482
				6428 337136.595171946
				6429 337147.390962411
				6430 337158.186752875
				6431 337168.98254334
				6432 337179.778333804
				6433 337190.574124269
				6434 337201.369914734
				6435 337212.165705198
				6436 337222.961495663
				6437 337233.757286127
				6438 337244.553076592
				6439 337255.348867056
				6440 337266.144657521
				6441 337276.940447986
				6442 337287.73623845
				6443 337298.532028915
				6444 337309.327819379
				6445 337320.123609844
				6446 337330.919400309
				6447 337341.715190773
				6448 337352.510981238
				6449 337363.306771702
				6450 337374.102562167
				6451 337384.898352631
				6452 337395.694143096
				6453 337406.489933561
				6454 337417.285724025
				6455 337428.08151449
				6456 337438.877304954
				6457 337449.673095419
				6458 337460.468885884
				6459 337471.264676348
				6460 337482.060466813
				6461 337492.856257277
				6462 337503.652047742
				6463 337514.447838206
				6464 337525.243628671
				6465 337536.039419136
				6466 337546.8352096
				6467 337557.631000065
				6468 337568.426790529
				6469 337579.222580994
				6470 337590.018371458
				6471 337600.814161923
				6472 337611.609952388
				6473 337622.405742852
				6474 337633.201533317
				6475 337643.997323781
				6476 337654.793114246
				6477 337665.58890471
				6478 337676.384695175
				6479 337687.18048564
				6480 337697.976276104
				6481 337708.772066569
				6482 337719.567857033
				6483 337730.363647498
				6484 337741.159437963
				6485 337751.955228427
				6486 337762.751018892
				6487 337773.546809356
				6488 337784.342599821
				6489 337795.138390285
				6490 337805.93418075
				6491 337816.729971215
				6492 337827.525761679
				6493 337838.321552144
				6494 337849.117342608
				6495 337859.913133073
				6496 337870.708923538
				6497 337881.504714002
				6498 337892.300504467
				6499 337903.096294931
				6500 337913.892085396
				6501 337924.68787586
				6502 337935.483666325
				6503 337946.27945679
				6504 337957.075247254
				6505 337967.871037719
				6506 337978.666828183
				6507 337989.462618648
				6508 338000.258409112
				6509 338011.054199577
				6510 338021.849990042
				6511 338032.645780506
				6512 338043.441570971
				6513 338054.237361435
				6514 338065.0331519
				6515 338075.828942365
				6516 338086.624732829
				6517 338097.420523294
				6518 338108.216313758
				6519 338119.012104223
				6520 338129.807894687
				6521 338140.603685152
				6522 338151.399475617
				6523 338162.195266081
				6524 338172.991056546
				6525 338183.78684701
				6526 338194.582637475
				6527 338205.378427939
				6528 338216.174218404
				6529 338226.970008869
				6530 338237.765799333
				6531 338248.561589798
				6532 338259.357380262
				6533 338270.153170727
				6534 338280.948961192
				6535 338291.744751656
				6536 338302.540542121
				6537 338313.336332585
				6538 338324.13212305
				6539 338334.927913514
				6540 338345.723703979
				6541 338356.519494444
				6542 338367.315284908
				6543 338378.111075373
				6544 338388.906865837
				6545 338399.702656302
				6546 338410.498446767
				6547 338421.294237231
				6548 338432.090027696
				6549 338442.88581816
				6550 338453.681608625
				6551 338464.477399089
				6552 338475.273189554
				6553 338486.068980019
				6554 338496.864770483
				6555 338507.660560948
				6556 338518.456351412
				6557 338529.252141877
				6558 338540.047932341
				6559 338550.843722806
				6560 338561.639513271
				6561 338572.435303735
				6562 338583.2310942
				6563 338594.026884664
				6564 338604.822675129
				6565 338615.618465594
				6566 338626.414256058
				6567 338637.210046523
				6568 338648.005836987
				6569 338658.801627452
				6570 338669.597417916
				6571 338680.393208381
				6572 338691.188998846
				6573 338701.98478931
				6574 338712.780579775
				6575 338723.576370239
				6576 338734.372160704
				6577 338745.167951168
				6578 338755.963741633
				6579 338766.759532098
				6580 338777.555322562
				6581 338788.351113027
				6582 338799.146903491
				6583 338809.942693956
				6584 338820.738484421
				6585 338831.534274885
				6586 338842.33006535
				6587 338853.125855814
				6588 338863.921646279
				6589 338874.717436743
				6590 338885.513227208
				6591 338896.309017673
				6592 338907.104808137
				6593 338917.900598602
				6594 338928.696389066
				6595 338939.492179531
				6596 338950.287969996
				6597 338961.08376046
				6598 338971.879550925
				6599 338982.675341389
				6600 338993.471131854
				6601 339004.266922318
				6602 339015.062712783
				6603 339025.858503248
				6604 339036.654293712
				6605 339047.450084177
				6606 339058.245874641
				6607 339069.041665106
				6608 339079.83745557
				6609 339090.633246035
				6610 339101.4290365
				6611 339112.224826964
				6612 339123.020617429
				6613 339133.816407893
				6614 339144.612198358
				6615 339155.407988823
				6616 339166.203779287
				6617 339176.999569752
				6618 339187.795360216
				6619 339198.591150681
				6620 339209.386941145
				6621 339220.18273161
				6622 339230.978522075
				6623 339241.774312539
				6624 339252.570103004
				6625 339263.365893468
				6626 339274.161683933
				6627 339284.957474397
				6628 339295.753264862
				6629 339306.549055327
				6630 339317.344845791
				6631 339328.140636256
				6632 339338.93642672
				6633 339349.732217185
				6634 339360.52800765
				6635 339371.323798114
				6636 339382.119588579
				6637 339392.915379043
				6638 339403.711169508
				6639 339414.506959972
				6640 339425.302750437
				6641 339436.098540902
				6642 339446.894331366
				6643 339457.690121831
				6644 339468.485912295
				6645 339479.28170276
				6646 339490.077493225
				6647 339500.873283689
				6648 339511.669074154
				6649 339522.464864618
				6650 339533.260655083
				6651 339544.056445547
				6652 339554.852236012
				6653 339565.648026477
				6654 339576.443816941
				6655 339587.239607406
				6656 339598.03539787
				6657 339608.831188335
				6658 339619.626978799
				6659 339630.422769264
				6660 339641.218559729
				6661 339652.014350193
				6662 339662.810140658
				6663 339673.605931122
				6664 339684.401721587
				6665 339695.197512052
				6666 339705.993302516
				6667 339716.789092981
				6668 339727.584883445
				6669 339738.38067391
				6670 339749.176464374
				6671 339759.972254839
				6672 339770.768045304
				6673 339781.563835768
				6674 339792.359626233
				6675 339803.155416697
				6676 339813.951207162
				6677 339824.746997626
				6678 339835.542788091
				6679 339846.338578556
				6680 339857.13436902
				6681 339867.930159485
				6682 339878.725949949
				6683 339889.521740414
				6684 339900.317530879
				6685 339911.113321343
				6686 339921.909111808
				6687 339932.704902272
				6688 339943.500692737
				6689 339954.296483201
				6690 339965.092273666
				6691 339975.888064131
				6692 339986.683854595
				6693 339997.47964506
				6694 340008.275435524
				6695 340019.071225989
				6696 340029.867016454
				6697 340040.662806918
				6698 340051.458597383
				6699 340062.254387847
				6700 340073.050178312
				6701 340083.845968776
				6702 340094.641759241
				6703 340105.437549706
				6704 340116.23334017
				6705 340127.029130635
				6706 340137.824921099
				6707 340148.620711564
				6708 340159.416502028
				6709 340170.212292493
				6710 340181.008082958
				6711 340191.803873422
				6712 340202.599663887
				6713 340213.395454351
				6714 340224.191244816
				6715 340234.987035281
				6716 340245.782825745
				6717 340256.57861621
				6718 340267.374406674
				6719 340278.170197139
				6720 340288.965987603
				6721 340299.761778068
				6722 340310.557568533
				6723 340321.353358997
				6724 340332.149149462
				6725 340342.944939926
				6726 340353.740730391
				6727 340364.536520856
				6728 340375.33231132
				6729 340386.128101785
				6730 340396.923892249
				6731 340407.719682714
				6732 340418.515473178
				6733 340429.311263643
				6734 340440.107054108
				6735 340450.902844572
				6736 340461.698635037
				6737 340472.494425501
				6738 340483.290215966
				6739 340494.08600643
				6740 340504.881796895
				6741 340515.67758736
				6742 340526.473377824
				6743 340537.269168289
				6744 340548.064958753
				6745 340558.860749218
				6746 340569.656539682
				6747 340580.452330147
				6748 340591.248120612
				6749 340602.043911076
				6750 340612.839701541
				6751 340623.635492005
				6752 340634.43128247
				6753 340645.227072935
				6754 340656.022863399
				6755 340666.818653864
				6756 340677.614444328
				6757 340688.410234793
				6758 340699.206025257
				6759 340710.001815722
				6760 340720.797606187
				6761 340731.593396651
				6762 340742.389187116
				6763 340753.18497758
				6764 340763.980768045
				6765 340774.77655851
				6766 340785.572348974
				6767 340796.368139439
				6768 340807.163929903
				6769 340817.959720368
				6770 340828.755510832
				6771 340839.551301297
				6772 340850.347091762
				6773 340861.142882226
				6774 340871.938672691
				6775 340882.734463155
				6776 340893.53025362
				6777 340904.326044084
				6778 340915.121834549
				6779 340925.917625014
				6780 340936.713415478
				6781 340947.509205943
				6782 340958.304996407
				6783 340969.100786872
				6784 340979.896577337
				6785 340990.692367801
				6786 341001.488158266
				6787 341012.28394873
				6788 341023.079739195
				6789 341033.875529659
				6790 341044.671320124
				6791 341055.467110589
				6792 341066.262901053
				6793 341077.058691518
				6794 341087.854481982
				6795 341098.650272447
				6796 341109.446062911
				6797 341120.241853376
				6798 341131.037643841
				6799 341141.833434305
				6800 341152.62922477
				6801 341163.425015234
				6802 341174.220805699
				6803 341185.016596164
				6804 341195.812386628
				6805 341206.608177093
				6806 341217.403967557
				6807 341228.199758022
				6808 341238.995548486
				6809 341249.791338951
				6810 341260.587129416
				6811 341271.38291988
				6812 341282.178710345
				6813 341292.974500809
				6814 341303.770291274
				6815 341314.566081739
				6816 341325.361872203
				6817 341336.157662668
				6818 341346.953453132
				6819 341357.749243597
				6820 341368.545034061
				6821 341379.340824526
				6822 341390.136614991
				6823 341400.932405455
				6824 341411.72819592
				6825 341422.523986384
				6826 341433.319776849
				6827 341444.115567313
				6828 341454.911357778
				6829 341465.707148243
				6830 341476.502938707
				6831 341487.298729172
				6832 341498.094519636
				6833 341508.890310101
				6834 341519.686100565
				6835 341530.48189103
				6836 341541.277681495
				6837 341552.073471959
				6838 341562.869262424
				6839 341573.665052888
				6840 341584.460843353
				6841 341595.256633818
				6842 341606.052424282
				6843 341616.848214747
				6844 341627.644005211
				6845 341638.439795676
				6846 341649.23558614
				6847 341660.031376605
				6848 341670.82716707
				6849 341681.622957534
				6850 341692.418747999
				6851 341703.214538463
				6852 341714.010328928
				6853 341724.806119393
				6854 341735.601909857
				6855 341746.397700322
				6856 341757.193490786
				6857 341767.989281251
				6858 341778.785071715
				6859 341789.58086218
				6860 341800.376652645
				6861 341811.172443109
				6862 341821.968233574
				6863 341832.764024038
				6864 341843.559814503
				6865 341854.355604967
				6866 341865.151395432
				6867 341875.947185897
				6868 341886.742976361
				6869 341897.538766826
				6870 341908.33455729
				6871 341919.130347755
				6872 341929.92613822
				6873 341940.721928684
				6874 341951.517719149
				6875 341962.313509613
				6876 341973.109300078
				6877 341983.905090542
				6878 341994.700881007
				6879 342005.496671472
				6880 342016.292461936
				6881 342027.088252401
				6882 342037.884042865
				6883 342048.67983333
				6884 342059.475623794
				6885 342070.271414259
				6886 342081.067204724
				6887 342091.862995188
				6888 342102.658785653
				6889 342113.454576117
				6890 342124.250366582
				6891 342135.046157047
				6892 342145.841947511
				6893 342156.637737976
				6894 342167.43352844
				6895 342178.229318905
				6896 342189.025109369
				6897 342199.820899834
				6898 342210.616690299
				6899 342221.412480763
				6900 342232.208271228
				6901 342243.004061692
				6902 342253.799852157
				6903 342264.595642622
				6904 342275.391433086
				6905 342286.187223551
				6906 342296.983014015
				6907 342307.77880448
				6908 342318.574594944
				6909 342329.370385409
				6910 342340.166175874
				6911 342350.961966338
				6912 342361.757756803
				6913 342372.553547267
				6914 342383.349337732
				6915 342394.145128196
				6916 342404.940918661
				6917 342415.736709126
				6918 342426.53249959
				6919 342437.328290055
				6920 342448.124080519
				6921 342458.919870984
				6922 342469.715661449
				6923 342480.511451913
				6924 342491.307242378
				6925 342502.103032842
				6926 342512.898823307
				6927 342523.694613771
				6928 342534.490404236
				6929 342545.286194701
				6930 342556.081985165
				6931 342566.87777563
				6932 342577.673566094
				6933 342588.469356559
				6934 342599.265147023
				6935 342610.060937488
				6936 342620.856727953
				6937 342631.652518417
				6938 342642.448308882
				6939 342653.244099346
				6940 342664.039889811
				6941 342674.835680276
				6942 342685.63147074
				6943 342696.427261205
				6944 342707.223051669
				6945 342718.018842134
				6946 342728.814632598
				6947 342739.610423063
				6948 342750.406213528
				6949 342761.202003992
				6950 342771.997794457
				6951 342782.793584921
				6952 342793.589375386
				6953 342804.385165851
				6954 342815.180956315
				6955 342825.97674678
				6956 342836.772537244
				6957 342847.568327709
				6958 342858.364118173
				6959 342869.159908638
				6960 342879.955699103
				6961 342890.751489567
				6962 342901.547280032
				6963 342912.343070496
				6964 342923.138860961
				6965 342933.934651425
				6966 342944.73044189
				6967 342955.526232355
				6968 342966.322022819
				6969 342977.117813284
				6970 342987.913603748
				6971 342998.709394213
				6972 343009.505184678
				6973 343020.300975142
				6974 343031.096765607
				6975 343041.892556071
				6976 343052.688346536
				6977 343063.484137
				6978 343074.279927465
				6979 343085.07571793
				6980 343095.871508394
				6981 343106.667298859
				6982 343117.463089323
				6983 343128.258879788
				6984 343139.054670252
				6985 343149.850460717
				6986 343160.646251182
				6987 343171.442041646
				6988 343182.237832111
				6989 343193.033622575
				6990 343203.82941304
				6991 343214.625203505
				6992 343225.420993969
				6993 343236.216784434
				6994 343247.012574898
				6995 343257.808365363
				6996 343268.604155827
				6997 343279.399946292
				6998 343290.195736757
				6999 343300.991527221
				7000 343311.787317686
				7001 343322.58310815
				7002 343333.378898615
				7003 343344.17468908
				7004 343354.970479544
				7005 343365.766270009
				7006 343376.562060473
				7007 343387.357850938
				7008 343398.153641402
				7009 343408.949431867
				7010 343419.745222332
				7011 343430.541012796
				7012 343441.336803261
				7013 343452.132593725
				7014 343462.92838419
				7015 343473.724174654
				7016 343484.519965119
				7017 343495.315755584
				7018 343506.111546048
				7019 343516.907336513
				7020 343527.703126977
				7021 343538.498917442
				7022 343549.294707907
				7023 343560.090498371
				7024 343570.886288836
				7025 343581.6820793
				7026 343592.477869765
				7027 343603.273660229
				7028 343614.069450694
				7029 343624.865241159
				7030 343635.661031623
				7031 343646.456822088
				7032 343657.252612552
				7033 343668.048403017
				7034 343678.844193481
				7035 343689.639983946
				7036 343700.435774411
				7037 343711.231564875
				7038 343722.02735534
				7039 343732.823145804
				7040 343743.618936269
				7041 343754.414726734
				7042 343765.210517198
				7043 343776.006307663
				7044 343786.802098127
				7045 343797.597888592
				7046 343808.393679056
				7047 343819.189469521
				7048 343829.985259986
				7049 343840.78105045
				7050 343851.576840915
				7051 343862.372631379
				7052 343873.168421844
				7053 343883.964212309
				7054 343894.760002773
				7055 343905.555793238
				7056 343916.351583702
				7057 343927.147374167
				7058 343937.943164631
				7059 343948.738955096
				7060 343959.534745561
				7061 343970.330536025
				7062 343981.12632649
				7063 343991.922116954
				7064 344002.717907419
				7065 344013.513697883
				7066 344024.309488348
				7067 344035.105278813
				7068 344045.901069277
				7069 344056.696859742
				7070 344067.492650206
				7071 344078.288440671
				7072 344089.084231135
				7073 344099.8800216
				7074 344110.675812065
				7075 344121.471602529
				7076 344132.267392994
				7077 344143.063183458
				7078 344153.858973923
				7079 344164.654764388
				7080 344175.450554852
				7081 344186.246345317
				7082 344197.042135781
				7083 344207.837926246
				7084 344218.63371671
				7085 344229.429507175
				7086 344240.22529764
				7087 344251.021088104
				7088 344261.816878569
				7089 344272.612669033
				7090 344283.408459498
				7091 344294.204249963
				7092 344305.000040427
				7093 344315.795830892
				7094 344326.591621356
				7095 344337.387411821
				7096 344348.183202285
				7097 344358.97899275
				7098 344369.774783215
				7099 344380.570573679
				7100 344391.366364144
				7101 344402.162154608
				7102 344412.957945073
				7103 344423.753735537
				7104 344434.549526002
				7105 344445.345316467
				7106 344456.141106931
				7107 344466.936897396
				7108 344477.73268786
				7109 344488.528478325
				7110 344499.32426879
				7111 344510.120059254
				7112 344520.915849719
				7113 344531.711640183
				7114 344542.507430648
				7115 344553.303221112
				7116 344564.099011577
				7117 344574.894802042
				7118 344585.690592506
				7119 344596.486382971
				7120 344607.282173435
				7121 344618.0779639
				7122 344628.873754364
				7123 344639.669544829
				7124 344650.465335294
				7125 344661.261125758
				7126 344672.056916223
				7127 344682.852706687
				7128 344693.648497152
				7129 344704.444287617
				7130 344715.240078081
				7131 344726.035868546
				7132 344736.83165901
				7133 344747.627449475
				7134 344758.423239939
				7135 344769.219030404
				7136 344780.014820869
				7137 344790.810611333
				7138 344801.606401798
				7139 344812.402192262
				7140 344823.197982727
				7141 344833.993773192
				7142 344844.789563656
				7143 344855.585354121
				7144 344866.381144585
				7145 344877.17693505
				7146 344887.972725514
				7147 344898.768515979
				7148 344909.564306444
				7149 344920.360096908
				7150 344931.155887373
				7151 344941.951677837
				7152 344952.747468302
				7153 344963.543258766
				7154 344974.339049231
				7155 344985.134839696
				7156 344995.93063016
				7157 345006.726420625
				7158 345017.522211089
				7159 345028.318001554
				7160 345039.113792018
				7161 345049.909582483
				7162 345060.705372948
				7163 345071.501163412
				7164 345082.296953877
				7165 345093.092744341
				7166 345103.888534806
				7167 345114.684325271
				7168 345125.480115735
				7169 345136.2759062
				7170 345147.071696664
				7171 345157.867487129
				7172 345168.663277593
				7173 345179.459068058
				7174 345190.254858523
				7175 345201.050648987
				7176 345211.846439452
				7177 345222.642229916
				7178 345233.438020381
				7179 345244.233810846
				7180 345255.02960131
				7181 345265.825391775
				7182 345276.621182239
				7183 345287.416972704
				7184 345298.212763168
				7185 345309.008553633
				7186 345319.804344098
				7187 345330.600134562
				7188 345341.395925027
				7189 345352.191715491
				7190 345362.987505956
				7191 345373.78329642
				7192 345384.579086885
				7193 345395.37487735
				7194 345406.170667814
				7195 345416.966458279
				7196 345427.762248743
				7197 345438.558039208
				7198 345449.353829673
				7199 345460.149620137
				7200 345470.945410602
				7201 345481.741201066
				7202 345492.536991531
				7203 345503.332781995
				7204 345514.12857246
				7205 345524.924362925
				7206 345535.720153389
				7207 345546.515943854
				7208 345557.311734318
				7209 345568.107524783
				7210 345578.903315248
				7211 345589.699105712
				7212 345600.494896177
				7213 345611.290686641
				7214 345622.086477106
				7215 345632.88226757
				7216 345643.678058035
				7217 345654.4738485
				7218 345665.269638964
				7219 345676.065429429
				7220 345686.861219893
				7221 345697.657010358
				7222 345708.452800822
				7223 345719.248591287
				7224 345730.044381752
				7225 345740.840172216
				7226 345751.635962681
				7227 345762.431753145
				7228 345773.22754361
				7229 345784.023334075
				7230 345794.819124539
				7231 345805.614915004
				7232 345816.410705468
				7233 345827.206495933
				7234 345838.002286397
				7235 345848.798076862
				7236 345859.593867327
				7237 345870.389657791
				7238 345881.185448256
				7239 345891.98123872
				7240 345902.777029185
				7241 345913.572819649
				7242 345924.368610114
				7243 345935.164400579
				7244 345945.960191043
				7245 345956.755981508
				7246 345967.551771972
				7247 345978.347562437
				7248 345989.143352902
				7249 345999.939143366
				7250 346010.734933831
				7251 346021.530724295
				7252 346032.32651476
				7253 346043.122305224
				7254 346053.918095689
				7255 346064.713886154
				7256 346075.509676618
				7257 346086.305467083
				7258 346097.101257547
				7259 346107.897048012
				7260 346118.692838477
				7261 346129.488628941
				7262 346140.284419406
				7263 346151.08020987
				7264 346161.876000335
				7265 346172.671790799
				7266 346183.467581264
				7267 346194.263371729
				7268 346205.059162193
				7269 346215.854952658
				7270 346226.650743122
				7271 346237.446533587
				7272 346248.242324051
				7273 346259.038114516
				7274 346269.833904981
				7275 346280.629695445
				7276 346291.42548591
				7277 346302.221276374
				7278 346313.017066839
				7279 346323.812857304
				7280 346334.608647768
				7281 346345.404438233
				7282 346356.200228697
				7283 346366.996019162
				7284 346377.791809626
				7285 346388.587600091
				7286 346399.383390556
				7287 346410.17918102
				7288 346420.974971485
				7289 346431.770761949
				7290 346442.566552414
				7291 346453.362342878
				7292 346464.158133343
				7293 346474.953923808
				7294 346485.749714272
				7295 346496.545504737
				7296 346507.341295201
				7297 346518.137085666
				7298 346528.932876131
				7299 346539.728666595
				7300 346550.52445706
				7301 346561.320247524
				7302 346572.116037989
				7303 346582.911828453
				7304 346593.707618918
				7305 346604.503409383
				7306 346615.299199847
				7307 346626.094990312
				7308 346636.890780776
				7309 346647.686571241
				7310 346658.482361706
				7311 346669.27815217
				7312 346680.073942635
				7313 346690.869733099
				7314 346701.665523564
				7315 346712.461314028
				7316 346723.257104493
				7317 346734.052894958
				7318 346744.848685422
				7319 346755.644475887
				7320 346766.440266351
				7321 346777.236056816
				7322 346788.03184728
				7323 346798.827637745
				7324 346809.62342821
				7325 346820.419218674
				7326 346831.215009139
				7327 346842.010799603
				7328 346852.806590068
				7329 346863.602380533
				7330 346874.398170997
				7331 346885.193961462
				7332 346895.989751926
				7333 346906.785542391
				7334 346917.581332855
				7335 346928.37712332
				7336 346939.172913785
				7337 346949.968704249
				7338 346960.764494714
				7339 346971.560285178
				7340 346982.356075643
				7341 346993.151866107
				7342 347003.947656572
				7343 347014.743447037
				7344 347025.539237501
				7345 347036.335027966
				7346 347047.13081843
				7347 347057.926608895
				7348 347068.72239936
				7349 347079.518189824
				7350 347090.313980289
				7351 347101.109770753
				7352 347111.905561218
				7353 347122.701351682
				7354 347133.497142147
				7355 347144.292932612
				7356 347155.088723076
				7357 347165.884513541
				7358 347176.680304005
				7359 347187.47609447
				7360 347198.271884935
				7361 347209.067675399
				7362 347219.863465864
				7363 347230.659256328
				7364 347241.455046793
				7365 347252.250837257
				7366 347263.046627722
				7367 347273.842418187
				7368 347284.638208651
				7369 347295.433999116
				7370 347306.22978958
				7371 347317.025580045
				7372 347327.821370509
				7373 347338.617160974
				7374 347349.412951439
				7375 347360.208741903
				7376 347371.004532368
				7377 347381.800322832
				7378 347392.596113297
				7379 347403.391903761
				7380 347414.187694226
				7381 347424.983484691
				7382 347435.779275155
				7383 347446.57506562
				7384 347457.370856084
				7385 347468.166646549
				7386 347478.962437014
				7387 347489.758227478
				7388 347500.554017943
				7389 347511.349808407
				7390 347522.145598872
				7391 347532.941389336
				7392 347543.737179801
				7393 347554.532970266
				7394 347565.32876073
				7395 347576.124551195
				7396 347586.920341659
				7397 347597.716132124
				7398 347608.511922589
				7399 347619.307713053
				7400 347630.103503518
				7401 347640.899293982
				7402 347651.695084447
				7403 347662.490874911
				7404 347673.286665376
				7405 347684.082455841
				7406 347694.878246305
				7407 347705.67403677
				7408 347716.469827234
				7409 347727.265617699
				7410 347738.061408164
				7411 347748.857198628
				7412 347759.652989093
				7413 347770.448779557
				7414 347781.244570022
				7415 347792.040360486
				7416 347802.836150951
				7417 347813.631941416
				7418 347824.42773188
				7419 347835.223522345
				7420 347846.019312809
				7421 347856.815103274
				7422 347867.610893738
				7423 347878.406684203
				7424 347889.202474668
				7425 347899.998265132
				7426 347910.794055597
				7427 347921.589846061
				7428 347932.385636526
				7429 347943.18142699
				7430 347953.977217455
				7431 347964.77300792
				7432 347975.568798384
				7433 347986.364588849
				7434 347997.160379313
				7435 348007.956169778
				7436 348018.751960243
				7437 348029.547750707
				7438 348040.343541172
				7439 348051.139331636
				7440 348061.935122101
				7441 348072.730912565
				7442 348083.52670303
				7443 348094.322493495
				7444 348105.118283959
				7445 348115.914074424
				7446 348126.709864888
				7447 348137.505655353
				7448 348148.301445818
				7449 348159.097236282
				7450 348169.893026747
				7451 348180.688817211
				7452 348191.484607676
				7453 348202.28039814
				7454 348213.076188605
				7455 348223.87197907
				7456 348234.667769534
				7457 348245.463559999
				7458 348256.259350463
				7459 348267.055140928
				7460 348277.850931392
				7461 348288.646721857
				7462 348299.442512322
				7463 348310.238302786
				7464 348321.034093251
				7465 348331.829883715
				7466 348342.62567418
				7467 348353.421464645
				7468 348364.217255109
				7469 348375.013045574
				7470 348385.808836038
				7471 348396.604626503
				7472 348407.400416967
				7473 348418.196207432
				7474 348428.991997897
				7475 348439.787788361
				7476 348450.583578826
				7477 348461.37936929
				7478 348472.175159755
				7479 348482.970950219
				7480 348493.766740684
				7481 348504.562531149
				7482 348515.358321613
				7483 348526.154112078
				7484 348536.949902542
				7485 348547.745693007
				7486 348558.541483472
				7487 348569.337273936
				7488 348580.133064401
				7489 348590.928854865
				7490 348601.72464533
				7491 348612.520435794
				7492 348623.316226259
				7493 348634.112016724
				7494 348644.907807188
				7495 348655.703597653
				7496 348666.499388117
				7497 348677.295178582
				7498 348688.090969047
				7499 348698.886759511
				7500 348709.682549976
				7501 348720.47834044
				7502 348731.274130905
				7503 348742.069921369
				7504 348752.865711834
				7505 348763.661502299
				7506 348774.457292763
				7507 348785.253083228
				7508 348796.048873692
				7509 348806.844664157
				7510 348817.640454621
				7511 348828.436245086
				7512 348839.232035551
				7513 348850.027826015
				7514 348860.82361648
				7515 348871.619406944
				7516 348882.415197409
				7517 348893.210987873
				7518 348904.006778338
				7519 348914.802568803
				7520 348925.598359267
				7521 348936.394149732
				7522 348947.189940196
				7523 348957.985730661
				7524 348968.781521126
				7525 348979.57731159
				7526 348990.373102055
				7527 349001.168892519
				7528 349011.964682984
				7529 349022.760473448
				7530 349033.556263913
				7531 349044.352054378
				7532 349055.147844842
				7533 349065.943635307
				7534 349076.739425771
				7535 349087.535216236
				7536 349098.331006701
				7537 349109.126797165
				7538 349119.92258763
				7539 349130.718378094
				7540 349141.514168559
				7541 349152.309959023
				7542 349163.105749488
				7543 349173.901539953
				7544 349184.697330417
				7545 349195.493120882
				7546 349206.288911346
				7547 349217.084701811
				7548 349227.880492275
				7549 349238.67628274
				7550 349249.472073205
				7551 349260.267863669
				7552 349271.063654134
				7553 349281.859444598
				7554 349292.655235063
				7555 349303.451025528
				7556 349314.246815992
				7557 349325.042606457
				7558 349335.838396921
				7559 349346.634187386
				7560 349357.42997785
				7561 349368.225768315
				7562 349379.02155878
				7563 349389.817349244
				7564 349400.613139709
				7565 349411.408930173
				7566 349422.204720638
				7567 349433.000511102
				7568 349443.796301567
				7569 349454.592092032
				7570 349465.387882496
				7571 349476.183672961
				7572 349486.979463425
				7573 349497.77525389
				7574 349508.571044355
				7575 349519.366834819
				7576 349530.162625284
				7577 349540.958415748
				7578 349551.754206213
				7579 349562.549996677
				7580 349573.345787142
				7581 349584.141577607
				7582 349594.937368071
				7583 349605.733158536
				7584 349616.528949
				7585 349627.324739465
				7586 349638.12052993
				7587 349648.916320394
				7588 349659.712110859
				7589 349670.507901323
				7590 349681.303691788
				7591 349692.099482252
				7592 349702.895272717
				7593 349713.691063182
				7594 349724.486853646
				7595 349735.282644111
				7596 349746.078434575
				7597 349756.87422504
				7598 349767.670015504
				7599 349778.465805969
				7600 349789.261596434
				7601 349800.057386898
				7602 349810.853177363
				7603 349821.648967827
				7604 349832.444758292
				7605 349843.240548757
				7606 349854.036339221
				7607 349864.832129686
				7608 349875.62792015
				7609 349886.423710615
				7610 349897.219501079
				7611 349908.015291544
				7612 349918.811082009
				7613 349929.606872473
				7614 349940.402662938
				7615 349951.198453402
				7616 349961.994243867
				7617 349972.790034331
				7618 349983.585824796
				7619 349994.381615261
				7620 350005.177405725
				7621 350015.97319619
				7622 350026.768986654
				7623 350037.564777119
				7624 350048.360567584
				7625 350059.156358048
				7626 350069.952148513
				7627 350080.747938977
				7628 350091.543729442
				7629 350102.339519906
				7630 350113.135310371
				7631 350123.931100836
				7632 350134.7268913
				7633 350145.522681765
				7634 350156.318472229
				7635 350167.114262694
				7636 350177.910053159
				7637 350188.705843623
				7638 350199.501634088
				7639 350210.297424552
				7640 350221.093215017
				7641 350231.889005481
				7642 350242.684795946
				7643 350253.480586411
				7644 350264.276376875
				7645 350275.07216734
				7646 350285.867957804
				7647 350296.663748269
				7648 350307.459538733
				7649 350318.255329198
				7650 350329.051119663
				7651 350339.846910127
				7652 350350.642700592
				7653 350361.438491056
				7654 350372.234281521
				7655 350383.030071986
				7656 350393.82586245
				7657 350404.621652915
				7658 350415.417443379
				7659 350426.213233844
				7660 350437.009024308
				7661 350447.804814773
				7662 350458.600605238
				7663 350469.396395702
				7664 350480.192186167
				7665 350490.987976631
				7666 350501.783767096
				7667 350512.579557561
				7668 350523.375348025
				7669 350534.17113849
				7670 350544.966928954
				7671 350555.762719419
				7672 350566.558509883
				7673 350577.354300348
				7674 350588.150090813
				7675 350598.945881277
				7676 350609.741671742
				7677 350620.537462206
				7678 350631.333252671
				7679 350642.129043135
				7680 350652.9248336
				7681 350663.720624065
				7682 350674.516414529
				7683 350685.312204994
				7684 350696.107995458
				7685 350706.903785923
				7686 350717.699576388
				7687 350728.495366852
				7688 350739.291157317
				7689 350750.086947781
				7690 350760.882738246
				7691 350771.67852871
				7692 350782.474319175
				7693 350793.27010964
				7694 350804.065900104
				7695 350814.861690569
				7696 350825.657481033
				7697 350836.453271498
				7698 350847.249061962
				7699 350858.044852427
				7700 350868.840642892
				7701 350879.636433356
				7702 350890.432223821
				7703 350901.228014285
				7704 350912.02380475
				7705 350922.819595215
				7706 350933.615385679
				7707 350944.411176144
				7708 350955.206966608
				7709 350966.002757073
				7710 350976.798547537
				7711 350987.594338002
				7712 350998.390128467
				7713 351009.185918931
				7714 351019.981709396
				7715 351030.77749986
				7716 351041.573290325
				7717 351052.36908079
				7718 351063.164871254
				7719 351073.960661719
				7720 351084.756452183
				7721 351095.552242648
				7722 351106.348033112
				7723 351117.143823577
				7724 351127.939614042
				7725 351138.735404506
				7726 351149.531194971
				7727 351160.326985435
				7728 351171.1227759
				7729 351181.918566364
				7730 351192.714356829
				7731 351203.510147294
				7732 351214.305937758
				7733 351225.101728223
				7734 351235.897518687
				7735 351246.693309152
				7736 351257.489099616
				7737 351268.284890081
				7738 351279.080680546
				7739 351289.87647101
				7740 351300.672261475
				7741 351311.468051939
				7742 351322.263842404
				7743 351333.059632869
				7744 351343.855423333
				7745 351354.651213798
				7746 351365.447004262
				7747 351376.242794727
				7748 351387.038585191
				7749 351397.834375656
				7750 351408.630166121
				7751 351419.425956585
				7752 351430.22174705
				7753 351441.017537514
				7754 351451.813327979
				7755 351462.609118444
				7756 351473.404908908
				7757 351484.200699373
				7758 351494.996489837
				7759 351505.792280302
				7760 351516.588070766
				7761 351527.383861231
				7762 351538.179651696
				7763 351548.97544216
				7764 351559.771232625
				7765 351570.567023089
				7766 351581.362813554
				7767 351592.158604018
				7768 351602.954394483
				7769 351613.750184948
				7770 351624.545975412
				7771 351635.341765877
				7772 351646.137556341
				7773 351656.933346806
				7774 351667.729137271
				7775 351678.524927735
				7776 351689.3207182
				7777 351700.116508664
				7778 351710.912299129
				7779 351721.708089593
				7780 351732.503880058
				7781 351743.299670523
				7782 351754.095460987
				7783 351764.891251452
				7784 351775.687041916
				7785 351786.482832381
				7786 351797.278622845
				7787 351808.07441331
				7788 351818.870203775
				7789 351829.665994239
				7790 351840.461784704
				7791 351851.257575168
				7792 351862.053365633
				7793 351872.849156098
				7794 351883.644946562
				7795 351894.440737027
				7796 351905.236527491
				7797 351916.032317956
				7798 351926.82810842
				7799 351937.623898885
				7800 351948.41968935
				7801 351959.215479814
				7802 351970.011270279
				7803 351980.807060743
				7804 351991.602851208
				7805 352002.398641673
				7806 352013.194432137
				7807 352023.990222602
				7808 352034.786013066
				7809 352045.581803531
				7810 352056.377593995
				7811 352067.17338446
				7812 352077.969174925
				7813 352088.764965389
				7814 352099.560755854
				7815 352110.356546318
				7816 352121.152336783
				7817 352131.948127247
				7818 352142.743917712
				7819 352153.539708177
				7820 352164.335498641
				7821 352175.131289106
				7822 352185.92707957
				7823 352196.722870035
				7824 352207.518660499
				7825 352218.314450964
				7826 352229.110241429
				7827 352239.906031893
				7828 352250.701822358
				7829 352261.497612822
				7830 352272.293403287
				7831 352283.089193752
				7832 352293.884984216
				7833 352304.680774681
				7834 352315.476565145
				7835 352326.27235561
				7836 352337.068146074
				7837 352347.863936539
				7838 352358.659727004
				7839 352369.455517468
				7840 352380.251307933
				7841 352391.047098397
				7842 352401.842888862
				7843 352412.638679327
				7844 352423.434469791
				7845 352434.230260256
				7846 352445.02605072
				7847 352455.821841185
				7848 352466.617631649
				7849 352477.413422114
				7850 352488.209212579
				7851 352499.005003043
				7852 352509.800793508
				7853 352520.596583972
				7854 352531.392374437
				7855 352542.188164902
				7856 352552.983955366
				7857 352563.779745831
				7858 352574.575536295
				7859 352585.37132676
				7860 352596.167117224
				7861 352606.962907689
				7862 352617.758698154
				7863 352628.554488618
				7864 352639.350279083
				7865 352650.146069547
				7866 352660.941860012
				7867 352671.737650476
				7868 352682.533440941
				7869 352693.329231406
				7870 352704.12502187
				7871 352714.920812335
				7872 352725.716602799
				7873 352736.512393264
				7874 352747.308183728
				7875 352758.103974193
				7876 352768.899764658
				7877 352779.695555122
				7878 352790.491345587
				7879 352801.287136051
				7880 352812.082926516
				7881 352822.878716981
				7882 352833.674507445
				7883 352844.47029791
				7884 352855.266088374
				7885 352866.061878839
				7886 352876.857669303
				7887 352887.653459768
				7888 352898.449250233
				7889 352909.245040697
				7890 352920.040831162
				7891 352930.836621626
				7892 352941.632412091
				7893 352952.428202556
				7894 352963.22399302
				7895 352974.019783485
				7896 352984.815573949
				7897 352995.611364414
				7898 353006.407154878
				7899 353017.202945343
				7900 353027.998735808
				7901 353038.794526272
				7902 353049.590316737
				7903 353060.386107201
				7904 353071.181897666
				7905 353081.97768813
				7906 353092.773478595
				7907 353103.56926906
				7908 353114.365059524
				7909 353125.160849989
				7910 353135.956640453
				7911 353146.752430918
				7912 353157.548221383
				7913 353168.344011847
				7914 353179.139802312
				7915 353189.935592776
				7916 353200.731383241
				7917 353211.527173705
				7918 353222.32296417
				7919 353233.118754635
				7920 353243.914545099
				7921 353254.710335564
				7922 353265.506126028
				7923 353276.301916493
				7924 353287.097706957
				7925 353297.893497422
				7926 353308.689287887
				7927 353319.485078351
				7928 353330.280868816
				7929 353341.07665928
				7930 353351.872449745
				7931 353362.66824021
				7932 353373.464030674
				7933 353384.259821139
				7934 353395.055611603
				7935 353405.851402068
				7936 353416.647192532
				7937 353427.442982997
				7938 353438.238773462
				7939 353449.034563926
				7940 353459.830354391
				7941 353470.626144855
				7942 353481.42193532
				7943 353492.217725785
				7944 353503.013516249
				7945 353513.809306714
				7946 353524.605097178
				7947 353535.400887643
				7948 353546.196678107
				7949 353556.992468572
				7950 353567.788259037
				7951 353578.584049501
				7952 353589.379839966
				7953 353600.17563043
				7954 353610.971420895
				7955 353621.767211359
				7956 353632.563001824
				7957 353643.358792289
				7958 353654.154582753
				7959 353664.950373218
				7960 353675.746163682
				7961 353686.541954147
				7962 353697.337744612
				7963 353708.133535076
				7964 353718.929325541
				7965 353729.725116005
				7966 353740.52090647
				7967 353751.316696934
				7968 353762.112487399
				7969 353772.908277864
				7970 353783.704068328
				7971 353794.499858793
				7972 353805.295649257
				7973 353816.091439722
				7974 353826.887230186
				7975 353837.683020651
				7976 353848.478811116
				7977 353859.27460158
				7978 353870.070392045
				7979 353880.866182509
				7980 353891.661972974
				7981 353902.457763439
				7982 353913.253553903
				7983 353924.049344368
				7984 353934.845134832
				7985 353945.640925297
				7986 353956.436715761
				7987 353967.232506226
				7988 353978.028296691
				7989 353988.824087155
				7990 353999.61987762
				7991 354010.415668084
				7992 354021.211458549
				7993 354032.007249014
				7994 354042.803039478
				7995 354053.598829943
				7996 354064.394620407
				7997 354075.190410872
				7998 354085.986201336
				7999 354096.781991801
				8000 354107.577782266
				8001 354118.37357273
				8002 354129.169363195
				8003 354139.965153659
				8004 354150.760944124
				8005 354161.556734588
				8006 354172.352525053
				8007 354183.148315518
				8008 354193.944105982
				8009 354204.739896447
				8010 354215.535686911
				8011 354226.331477376
				8012 354237.127267841
				8013 354247.923058305
				8014 354258.71884877
				8015 354269.514639234
				8016 354280.310429699
				8017 354291.106220163
				8018 354301.902010628
				8019 354312.697801093
				8020 354323.493591557
				8021 354334.289382022
				8022 354345.085172486
				8023 354355.880962951
				8024 354366.676753415
				8025 354377.47254388
				8026 354388.268334345
				8027 354399.064124809
				8028 354409.859915274
				8029 354420.655705738
				8030 354431.451496203
				8031 354442.247286668
				8032 354453.043077132
				8033 354463.838867597
				8034 354474.634658061
				8035 354485.430448526
				8036 354496.22623899
				8037 354507.022029455
				8038 354517.81781992
				8039 354528.613610384
				8040 354539.409400849
				8041 354550.205191313
				8042 354561.000981778
				8043 354571.796772243
				8044 354582.592562707
				8045 354593.388353172
				8046 354604.184143636
				8047 354614.979934101
				8048 354625.775724565
				8049 354636.57151503
				8050 354647.367305495
				8051 354658.163095959
				8052 354668.958886424
				8053 354679.754676888
				8054 354690.550467353
				8055 354701.346257817
				8056 354712.142048282
				8057 354722.937838747
				8058 354733.733629211
				8059 354744.529419676
				8060 354755.32521014
				8061 354766.121000605
				8062 354776.916791069
				8063 354787.712581534
				8064 354798.508371999
				8065 354809.304162463
				8066 354820.099952928
				8067 354830.895743392
				8068 354841.691533857
				8069 354852.487324322
				8070 354863.283114786
				8071 354874.078905251
				8072 354884.874695715
				8073 354895.67048618
				8074 354906.466276644
				8075 354917.262067109
				8076 354928.057857574
				8077 354938.853648038
				8078 354949.649438503
				8079 354960.445228967
				8080 354971.241019432
				8081 354982.036809897
				8082 354992.832600361
				8083 355003.628390826
				8084 355014.42418129
				8085 355025.219971755
				8086 355036.015762219
				8087 355046.811552684
				8088 355057.607343149
				8089 355068.403133613
				8090 355079.198924078
				8091 355089.994714542
				8092 355100.790505007
				8093 355111.586295471
				8094 355122.382085936
				8095 355133.177876401
				8096 355143.973666865
				8097 355154.76945733
				8098 355165.565247794
				8099 355176.361038259
				8100 355187.156828724
				8101 355197.952619188
				8102 355208.748409653
				8103 355219.544200117
				8104 355230.339990582
				8105 355241.135781046
				8106 355251.931571511
				8107 355262.727361976
				8108 355273.52315244
				8109 355284.318942905
				8110 355295.114733369
				8111 355305.910523834
				8112 355316.706314298
				8113 355327.502104763
				8114 355338.297895228
				8115 355349.093685692
				8116 355359.889476157
				8117 355370.685266621
				8118 355381.481057086
				8119 355392.276847551
				8120 355403.072638015
				8121 355413.86842848
				8122 355424.664218944
				8123 355435.460009409
				8124 355446.255799873
				8125 355457.051590338
				8126 355467.847380803
				8127 355478.643171267
				8128 355489.438961732
				8129 355500.234752196
				8130 355511.030542661
				8131 355521.826333126
				8132 355532.62212359
				8133 355543.417914055
				8134 355554.213704519
				8135 355565.009494984
				8136 355575.805285448
				8137 355586.601075913
				8138 355597.396866378
				8139 355608.192656842
				8140 355618.988447307
				8141 355629.784237771
				8142 355640.580028236
				8143 355651.3758187
				8144 355662.171609165
				8145 355672.96739963
				8146 355683.763190094
				8147 355694.558980559
				8148 355705.354771023
				8149 355716.150561488
				8150 355726.946351953
				8151 355737.742142417
				8152 355748.537932882
				8153 355759.333723346
				8154 355770.129513811
				8155 355780.925304275
				8156 355791.72109474
				8157 355802.516885205
				8158 355813.312675669
				8159 355824.108466134
				8160 355834.904256598
				8161 355845.700047063
				8162 355856.495837528
				8163 355867.291627992
				8164 355878.087418457
				8165 355888.883208921
				8166 355899.678999386
				8167 355910.47478985
				8168 355921.270580315
				8169 355932.06637078
				8170 355942.862161244
				8171 355953.657951709
				8172 355964.453742173
				8173 355975.249532638
				8174 355986.045323102
				8175 355996.841113567
				8176 356007.636904032
				8177 356018.432694496
				8178 356029.228484961
				8179 356040.024275425
				8180 356050.82006589
				8181 356061.615856354
				8182 356072.411646819
				8183 356083.207437284
				8184 356094.003227748
				8185 356104.799018213
				8186 356115.594808677
				8187 356126.390599142
				8188 356137.186389607
				8189 356147.982180071
				8190 356158.777970536
				8191 356169.573761
				8192 356180.369551465
				8193 356191.165341929
				8194 356201.961132394
				8195 356212.756922859
				8196 356223.552713323
				8197 356234.348503788
				8198 356245.144294252
				8199 356255.940084717
				8200 356266.735875182
				8201 356277.531665646
				8202 356288.327456111
				8203 356299.123246575
				8204 356309.91903704
				8205 356320.714827504
				8206 356331.510617969
				8207 356342.306408434
				8208 356353.102198898
				8209 356363.897989363
				8210 356374.693779827
				8211 356385.489570292
				8212 356396.285360756
				8213 356407.081151221
				8214 356417.876941686
				8215 356428.67273215
				8216 356439.468522615
				8217 356450.264313079
				8218 356461.060103544
				8219 356471.855894009
				8220 356482.651684473
				8221 356493.447474938
				8222 356504.243265402
				8223 356515.039055867
				8224 356525.834846331
				8225 356536.630636796
				8226 356547.426427261
				8227 356558.222217725
				8228 356569.01800819
				8229 356579.813798654
				8230 356590.609589119
				8231 356601.405379583
				8232 356612.201170048
				8233 356622.996960513
				8234 356633.792750977
				8235 356644.588541442
				8236 356655.384331906
				8237 356666.180122371
				8238 356676.975912836
				8239 356687.7717033
				8240 356698.567493765
				8241 356709.363284229
				8242 356720.159074694
				8243 356730.954865158
				8244 356741.750655623
				8245 356752.546446088
				8246 356763.342236552
				8247 356774.138027017
				8248 356784.933817481
				8249 356795.729607946
				8250 356806.525398411
				8251 356817.321188875
				8252 356828.11697934
				8253 356838.912769804
				8254 356849.708560269
				8255 356860.504350733
				8256 356871.300141198
				8257 356882.095931663
				8258 356892.891722127
				8259 356903.687512592
				8260 356914.483303056
				8261 356925.279093521
				8262 356936.074883985
				8263 356946.87067445
				8264 356957.666464915
				8265 356968.462255379
				8266 356979.258045844
				8267 356990.053836308
				8268 357000.849626773
				8269 357011.645417238
				8270 357022.441207702
				8271 357033.236998167
				8272 357044.032788631
				8273 357054.828579096
				8274 357065.62436956
				8275 357076.420160025
				8276 357087.21595049
				8277 357098.011740954
				8278 357108.807531419
				8279 357119.603321883
				8280 357130.399112348
				8281 357141.194902812
				8282 357151.990693277
				8283 357162.786483742
				8284 357173.582274206
				8285 357184.378064671
				8286 357195.173855135
				8287 357205.9696456
				8288 357216.765436065
				8289 357227.561226529
				8290 357238.357016994
				8291 357249.152807458
				8292 357259.948597923
				8293 357270.744388387
				8294 357281.540178852
				8295 357292.335969317
				8296 357303.131759781
				8297 357313.927550246
				8298 357324.72334071
				8299 357335.519131175
				8300 357346.31492164
				8301 357357.110712104
				8302 357367.906502569
				8303 357378.702293033
				8304 357389.498083498
				8305 357400.293873962
				8306 357411.089664427
				8307 357421.885454892
				8308 357432.681245356
				8309 357443.477035821
				8310 357454.272826285
				8311 357465.06861675
				8312 357475.864407214
				8313 357486.660197679
				8314 357497.455988144
				8315 357508.251778608
				8316 357519.047569073
				8317 357529.843359537
				8318 357540.639150002
				8319 357551.434940467
				8320 357562.230730931
				8321 357573.026521396
				8322 357583.82231186
				8323 357594.618102325
				8324 357605.413892789
				8325 357616.209683254
				8326 357627.005473719
				8327 357637.801264183
				8328 357648.597054648
				8329 357659.392845112
				8330 357670.188635577
				8331 357680.984426041
				8332 357691.780216506
				8333 357702.576006971
				8334 357713.371797435
				8335 357724.1675879
				8336 357734.963378364
				8337 357745.759168829
				8338 357756.554959294
				8339 357767.350749758
				8340 357778.146540223
				8341 357788.942330687
				8342 357799.738121152
				8343 357810.533911616
				8344 357821.329702081
				8345 357832.125492546
				8346 357842.92128301
				8347 357853.717073475
				8348 357864.512863939
				8349 357875.308654404
				8350 357886.104444869
				8351 357896.900235333
				8352 357907.696025798
				8353 357918.491816262
				8354 357929.287606727
				8355 357940.083397191
				8356 357950.879187656
				8357 357961.674978121
				8358 357972.470768585
				8359 357983.26655905
				8360 357994.062349514
				8361 358004.858139979
				8362 358015.653930443
				8363 358026.449720908
				8364 358037.245511373
				8365 358048.041301837
				8366 358058.837092302
				8367 358069.632882766
				8368 358080.428673231
				8369 358091.224463696
				8370 358102.02025416
				8371 358112.816044625
				8372 358123.611835089
				8373 358134.407625554
				8374 358145.203416018
				8375 358155.999206483
				8376 358166.794996948
				8377 358177.590787412
				8378 358188.386577877
				8379 358199.182368341
				8380 358209.978158806
				8381 358220.77394927
				8382 358231.569739735
				8383 358242.3655302
				8384 358253.161320664
				8385 358263.957111129
				8386 358274.752901593
				8387 358285.548692058
				8388 358296.344482523
				8389 358307.140272987
				8390 358317.936063452
				8391 358328.731853916
				8392 358339.527644381
				8393 358350.323434845
				8394 358361.11922531
				8395 358371.915015775
				8396 358382.710806239
				8397 358393.506596704
				8398 358404.302387168
				8399 358415.098177633
				8400 358425.893968098
				8401 358436.689758562
				8402 358447.485549027
				8403 358458.281339491
				8404 358469.077129956
				8405 358479.87292042
				8406 358490.668710885
				8407 358501.46450135
				8408 358512.260291814
				8409 358523.056082279
				8410 358533.851872743
				8411 358544.647663208
				8412 358555.443453672
				8413 358566.239244137
				8414 358577.035034602
				8415 358587.830825066
				8416 358598.626615531
				8417 358609.422405995
				8418 358620.21819646
				8419 358631.013986924
				8420 358641.809777389
				8421 358652.605567854
				8422 358663.401358318
				8423 358674.197148783
				8424 358684.992939247
				8425 358695.788729712
				8426 358706.584520177
				8427 358717.380310641
				8428 358728.176101106
				8429 358738.97189157
				8430 358749.767682035
				8431 358760.563472499
				8432 358771.359262964
				8433 358782.155053429
				8434 358792.950843893
				8435 358803.746634358
				8436 358814.542424822
				8437 358825.338215287
				8438 358836.134005752
				8439 358846.929796216
				8440 358857.725586681
				8441 358868.521377145
				8442 358879.31716761
				8443 358890.112958074
				8444 358900.908748539
				8445 358911.704539004
				8446 358922.500329468
				8447 358933.296119933
				8448 358944.091910397
				8449 358954.887700862
				8450 358965.683491326
				8451 358976.479281791
				8452 358987.275072256
				8453 358998.07086272
				8454 359008.866653185
				8455 359019.662443649
				8456 359030.458234114
				8457 359041.254024579
				8458 359052.049815043
				8459 359062.845605508
				8460 359073.641395972
				8461 359084.437186437
				8462 359095.232976901
				8463 359106.028767366
				8464 359116.824557831
				8465 359127.620348295
				8466 359138.41613876
				8467 359149.211929224
				8468 359160.007719689
				8469 359170.803510153
				8470 359181.599300618
				8471 359192.395091083
				8472 359203.190881547
				8473 359213.986672012
				8474 359224.782462476
				8475 359235.578252941
				8476 359246.374043406
				8477 359257.16983387
				8478 359267.965624335
				8479 359278.761414799
				8480 359289.557205264
				8481 359300.352995728
				8482 359311.148786193
				8483 359321.944576658
				8484 359332.740367122
				8485 359343.536157587
				8486 359354.331948051
				8487 359365.127738516
				8488 359375.923528981
				8489 359386.719319445
				8490 359397.51510991
				8491 359408.310900374
				8492 359419.106690839
				8493 359429.902481303
				8494 359440.698271768
				8495 359451.494062233
				8496 359462.289852697
				8497 359473.085643162
				8498 359483.881433626
				8499 359494.677224091
				8500 359505.473014555
				8501 359516.26880502
				8502 359527.064595485
				8503 359537.860385949
				8504 359548.656176414
				8505 359559.451966878
				8506 359570.247757343
				8507 359581.043547807
				8508 359591.839338272
				8509 359602.635128737
				8510 359613.430919201
				8511 359624.226709666
				8512 359635.02250013
				8513 359645.818290595
				8514 359656.61408106
				8515 359667.409871524
				8516 359678.205661989
				8517 359689.001452453
				8518 359699.797242918
				8519 359710.593033382
				8520 359721.388823847
				8521 359732.184614312
				8522 359742.980404776
				8523 359753.776195241
				8524 359764.571985705
				8525 359775.36777617
				8526 359786.163566635
				8527 359796.959357099
				8528 359807.755147564
				8529 359818.550938028
				8530 359829.346728493
				8531 359840.142518957
				8532 359850.938309422
				8533 359861.734099887
				8534 359872.529890351
				8535 359883.325680816
				8536 359894.12147128
				8537 359904.917261745
				8538 359915.713052209
				8539 359926.508842674
				8540 359937.304633139
				8541 359948.100423603
				8542 359958.896214068
				8543 359969.692004532
				8544 359980.487794997
				8545 359991.283585462
				8546 360002.079375926
				8547 360012.875166391
				8548 360023.670956855
				8549 360034.46674732
				8550 360045.262537784
				8551 360056.058328249
				8552 360066.854118714
				8553 360077.649909178
				8554 360088.445699643
				8555 360099.241490107
				8556 360110.037280572
				8557 360120.833071036
				8558 360131.628861501
				8559 360142.424651966
				8560 360153.22044243
				8561 360164.016232895
				8562 360174.812023359
				8563 360185.607813824
				8564 360196.403604289
				8565 360207.199394753
				8566 360217.995185218
				8567 360228.790975682
				8568 360239.586766147
				8569 360250.382556611
				8570 360261.178347076
				8571 360271.974137541
				8572 360282.769928005
				8573 360293.56571847
				8574 360304.361508934
				8575 360315.157299399
				8576 360325.953089864
				8577 360336.748880328
				8578 360347.544670793
				8579 360358.340461257
				8580 360369.136251722
				8581 360379.932042186
				8582 360390.727832651
				8583 360401.523623116
				8584 360412.31941358
				8585 360423.115204045
				8586 360433.910994509
				8587 360444.706784974
				8588 360455.502575438
				8589 360466.298365903
				8590 360477.094156368
				8591 360487.889946832
				8592 360498.685737297
				8593 360509.481527761
				8594 360520.277318226
				8595 360531.073108691
				8596 360541.868899155
				8597 360552.66468962
				8598 360563.460480084
				8599 360574.256270549
				8600 360585.052061013
				8601 360595.847851478
				8602 360606.643641943
				8603 360617.439432407
				8604 360628.235222872
				8605 360639.031013336
				8606 360649.826803801
				8607 360660.622594266
				8608 360671.41838473
				8609 360682.214175195
				8610 360693.009965659
				8611 360703.805756124
				8612 360714.601546588
				8613 360725.397337053
				8614 360736.193127518
				8615 360746.988917982
				8616 360757.784708447
				8617 360768.580498911
				8618 360779.376289376
				8619 360790.17207984
				8620 360800.967870305
				8621 360811.76366077
				8622 360822.559451234
				8623 360833.355241699
				8624 360844.151032163
				8625 360854.946822628
				8626 360865.742613093
				8627 360876.538403557
				8628 360887.334194022
				8629 360898.129984486
				8630 360908.925774951
				8631 360919.721565415
				8632 360930.51735588
				8633 360941.313146345
				8634 360952.108936809
				8635 360962.904727274
				8636 360973.700517738
				8637 360984.496308203
				8638 360995.292098667
				8639 361006.087889132
				8640 361016.883679597
				8641 361027.679470061
				8642 361038.475260526
				8643 361049.27105099
				8644 361060.066841455
				8645 361070.86263192
				8646 361081.658422384
				8647 361092.454212849
				8648 361103.250003313
				8649 361114.045793778
				8650 361124.841584242
				8651 361135.637374707
				8652 361146.433165172
				8653 361157.228955636
				8654 361168.024746101
				8655 361178.820536565
				8656 361189.61632703
				8657 361200.412117495
				8658 361211.207907959
				8659 361222.003698424
				8660 361232.799488888
				8661 361243.595279353
				8662 361254.391069817
				8663 361265.186860282
				8664 361275.982650747
				8665 361286.778441211
				8666 361297.574231676
				8667 361308.37002214
				8668 361319.165812605
				8669 361329.961603069
				8670 361340.757393534
				8671 361351.553183999
				8672 361362.348974463
				8673 361373.144764928
				8674 361383.940555392
				8675 361394.736345857
				8676 361405.532136322
				8677 361416.327926786
				8678 361427.123717251
				8679 361437.919507715
				8680 361448.71529818
				8681 361459.511088644
				8682 361470.306879109
				8683 361481.102669574
				8684 361491.898460038
				8685 361502.694250503
				8686 361513.490040967
				8687 361524.285831432
				8688 361535.081621896
				8689 361545.877412361
				8690 361556.673202826
				8691 361567.46899329
				8692 361578.264783755
				8693 361589.060574219
				8694 361599.856364684
				8695 361610.652155149
				8696 361621.447945613
				8697 361632.243736078
				8698 361643.039526542
				8699 361653.835317007
				8700 361664.631107471
				8701 361675.426897936
				8702 361686.222688401
				8703 361697.018478865
				8704 361707.81426933
				8705 361718.610059794
				8706 361729.405850259
				8707 361740.201640724
				8708 361750.997431188
				8709 361761.793221653
				8710 361772.589012117
				8711 361783.384802582
				8712 361794.180593046
				8713 361804.976383511
				8714 361815.772173976
				8715 361826.56796444
				8716 361837.363754905
				8717 361848.159545369
				8718 361858.955335834
				8719 361869.751126298
				8720 361880.546916763
				8721 361891.342707228
				8722 361902.138497692
				8723 361912.934288157
				8724 361923.730078621
				8725 361934.525869086
				8726 361945.321659551
				8727 361956.117450015
				8728 361966.91324048
				8729 361977.709030944
				8730 361988.504821409
				8731 361999.300611873
				8732 362010.096402338
				8733 362020.892192803
				8734 362031.687983267
				8735 362042.483773732
				8736 362053.279564196
				8737 362064.075354661
				8738 362074.871145125
				8739 362085.66693559
				8740 362096.462726055
				8741 362107.258516519
				8742 362118.054306984
				8743 362128.850097448
				8744 362139.645887913
				8745 362150.441678378
				8746 362161.237468842
				8747 362172.033259307
				8748 362182.829049771
				8749 362193.624840236
				8750 362204.4206307
				8751 362215.216421165
				8752 362226.01221163
				8753 362236.808002094
				8754 362247.603792559
				8755 362258.399583023
				8756 362269.195373488
				8757 362279.991163953
				8758 362290.786954417
				8759 362301.582744882
				8760 362312.378535346
				8761 362323.174325811
				8762 362333.970116275
				8763 362344.76590674
				8764 362355.561697205
				8765 362366.357487669
				8766 362377.153278134
				8767 362387.949068598
				8768 362398.744859063
				8769 362409.540649527
				8770 362420.336439992
				8771 362431.132230457
				8772 362441.928020921
				8773 362452.723811386
				8774 362463.51960185
				8775 362474.315392315
				8776 362485.111182779
				8777 362495.906973244
				8778 362506.702763709
				8779 362517.498554173
				8780 362528.294344638
				8781 362539.090135102
				8782 362549.885925567
				8783 362560.681716032
				8784 362571.477506496
				8785 362582.273296961
				8786 362593.069087425
				8787 362603.86487789
				8788 362614.660668354
				8789 362625.456458819
				8790 362636.252249284
				8791 362647.048039748
				8792 362657.843830213
				8793 362668.639620677
				8794 362679.435411142
				8795 362690.231201607
				8796 362701.026992071
				8797 362711.822782536
				8798 362722.618573
				8799 362733.414363465
				8800 362744.210153929
				8801 362755.005944394
				8802 362765.801734859
				8803 362776.597525323
				8804 362787.393315788
				8805 362798.189106252
				8806 362808.984896717
				8807 362819.780687181
				8808 362830.576477646
				8809 362841.372268111
				8810 362852.168058575
				8811 362862.96384904
				8812 362873.759639504
				8813 362884.555429969
				8814 362895.351220434
				8815 362906.147010898
				8816 362916.942801363
				8817 362927.738591827
				8818 362938.534382292
				8819 362949.330172756
				8820 362960.125963221
				8821 362970.921753686
				8822 362981.71754415
				8823 362992.513334615
				8824 363003.309125079
				8825 363014.104915544
				8826 363024.900706008
				8827 363035.696496473
				8828 363046.492286938
				8829 363057.288077402
				8830 363068.083867867
				8831 363078.879658331
				8832 363089.675448796
				8833 363100.471239261
				8834 363111.267029725
				8835 363122.06282019
				8836 363132.858610654
				8837 363143.654401119
				8838 363154.450191583
				8839 363165.245982048
				8840 363176.041772513
				8841 363186.837562977
				8842 363197.633353442
				8843 363208.429143906
				8844 363219.224934371
				8845 363230.020724836
				8846 363240.8165153
				8847 363251.612305765
				8848 363262.408096229
				8849 363273.203886694
				8850 363283.999677158
				8851 363294.795467623
				8852 363305.591258088
				8853 363316.387048552
				8854 363327.182839017
				8855 363337.978629481
				8856 363348.774419946
				8857 363359.57021041
				8858 363370.366000875
				8859 363381.16179134
				8860 363391.957581804
				8861 363402.753372269
				8862 363413.549162733
				8863 363424.344953198
				8864 363435.140743662
				8865 363445.936534127
				8866 363456.732324592
				8867 363467.528115056
				8868 363478.323905521
				8869 363489.119695985
				8870 363499.91548645
				8871 363510.711276915
				8872 363521.507067379
				8873 363532.302857844
				8874 363543.098648308
				8875 363553.894438773
				8876 363564.690229237
				8877 363575.486019702
				8878 363586.281810167
				8879 363597.077600631
				8880 363607.873391096
				8881 363618.66918156
				8882 363629.464972025
				8883 363640.26076249
				8884 363651.056552954
				8885 363661.852343419
				8886 363672.648133883
				8887 363683.443924348
				8888 363694.239714812
				8889 363705.035505277
				8890 363715.831295742
				8891 363726.627086206
				8892 363737.422876671
				8893 363748.218667135
				8894 363759.0144576
				8895 363769.810248064
				8896 363780.606038529
				8897 363791.401828994
				8898 363802.197619458
				8899 363812.993409923
				8900 363823.789200387
				8901 363834.584990852
				8902 363845.380781317
				8903 363856.176571781
				8904 363866.972362246
				8905 363877.76815271
				8906 363888.563943175
				8907 363899.359733639
				8908 363910.155524104
				8909 363920.951314569
				8910 363931.747105033
				8911 363942.542895498
				8912 363953.338685962
				8913 363964.134476427
				8914 363974.930266891
				8915 363985.726057356
				8916 363996.521847821
				8917 364007.317638285
				8918 364018.11342875
				8919 364028.909219214
				8920 364039.705009679
				8921 364050.500800144
				8922 364061.296590608
				8923 364072.092381073
				8924 364082.888171537
				8925 364093.683962002
				8926 364104.479752466
				8927 364115.275542931
				8928 364126.071333396
				8929 364136.86712386
				8930 364147.662914325
				8931 364158.458704789
				8932 364169.254495254
				8933 364180.050285719
				8934 364190.846076183
				8935 364201.641866648
				8936 364212.437657112
				8937 364223.233447577
				8938 364234.029238041
				8939 364244.825028506
				8940 364255.620818971
				8941 364266.416609435
				8942 364277.2123999
				8943 364288.008190364
				8944 364298.803980829
				8945 364309.599771293
				8946 364320.395561758
				8947 364331.191352223
				8948 364341.987142687
				8949 364352.782933152
				8950 364363.578723616
				8951 364374.374514081
				8952 364385.170304546
				8953 364395.96609501
				8954 364406.761885475
				8955 364417.557675939
				8956 364428.353466404
				8957 364439.149256868
				8958 364449.945047333
				8959 364460.740837798
				8960 364471.536628262
				8961 364482.332418727
				8962 364493.128209191
				8963 364503.923999656
				8964 364514.71979012
				8965 364525.515580585
				8966 364536.31137105
				8967 364547.107161514
				8968 364557.902951979
				8969 364568.698742443
				8970 364579.494532908
				8971 364590.290323373
				8972 364601.086113837
				8973 364611.881904302
				8974 364622.677694766
				8975 364633.473485231
				8976 364644.269275695
				8977 364655.06506616
				8978 364665.860856625
				8979 364676.656647089
				8980 364687.452437554
				8981 364698.248228018
				8982 364709.044018483
				8983 364719.839808948
				8984 364730.635599412
				8985 364741.431389877
				8986 364752.227180341
				8987 364763.022970806
				8988 364773.81876127
				8989 364784.614551735
				8990 364795.4103422
				8991 364806.206132664
				8992 364817.001923129
				8993 364827.797713593
				8994 364838.593504058
				8995 364849.389294522
				8996 364860.185084987
				8997 364870.980875452
				8998 364881.776665916
				8999 364892.572456381
				9000 364903.368246845
				9001 364914.16403731
				9002 364924.959827775
				9003 364935.755618239
				9004 364946.551408704
				9005 364957.347199168
				9006 364968.142989633
				9007 364978.938780097
				9008 364989.734570562
				9009 365000.530361027
				9010 365011.326151491
				9011 365022.121941956
				9012 365032.91773242
				9013 365043.713522885
				9014 365054.509313349
				9015 365065.305103814
				9016 365076.100894279
				9017 365086.896684743
				9018 365097.692475208
				9019 365108.488265672
				9020 365119.284056137
				9021 365130.079846602
				9022 365140.875637066
				9023 365151.671427531
				9024 365162.467217995
				9025 365173.26300846
				9026 365184.058798924
				9027 365194.854589389
				9028 365205.650379854
				9029 365216.446170318
				9030 365227.241960783
				9031 365238.037751247
				9032 365248.833541712
				9033 365259.629332177
				9034 365270.425122641
				9035 365281.220913106
				9036 365292.01670357
				9037 365302.812494035
				9038 365313.608284499
				9039 365324.404074964
				9040 365335.199865429
				9041 365345.995655893
				9042 365356.791446358
				9043 365367.587236822
				9044 365378.383027287
				9045 365389.178817751
				9046 365399.974608216
				9047 365410.770398681
				9048 365421.566189145
				9049 365432.36197961
				9050 365443.157770074
				9051 365453.953560539
				9052 365464.749351003
				9053 365475.545141468
				9054 365486.340931933
				9055 365497.136722397
				9056 365507.932512862
				9057 365518.728303326
				9058 365529.524093791
				9059 365540.319884256
				9060 365551.11567472
				9061 365561.911465185
				9062 365572.707255649
				9063 365583.503046114
				9064 365594.298836578
				9065 365605.094627043
				9066 365615.890417508
				9067 365626.686207972
				9068 365637.481998437
				9069 365648.277788901
				9070 365659.073579366
				9071 365669.869369831
				9072 365680.665160295
				9073 365691.46095076
				9074 365702.256741224
				9075 365713.052531689
				9076 365723.848322153
				9077 365734.644112618
				9078 365745.439903083
				9079 365756.235693547
				9080 365767.031484012
				9081 365777.827274476
				9082 365788.623064941
				9083 365799.418855406
				9084 365810.21464587
				9085 365821.010436335
				9086 365831.806226799
				9087 365842.602017264
				9088 365853.397807728
				9089 365864.193598193
				9090 365874.989388658
				9091 365885.785179122
				9092 365896.580969587
				9093 365907.376760051
				9094 365918.172550516
				9095 365928.96834098
				9096 365939.764131445
				9097 365950.55992191
				9098 365961.355712374
				9099 365972.151502839
				9100 365982.947293303
				9101 365993.743083768
				9102 366004.538874233
				9103 366015.334664697
				9104 366026.130455162
				9105 366036.926245626
				9106 366047.722036091
				9107 366058.517826555
				9108 366069.31361702
				9109 366080.109407485
				9110 366090.905197949
				9111 366101.700988414
				9112 366112.496778878
				9113 366123.292569343
				9114 366134.088359808
				9115 366144.884150272
				9116 366155.679940737
				9117 366166.475731201
				9118 366177.271521666
				9119 366188.06731213
				9120 366198.863102595
				9121 366209.65889306
				9122 366220.454683524
				9123 366231.250473989
				9124 366242.046264453
				9125 366252.842054918
				9126 366263.637845382
				9127 366274.433635847
				9128 366285.229426312
				9129 366296.025216776
				9130 366306.821007241
				9131 366317.616797705
				9132 366328.41258817
				9133 366339.208378634
				9134 366350.004169099
				9135 366360.799959564
				9136 366371.595750028
				9137 366382.391540493
				9138 366393.187330957
				9139 366403.983121422
				9140 366414.778911887
				9141 366425.574702351
				9142 366436.370492816
				9143 366447.16628328
				9144 366457.962073745
				9145 366468.757864209
				9146 366479.553654674
				9147 366490.349445139
				9148 366501.145235603
				9149 366511.941026068
				9150 366522.736816532
				9151 366533.532606997
				9152 366544.328397462
				9153 366555.124187926
				9154 366565.919978391
				9155 366576.715768855
				9156 366587.51155932
				9157 366598.307349784
				9158 366609.103140249
				9159 366619.898930714
				9160 366630.694721178
				9161 366641.490511643
				9162 366652.286302107
				9163 366663.082092572
				9164 366673.877883036
				9165 366684.673673501
				9166 366695.469463966
				9167 366706.26525443
				9168 366717.061044895
				9169 366727.856835359
				9170 366738.652625824
				9171 366749.448416289
				9172 366760.244206753
				9173 366771.039997218
				9174 366781.835787682
				9175 366792.631578147
				9176 366803.427368611
				9177 366814.223159076
				9178 366825.018949541
				9179 366835.814740005
				9180 366846.61053047
				9181 366857.406320934
				9182 366868.202111399
				9183 366878.997901863
				9184 366889.793692328
				9185 366900.589482793
				9186 366911.385273257
				9187 366922.181063722
				9188 366932.976854186
				9189 366943.772644651
				9190 366954.568435116
				9191 366965.36422558
				9192 366976.160016045
				9193 366986.955806509
				9194 366997.751596974
				9195 367008.547387438
				9196 367019.343177903
				9197 367030.138968368
				9198 367040.934758832
				9199 367051.730549297
				9200 367062.526339761
				9201 367073.322130226
				9202 367084.117920691
				9203 367094.913711155
				9204 367105.70950162
				9205 367116.505292084
				9206 367127.301082549
				9207 367138.096873013
				9208 367148.892663478
				9209 367159.688453943
				9210 367170.484244407
				9211 367181.280034872
				9212 367192.075825336
				9213 367202.871615801
				9214 367213.667406265
				9215 367224.46319673
				9216 367235.258987195
				9217 367246.054777659
				9218 367256.850568124
				9219 367267.646358588
				9220 367278.442149053
				9221 367289.237939517
				9222 367300.033729982
				9223 367310.829520447
				9224 367321.625310911
				9225 367332.421101376
				9226 367343.21689184
				9227 367354.012682305
				9228 367364.80847277
				9229 367375.604263234
				9230 367386.400053699
				9231 367397.195844163
				9232 367407.991634628
				9233 367418.787425092
				9234 367429.583215557
				9235 367440.379006022
				9236 367451.174796486
				9237 367461.970586951
				9238 367472.766377415
				9239 367483.56216788
				9240 367494.357958345
				9241 367505.153748809
				9242 367515.949539274
				9243 367526.745329738
				9244 367537.541120203
				9245 367548.336910667
				9246 367559.132701132
				9247 367569.928491597
				9248 367580.724282061
				9249 367591.520072526
				9250 367602.31586299
				9251 367613.111653455
				9252 367623.907443919
				9253 367634.703234384
				9254 367645.499024849
				9255 367656.294815313
				9256 367667.090605778
				9257 367677.886396242
				9258 367688.682186707
				9259 367699.477977172
				9260 367710.273767636
				9261 367721.069558101
				9262 367731.865348565
				9263 367742.66113903
				9264 367753.456929494
				9265 367764.252719959
				9266 367775.048510424
				9267 367785.844300888
				9268 367796.640091353
				9269 367807.435881817
				9270 367818.231672282
				9271 367829.027462746
				9272 367839.823253211
				9273 367850.619043676
				9274 367861.41483414
				9275 367872.210624605
				9276 367883.006415069
				9277 367893.802205534
				9278 367904.597995999
				9279 367915.393786463
				9280 367926.189576928
				9281 367936.985367392
				9282 367947.781157857
				9283 367958.576948321
				9284 367969.372738786
				9285 367980.168529251
				9286 367990.964319715
				9287 368001.76011018
				9288 368012.555900644
				9289 368023.351691109
				9290 368034.147481574
				9291 368044.943272038
				9292 368055.739062503
				9293 368066.534852967
				9294 368077.330643432
				9295 368088.126433896
				9296 368098.922224361
				9297 368109.718014826
				9298 368120.51380529
				9299 368131.309595755
				9300 368142.105386219
				9301 368152.901176684
				9302 368163.696967148
				9303 368174.492757613
				9304 368185.288548078
				9305 368196.084338542
				9306 368206.880129007
				9307 368217.675919471
				9308 368228.471709936
				9309 368239.267500401
				9310 368250.063290865
				9311 368260.85908133
				9312 368271.654871794
				9313 368282.450662259
				9314 368293.246452723
				9315 368304.042243188
				9316 368314.838033653
				9317 368325.633824117
				9318 368336.429614582
				9319 368347.225405046
				9320 368358.021195511
				9321 368368.816985975
				9322 368379.61277644
				9323 368390.408566905
				9324 368401.204357369
				9325 368412.000147834
				9326 368422.795938298
				9327 368433.591728763
				9328 368444.387519228
				9329 368455.183309692
				9330 368465.979100157
				9331 368476.774890621
				9332 368487.570681086
				9333 368498.36647155
				9334 368509.162262015
				9335 368519.95805248
				9336 368530.753842944
				9337 368541.549633409
				9338 368552.345423873
				9339 368563.141214338
				9340 368573.937004803
				9341 368584.732795267
				9342 368595.528585732
				9343 368606.324376196
				9344 368617.120166661
				9345 368627.915957125
				9346 368638.71174759
				9347 368649.507538055
				9348 368660.303328519
				9349 368671.099118984
				9350 368681.894909448
				9351 368692.690699913
				9352 368703.486490377
				9353 368714.282280842
				9354 368725.078071307
				9355 368735.873861771
				9356 368746.669652236
				9357 368757.4654427
				9358 368768.261233165
				9359 368779.05702363
				9360 368789.852814094
				9361 368800.648604559
				9362 368811.444395023
				9363 368822.240185488
				9364 368833.035975952
				9365 368843.831766417
				9366 368854.627556882
				9367 368865.423347346
				9368 368876.219137811
				9369 368887.014928275
				9370 368897.81071874
				9371 368908.606509204
				9372 368919.402299669
				9373 368930.198090134
				9374 368940.993880598
				9375 368951.789671063
				9376 368962.585461527
				9377 368973.381251992
				9378 368984.177042457
				9379 368994.972832921
				9380 369005.768623386
				9381 369016.56441385
				9382 369027.360204315
				9383 369038.155994779
				9384 369048.951785244
				9385 369059.747575709
				9386 369070.543366173
				9387 369081.339156638
				9388 369092.134947102
				9389 369102.930737567
				9390 369113.726528032
				9391 369124.522318496
				9392 369135.318108961
				9393 369146.113899425
				9394 369156.90968989
				9395 369167.705480354
				9396 369178.501270819
				9397 369189.297061284
				9398 369200.092851748
				9399 369210.888642213
				9400 369221.684432677
				9401 369232.480223142
				9402 369243.276013606
				9403 369254.071804071
				9404 369264.867594536
				9405 369275.663385
				9406 369286.459175465
				9407 369297.254965929
				9408 369308.050756394
				9409 369318.846546858
				9410 369329.642337323
				9411 369340.438127788
				9412 369351.233918252
				9413 369362.029708717
				9414 369372.825499181
				9415 369383.621289646
				9416 369394.417080111
				9417 369405.212870575
				9418 369416.00866104
				9419 369426.804451504
				9420 369437.600241969
				9421 369448.396032433
				9422 369459.191822898
				9423 369469.987613363
				9424 369480.783403827
				9425 369491.579194292
				9426 369502.374984756
				9427 369513.170775221
				9428 369523.966565686
				9429 369534.76235615
				9430 369545.558146615
				9431 369556.353937079
				9432 369567.149727544
				9433 369577.945518008
				9434 369588.741308473
				9435 369599.537098938
				9436 369610.332889402
				9437 369621.128679867
				9438 369631.924470331
				9439 369642.720260796
				9440 369653.51605126
				9441 369664.311841725
				9442 369675.10763219
				9443 369685.903422654
				9444 369696.699213119
				9445 369707.495003583
				9446 369718.290794048
				9447 369729.086584513
				9448 369739.882374977
				9449 369750.678165442
				9450 369761.473955906
				9451 369772.269746371
				9452 369783.065536835
				9453 369793.8613273
				9454 369804.657117765
				9455 369815.452908229
				9456 369826.248698694
				9457 369837.044489158
				9458 369847.840279623
				9459 369858.636070087
				9460 369869.431860552
				9461 369880.227651017
				9462 369891.023441481
				9463 369901.819231946
				9464 369912.61502241
				9465 369923.410812875
				9466 369934.20660334
				9467 369945.002393804
				9468 369955.798184269
				9469 369966.593974733
				9470 369977.389765198
				9471 369988.185555662
				9472 369998.981346127
				9473 370009.777136592
				9474 370020.572927056
				9475 370031.368717521
				9476 370042.164507985
				9477 370052.96029845
				9478 370063.756088915
				9479 370074.551879379
				9480 370085.347669844
				9481 370096.143460308
				9482 370106.939250773
				9483 370117.735041237
				9484 370128.530831702
				9485 370139.326622167
				9486 370150.122412631
				9487 370160.918203096
				9488 370171.71399356
				9489 370182.509784025
				9490 370193.305574489
				9491 370204.101364954
				9492 370214.897155419
				9493 370225.692945883
				9494 370236.488736348
				9495 370247.284526812
				9496 370258.080317277
				9497 370268.876107742
				9498 370279.671898206
				9499 370290.467688671
				9500 370301.263479135
				9501 370312.0592696
				9502 370322.855060064
				9503 370333.650850529
				9504 370344.446640994
				9505 370355.242431458
				9506 370366.038221923
				9507 370376.834012387
				9508 370387.629802852
				9509 370398.425593316
				9510 370409.221383781
				9511 370420.017174246
				9512 370430.81296471
				9513 370441.608755175
				9514 370452.404545639
				9515 370463.200336104
				9516 370473.996126569
				9517 370484.791917033
				9518 370495.587707498
				9519 370506.383497962
				9520 370517.179288427
				9521 370527.975078891
				9522 370538.770869356
				9523 370549.566659821
				9524 370560.362450285
				9525 370571.15824075
				9526 370581.954031214
				9527 370592.749821679
				9528 370603.545612144
				9529 370614.341402608
				9530 370625.137193073
				9531 370635.932983537
				9532 370646.728774002
				9533 370657.524564466
				9534 370668.320354931
				9535 370679.116145396
				9536 370689.91193586
				9537 370700.707726325
				9538 370711.503516789
				9539 370722.299307254
				9540 370733.095097718
				9541 370743.890888183
				9542 370754.686678648
				9543 370765.482469112
				9544 370776.278259577
				9545 370787.074050041
				9546 370797.869840506
				9547 370808.66563097
				9548 370819.461421435
				9549 370830.2572119
				9550 370841.053002364
				9551 370851.848792829
				9552 370862.644583293
				9553 370873.440373758
				9554 370884.236164223
				9555 370895.031954687
				9556 370905.827745152
				9557 370916.623535616
				9558 370927.419326081
				9559 370938.215116545
				9560 370949.01090701
				9561 370959.806697475
				9562 370970.602487939
				9563 370981.398278404
				9564 370992.194068868
				9565 371002.989859333
				9566 371013.785649798
				9567 371024.581440262
				9568 371035.377230727
				9569 371046.173021191
				9570 371056.968811656
				9571 371067.76460212
				9572 371078.560392585
				9573 371089.35618305
				9574 371100.151973514
				9575 371110.947763979
				9576 371121.743554443
				9577 371132.539344908
				9578 371143.335135372
				9579 371154.130925837
				9580 371164.926716302
				9581 371175.722506766
				9582 371186.518297231
				9583 371197.314087695
				9584 371208.10987816
				9585 371218.905668625
				9586 371229.701459089
				9587 371240.497249554
				9588 371251.293040018
				9589 371262.088830483
				9590 371272.884620947
				9591 371283.680411412
				9592 371294.476201877
				9593 371305.271992341
				9594 371316.067782806
				9595 371326.86357327
				9596 371337.659363735
				9597 371348.4551542
				9598 371359.250944664
				9599 371370.046735129
				9600 371380.842525593
				9601 371391.638316058
				9602 371402.434106522
				9603 371413.229896987
				9604 371424.025687452
				9605 371434.821477916
				9606 371445.617268381
				9607 371456.413058845
				9608 371467.20884931
				9609 371478.004639774
				9610 371488.800430239
				9611 371499.596220704
				9612 371510.392011168
				9613 371521.187801633
				9614 371531.983592097
				9615 371542.779382562
				9616 371553.575173027
				9617 371564.370963491
				9618 371575.166753956
				9619 371585.96254442
				9620 371596.758334885
				9621 371607.554125349
				9622 371618.349915814
				9623 371629.145706279
				9624 371639.941496743
				9625 371650.737287208
				9626 371661.533077672
				9627 371672.328868137
				9628 371683.124658601
				9629 371693.920449066
				9630 371704.716239531
				9631 371715.512029995
				9632 371726.30782046
				9633 371737.103610924
				9634 371747.899401389
				9635 371758.695191854
				9636 371769.490982318
				9637 371780.286772783
				9638 371791.082563247
				9639 371801.878353712
				9640 371812.674144176
				9641 371823.469934641
				9642 371834.265725106
				9643 371845.06151557
				9644 371855.857306035
				9645 371866.653096499
				9646 371877.448886964
				9647 371888.244677429
				9648 371899.040467893
				9649 371909.836258358
				9650 371920.632048822
				9651 371931.427839287
				9652 371942.223629751
				9653 371953.019420216
				9654 371963.815210681
				9655 371974.611001145
				9656 371985.40679161
				9657 371996.202582074
				9658 372006.998372539
				9659 372017.794163003
				9660 372028.589953468
				9661 372039.385743933
				9662 372050.181534397
				9663 372060.977324862
				9664 372071.773115326
				9665 372082.568905791
				9666 372093.364696256
				9667 372104.16048672
				9668 372114.956277185
				9669 372125.752067649
				9670 372136.547858114
				9671 372147.343648578
				9672 372158.139439043
				9673 372168.935229508
				9674 372179.731019972
				9675 372190.526810437
				9676 372201.322600901
				9677 372212.118391366
				9678 372222.91418183
				9679 372233.709972295
				9680 372244.50576276
				9681 372255.301553224
				9682 372266.097343689
				9683 372276.893134153
				9684 372287.688924618
				9685 372298.484715083
				9686 372309.280505547
				9687 372320.076296012
				9688 372330.872086476
				9689 372341.667876941
				9690 372352.463667405
				9691 372363.25945787
				9692 372374.055248335
				9693 372384.851038799
				9694 372395.646829264
				9695 372406.442619728
				9696 372417.238410193
				9697 372428.034200658
				9698 372438.829991122
				9699 372449.625781587
				9700 372460.421572051
				9701 372471.217362516
				9702 372482.01315298
				9703 372492.808943445
				9704 372503.60473391
				9705 372514.400524374
				9706 372525.196314839
				9707 372535.992105303
				9708 372546.787895768
				9709 372557.583686232
				9710 372568.379476697
				9711 372579.175267162
				9712 372589.971057626
				9713 372600.766848091
				9714 372611.562638555
				9715 372622.35842902
				9716 372633.154219485
				9717 372643.950009949
				9718 372654.745800414
				9719 372665.541590878
				9720 372676.337381343
				9721 372687.133171807
				9722 372697.928962272
				9723 372708.724752737
				9724 372719.520543201
				9725 372730.316333666
				9726 372741.11212413
				9727 372751.907914595
				9728 372762.703705059
				9729 372773.499495524
				9730 372784.295285989
				9731 372795.091076453
				9732 372805.886866918
				9733 372816.682657382
				9734 372827.478447847
				9735 372838.274238312
				9736 372849.070028776
				9737 372859.865819241
				9738 372870.661609705
				9739 372881.45740017
				9740 372892.253190634
				9741 372903.048981099
				9742 372913.844771564
				9743 372924.640562028
				9744 372935.436352493
				9745 372946.232142957
				9746 372957.027933422
				9747 372967.823723887
				9748 372978.619514351
				9749 372989.415304816
				9750 373000.21109528
				9751 373011.006885745
				9752 373021.802676209
				9753 373032.598466674
				9754 373043.394257139
				9755 373054.190047603
				9756 373064.985838068
				9757 373075.781628532
				9758 373086.577418997
				9759 373097.373209461
				9760 373108.168999926
				9761 373118.964790391
				9762 373129.760580855
				9763 373140.55637132
				9764 373151.352161784
				9765 373162.147952249
				9766 373172.943742713
				9767 373183.739533178
				9768 373194.535323643
				9769 373205.331114107
				9770 373216.126904572
				9771 373226.922695036
				9772 373237.718485501
				9773 373248.514275966
				9774 373259.31006643
				9775 373270.105856895
				9776 373280.901647359
				9777 373291.697437824
				9778 373302.493228288
				9779 373313.289018753
				9780 373324.084809218
				9781 373334.880599682
				9782 373345.676390147
				9783 373356.472180611
				9784 373367.267971076
				9785 373378.063761541
				9786 373388.859552005
				9787 373399.65534247
				9788 373410.451132934
				9789 373421.246923399
				9790 373432.042713863
				9791 373442.838504328
				9792 373453.634294793
				9793 373464.430085257
				9794 373475.225875722
				9795 373486.021666186
				9796 373496.817456651
				9797 373507.613247115
				9798 373518.40903758
				9799 373529.204828045
				9800 373540.000618509
				9801 373550.796408974
				9802 373561.592199438
				9803 373572.387989903
				9804 373583.183780368
				9805 373593.979570832
				9806 373604.775361297
				9807 373615.571151761
				9808 373626.366942226
				9809 373637.16273269
				9810 373647.958523155
				9811 373658.75431362
				9812 373669.550104084
				9813 373680.345894549
				9814 373691.141685013
				9815 373701.937475478
				9816 373712.733265942
				9817 373723.529056407
				9818 373734.324846872
				9819 373745.120637336
				9820 373755.916427801
				9821 373766.712218265
				9822 373777.50800873
				9823 373788.303799195
				9824 373799.099589659
				9825 373809.895380124
				9826 373820.691170588
				9827 373831.486961053
				9828 373842.282751517
				9829 373853.078541982
				9830 373863.874332447
				9831 373874.670122911
				9832 373885.465913376
				9833 373896.26170384
				9834 373907.057494305
				9835 373917.85328477
				9836 373928.649075234
				9837 373939.444865699
				9838 373950.240656163
				9839 373961.036446628
				9840 373971.832237092
				9841 373982.628027557
				9842 373993.423818022
				9843 374004.219608486
				9844 374015.015398951
				9845 374025.811189415
				9846 374036.60697988
				9847 374047.402770344
				9848 374058.198560809
				9849 374068.994351274
				9850 374079.790141738
				9851 374090.585932203
				9852 374101.381722667
				9853 374112.177513132
				9854 374122.973303596
				9855 374133.769094061
				9856 374144.564884526
				9857 374155.36067499
				9858 374166.156465455
				9859 374176.952255919
				9860 374187.748046384
				9861 374198.543836849
				9862 374209.339627313
				9863 374220.135417778
				9864 374230.931208242
				9865 374241.726998707
				9866 374252.522789171
				9867 374263.318579636
				9868 374274.114370101
				9869 374284.910160565
				9870 374295.70595103
				9871 374306.501741494
				9872 374317.297531959
				9873 374328.093322424
				9874 374338.889112888
				9875 374349.684903353
				9876 374360.480693817
				9877 374371.276484282
				9878 374382.072274746
				9879 374392.868065211
				9880 374403.663855676
				9881 374414.45964614
				9882 374425.255436605
				9883 374436.051227069
				9884 374446.847017534
				9885 374457.642807998
				9886 374468.438598463
				9887 374479.234388928
				9888 374490.030179392
				9889 374500.825969857
				9890 374511.621760321
				9891 374522.417550786
				9892 374533.213341251
				9893 374544.009131715
				9894 374554.80492218
				9895 374565.600712644
				9896 374576.396503109
				9897 374587.192293573
				9898 374597.988084038
				9899 374608.783874503
				9900 374619.579664967
				9901 374630.375455432
				9902 374641.171245896
				9903 374651.967036361
				9904 374662.762826825
				9905 374673.55861729
				9906 374684.354407755
				9907 374695.150198219
				9908 374705.945988684
				9909 374716.741779148
				9910 374727.537569613
				9911 374738.333360078
				9912 374749.129150542
				9913 374759.924941007
				9914 374770.720731471
				9915 374781.516521936
				9916 374792.3123124
				9917 374803.108102865
				9918 374813.90389333
				9919 374824.699683794
				9920 374835.495474259
				9921 374846.291264723
				9922 374857.087055188
				9923 374867.882845653
				9924 374878.678636117
				9925 374889.474426582
				9926 374900.270217046
				9927 374911.066007511
				9928 374921.861797975
				9929 374932.65758844
				9930 374943.453378905
				9931 374954.249169369
				9932 374965.044959834
				9933 374975.840750298
				9934 374986.636540763
				9935 374997.432331227
				9936 375008.228121692
				9937 375019.023912157
				9938 375029.819702621
				9939 375040.615493086
				9940 375051.41128355
				9941 375062.207074015
				9942 375073.00286448
				9943 375083.798654944
				9944 375094.594445409
				9945 375105.390235873
				9946 375116.186026338
				9947 375126.981816802
				9948 375137.777607267
				9949 375148.573397732
				9950 375159.369188196
				9951 375170.164978661
				9952 375180.960769125
				9953 375191.75655959
				9954 375202.552350054
				9955 375213.348140519
				9956 375224.143930984
				9957 375234.939721448
				9958 375245.735511913
				9959 375256.531302377
				9960 375267.327092842
				9961 375278.122883307
				9962 375288.918673771
				9963 375299.714464236
				9964 375310.5102547
				9965 375321.306045165
				9966 375332.101835629
				9967 375342.897626094
				9968 375353.693416559
				9969 375364.489207023
				9970 375375.284997488
				9971 375386.080787952
				9972 375396.876578417
				9973 375407.672368882
				9974 375418.468159346
				9975 375429.263949811
				9976 375440.059740275
				9977 375450.85553074
				9978 375461.651321204
				9979 375472.447111669
				9980 375483.242902134
				9981 375494.038692598
				9982 375504.834483063
				9983 375515.630273527
				9984 375526.426063992
				9985 375537.221854456
				9986 375548.017644921
				9987 375558.813435386
				9988 375569.60922585
				9989 375580.405016315
				9990 375591.200806779
				9991 375601.996597244
				9992 375612.792387709
				9993 375623.588178173
				9994 375634.383968638
				9995 375645.179759102
				9996 375655.975549567
				9997 375666.771340031
				9998 375677.567130496
				9999 375688.362920961
				10000 375699.158711425
				10001 375709.95450189
				10002 375720.750292354
				10003 375731.546082819
				10004 375742.341873283
				10005 375753.137663748
				10006 375763.933454213
				10007 375774.729244677
				10008 375785.525035142
				10009 375796.320825606
				10010 375807.116616071
				10011 375817.912406536
				10012 375828.708197
				10013 375839.503987465
				10014 375850.299777929
				10015 375861.095568394
				10016 375871.891358858
				10017 375882.687149323
				10018 375893.482939788
				10019 375904.278730252
				10020 375915.074520717
				10021 375925.870311181
				10022 375936.666101646
				10023 375947.461892111
				10024 375958.257682575
				10025 375969.05347304
				10026 375979.849263504
				10027 375990.645053969
				10028 376001.440844433
				10029 376012.236634898
				10030 376023.032425363
				10031 376033.828215827
				10032 376044.624006292
				10033 376055.419796756
				10034 376066.215587221
				10035 376077.011377685
				10036 376087.80716815
				10037 376098.602958615
				10038 376109.398749079
				10039 376120.194539544
				10040 376130.990330008
				10041 376141.786120473
				10042 376152.581910938
				10043 376163.377701402
				10044 376174.173491867
				10045 376184.969282331
				10046 376195.765072796
				10047 376206.56086326
				10048 376217.356653725
				10049 376228.15244419
				10050 376238.948234654
				10051 376249.744025119
				10052 376260.539815583
				10053 376271.335606048
				10054 376282.131396513
				10055 376292.927186977
				10056 376303.722977442
				10057 376314.518767906
				10058 376325.314558371
				10059 376336.110348835
				10060 376346.9061393
				10061 376357.701929765
				10062 376368.497720229
				10063 376379.293510694
				10064 376390.089301158
				10065 376400.885091623
				10066 376411.680882087
				10067 376422.476672552
				10068 376433.272463017
				10069 376444.068253481
				10070 376454.864043946
				10071 376465.65983441
				10072 376476.455624875
				10073 376487.25141534
				10074 376498.047205804
				10075 376508.842996269
				10076 376519.638786733
				10077 376530.434577198
				10078 376541.230367662
				10079 376552.026158127
				10080 376562.821948592
				10081 376573.617739056
				10082 376584.413529521
				10083 376595.209319985
				10084 376606.00511045
				10085 376616.800900914
				10086 376627.596691379
				10087 376638.392481844
				10088 376649.188272308
				10089 376659.984062773
				10090 376670.779853237
				10091 376681.575643702
				10092 376692.371434167
				10093 376703.167224631
				10094 376713.963015096
				10095 376724.75880556
				10096 376735.554596025
				10097 376746.350386489
				10098 376757.146176954
				10099 376767.941967419
				10100 376778.737757883
				10101 376789.533548348
				10102 376800.329338812
				10103 376811.125129277
				10104 376821.920919742
				10105 376832.716710206
				10106 376843.512500671
				10107 376854.308291135
				10108 376865.1040816
				10109 376875.899872064
				10110 376886.695662529
				10111 376897.491452994
				10112 376908.287243458
				10113 376919.083033923
				10114 376929.878824387
				10115 376940.674614852
				10116 376951.470405316
				10117 376962.266195781
				10118 376973.061986246
				10119 376983.85777671
				10120 376994.653567175
				10121 377005.449357639
				10122 377016.245148104
				10123 377027.040938568
				10124 377037.836729033
				10125 377048.632519498
				10126 377059.428309962
				10127 377070.224100427
				10128 377081.019890891
				10129 377091.815681356
				10130 377102.611471821
				10131 377113.407262285
				10132 377124.20305275
				10133 377134.998843214
				10134 377145.794633679
				10135 377156.590424143
				10136 377167.386214608
				10137 377178.182005073
				10138 377188.977795537
				10139 377199.773586002
				10140 377210.569376466
				10141 377221.365166931
				10142 377232.160957396
				10143 377242.95674786
				10144 377253.752538325
				10145 377264.548328789
				10146 377275.344119254
				10147 377286.139909718
				10148 377296.935700183
				10149 377307.731490648
				10150 377318.527281112
				10151 377329.323071577
				10152 377340.118862041
				10153 377350.914652506
				10154 377361.71044297
				10155 377372.506233435
				10156 377383.3020239
				10157 377394.097814364
				10158 377404.893604829
				10159 377415.689395293
				10160 377426.485185758
				10161 377437.280976223
				10162 377448.076766687
				10163 377458.872557152
				10164 377469.668347616
				10165 377480.464138081
				10166 377491.259928545
				10167 377502.05571901
				10168 377512.851509475
				10169 377523.647299939
				10170 377534.443090404
				10171 377545.238880868
				10172 377556.034671333
				10173 377566.830461797
				10174 377577.626252262
				10175 377588.422042727
				10176 377599.217833191
				10177 377610.013623656
				10178 377620.80941412
				10179 377631.605204585
				10180 377642.40099505
				10181 377653.196785514
				10182 377663.992575979
				10183 377674.788366443
				10184 377685.584156908
				10185 377696.379947372
				10186 377707.175737837
				10187 377717.971528302
				10188 377728.767318766
				10189 377739.563109231
				10190 377750.358899695
				10191 377761.15469016
				10192 377771.950480625
				10193 377782.746271089
				10194 377793.542061554
				10195 377804.337852018
				10196 377815.133642483
				10197 377825.929432947
				10198 377836.725223412
				10199 377847.521013877
				10200 377858.316804341
				10201 377869.112594806
				10202 377879.90838527
				10203 377890.704175735
				10204 377901.499966199
				10205 377912.295756664
				10206 377923.091547129
				10207 377933.887337593
				10208 377944.683128058
				10209 377955.478918522
				10210 377966.274708987
				10211 377977.070499451
				10212 377987.866289916
				10213 377998.662080381
				10214 378009.457870845
				10215 378020.25366131
				10216 378031.049451774
				10217 378041.845242239
				10218 378052.641032704
				10219 378063.436823168
				10220 378074.232613633
				10221 378085.028404097
				10222 378095.824194562
				10223 378106.619985026
				10224 378117.415775491
				10225 378128.211565956
				10226 378139.00735642
				10227 378149.803146885
				10228 378160.598937349
				10229 378171.394727814
				10230 378182.190518279
				10231 378192.986308743
				10232 378203.782099208
				10233 378214.577889672
				10234 378225.373680137
				10235 378236.169470601
				10236 378246.965261066
				10237 378257.761051531
				10238 378268.556841995
				10239 378279.35263246
				10240 378290.148422924
				10241 378300.944213389
				10242 378311.740003853
				10243 378322.535794318
				10244 378333.331584783
				10245 378344.127375247
				10246 378354.923165712
				10247 378365.718956176
				10248 378376.514746641
				10249 378387.310537106
				10250 378398.10632757
				10251 378408.902118035
				10252 378419.697908499
				10253 378430.493698964
				10254 378441.289489428
				10255 378452.085279893
				10256 378462.881070358
				10257 378473.676860822
				10258 378484.472651287
				10259 378495.268441751
				10260 378506.064232216
				10261 378516.86002268
				10262 378527.655813145
				10263 378538.45160361
				10264 378549.247394074
				10265 378560.043184539
				10266 378570.838975003
				10267 378581.634765468
				10268 378592.430555933
				10269 378603.226346397
				10270 378614.022136862
				10271 378624.817927326
				10272 378635.613717791
				10273 378646.409508255
				10274 378657.20529872
				10275 378668.001089185
				10276 378678.796879649
				10277 378689.592670114
				10278 378700.388460578
				10279 378711.184251043
				10280 378721.980041508
				10281 378732.775831972
				10282 378743.571622437
				10283 378754.367412901
				10284 378765.163203366
				10285 378775.95899383
				10286 378786.754784295
				10287 378797.55057476
				10288 378808.346365224
				10289 378819.142155689
				10290 378829.937946153
				10291 378840.733736618
				10292 378851.529527082
				10293 378862.325317547
				10294 378873.121108012
				10295 378883.916898476
				10296 378894.712688941
				10297 378905.508479405
				10298 378916.30426987
				10299 378927.100060335
				10300 378937.895850799
				10301 378948.691641264
				10302 378959.487431728
				10303 378970.283222193
				10304 378981.079012657
				10305 378991.874803122
				10306 379002.670593587
				10307 379013.466384051
				10308 379024.262174516
				10309 379035.05796498
				10310 379045.853755445
				10311 379056.649545909
				10312 379067.445336374
				10313 379078.241126839
				10314 379089.036917303
				10315 379099.832707768
				10316 379110.628498232
				10317 379121.424288697
				10318 379132.220079162
				10319 379143.015869626
				10320 379153.811660091
				10321 379164.607450555
				10322 379175.40324102
				10323 379186.199031484
				10324 379196.994821949
				10325 379207.790612414
				10326 379218.586402878
				10327 379229.382193343
				10328 379240.177983807
				10329 379250.973774272
				10330 379261.769564737
				10331 379272.565355201
				10332 379283.361145666
				10333 379294.15693613
				10334 379304.952726595
				10335 379315.748517059
				10336 379326.544307524
				10337 379337.340097989
				10338 379348.135888453
				10339 379358.931678918
				10340 379369.727469382
				10341 379380.523259847
				10342 379391.319050311
				10343 379402.114840776
				10344 379412.910631241
				10345 379423.706421705
				10346 379434.50221217
				10347 379445.298002634
				10348 379456.093793099
				10349 379466.889583564
				10350 379477.685374028
				10351 379488.481164493
				10352 379499.276954957
				10353 379510.072745422
				10354 379520.868535886
				10355 379531.664326351
				10356 379542.460116816
				10357 379553.25590728
				10358 379564.051697745
				10359 379574.847488209
				10360 379585.643278674
				10361 379596.439069138
				10362 379607.234859603
				10363 379618.030650068
				10364 379628.826440532
				10365 379639.622230997
				10366 379650.418021461
				10367 379661.213811926
				10368 379672.009602391
				10369 379682.805392855
				10370 379693.60118332
				10371 379704.396973784
				10372 379715.192764249
				10373 379725.988554713
				10374 379736.784345178
				10375 379747.580135643
				10376 379758.375926107
				10377 379769.171716572
				10378 379779.967507036
				10379 379790.763297501
				10380 379801.559087966
				10381 379812.35487843
				10382 379823.150668895
				10383 379833.946459359
				10384 379844.742249824
				10385 379855.538040288
				10386 379866.333830753
				10387 379877.129621218
				10388 379887.925411682
				10389 379898.721202147
				10390 379909.516992611
				10391 379920.312783076
				10392 379931.10857354
				10393 379941.904364005
				10394 379952.70015447
				10395 379963.495944934
				10396 379974.291735399
				10397 379985.087525863
				10398 379995.883316328
				10399 380006.679106793
				10400 380017.474897257
				10401 380028.270687722
				10402 380039.066478186
				10403 380049.862268651
				10404 380060.658059115
				10405 380071.45384958
				10406 380082.249640045
				10407 380093.045430509
				10408 380103.841220974
				10409 380114.637011438
				10410 380125.432801903
				10411 380136.228592367
				10412 380147.024382832
				10413 380157.820173297
				10414 380168.615963761
				10415 380179.411754226
				10416 380190.20754469
				10417 380201.003335155
				10418 380211.79912562
				10419 380222.594916084
				10420 380233.390706549
				10421 380244.186497013
				10422 380254.982287478
				10423 380265.778077942
				10424 380276.573868407
				10425 380287.369658872
				10426 380298.165449336
				10427 380308.961239801
				10428 380319.757030265
				10429 380330.55282073
				10430 380341.348611195
				10431 380352.144401659
				10432 380362.940192124
				10433 380373.735982588
				10434 380384.531773053
				10435 380395.327563517
				10436 380406.123353982
				10437 380416.919144447
				10438 380427.714934911
				10439 380438.510725376
				10440 380449.30651584
				10441 380460.102306305
				10442 380470.898096769
				10443 380481.693887234
				10444 380492.489677699
				10445 380503.285468163
				10446 380514.081258628
				10447 380524.877049092
				10448 380535.672839557
				10449 380546.468630021
				10450 380557.264420486
				10451 380568.060210951
				10452 380578.856001415
				10453 380589.65179188
				10454 380600.447582344
				10455 380611.243372809
				10456 380622.039163274
				10457 380632.834953738
				10458 380643.630744203
				10459 380654.426534667
				10460 380665.222325132
				10461 380676.018115596
				10462 380686.813906061
				10463 380697.609696526
				10464 380708.40548699
				10465 380719.201277455
				10466 380729.997067919
				10467 380740.792858384
				10468 380751.588648849
				10469 380762.384439313
				10470 380773.180229778
				10471 380783.976020242
				10472 380794.771810707
				10473 380805.567601171
				10474 380816.363391636
				10475 380827.159182101
				10476 380837.954972565
				10477 380848.75076303
				10478 380859.546553494
				10479 380870.342343959
				10480 380881.138134423
				10481 380891.933924888
				10482 380902.729715353
				10483 380913.525505817
				10484 380924.321296282
				10485 380935.117086746
				10486 380945.912877211
				10487 380956.708667676
				10488 380967.50445814
				10489 380978.300248605
				10490 380989.096039069
				10491 380999.891829534
				10492 381010.687619998
				10493 381021.483410463
				10494 381032.279200928
				10495 381043.074991392
				10496 381053.870781857
				10497 381064.666572321
				10498 381075.462362786
				10499 381086.25815325
				10500 381097.053943715
				10501 381107.84973418
				10502 381118.645524644
				10503 381129.441315109
				10504 381140.237105573
				10505 381151.032896038
				10506 381161.828686503
				10507 381172.624476967
				10508 381183.420267432
				10509 381194.216057896
				10510 381205.011848361
				10511 381215.807638825
				10512 381226.60342929
				10513 381237.399219755
				10514 381248.195010219
				10515 381258.990800684
				10516 381269.786591148
				10517 381280.582381613
				10518 381291.378172078
				10519 381302.173962542
				10520 381312.969753007
				10521 381323.765543471
				10522 381334.561333936
				10523 381345.3571244
				10524 381356.152914865
				10525 381366.94870533
				10526 381377.744495794
				10527 381388.540286259
				10528 381399.336076723
				10529 381410.131867188
				10530 381420.927657652
				10531 381431.723448117
				10532 381442.519238582
				10533 381453.315029046
				10534 381464.110819511
				10535 381474.906609975
				10536 381485.70240044
				10537 381496.498190905
				10538 381507.293981369
				10539 381518.089771834
				10540 381528.885562298
				10541 381539.681352763
				10542 381550.477143227
				10543 381561.272933692
				10544 381572.068724157
				10545 381582.864514621
				10546 381593.660305086
				10547 381604.45609555
				10548 381615.251886015
				10549 381626.04767648
				10550 381636.843466944
				10551 381647.639257409
				10552 381658.435047873
				10553 381669.230838338
				10554 381680.026628802
				10555 381690.822419267
				10556 381701.618209732
				10557 381712.414000196
				10558 381723.209790661
				10559 381734.005581125
				10560 381744.80137159
				10561 381755.597162054
				10562 381766.392952519
				10563 381777.188742984
				10564 381787.984533448
				10565 381798.780323913
				10566 381809.576114377
				10567 381820.371904842
				10568 381831.167695306
				10569 381841.963485771
				10570 381852.759276236
				10571 381863.5550667
				10572 381874.350857165
				10573 381885.146647629
				10574 381895.942438094
				10575 381906.738228559
				10576 381917.534019023
				10577 381928.329809488
				10578 381939.125599952
				10579 381949.921390417
				10580 381960.717180881
				10581 381971.512971346
				10582 381982.308761811
				10583 381993.104552275
				10584 382003.90034274
				10585 382014.696133204
				10586 382025.491923669
				10587 382036.287714134
				10588 382047.083504598
				10589 382057.879295063
				10590 382068.675085527
				10591 382079.470875992
				10592 382090.266666456
				10593 382101.062456921
				10594 382111.858247386
				10595 382122.65403785
				10596 382133.449828315
				10597 382144.245618779
				10598 382155.041409244
				10599 382165.837199708
				10600 382176.632990173
				10601 382187.428780638
				10602 382198.224571102
				10603 382209.020361567
				10604 382219.816152031
				10605 382230.611942496
				10606 382241.407732961
				10607 382252.203523425
				10608 382262.99931389
				10609 382273.795104354
				10610 382284.590894819
				10611 382295.386685283
				10612 382306.182475748
				10613 382316.978266213
				10614 382327.774056677
				10615 382338.569847142
				10616 382349.365637606
				10617 382360.161428071
				10618 382370.957218535
				10619 382381.753009
				10620 382392.548799465
				10621 382403.344589929
				10622 382414.140380394
				10623 382424.936170858
				10624 382435.731961323
				10625 382446.527751788
				10626 382457.323542252
				10627 382468.119332717
				10628 382478.915123181
				10629 382489.710913646
				10630 382500.50670411
				10631 382511.302494575
				10632 382522.09828504
				10633 382532.894075504
				10634 382543.689865969
				10635 382554.485656433
				10636 382565.281446898
				10637 382576.077237363
				10638 382586.873027827
				10639 382597.668818292
				10640 382608.464608756
				10641 382619.260399221
				10642 382630.056189685
				10643 382640.85198015
				10644 382651.647770615
				10645 382662.443561079
				10646 382673.239351544
				10647 382684.035142008
				10648 382694.830932473
				10649 382705.626722937
				10650 382716.422513402
				10651 382727.218303867
				10652 382738.014094331
				10653 382748.809884796
				10654 382759.60567526
				10655 382770.401465725
				10656 382781.19725619
				10657 382791.993046654
				10658 382802.788837119
				10659 382813.584627583
				10660 382824.380418048
				10661 382835.176208512
				10662 382845.971998977
				10663 382856.767789442
				10664 382867.563579906
				10665 382878.359370371
				10666 382889.155160835
				10667 382899.9509513
				10668 382910.746741764
				10669 382921.542532229
				10670 382932.338322694
				10671 382943.134113158
				10672 382953.929903623
				10673 382964.725694087
				10674 382975.521484552
				10675 382986.317275017
				10676 382997.113065481
				10677 383007.908855946
				10678 383018.70464641
				10679 383029.500436875
				10680 383040.296227339
				10681 383051.092017804
				10682 383061.887808269
				10683 383072.683598733
				10684 383083.479389198
				10685 383094.275179662
				10686 383105.070970127
				10687 383115.866760592
				10688 383126.662551056
				10689 383137.458341521
				10690 383148.254131985
				10691 383159.04992245
				10692 383169.845712914
				10693 383180.641503379
				10694 383191.437293844
				10695 383202.233084308
				10696 383213.028874773
				10697 383223.824665237
				10698 383234.620455702
				10699 383245.416246166
				10700 383256.212036631
				10701 383267.007827096
				10702 383277.80361756
				10703 383288.599408025
				10704 383299.395198489
				10705 383310.190988954
				10706 383320.986779419
				10707 383331.782569883
				10708 383342.578360348
				10709 383353.374150812
				10710 383364.169941277
				10711 383374.965731741
				10712 383385.761522206
				10713 383396.557312671
				10714 383407.353103135
				10715 383418.1488936
				10716 383428.944684064
				10717 383439.740474529
				10718 383450.536264993
				10719 383461.332055458
				10720 383472.127845923
				10721 383482.923636387
				10722 383493.719426852
				10723 383504.515217316
				10724 383515.311007781
				10725 383526.106798246
				10726 383536.90258871
				10727 383547.698379175
				10728 383558.494169639
				10729 383569.289960104
				10730 383580.085750568
				10731 383590.881541033
				10732 383601.677331498
				10733 383612.473121962
				10734 383623.268912427
				10735 383634.064702891
				10736 383644.860493356
				10737 383655.656283821
				10738 383666.452074285
				10739 383677.24786475
				10740 383688.043655214
				10741 383698.839445679
				10742 383709.635236143
				10743 383720.431026608
				10744 383731.226817073
				10745 383742.022607537
				10746 383752.818398002
				10747 383763.614188466
				10748 383774.409978931
				10749 383785.205769395
				10750 383796.00155986
				10751 383806.797350325
				10752 383817.593140789
				10753 383828.388931254
				10754 383839.184721718
				10755 383849.980512183
				10756 383860.776302648
				10757 383871.572093112
				10758 383882.367883577
				10759 383893.163674041
				10760 383903.959464506
				10761 383914.75525497
				10762 383925.551045435
				10763 383936.3468359
				10764 383947.142626364
				10765 383957.938416829
				10766 383968.734207293
				10767 383979.529997758
				10768 383990.325788222
				10769 384001.121578687
				10770 384011.917369152
				10771 384022.713159616
				10772 384033.508950081
				10773 384044.304740545
				10774 384055.10053101
				10775 384065.896321475
				10776 384076.692111939
				10777 384087.487902404
				10778 384098.283692868
				10779 384109.079483333
				10780 384119.875273797
				10781 384130.671064262
				10782 384141.466854727
				10783 384152.262645191
				10784 384163.058435656
				10785 384173.85422612
				10786 384184.650016585
				10787 384195.44580705
				10788 384206.241597514
				10789 384217.037387979
				10790 384227.833178443
				10791 384238.628968908
				10792 384249.424759372
				10793 384260.220549837
				10794 384271.016340302
				10795 384281.812130766
				10796 384292.607921231
				10797 384303.403711695
				10798 384314.19950216
				10799 384324.995292624
				10800 384335.791083089
				10801 384346.586873554
				10802 384357.382664018
				10803 384368.178454483
				10804 384378.974244947
				10805 384389.770035412
				10806 384400.565825876
				10807 384411.361616341
				10808 384422.157406806
				10809 384432.95319727
				10810 384443.748987735
				10811 384454.544778199
				10812 384465.340568664
				10813 384476.136359129
				10814 384486.932149593
				10815 384497.727940058
				10816 384508.523730522
				10817 384519.319520987
				10818 384530.115311451
				10819 384540.911101916
				10820 384551.706892381
				10821 384562.502682845
				10822 384573.29847331
				10823 384584.094263774
				10824 384594.890054239
				10825 384605.685844704
				10826 384616.481635168
				10827 384627.277425633
				10828 384638.073216097
				10829 384648.869006562
				10830 384659.664797026
				10831 384670.460587491
				10832 384681.256377956
				10833 384692.05216842
				10834 384702.847958885
				10835 384713.643749349
				10836 384724.439539814
				10837 384735.235330278
				10838 384746.031120743
				10839 384756.826911208
				10840 384767.622701672
				10841 384778.418492137
				10842 384789.214282601
				10843 384800.010073066
				10844 384810.805863531
				10845 384821.601653995
				10846 384832.39744446
				10847 384843.193234924
				10848 384853.989025389
				10849 384864.784815853
				10850 384875.580606318
				10851 384886.376396783
				10852 384897.172187247
				10853 384907.967977712
				10854 384918.763768176
				10855 384929.559558641
				10856 384940.355349105
				10857 384951.15113957
				10858 384961.946930035
				10859 384972.742720499
				10860 384983.538510964
				10861 384994.334301428
				10862 385005.130091893
				10863 385015.925882358
				10864 385026.721672822
				10865 385037.517463287
				10866 385048.313253751
				10867 385059.109044216
				10868 385069.90483468
				10869 385080.700625145
				10870 385091.49641561
				10871 385102.292206074
				10872 385113.087996539
				10873 385123.883787003
				10874 385134.679577468
				10875 385145.475367933
				10876 385156.271158397
				10877 385167.066948862
				10878 385177.862739326
				10879 385188.658529791
				10880 385199.454320255
				10881 385210.25011072
				10882 385221.045901185
				10883 385231.841691649
				10884 385242.637482114
				10885 385253.433272578
				10886 385264.229063043
				10887 385275.024853507
				10888 385285.820643972
				10889 385296.616434437
				10890 385307.412224901
				10891 385318.208015366
				10892 385329.00380583
				10893 385339.799596295
				10894 385350.595386759
				10895 385361.391177224
				10896 385372.186967689
				10897 385382.982758153
				10898 385393.778548618
				10899 385404.574339082
				10900 385415.370129547
				10901 385426.165920012
				10902 385436.961710476
				10903 385447.757500941
				10904 385458.553291405
				10905 385469.34908187
				10906 385480.144872334
				10907 385490.940662799
				10908 385501.736453264
				10909 385512.532243728
				10910 385523.328034193
				10911 385534.123824657
				10912 385544.919615122
				10913 385555.715405587
				10914 385566.511196051
				10915 385577.306986516
				10916 385588.10277698
				10917 385598.898567445
				10918 385609.694357909
				10919 385620.490148374
				10920 385631.285938839
				10921 385642.081729303
				10922 385652.877519768
				10923 385663.673310232
				10924 385674.469100697
				10925 385685.264891161
				10926 385696.060681626
				10927 385706.856472091
				10928 385717.652262555
				10929 385728.44805302
				10930 385739.243843484
				10931 385750.039633949
				10932 385760.835424414
				10933 385771.631214878
				10934 385782.427005343
				10935 385793.222795807
				10936 385804.018586272
				10937 385814.814376736
				10938 385825.610167201
				10939 385836.405957666
				10940 385847.20174813
				10941 385857.997538595
				10942 385868.793329059
				10943 385879.589119524
				10944 385890.384909988
				10945 385901.180700453
				10946 385911.976490918
				10947 385922.772281382
				10948 385933.568071847
				10949 385944.363862311
				10950 385955.159652776
				10951 385965.955443241
				10952 385976.751233705
				10953 385987.54702417
				10954 385998.342814634
				10955 386009.138605099
				10956 386019.934395563
				10957 386030.730186028
				10958 386041.525976493
				10959 386052.321766957
				10960 386063.117557422
				10961 386073.913347886
				10962 386084.709138351
				10963 386095.504928816
				10964 386106.30071928
				10965 386117.096509745
				10966 386127.892300209
				10967 386138.688090674
				10968 386149.483881138
				10969 386160.279671603
				10970 386171.075462068
				10971 386181.871252532
				10972 386192.667042997
				10973 386203.462833461
				10974 386214.258623926
				10975 386225.05441439
				10976 386235.850204855
				10977 386246.64599532
				10978 386257.441785784
				10979 386268.237576249
				10980 386279.033366713
				10981 386289.829157178
				10982 386300.624947643
				10983 386311.420738107
				10984 386322.216528572
				10985 386333.012319036
				10986 386343.808109501
				10987 386354.603899965
				10988 386365.39969043
				10989 386376.195480895
				10990 386386.991271359
				10991 386397.787061824
				10992 386408.582852288
				10993 386419.378642753
				10994 386430.174433218
				10995 386440.970223682
				10996 386451.766014147
				10997 386462.561804611
				10998 386473.357595076
				10999 386484.15338554
				11000 386494.949176005
				11001 386505.74496647
				11002 386516.540756934
				11003 386527.336547399
				11004 386538.132337863
				11005 386548.928128328
				11006 386559.723918792
				11007 386570.519709257
				11008 386581.315499722
				11009 386592.111290186
				11010 386602.907080651
				11011 386613.702871115
				11012 386624.49866158
				11013 386635.294452045
				11014 386646.090242509
				11015 386656.886032974
				11016 386667.681823438
				11017 386678.477613903
				11018 386689.273404367
				11019 386700.069194832
				11020 386710.864985297
				11021 386721.660775761
				11022 386732.456566226
				11023 386743.25235669
				11024 386754.048147155
				11025 386764.843937619
				11026 386775.639728084
				11027 386786.435518549
				11028 386797.231309013
				11029 386808.027099478
				11030 386818.822889942
				11031 386829.618680407
				11032 386840.414470872
				11033 386851.210261336
				11034 386862.006051801
				11035 386872.801842265
				11036 386883.59763273
				11037 386894.393423194
				11038 386905.189213659
				11039 386915.985004124
				11040 386926.780794588
				11041 386937.576585053
				11042 386948.372375517
				11043 386959.168165982
				11044 386969.963956447
				11045 386980.759746911
				11046 386991.555537376
				11047 387002.35132784
				11048 387013.147118305
				11049 387023.942908769
				11050 387034.738699234
				11051 387045.534489699
				11052 387056.330280163
				11053 387067.126070628
				11054 387077.921861092
				11055 387088.717651557
				11056 387099.513442021
				11057 387110.309232486
				11058 387121.105022951
				11059 387131.900813415
				11060 387142.69660388
				11061 387153.492394344
				11062 387164.288184809
				11063 387175.083975274
				11064 387185.879765738
				11065 387196.675556203
				11066 387207.471346667
				11067 387218.267137132
				11068 387229.062927596
				11069 387239.858718061
				11070 387250.654508526
				11071 387261.45029899
				11072 387272.246089455
				11073 387283.041879919
				11074 387293.837670384
				11075 387304.633460848
				11076 387315.429251313
				11077 387326.225041778
				11078 387337.020832242
				11079 387347.816622707
				11080 387358.612413171
				11081 387369.408203636
				11082 387380.203994101
				11083 387390.999784565
				11084 387401.79557503
				11085 387412.591365494
				11086 387423.387155959
				11087 387434.182946423
				11088 387444.978736888
				11089 387455.774527353
				11090 387466.570317817
				11091 387477.366108282
				11092 387488.161898746
				11093 387498.957689211
				11094 387509.753479676
				11095 387520.54927014
				11096 387531.345060605
				11097 387542.140851069
				11098 387552.936641534
				11099 387563.732431998
				11100 387574.528222463
				11101 387585.324012928
				11102 387596.119803392
				11103 387606.915593857
				11104 387617.711384321
				11105 387628.507174786
				11106 387639.30296525
				11107 387650.098755715
				11108 387660.89454618
				11109 387671.690336644
				11110 387682.486127109
				11111 387693.281917573
				11112 387704.077708038
				11113 387714.873498502
				11114 387725.669288967
				11115 387736.465079432
				11116 387747.260869896
				11117 387758.056660361
				11118 387768.852450825
				11119 387779.64824129
				11120 387790.444031755
				11121 387801.239822219
				11122 387812.035612684
				11123 387822.831403148
				11124 387833.627193613
				11125 387844.422984077
				11126 387855.218774542
				11127 387866.014565007
				11128 387876.810355471
				11129 387887.606145936
				11130 387898.4019364
				11131 387909.197726865
				11132 387919.99351733
				11133 387930.789307794
				11134 387941.585098259
				11135 387952.380888723
				11136 387963.176679188
				11137 387973.972469652
				11138 387984.768260117
				11139 387995.564050582
				11140 388006.359841046
				11141 388017.155631511
				11142 388027.951421975
				11143 388038.74721244
				11144 388049.543002905
				11145 388060.338793369
				11146 388071.134583834
				11147 388081.930374298
				11148 388092.726164763
				11149 388103.521955227
				11150 388114.317745692
				11151 388125.113536157
				11152 388135.909326621
				11153 388146.705117086
				11154 388157.50090755
				11155 388168.296698015
				11156 388179.092488479
				11157 388189.888278944
				11158 388200.684069409
				11159 388211.479859873
				11160 388222.275650338
				11161 388233.071440802
				11162 388243.867231267
				11163 388254.663021731
				11164 388265.458812196
				11165 388276.254602661
				11166 388287.050393125
				11167 388297.84618359
				11168 388308.641974054
				11169 388319.437764519
				11170 388330.233554984
				11171 388341.029345448
				11172 388351.825135913
				11173 388362.620926377
				11174 388373.416716842
				11175 388384.212507306
				11176 388395.008297771
				11177 388405.804088236
				11178 388416.5998787
				11179 388427.395669165
				11180 388438.191459629
				11181 388448.987250094
				11182 388459.783040559
				11183 388470.578831023
				11184 388481.374621488
				11185 388492.170411952
				11186 388502.966202417
				11187 388513.761992881
				11188 388524.557783346
				11189 388535.353573811
				11190 388546.149364275
				11191 388556.94515474
				11192 388567.740945204
				11193 388578.536735669
				11194 388589.332526133
				11195 388600.128316598
				11196 388610.924107063
				11197 388621.719897527
				11198 388632.515687992
				11199 388643.311478456
				11200 388654.107268921
				11201 388664.903059386
				11202 388675.69884985
				11203 388686.494640315
				11204 388697.290430779
				11205 388708.086221244
				11206 388718.882011708
				11207 388729.677802173
				11208 388740.473592638
				11209 388751.269383102
				11210 388762.065173567
				11211 388772.860964031
				11212 388783.656754496
				11213 388794.45254496
				11214 388805.248335425
				11215 388816.04412589
				11216 388826.839916354
				11217 388837.635706819
				11218 388848.431497283
				11219 388859.227287748
				11220 388870.023078213
				11221 388880.818868677
				11222 388891.614659142
				11223 388902.410449606
				11224 388913.206240071
				11225 388924.002030535
				11226 388934.797821
				11227 388945.593611465
				11228 388956.389401929
				11229 388967.185192394
				11230 388977.980982858
				11231 388988.776773323
				11232 388999.572563788
				11233 389010.368354252
				11234 389021.164144717
				11235 389031.959935181
				11236 389042.755725646
				11237 389053.55151611
				11238 389064.347306575
				11239 389075.14309704
				11240 389085.938887504
				11241 389096.734677969
				11242 389107.530468433
				11243 389118.326258898
				11244 389129.122049362
				11245 389139.917839827
				11246 389150.713630292
				11247 389161.509420756
				11248 389172.305211221
				11249 389183.101001685
				11250 389193.89679215
				11251 389204.692582614
				11252 389215.488373079
				11253 389226.284163544
				11254 389237.079954008
				11255 389247.875744473
				11256 389258.671534937
				11257 389269.467325402
				11258 389280.263115867
				11259 389291.058906331
				11260 389301.854696796
				11261 389312.65048726
				11262 389323.446277725
				11263 389334.242068189
				11264 389345.037858654
				11265 389355.833649119
				11266 389366.629439583
				11267 389377.425230048
				11268 389388.221020512
				11269 389399.016810977
				11270 389409.812601442
				11271 389420.608391906
				11272 389431.404182371
				11273 389442.199972835
				11274 389452.9957633
				11275 389463.791553764
				11276 389474.587344229
				11277 389485.383134694
				11278 389496.178925158
				11279 389506.974715623
				11280 389517.770506087
				11281 389528.566296552
				11282 389539.362087016
				11283 389550.157877481
				11284 389560.953667946
				11285 389571.74945841
				11286 389582.545248875
				11287 389593.341039339
				11288 389604.136829804
				11289 389614.932620269
				11290 389625.728410733
				11291 389636.524201198
				11292 389647.319991662
				11293 389658.115782127
				11294 389668.911572591
				11295 389679.707363056
				11296 389690.503153521
				11297 389701.298943985
				11298 389712.09473445
				11299 389722.890524914
				11300 389733.686315379
				11301 389744.482105843
				11302 389755.277896308
				11303 389766.073686773
				11304 389776.869477237
				11305 389787.665267702
				11306 389798.461058166
				11307 389809.256848631
				11308 389820.052639096
				11309 389830.84842956
				11310 389841.644220025
				11311 389852.440010489
				11312 389863.235800954
				11313 389874.031591418
				11314 389884.827381883
				11315 389895.623172348
				11316 389906.418962812
				11317 389917.214753277
				11318 389928.010543741
				11319 389938.806334206
				11320 389949.602124671
				11321 389960.397915135
				11322 389971.1937056
				11323 389981.989496064
				11324 389992.785286529
				11325 390003.581076993
				11326 390014.376867458
				11327 390025.172657923
				11328 390035.968448387
				11329 390046.764238852
				11330 390057.560029316
				11331 390068.355819781
				11332 390079.151610245
				11333 390089.94740071
				11334 390100.743191175
				11335 390111.538981639
				11336 390122.334772104
				11337 390133.130562568
				11338 390143.926353033
				11339 390154.722143498
				11340 390165.517933962
				11341 390176.313724427
				11342 390187.109514891
				11343 390197.905305356
				11344 390208.70109582
				11345 390219.496886285
				11346 390230.29267675
				11347 390241.088467214
				11348 390251.884257679
				11349 390262.680048143
				11350 390273.475838608
				11351 390284.271629072
				11352 390295.067419537
				11353 390305.863210002
				11354 390316.659000466
				11355 390327.454790931
				11356 390338.250581395
				11357 390349.04637186
				11358 390359.842162325
				11359 390370.637952789
				11360 390381.433743254
				11361 390392.229533718
				11362 390403.025324183
				11363 390413.821114647
				11364 390424.616905112
				11365 390435.412695577
				11366 390446.208486041
				11367 390457.004276506
				11368 390467.80006697
				11369 390478.595857435
				11370 390489.3916479
				11371 390500.187438364
				11372 390510.983228829
				11373 390521.779019293
				11374 390532.574809758
				11375 390543.370600222
				11376 390554.166390687
				11377 390564.962181152
				11378 390575.757971616
				11379 390586.553762081
				11380 390597.349552545
				11381 390608.14534301
				11382 390618.941133474
				11383 390629.736923939
				11384 390640.532714404
				11385 390651.328504868
				11386 390662.124295333
				11387 390672.920085797
				11388 390683.715876262
				11389 390694.511666727
				11390 390705.307457191
				11391 390716.103247656
				11392 390726.89903812
				11393 390737.694828585
				11394 390748.490619049
				11395 390759.286409514
				11396 390770.082199979
				11397 390780.877990443
				11398 390791.673780908
				11399 390802.469571372
				11400 390813.265361837
				11401 390824.061152301
				11402 390834.856942766
				11403 390845.652733231
				11404 390856.448523695
				11405 390867.24431416
				11406 390878.040104624
				11407 390888.835895089
				11408 390899.631685554
				11409 390910.427476018
				11410 390921.223266483
				11411 390932.019056947
				11412 390942.814847412
				11413 390953.610637876
				11414 390964.406428341
				11415 390975.202218806
				11416 390985.99800927
				11417 390996.793799735
				11418 391007.589590199
				11419 391018.385380664
				11420 391029.181171129
				11421 391039.976961593
				11422 391050.772752058
				11423 391061.568542522
				11424 391072.364332987
				11425 391083.160123451
				11426 391093.955913916
				11427 391104.751704381
				11428 391115.547494845
				11429 391126.34328531
				11430 391137.139075774
				11431 391147.934866239
				11432 391158.730656703
				11433 391169.526447168
				11434 391180.322237633
				11435 391191.118028097
				11436 391201.913818562
				11437 391212.709609026
				11438 391223.505399491
				11439 391234.301189955
				11440 391245.09698042
				11441 391255.892770885
				11442 391266.688561349
				11443 391277.484351814
				11444 391288.280142278
				11445 391299.075932743
				11446 391309.871723208
				11447 391320.667513672
				11448 391331.463304137
				11449 391342.259094601
				11450 391353.054885066
				11451 391363.85067553
				11452 391374.646465995
				11453 391385.44225646
				11454 391396.238046924
				11455 391407.033837389
				11456 391417.829627853
				11457 391428.625418318
				11458 391439.421208783
				11459 391450.216999247
				11460 391461.012789712
				11461 391471.808580176
				11462 391482.604370641
				11463 391493.400161105
				11464 391504.19595157
				11465 391514.991742035
				11466 391525.787532499
				11467 391536.583322964
				11468 391547.379113428
				11469 391558.174903893
				11470 391568.970694357
				11471 391579.766484822
				11472 391590.562275287
				11473 391601.358065751
				11474 391612.153856216
				11475 391622.94964668
				11476 391633.745437145
				11477 391644.54122761
				11478 391655.337018074
				11479 391666.132808539
				11480 391676.928599003
				11481 391687.724389468
				11482 391698.520179932
				11483 391709.315970397
				11484 391720.111760862
				11485 391730.907551326
				11486 391741.703341791
				11487 391752.499132255
				11488 391763.29492272
				11489 391774.090713185
				11490 391784.886503649
				11491 391795.682294114
				11492 391806.478084578
				11493 391817.273875043
				11494 391828.069665507
				11495 391838.865455972
				11496 391849.661246437
				11497 391860.457036901
				11498 391871.252827366
				11499 391882.04861783
				11500 391892.844408295
				11501 391903.640198759
				11502 391914.435989224
				11503 391925.231779689
				11504 391936.027570153
				11505 391946.823360618
				11506 391957.619151082
				11507 391968.414941547
				11508 391979.210732012
				11509 391990.006522476
				11510 392000.802312941
				11511 392011.598103405
				11512 392022.39389387
				11513 392033.189684334
				11514 392043.985474799
				11515 392054.781265264
				11516 392065.577055728
				11517 392076.372846193
				11518 392087.168636657
				11519 392097.964427122
				11520 392108.760217586
				11521 392119.556008051
				11522 392130.351798516
				11523 392141.14758898
				11524 392151.943379445
				11525 392162.739169909
				11526 392173.534960374
				11527 392184.330750839
				11528 392195.126541303
				11529 392205.922331768
				11530 392216.718122232
				11531 392227.513912697
				11532 392238.309703161
				11533 392249.105493626
				11534 392259.901284091
				11535 392270.697074555
				11536 392281.49286502
				11537 392292.288655484
				11538 392303.084445949
				11539 392313.880236414
				11540 392324.676026878
				11541 392335.471817343
				11542 392346.267607807
				11543 392357.063398272
				11544 392367.859188736
				11545 392378.654979201
				11546 392389.450769666
				11547 392400.24656013
				11548 392411.042350595
				11549 392421.838141059
				11550 392432.633931524
				11551 392443.429721988
				11552 392454.225512453
				11553 392465.021302918
				11554 392475.817093382
				11555 392486.612883847
				11556 392497.408674311
				11557 392508.204464776
				11558 392519.00025524
				11559 392529.796045705
				11560 392540.59183617
				11561 392551.387626634
				11562 392562.183417099
				11563 392572.979207563
				11564 392583.774998028
				11565 392594.570788493
				11566 392605.366578957
				11567 392616.162369422
				11568 392626.958159886
				11569 392637.753950351
				11570 392648.549740815
				11571 392659.34553128
				11572 392670.141321745
				11573 392680.937112209
				11574 392691.732902674
				11575 392702.528693138
				11576 392713.324483603
				11577 392724.120274068
				11578 392734.916064532
				11579 392745.711854997
				11580 392756.507645461
				11581 392767.303435926
				11582 392778.09922639
				11583 392788.895016855
				11584 392799.69080732
				11585 392810.486597784
				11586 392821.282388249
				11587 392832.078178713
				11588 392842.873969178
				11589 392853.669759643
				11590 392864.465550107
				11591 392875.261340572
				11592 392886.057131036
				11593 392896.852921501
				11594 392907.648711965
				11595 392918.44450243
				11596 392929.240292895
				11597 392940.036083359
				11598 392950.831873824
				11599 392961.627664288
				11600 392972.423454753
				11601 392983.219245217
				11602 392994.015035682
				11603 393004.810826147
				11604 393015.606616611
				11605 393026.402407076
				11606 393037.19819754
				11607 393047.993988005
				11608 393058.789778469
				11609 393069.585568934
				11610 393080.381359399
				11611 393091.177149863
				11612 393101.972940328
				11613 393112.768730792
				11614 393123.564521257
				11615 393134.360311722
				11616 393145.156102186
				11617 393155.951892651
				11618 393166.747683115
				11619 393177.54347358
				11620 393188.339264044
				11621 393199.135054509
				11622 393209.930844974
				11623 393220.726635438
				11624 393231.522425903
				11625 393242.318216367
				11626 393253.114006832
				11627 393263.909797297
				11628 393274.705587761
				11629 393285.501378226
				11630 393296.29716869
				11631 393307.092959155
				11632 393317.888749619
				11633 393328.684540084
				11634 393339.480330549
				11635 393350.276121013
				11636 393361.071911478
				11637 393371.867701942
				11638 393382.663492407
				11639 393393.459282871
				11640 393404.255073336
				11641 393415.050863801
				11642 393425.846654265
				11643 393436.64244473
				11644 393447.438235194
				11645 393458.234025659
				11646 393469.029816124
				11647 393479.825606588
				11648 393490.621397053
				11649 393501.417187517
				11650 393512.212977982
				11651 393523.008768446
				11652 393533.804558911
				11653 393544.600349376
				11654 393555.39613984
				11655 393566.191930305
				11656 393576.987720769
				11657 393587.783511234
				11658 393598.579301698
				11659 393609.375092163
				11660 393620.170882628
				11661 393630.966673092
				11662 393641.762463557
				11663 393652.558254021
				11664 393663.354044486
				11665 393674.149834951
				11666 393684.945625415
				11667 393695.74141588
				11668 393706.537206344
				11669 393717.332996809
				11670 393728.128787273
				11671 393738.924577738
				11672 393749.720368203
				11673 393760.516158667
				11674 393771.311949132
				11675 393782.107739596
				11676 393792.903530061
				11677 393803.699320526
				11678 393814.49511099
				11679 393825.290901455
				11680 393836.086691919
				11681 393846.882482384
				11682 393857.678272848
				11683 393868.474063313
				11684 393879.269853778
				11685 393890.065644242
				11686 393900.861434707
				11687 393911.657225171
				11688 393922.453015636
				11689 393933.2488061
				11690 393944.044596565
				11691 393954.84038703
				11692 393965.636177494
				11693 393976.431967959
				11694 393987.227758423
				11695 393998.023548888
				11696 394008.819339353
				11697 394019.615129817
				11698 394030.410920282
				11699 394041.206710746
				11700 394052.002501211
				11701 394062.798291675
				11702 394073.59408214
				11703 394084.389872605
				11704 394095.185663069
				11705 394105.981453534
				11706 394116.777243998
				11707 394127.573034463
				11708 394138.368824927
				11709 394149.164615392
				11710 394159.960405857
				11711 394170.756196321
				11712 394181.551986786
				11713 394192.34777725
				11714 394203.143567715
				11715 394213.93935818
				11716 394224.735148644
				11717 394235.530939109
				11718 394246.326729573
				11719 394257.122520038
				11720 394267.918310502
				11721 394278.714100967
				11722 394289.509891432
				11723 394300.305681896
				11724 394311.101472361
				11725 394321.897262825
				11726 394332.69305329
				11727 394343.488843755
				11728 394354.284634219
				11729 394365.080424684
				11730 394375.876215148
				11731 394386.672005613
				11732 394397.467796077
				11733 394408.263586542
				11734 394419.059377007
				11735 394429.855167471
				11736 394440.650957936
				11737 394451.4467484
				11738 394462.242538865
				11739 394473.038329329
				11740 394483.834119794
				11741 394494.629910259
				11742 394505.425700723
				11743 394516.221491188
				11744 394527.017281652
				11745 394537.813072117
				11746 394548.608862582
				11747 394559.404653046
				11748 394570.200443511
				11749 394580.996233975
				11750 394591.79202444
				11751 394602.587814904
				11752 394613.383605369
				11753 394624.179395834
				11754 394634.975186298
				11755 394645.770976763
				11756 394656.566767227
				11757 394667.362557692
				11758 394678.158348156
				11759 394688.954138621
				11760 394699.749929086
				11761 394710.54571955
				11762 394721.341510015
				11763 394732.137300479
				11764 394742.933090944
				11765 394753.728881409
				11766 394764.524671873
				11767 394775.320462338
				11768 394786.116252802
				11769 394796.912043267
				11770 394807.707833731
				11771 394818.503624196
				11772 394829.299414661
				11773 394840.095205125
				11774 394850.89099559
				11775 394861.686786054
				11776 394872.482576519
				11777 394883.278366984
				11778 394894.074157448
				11779 394904.869947913
				11780 394915.665738377
				11781 394926.461528842
				11782 394937.257319306
				11783 394948.053109771
				11784 394958.848900236
				11785 394969.6446907
				11786 394980.440481165
				11787 394991.236271629
				11788 395002.032062094
				11789 395012.827852558
				11790 395023.623643023
				11791 395034.419433488
				11792 395045.215223952
				11793 395056.011014417
				11794 395066.806804881
				11795 395077.602595346
				11796 395088.39838581
				11797 395099.194176275
				11798 395109.98996674
				11799 395120.785757204
				11800 395131.581547669
				11801 395142.377338133
				11802 395153.173128598
				11803 395163.968919063
				11804 395174.764709527
				11805 395185.560499992
				11806 395196.356290456
				11807 395207.152080921
				11808 395217.947871385
				11809 395228.74366185
				11810 395239.539452315
				11811 395250.335242779
				11812 395261.131033244
				11813 395271.926823708
				11814 395282.722614173
				11815 395293.518404638
				11816 395304.314195102
				11817 395315.109985567
				11818 395325.905776031
				11819 395336.701566496
				11820 395347.49735696
				11821 395358.293147425
				11822 395369.08893789
				11823 395379.884728354
				11824 395390.680518819
				11825 395401.476309283
				11826 395412.272099748
				11827 395423.067890212
				11828 395433.863680677
				11829 395444.659471142
				11830 395455.455261606
				11831 395466.251052071
				11832 395477.046842535
				11833 395487.842633
				11834 395498.638423465
				11835 395509.434213929
				11836 395520.230004394
				11837 395531.025794858
				11838 395541.821585323
				11839 395552.617375787
				11840 395563.413166252
				11841 395574.208956717
				11842 395585.004747181
				11843 395595.800537646
				11844 395606.59632811
				11845 395617.392118575
				11846 395628.187909039
				11847 395638.983699504
				11848 395649.779489969
				11849 395660.575280433
				11850 395671.371070898
				11851 395682.166861362
				11852 395692.962651827
				11853 395703.758442292
				11854 395714.554232756
				11855 395725.350023221
				11856 395736.145813685
				11857 395746.94160415
				11858 395757.737394614
				11859 395768.533185079
				11860 395779.328975544
				11861 395790.124766008
				11862 395800.920556473
				11863 395811.716346937
				11864 395822.512137402
				11865 395833.307927867
				11866 395844.103718331
				11867 395854.899508796
				11868 395865.69529926
				11869 395876.491089725
				11870 395887.286880189
				11871 395898.082670654
				11872 395908.878461119
				11873 395919.674251583
				11874 395930.470042048
				11875 395941.265832512
				11876 395952.061622977
				11877 395962.857413441
				11878 395973.653203906
				11879 395984.448994371
				11880 395995.244784835
				11881 396006.0405753
				11882 396016.836365764
				11883 396027.632156229
				11884 396038.427946693
				11885 396049.223737158
				11886 396060.019527623
				11887 396070.815318087
				11888 396081.611108552
				11889 396092.406899016
				11890 396103.202689481
				11891 396113.998479946
				11892 396124.79427041
				11893 396135.590060875
				11894 396146.385851339
				11895 396157.181641804
				11896 396167.977432268
				11897 396178.773222733
				11898 396189.569013198
				11899 396200.364803662
				11900 396211.160594127
				11901 396221.956384591
				11902 396232.752175056
				11903 396243.547965521
				11904 396254.343755985
				11905 396265.13954645
				11906 396275.935336914
				11907 396286.731127379
				11908 396297.526917843
				11909 396308.322708308
				11910 396319.118498773
				11911 396329.914289237
				11912 396340.710079702
				11913 396351.505870166
				11914 396362.301660631
				11915 396373.097451095
				11916 396383.89324156
				11917 396394.689032025
				11918 396405.484822489
				11919 396416.280612954
				11920 396427.076403418
				11921 396437.872193883
				11922 396448.667984348
				11923 396459.463774812
				11924 396470.259565277
				11925 396481.055355741
				11926 396491.851146206
				11927 396502.64693667
				11928 396513.442727135
				11929 396524.2385176
				11930 396535.034308064
				11931 396545.830098529
				11932 396556.625888993
				11933 396567.421679458
				11934 396578.217469922
				11935 396589.013260387
				11936 396599.809050852
				11937 396610.604841316
				11938 396621.400631781
				11939 396632.196422245
				11940 396642.99221271
				11941 396653.788003175
				11942 396664.583793639
				11943 396675.379584104
				11944 396686.175374568
				11945 396696.971165033
				11946 396707.766955497
				11947 396718.562745962
				11948 396729.358536427
				11949 396740.154326891
				11950 396750.950117356
				11951 396761.74590782
				11952 396772.541698285
				11953 396783.33748875
				11954 396794.133279214
				11955 396804.929069679
				11956 396815.724860143
				11957 396826.520650608
				11958 396837.316441072
				11959 396848.112231537
				11960 396858.908022002
				11961 396869.703812466
				11962 396880.499602931
				11963 396891.295393395
				11964 396902.09118386
				11965 396912.886974324
				11966 396923.682764789
				11967 396934.478555254
				11968 396945.274345718
				11969 396956.070136183
				11970 396966.865926647
				11971 396977.661717112
				11972 396988.457507577
				11973 396999.253298041
				11974 397010.049088506
				11975 397020.84487897
				11976 397031.640669435
				11977 397042.436459899
				11978 397053.232250364
				11979 397064.028040829
				11980 397074.823831293
				11981 397085.619621758
				11982 397096.415412222
				11983 397107.211202687
				11984 397118.006993152
				11985 397128.802783616
				11986 397139.598574081
				11987 397150.394364545
				11988 397161.19015501
				11989 397171.985945474
				11990 397182.781735939
				11991 397193.577526404
				11992 397204.373316868
				11993 397215.169107333
				11994 397225.964897797
				11995 397236.760688262
				11996 397247.556478726
				11997 397258.352269191
				11998 397269.148059656
				11999 397279.94385012
				12000 397290.739640585
				12001 397301.535431049
				12002 397312.331221514
				12003 397323.127011979
				12004 397333.922802443
				12005 397344.718592908
				12006 397355.514383372
				12007 397366.310173837
				12008 397377.105964301
				12009 397387.901754766
				12010 397398.697545231
				12011 397409.493335695
				12012 397420.28912616
				12013 397431.084916624
				12014 397441.880707089
				12015 397452.676497553
				12016 397463.472288018
				12017 397474.268078483
				12018 397485.063868947
				12019 397495.859659412
				12020 397506.655449876
				12021 397517.451240341
				12022 397528.247030806
				12023 397539.04282127
				12024 397549.838611735
				12025 397560.634402199
				12026 397571.430192664
				12027 397582.225983128
				12028 397593.021773593
				12029 397603.817564058
				12030 397614.613354522
				12031 397625.409144987
				12032 397636.204935451
				12033 397647.000725916
				12034 397657.796516381
				12035 397668.592306845
				12036 397679.38809731
				12037 397690.183887774
				12038 397700.979678239
				12039 397711.775468703
				12040 397722.571259168
				12041 397733.367049633
				12042 397744.162840097
				12043 397754.958630562
				12044 397765.754421026
				12045 397776.550211491
				12046 397787.346001955
				12047 397798.14179242
				12048 397808.937582885
				12049 397819.733373349
				12050 397830.529163814
				12051 397841.324954278
				12052 397852.120744743
				12053 397862.916535208
				12054 397873.712325672
				12055 397884.508116137
				12056 397895.303906601
				12057 397906.099697066
				12058 397916.89548753
				12059 397927.691277995
				12060 397938.48706846
				12061 397949.282858924
				12062 397960.078649389
				12063 397970.874439853
				12064 397981.670230318
				12065 397992.466020782
				12066 398003.261811247
				12067 398014.057601712
				12068 398024.853392176
				12069 398035.649182641
				12070 398046.444973105
				12071 398057.24076357
				12072 398068.036554035
				12073 398078.832344499
				12074 398089.628134964
				12075 398100.423925428
				12076 398111.219715893
				12077 398122.015506357
				12078 398132.811296822
				12079 398143.607087287
				12080 398154.402877751
				12081 398165.198668216
				12082 398175.99445868
				12083 398186.790249145
				12084 398197.58603961
				12085 398208.381830074
				12086 398219.177620539
				12087 398229.973411003
				12088 398240.769201468
				12089 398251.564991932
				12090 398262.360782397
				12091 398273.156572862
				12092 398283.952363326
				12093 398294.748153791
				12094 398305.543944255
				12095 398316.33973472
				12096 398327.135525184
				12097 398337.931315649
				12098 398348.727106114
				12099 398359.522896578
				12100 398370.318687043
				12101 398381.114477507
				12102 398391.910267972
				12103 398402.706058437
				12104 398413.501848901
				12105 398424.297639366
				12106 398435.09342983
				12107 398445.889220295
				12108 398456.685010759
				12109 398467.480801224
				12110 398478.276591689
				12111 398489.072382153
				12112 398499.868172618
				12113 398510.663963082
				12114 398521.459753547
				12115 398532.255544011
				12116 398543.051334476
				12117 398553.847124941
				12118 398564.642915405
				12119 398575.43870587
				12120 398586.234496334
				12121 398597.030286799
				12122 398607.826077264
				12123 398618.621867728
				12124 398629.417658193
				12125 398640.213448657
				12126 398651.009239122
				12127 398661.805029586
				12128 398672.600820051
				12129 398683.396610516
				12130 398694.19240098
				12131 398704.988191445
				12132 398715.783981909
				12133 398726.579772374
				12134 398737.375562839
				12135 398748.171353303
				12136 398758.967143768
				12137 398769.762934232
				12138 398780.558724697
				12139 398791.354515161
				12140 398802.150305626
				12141 398812.946096091
				12142 398823.741886555
				12143 398834.53767702
				12144 398845.333467484
				12145 398856.129257949
				12146 398866.925048413
				12147 398877.720838878
				12148 398888.516629343
				12149 398899.312419807
				12150 398910.108210272
				12151 398920.904000736
				12152 398931.699791201
				12153 398942.495581665
				12154 398953.29137213
				12155 398964.087162595
				12156 398974.882953059
				12157 398985.678743524
				12158 398996.474533988
				12159 399007.270324453
				12160 399018.066114918
				12161 399028.861905382
				12162 399039.657695847
				12163 399050.453486311
				12164 399061.249276776
				12165 399072.04506724
				12166 399082.840857705
				12167 399093.63664817
				12168 399104.432438634
				12169 399115.228229099
				12170 399126.024019563
				12171 399136.819810028
				12172 399147.615600493
				12173 399158.411390957
				12174 399169.207181422
				12175 399180.002971886
				12176 399190.798762351
				12177 399201.594552815
				12178 399212.39034328
				12179 399223.186133745
				12180 399233.981924209
				12181 399244.777714674
				12182 399255.573505138
				12183 399266.369295603
				12184 399277.165086068
				12185 399287.960876532
				12186 399298.756666997
				12187 399309.552457461
				12188 399320.348247926
				12189 399331.14403839
				12190 399341.939828855
				12191 399352.73561932
				12192 399363.531409784
				12193 399374.327200249
				12194 399385.122990713
				12195 399395.918781178
				12196 399406.714571642
				12197 399417.510362107
				12198 399428.306152572
				12199 399439.101943036
				12200 399449.897733501
				12201 399460.693523965
				12202 399471.48931443
				12203 399482.285104894
				12204 399493.080895359
				12205 399503.876685824
				12206 399514.672476288
				12207 399525.468266753
				12208 399536.264057217
				12209 399547.059847682
				12210 399557.855638147
				12211 399568.651428611
				12212 399579.447219076
				12213 399590.24300954
				12214 399601.038800005
				12215 399611.834590469
				12216 399622.630380934
				12217 399633.426171399
				12218 399644.221961863
				12219 399655.017752328
				12220 399665.813542792
				12221 399676.609333257
				12222 399687.405123722
				12223 399698.200914186
				12224 399708.996704651
				12225 399719.792495115
				12226 399730.58828558
				12227 399741.384076044
				12228 399752.179866509
				12229 399762.975656974
				12230 399773.771447438
				12231 399784.567237903
				12232 399795.363028367
				12233 399806.158818832
				12234 399816.954609296
				12235 399827.750399761
				12236 399838.546190226
				12237 399849.34198069
				12238 399860.137771155
				12239 399870.933561619
				12240 399881.729352084
				12241 399892.525142548
				12242 399903.320933013
				12243 399914.116723478
				12244 399924.912513942
				12245 399935.708304407
				12246 399946.504094871
				12247 399957.299885336
				12248 399968.095675801
				12249 399978.891466265
				12250 399989.68725673
				12251 400000.483047194
				12252 400011.278837659
				12253 400022.074628123
				12254 400032.870418588
				12255 400043.666209053
				12256 400054.461999517
				12257 400065.257789982
				12258 400076.053580446
				12259 400086.849370911
				12260 400097.645161376
				12261 400108.44095184
				12262 400119.236742305
				12263 400130.032532769
				12264 400140.828323234
				12265 400151.624113698
				12266 400162.419904163
				12267 400173.215694628
				12268 400184.011485092
				12269 400194.807275557
				12270 400205.603066021
				12271 400216.398856486
				12272 400227.19464695
				12273 400237.990437415
				12274 400248.78622788
				12275 400259.582018344
				12276 400270.377808809
				12277 400281.173599273
				12278 400291.969389738
				12279 400302.765180203
				12280 400313.560970667
				12281 400324.356761132
				12282 400335.152551596
				12283 400345.948342061
				12284 400356.744132525
				12285 400367.53992299
				12286 400378.335713455
				12287 400389.131503919
				12288 400399.927294384
				12289 400410.723084848
				12290 400421.518875313
				12291 400432.314665777
				12292 400443.110456242
				12293 400453.906246707
				12294 400464.702037171
				12295 400475.497827636
				12296 400486.2936181
				12297 400497.089408565
				12298 400507.88519903
				12299 400518.680989494
				12300 400529.476779959
				12301 400540.272570423
				12302 400551.068360888
				12303 400561.864151352
				12304 400572.659941817
				12305 400583.455732282
				12306 400594.251522746
				12307 400605.047313211
				12308 400615.843103675
				12309 400626.63889414
				12310 400637.434684605
				12311 400648.230475069
				12312 400659.026265534
				12313 400669.822055998
				12314 400680.617846463
				12315 400691.413636927
				12316 400702.209427392
				12317 400713.005217857
				12318 400723.801008321
				12319 400734.596798786
				12320 400745.39258925
				12321 400756.188379715
				12322 400766.984170179
				12323 400777.779960644
				12324 400788.575751109
				12325 400799.371541573
				12326 400810.167332038
				12327 400820.963122502
				12328 400831.758912967
				12329 400842.554703432
				12330 400853.350493896
				12331 400864.146284361
				12332 400874.942074825
				12333 400885.73786529
				12334 400896.533655754
				12335 400907.329446219
				12336 400918.125236684
				12337 400928.921027148
				12338 400939.716817613
				12339 400950.512608077
				12340 400961.308398542
				12341 400972.104189006
				12342 400982.899979471
				12343 400993.695769936
				12344 401004.4915604
				12345 401015.287350865
				12346 401026.083141329
				12347 401036.878931794
				12348 401047.674722259
				12349 401058.470512723
				12350 401069.266303188
				12351 401080.062093652
				12352 401090.857884117
				12353 401101.653674581
				12354 401112.449465046
				12355 401123.245255511
				12356 401134.041045975
				12357 401144.83683644
				12358 401155.632626904
				12359 401166.428417369
				12360 401177.224207834
				12361 401188.019998298
				12362 401198.815788763
				12363 401209.611579227
				12364 401220.407369692
				12365 401231.203160156
				12366 401241.998950621
				12367 401252.794741086
				12368 401263.59053155
				12369 401274.386322015
				12370 401285.182112479
				12371 401295.977902944
				12372 401306.773693408
				12373 401317.569483873
				12374 401328.365274338
				12375 401339.161064802
				12376 401349.956855267
				12377 401360.752645731
				12378 401371.548436196
				12379 401382.344226661
				12380 401393.140017125
				12381 401403.93580759
				12382 401414.731598054
				12383 401425.527388519
				12384 401436.323178983
				12385 401447.118969448
				12386 401457.914759913
				12387 401468.710550377
				12388 401479.506340842
				12389 401490.302131306
				12390 401501.097921771
				12391 401511.893712235
				12392 401522.6895027
				12393 401533.485293165
				12394 401544.281083629
				12395 401555.076874094
				12396 401565.872664558
				12397 401576.668455023
				12398 401587.464245488
				12399 401598.260035952
				12400 401609.055826417
				12401 401619.851616881
				12402 401630.647407346
				12403 401641.44319781
				12404 401652.238988275
				12405 401663.03477874
				12406 401673.830569204
				12407 401684.626359669
				12408 401695.422150133
				12409 401706.217940598
				12410 401717.013731063
				12411 401727.809521527
				12412 401738.605311992
				12413 401749.401102456
				12414 401760.196892921
				12415 401770.992683385
				12416 401781.78847385
				12417 401792.584264315
				12418 401803.380054779
				12419 401814.175845244
				12420 401824.971635708
				12421 401835.767426173
				12422 401846.563216637
				12423 401857.359007102
				12424 401868.154797567
				12425 401878.950588031
				12426 401889.746378496
				12427 401900.54216896
				12428 401911.337959425
				12429 401922.13374989
				12430 401932.929540354
				12431 401943.725330819
				12432 401954.521121283
				12433 401965.316911748
				12434 401976.112702212
				12435 401986.908492677
				12436 401997.704283142
				12437 402008.500073606
				12438 402019.295864071
				12439 402030.091654535
				12440 402040.887445
				12441 402051.683235465
				12442 402062.479025929
				12443 402073.274816394
				12444 402084.070606858
				12445 402094.866397323
				12446 402105.662187787
				12447 402116.457978252
				12448 402127.253768717
				12449 402138.049559181
				12450 402148.845349646
				12451 402159.64114011
				12452 402170.436930575
				12453 402181.232721039
				12454 402192.028511504
				12455 402202.824301969
				12456 402213.620092433
				12457 402224.415882898
				12458 402235.211673362
				12459 402246.007463827
				12460 402256.803254292
				12461 402267.599044756
				12462 402278.394835221
				12463 402289.190625685
				12464 402299.98641615
				12465 402310.782206614
				12466 402321.577997079
				12467 402332.373787544
				12468 402343.169578008
				12469 402353.965368473
				12470 402364.761158937
				12471 402375.556949402
				12472 402386.352739866
				12473 402397.148530331
				12474 402407.944320796
				12475 402418.74011126
				12476 402429.535901725
				12477 402440.331692189
				12478 402451.127482654
				12479 402461.923273119
				12480 402472.719063583
				12481 402483.514854048
				12482 402494.310644512
				12483 402505.106434977
				12484 402515.902225441
				12485 402526.698015906
				12486 402537.493806371
				12487 402548.289596835
				12488 402559.0853873
				12489 402569.881177764
				12490 402580.676968229
				12491 402591.472758694
				12492 402602.268549158
				12493 402613.064339623
				12494 402623.860130087
				12495 402634.655920552
				12496 402645.451711016
				12497 402656.247501481
				12498 402667.043291946
				12499 402677.83908241
				12500 402688.634872875
				12501 402699.430663339
				12502 402710.226453804
				12503 402721.022244268
				12504 402731.818034733
				12505 402742.613825198
				12506 402753.409615662
				12507 402764.205406127
				12508 402775.001196591
				12509 402785.796987056
				12510 402796.59277752
				12511 402807.388567985
				12512 402818.18435845
				12513 402828.980148914
				12514 402839.775939379
				12515 402850.571729843
				12516 402861.367520308
				12517 402872.163310773
				12518 402882.959101237
				12519 402893.754891702
				12520 402904.550682166
				12521 402915.346472631
				12522 402926.142263095
				12523 402936.93805356
				12524 402947.733844025
				12525 402958.529634489
				12526 402969.325424954
				12527 402980.121215418
				12528 402990.917005883
				12529 403001.712796348
				12530 403012.508586812
				12531 403023.304377277
				12532 403034.100167741
				12533 403044.895958206
				12534 403055.69174867
				12535 403066.487539135
				12536 403077.2833296
				12537 403088.079120064
				12538 403098.874910529
				12539 403109.670700993
				12540 403120.466491458
				12541 403131.262281922
				12542 403142.058072387
				12543 403152.853862852
				12544 403163.649653316
				12545 403174.445443781
				12546 403185.241234245
				12547 403196.03702471
				12548 403206.832815174
				12549 403217.628605639
				12550 403228.424396104
				12551 403239.220186568
				12552 403250.015977033
				12553 403260.811767497
				12554 403271.607557962
				12555 403282.403348427
				12556 403293.199138891
				12557 403303.994929356
				12558 403314.79071982
				12559 403325.586510285
				12560 403336.382300749
				12561 403347.178091214
				12562 403357.973881679
				12563 403368.769672143
				12564 403379.565462608
				12565 403390.361253072
				12566 403401.157043537
				12567 403411.952834002
				12568 403422.748624466
				12569 403433.544414931
				12570 403444.340205395
				12571 403455.13599586
				12572 403465.931786324
				12573 403476.727576789
				12574 403487.523367254
				12575 403498.319157718
				12576 403509.114948183
				12577 403519.910738647
				12578 403530.706529112
				12579 403541.502319576
				12580 403552.298110041
				12581 403563.093900506
				12582 403573.88969097
				12583 403584.685481435
				12584 403595.481271899
				12585 403606.277062364
				12586 403617.072852829
				12587 403627.868643293
				12588 403638.664433758
				12589 403649.460224222
				12590 403660.256014687
				12591 403671.051805151
				12592 403681.847595616
				12593 403692.643386081
				12594 403703.439176545
				12595 403714.23496701
				12596 403725.030757474
				12597 403735.826547939
				12598 403746.622338403
				12599 403757.418128868
				12600 403768.213919333
				12601 403779.009709797
				12602 403789.805500262
				12603 403800.601290726
				12604 403811.397081191
				12605 403822.192871656
				12606 403832.98866212
				12607 403843.784452585
				12608 403854.580243049
				12609 403865.376033514
				12610 403876.171823978
				12611 403886.967614443
				12612 403897.763404908
				12613 403908.559195372
				12614 403919.354985837
				12615 403930.150776301
				12616 403940.946566766
				12617 403951.742357231
				12618 403962.538147695
				12619 403973.33393816
				12620 403984.129728624
				12621 403994.925519089
				12622 404005.721309553
				12623 404016.517100018
				12624 404027.312890483
				12625 404038.108680947
				12626 404048.904471412
				12627 404059.700261876
				12628 404070.496052341
				12629 404081.291842805
				12630 404092.08763327
				12631 404102.883423735
				12632 404113.679214199
				12633 404124.475004664
				12634 404135.270795128
				12635 404146.066585593
				12636 404156.862376058
				12637 404167.658166522
				12638 404178.453956987
				12639 404189.249747451
				12640 404200.045537916
				12641 404210.84132838
				12642 404221.637118845
				12643 404232.43290931
				12644 404243.228699774
				12645 404254.024490239
				12646 404264.820280703
				12647 404275.616071168
				12648 404286.411861632
				12649 404297.207652097
				12650 404308.003442562
				12651 404318.799233026
				12652 404329.595023491
				12653 404340.390813955
				12654 404351.18660442
				12655 404361.982394885
				12656 404372.778185349
				12657 404383.573975814
				12658 404394.369766278
				12659 404405.165556743
				12660 404415.961347207
				12661 404426.757137672
				12662 404437.552928137
				12663 404448.348718601
				12664 404459.144509066
				12665 404469.94029953
				12666 404480.736089995
				12667 404491.53188046
				12668 404502.327670924
				12669 404513.123461389
				12670 404523.919251853
				12671 404534.715042318
				12672 404545.510832782
				12673 404556.306623247
				12674 404567.102413712
				12675 404577.898204176
				12676 404588.693994641
				12677 404599.489785105
				12678 404610.28557557
				12679 404621.081366034
				12680 404631.877156499
				12681 404642.672946964
				12682 404653.468737428
				12683 404664.264527893
				12684 404675.060318357
				12685 404685.856108822
				12686 404696.651899287
				12687 404707.447689751
				12688 404718.243480216
				12689 404729.03927068
				12690 404739.835061145
				12691 404750.630851609
				12692 404761.426642074
				12693 404772.222432539
				12694 404783.018223003
				12695 404793.814013468
				12696 404804.609803932
				12697 404815.405594397
				12698 404826.201384861
				12699 404836.997175326
				12700 404847.792965791
				12701 404858.588756255
				12702 404869.38454672
				12703 404880.180337184
				12704 404890.976127649
				12705 404901.771918114
				12706 404912.567708578
				12707 404923.363499043
				12708 404934.159289507
				12709 404944.955079972
				12710 404955.750870436
				12711 404966.546660901
				12712 404977.342451366
				12713 404988.13824183
				12714 404998.934032295
				12715 405009.729822759
				12716 405020.525613224
				12717 405031.321403689
				12718 405042.117194153
				12719 405052.912984618
				12720 405063.708775082
				12721 405074.504565547
				12722 405085.300356011
				12723 405096.096146476
				12724 405106.891936941
				12725 405117.687727405
				12726 405128.48351787
				12727 405139.279308334
				12728 405150.075098799
				12729 405160.870889263
				12730 405171.666679728
				12731 405182.462470193
				12732 405193.258260657
				12733 405204.054051122
				12734 405214.849841586
				12735 405225.645632051
				12736 405236.441422516
				12737 405247.23721298
				12738 405258.033003445
				12739 405268.828793909
				12740 405279.624584374
				12741 405290.420374838
				12742 405301.216165303
				12743 405312.011955768
				12744 405322.807746232
				12745 405333.603536697
				12746 405344.399327161
				12747 405355.195117626
				12748 405365.99090809
				12749 405376.786698555
				12750 405387.58248902
				12751 405398.378279484
				12752 405409.174069949
				12753 405419.969860413
				12754 405430.765650878
				12755 405441.561441343
				12756 405452.357231807
				12757 405463.153022272
				12758 405473.948812736
				12759 405484.744603201
				12760 405495.540393665
				12761 405506.33618413
				12762 405517.131974595
				12763 405527.927765059
				12764 405538.723555524
				12765 405549.519345988
				12766 405560.315136453
				12767 405571.110926918
				12768 405581.906717382
				12769 405592.702507847
				12770 405603.498298311
				12771 405614.294088776
				12772 405625.08987924
				12773 405635.885669705
				12774 405646.68146017
				12775 405657.477250634
				12776 405668.273041099
				12777 405679.068831563
				12778 405689.864622028
				12779 405700.660412492
				12780 405711.456202957
				12781 405722.251993422
				12782 405733.047783886
				12783 405743.843574351
				12784 405754.639364815
				12785 405765.43515528
				12786 405776.230945745
				12787 405787.026736209
				12788 405797.822526674
				12789 405808.618317138
				12790 405819.414107603
				12791 405830.209898067
				12792 405841.005688532
				12793 405851.801478997
				12794 405862.597269461
				12795 405873.393059926
				12796 405884.18885039
				12797 405894.984640855
				12798 405905.780431319
				12799 405916.576221784
				12800 405927.372012249
				12801 405938.167802713
				12802 405948.963593178
				12803 405959.759383642
				12804 405970.555174107
				12805 405981.350964572
				12806 405992.146755036
				12807 406002.942545501
				12808 406013.738335965
				12809 406024.53412643
				12810 406035.329916894
				12811 406046.125707359
				12812 406056.921497824
				12813 406067.717288288
				12814 406078.513078753
				12815 406089.308869217
				12816 406100.104659682
				12817 406110.900450147
				12818 406121.696240611
				12819 406132.492031076
				12820 406143.28782154
				12821 406154.083612005
				12822 406164.879402469
				12823 406175.675192934
				12824 406186.470983399
				12825 406197.266773863
				12826 406208.062564328
				12827 406218.858354792
				12828 406229.654145257
				12829 406240.449935721
				12830 406251.245726186
				12831 406262.041516651
				12832 406272.837307115
				12833 406283.63309758
				12834 406294.428888044
				12835 406305.224678509
				12836 406316.020468973
				12837 406326.816259438
				12838 406337.612049903
				12839 406348.407840367
				12840 406359.203630832
				12841 406369.999421296
				12842 406380.795211761
				12843 406391.591002226
				12844 406402.38679269
				12845 406413.182583155
				12846 406423.978373619
				12847 406434.774164084
				12848 406445.569954548
				12849 406456.365745013
				12850 406467.161535478
				12851 406477.957325942
				12852 406488.753116407
				12853 406499.548906871
				12854 406510.344697336
				12855 406521.140487801
				12856 406531.936278265
				12857 406542.73206873
				12858 406553.527859194
				12859 406564.323649659
				12860 406575.119440123
				12861 406585.915230588
				12862 406596.711021053
				12863 406607.506811517
				12864 406618.302601982
				12865 406629.098392446
				12866 406639.894182911
				12867 406650.689973375
				12868 406661.48576384
				12869 406672.281554305
				12870 406683.077344769
				12871 406693.873135234
				12872 406704.668925698
				12873 406715.464716163
				12874 406726.260506627
				12875 406737.056297092
				12876 406747.852087557
				12877 406758.647878021
				12878 406769.443668486
				12879 406780.23945895
				12880 406791.035249415
				12881 406801.83103988
				12882 406812.626830344
				12883 406823.422620809
				12884 406834.218411273
				12885 406845.014201738
				12886 406855.809992202
				12887 406866.605782667
				12888 406877.401573132
				12889 406888.197363596
				12890 406898.993154061
				12891 406909.788944525
				12892 406920.58473499
				12893 406931.380525455
				12894 406942.176315919
				12895 406952.972106384
				12896 406963.767896848
				12897 406974.563687313
				12898 406985.359477777
				12899 406996.155268242
				12900 407006.951058707
				12901 407017.746849171
				12902 407028.542639636
				12903 407039.3384301
				12904 407050.134220565
				12905 407060.930011029
				12906 407071.725801494
				12907 407082.521591959
				12908 407093.317382423
				12909 407104.113172888
				12910 407114.908963352
				12911 407125.704753817
				12912 407136.500544282
				12913 407147.296334746
				12914 407158.092125211
				12915 407168.887915675
				12916 407179.68370614
				12917 407190.479496604
				12918 407201.275287069
				12919 407212.071077534
				12920 407222.866867998
				12921 407233.662658463
				12922 407244.458448927
				12923 407255.254239392
				12924 407266.050029857
				12925 407276.845820321
				12926 407287.641610786
				12927 407298.43740125
				12928 407309.233191715
				12929 407320.028982179
				12930 407330.824772644
				12931 407341.620563109
				12932 407352.416353573
				12933 407363.212144038
				12934 407374.007934502
				12935 407384.803724967
				12936 407395.599515431
				12937 407406.395305896
				12938 407417.191096361
				12939 407427.986886825
				12940 407438.78267729
				12941 407449.578467754
				12942 407460.374258219
				12943 407471.170048684
				12944 407481.965839148
				12945 407492.761629613
				12946 407503.557420077
				12947 407514.353210542
				12948 407525.149001006
				12949 407535.944791471
				12950 407546.740581936
				12951 407557.5363724
				12952 407568.332162865
				12953 407579.127953329
				12954 407589.923743794
				12955 407600.719534258
				12956 407611.515324723
				12957 407622.311115188
				12958 407633.106905652
				12959 407643.902696117
				12960 407654.698486581
				12961 407665.494277046
				12962 407676.290067511
				12963 407687.085857975
				12964 407697.88164844
				12965 407708.677438904
				12966 407719.473229369
				12967 407730.269019833
				12968 407741.064810298
				12969 407751.860600763
				12970 407762.656391227
				12971 407773.452181692
				12972 407784.247972156
				12973 407795.043762621
				12974 407805.839553086
				12975 407816.63534355
				12976 407827.431134015
				12977 407838.226924479
				12978 407849.022714944
				12979 407859.818505408
				12980 407870.614295873
				12981 407881.410086338
				12982 407892.205876802
				12983 407903.001667267
				12984 407913.797457731
				12985 407924.593248196
				12986 407935.38903866
				12987 407946.184829125
				12988 407956.98061959
				12989 407967.776410054
				12990 407978.572200519
				12991 407989.367990983
				12992 408000.163781448
				12993 408010.959571913
				12994 408021.755362377
				12995 408032.551152842
				12996 408043.346943306
				12997 408054.142733771
				12998 408064.938524235
				12999 408075.7343147
				13000 408086.530105165
				13001 408097.325895629
				13002 408108.121686094
				13003 408118.917476558
				13004 408129.713267023
				13005 408140.509057487
				13006 408151.304847952
				13007 408162.100638417
				13008 408172.896428881
				13009 408183.692219346
				13010 408194.48800981
				13011 408205.283800275
				13012 408216.07959074
				13013 408226.875381204
				13014 408237.671171669
				13015 408248.466962133
				13016 408259.262752598
				13017 408270.058543062
				13018 408280.854333527
				13019 408291.650123992
				13020 408302.445914456
				13021 408313.241704921
				13022 408324.037495385
				13023 408334.83328585
				13024 408345.629076315
				13025 408356.424866779
				13026 408367.220657244
				13027 408378.016447708
				13028 408388.812238173
				13029 408399.608028637
				13030 408410.403819102
				13031 408421.199609567
				13032 408431.995400031
				13033 408442.791190496
				13034 408453.58698096
				13035 408464.382771425
				13036 408475.178561889
				13037 408485.974352354
				13038 408496.770142819
				13039 408507.565933283
				13040 408518.361723748
				13041 408529.157514212
				13042 408539.953304677
				13043 408550.749095142
				13044 408561.544885606
				13045 408572.340676071
				13046 408583.136466535
				13047 408593.932257
				13048 408604.728047464
				13049 408615.523837929
				13050 408626.319628394
				13051 408637.115418858
				13052 408647.911209323
				13053 408658.706999787
				13054 408669.502790252
				13055 408680.298580716
				13056 408691.094371181
				13057 408701.890161646
				13058 408712.68595211
				13059 408723.481742575
				13060 408734.277533039
				13061 408745.073323504
				13062 408755.869113969
				13063 408766.664904433
				13064 408777.460694898
				13065 408788.256485362
				13066 408799.052275827
				13067 408809.848066291
				13068 408820.643856756
				13069 408831.439647221
				13070 408842.235437685
				13071 408853.03122815
				13072 408863.827018614
				13073 408874.622809079
				13074 408885.418599544
				13075 408896.214390008
				13076 408907.010180473
				13077 408917.805970937
				13078 408928.601761402
				13079 408939.397551866
				13080 408950.193342331
				13081 408960.989132796
				13082 408971.78492326
				13083 408982.580713725
				13084 408993.376504189
				13085 409004.172294654
				13086 409014.968085118
				13087 409025.763875583
				13088 409036.559666048
				13089 409047.355456512
				13090 409058.151246977
				13091 409068.947037441
				13092 409079.742827906
				13093 409090.538618371
				13094 409101.334408835
				13095 409112.1301993
				13096 409122.925989764
				13097 409133.721780229
				13098 409144.517570693
				13099 409155.313361158
				13100 409166.109151623
				13101 409176.904942087
				13102 409187.700732552
				13103 409198.496523016
				13104 409209.292313481
				13105 409220.088103945
				13106 409230.88389441
				13107 409241.679684875
				13108 409252.475475339
				13109 409263.271265804
				13110 409274.067056268
				13111 409284.862846733
				13112 409295.658637198
				13113 409306.454427662
				13114 409317.250218127
				13115 409328.046008591
				13116 409338.841799056
				13117 409349.63758952
				13118 409360.433379985
				13119 409371.22917045
				13120 409382.024960914
				13121 409392.820751379
				13122 409403.616541843
				13123 409414.412332308
				13124 409425.208122773
				13125 409436.003913237
				13126 409446.799703702
				13127 409457.595494166
				13128 409468.391284631
				13129 409479.187075095
				13130 409489.98286556
				13131 409500.778656025
				13132 409511.574446489
				13133 409522.370236954
				13134 409533.166027418
				13135 409543.961817883
				13136 409554.757608347
				13137 409565.553398812
				13138 409576.349189277
				13139 409587.144979741
				13140 409597.940770206
				13141 409608.73656067
				13142 409619.532351135
				13143 409630.3281416
				13144 409641.123932064
				13145 409651.919722529
				13146 409662.715512993
				13147 409673.511303458
				13148 409684.307093922
				13149 409695.102884387
				13150 409705.898674852
				13151 409716.694465316
				13152 409727.490255781
				13153 409738.286046245
				13154 409749.08183671
				13155 409759.877627174
				13156 409770.673417639
				13157 409781.469208104
				13158 409792.264998568
				13159 409803.060789033
				13160 409813.856579497
				13161 409824.652369962
				13162 409835.448160427
				13163 409846.243950891
				13164 409857.039741356
				13165 409867.83553182
				13166 409878.631322285
				13167 409889.427112749
				13168 409900.222903214
				13169 409911.018693679
				13170 409921.814484143
				13171 409932.610274608
				13172 409943.406065072
				13173 409954.201855537
				13174 409964.997646002
				13175 409975.793436466
				13176 409986.589226931
				13177 409997.385017395
				13178 410008.18080786
				13179 410018.976598324
				13180 410029.772388789
				13181 410040.568179254
				13182 410051.363969718
				13183 410062.159760183
				13184 410072.955550647
				13185 410083.751341112
				13186 410094.547131576
				13187 410105.342922041
				13188 410116.138712506
				13189 410126.93450297
				13190 410137.730293435
				13191 410148.526083899
				13192 410159.321874364
				13193 410170.117664828
				13194 410180.913455293
				13195 410191.709245758
				13196 410202.505036222
				13197 410213.300826687
				13198 410224.096617151
				13199 410234.892407616
				13200 410245.688198081
				13201 410256.483988545
				13202 410267.27977901
				13203 410278.075569474
				13204 410288.871359939
				13205 410299.667150403
				13206 410310.462940868
				13207 410321.258731333
				13208 410332.054521797
				13209 410342.850312262
				13210 410353.646102726
				13211 410364.441893191
				13212 410375.237683656
				13213 410386.03347412
				13214 410396.829264585
				13215 410407.625055049
				13216 410418.420845514
				13217 410429.216635978
				13218 410440.012426443
				13219 410450.808216908
				13220 410461.604007372
				13221 410472.399797837
				13222 410483.195588301
				13223 410493.991378766
				13224 410504.78716923
				13225 410515.582959695
				13226 410526.37875016
				13227 410537.174540624
				13228 410547.970331089
				13229 410558.766121553
				13230 410569.561912018
				13231 410580.357702482
				13232 410591.153492947
				13233 410601.949283412
				13234 410612.745073876
				13235 410623.540864341
				13236 410634.336654805
				13237 410645.13244527
				13238 410655.928235735
				13239 410666.724026199
				13240 410677.519816664
				13241 410688.315607128
				13242 410699.111397593
				13243 410709.907188057
				13244 410720.702978522
				13245 410731.498768987
				13246 410742.294559451
				13247 410753.090349916
				13248 410763.88614038
				13249 410774.681930845
				13250 410785.47772131
				13251 410796.273511774
				13252 410807.069302239
				13253 410817.865092703
				13254 410828.660883168
				13255 410839.456673632
				13256 410850.252464097
				13257 410861.048254562
				13258 410871.844045026
				13259 410882.639835491
				13260 410893.435625955
				13261 410904.23141642
				13262 410915.027206884
				13263 410925.822997349
				13264 410936.618787814
				13265 410947.414578278
				13266 410958.210368743
				13267 410969.006159207
				13268 410979.801949672
				13269 410990.597740137
				13270 411001.393530601
				13271 411012.189321066
				13272 411022.98511153
				13273 411033.780901995
				13274 411044.576692459
				13275 411055.372482924
				13276 411066.168273389
				13277 411076.964063853
				13278 411087.759854318
				13279 411098.555644782
				13280 411109.351435247
				13281 411120.147225711
				13282 411130.943016176
				13283 411141.738806641
				13284 411152.534597105
				13285 411163.33038757
				13286 411174.126178034
				13287 411184.921968499
				13288 411195.717758964
				13289 411206.513549428
				13290 411217.309339893
				13291 411228.105130357
				13292 411238.900920822
				13293 411249.696711286
				13294 411260.492501751
				13295 411271.288292216
				13296 411282.08408268
				13297 411292.879873145
				13298 411303.675663609
				13299 411314.471454074
				13300 411325.267244539
				13301 411336.063035003
				13302 411346.858825468
				13303 411357.654615932
				13304 411368.450406397
				13305 411379.246196861
				13306 411390.041987326
				13307 411400.837777791
				13308 411411.633568255
				13309 411422.42935872
				13310 411433.225149184
				13311 411444.020939649
				13312 411454.816730113
				13313 411465.612520578
				13314 411476.408311043
				13315 411487.204101507
				13316 411497.999891972
				13317 411508.795682436
				13318 411519.591472901
				13319 411530.387263366
				13320 411541.18305383
				13321 411551.978844295
				13322 411562.774634759
				13323 411573.570425224
				13324 411584.366215688
				13325 411595.162006153
				13326 411605.957796618
				13327 411616.753587082
				13328 411627.549377547
				13329 411638.345168011
				13330 411649.140958476
				13331 411659.93674894
				13332 411670.732539405
				13333 411681.52832987
				13334 411692.324120334
				13335 411703.119910799
				13336 411713.915701263
				13337 411724.711491728
				13338 411735.507282193
				13339 411746.303072657
				13340 411757.098863122
				13341 411767.894653586
				13342 411778.690444051
				13343 411789.486234515
				13344 411800.28202498
				13345 411811.077815445
				13346 411821.873605909
				13347 411832.669396374
				13348 411843.465186838
				13349 411854.260977303
				13350 411865.056767768
				13351 411875.852558232
				13352 411886.648348697
				13353 411897.444139161
				13354 411908.239929626
				13355 411919.03572009
				13356 411929.831510555
				13357 411940.62730102
				13358 411951.423091484
				13359 411962.218881949
				13360 411973.014672413
				13361 411983.810462878
				13362 411994.606253342
				13363 412005.402043807
				13364 412016.197834272
				13365 412026.993624736
				13366 412037.789415201
				13367 412048.585205665
				13368 412059.38099613
				13369 412070.176786595
				13370 412080.972577059
				13371 412091.768367524
				13372 412102.564157988
				13373 412113.359948453
				13374 412124.155738917
				13375 412134.951529382
				13376 412145.747319847
				13377 412156.543110311
				13378 412167.338900776
				13379 412178.13469124
				13380 412188.930481705
				13381 412199.72627217
				13382 412210.522062634
				13383 412221.317853099
				13384 412232.113643563
				13385 412242.909434028
				13386 412253.705224492
				13387 412264.501014957
				13388 412275.296805422
				13389 412286.092595886
				13390 412296.888386351
				13391 412307.684176815
				13392 412318.47996728
				13393 412329.275757744
				13394 412340.071548209
				13395 412350.867338674
				13396 412361.663129138
				13397 412372.458919603
				13398 412383.254710067
				13399 412394.050500532
				13400 412404.846290997
				13401 412415.642081461
				13402 412426.437871926
				13403 412437.23366239
				13404 412448.029452855
				13405 412458.825243319
				13406 412469.621033784
				13407 412480.416824249
				13408 412491.212614713
				13409 412502.008405178
				13410 412512.804195642
				13411 412523.599986107
				13412 412534.395776571
				13413 412545.191567036
				13414 412555.987357501
				13415 412566.783147965
				13416 412577.57893843
				13417 412588.374728894
				13418 412599.170519359
				13419 412609.966309824
				13420 412620.762100288
				13421 412631.557890753
				13422 412642.353681217
				13423 412653.149471682
				13424 412663.945262146
				13425 412674.741052611
				13426 412685.536843076
				13427 412696.33263354
				13428 412707.128424005
				13429 412717.924214469
				13430 412728.720004934
				13431 412739.515795399
				13432 412750.311585863
				13433 412761.107376328
				13434 412771.903166792
				13435 412782.698957257
				13436 412793.494747721
				13437 412804.290538186
				13438 412815.086328651
				13439 412825.882119115
				13440 412836.67790958
				13441 412847.473700044
				13442 412858.269490509
				13443 412869.065280973
				13444 412879.861071438
				13445 412890.656861903
				13446 412901.452652367
				13447 412912.248442832
				13448 412923.044233296
				13449 412933.840023761
				13450 412944.635814226
				13451 412955.43160469
				13452 412966.227395155
				13453 412977.023185619
				13454 412987.818976084
				13455 412998.614766548
				13456 413009.410557013
				13457 413020.206347478
				13458 413031.002137942
				13459 413041.797928407
				13460 413052.593718871
				13461 413063.389509336
				13462 413074.1852998
				13463 413084.981090265
				13464 413095.77688073
				13465 413106.572671194
				13466 413117.368461659
				13467 413128.164252123
				13468 413138.960042588
				13469 413149.755833053
				13470 413160.551623517
				13471 413171.347413982
				13472 413182.143204446
				13473 413192.938994911
				13474 413203.734785375
				13475 413214.53057584
				13476 413225.326366305
				13477 413236.122156769
				13478 413246.917947234
				13479 413257.713737698
				13480 413268.509528163
				13481 413279.305318628
				13482 413290.101109092
				13483 413300.896899557
				13484 413311.692690021
				13485 413322.488480486
				13486 413333.28427095
				13487 413344.080061415
				13488 413354.87585188
				13489 413365.671642344
				13490 413376.467432809
				13491 413387.263223273
				13492 413398.059013738
				13493 413408.854804202
				13494 413419.650594667
				13495 413430.446385132
				13496 413441.242175596
				13497 413452.037966061
				13498 413462.833756525
				13499 413473.62954699
				13500 413484.425337455
				13501 413495.221127919
				13502 413506.016918384
				13503 413516.812708848
				13504 413527.608499313
				13505 413538.404289777
				13506 413549.200080242
				13507 413559.995870707
				13508 413570.791661171
				13509 413581.587451636
				13510 413592.3832421
				13511 413603.179032565
				13512 413613.974823029
				13513 413624.770613494
				13514 413635.566403959
				13515 413646.362194423
				13516 413657.157984888
				13517 413667.953775352
				13518 413678.749565817
				13519 413689.545356282
				13520 413700.341146746
				13521 413711.136937211
				13522 413721.932727675
				13523 413732.72851814
				13524 413743.524308604
				13525 413754.320099069
				13526 413765.115889534
				13527 413775.911679998
				13528 413786.707470463
				13529 413797.503260927
				13530 413808.299051392
				13531 413819.094841857
				13532 413829.890632321
				13533 413840.686422786
				13534 413851.48221325
				13535 413862.278003715
				13536 413873.073794179
				13537 413883.869584644
				13538 413894.665375109
				13539 413905.461165573
				13540 413916.256956038
				13541 413927.052746502
				13542 413937.848536967
				13543 413948.644327431
				13544 413959.440117896
				13545 413970.235908361
				13546 413981.031698825
				13547 413991.82748929
				13548 414002.623279754
				13549 414013.419070219
				13550 414024.214860683
				13551 414035.010651148
				13552 414045.806441613
				13553 414056.602232077
				13554 414067.398022542
				13555 414078.193813006
				13556 414088.989603471
				13557 414099.785393936
				13558 414110.5811844
				13559 414121.376974865
				13560 414132.172765329
				13561 414142.968555794
				13562 414153.764346258
				13563 414164.560136723
				13564 414175.355927188
				13565 414186.151717652
				13566 414196.947508117
				13567 414207.743298581
				13568 414218.539089046
				13569 414229.334879511
				13570 414240.130669975
				13571 414250.92646044
				13572 414261.722250904
				13573 414272.518041369
				13574 414283.313831833
				13575 414294.109622298
				13576 414304.905412763
				13577 414315.701203227
				13578 414326.496993692
				13579 414337.292784156
				13580 414348.088574621
				13581 414358.884365085
				13582 414369.68015555
				13583 414380.475946015
				13584 414391.271736479
				13585 414402.067526944
				13586 414412.863317408
				13587 414423.659107873
				13588 414434.454898337
				13589 414445.250688802
				13590 414456.046479267
				13591 414466.842269731
				13592 414477.638060196
				13593 414488.43385066
				13594 414499.229641125
				13595 414510.02543159
				13596 414520.821222054
				13597 414531.617012519
				13598 414542.412802983
				13599 414553.208593448
				13600 414564.004383912
				13601 414574.800174377
				13602 414585.595964842
				13603 414596.391755306
				13604 414607.187545771
				13605 414617.983336235
				13606 414628.7791267
				13607 414639.574917165
				13608 414650.370707629
				13609 414661.166498094
				13610 414671.962288558
				13611 414682.758079023
				13612 414693.553869487
				13613 414704.349659952
				13614 414715.145450417
				13615 414725.941240881
				13616 414736.737031346
				13617 414747.53282181
				13618 414758.328612275
				13619 414769.124402739
				13620 414779.920193204
				13621 414790.715983669
				13622 414801.511774133
				13623 414812.307564598
				13624 414823.103355062
				13625 414833.899145527
				13626 414844.694935992
				13627 414855.490726456
				13628 414866.286516921
				13629 414877.082307385
				13630 414887.87809785
				13631 414898.673888314
				13632 414909.469678779
				13633 414920.265469244
				13634 414931.061259708
				13635 414941.857050173
				13636 414952.652840637
				13637 414963.448631102
				13638 414974.244421566
				13639 414985.040212031
				13640 414995.836002496
				13641 415006.63179296
				13642 415017.427583425
				13643 415028.223373889
				13644 415039.019164354
				13645 415049.814954819
				13646 415060.610745283
				13647 415071.406535748
				13648 415082.202326212
				13649 415092.998116677
				13650 415103.793907141
				13651 415114.589697606
				13652 415125.385488071
				13653 415136.181278535
				13654 415146.977069
				13655 415157.772859464
				13656 415168.568649929
				13657 415179.364440394
				13658 415190.160230858
				13659 415200.956021323
				13660 415211.751811787
				13661 415222.547602252
				13662 415233.343392716
				13663 415244.139183181
				13664 415254.934973646
				13665 415265.73076411
				13666 415276.526554575
				13667 415287.322345039
				13668 415298.118135504
				13669 415308.913925968
				13670 415319.709716433
				13671 415330.505506898
				13672 415341.301297362
				13673 415352.097087827
				13674 415362.892878291
				13675 415373.688668756
				13676 415384.484459221
				13677 415395.280249685
				13678 415406.07604015
				13679 415416.871830614
				13680 415427.667621079
				13681 415438.463411543
				13682 415449.259202008
				13683 415460.054992473
				13684 415470.850782937
				13685 415481.646573402
				13686 415492.442363866
				13687 415503.238154331
				13688 415514.033944795
				13689 415524.82973526
				13690 415535.625525725
				13691 415546.421316189
				13692 415557.217106654
				13693 415568.012897118
				13694 415578.808687583
				13695 415589.604478048
				13696 415600.400268512
				13697 415611.196058977
				13698 415621.991849441
				13699 415632.787639906
				13700 415643.58343037
				13701 415654.379220835
				13702 415665.1750113
				13703 415675.970801764
				13704 415686.766592229
				13705 415697.562382693
				13706 415708.358173158
				13707 415719.153963623
				13708 415729.949754087
				13709 415740.745544552
				13710 415751.541335016
				13711 415762.337125481
				13712 415773.132915945
				13713 415783.92870641
				13714 415794.724496875
				13715 415805.520287339
				13716 415816.316077804
				13717 415827.111868268
				13718 415837.907658733
				13719 415848.703449197
				13720 415859.499239662
				13721 415870.295030127
				13722 415881.090820591
				13723 415891.886611056
				13724 415902.68240152
				13725 415913.478191985
				13726 415924.27398245
				13727 415935.069772914
				13728 415945.865563379
				13729 415956.661353843
				13730 415967.457144308
				13731 415978.252934772
				13732 415989.048725237
				13733 415999.844515702
				13734 416010.640306166
				13735 416021.436096631
				13736 416032.231887095
				13737 416043.02767756
				13738 416053.823468024
				13739 416064.619258489
				13740 416075.415048954
				13741 416086.210839418
				13742 416097.006629883
				13743 416107.802420347
				13744 416118.598210812
				13745 416129.394001277
				13746 416140.189791741
				13747 416150.985582206
				13748 416161.78137267
				13749 416172.577163135
				13750 416183.372953599
				13751 416194.168744064
				13752 416204.964534529
				13753 416215.760324993
				13754 416226.556115458
				13755 416237.351905922
				13756 416248.147696387
				13757 416258.943486852
				13758 416269.739277316
				13759 416280.535067781
				13760 416291.330858245
				13761 416302.12664871
				13762 416312.922439174
				13763 416323.718229639
				13764 416334.514020104
				13765 416345.309810568
				13766 416356.105601033
				13767 416366.901391497
				13768 416377.697181962
				13769 416388.492972426
				13770 416399.288762891
				13771 416410.084553356
				13772 416420.88034382
				13773 416431.676134285
				13774 416442.471924749
				13775 416453.267715214
				13776 416464.063505679
				13777 416474.859296143
				13778 416485.655086608
				13779 416496.450877072
				13780 416507.246667537
				13781 416518.042458001
				13782 416528.838248466
				13783 416539.634038931
				13784 416550.429829395
				13785 416561.22561986
				13786 416572.021410324
				13787 416582.817200789
				13788 416593.612991253
				13789 416604.408781718
				13790 416615.204572183
				13791 416626.000362647
				13792 416636.796153112
				13793 416647.591943576
				13794 416658.387734041
				13795 416669.183524506
				13796 416679.97931497
				13797 416690.775105435
				13798 416701.570895899
				13799 416712.366686364
				13800 416723.162476828
				13801 416733.958267293
				13802 416744.754057758
				13803 416755.549848222
				13804 416766.345638687
				13805 416777.141429151
				13806 416787.937219616
				13807 416798.733010081
				13808 416809.528800545
				13809 416820.32459101
				13810 416831.120381474
				13811 416841.916171939
				13812 416852.711962403
				13813 416863.507752868
				13814 416874.303543333
				13815 416885.099333797
				13816 416895.895124262
				13817 416906.690914726
				13818 416917.486705191
				13819 416928.282495655
				13820 416939.07828612
				13821 416949.874076585
				13822 416960.669867049
				13823 416971.465657514
				13824 416982.261447978
				13825 416993.057238443
				13826 417003.853028907
				13827 417014.648819372
				13828 417025.444609837
				13829 417036.240400301
				13830 417047.036190766
				13831 417057.83198123
				13832 417068.627771695
				13833 417079.42356216
				13834 417090.219352624
				13835 417101.015143089
				13836 417111.810933553
				13837 417122.606724018
				13838 417133.402514482
				13839 417144.198304947
				13840 417154.994095412
				13841 417165.789885876
				13842 417176.585676341
				13843 417187.381466805
				13844 417198.17725727
				13845 417208.973047735
				13846 417219.768838199
				13847 417230.564628664
				13848 417241.360419128
				13849 417252.156209593
				13850 417262.952000057
				13851 417273.747790522
				13852 417284.543580987
				13853 417295.339371451
				13854 417306.135161916
				13855 417316.93095238
				13856 417327.726742845
				13857 417338.52253331
				13858 417349.318323774
				13859 417360.114114239
				13860 417370.909904703
				13861 417381.705695168
				13862 417392.501485632
				13863 417403.297276097
				13864 417414.093066562
				13865 417424.888857026
				13866 417435.684647491
				13867 417446.480437955
				13868 417457.27622842
				13869 417468.072018884
				13870 417478.867809349
				13871 417489.663599814
				13872 417500.459390278
				13873 417511.255180743
				13874 417522.050971207
				13875 417532.846761672
				13876 417543.642552137
				13877 417554.438342601
				13878 417565.234133066
				13879 417576.02992353
				13880 417586.825713995
				13881 417597.621504459
				13882 417608.417294924
				13883 417619.213085389
				13884 417630.008875853
				13885 417640.804666318
				13886 417651.600456782
				13887 417662.396247247
				13888 417673.192037712
				13889 417683.987828176
				13890 417694.783618641
				13891 417705.579409105
				13892 417716.37519957
				13893 417727.170990034
				13894 417737.966780499
				13895 417748.762570964
				13896 417759.558361428
				13897 417770.354151893
				13898 417781.149942357
				13899 417791.945732822
				13900 417802.741523286
				13901 417813.537313751
				13902 417824.333104216
				13903 417835.12889468
				13904 417845.924685145
				13905 417856.720475609
				13906 417867.516266074
				13907 417878.312056538
				13908 417889.107847003
				13909 417899.903637468
				13910 417910.699427932
				13911 417921.495218397
				13912 417932.291008861
				13913 417943.086799326
				13914 417953.882589791
				13915 417964.678380255
				13916 417975.47417072
				13917 417986.269961184
				13918 417997.065751649
				13919 418007.861542113
				13920 418018.657332578
				13921 418029.453123043
				13922 418040.248913507
				13923 418051.044703972
				13924 418061.840494436
				13925 418072.636284901
				13926 418083.432075366
				13927 418094.22786583
				13928 418105.023656295
				13929 418115.819446759
				13930 418126.615237224
				13931 418137.411027688
				13932 418148.206818153
				13933 418159.002608618
				13934 418169.798399082
				13935 418180.594189547
				13936 418191.389980011
				13937 418202.185770476
				13938 418212.98156094
				13939 418223.777351405
				13940 418234.57314187
				13941 418245.368932334
				13942 418256.164722799
				13943 418266.960513263
				13944 418277.756303728
				13945 418288.552094192
				13946 418299.347884657
				13947 418310.143675122
				13948 418320.939465586
				13949 418331.735256051
				13950 418342.531046515
				13951 418353.32683698
				13952 418364.122627445
				13953 418374.918417909
				13954 418385.714208374
				13955 418396.509998838
				13956 418407.305789303
				13957 418418.101579767
				13958 418428.897370232
				13959 418439.693160697
				13960 418450.488951161
				13961 418461.284741626
				13962 418472.08053209
				13963 418482.876322555
				13964 418493.67211302
				13965 418504.467903484
				13966 418515.263693949
				13967 418526.059484413
				13968 418536.855274878
				13969 418547.651065342
				13970 418558.446855807
				13971 418569.242646272
				13972 418580.038436736
				13973 418590.834227201
				13974 418601.630017665
				13975 418612.42580813
				13976 418623.221598594
				13977 418634.017389059
				13978 418644.813179524
				13979 418655.608969988
				13980 418666.404760453
				13981 418677.200550917
				13982 418687.996341382
				13983 418698.792131847
				13984 418709.587922311
				13985 418720.383712776
				13986 418731.17950324
				13987 418741.975293705
				13988 418752.771084169
				13989 418763.566874634
				13990 418774.362665099
				13991 418785.158455563
				13992 418795.954246028
				13993 418806.750036492
				13994 418817.545826957
				13995 418828.341617421
				13996 418839.137407886
				13997 418849.933198351
				13998 418860.728988815
				13999 418871.52477928
				14000 418882.320569744
				14001 418893.116360209
				14002 418903.912150674
				14003 418914.707941138
				14004 418925.503731603
				14005 418936.299522067
				14006 418947.095312532
				14007 418957.891102996
				14008 418968.686893461
				14009 418979.482683926
				14010 418990.27847439
				14011 419001.074264855
				14012 419011.870055319
				14013 419022.665845784
				14014 419033.461636249
				14015 419044.257426713
				14016 419055.053217178
				14017 419065.849007642
				14018 419076.644798107
				14019 419087.440588571
				14020 419098.236379036
				14021 419109.032169501
				14022 419119.827959965
				14023 419130.62375043
				14024 419141.419540894
				14025 419152.215331359
				14026 419163.011121823
				14027 419173.806912288
				14028 419184.602702753
				14029 419195.398493217
				14030 419206.194283682
				14031 419216.990074146
				14032 419227.785864611
				14033 419238.581655076
				14034 419249.37744554
				14035 419260.173236005
				14036 419270.969026469
				14037 419281.764816934
				14038 419292.560607398
				14039 419303.356397863
				14040 419314.152188328
				14041 419324.947978792
				14042 419335.743769257
				14043 419346.539559721
				14044 419357.335350186
				14045 419368.13114065
				14046 419378.926931115
				14047 419389.72272158
				14048 419400.518512044
				14049 419411.314302509
				14050 419422.110092973
				14051 419432.905883438
				14052 419443.701673903
				14053 419454.497464367
				14054 419465.293254832
				14055 419476.089045296
				14056 419486.884835761
				14057 419497.680626225
				14058 419508.47641669
				14059 419519.272207155
				14060 419530.067997619
				14061 419540.863788084
				14062 419551.659578548
				14063 419562.455369013
				14064 419573.251159478
				14065 419584.046949942
				14066 419594.842740407
				14067 419605.638530871
				14068 419616.434321336
				14069 419627.2301118
				14070 419638.025902265
				14071 419648.82169273
				14072 419659.617483194
				14073 419670.413273659
				14074 419681.209064123
				14075 419692.004854588
				14076 419702.800645052
				14077 419713.596435517
				14078 419724.392225982
				14079 419735.188016446
				14080 419745.983806911
				14081 419756.779597375
				14082 419767.57538784
				14083 419778.371178305
				14084 419789.166968769
				14085 419799.962759234
				14086 419810.758549698
				14087 419821.554340163
				14088 419832.350130627
				14089 419843.145921092
				14090 419853.941711557
				14091 419864.737502021
				14092 419875.533292486
				14093 419886.32908295
				14094 419897.124873415
				14095 419907.920663879
				14096 419918.716454344
				14097 419929.512244809
				14098 419940.308035273
				14099 419951.103825738
				14100 419961.899616202
				14101 419972.695406667
				14102 419983.491197132
				14103 419994.286987596
				14104 420005.082778061
				14105 420015.878568525
				14106 420026.67435899
				14107 420037.470149454
				14108 420048.265939919
				14109 420059.061730384
				14110 420069.857520848
				14111 420080.653311313
				14112 420091.449101777
				14113 420102.244892242
				14114 420113.040682707
				14115 420123.836473171
				14116 420134.632263636
				14117 420145.4280541
				14118 420156.223844565
				14119 420167.019635029
				14120 420177.815425494
				14121 420188.611215959
				14122 420199.407006423
				14123 420210.202796888
				14124 420220.998587352
				14125 420231.794377817
				14126 420242.590168281
				14127 420253.385958746
				14128 420264.181749211
				14129 420274.977539675
				14130 420285.77333014
				14131 420296.569120604
				14132 420307.364911069
				14133 420318.160701534
				14134 420328.956491998
				14135 420339.752282463
				14136 420350.548072927
				14137 420361.343863392
				14138 420372.139653856
				14139 420382.935444321
				14140 420393.731234786
				14141 420404.52702525
				14142 420415.322815715
				14143 420426.118606179
				14144 420436.914396644
				14145 420447.710187108
				14146 420458.505977573
				14147 420469.301768038
				14148 420480.097558502
				14149 420490.893348967
				14150 420501.689139431
				14151 420512.484929896
				14152 420523.280720361
				14153 420534.076510825
				14154 420544.87230129
				14155 420555.668091754
				14156 420566.463882219
				14157 420577.259672683
				14158 420588.055463148
				14159 420598.851253613
				14160 420609.647044077
				14161 420620.442834542
				14162 420631.238625006
				14163 420642.034415471
				14164 420652.830205936
				14165 420663.6259964
				14166 420674.421786865
				14167 420685.217577329
				14168 420696.013367794
				14169 420706.809158258
				14170 420717.604948723
				14171 420728.400739188
				14172 420739.196529652
				14173 420749.992320117
				14174 420760.788110581
				14175 420771.583901046
				14176 420782.37969151
				14177 420793.175481975
				14178 420803.97127244
				14179 420814.767062904
				14180 420825.562853369
				14181 420836.358643833
				14182 420847.154434298
				14183 420857.950224762
				14184 420868.746015227
				14185 420879.541805692
				14186 420890.337596156
				14187 420901.133386621
				14188 420911.929177085
				14189 420922.72496755
				14190 420933.520758015
				14191 420944.316548479
				14192 420955.112338944
				14193 420965.908129408
				14194 420976.703919873
				14195 420987.499710337
				14196 420998.295500802
				14197 421009.091291267
				14198 421019.887081731
				14199 421030.682872196
				14200 421041.47866266
				14201 421052.274453125
				14202 421063.07024359
				14203 421073.866034054
				14204 421084.661824519
				14205 421095.457614983
				14206 421106.253405448
				14207 421117.049195912
				14208 421127.844986377
				14209 421138.640776842
				14210 421149.436567306
				14211 421160.232357771
				14212 421171.028148235
				14213 421181.8239387
				14214 421192.619729164
				14215 421203.415519629
				14216 421214.211310094
				14217 421225.007100558
				14218 421235.802891023
				14219 421246.598681487
				14220 421257.394471952
				14221 421268.190262416
				14222 421278.986052881
				14223 421289.781843346
				14224 421300.57763381
				14225 421311.373424275
				14226 421322.169214739
				14227 421332.965005204
				14228 421343.760795669
				14229 421354.556586133
				14230 421365.352376598
				14231 421376.148167062
				14232 421386.943957527
				14233 421397.739747991
				14234 421408.535538456
				14235 421419.331328921
				14236 421430.127119385
				14237 421440.92290985
				14238 421451.718700314
				14239 421462.514490779
				14240 421473.310281244
				14241 421484.106071708
				14242 421494.901862173
				14243 421505.697652637
				14244 421516.493443102
				14245 421527.289233566
				14246 421538.085024031
				14247 421548.880814496
				14248 421559.67660496
				14249 421570.472395425
				14250 421581.268185889
				14251 421592.063976354
				14252 421602.859766819
				14253 421613.655557283
				14254 421624.451347748
				14255 421635.247138212
				14256 421646.042928677
				14257 421656.838719141
				14258 421667.634509606
				14259 421678.430300071
				14260 421689.226090535
				14261 421700.021881
				14262 421710.817671464
				14263 421721.613461929
				14264 421732.409252393
				14265 421743.205042858
				14266 421754.000833323
				14267 421764.796623787
				14268 421775.592414252
				14269 421786.388204716
				14270 421797.183995181
				14271 421807.979785645
				14272 421818.77557611
				14273 421829.571366575
				14274 421840.367157039
				14275 421851.162947504
				14276 421861.958737968
				14277 421872.754528433
				14278 421883.550318898
				14279 421894.346109362
				14280 421905.141899827
				14281 421915.937690291
				14282 421926.733480756
				14283 421937.52927122
				14284 421948.325061685
				14285 421959.12085215
				14286 421969.916642614
				14287 421980.712433079
				14288 421991.508223543
				14289 422002.304014008
				14290 422013.099804473
				14291 422023.895594937
				14292 422034.691385402
				14293 422045.487175866
				14294 422056.282966331
				14295 422067.078756795
				14296 422077.87454726
				14297 422088.670337725
				14298 422099.466128189
				14299 422110.261918654
				14300 422121.057709118
				14301 422131.853499583
				14302 422142.649290047
				14303 422153.445080512
				14304 422164.240870977
				14305 422175.036661441
				14306 422185.832451906
				14307 422196.62824237
				14308 422207.424032835
				14309 422218.2198233
				14310 422229.015613764
				14311 422239.811404229
				14312 422250.607194693
				14313 422261.402985158
				14314 422272.198775622
				14315 422282.994566087
				14316 422293.790356552
				14317 422304.586147016
				14318 422315.381937481
				14319 422326.177727945
				14320 422336.97351841
				14321 422347.769308874
				14322 422358.565099339
				14323 422369.360889804
				14324 422380.156680268
				14325 422390.952470733
				14326 422401.748261197
				14327 422412.544051662
				14328 422423.339842127
				14329 422434.135632591
				14330 422444.931423056
				14331 422455.72721352
				14332 422466.523003985
				14333 422477.318794449
				14334 422488.114584914
				14335 422498.910375379
				14336 422509.706165843
				14337 422520.501956308
				14338 422531.297746772
				14339 422542.093537237
				14340 422552.889327702
				14341 422563.685118166
				14342 422574.480908631
				14343 422585.276699095
				14344 422596.07248956
				14345 422606.868280024
				14346 422617.664070489
				14347 422628.459860954
				14348 422639.255651418
				14349 422650.051441883
				14350 422660.847232347
				14351 422671.643022812
				14352 422682.438813276
				14353 422693.234603741
				14354 422704.030394206
				14355 422714.82618467
				14356 422725.621975135
				14357 422736.417765599
				14358 422747.213556064
				14359 422758.009346529
				14360 422768.805136993
				14361 422779.600927458
				14362 422790.396717922
				14363 422801.192508387
				14364 422811.988298851
				14365 422822.784089316
				14366 422833.579879781
				14367 422844.375670245
				14368 422855.17146071
				14369 422865.967251174
				14370 422876.763041639
				14371 422887.558832104
				14372 422898.354622568
				14373 422909.150413033
				14374 422919.946203497
				14375 422930.741993962
				14376 422941.537784426
				14377 422952.333574891
				14378 422963.129365356
				14379 422973.92515582
				14380 422984.720946285
				14381 422995.516736749
				14382 423006.312527214
				14383 423017.108317678
				14384 423027.904108143
				14385 423038.699898608
				14386 423049.495689072
				14387 423060.291479537
				14388 423071.087270001
				14389 423081.883060466
				14390 423092.678850931
				14391 423103.474641395
				14392 423114.27043186
				14393 423125.066222324
				14394 423135.862012789
				14395 423146.657803253
				14396 423157.453593718
				14397 423168.249384183
				14398 423179.045174647
				14399 423189.840965112
				14400 423200.636755576
				14401 423211.432546041
				14402 423222.228336505
				14403 423233.02412697
				14404 423243.819917435
				14405 423254.615707899
				14406 423265.411498364
				14407 423276.207288828
				14408 423287.003079293
				14409 423297.798869758
				14410 423308.594660222
				14411 423319.390450687
				14412 423330.186241151
				14413 423340.982031616
				14414 423351.77782208
				14415 423362.573612545
				14416 423373.36940301
				14417 423384.165193474
				14418 423394.960983939
				14419 423405.756774403
				14420 423416.552564868
				14421 423427.348355333
				14422 423438.144145797
				14423 423448.939936262
				14424 423459.735726726
				14425 423470.531517191
				14426 423481.327307655
				14427 423492.12309812
				14428 423502.918888585
				14429 423513.714679049
				14430 423524.510469514
				14431 423535.306259978
				14432 423546.102050443
				14433 423556.897840907
				14434 423567.693631372
				14435 423578.489421837
				14436 423589.285212301
				14437 423600.081002766
				14438 423610.87679323
				14439 423621.672583695
				14440 423632.46837416
				14441 423643.264164624
				14442 423654.059955089
				14443 423664.855745553
				14444 423675.651536018
				14445 423686.447326482
				14446 423697.243116947
				14447 423708.038907412
				14448 423718.834697876
				14449 423729.630488341
				14450 423740.426278805
				14451 423751.22206927
				14452 423762.017859734
				14453 423772.813650199
				14454 423783.609440664
				14455 423794.405231128
				14456 423805.201021593
				14457 423815.996812057
				14458 423826.792602522
				14459 423837.588392987
				14460 423848.384183451
				14461 423859.179973916
				14462 423869.97576438
				14463 423880.771554845
				14464 423891.567345309
				14465 423902.363135774
				14466 423913.158926239
				14467 423923.954716703
				14468 423934.750507168
				14469 423945.546297632
				14470 423956.342088097
				14471 423967.137878562
				14472 423977.933669026
				14473 423988.729459491
				14474 423999.525249955
				14475 424010.32104042
				14476 424021.116830884
				14477 424031.912621349
				14478 424042.708411814
				14479 424053.504202278
				14480 424064.299992743
				14481 424075.095783207
				14482 424085.891573672
				14483 424096.687364136
				14484 424107.483154601
				14485 424118.278945066
				14486 424129.07473553
				14487 424139.870525995
				14488 424150.666316459
				14489 424161.462106924
				14490 424172.257897389
				14491 424183.053687853
				14492 424193.849478318
				14493 424204.645268782
				14494 424215.441059247
				14495 424226.236849711
				14496 424237.032640176
				14497 424247.828430641
				14498 424258.624221105
				14499 424269.42001157
				14500 424280.215802034
				14501 424291.011592499
				14502 424301.807382963
				14503 424312.603173428
				14504 424323.398963893
				14505 424334.194754357
				14506 424344.990544822
				14507 424355.786335286
				14508 424366.582125751
				14509 424377.377916216
				14510 424388.17370668
				14511 424398.969497145
				14512 424409.765287609
				14513 424420.561078074
				14514 424431.356868538
				14515 424442.152659003
				14516 424452.948449468
				14517 424463.744239932
				14518 424474.540030397
				14519 424485.335820861
				14520 424496.131611326
				14521 424506.927401791
				14522 424517.723192255
				14523 424528.51898272
				14524 424539.314773184
				14525 424550.110563649
				14526 424560.906354113
				14527 424571.702144578
				14528 424582.497935043
				14529 424593.293725507
				14530 424604.089515972
				14531 424614.885306436
				14532 424625.681096901
				14533 424636.476887365
				14534 424647.27267783
				14535 424658.068468295
				14536 424668.864258759
				14537 424679.660049224
				14538 424690.455839688
				14539 424701.251630153
				14540 424712.047420617
				14541 424722.843211082
				14542 424733.639001547
				14543 424744.434792011
				14544 424755.230582476
				14545 424766.02637294
				14546 424776.822163405
				14547 424787.61795387
				14548 424798.413744334
				14549 424809.209534799
				14550 424820.005325263
				14551 424830.801115728
				14552 424841.596906192
				14553 424852.392696657
				14554 424863.188487122
				14555 424873.984277586
				14556 424884.780068051
				14557 424895.575858515
				14558 424906.37164898
				14559 424917.167439445
				14560 424927.963229909
				14561 424938.759020374
				14562 424949.554810838
				14563 424960.350601303
				14564 424971.146391767
				14565 424981.942182232
				14566 424992.737972697
				14567 425003.533763161
				14568 425014.329553626
				14569 425025.12534409
				14570 425035.921134555
				14571 425046.716925019
				14572 425057.512715484
				14573 425068.308505949
				14574 425079.104296413
				14575 425089.900086878
				14576 425100.695877342
				14577 425111.491667807
				14578 425122.287458271
				14579 425133.083248736
				14580 425143.879039201
				14581 425154.674829665
				14582 425165.47062013
				14583 425176.266410594
				14584 425187.062201059
				14585 425197.857991524
				14586 425208.653781988
				14587 425219.449572453
				14588 425230.245362917
				14589 425241.041153382
				14590 425251.836943846
				14591 425262.632734311
				14592 425273.428524776
				14593 425284.22431524
				14594 425295.020105705
				14595 425305.815896169
				14596 425316.611686634
				14597 425327.407477099
				14598 425338.203267563
				14599 425348.999058028
				14600 425359.794848492
				14601 425370.590638957
				14602 425381.386429421
				14603 425392.182219886
				14604 425402.978010351
				14605 425413.773800815
				14606 425424.56959128
				14607 425435.365381744
				14608 425446.161172209
				14609 425456.956962673
				14610 425467.752753138
				14611 425478.548543603
				14612 425489.344334067
				14613 425500.140124532
				14614 425510.935914996
				14615 425521.731705461
				14616 425532.527495926
				14617 425543.32328639
				14618 425554.119076855
				14619 425564.914867319
				14620 425575.710657784
				14621 425586.506448248
				14622 425597.302238713
				14623 425608.098029178
				14624 425618.893819642
				14625 425629.689610107
				14626 425640.485400571
				14627 425651.281191036
				14628 425662.0769815
				14629 425672.872771965
				14630 425683.66856243
				14631 425694.464352894
				14632 425705.260143359
				14633 425716.055933823
				14634 425726.851724288
				14635 425737.647514753
				14636 425748.443305217
				14637 425759.239095682
				14638 425770.034886146
				14639 425780.830676611
				14640 425791.626467075
				14641 425802.42225754
				14642 425813.218048005
				14643 425824.013838469
				14644 425834.809628934
				14645 425845.605419398
				14646 425856.401209863
				14647 425867.197000328
				14648 425877.992790792
				14649 425888.788581257
				14650 425899.584371721
				14651 425910.380162186
				14652 425921.17595265
				14653 425931.971743115
				14654 425942.76753358
				14655 425953.563324044
				14656 425964.359114509
				14657 425975.154904973
				14658 425985.950695438
				14659 425996.746485902
				14660 426007.542276367
				14661 426018.338066832
				14662 426029.133857296
				14663 426039.929647761
				14664 426050.725438225
				14665 426061.52122869
				14666 426072.317019155
				14667 426083.112809619
				14668 426093.908600084
				14669 426104.704390548
				14670 426115.500181013
				14671 426126.295971477
				14672 426137.091761942
				14673 426147.887552407
				14674 426158.683342871
				14675 426169.479133336
				14676 426180.2749238
				14677 426191.070714265
				14678 426201.866504729
				14679 426212.662295194
				14680 426223.458085659
				14681 426234.253876123
				14682 426245.049666588
				14683 426255.845457052
				14684 426266.641247517
				14685 426277.437037982
				14686 426288.232828446
				14687 426299.028618911
				14688 426309.824409375
				14689 426320.62019984
				14690 426331.415990304
				14691 426342.211780769
				14692 426353.007571234
				14693 426363.803361698
				14694 426374.599152163
				14695 426385.394942627
				14696 426396.190733092
				14697 426406.986523557
				14698 426417.782314021
				14699 426428.578104486
				14700 426439.37389495
				14701 426450.169685415
				14702 426460.965475879
				14703 426471.761266344
				14704 426482.557056809
				14705 426493.352847273
				14706 426504.148637738
				14707 426514.944428202
				14708 426525.740218667
				14709 426536.536009131
				14710 426547.331799596
				14711 426558.127590061
				14712 426568.923380525
				14713 426579.71917099
				14714 426590.514961454
				14715 426601.310751919
				14716 426612.106542384
				14717 426622.902332848
				14718 426633.698123313
				14719 426644.493913777
				14720 426655.289704242
				14721 426666.085494706
				14722 426676.881285171
				14723 426687.677075636
				14724 426698.4728661
				14725 426709.268656565
				14726 426720.064447029
				14727 426730.860237494
				14728 426741.656027958
				14729 426752.451818423
				14730 426763.247608888
				14731 426774.043399352
				14732 426784.839189817
				14733 426795.634980281
				14734 426806.430770746
				14735 426817.226561211
				14736 426828.022351675
				14737 426838.81814214
				14738 426849.613932604
				14739 426860.409723069
				14740 426871.205513533
				14741 426882.001303998
				14742 426892.797094463
				14743 426903.592884927
				14744 426914.388675392
				14745 426925.184465856
				14746 426935.980256321
				14747 426946.776046786
				14748 426957.57183725
				14749 426968.367627715
				14750 426979.163418179
				14751 426989.959208644
				14752 427000.754999108
				14753 427011.550789573
				14754 427022.346580038
				14755 427033.142370502
				14756 427043.938160967
				14757 427054.733951431
				14758 427065.529741896
				14759 427076.32553236
				14760 427087.121322825
				14761 427097.91711329
				14762 427108.712903754
				14763 427119.508694219
				14764 427130.304484683
				14765 427141.100275148
				14766 427151.896065613
				14767 427162.691856077
				14768 427173.487646542
				14769 427184.283437006
				14770 427195.079227471
				14771 427205.875017935
				14772 427216.6708084
				14773 427227.466598865
				14774 427238.262389329
				14775 427249.058179794
				14776 427259.853970258
				14777 427270.649760723
				14778 427281.445551187
				14779 427292.241341652
				14780 427303.037132117
				14781 427313.832922581
				14782 427324.628713046
				14783 427335.42450351
				14784 427346.220293975
				14785 427357.01608444
				14786 427367.811874904
				14787 427378.607665369
				14788 427389.403455833
				14789 427400.199246298
				14790 427410.995036762
				14791 427421.790827227
				14792 427432.586617692
				14793 427443.382408156
				14794 427454.178198621
				14795 427464.973989085
				14796 427475.76977955
				14797 427486.565570015
				14798 427497.361360479
				14799 427508.157150944
				14800 427518.952941408
				14801 427529.748731873
				14802 427540.544522337
				14803 427551.340312802
				14804 427562.136103267
				14805 427572.931893731
				14806 427583.727684196
				14807 427594.52347466
				14808 427605.319265125
				14809 427616.115055589
				14810 427626.910846054
				14811 427637.706636519
				14812 427648.502426983
				14813 427659.298217448
				14814 427670.094007912
				14815 427680.889798377
				14816 427691.685588842
				14817 427702.481379306
				14818 427713.277169771
				14819 427724.072960235
				14820 427734.8687507
				14821 427745.664541164
				14822 427756.460331629
				14823 427767.256122094
				14824 427778.051912558
				14825 427788.847703023
				14826 427799.643493487
				14827 427810.439283952
				14828 427821.235074417
				14829 427832.030864881
				14830 427842.826655346
				14831 427853.62244581
				14832 427864.418236275
				14833 427875.214026739
				14834 427886.009817204
				14835 427896.805607669
				14836 427907.601398133
				14837 427918.397188598
				14838 427929.192979062
				14839 427939.988769527
				14840 427950.784559991
				14841 427961.580350456
				14842 427972.376140921
				14843 427983.171931385
				14844 427993.96772185
				14845 428004.763512314
				14846 428015.559302779
				14847 428026.355093244
				14848 428037.150883708
				14849 428047.946674173
				14850 428058.742464637
				14851 428069.538255102
				14852 428080.334045566
				14853 428091.129836031
				14854 428101.925626496
				14855 428112.72141696
				14856 428123.517207425
				14857 428134.312997889
				14858 428145.108788354
				14859 428155.904578818
				14860 428166.700369283
				14861 428177.496159748
				14862 428188.291950212
				14863 428199.087740677
				14864 428209.883531141
				14865 428220.679321606
				14866 428231.475112071
				14867 428242.270902535
				14868 428253.066693
				14869 428263.862483464
				14870 428274.658273929
				14871 428285.454064393
				14872 428296.249854858
				14873 428307.045645323
				14874 428317.841435787
				14875 428328.637226252
				14876 428339.433016716
				14877 428350.228807181
				14878 428361.024597646
				14879 428371.82038811
				14880 428382.616178575
				14881 428393.411969039
				14882 428404.207759504
				14883 428415.003549968
				14884 428425.799340433
				14885 428436.595130898
				14886 428447.390921362
				14887 428458.186711827
				14888 428468.982502291
				14889 428479.778292756
				14890 428490.57408322
				14891 428501.369873685
				14892 428512.16566415
				14893 428522.961454614
				14894 428533.757245079
				14895 428544.553035543
				14896 428555.348826008
				14897 428566.144616472
				14898 428576.940406937
				14899 428587.736197402
				14900 428598.531987866
				14901 428609.327778331
				14902 428620.123568795
				14903 428630.91935926
				14904 428641.715149725
				14905 428652.510940189
				14906 428663.306730654
				14907 428674.102521118
				14908 428684.898311583
				14909 428695.694102047
				14910 428706.489892512
				14911 428717.285682977
				14912 428728.081473441
				14913 428738.877263906
				14914 428749.67305437
				14915 428760.468844835
				14916 428771.2646353
				14917 428782.060425764
				14918 428792.856216229
				14919 428803.652006693
				14920 428814.447797158
				14921 428825.243587622
				14922 428836.039378087
				14923 428846.835168552
				14924 428857.630959016
				14925 428868.426749481
				14926 428879.222539945
				14927 428890.01833041
				14928 428900.814120874
				14929 428911.609911339
				14930 428922.405701804
				14931 428933.201492268
				14932 428943.997282733
				14933 428954.793073197
				14934 428965.588863662
				14935 428976.384654126
				14936 428987.180444591
				14937 428997.976235056
				14938 429008.77202552
				14939 429019.567815985
				14940 429030.363606449
				14941 429041.159396914
				14942 429051.955187379
				14943 429062.750977843
				14944 429073.546768308
				14945 429084.342558772
				14946 429095.138349237
				14947 429105.934139701
				14948 429116.729930166
				14949 429127.525720631
				14950 429138.321511095
				14951 429149.11730156
				14952 429159.913092024
				14953 429170.708882489
				14954 429181.504672954
				14955 429192.300463418
				14956 429203.096253883
				14957 429213.892044347
				14958 429224.687834812
				14959 429235.483625276
				14960 429246.279415741
				14961 429257.075206206
				14962 429267.87099667
				14963 429278.666787135
				14964 429289.462577599
				14965 429300.258368064
				14966 429311.054158528
				14967 429321.849948993
				14968 429332.645739458
				14969 429343.441529922
				14970 429354.237320387
				14971 429365.033110851
				14972 429375.828901316
				14973 429386.624691781
				14974 429397.420482245
				14975 429408.21627271
				14976 429419.012063174
				14977 429429.807853639
				14978 429440.603644103
				14979 429451.399434568
				14980 429462.195225033
				14981 429472.991015497
				14982 429483.786805962
				14983 429494.582596426
				14984 429505.378386891
				14985 429516.174177355
				14986 429526.96996782
				14987 429537.765758285
				14988 429548.561548749
				14989 429559.357339214
				14990 429570.153129678
				14991 429580.948920143
				14992 429591.744710608
				14993 429602.540501072
				14994 429613.336291537
				14995 429624.132082001
				14996 429634.927872466
				14997 429645.72366293
				14998 429656.519453395
				14999 429667.31524386
				15000 429678.111034324
				15001 429688.906824789
				15002 429699.702615253
				15003 429710.498405718
				15004 429721.294196183
				15005 429732.089986647
				15006 429742.885777112
				15007 429753.681567576
				15008 429764.477358041
				15009 429775.273148505
				15010 429786.06893897
				15011 429796.864729435
				15012 429807.660519899
				15013 429818.456310364
				15014 429829.252100828
				15015 429840.047891293
				15016 429850.843681757
				15017 429861.639472222
				15018 429872.435262687
				15019 429883.231053151
				15020 429894.026843616
				15021 429904.82263408
				15022 429915.618424545
				15023 429926.41421501
				15024 429937.210005474
				15025 429948.005795939
				15026 429958.801586403
				15027 429969.597376868
				15028 429980.393167332
				15029 429991.188957797
				15030 430001.984748262
				15031 430012.780538726
				15032 430023.576329191
				15033 430034.372119655
				15034 430045.16791012
				15035 430055.963700584
				15036 430066.759491049
				15037 430077.555281514
				15038 430088.351071978
				15039 430099.146862443
				15040 430109.942652907
				15041 430120.738443372
				15042 430131.534233837
				15043 430142.330024301
				15044 430153.125814766
				15045 430163.92160523
				15046 430174.717395695
				15047 430185.513186159
				15048 430196.308976624
				15049 430207.104767089
				15050 430217.900557553
				15051 430228.696348018
				15052 430239.492138482
				15053 430250.287928947
				15054 430261.083719412
				15055 430271.879509876
				15056 430282.675300341
				15057 430293.471090805
				15058 430304.26688127
				15059 430315.062671734
				15060 430325.858462199
				15061 430336.654252664
				15062 430347.450043128
				15063 430358.245833593
				15064 430369.041624057
				15065 430379.837414522
				15066 430390.633204986
				15067 430401.428995451
				15068 430412.224785916
				15069 430423.02057638
				15070 430433.816366845
				15071 430444.612157309
				15072 430455.407947774
				15073 430466.203738239
				15074 430476.999528703
				15075 430487.795319168
				15076 430498.591109632
				15077 430509.386900097
				15078 430520.182690561
				15079 430530.978481026
				15080 430541.774271491
				15081 430552.570061955
				15082 430563.36585242
				15083 430574.161642884
				15084 430584.957433349
				15085 430595.753223813
				15086 430606.549014278
				15087 430617.344804743
				15088 430628.140595207
				15089 430638.936385672
				15090 430649.732176136
				15091 430660.527966601
				15092 430671.323757066
				15093 430682.11954753
				15094 430692.915337995
				15095 430703.711128459
				15096 430714.506918924
				15097 430725.302709388
				15098 430736.098499853
				15099 430746.894290318
				15100 430757.690080782
				15101 430768.485871247
				15102 430779.281661711
				15103 430790.077452176
				15104 430800.873242641
				15105 430811.669033105
				15106 430822.46482357
				15107 430833.260614034
				15108 430844.056404499
				15109 430854.852194963
				15110 430865.647985428
				15111 430876.443775893
				15112 430887.239566357
				15113 430898.035356822
				15114 430908.831147286
				15115 430919.626937751
				15116 430930.422728215
				15117 430941.21851868
				15118 430952.014309145
				15119 430962.810099609
				15120 430973.605890074
				15121 430984.401680538
				15122 430995.197471003
				15123 431005.993261468
				15124 431016.789051932
				15125 431027.584842397
				15126 431038.380632861
				15127 431049.176423326
				15128 431059.97221379
				15129 431070.768004255
				15130 431081.56379472
				15131 431092.359585184
				15132 431103.155375649
				15133 431113.951166113
				15134 431124.746956578
				15135 431135.542747042
				15136 431146.338537507
				15137 431157.134327972
				15138 431167.930118436
				15139 431178.725908901
				15140 431189.521699365
				15141 431200.31748983
				15142 431211.113280295
				15143 431221.909070759
				15144 431232.704861224
				15145 431243.500651688
				15146 431254.296442153
				15147 431265.092232617
				15148 431275.888023082
				15149 431286.683813547
				15150 431297.479604011
				15151 431308.275394476
				15152 431319.07118494
				15153 431329.866975405
				15154 431340.66276587
				15155 431351.458556334
				15156 431362.254346799
				15157 431373.050137263
				15158 431383.845927728
				15159 431394.641718192
				15160 431405.437508657
				15161 431416.233299122
				15162 431427.029089586
				15163 431437.824880051
				15164 431448.620670515
				15165 431459.41646098
				15166 431470.212251444
				15167 431481.008041909
				15168 431491.803832374
				15169 431502.599622838
				15170 431513.395413303
				15171 431524.191203767
				15172 431534.986994232
				15173 431545.782784697
				15174 431556.578575161
				15175 431567.374365626
				15176 431578.17015609
				15177 431588.965946555
				15178 431599.761737019
				15179 431610.557527484
				15180 431621.353317949
				15181 431632.149108413
				15182 431642.944898878
				15183 431653.740689342
				15184 431664.536479807
				15185 431675.332270271
				15186 431686.128060736
				15187 431696.923851201
				15188 431707.719641665
				15189 431718.51543213
				15190 431729.311222594
				15191 431740.107013059
				15192 431750.902803524
				15193 431761.698593988
				15194 431772.494384453
				15195 431783.290174917
				15196 431794.085965382
				15197 431804.881755846
				15198 431815.677546311
				15199 431826.473336776
				15200 431837.26912724
				15201 431848.064917705
				15202 431858.860708169
				15203 431869.656498634
				15204 431880.452289099
				15205 431891.248079563
				15206 431902.043870028
				15207 431912.839660492
				15208 431923.635450957
				15209 431934.431241421
				15210 431945.227031886
				15211 431956.022822351
				15212 431966.818612815
				15213 431977.61440328
				15214 431988.410193744
				15215 431999.205984209
				15216 432010.001774673
				15217 432020.797565138
				15218 432031.593355603
				15219 432042.389146067
				15220 432053.184936532
				15221 432063.980726996
				15222 432074.776517461
				15223 432085.572307925
				15224 432096.36809839
				15225 432107.163888855
				15226 432117.959679319
				15227 432128.755469784
				15228 432139.551260248
				15229 432150.347050713
				15230 432161.142841178
				15231 432171.938631642
				15232 432182.734422107
				15233 432193.530212571
				15234 432204.326003036
				15235 432215.1217935
				15236 432225.917583965
				15237 432236.71337443
				15238 432247.509164894
				15239 432258.304955359
				15240 432269.100745823
				15241 432279.896536288
				15242 432290.692326753
				15243 432301.488117217
				15244 432312.283907682
				15245 432323.079698146
				15246 432333.875488611
				15247 432344.671279075
				15248 432355.46706954
				15249 432366.262860005
				15250 432377.058650469
				15251 432387.854440934
				15252 432398.650231398
				15253 432409.446021863
				15254 432420.241812327
				15255 432431.037602792
				15256 432441.833393257
				15257 432452.629183721
				15258 432463.424974186
				15259 432474.22076465
				15260 432485.016555115
				15261 432495.812345579
				15262 432506.608136044
				15263 432517.403926509
				15264 432528.199716973
				15265 432538.995507438
				15266 432549.791297902
				15267 432560.587088367
				15268 432571.382878832
				15269 432582.178669296
				15270 432592.974459761
				15271 432603.770250225
				15272 432614.56604069
				15273 432625.361831154
				15274 432636.157621619
				15275 432646.953412084
				15276 432657.749202548
				15277 432668.544993013
				15278 432679.340783477
				15279 432690.136573942
				15280 432700.932364407
				15281 432711.728154871
				15282 432722.523945336
				15283 432733.3197358
				15284 432744.115526265
				15285 432754.911316729
				15286 432765.707107194
				15287 432776.502897659
				15288 432787.298688123
				15289 432798.094478588
				15290 432808.890269052
				15291 432819.686059517
				15292 432830.481849981
				15293 432841.277640446
				15294 432852.073430911
				15295 432862.869221375
				15296 432873.66501184
				15297 432884.460802304
				15298 432895.256592769
				15299 432906.052383234
				15300 432916.848173698
				15301 432927.643964163
				15302 432938.439754627
				15303 432949.235545092
				15304 432960.031335556
				15305 432970.827126021
				15306 432981.622916486
				15307 432992.41870695
				15308 433003.214497415
				15309 433014.010287879
				15310 433024.806078344
				15311 433035.601868809
				15312 433046.397659273
				15313 433057.193449738
				15314 433067.989240202
				15315 433078.785030667
				15316 433089.580821131
				15317 433100.376611596
				15318 433111.172402061
				15319 433121.968192525
				15320 433132.76398299
				15321 433143.559773454
				15322 433154.355563919
				15323 433165.151354383
				15324 433175.947144848
				15325 433186.742935313
				15326 433197.538725777
				15327 433208.334516242
				15328 433219.130306706
				15329 433229.926097171
				15330 433240.721887636
				15331 433251.5176781
				15332 433262.313468565
				15333 433273.109259029
				15334 433283.905049494
				15335 433294.700839958
				15336 433305.496630423
				15337 433316.292420888
				15338 433327.088211352
				15339 433337.884001817
				15340 433348.679792281
				15341 433359.475582746
				15342 433370.27137321
				15343 433381.067163675
				15344 433391.86295414
				15345 433402.658744604
				15346 433413.454535069
				15347 433424.250325533
				15348 433435.046115998
				15349 433445.841906463
				15350 433456.637696927
				15351 433467.433487392
				15352 433478.229277856
				15353 433489.025068321
				15354 433499.820858785
				15355 433510.61664925
				15356 433521.412439715
				15357 433532.208230179
				15358 433543.004020644
				15359 433553.799811108
				15360 433564.595601573
				15361 433575.391392038
				15362 433586.187182502
				15363 433596.982972967
				15364 433607.778763431
				15365 433618.574553896
				15366 433629.37034436
				15367 433640.166134825
				15368 433650.96192529
				15369 433661.757715754
				15370 433672.553506219
				15371 433683.349296683
				15372 433694.145087148
				15373 433704.940877612
				15374 433715.736668077
				15375 433726.532458542
				15376 433737.328249006
				15377 433748.124039471
				15378 433758.919829935
				15379 433769.7156204
				15380 433780.511410865
				15381 433791.307201329
				15382 433802.102991794
				15383 433812.898782258
				15384 433823.694572723
				15385 433834.490363187
				15386 433845.286153652
				15387 433856.081944117
				15388 433866.877734581
				15389 433877.673525046
				15390 433888.46931551
				15391 433899.265105975
				15392 433910.060896439
				15393 433920.856686904
				15394 433931.652477369
				15395 433942.448267833
				15396 433953.244058298
				15397 433964.039848762
				15398 433974.835639227
				15399 433985.631429692
				15400 433996.427220156
				15401 434007.223010621
				15402 434018.018801085
				15403 434028.81459155
				15404 434039.610382014
				15405 434050.406172479
				15406 434061.201962944
				15407 434071.997753408
				15408 434082.793543873
				15409 434093.589334337
				15410 434104.385124802
				15411 434115.180915267
				15412 434125.976705731
				15413 434136.772496196
				15414 434147.56828666
				15415 434158.364077125
				15416 434169.159867589
				15417 434179.955658054
				15418 434190.751448519
				15419 434201.547238983
				15420 434212.343029448
				15421 434223.138819912
				15422 434233.934610377
				15423 434244.730400841
				15424 434255.526191306
				15425 434266.321981771
				15426 434277.117772235
				15427 434287.9135627
				15428 434298.709353164
				15429 434309.505143629
				15430 434320.300934094
				15431 434331.096724558
				15432 434341.892515023
				15433 434352.688305487
				15434 434363.484095952
				15435 434374.279886416
				15436 434385.075676881
				15437 434395.871467346
				15438 434406.66725781
				15439 434417.463048275
				15440 434428.258838739
				15441 434439.054629204
				15442 434449.850419668
				15443 434460.646210133
				15444 434471.442000598
				15445 434482.237791062
				15446 434493.033581527
				15447 434503.829371991
				15448 434514.625162456
				15449 434525.420952921
				15450 434536.216743385
				15451 434547.01253385
				15452 434557.808324314
				15453 434568.604114779
				15454 434579.399905243
				15455 434590.195695708
				15456 434600.991486173
				15457 434611.787276637
				15458 434622.583067102
				15459 434633.378857566
				15460 434644.174648031
				15461 434654.970438496
				15462 434665.76622896
				15463 434676.562019425
				15464 434687.357809889
				15465 434698.153600354
				15466 434708.949390818
				15467 434719.745181283
				15468 434730.540971748
				15469 434741.336762212
				15470 434752.132552677
				15471 434762.928343141
				15472 434773.724133606
				15473 434784.51992407
				15474 434795.315714535
				15475 434806.111505
				15476 434816.907295464
				15477 434827.703085929
				15478 434838.498876393
				15479 434849.294666858
				15480 434860.090457323
				15481 434870.886247787
				15482 434881.682038252
				15483 434892.477828716
				15484 434903.273619181
				15485 434914.069409645
				15486 434924.86520011
				15487 434935.660990575
				15488 434946.456781039
				15489 434957.252571504
				15490 434968.048361968
				15491 434978.844152433
				15492 434989.639942897
				15493 435000.435733362
				15494 435011.231523827
				15495 435022.027314291
				15496 435032.823104756
				15497 435043.61889522
				15498 435054.414685685
				15499 435065.21047615
				15500 435076.006266614
				15501 435086.802057079
				15502 435097.597847543
				15503 435108.393638008
				15504 435119.189428472
				15505 435129.985218937
				15506 435140.781009402
				15507 435151.576799866
				15508 435162.372590331
				15509 435173.168380795
				15510 435183.96417126
				15511 435194.759961725
				15512 435205.555752189
				15513 435216.351542654
				15514 435227.147333118
				15515 435237.943123583
				15516 435248.738914047
				15517 435259.534704512
				15518 435270.330494977
				15519 435281.126285441
				15520 435291.922075906
				15521 435302.71786637
				15522 435313.513656835
				15523 435324.309447299
				15524 435335.105237764
				15525 435345.901028229
				15526 435356.696818693
				15527 435367.492609158
				15528 435378.288399622
				15529 435389.084190087
				15530 435399.879980552
				15531 435410.675771016
				15532 435421.471561481
				15533 435432.267351945
				15534 435443.06314241
				15535 435453.858932874
				15536 435464.654723339
				15537 435475.450513804
				15538 435486.246304268
				15539 435497.042094733
				15540 435507.837885197
				15541 435518.633675662
				15542 435529.429466126
				15543 435540.225256591
				15544 435551.021047056
				15545 435561.81683752
				15546 435572.612627985
				15547 435583.408418449
				15548 435594.204208914
				15549 435604.999999379
				15550 435615.795789843
				15551 435626.591580308
				15552 435637.387370772
				15553 435648.183161237
				15554 435658.978951701
				15555 435669.774742166
				15556 435680.570532631
				15557 435691.366323095
				15558 435702.16211356
				15559 435712.957904024
				15560 435723.753694489
				15561 435734.549484954
				15562 435745.345275418
				15563 435756.141065883
				15564 435766.936856347
				15565 435777.732646812
				15566 435788.528437276
				15567 435799.324227741
				15568 435810.120018206
				15569 435820.91580867
				15570 435831.711599135
				15571 435842.507389599
				15572 435853.303180064
				15573 435864.098970528
				15574 435874.894760993
				15575 435885.690551458
				15576 435896.486341922
				15577 435907.282132387
				15578 435918.077922851
				15579 435928.873713316
				15580 435939.66950378
				15581 435950.465294245
				15582 435961.26108471
				15583 435972.056875174
				15584 435982.852665639
				15585 435993.648456103
				15586 436004.444246568
				15587 436015.240037033
				15588 436026.035827497
				15589 436036.831617962
				15590 436047.627408426
				15591 436058.423198891
				15592 436069.218989355
				15593 436080.01477982
				15594 436090.810570285
				15595 436101.606360749
				15596 436112.402151214
				15597 436123.197941678
				15598 436133.993732143
				15599 436144.789522608
				15600 436155.585313072
				15601 436166.381103537
				15602 436177.176894001
				15603 436187.972684466
				15604 436198.76847493
				15605 436209.564265395
				15606 436220.36005586
				15607 436231.155846324
				15608 436241.951636789
				15609 436252.747427253
				15610 436263.543217718
				15611 436274.339008182
				15612 436285.134798647
				15613 436295.930589112
				15614 436306.726379576
				15615 436317.522170041
				15616 436328.317960505
				15617 436339.11375097
				15618 436349.909541434
				15619 436360.705331899
				15620 436371.501122364
				15621 436382.296912828
				15622 436393.092703293
				15623 436403.888493757
				15624 436414.684284222
				15625 436425.480074687
				15626 436436.275865151
				15627 436447.071655616
				15628 436457.86744608
				15629 436468.663236545
				15630 436479.459027009
				15631 436490.254817474
				15632 436501.050607939
				15633 436511.846398403
				15634 436522.642188868
				15635 436533.437979332
				15636 436544.233769797
				15637 436555.029560262
				15638 436565.825350726
				15639 436576.621141191
				15640 436587.416931655
				15641 436598.21272212
				15642 436609.008512584
				15643 436619.804303049
				15644 436630.600093514
				15645 436641.395883978
				15646 436652.191674443
				15647 436662.987464907
				15648 436673.783255372
				15649 436684.579045836
				15650 436695.374836301
				15651 436706.170626766
				15652 436716.96641723
				15653 436727.762207695
				15654 436738.557998159
				15655 436749.353788624
				15656 436760.149579089
				15657 436770.945369553
				15658 436781.741160018
				15659 436792.536950482
				15660 436803.332740947
				15661 436814.128531411
				15662 436824.924321876
				15663 436835.720112341
				15664 436846.515902805
				15665 436857.31169327
				15666 436868.107483734
				15667 436878.903274199
				15668 436889.699064663
				15669 436900.494855128
				15670 436911.290645593
				15671 436922.086436057
				15672 436932.882226522
				15673 436943.678016986
				15674 436954.473807451
				15675 436965.269597916
				15676 436976.06538838
				15677 436986.861178845
				15678 436997.656969309
				15679 437008.452759774
				15680 437019.248550238
				15681 437030.044340703
				15682 437040.840131168
				15683 437051.635921632
				15684 437062.431712097
				15685 437073.227502561
				15686 437084.023293026
				15687 437094.819083491
				15688 437105.614873955
				15689 437116.41066442
				15690 437127.206454884
				15691 437138.002245349
				15692 437148.798035813
				15693 437159.593826278
				15694 437170.389616743
				15695 437181.185407207
				15696 437191.981197672
				15697 437202.776988136
				15698 437213.572778601
				15699 437224.368569065
				15700 437235.16435953
				15701 437245.960149995
				15702 437256.755940459
				15703 437267.551730924
				15704 437278.347521388
				15705 437289.143311853
				15706 437299.939102318
				15707 437310.734892782
				15708 437321.530683247
				15709 437332.326473711
				15710 437343.122264176
				15711 437353.91805464
				15712 437364.713845105
				15713 437375.50963557
				15714 437386.305426034
				15715 437397.101216499
				15716 437407.897006963
				15717 437418.692797428
				15718 437429.488587892
				15719 437440.284378357
				15720 437451.080168822
				15721 437461.875959286
				15722 437472.671749751
				15723 437483.467540215
				15724 437494.26333068
				15725 437505.059121145
				15726 437515.854911609
				15727 437526.650702074
				15728 437537.446492538
				15729 437548.242283003
				15730 437559.038073467
				15731 437569.833863932
				15732 437580.629654397
				15733 437591.425444861
				15734 437602.221235326
				15735 437613.01702579
				15736 437623.812816255
				15737 437634.60860672
				15738 437645.404397184
				15739 437656.200187649
				15740 437666.995978113
				15741 437677.791768578
				15742 437688.587559042
				15743 437699.383349507
				15744 437710.179139972
				15745 437720.974930436
				15746 437731.770720901
				15747 437742.566511365
				15748 437753.36230183
				15749 437764.158092294
				15750 437774.953882759
				15751 437785.749673224
				15752 437796.545463688
				15753 437807.341254153
				15754 437818.137044617
				15755 437828.932835082
				15756 437839.728625547
				15757 437850.524416011
				15758 437861.320206476
				15759 437872.11599694
				15760 437882.911787405
				15761 437893.707577869
				15762 437904.503368334
				15763 437915.299158799
				15764 437926.094949263
				15765 437936.890739728
				15766 437947.686530192
				15767 437958.482320657
				15768 437969.278111122
				15769 437980.073901586
				15770 437990.869692051
				15771 438001.665482515
				15772 438012.46127298
				15773 438023.257063444
				15774 438034.052853909
				15775 438044.848644374
				15776 438055.644434838
				15777 438066.440225303
				15778 438077.236015767
				15779 438088.031806232
				15780 438098.827596696
				15781 438109.623387161
				15782 438120.419177626
				15783 438131.21496809
				15784 438142.010758555
				15785 438152.806549019
				15786 438163.602339484
				15787 438174.398129949
				15788 438185.193920413
				15789 438195.989710878
				15790 438206.785501342
				15791 438217.581291807
				15792 438228.377082271
				15793 438239.172872736
				15794 438249.968663201
				15795 438260.764453665
				15796 438271.56024413
				15797 438282.356034594
				15798 438293.151825059
				15799 438303.947615523
				15800 438314.743405988
				15801 438325.539196453
				15802 438336.334986917
				15803 438347.130777382
				15804 438357.926567846
				15805 438368.722358311
				15806 438379.518148776
				15807 438390.31393924
				15808 438401.109729705
				15809 438411.905520169
				15810 438422.701310634
				15811 438433.497101098
				15812 438444.292891563
				15813 438455.088682028
				15814 438465.884472492
				15815 438476.680262957
				15816 438487.476053421
				15817 438498.271843886
				15818 438509.067634351
				15819 438519.863424815
				15820 438530.65921528
				15821 438541.455005744
				15822 438552.250796209
				15823 438563.046586673
				15824 438573.842377138
				15825 438584.638167603
				15826 438595.433958067
				15827 438606.229748532
				15828 438617.025538996
				15829 438627.821329461
				15830 438638.617119925
				15831 438649.41291039
				15832 438660.208700855
				15833 438671.004491319
				15834 438681.800281784
				15835 438692.596072248
				15836 438703.391862713
				15837 438714.187653178
				15838 438724.983443642
				15839 438735.779234107
				15840 438746.575024571
				15841 438757.370815036
				15842 438768.1666055
				15843 438778.962395965
				15844 438789.75818643
				15845 438800.553976894
				15846 438811.349767359
				15847 438822.145557823
				15848 438832.941348288
				15849 438843.737138752
				15850 438854.532929217
				15851 438865.328719682
				15852 438876.124510146
				15853 438886.920300611
				15854 438897.716091075
				15855 438908.51188154
				15856 438919.307672005
				15857 438930.103462469
				15858 438940.899252934
				15859 438951.695043398
				15860 438962.490833863
				15861 438973.286624327
				15862 438984.082414792
				15863 438994.878205257
				15864 439005.673995721
				15865 439016.469786186
				15866 439027.26557665
				15867 439038.061367115
				15868 439048.85715758
				15869 439059.652948044
				15870 439070.448738509
				15871 439081.244528973
				15872 439092.040319438
				15873 439102.836109902
				15874 439113.631900367
				15875 439124.427690832
				15876 439135.223481296
				15877 439146.019271761
				15878 439156.815062225
				15879 439167.61085269
				15880 439178.406643154
				15881 439189.202433619
				15882 439199.998224084
				15883 439210.794014548
				15884 439221.589805013
				15885 439232.385595477
				15886 439243.181385942
				15887 439253.977176406
				15888 439264.772966871
				15889 439275.568757336
				15890 439286.3645478
				15891 439297.160338265
				15892 439307.956128729
				15893 439318.751919194
				15894 439329.547709659
				15895 439340.343500123
				15896 439351.139290588
				15897 439361.935081052
				15898 439372.730871517
				15899 439383.526661981
				15900 439394.322452446
				15901 439405.118242911
				15902 439415.914033375
				15903 439426.70982384
				15904 439437.505614304
				15905 439448.301404769
				15906 439459.097195234
				15907 439469.892985698
				15908 439480.688776163
				15909 439491.484566627
				15910 439502.280357092
				15911 439513.076147556
				15912 439523.871938021
				15913 439534.667728486
				15914 439545.46351895
				15915 439556.259309415
				15916 439567.055099879
				15917 439577.850890344
				15918 439588.646680809
				15919 439599.442471273
				15920 439610.238261738
				15921 439621.034052202
				15922 439631.829842667
				15923 439642.625633131
				15924 439653.421423596
				15925 439664.217214061
				15926 439675.013004525
				15927 439685.80879499
				15928 439696.604585454
				15929 439707.400375919
				15930 439718.196166383
				15931 439728.991956848
				15932 439739.787747313
				15933 439750.583537777
				15934 439761.379328242
				15935 439772.175118706
				15936 439782.970909171
				15937 439793.766699635
				15938 439804.5624901
				15939 439815.358280565
				15940 439826.154071029
				15941 439836.949861494
				15942 439847.745651958
				15943 439858.541442423
				15944 439869.337232888
				15945 439880.133023352
				15946 439890.928813817
				15947 439901.724604281
				15948 439912.520394746
				15949 439923.31618521
				15950 439934.111975675
				15951 439944.90776614
				15952 439955.703556604
				15953 439966.499347069
				15954 439977.295137533
				15955 439988.090927998
				15956 439998.886718463
				15957 440009.682508927
				15958 440020.478299392
				15959 440031.274089856
				15960 440042.069880321
				15961 440052.865670785
				15962 440063.66146125
				15963 440074.457251715
				15964 440085.253042179
				15965 440096.048832644
				15966 440106.844623108
				15967 440117.640413573
				15968 440128.436204037
				15969 440139.231994502
				15970 440150.027784967
				15971 440160.823575431
				15972 440171.619365896
				15973 440182.41515636
				15974 440193.210946825
				15975 440204.006737289
				15976 440214.802527754
				15977 440225.598318219
				15978 440236.394108683
				15979 440247.189899148
				15980 440257.985689612
				15981 440268.781480077
				15982 440279.577270542
				15983 440290.373061006
				15984 440301.168851471
				15985 440311.964641935
				15986 440322.7604324
				15987 440333.556222864
				15988 440344.352013329
				15989 440355.147803794
				15990 440365.943594258
				15991 440376.739384723
				15992 440387.535175187
				15993 440398.330965652
				15994 440409.126756117
				15995 440419.922546581
				15996 440430.718337046
				15997 440441.51412751
				15998 440452.309917975
				15999 440463.105708439
				16000 440473.901498904
				16001 440484.697289369
				16002 440495.493079833
				16003 440506.288870298
				16004 440517.084660762
				16005 440527.880451227
				16006 440538.676241691
				16007 440549.472032156
				16008 440560.267822621
				16009 440571.063613085
				16010 440581.85940355
				16011 440592.655194014
				16012 440603.450984479
				16013 440614.246774944
				16014 440625.042565408
				16015 440635.838355873
				16016 440646.634146337
				16017 440657.429936802
				16018 440668.225727266
				16019 440679.021517731
				16020 440689.817308196
				16021 440700.61309866
				16022 440711.408889125
				16023 440722.204679589
				16024 440733.000470054
				16025 440743.796260518
				16026 440754.592050983
				16027 440765.387841448
				16028 440776.183631912
				16029 440786.979422377
				16030 440797.775212841
				16031 440808.571003306
				16032 440819.366793771
				16033 440830.162584235
				16034 440840.9583747
				16035 440851.754165164
				16036 440862.549955629
				16037 440873.345746093
				16038 440884.141536558
				16039 440894.937327023
				16040 440905.733117487
				16041 440916.528907952
				16042 440927.324698416
				16043 440938.120488881
				16044 440948.916279346
				16045 440959.71206981
				16046 440970.507860275
				16047 440981.303650739
				16048 440992.099441204
				16049 441002.895231668
				16050 441013.691022133
				16051 441024.486812598
				16052 441035.282603062
				16053 441046.078393527
				16054 441056.874183991
				16055 441067.669974456
				16056 441078.46576492
				16057 441089.261555385
				16058 441100.05734585
				16059 441110.853136314
				16060 441121.648926779
				16061 441132.444717243
				16062 441143.240507708
				16063 441154.036298173
				16064 441164.832088637
				16065 441175.627879102
				16066 441186.423669566
				16067 441197.219460031
				16068 441208.015250495
				16069 441218.81104096
				16070 441229.606831425
				16071 441240.402621889
				16072 441251.198412354
				16073 441261.994202818
				16074 441272.789993283
				16075 441283.585783747
				16076 441294.381574212
				16077 441305.177364677
				16078 441315.973155141
				16079 441326.768945606
				16080 441337.56473607
				16081 441348.360526535
				16082 441359.156317
				16083 441369.952107464
				16084 441380.747897929
				16085 441391.543688393
				16086 441402.339478858
				16087 441413.135269322
				16088 441423.931059787
				16089 441434.726850252
				16090 441445.522640716
				16091 441456.318431181
				16092 441467.114221645
				16093 441477.91001211
				16094 441488.705802575
				16095 441499.501593039
				16096 441510.297383504
				16097 441521.093173968
				16098 441531.888964433
				16099 441542.684754897
				16100 441553.480545362
				16101 441564.276335827
				16102 441575.072126291
				16103 441585.867916756
				16104 441596.66370722
				16105 441607.459497685
				16106 441618.255288149
				16107 441629.051078614
				16108 441639.846869079
				16109 441650.642659543
				16110 441661.438450008
				16111 441672.234240472
				16112 441683.030030937
				16113 441693.825821402
				16114 441704.621611866
				16115 441715.417402331
				16116 441726.213192795
				16117 441737.00898326
				16118 441747.804773724
				16119 441758.600564189
				16120 441769.396354654
				16121 441780.192145118
				16122 441790.987935583
				16123 441801.783726047
				16124 441812.579516512
				16125 441823.375306976
				16126 441834.171097441
				16127 441844.966887906
				16128 441855.76267837
				16129 441866.558468835
				16130 441877.354259299
				16131 441888.150049764
				16132 441898.945840229
				16133 441909.741630693
				16134 441920.537421158
				16135 441931.333211622
				16136 441942.129002087
				16137 441952.924792551
				16138 441963.720583016
				16139 441974.516373481
				16140 441985.312163945
				16141 441996.10795441
				16142 442006.903744874
				16143 442017.699535339
				16144 442028.495325804
				16145 442039.291116268
				16146 442050.086906733
				16147 442060.882697197
				16148 442071.678487662
				16149 442082.474278126
				16150 442093.270068591
				16151 442104.065859056
				16152 442114.86164952
				16153 442125.657439985
				16154 442136.453230449
				16155 442147.249020914
				16156 442158.044811378
				16157 442168.840601843
				16158 442179.636392308
				16159 442190.432182772
				16160 442201.227973237
				16161 442212.023763701
				16162 442222.819554166
				16163 442233.615344631
				16164 442244.411135095
				16165 442255.20692556
				16166 442266.002716024
				16167 442276.798506489
				16168 442287.594296953
				16169 442298.390087418
				16170 442309.185877883
				16171 442319.981668347
				16172 442330.777458812
				16173 442341.573249276
				16174 442352.369039741
				16175 442363.164830205
				16176 442373.96062067
				16177 442384.756411135
				16178 442395.552201599
				16179 442406.347992064
				16180 442417.143782528
				16181 442427.939572993
				16182 442438.735363458
				16183 442449.531153922
				16184 442460.326944387
				16185 442471.122734851
				16186 442481.918525316
				16187 442492.71431578
				16188 442503.510106245
				16189 442514.30589671
				16190 442525.101687174
				16191 442535.897477639
				16192 442546.693268103
				16193 442557.489058568
				16194 442568.284849033
				16195 442579.080639497
				16196 442589.876429962
				16197 442600.672220426
				16198 442611.468010891
				16199 442622.263801355
				16200 442633.05959182
				16201 442643.855382285
				16202 442654.651172749
				16203 442665.446963214
				16204 442676.242753678
				16205 442687.038544143
				16206 442697.834334607
				16207 442708.630125072
				16208 442719.425915537
				16209 442730.221706001
				16210 442741.017496466
				16211 442751.81328693
				16212 442762.609077395
				16213 442773.404867859
				16214 442784.200658324
				16215 442794.996448789
				16216 442805.792239253
				16217 442816.588029718
				16218 442827.383820182
				16219 442838.179610647
				16220 442848.975401112
				16221 442859.771191576
				16222 442870.566982041
				16223 442881.362772505
				16224 442892.15856297
				16225 442902.954353434
				16226 442913.750143899
				16227 442924.545934364
				16228 442935.341724828
				16229 442946.137515293
				16230 442956.933305757
				16231 442967.729096222
				16232 442978.524886687
				16233 442989.320677151
				16234 443000.116467616
				16235 443010.91225808
				16236 443021.708048545
				16237 443032.503839009
				16238 443043.299629474
				16239 443054.095419939
				16240 443064.891210403
				16241 443075.687000868
				16242 443086.482791332
				16243 443097.278581797
				16244 443108.074372261
				16245 443118.870162726
				16246 443129.665953191
				16247 443140.461743655
				16248 443151.25753412
				16249 443162.053324584
				16250 443172.849115049
				16251 443183.644905514
				16252 443194.440695978
				16253 443205.236486443
				16254 443216.032276907
				16255 443226.828067372
				16256 443237.623857836
				16257 443248.419648301
				16258 443259.215438766
				16259 443270.01122923
				16260 443280.807019695
				16261 443291.602810159
				16262 443302.398600624
				16263 443313.194391089
				16264 443323.990181553
				16265 443334.785972018
				16266 443345.581762482
				16267 443356.377552947
				16268 443367.173343411
				16269 443377.969133876
				16270 443388.764924341
				16271 443399.560714805
				16272 443410.35650527
				16273 443421.152295734
				16274 443431.948086199
				16275 443442.743876663
				16276 443453.539667128
				16277 443464.335457593
				16278 443475.131248057
				16279 443485.927038522
				16280 443496.722828986
				16281 443507.518619451
				16282 443518.314409915
				16283 443529.11020038
				16284 443539.905990845
				16285 443550.701781309
				16286 443561.497571774
				16287 443572.293362238
				16288 443583.089152703
				16289 443593.884943168
				16290 443604.680733632
				16291 443615.476524097
				16292 443626.272314561
				16293 443637.068105026
				16294 443647.86389549
				16295 443658.659685955
				16296 443669.45547642
				16297 443680.251266884
				16298 443691.047057349
				16299 443701.842847813
				16300 443712.638638278
				16301 443723.434428743
				16302 443734.230219207
				16303 443745.026009672
				16304 443755.821800136
				16305 443766.617590601
				16306 443777.413381065
				16307 443788.20917153
				16308 443799.004961995
				16309 443809.800752459
				16310 443820.596542924
				16311 443831.392333388
				16312 443842.188123853
				16313 443852.983914317
				16314 443863.779704782
				16315 443874.575495247
				16316 443885.371285711
				16317 443896.167076176
				16318 443906.96286664
				16319 443917.758657105
				16320 443928.55444757
				16321 443939.350238034
				16322 443950.146028499
				16323 443960.941818963
				16324 443971.737609428
				16325 443982.533399892
				16326 443993.329190357
				16327 444004.124980822
				16328 444014.920771286
				16329 444025.716561751
				16330 444036.512352215
				16331 444047.30814268
				16332 444058.103933144
				16333 444068.899723609
				16334 444079.695514074
				16335 444090.491304538
				16336 444101.287095003
				16337 444112.082885467
				16338 444122.878675932
				16339 444133.674466397
				16340 444144.470256861
				16341 444155.266047326
				16342 444166.06183779
				16343 444176.857628255
				16344 444187.653418719
				16345 444198.449209184
				16346 444209.244999649
				16347 444220.040790113
				16348 444230.836580578
				16349 444241.632371042
				16350 444252.428161507
				16351 444263.223951972
				16352 444274.019742436
				16353 444284.815532901
				16354 444295.611323365
				16355 444306.40711383
				16356 444317.202904294
				16357 444327.998694759
				16358 444338.794485224
				16359 444349.590275688
				16360 444360.386066153
				16361 444371.181856617
				16362 444381.977647082
				16363 444392.773437546
				16364 444403.569228011
				16365 444414.365018476
				16366 444425.16080894
				16367 444435.956599405
				16368 444446.752389869
				16369 444457.548180334
				16370 444468.343970799
				16371 444479.139761263
				16372 444489.935551728
				16373 444500.731342192
				16374 444511.527132657
				16375 444522.322923121
				16376 444533.118713586
				16377 444543.914504051
				16378 444554.710294515
				16379 444565.50608498
				16380 444576.301875444
				16381 444587.097665909
				16382 444597.893456373
				16383 444608.689246838
				16384 444619.485037303
				16385 444630.280827767
				16386 444641.076618232
				16387 444651.872408696
				16388 444662.668199161
				16389 444673.463989626
				16390 444684.25978009
				16391 444695.055570555
				16392 444705.851361019
				16393 444716.647151484
				16394 444727.442941948
				16395 444738.238732413
				16396 444749.034522878
				16397 444759.830313342
				16398 444770.626103807
				16399 444781.421894271
				16400 444792.217684736
				16401 444803.013475201
				16402 444813.809265665
				16403 444824.60505613
				16404 444835.400846594
				16405 444846.196637059
				16406 444856.992427523
				16407 444867.788217988
				16408 444878.584008453
				16409 444889.379798917
				16410 444900.175589382
				16411 444910.971379846
				16412 444921.767170311
				16413 444932.562960775
				16414 444943.35875124
				16415 444954.154541705
				16416 444964.950332169
				16417 444975.746122634
				16418 444986.541913098
				16419 444997.337703563
				16420 445008.133494028
				16421 445018.929284492
				16422 445029.725074957
				16423 445040.520865421
				16424 445051.316655886
				16425 445062.11244635
				16426 445072.908236815
				16427 445083.70402728
				16428 445094.499817744
				16429 445105.295608209
				16430 445116.091398673
				16431 445126.887189138
				16432 445137.682979602
				16433 445148.478770067
				16434 445159.274560532
				16435 445170.070350996
				16436 445180.866141461
				16437 445191.661931925
				16438 445202.45772239
				16439 445213.253512855
				16440 445224.049303319
				16441 445234.845093784
				16442 445245.640884248
				16443 445256.436674713
				16444 445267.232465177
				16445 445278.028255642
				16446 445288.824046107
				16447 445299.619836571
				16448 445310.415627036
				16449 445321.2114175
				16450 445332.007207965
				16451 445342.80299843
				16452 445353.598788894
				16453 445364.394579359
				16454 445375.190369823
				16455 445385.986160288
				16456 445396.781950752
				16457 445407.577741217
				16458 445418.373531682
				16459 445429.169322146
				16460 445439.965112611
				16461 445450.760903075
				16462 445461.55669354
				16463 445472.352484004
				16464 445483.148274469
				16465 445493.944064934
				16466 445504.739855398
				16467 445515.535645863
				16468 445526.331436327
				16469 445537.127226792
				16470 445547.923017257
				16471 445558.718807721
				16472 445569.514598186
				16473 445580.31038865
				16474 445591.106179115
				16475 445601.901969579
				16476 445612.697760044
				16477 445623.493550509
				16478 445634.289340973
				16479 445645.085131438
				16480 445655.880921902
				16481 445666.676712367
				16482 445677.472502831
				16483 445688.268293296
				16484 445699.064083761
				16485 445709.859874225
				16486 445720.65566469
				16487 445731.451455154
				16488 445742.247245619
				16489 445753.043036084
				16490 445763.838826548
				16491 445774.634617013
				16492 445785.430407477
				16493 445796.226197942
				16494 445807.021988406
				16495 445817.817778871
				16496 445828.613569336
				16497 445839.4093598
				16498 445850.205150265
				16499 445861.000940729
				16500 445871.796731194
				16501 445882.592521659
				16502 445893.388312123
				16503 445904.184102588
				16504 445914.979893052
				16505 445925.775683517
				16506 445936.571473981
				16507 445947.367264446
				16508 445958.163054911
				16509 445968.958845375
				16510 445979.75463584
				16511 445990.550426304
				16512 446001.346216769
				16513 446012.142007233
				16514 446022.937797698
				16515 446033.733588163
				16516 446044.529378627
				16517 446055.325169092
				16518 446066.120959556
				16519 446076.916750021
				16520 446087.712540486
				16521 446098.50833095
				16522 446109.304121415
				16523 446120.099911879
				16524 446130.895702344
				16525 446141.691492808
				16526 446152.487283273
				16527 446163.283073738
				16528 446174.078864202
				16529 446184.874654667
				16530 446195.670445131
				16531 446206.466235596
				16532 446217.26202606
				16533 446228.057816525
				16534 446238.85360699
				16535 446249.649397454
				16536 446260.445187919
				16537 446271.240978383
				16538 446282.036768848
				16539 446292.832559313
				16540 446303.628349777
				16541 446314.424140242
				16542 446325.219930706
				16543 446336.015721171
				16544 446346.811511635
				16545 446357.6073021
				16546 446368.403092565
				16547 446379.198883029
				16548 446389.994673494
				16549 446400.790463958
				16550 446411.586254423
				16551 446422.382044888
				16552 446433.177835352
				16553 446443.973625817
				16554 446454.769416281
				16555 446465.565206746
				16556 446476.36099721
				16557 446487.156787675
				16558 446497.95257814
				16559 446508.748368604
				16560 446519.544159069
				16561 446530.339949533
				16562 446541.135739998
				16563 446551.931530462
				16564 446562.727320927
				16565 446573.523111392
				16566 446584.318901856
				16567 446595.114692321
				16568 446605.910482785
				16569 446616.70627325
				16570 446627.502063714
				16571 446638.297854179
				16572 446649.093644644
				16573 446659.889435108
				16574 446670.685225573
				16575 446681.481016037
				16576 446692.276806502
				16577 446703.072596967
				16578 446713.868387431
				16579 446724.664177896
				16580 446735.45996836
				16581 446746.255758825
				16582 446757.051549289
				16583 446767.847339754
				16584 446778.643130219
				16585 446789.438920683
				16586 446800.234711148
				16587 446811.030501612
				16588 446821.826292077
				16589 446832.622082542
				16590 446843.417873006
				16591 446854.213663471
				16592 446865.009453935
				16593 446875.8052444
				16594 446886.601034864
				16595 446897.396825329
				16596 446908.192615794
				16597 446918.988406258
				16598 446929.784196723
				16599 446940.579987187
				16600 446951.375777652
				16601 446962.171568116
				16602 446972.967358581
				16603 446983.763149046
				16604 446994.55893951
				16605 447005.354729975
				16606 447016.150520439
				16607 447026.946310904
				16608 447037.742101368
				16609 447048.537891833
				16610 447059.333682298
				16611 447070.129472762
				16612 447080.925263227
				16613 447091.721053691
				16614 447102.516844156
				16615 447113.312634621
				16616 447124.108425085
				16617 447134.90421555
				16618 447145.700006014
				16619 447156.495796479
				16620 447167.291586943
				16621 447178.087377408
				16622 447188.883167873
				16623 447199.678958337
				16624 447210.474748802
				16625 447221.270539266
				16626 447232.066329731
				16627 447242.862120196
				16628 447253.65791066
				16629 447264.453701125
				16630 447275.249491589
				16631 447286.045282054
				16632 447296.841072518
				16633 447307.636862983
				16634 447318.432653448
				16635 447329.228443912
				16636 447340.024234377
				16637 447350.820024841
				16638 447361.615815306
				16639 447372.41160577
				16640 447383.207396235
				16641 447394.0031867
				16642 447404.798977164
				16643 447415.594767629
				16644 447426.390558093
				16645 447437.186348558
				16646 447447.982139023
				16647 447458.777929487
				16648 447469.573719952
				16649 447480.369510416
				16650 447491.165300881
				16651 447501.961091345
				16652 447512.75688181
				16653 447523.552672275
				16654 447534.348462739
				16655 447545.144253204
				16656 447555.940043668
				16657 447566.735834133
				16658 447577.531624597
				16659 447588.327415062
				16660 447599.123205527
				16661 447609.918995991
				16662 447620.714786456
				16663 447631.51057692
				16664 447642.306367385
				16665 447653.10215785
				16666 447663.897948314
				16667 447674.693738779
				16668 447685.489529243
				16669 447696.285319708
				16670 447707.081110172
				16671 447717.876900637
				16672 447728.672691102
				16673 447739.468481566
				16674 447750.264272031
				16675 447761.060062495
				16676 447771.85585296
				16677 447782.651643425
				16678 447793.447433889
				16679 447804.243224354
				16680 447815.039014818
				16681 447825.834805283
				16682 447836.630595747
				16683 447847.426386212
				16684 447858.222176677
				16685 447869.017967141
				16686 447879.813757606
				16687 447890.60954807
				16688 447901.405338535
				16689 447912.201128999
				16690 447922.996919464
				16691 447933.792709929
				16692 447944.588500393
				16693 447955.384290858
				16694 447966.180081322
				16695 447976.975871787
				16696 447987.771662252
				16697 447998.567452716
				16698 448009.363243181
				16699 448020.159033645
				16700 448030.95482411
				16701 448041.750614574
				16702 448052.546405039
				16703 448063.342195504
				16704 448074.137985968
				16705 448084.933776433
				16706 448095.729566897
				16707 448106.525357362
				16708 448117.321147827
				16709 448128.116938291
				16710 448138.912728756
				16711 448149.70851922
				16712 448160.504309685
				16713 448171.300100149
				16714 448182.095890614
				16715 448192.891681079
				16716 448203.687471543
				16717 448214.483262008
				16718 448225.279052472
				16719 448236.074842937
				16720 448246.870633401
				16721 448257.666423866
				16722 448268.462214331
				16723 448279.258004795
				16724 448290.05379526
				16725 448300.849585724
				16726 448311.645376189
				16727 448322.441166654
				16728 448333.236957118
				16729 448344.032747583
				16730 448354.828538047
				16731 448365.624328512
				16732 448376.420118976
				16733 448387.215909441
				16734 448398.011699906
				16735 448408.80749037
				16736 448419.603280835
				16737 448430.399071299
				16738 448441.194861764
				16739 448451.990652228
				16740 448462.786442693
				16741 448473.582233158
				16742 448484.378023622
				16743 448495.173814087
				16744 448505.969604551
				16745 448516.765395016
				16746 448527.561185481
				16747 448538.356975945
				16748 448549.15276641
				16749 448559.948556874
				16750 448570.744347339
				16751 448581.540137803
				16752 448592.335928268
				16753 448603.131718733
				16754 448613.927509197
				16755 448624.723299662
				16756 448635.519090126
				16757 448646.314880591
				16758 448657.110671056
				16759 448667.90646152
				16760 448678.702251985
				16761 448689.498042449
				16762 448700.293832914
				16763 448711.089623378
				16764 448721.885413843
				16765 448732.681204308
				16766 448743.476994772
				16767 448754.272785237
				16768 448765.068575701
				16769 448775.864366166
				16770 448786.66015663
				16771 448797.455947095
				16772 448808.25173756
				16773 448819.047528024
				16774 448829.843318489
				16775 448840.639108953
				16776 448851.434899418
				16777 448862.230689883
				16778 448873.026480347
				16779 448883.822270812
				16780 448894.618061276
				16781 448905.413851741
				16782 448916.209642205
				16783 448927.00543267
				16784 448937.801223135
				16785 448948.597013599
				16786 448959.392804064
				16787 448970.188594528
				16788 448980.984384993
				16789 448991.780175457
				16790 449002.575965922
				16791 449013.371756387
				16792 449024.167546851
				16793 449034.963337316
				16794 449045.75912778
				16795 449056.554918245
				16796 449067.35070871
				16797 449078.146499174
				16798 449088.942289639
				16799 449099.738080103
				16800 449110.533870568
				16801 449121.329661032
				16802 449132.125451497
				16803 449142.921241962
				16804 449153.717032426
				16805 449164.512822891
				16806 449175.308613355
				16807 449186.10440382
				16808 449196.900194285
				16809 449207.695984749
				16810 449218.491775214
				16811 449229.287565678
				16812 449240.083356143
				16813 449250.879146607
				16814 449261.674937072
				16815 449272.470727537
				16816 449283.266518001
				16817 449294.062308466
				16818 449304.85809893
				16819 449315.653889395
				16820 449326.449679859
				16821 449337.245470324
				16822 449348.041260789
				16823 449358.837051253
				16824 449369.632841718
				16825 449380.428632182
				16826 449391.224422647
				16827 449402.020213112
				16828 449412.816003576
				16829 449423.611794041
				16830 449434.407584505
				16831 449445.20337497
				16832 449455.999165434
				16833 449466.794955899
				16834 449477.590746364
				16835 449488.386536828
				16836 449499.182327293
				16837 449509.978117757
				16838 449520.773908222
				16839 449531.569698686
				16840 449542.365489151
				16841 449553.161279616
				16842 449563.95707008
				16843 449574.752860545
				16844 449585.548651009
				16845 449596.344441474
				16846 449607.140231939
				16847 449617.936022403
				16848 449628.731812868
				16849 449639.527603332
				16850 449650.323393797
				16851 449661.119184261
				16852 449671.914974726
				16853 449682.710765191
				16854 449693.506555655
				16855 449704.30234612
				16856 449715.098136584
				16857 449725.893927049
				16858 449736.689717514
				16859 449747.485507978
				16860 449758.281298443
				16861 449769.077088907
				16862 449779.872879372
				16863 449790.668669836
				16864 449801.464460301
				16865 449812.260250766
				16866 449823.05604123
				16867 449833.851831695
				16868 449844.647622159
				16869 449855.443412624
				16870 449866.239203088
				16871 449877.034993553
				16872 449887.830784018
				16873 449898.626574482
				16874 449909.422364947
				16875 449920.218155411
				16876 449931.013945876
				16877 449941.809736341
				16878 449952.605526805
				16879 449963.40131727
				16880 449974.197107734
				16881 449984.992898199
				16882 449995.788688663
				16883 450006.584479128
				16884 450017.380269593
				16885 450028.176060057
				16886 450038.971850522
				16887 450049.767640986
				16888 450060.563431451
				16889 450071.359221915
				16890 450082.15501238
				16891 450092.950802845
				16892 450103.746593309
				16893 450114.542383774
				16894 450125.338174238
				16895 450136.133964703
				16896 450146.929755168
				16897 450157.725545632
				16898 450168.521336097
				16899 450179.317126561
				16900 450190.112917026
				16901 450200.90870749
				16902 450211.704497955
				16903 450222.50028842
				16904 450233.296078884
				16905 450244.091869349
				16906 450254.887659813
				16907 450265.683450278
				16908 450276.479240743
				16909 450287.275031207
				16910 450298.070821672
				16911 450308.866612136
				16912 450319.662402601
				16913 450330.458193065
				16914 450341.25398353
				16915 450352.049773995
				16916 450362.845564459
				16917 450373.641354924
				16918 450384.437145388
				16919 450395.232935853
				16920 450406.028726317
				16921 450416.824516782
				16922 450427.620307247
				16923 450438.416097711
				16924 450449.211888176
				16925 450460.00767864
				16926 450470.803469105
				16927 450481.599259569
				16928 450492.395050034
				16929 450503.190840499
				16930 450513.986630963
				16931 450524.782421428
				16932 450535.578211892
				16933 450546.374002357
				16934 450557.169792822
				16935 450567.965583286
				16936 450578.761373751
				16937 450589.557164215
				16938 450600.35295468
				16939 450611.148745144
				16940 450621.944535609
				16941 450632.740326074
				16942 450643.536116538
				16943 450654.331907003
				16944 450665.127697467
				16945 450675.923487932
				16946 450686.719278397
				16947 450697.515068861
				16948 450708.310859326
				16949 450719.10664979
				16950 450729.902440255
				16951 450740.698230719
				16952 450751.494021184
				16953 450762.289811649
				16954 450773.085602113
				16955 450783.881392578
				16956 450794.677183042
				16957 450805.472973507
				16958 450816.268763971
				16959 450827.064554436
				16960 450837.860344901
				16961 450848.656135365
				16962 450859.45192583
				16963 450870.247716294
				16964 450881.043506759
				16965 450891.839297223
				16966 450902.635087688
				16967 450913.430878153
				16968 450924.226668617
				16969 450935.022459082
				16970 450945.818249546
				16971 450956.614040011
				16972 450967.409830476
				16973 450978.20562094
				16974 450989.001411405
				16975 450999.797201869
				16976 451010.592992334
				16977 451021.388782798
				16978 451032.184573263
				16979 451042.980363728
				16980 451053.776154192
				16981 451064.571944657
				16982 451075.367735121
				16983 451086.163525586
				16984 451096.959316051
				16985 451107.755106515
				16986 451118.55089698
				16987 451129.346687444
				16988 451140.142477909
				16989 451150.938268373
				16990 451161.734058838
				16991 451172.529849303
				16992 451183.325639767
				16993 451194.121430232
				16994 451204.917220696
				16995 451215.713011161
				16996 451226.508801625
				16997 451237.30459209
				16998 451248.100382555
				16999 451258.896173019
				17000 451269.691963484
				17001 451280.487753948
				17002 451291.283544413
				17003 451302.079334878
				17004 451312.875125342
				17005 451323.670915807
				17006 451334.466706271
				17007 451345.262496736
				17008 451356.0582872
				17009 451366.854077665
				17010 451377.64986813
				17011 451388.445658594
				17012 451399.241449059
				17013 451410.037239523
				17014 451420.833029988
				17015 451431.628820452
				17016 451442.424610917
				17017 451453.220401382
				17018 451464.016191846
				17019 451474.811982311
				17020 451485.607772775
				17021 451496.40356324
				17022 451507.199353705
				17023 451517.995144169
				17024 451528.790934634
				17025 451539.586725098
				17026 451550.382515563
				17027 451561.178306027
				17028 451571.974096492
				17029 451582.769886957
				17030 451593.565677421
				17031 451604.361467886
				17032 451615.15725835
				17033 451625.953048815
				17034 451636.74883928
				17035 451647.544629744
				17036 451658.340420209
				17037 451669.136210673
				17038 451679.932001138
				17039 451690.727791602
				17040 451701.523582067
				17041 451712.319372532
				17042 451723.115162996
				17043 451733.910953461
				17044 451744.706743925
				17045 451755.50253439
				17046 451766.298324854
				17047 451777.094115319
				17048 451787.889905784
				17049 451798.685696248
				17050 451809.481486713
				17051 451820.277277177
				17052 451831.073067642
				17053 451841.868858107
				17054 451852.664648571
				17055 451863.460439036
				17056 451874.2562295
				17057 451885.052019965
				17058 451895.847810429
				17059 451906.643600894
				17060 451917.439391359
				17061 451928.235181823
				17062 451939.030972288
				17063 451949.826762752
				17064 451960.622553217
				17065 451971.418343681
				17066 451982.214134146
				17067 451993.009924611
				17068 452003.805715075
				17069 452014.60150554
				17070 452025.397296004
				17071 452036.193086469
				17072 452046.988876934
				17073 452057.784667398
				17074 452068.580457863
				17075 452079.376248327
				17076 452090.172038792
				17077 452100.967829256
				17078 452111.763619721
				17079 452122.559410186
				17080 452133.35520065
				17081 452144.150991115
				17082 452154.946781579
				17083 452165.742572044
				17084 452176.538362509
				17085 452187.334152973
				17086 452198.129943438
				17087 452208.925733902
				17088 452219.721524367
				17089 452230.517314831
				17090 452241.313105296
				17091 452252.108895761
				17092 452262.904686225
				17093 452273.70047669
				17094 452284.496267154
				17095 452295.292057619
				17096 452306.087848083
				17097 452316.883638548
				17098 452327.679429013
				17099 452338.475219477
				17100 452349.271009942
				17101 452360.066800406
				17102 452370.862590871
				17103 452381.658381336
				17104 452392.4541718
				17105 452403.249962265
				17106 452414.045752729
				17107 452424.841543194
				17108 452435.637333658
				17109 452446.433124123
				17110 452457.228914588
				17111 452468.024705052
				17112 452478.820495517
				17113 452489.616285981
				17114 452500.412076446
				17115 452511.20786691
				17116 452522.003657375
				17117 452532.79944784
				17118 452543.595238304
				17119 452554.391028769
				17120 452565.186819233
				17121 452575.982609698
				17122 452586.778400163
				17123 452597.574190627
				17124 452608.369981092
				17125 452619.165771556
				17126 452629.961562021
				17127 452640.757352485
				17128 452651.55314295
				17129 452662.348933415
				17130 452673.144723879
				17131 452683.940514344
				17132 452694.736304808
				17133 452705.532095273
				17134 452716.327885738
				17135 452727.123676202
				17136 452737.919466667
				17137 452748.715257131
				17138 452759.511047596
				17139 452770.30683806
				17140 452781.102628525
				17141 452791.89841899
				17142 452802.694209454
				17143 452813.489999919
				17144 452824.285790383
				17145 452835.081580848
				17146 452845.877371312
				17147 452856.673161777
				17148 452867.468952242
				17149 452878.264742706
				17150 452889.060533171
				17151 452899.856323635
				17152 452910.6521141
				17153 452921.447904565
				17154 452932.243695029
				17155 452943.039485494
				17156 452953.835275958
				17157 452964.631066423
				17158 452975.426856887
				17159 452986.222647352
				17160 452997.018437817
				17161 453007.814228281
				17162 453018.610018746
				17163 453029.40580921
				17164 453040.201599675
				17165 453050.997390139
				17166 453061.793180604
				17167 453072.588971069
				17168 453083.384761533
				17169 453094.180551998
				17170 453104.976342462
				17171 453115.772132927
				17172 453126.567923392
				17173 453137.363713856
				17174 453148.159504321
				17175 453158.955294785
				17176 453169.75108525
				17177 453180.546875714
				17178 453191.342666179
				17179 453202.138456644
				17180 453212.934247108
				17181 453223.730037573
				17182 453234.525828037
				17183 453245.321618502
				17184 453256.117408967
				17185 453266.913199431
				17186 453277.708989896
				17187 453288.50478036
				17188 453299.300570825
				17189 453310.096361289
				17190 453320.892151754
				17191 453331.687942219
				17192 453342.483732683
				17193 453353.279523148
				17194 453364.075313612
				17195 453374.871104077
				17196 453385.666894541
				17197 453396.462685006
				17198 453407.258475471
				17199 453418.054265935
				17200 453428.8500564
				17201 453439.645846864
				17202 453450.441637329
				17203 453461.237427794
				17204 453472.033218258
				17205 453482.829008723
				17206 453493.624799187
				17207 453504.420589652
				17208 453515.216380116
				17209 453526.012170581
				17210 453536.807961046
				17211 453547.60375151
				17212 453558.399541975
				17213 453569.195332439
				17214 453579.991122904
				17215 453590.786913369
				17216 453601.582703833
				17217 453612.378494298
				17218 453623.174284762
				17219 453633.970075227
				17220 453644.765865691
				17221 453655.561656156
				17222 453666.357446621
				17223 453677.153237085
				17224 453687.94902755
				17225 453698.744818014
				17226 453709.540608479
				17227 453720.336398943
				17228 453731.132189408
				17229 453741.927979873
				17230 453752.723770337
				17231 453763.519560802
				17232 453774.315351266
				17233 453785.111141731
				17234 453795.906932196
				17235 453806.70272266
				17236 453817.498513125
				17237 453828.294303589
				17238 453839.090094054
				17239 453849.885884518
				17240 453860.681674983
				17241 453871.477465448
				17242 453882.273255912
				17243 453893.069046377
				17244 453903.864836841
				17245 453914.660627306
				17246 453925.45641777
				17247 453936.252208235
				17248 453947.0479987
				17249 453957.843789164
				17250 453968.639579629
				17251 453979.435370093
				17252 453990.231160558
				17253 454001.026951023
				17254 454011.822741487
				17255 454022.618531952
				17256 454033.414322416
				17257 454044.210112881
				17258 454055.005903345
				17259 454065.80169381
				17260 454076.597484275
				17261 454087.393274739
				17262 454098.189065204
				17263 454108.984855668
				17264 454119.780646133
				17265 454130.576436598
				17266 454141.372227062
				17267 454152.168017527
				17268 454162.963807991
				17269 454173.759598456
				17270 454184.55538892
				17271 454195.351179385
				17272 454206.14696985
				17273 454216.942760314
				17274 454227.738550779
				17275 454238.534341243
				17276 454249.330131708
				17277 454260.125922172
				17278 454270.921712637
				17279 454281.717503102
				17280 454292.513293566
				17281 454303.309084031
				17282 454314.104874495
				17283 454324.90066496
				17284 454335.696455424
				17285 454346.492245889
				17286 454357.288036354
				17287 454368.083826818
				17288 454378.879617283
				17289 454389.675407747
				17290 454400.471198212
				17291 454411.266988677
				17292 454422.062779141
				17293 454432.858569606
				17294 454443.65436007
				17295 454454.450150535
				17296 454465.245940999
				17297 454476.041731464
				17298 454486.837521929
				17299 454497.633312393
				17300 454508.429102858
				17301 454519.224893322
				17302 454530.020683787
				17303 454540.816474252
				17304 454551.612264716
				17305 454562.408055181
				17306 454573.203845645
				17307 454583.99963611
				17308 454594.795426574
				17309 454605.591217039
				17310 454616.387007504
				17311 454627.182797968
				17312 454637.978588433
				17313 454648.774378897
				17314 454659.570169362
				17315 454670.365959826
				17316 454681.161750291
				17317 454691.957540756
				17318 454702.75333122
				17319 454713.549121685
				17320 454724.344912149
				17321 454735.140702614
				17322 454745.936493078
				17323 454756.732283543
				17324 454767.528074008
				17325 454778.323864472
				17326 454789.119654937
				17327 454799.915445401
				17328 454810.711235866
				17329 454821.507026331
				17330 454832.302816795
				17331 454843.09860726
				17332 454853.894397724
				17333 454864.690188189
				17334 454875.485978653
				17335 454886.281769118
				17336 454897.077559583
				17337 454907.873350047
				17338 454918.669140512
				17339 454929.464930976
				17340 454940.260721441
				17341 454951.056511906
				17342 454961.85230237
				17343 454972.648092835
				17344 454983.443883299
				17345 454994.239673764
				17346 455005.035464228
				17347 455015.831254693
				17348 455026.627045158
				17349 455037.422835622
				17350 455048.218626087
				17351 455059.014416551
				17352 455069.810207016
				17353 455080.60599748
				17354 455091.401787945
				17355 455102.19757841
				17356 455112.993368874
				17357 455123.789159339
				17358 455134.584949803
				17359 455145.380740268
				17360 455156.176530733
				17361 455166.972321197
				17362 455177.768111662
				17363 455188.563902126
				17364 455199.359692591
				17365 455210.155483055
				17366 455220.95127352
				17367 455231.747063985
				17368 455242.542854449
				17369 455253.338644914
				17370 455264.134435378
				17371 455274.930225843
				17372 455285.726016307
				17373 455296.521806772
				17374 455307.317597237
				17375 455318.113387701
				17376 455328.909178166
				17377 455339.70496863
				17378 455350.500759095
				17379 455361.29654956
				17380 455372.092340024
				17381 455382.888130489
				17382 455393.683920953
				17383 455404.479711418
				17384 455415.275501882
				17385 455426.071292347
				17386 455436.867082812
				17387 455447.662873276
				17388 455458.458663741
				17389 455469.254454205
				17390 455480.05024467
				17391 455490.846035135
				17392 455501.641825599
				17393 455512.437616064
				17394 455523.233406528
				17395 455534.029196993
				17396 455544.824987457
				17397 455555.620777922
				17398 455566.416568387
				17399 455577.212358851
				17400 455588.008149316
				17401 455598.80393978
				17402 455609.599730245
				17403 455620.395520709
				17404 455631.191311174
				17405 455641.987101639
				17406 455652.782892103
				17407 455663.578682568
				17408 455674.374473032
				17409 455685.170263497
				17410 455695.966053962
				17411 455706.761844426
				17412 455717.557634891
				17413 455728.353425355
				17414 455739.14921582
				17415 455749.945006284
				17416 455760.740796749
				17417 455771.536587214
				17418 455782.332377678
				17419 455793.128168143
				17420 455803.923958607
				17421 455814.719749072
				17422 455825.515539536
				17423 455836.311330001
				17424 455847.107120466
				17425 455857.90291093
				17426 455868.698701395
				17427 455879.494491859
				17428 455890.290282324
				17429 455901.086072789
				17430 455911.881863253
				17431 455922.677653718
				17432 455933.473444182
				17433 455944.269234647
				17434 455955.065025111
				17435 455965.860815576
				17436 455976.656606041
				17437 455987.452396505
				17438 455998.24818697
				17439 456009.043977434
				17440 456019.839767899
				17441 456030.635558364
				17442 456041.431348828
				17443 456052.227139293
				17444 456063.022929757
				17445 456073.818720222
				17446 456084.614510686
				17447 456095.410301151
				17448 456106.206091616
				17449 456117.00188208
				17450 456127.797672545
				17451 456138.593463009
				17452 456149.389253474
				17453 456160.185043938
				17454 456170.980834403
				17455 456181.776624868
				17456 456192.572415332
				17457 456203.368205797
				17458 456214.163996261
				17459 456224.959786726
				17460 456235.755577191
				17461 456246.551367655
				17462 456257.34715812
				17463 456268.142948584
				17464 456278.938739049
				17465 456289.734529513
				17466 456300.530319978
				17467 456311.326110443
				17468 456322.121900907
				17469 456332.917691372
				17470 456343.713481836
				17471 456354.509272301
				17472 456365.305062765
				17473 456376.10085323
				17474 456386.896643695
				17475 456397.692434159
				17476 456408.488224624
				17477 456419.284015088
				17478 456430.079805553
				17479 456440.875596018
				17480 456451.671386482
				17481 456462.467176947
				17482 456473.262967411
				17483 456484.058757876
				17484 456494.85454834
				17485 456505.650338805
				17486 456516.44612927
				17487 456527.241919734
				17488 456538.037710199
				17489 456548.833500663
				17490 456559.629291128
				17491 456570.425081593
				17492 456581.220872057
				17493 456592.016662522
				17494 456602.812452986
				17495 456613.608243451
				17496 456624.404033915
				17497 456635.19982438
				17498 456645.995614845
				17499 456656.791405309
				17500 456667.587195774
				17501 456678.382986238
				17502 456689.178776703
				17503 456699.974567167
				17504 456710.770357632
				17505 456721.566148097
				17506 456732.361938561
				17507 456743.157729026
				17508 456753.95351949
				17509 456764.749309955
				17510 456775.54510042
				17511 456786.340890884
				17512 456797.136681349
				17513 456807.932471813
				17514 456818.728262278
				17515 456829.524052742
				17516 456840.319843207
				17517 456851.115633672
				17518 456861.911424136
				17519 456872.707214601
				17520 456883.503005065
				17521 456894.29879553
				17522 456905.094585994
				17523 456915.890376459
				17524 456926.686166924
				17525 456937.481957388
				17526 456948.277747853
				17527 456959.073538317
				17528 456969.869328782
				17529 456980.665119247
				17530 456991.460909711
				17531 457002.256700176
				17532 457013.05249064
				17533 457023.848281105
				17534 457034.644071569
				17535 457045.439862034
				17536 457056.235652499
				17537 457067.031442963
				17538 457077.827233428
				17539 457088.623023892
				17540 457099.418814357
				17541 457110.214604822
				17542 457121.010395286
				17543 457131.806185751
				17544 457142.601976215
				17545 457153.39776668
				17546 457164.193557144
				17547 457174.989347609
				17548 457185.785138074
				17549 457196.580928538
				17550 457207.376719003
				17551 457218.172509467
				17552 457228.968299932
				17553 457239.764090396
				17554 457250.559880861
				17555 457261.355671326
				17556 457272.15146179
				17557 457282.947252255
				17558 457293.743042719
				17559 457304.538833184
				17560 457315.334623648
				17561 457326.130414113
				17562 457336.926204578
				17563 457347.721995042
				17564 457358.517785507
				17565 457369.313575971
				17566 457380.109366436
				17567 457390.905156901
				17568 457401.700947365
				17569 457412.49673783
				17570 457423.292528294
				17571 457434.088318759
				17572 457444.884109223
				17573 457455.679899688
				17574 457466.475690153
				17575 457477.271480617
				17576 457488.067271082
				17577 457498.863061546
				17578 457509.658852011
				17579 457520.454642476
				17580 457531.25043294
				17581 457542.046223405
				17582 457552.842013869
				17583 457563.637804334
				17584 457574.433594798
				17585 457585.229385263
				17586 457596.025175728
				17587 457606.820966192
				17588 457617.616756657
				17589 457628.412547121
				17590 457639.208337586
				17591 457650.004128051
				17592 457660.799918515
				17593 457671.59570898
				17594 457682.391499444
				17595 457693.187289909
				17596 457703.983080373
				17597 457714.778870838
				17598 457725.574661303
				17599 457736.370451767
				17600 457747.166242232
				17601 457757.962032696
				17602 457768.757823161
				17603 457779.553613625
				17604 457790.34940409
				17605 457801.145194555
				17606 457811.940985019
				17607 457822.736775484
				17608 457833.532565948
				17609 457844.328356413
				17610 457855.124146877
				17611 457865.919937342
				17612 457876.715727807
				17613 457887.511518271
				17614 457898.307308736
				17615 457909.1030992
				17616 457919.898889665
				17617 457930.69468013
				17618 457941.490470594
				17619 457952.286261059
				17620 457963.082051523
				17621 457973.877841988
				17622 457984.673632452
				17623 457995.469422917
				17624 458006.265213382
				17625 458017.061003846
				17626 458027.856794311
				17627 458038.652584775
				17628 458049.44837524
				17629 458060.244165705
				17630 458071.039956169
				17631 458081.835746634
				17632 458092.631537098
				17633 458103.427327563
				17634 458114.223118027
				17635 458125.018908492
				17636 458135.814698957
				17637 458146.610489421
				17638 458157.406279886
				17639 458168.20207035
				17640 458178.997860815
				17641 458189.793651279
				17642 458200.589441744
				17643 458211.385232209
				17644 458222.181022673
				17645 458232.976813138
				17646 458243.772603602
				17647 458254.568394067
				17648 458265.364184531
				17649 458276.159974996
				17650 458286.955765461
				17651 458297.751555925
				17652 458308.54734639
				17653 458319.343136854
				17654 458330.138927319
				17655 458340.934717784
				17656 458351.730508248
				17657 458362.526298713
				17658 458373.322089177
				17659 458384.117879642
				17660 458394.913670106
				17661 458405.709460571
				17662 458416.505251036
				17663 458427.3010415
				17664 458438.096831965
				17665 458448.892622429
				17666 458459.688412894
				17667 458470.484203359
				17668 458481.279993823
				17669 458492.075784288
				17670 458502.871574752
				17671 458513.667365217
				17672 458524.463155681
				17673 458535.258946146
				17674 458546.054736611
				17675 458556.850527075
				17676 458567.64631754
				17677 458578.442108004
				17678 458589.237898469
				17679 458600.033688933
				17680 458610.829479398
				17681 458621.625269863
				17682 458632.421060327
				17683 458643.216850792
				17684 458654.012641256
				17685 458664.808431721
				17686 458675.604222186
				17687 458686.40001265
				17688 458697.195803115
				17689 458707.991593579
				17690 458718.787384044
				17691 458729.583174508
				17692 458740.378964973
				17693 458751.174755438
				17694 458761.970545902
				17695 458772.766336367
				17696 458783.562126831
				17697 458794.357917296
				17698 458805.153707761
				17699 458815.949498225
				17700 458826.74528869
				17701 458837.541079154
				17702 458848.336869619
				17703 458859.132660083
				17704 458869.928450548
				17705 458880.724241013
				17706 458891.520031477
				17707 458902.315821942
				17708 458913.111612406
				17709 458923.907402871
				17710 458934.703193335
				17711 458945.4989838
				17712 458956.294774265
				17713 458967.090564729
				17714 458977.886355194
				17715 458988.682145658
				17716 458999.477936123
				17717 459010.273726588
				17718 459021.069517052
				17719 459031.865307517
				17720 459042.661097981
				17721 459053.456888446
				17722 459064.25267891
				17723 459075.048469375
				17724 459085.84425984
				17725 459096.640050304
				17726 459107.435840769
				17727 459118.231631233
				17728 459129.027421698
				17729 459139.823212162
				17730 459150.619002627
				17731 459161.414793092
				17732 459172.210583556
				17733 459183.006374021
				17734 459193.802164485
				17735 459204.59795495
				17736 459215.393745415
				17737 459226.189535879
				17738 459236.985326344
				17739 459247.781116808
				17740 459258.576907273
				17741 459269.372697737
				17742 459280.168488202
				17743 459290.964278667
				17744 459301.760069131
				17745 459312.555859596
				17746 459323.35165006
				17747 459334.147440525
				17748 459344.94323099
				17749 459355.739021454
				17750 459366.534811919
				17751 459377.330602383
				17752 459388.126392848
				17753 459398.922183312
				17754 459409.717973777
				17755 459420.513764242
				17756 459431.309554706
				17757 459442.105345171
				17758 459452.901135635
				17759 459463.6969261
				17760 459474.492716564
				17761 459485.288507029
				17762 459496.084297494
				17763 459506.880087958
				17764 459517.675878423
				17765 459528.471668887
				17766 459539.267459352
				17767 459550.063249817
				17768 459560.859040281
				17769 459571.654830746
				17770 459582.45062121
				17771 459593.246411675
				17772 459604.042202139
				17773 459614.837992604
				17774 459625.633783069
				17775 459636.429573533
				17776 459647.225363998
				17777 459658.021154462
				17778 459668.816944927
				17779 459679.612735391
				17780 459690.408525856
				17781 459701.204316321
				17782 459712.000106785
				17783 459722.79589725
				17784 459733.591687714
				17785 459744.387478179
				17786 459755.183268644
				17787 459765.979059108
				17788 459776.774849573
				17789 459787.570640037
				17790 459798.366430502
				17791 459809.162220966
				17792 459819.958011431
				17793 459830.753801896
				17794 459841.54959236
				17795 459852.345382825
				17796 459863.141173289
				17797 459873.936963754
				17798 459884.732754219
				17799 459895.528544683
				17800 459906.324335148
				17801 459917.120125612
				17802 459927.915916077
				17803 459938.711706541
				17804 459949.507497006
				17805 459960.303287471
				17806 459971.099077935
				17807 459981.8948684
				17808 459992.690658864
				17809 460003.486449329
				17810 460014.282239793
				17811 460025.078030258
				17812 460035.873820723
				17813 460046.669611187
				17814 460057.465401652
				17815 460068.261192116
				17816 460079.056982581
				17817 460089.852773046
				17818 460100.64856351
				17819 460111.444353975
				17820 460122.240144439
				17821 460133.035934904
				17822 460143.831725368
				17823 460154.627515833
				17824 460165.423306298
				17825 460176.219096762
				17826 460187.014887227
				17827 460197.810677691
				17828 460208.606468156
				17829 460219.40225862
				17830 460230.198049085
				17831 460240.99383955
				17832 460251.789630014
				17833 460262.585420479
				17834 460273.381210943
				17835 460284.177001408
				17836 460294.972791873
				17837 460305.768582337
				17838 460316.564372802
				17839 460327.360163266
				17840 460338.155953731
				17841 460348.951744195
				17842 460359.74753466
				17843 460370.543325125
				17844 460381.339115589
				17845 460392.134906054
				17846 460402.930696518
				17847 460413.726486983
				17848 460424.522277448
				17849 460435.318067912
				17850 460446.113858377
				17851 460456.909648841
				17852 460467.705439306
				17853 460478.50122977
				17854 460489.297020235
				17855 460500.0928107
				17856 460510.888601164
				17857 460521.684391629
				17858 460532.480182093
				17859 460543.275972558
				17860 460554.071763022
				17861 460564.867553487
				17862 460575.663343952
				17863 460586.459134416
				17864 460597.254924881
				17865 460608.050715345
				17866 460618.84650581
				17867 460629.642296275
				17868 460640.438086739
				17869 460651.233877204
				17870 460662.029667668
				17871 460672.825458133
				17872 460683.621248597
				17873 460694.417039062
				17874 460705.212829527
				17875 460716.008619991
				17876 460726.804410456
				17877 460737.60020092
				17878 460748.395991385
				17879 460759.191781849
				17880 460769.987572314
				17881 460780.783362779
				17882 460791.579153243
				17883 460802.374943708
				17884 460813.170734172
				17885 460823.966524637
				17886 460834.762315102
				17887 460845.558105566
				17888 460856.353896031
				17889 460867.149686495
				17890 460877.94547696
				17891 460888.741267424
				17892 460899.537057889
				17893 460910.332848354
				17894 460921.128638818
				17895 460931.924429283
				17896 460942.720219747
				17897 460953.516010212
				17898 460964.311800677
				17899 460975.107591141
				17900 460985.903381606
				17901 460996.69917207
				17902 461007.494962535
				17903 461018.290752999
				17904 461029.086543464
				17905 461039.882333929
				17906 461050.678124393
				17907 461061.473914858
				17908 461072.269705322
				17909 461083.065495787
				17910 461093.861286251
				17911 461104.657076716
				17912 461115.452867181
				17913 461126.248657645
				17914 461137.04444811
				17915 461147.840238574
				17916 461158.636029039
				17917 461169.431819503
				17918 461180.227609968
				17919 461191.023400433
				17920 461201.819190897
				17921 461212.614981362
				17922 461223.410771826
				17923 461234.206562291
				17924 461245.002352756
				17925 461255.79814322
				17926 461266.593933685
				17927 461277.389724149
				17928 461288.185514614
				17929 461298.981305078
				17930 461309.777095543
				17931 461320.572886008
				17932 461331.368676472
				17933 461342.164466937
				17934 461352.960257401
				17935 461363.756047866
				17936 461374.551838331
				17937 461385.347628795
				17938 461396.14341926
				17939 461406.939209724
				17940 461417.735000189
				17941 461428.530790653
				17942 461439.326581118
				17943 461450.122371583
				17944 461460.918162047
				17945 461471.713952512
				17946 461482.509742976
				17947 461493.305533441
				17948 461504.101323905
				17949 461514.89711437
				17950 461525.692904835
				17951 461536.488695299
				17952 461547.284485764
				17953 461558.080276228
				17954 461568.876066693
				17955 461579.671857157
				17956 461590.467647622
				17957 461601.263438087
				17958 461612.059228551
				17959 461622.855019016
				17960 461633.65080948
				17961 461644.446599945
				17962 461655.24239041
				17963 461666.038180874
				17964 461676.833971339
				17965 461687.629761803
				17966 461698.425552268
				17967 461709.221342732
				17968 461720.017133197
				17969 461730.812923662
				17970 461741.608714126
				17971 461752.404504591
				17972 461763.200295055
				17973 461773.99608552
				17974 461784.791875985
				17975 461795.587666449
				17976 461806.383456914
				17977 461817.179247378
				17978 461827.975037843
				17979 461838.770828307
				17980 461849.566618772
				17981 461860.362409237
				17982 461871.158199701
				17983 461881.953990166
				17984 461892.74978063
				17985 461903.545571095
				17986 461914.34136156
				17987 461925.137152024
				17988 461935.932942489
				17989 461946.728732953
				17990 461957.524523418
				17991 461968.320313882
				17992 461979.116104347
				17993 461989.911894812
				17994 462000.707685276
				17995 462011.503475741
				17996 462022.299266205
				17997 462033.09505667
				17998 462043.890847134
				17999 462054.686637599
				18000 462065.482428064
				18001 462076.278218528
				18002 462087.074008993
				18003 462097.869799457
				18004 462108.665589922
				18005 462119.461380386
				18006 462130.257170851
				18007 462141.052961316
				18008 462151.84875178
				18009 462162.644542245
				18010 462173.440332709
				18011 462184.236123174
				18012 462195.031913639
				18013 462205.827704103
				18014 462216.623494568
				18015 462227.419285032
				18016 462238.215075497
				18017 462249.010865961
				18018 462259.806656426
				18019 462270.602446891
				18020 462281.398237355
				18021 462292.19402782
				18022 462302.989818284
				18023 462313.785608749
				18024 462324.581399214
				18025 462335.377189678
				18026 462346.172980143
				18027 462356.968770607
				18028 462367.764561072
				18029 462378.560351536
				18030 462389.356142001
				18031 462400.151932466
				18032 462410.94772293
				18033 462421.743513395
				18034 462432.539303859
				18035 462443.335094324
				18036 462454.130884788
				18037 462464.926675253
				18038 462475.722465718
				18039 462486.518256182
				18040 462497.314046647
				18041 462508.109837111
				18042 462518.905627576
				18043 462529.701418041
				18044 462540.497208505
				18045 462551.29299897
				18046 462562.088789434
				18047 462572.884579899
				18048 462583.680370363
				18049 462594.476160828
				18050 462605.271951293
				18051 462616.067741757
				18052 462626.863532222
				18053 462637.659322686
				18054 462648.455113151
				18055 462659.250903615
				18056 462670.04669408
				18057 462680.842484545
				18058 462691.638275009
				18059 462702.434065474
				18060 462713.229855938
				18061 462724.025646403
				18062 462734.821436868
				18063 462745.617227332
				18064 462756.413017797
				18065 462767.208808261
				18066 462778.004598726
				18067 462788.80038919
				18068 462799.596179655
				18069 462810.39197012
				18070 462821.187760584
				18071 462831.983551049
				18072 462842.779341513
				18073 462853.575131978
				18074 462864.370922443
				18075 462875.166712907
				18076 462885.962503372
				18077 462896.758293836
				18078 462907.554084301
				18079 462918.349874765
				18080 462929.14566523
				18081 462939.941455695
				18082 462950.737246159
				18083 462961.533036624
				18084 462972.328827088
				18085 462983.124617553
				18086 462993.920408017
				18087 463004.716198482
				18088 463015.511988947
				18089 463026.307779411
				18090 463037.103569876
				18091 463047.89936034
				18092 463058.695150805
				18093 463069.49094127
				18094 463080.286731734
				18095 463091.082522199
				18096 463101.878312663
				18097 463112.674103128
				18098 463123.469893592
				18099 463134.265684057
				18100 463145.061474522
				18101 463155.857264986
				18102 463166.653055451
				18103 463177.448845915
				18104 463188.24463638
				18105 463199.040426844
				18106 463209.836217309
				18107 463220.632007774
				18108 463231.427798238
				18109 463242.223588703
				18110 463253.019379167
				18111 463263.815169632
				18112 463274.610960097
				18113 463285.406750561
				18114 463296.202541026
				18115 463306.99833149
				18116 463317.794121955
				18117 463328.589912419
				18118 463339.385702884
				18119 463350.181493349
				18120 463360.977283813
				18121 463371.773074278
				18122 463382.568864742
				18123 463393.364655207
				18124 463404.160445672
				18125 463414.956236136
				18126 463425.752026601
				18127 463436.547817065
				18128 463447.34360753
				18129 463458.139397994
				18130 463468.935188459
				18131 463479.730978924
				18132 463490.526769388
				18133 463501.322559853
				18134 463512.118350317
				18135 463522.914140782
				18136 463533.709931246
				18137 463544.505721711
				18138 463555.301512176
				18139 463566.09730264
				18140 463576.893093105
				18141 463587.688883569
				18142 463598.484674034
				18143 463609.280464499
				18144 463620.076254963
				18145 463630.872045428
				18146 463641.667835892
				18147 463652.463626357
				18148 463663.259416821
				18149 463674.055207286
				18150 463684.850997751
				18151 463695.646788215
				18152 463706.44257868
				18153 463717.238369144
				18154 463728.034159609
				18155 463738.829950074
				18156 463749.625740538
				18157 463760.421531003
				18158 463771.217321467
				18159 463782.013111932
				18160 463792.808902396
				18161 463803.604692861
				18162 463814.400483326
				18163 463825.19627379
				18164 463835.992064255
				18165 463846.787854719
				18166 463857.583645184
				18167 463868.379435648
				18168 463879.175226113
				18169 463889.971016578
				18170 463900.766807042
				18171 463911.562597507
				18172 463922.358387971
				18173 463933.154178436
				18174 463943.949968901
				18175 463954.745759365
				18176 463965.54154983
				18177 463976.337340294
				18178 463987.133130759
				18179 463997.928921223
				18180 464008.724711688
				18181 464019.520502153
				18182 464030.316292617
				18183 464041.112083082
				18184 464051.907873546
				18185 464062.703664011
				18186 464073.499454475
				18187 464084.29524494
				18188 464095.091035405
				18189 464105.886825869
				18190 464116.682616334
				18191 464127.478406798
				18192 464138.274197263
				18193 464149.069987728
				18194 464159.865778192
				18195 464170.661568657
				18196 464181.457359121
				18197 464192.253149586
				18198 464203.04894005
				18199 464213.844730515
				18200 464224.64052098
				18201 464235.436311444
				18202 464246.232101909
				18203 464257.027892373
				18204 464267.823682838
				18205 464278.619473303
				18206 464289.415263767
				18207 464300.211054232
				18208 464311.006844696
				18209 464321.802635161
				18210 464332.598425625
				18211 464343.39421609
				18212 464354.190006555
				18213 464364.985797019
				18214 464375.781587484
				18215 464386.577377948
				18216 464397.373168413
				18217 464408.168958877
				18218 464418.964749342
				18219 464429.760539807
				18220 464440.556330271
				18221 464451.352120736
				18222 464462.1479112
				18223 464472.943701665
				18224 464483.73949213
				18225 464494.535282594
				18226 464505.331073059
				18227 464516.126863523
				18228 464526.922653988
				18229 464537.718444452
				18230 464548.514234917
				18231 464559.310025382
				18232 464570.105815846
				18233 464580.901606311
				18234 464591.697396775
				18235 464602.49318724
				18236 464613.288977704
				18237 464624.084768169
				18238 464634.880558634
				18239 464645.676349098
				18240 464656.472139563
				18241 464667.267930027
				18242 464678.063720492
				18243 464688.859510957
				18244 464699.655301421
				18245 464710.451091886
				18246 464721.24688235
				18247 464732.042672815
				18248 464742.838463279
				18249 464753.634253744
				18250 464764.430044209
				18251 464775.225834673
				18252 464786.021625138
				18253 464796.817415602
				18254 464807.613206067
				18255 464818.408996532
				18256 464829.204786996
				18257 464840.000577461
				18258 464850.796367925
				18259 464861.59215839
				18260 464872.387948854
				18261 464883.183739319
				18262 464893.979529784
				18263 464904.775320248
				18264 464915.571110713
				18265 464926.366901177
				18266 464937.162691642
				18267 464947.958482106
				18268 464958.754272571
				18269 464969.550063036
				18270 464980.3458535
				18271 464991.141643965
				18272 465001.937434429
				18273 465012.733224894
				18274 465023.529015358
				18275 465034.324805823
				18276 465045.120596288
				18277 465055.916386752
				18278 465066.712177217
				18279 465077.507967681
				18280 465088.303758146
				18281 465099.099548611
				18282 465109.895339075
				18283 465120.69112954
				18284 465131.486920004
				18285 465142.282710469
				18286 465153.078500933
				18287 465163.874291398
				18288 465174.670081863
				18289 465185.465872327
				18290 465196.261662792
				18291 465207.057453256
				18292 465217.853243721
				18293 465228.649034186
				18294 465239.44482465
				18295 465250.240615115
				18296 465261.036405579
				18297 465271.832196044
				18298 465282.627986508
				18299 465293.423776973
				18300 465304.219567438
				18301 465315.015357902
				18302 465325.811148367
				18303 465336.606938831
				18304 465347.402729296
				18305 465358.19851976
				18306 465368.994310225
				18307 465379.79010069
				18308 465390.585891154
				18309 465401.381681619
				18310 465412.177472083
				18311 465422.973262548
				18312 465433.769053012
				18313 465444.564843477
				18314 465455.360633942
				18315 465466.156424406
				18316 465476.952214871
				18317 465487.748005335
				18318 465498.5437958
				18319 465509.339586265
				18320 465520.135376729
				18321 465530.931167194
				18322 465541.726957658
				18323 465552.522748123
				18324 465563.318538587
				18325 465574.114329052
				18326 465584.910119517
				18327 465595.705909981
				18328 465606.501700446
				18329 465617.29749091
				18330 465628.093281375
				18331 465638.88907184
				18332 465649.684862304
				18333 465660.480652769
				18334 465671.276443233
				18335 465682.072233698
				18336 465692.868024162
				18337 465703.663814627
				18338 465714.459605092
				18339 465725.255395556
				18340 465736.051186021
				18341 465746.846976485
				18342 465757.64276695
				18343 465768.438557414
				18344 465779.234347879
				18345 465790.030138344
				18346 465800.825928808
				18347 465811.621719273
				18348 465822.417509737
				18349 465833.213300202
				18350 465844.009090667
				18351 465854.804881131
				18352 465865.600671596
				18353 465876.39646206
				18354 465887.192252525
				18355 465897.988042989
				18356 465908.783833454
				18357 465919.579623919
				18358 465930.375414383
				18359 465941.171204848
				18360 465951.966995312
				18361 465962.762785777
				18362 465973.558576241
				18363 465984.354366706
				18364 465995.150157171
				18365 466005.945947635
				18366 466016.7417381
				18367 466027.537528564
				18368 466038.333319029
				18369 466049.129109494
				18370 466059.924899958
				18371 466070.720690423
				18372 466081.516480887
				18373 466092.312271352
				18374 466103.108061816
				18375 466113.903852281
				18376 466124.699642746
				18377 466135.49543321
				18378 466146.291223675
				18379 466157.087014139
				18380 466167.882804604
				18381 466178.678595069
				18382 466189.474385533
				18383 466200.270175998
				18384 466211.065966462
				18385 466221.861756927
				18386 466232.657547391
				18387 466243.453337856
				18388 466254.249128321
				18389 466265.044918785
				18390 466275.84070925
				18391 466286.636499714
				18392 466297.432290179
				18393 466308.228080643
				18394 466319.023871108
				18395 466329.819661573
				18396 466340.615452037
				18397 466351.411242502
				18398 466362.207032966
				18399 466373.002823431
				18400 466383.798613896
				18401 466394.59440436
				18402 466405.390194825
				18403 466416.185985289
				18404 466426.981775754
				18405 466437.777566218
				18406 466448.573356683
				18407 466459.369147148
				18408 466470.164937612
				18409 466480.960728077
				18410 466491.756518541
				18411 466502.552309006
				18412 466513.34809947
				18413 466524.143889935
				18414 466534.9396804
				18415 466545.735470864
				18416 466556.531261329
				18417 466567.327051793
				18418 466578.122842258
				18419 466588.918632723
				18420 466599.714423187
				18421 466610.510213652
				18422 466621.306004116
				18423 466632.101794581
				18424 466642.897585045
				18425 466653.69337551
				18426 466664.489165975
				18427 466675.284956439
				18428 466686.080746904
				18429 466696.876537368
				18430 466707.672327833
				18431 466718.468118298
				18432 466729.263908762
				18433 466740.059699227
				18434 466750.855489691
				18435 466761.651280156
				18436 466772.44707062
				18437 466783.242861085
				18438 466794.03865155
				18439 466804.834442014
				18440 466815.630232479
				18441 466826.426022943
				18442 466837.221813408
				18443 466848.017603872
				18444 466858.813394337
				18445 466869.609184802
				18446 466880.404975266
				18447 466891.200765731
				18448 466901.996556195
				18449 466912.79234666
				18450 466923.588137125
				18451 466934.383927589
				18452 466945.179718054
				18453 466955.975508518
				18454 466966.771298983
				18455 466977.567089447
				18456 466988.362879912
				18457 466999.158670377
				18458 467009.954460841
				18459 467020.750251306
				18460 467031.54604177
				18461 467042.341832235
				18462 467053.137622699
				18463 467063.933413164
				18464 467074.729203629
				18465 467085.524994093
				18466 467096.320784558
				18467 467107.116575022
				18468 467117.912365487
				18469 467128.708155952
				18470 467139.503946416
				18471 467150.299736881
				18472 467161.095527345
				18473 467171.89131781
				18474 467182.687108274
				18475 467193.482898739
				18476 467204.278689204
				18477 467215.074479668
				18478 467225.870270133
				18479 467236.666060597
				18480 467247.461851062
				18481 467258.257641527
				18482 467269.053431991
				18483 467279.849222456
				18484 467290.64501292
				18485 467301.440803385
				18486 467312.236593849
				18487 467323.032384314
				18488 467333.828174779
				18489 467344.623965243
				18490 467355.419755708
				18491 467366.215546172
				18492 467377.011336637
				18493 467387.807127101
				18494 467398.602917566
				18495 467409.398708031
				18496 467420.194498495
				18497 467430.99028896
				18498 467441.786079424
				18499 467452.581869889
				18500 467463.377660354
				18501 467474.173450818
				18502 467484.969241283
				18503 467495.765031747
				18504 467506.560822212
				18505 467517.356612676
				18506 467528.152403141
				18507 467538.948193606
				18508 467549.74398407
				18509 467560.539774535
				18510 467571.335564999
				18511 467582.131355464
				18512 467592.927145928
				18513 467603.722936393
				18514 467614.518726858
				18515 467625.314517322
				18516 467636.110307787
				18517 467646.906098251
				18518 467657.701888716
				18519 467668.497679181
				18520 467679.293469645
				18521 467690.08926011
				18522 467700.885050574
				18523 467711.680841039
				18524 467722.476631503
				18525 467733.272421968
				18526 467744.068212433
				18527 467754.864002897
				18528 467765.659793362
				18529 467776.455583826
				18530 467787.251374291
				18531 467798.047164756
				18532 467808.84295522
				18533 467819.638745685
				18534 467830.434536149
				18535 467841.230326614
				18536 467852.026117078
				18537 467862.821907543
				18538 467873.617698008
				18539 467884.413488472
				18540 467895.209278937
				18541 467906.005069401
				18542 467916.800859866
				18543 467927.59665033
				18544 467938.392440795
				18545 467949.18823126
				18546 467959.984021724
				18547 467970.779812189
				18548 467981.575602653
				18549 467992.371393118
				18550 468003.167183583
				18551 468013.962974047
				18552 468024.758764512
				18553 468035.554554976
				18554 468046.350345441
				18555 468057.146135905
				18556 468067.94192637
				18557 468078.737716835
				18558 468089.533507299
				18559 468100.329297764
				18560 468111.125088228
				18561 468121.920878693
				18562 468132.716669157
				18563 468143.512459622
				18564 468154.308250087
				18565 468165.104040551
				18566 468175.899831016
				18567 468186.69562148
				18568 468197.491411945
				18569 468208.28720241
				18570 468219.082992874
				18571 468229.878783339
				18572 468240.674573803
				18573 468251.470364268
				18574 468262.266154732
				18575 468273.061945197
				18576 468283.857735662
				18577 468294.653526126
				18578 468305.449316591
				18579 468316.245107055
				18580 468327.04089752
				18581 468337.836687985
				18582 468348.632478449
				18583 468359.428268914
				18584 468370.224059378
				18585 468381.019849843
				18586 468391.815640307
				18587 468402.611430772
				18588 468413.407221237
				18589 468424.203011701
				18590 468434.998802166
				18591 468445.79459263
				18592 468456.590383095
				18593 468467.386173559
				18594 468478.181964024
				18595 468488.977754489
				18596 468499.773544953
				18597 468510.569335418
				18598 468521.365125882
				18599 468532.160916347
				18600 468542.956706811
				18601 468553.752497276
				18602 468564.548287741
				18603 468575.344078205
				18604 468586.13986867
				18605 468596.935659134
				18606 468607.731449599
				18607 468618.527240064
				18608 468629.323030528
				18609 468640.118820993
				18610 468650.914611457
				18611 468661.710401922
				18612 468672.506192386
				18613 468683.301982851
				18614 468694.097773316
				18615 468704.89356378
				18616 468715.689354245
				18617 468726.485144709
				18618 468737.280935174
				18619 468748.076725639
				18620 468758.872516103
				18621 468769.668306568
				18622 468780.464097032
				18623 468791.259887497
				18624 468802.055677961
				18625 468812.851468426
				18626 468823.647258891
				18627 468834.443049355
				18628 468845.23883982
				18629 468856.034630284
				18630 468866.830420749
				18631 468877.626211213
				18632 468888.422001678
				18633 468899.217792143
				18634 468910.013582607
				18635 468920.809373072
				18636 468931.605163536
				18637 468942.400954001
				18638 468953.196744466
				18639 468963.99253493
				18640 468974.788325395
				18641 468985.584115859
				18642 468996.379906324
				18643 469007.175696788
				18644 469017.971487253
				18645 469028.767277718
				18646 469039.563068182
				18647 469050.358858647
				18648 469061.154649111
				18649 469071.950439576
				18650 469082.746230041
				18651 469093.542020505
				18652 469104.33781097
				18653 469115.133601434
				18654 469125.929391899
				18655 469136.725182363
				18656 469147.520972828
				18657 469158.316763293
				18658 469169.112553757
				18659 469179.908344222
				18660 469190.704134686
				18661 469201.499925151
				18662 469212.295715615
				18663 469223.09150608
				18664 469233.887296545
				18665 469244.683087009
				18666 469255.478877474
				18667 469266.274667938
				18668 469277.070458403
				18669 469287.866248867
				18670 469298.662039332
				18671 469309.457829797
				18672 469320.253620261
				18673 469331.049410726
				18674 469341.84520119
				18675 469352.640991655
				18676 469363.43678212
				18677 469374.232572584
				18678 469385.028363049
				18679 469395.824153513
				18680 469406.619943978
				18681 469417.415734442
				18682 469428.211524907
				18683 469439.007315372
				18684 469449.803105836
				18685 469460.598896301
				18686 469471.394686765
				18687 469482.19047723
				18688 469492.986267695
				18689 469503.782058159
				18690 469514.577848624
				18691 469525.373639088
				18692 469536.169429553
				18693 469546.965220017
				18694 469557.761010482
				18695 469568.556800947
				18696 469579.352591411
				18697 469590.148381876
				18698 469600.94417234
				18699 469611.739962805
				18700 469622.535753269
				18701 469633.331543734
				18702 469644.127334199
				18703 469654.923124663
				18704 469665.718915128
				18705 469676.514705592
				18706 469687.310496057
				18707 469698.106286522
				18708 469708.902076986
				18709 469719.697867451
				18710 469730.493657915
				18711 469741.28944838
				18712 469752.085238844
				18713 469762.881029309
				18714 469773.676819774
				18715 469784.472610238
				18716 469795.268400703
				18717 469806.064191167
				18718 469816.859981632
				18719 469827.655772096
				18720 469838.451562561
				18721 469849.247353026
				18722 469860.04314349
				18723 469870.838933955
				18724 469881.634724419
				18725 469892.430514884
				18726 469903.226305349
				18727 469914.022095813
				18728 469924.817886278
				18729 469935.613676742
				18730 469946.409467207
				18731 469957.205257671
				18732 469968.001048136
				18733 469978.796838601
				18734 469989.592629065
				18735 470000.38841953
				18736 470011.184209994
				18737 470021.980000459
				18738 470032.775790924
				18739 470043.571581388
				18740 470054.367371853
				18741 470065.163162317
				18742 470075.958952782
				18743 470086.754743246
				18744 470097.550533711
				18745 470108.346324176
				18746 470119.14211464
				18747 470129.937905105
				18748 470140.733695569
				18749 470151.529486034
				18750 470162.325276498
				18751 470173.121066963
				18752 470183.916857428
				18753 470194.712647892
				18754 470205.508438357
				18755 470216.304228821
				18756 470227.100019286
				18757 470237.895809751
				18758 470248.691600215
				18759 470259.48739068
				18760 470270.283181144
				18761 470281.078971609
				18762 470291.874762073
				18763 470302.670552538
				18764 470313.466343003
				18765 470324.262133467
				18766 470335.057923932
				18767 470345.853714396
				18768 470356.649504861
				18769 470367.445295325
				18770 470378.24108579
				18771 470389.036876255
				18772 470399.832666719
				18773 470410.628457184
				18774 470421.424247648
				18775 470432.220038113
				18776 470443.015828578
				18777 470453.811619042
				18778 470464.607409507
				18779 470475.403199971
				18780 470486.198990436
				18781 470496.9947809
				18782 470507.790571365
				18783 470518.58636183
				18784 470529.382152294
				18785 470540.177942759
				18786 470550.973733223
				18787 470561.769523688
				18788 470572.565314153
				18789 470583.361104617
				18790 470594.156895082
				18791 470604.952685546
				18792 470615.748476011
				18793 470626.544266475
				18794 470637.34005694
				18795 470648.135847405
				18796 470658.931637869
				18797 470669.727428334
				18798 470680.523218798
				18799 470691.319009263
				18800 470702.114799727
				18801 470712.910590192
				18802 470723.706380657
				18803 470734.502171121
				18804 470745.297961586
				18805 470756.09375205
				18806 470766.889542515
				18807 470777.68533298
				18808 470788.481123444
				18809 470799.276913909
				18810 470810.072704373
				18811 470820.868494838
				18812 470831.664285302
				18813 470842.460075767
				18814 470853.255866232
				18815 470864.051656696
				18816 470874.847447161
				18817 470885.643237625
				18818 470896.43902809
				18819 470907.234818554
				18820 470918.030609019
				18821 470928.826399484
				18822 470939.622189948
				18823 470950.417980413
				18824 470961.213770877
				18825 470972.009561342
				18826 470982.805351807
				18827 470993.601142271
				18828 471004.396932736
				18829 471015.1927232
				18830 471025.988513665
				18831 471036.784304129
				18832 471047.580094594
				18833 471058.375885059
				18834 471069.171675523
				18835 471079.967465988
				18836 471090.763256452
				18837 471101.559046917
				18838 471112.354837382
				18839 471123.150627846
				18840 471133.946418311
				18841 471144.742208775
				18842 471155.53799924
				18843 471166.333789704
				18844 471177.129580169
				18845 471187.925370634
				18846 471198.721161098
				18847 471209.516951563
				18848 471220.312742027
				18849 471231.108532492
				18850 471241.904322956
				18851 471252.700113421
				18852 471263.495903886
				18853 471274.29169435
				18854 471285.087484815
				18855 471295.883275279
				18856 471306.679065744
				18857 471317.474856209
				18858 471328.270646673
				18859 471339.066437138
				18860 471349.862227602
				18861 471360.658018067
				18862 471371.453808531
				18863 471382.249598996
				18864 471393.045389461
				18865 471403.841179925
				18866 471414.63697039
				18867 471425.432760854
				18868 471436.228551319
				18869 471447.024341783
				18870 471457.820132248
				18871 471468.615922713
				18872 471479.411713177
				18873 471490.207503642
				18874 471501.003294106
				18875 471511.799084571
				18876 471522.594875036
				18877 471533.3906655
				18878 471544.186455965
				18879 471554.982246429
				18880 471565.778036894
				18881 471576.573827358
				18882 471587.369617823
				18883 471598.165408288
				18884 471608.961198752
				18885 471619.756989217
				18886 471630.552779681
				18887 471641.348570146
				18888 471652.144360611
				18889 471662.940151075
				18890 471673.73594154
				18891 471684.531732004
				18892 471695.327522469
				18893 471706.123312933
				18894 471716.919103398
				18895 471727.714893863
				18896 471738.510684327
				18897 471749.306474792
				18898 471760.102265256
				18899 471770.898055721
				18900 471781.693846185
				18901 471792.48963665
				18902 471803.285427115
				18903 471814.081217579
				18904 471824.877008044
				18905 471835.672798508
				18906 471846.468588973
				18907 471857.264379438
				18908 471868.060169902
				18909 471878.855960367
				18910 471889.651750831
				18911 471900.447541296
				18912 471911.24333176
				18913 471922.039122225
				18914 471932.83491269
				18915 471943.630703154
				18916 471954.426493619
				18917 471965.222284083
				18918 471976.018074548
				18919 471986.813865012
				18920 471997.609655477
				18921 472008.405445942
				18922 472019.201236406
				18923 472029.997026871
				18924 472040.792817335
				18925 472051.5886078
				18926 472062.384398265
				18927 472073.180188729
				18928 472083.975979194
				18929 472094.771769658
				18930 472105.567560123
				18931 472116.363350587
				18932 472127.159141052
				18933 472137.954931517
				18934 472148.750721981
				18935 472159.546512446
				18936 472170.34230291
				18937 472181.138093375
				18938 472191.93388384
				18939 472202.729674304
				18940 472213.525464769
				18941 472224.321255233
				18942 472235.117045698
				18943 472245.912836162
				18944 472256.708626627
				18945 472267.504417092
				18946 472278.300207556
				18947 472289.095998021
				18948 472299.891788485
				18949 472310.68757895
				18950 472321.483369414
				18951 472332.279159879
				18952 472343.074950344
				18953 472353.870740808
				18954 472364.666531273
				18955 472375.462321737
				18956 472386.258112202
				18957 472397.053902666
				18958 472407.849693131
				18959 472418.645483596
				18960 472429.44127406
				18961 472440.237064525
				18962 472451.032854989
				18963 472461.828645454
				18964 472472.624435919
				18965 472483.420226383
				18966 472494.216016848
				18967 472505.011807312
				18968 472515.807597777
				18969 472526.603388241
				18970 472537.399178706
				18971 472548.194969171
				18972 472558.990759635
				18973 472569.7865501
				18974 472580.582340564
				18975 472591.378131029
				18976 472602.173921494
				18977 472612.969711958
				18978 472623.765502423
				18979 472634.561292887
				18980 472645.357083352
				18981 472656.152873816
				18982 472666.948664281
				18983 472677.744454746
				18984 472688.54024521
				18985 472699.336035675
				18986 472710.131826139
				18987 472720.927616604
				18988 472731.723407068
				18989 472742.519197533
				18990 472753.314987998
				18991 472764.110778462
				18992 472774.906568927
				18993 472785.702359391
				18994 472796.498149856
				18995 472807.29394032
				18996 472818.089730785
				18997 472828.88552125
				18998 472839.681311714
				18999 472850.477102179
				19000 472861.272892643
				19001 472872.068683108
				19002 472882.864473573
				19003 472893.660264037
				19004 472904.456054502
				19005 472915.251844966
				19006 472926.047635431
				19007 472936.843425895
				19008 472947.63921636
				19009 472958.435006825
				19010 472969.230797289
				19011 472980.026587754
				19012 472990.822378218
				19013 473001.618168683
				19014 473012.413959148
				19015 473023.209749612
				19016 473034.005540077
				19017 473044.801330541
				19018 473055.597121006
				19019 473066.39291147
				19020 473077.188701935
				19021 473087.9844924
				19022 473098.780282864
				19023 473109.576073329
				19024 473120.371863793
				19025 473131.167654258
				19026 473141.963444722
				19027 473152.759235187
				19028 473163.555025652
				19029 473174.350816116
				19030 473185.146606581
				19031 473195.942397045
				19032 473206.73818751
				19033 473217.533977975
				19034 473228.329768439
				19035 473239.125558904
				19036 473249.921349368
				19037 473260.717139833
				19038 473271.512930297
				19039 473282.308720762
				19040 473293.104511227
				19041 473303.900301691
				19042 473314.696092156
				19043 473325.49188262
				19044 473336.287673085
				19045 473347.083463549
				19046 473357.879254014
				19047 473368.675044479
				19048 473379.470834943
				19049 473390.266625408
				19050 473401.062415872
				19051 473411.858206337
				19052 473422.653996802
				19053 473433.449787266
				19054 473444.245577731
				19055 473455.041368195
				19056 473465.83715866
				19057 473476.632949124
				19058 473487.428739589
				19059 473498.224530054
				19060 473509.020320518
				19061 473519.816110983
				19062 473530.611901447
				19063 473541.407691912
				19064 473552.203482377
				19065 473562.999272841
				19066 473573.795063306
				19067 473584.59085377
				19068 473595.386644235
				19069 473606.182434699
				19070 473616.978225164
				19071 473627.774015629
				19072 473638.569806093
				19073 473649.365596558
				19074 473660.161387022
				19075 473670.957177487
				19076 473681.752967951
				19077 473692.548758416
				19078 473703.344548881
				19079 473714.140339345
				19080 473724.93612981
				19081 473735.731920274
				19082 473746.527710739
				19083 473757.323501204
				19084 473768.119291668
				19085 473778.915082133
				19086 473789.710872597
				19087 473800.506663062
				19088 473811.302453526
				19089 473822.098243991
				19090 473832.894034456
				19091 473843.68982492
				19092 473854.485615385
				19093 473865.281405849
				19094 473876.077196314
				19095 473886.872986779
				19096 473897.668777243
				19097 473908.464567708
				19098 473919.260358172
				19099 473930.056148637
				19100 473940.851939101
				19101 473951.647729566
				19102 473962.443520031
				19103 473973.239310495
				19104 473984.03510096
				19105 473994.830891424
				19106 474005.626681889
				19107 474016.422472353
				19108 474027.218262818
				19109 474038.014053283
				19110 474048.809843747
				19111 474059.605634212
				19112 474070.401424676
				19113 474081.197215141
				19114 474091.993005606
				19115 474102.78879607
				19116 474113.584586535
				19117 474124.380376999
				19118 474135.176167464
				19119 474145.971957928
				19120 474156.767748393
				19121 474167.563538858
				19122 474178.359329322
				19123 474189.155119787
				19124 474199.950910251
				19125 474210.746700716
				19126 474221.54249118
				19127 474232.338281645
				19128 474243.13407211
				19129 474253.929862574
				19130 474264.725653039
				19131 474275.521443503
				19132 474286.317233968
				19133 474297.113024433
				19134 474307.908814897
				19135 474318.704605362
				19136 474329.500395826
				19137 474340.296186291
				19138 474351.091976755
				19139 474361.88776722
				19140 474372.683557685
				19141 474383.479348149
				19142 474394.275138614
				19143 474405.070929078
				19144 474415.866719543
				19145 474426.662510008
				19146 474437.458300472
				19147 474448.254090937
				19148 474459.049881401
				19149 474469.845671866
				19150 474480.64146233
				19151 474491.437252795
				19152 474502.23304326
				19153 474513.028833724
				19154 474523.824624189
				19155 474534.620414653
				19156 474545.416205118
				19157 474556.211995582
				19158 474567.007786047
				19159 474577.803576512
				19160 474588.599366976
				19161 474599.395157441
				19162 474610.190947905
				19163 474620.98673837
				19164 474631.782528835
				19165 474642.578319299
				19166 474653.374109764
				19167 474664.169900228
				19168 474674.965690693
				19169 474685.761481157
				19170 474696.557271622
				19171 474707.353062087
				19172 474718.148852551
				19173 474728.944643016
				19174 474739.74043348
				19175 474750.536223945
				19176 474761.332014409
				19177 474772.127804874
				19178 474782.923595339
				19179 474793.719385803
				19180 474804.515176268
				19181 474815.310966732
				19182 474826.106757197
				19183 474836.902547662
				19184 474847.698338126
				19185 474858.494128591
				19186 474869.289919055
				19187 474880.08570952
				19188 474890.881499984
				19189 474901.677290449
				19190 474912.473080914
				19191 474923.268871378
				19192 474934.064661843
				19193 474944.860452307
				19194 474955.656242772
				19195 474966.452033237
				19196 474977.247823701
				19197 474988.043614166
				19198 474998.83940463
				19199 475009.635195095
				19200 475020.430985559
				19201 475031.226776024
				19202 475042.022566489
				19203 475052.818356953
				19204 475063.614147418
				19205 475074.409937882
				19206 475085.205728347
				19207 475096.001518811
				19208 475106.797309276
				19209 475117.593099741
				19210 475128.388890205
				19211 475139.18468067
				19212 475149.980471134
				19213 475160.776261599
				19214 475171.572052064
				19215 475182.367842528
				19216 475193.163632993
				19217 475203.959423457
				19218 475214.755213922
				19219 475225.551004386
				19220 475236.346794851
				19221 475247.142585316
				19222 475257.93837578
				19223 475268.734166245
				19224 475279.529956709
				19225 475290.325747174
				19226 475301.121537638
				19227 475311.917328103
				19228 475322.713118568
				19229 475333.508909032
				19230 475344.304699497
				19231 475355.100489961
				19232 475365.896280426
				19233 475376.692070891
				19234 475387.487861355
				19235 475398.28365182
				19236 475409.079442284
				19237 475419.875232749
				19238 475430.671023213
				19239 475441.466813678
				19240 475452.262604143
				19241 475463.058394607
				19242 475473.854185072
				19243 475484.649975536
				19244 475495.445766001
				19245 475506.241556466
				19246 475517.03734693
				19247 475527.833137395
				19248 475538.628927859
				19249 475549.424718324
				19250 475560.220508788
				19251 475571.016299253
				19252 475581.812089718
				19253 475592.607880182
				19254 475603.403670647
				19255 475614.199461111
				19256 475624.995251576
				19257 475635.79104204
				19258 475646.586832505
				19259 475657.38262297
				19260 475668.178413434
				19261 475678.974203899
				19262 475689.769994363
				19263 475700.565784828
				19264 475711.361575293
				19265 475722.157365757
				19266 475732.953156222
				19267 475743.748946686
				19268 475754.544737151
				19269 475765.340527615
				19270 475776.13631808
				19271 475786.932108545
				19272 475797.727899009
				19273 475808.523689474
				19274 475819.319479938
				19275 475830.115270403
				19276 475840.911060867
				19277 475851.706851332
				19278 475862.502641797
				19279 475873.298432261
				19280 475884.094222726
				19281 475894.89001319
				19282 475905.685803655
				19283 475916.48159412
				19284 475927.277384584
				19285 475938.073175049
				19286 475948.868965513
				19287 475959.664755978
				19288 475970.460546442
				19289 475981.256336907
				19290 475992.052127372
				19291 476002.847917836
				19292 476013.643708301
				19293 476024.439498765
				19294 476035.23528923
				19295 476046.031079695
				19296 476056.826870159
				19297 476067.622660624
				19298 476078.418451088
				19299 476089.214241553
				19300 476100.010032017
				19301 476110.805822482
				19302 476121.601612947
				19303 476132.397403411
				19304 476143.193193876
				19305 476153.98898434
				19306 476164.784774805
				19307 476175.580565269
				19308 476186.376355734
				19309 476197.172146199
				19310 476207.967936663
				19311 476218.763727128
				19312 476229.559517592
				19313 476240.355308057
				19314 476251.151098521
				19315 476261.946888986
				19316 476272.742679451
				19317 476283.538469915
				19318 476294.33426038
				19319 476305.130050844
				19320 476315.925841309
				19321 476326.721631774
				19322 476337.517422238
				19323 476348.313212703
				19324 476359.109003167
				19325 476369.904793632
				19326 476380.700584096
				19327 476391.496374561
				19328 476402.292165026
				19329 476413.08795549
				19330 476423.883745955
				19331 476434.679536419
				19332 476445.475326884
				19333 476456.271117349
				19334 476467.066907813
				19335 476477.862698278
				19336 476488.658488742
				19337 476499.454279207
				19338 476510.250069671
				19339 476521.045860136
				19340 476531.841650601
				19341 476542.637441065
				19342 476553.43323153
				19343 476564.229021994
				19344 476575.024812459
				19345 476585.820602923
				19346 476596.616393388
				19347 476607.412183853
				19348 476618.207974317
				19349 476629.003764782
				19350 476639.799555246
				19351 476650.595345711
				19352 476661.391136175
				19353 476672.18692664
				19354 476682.982717105
				19355 476693.778507569
				19356 476704.574298034
				19357 476715.370088498
				19358 476726.165878963
				19359 476736.961669428
				19360 476747.757459892
				19361 476758.553250357
				19362 476769.349040821
				19363 476780.144831286
				19364 476790.94062175
				19365 476801.736412215
				19366 476812.53220268
				19367 476823.327993144
				19368 476834.123783609
				19369 476844.919574073
				19370 476855.715364538
				19371 476866.511155003
				19372 476877.306945467
				19373 476888.102735932
				19374 476898.898526396
				19375 476909.694316861
				19376 476920.490107325
				19377 476931.28589779
				19378 476942.081688255
				19379 476952.877478719
				19380 476963.673269184
				19381 476974.469059648
				19382 476985.264850113
				19383 476996.060640577
				19384 477006.856431042
				19385 477017.652221507
				19386 477028.448011971
				19387 477039.243802436
				19388 477050.0395929
				19389 477060.835383365
				19390 477071.63117383
				19391 477082.426964294
				19392 477093.222754759
				19393 477104.018545223
				19394 477114.814335688
				19395 477125.610126152
				19396 477136.405916617
				19397 477147.201707082
				19398 477157.997497546
				19399 477168.793288011
				19400 477179.589078475
				19401 477190.38486894
				19402 477201.180659404
				19403 477211.976449869
				19404 477222.772240334
				19405 477233.568030798
				19406 477244.363821263
				19407 477255.159611727
				19408 477265.955402192
				19409 477276.751192657
				19410 477287.546983121
				19411 477298.342773586
				19412 477309.13856405
				19413 477319.934354515
				19414 477330.730144979
				19415 477341.525935444
				19416 477352.321725909
				19417 477363.117516373
				19418 477373.913306838
				19419 477384.709097302
				19420 477395.504887767
				19421 477406.300678232
				19422 477417.096468696
				19423 477427.892259161
				19424 477438.688049625
				19425 477449.48384009
				19426 477460.279630554
				19427 477471.075421019
				19428 477481.871211484
				19429 477492.667001948
				19430 477503.462792413
				19431 477514.258582877
				19432 477525.054373342
				19433 477535.850163806
				19434 477546.645954271
				19435 477557.441744736
				19436 477568.2375352
				19437 477579.033325665
				19438 477589.829116129
				19439 477600.624906594
				19440 477611.420697059
				19441 477622.216487523
				19442 477633.012277988
				19443 477643.808068452
				19444 477654.603858917
				19445 477665.399649381
				19446 477676.195439846
				19447 477686.991230311
				19448 477697.787020775
				19449 477708.58281124
				19450 477719.378601704
				19451 477730.174392169
				19452 477740.970182633
				19453 477751.765973098
				19454 477762.561763563
				19455 477773.357554027
				19456 477784.153344492
				19457 477794.949134956
				19458 477805.744925421
				19459 477816.540715886
				19460 477827.33650635
				19461 477838.132296815
				19462 477848.928087279
				19463 477859.723877744
				19464 477870.519668208
				19465 477881.315458673
				19466 477892.111249138
				19467 477902.907039602
				19468 477913.702830067
				19469 477924.498620531
				19470 477935.294410996
				19471 477946.090201461
				19472 477956.885991925
				19473 477967.68178239
				19474 477978.477572854
				19475 477989.273363319
				19476 478000.069153783
				19477 478010.864944248
				19478 478021.660734713
				19479 478032.456525177
				19480 478043.252315642
				19481 478054.048106106
				19482 478064.843896571
				19483 478075.639687035
				19484 478086.4354775
				19485 478097.231267965
				19486 478108.027058429
				19487 478118.822848894
				19488 478129.618639358
				19489 478140.414429823
				19490 478151.210220288
				19491 478162.006010752
				19492 478172.801801217
				19493 478183.597591681
				19494 478194.393382146
				19495 478205.18917261
				19496 478215.984963075
				19497 478226.78075354
				19498 478237.576544004
				19499 478248.372334469
				19500 478259.168124933
				19501 478269.963915398
				19502 478280.759705862
				19503 478291.555496327
				19504 478302.351286792
				19505 478313.147077256
				19506 478323.942867721
				19507 478334.738658185
				19508 478345.53444865
				19509 478356.330239115
				19510 478367.126029579
				19511 478377.921820044
				19512 478388.717610508
				19513 478399.513400973
				19514 478410.309191437
				19515 478421.104981902
				19516 478431.900772367
				19517 478442.696562831
				19518 478453.492353296
				19519 478464.28814376
				19520 478475.083934225
				19521 478485.87972469
				19522 478496.675515154
				19523 478507.471305619
				19524 478518.267096083
				19525 478529.062886548
				19526 478539.858677012
				19527 478550.654467477
				19528 478561.450257942
				19529 478572.246048406
				19530 478583.041838871
				19531 478593.837629335
				19532 478604.6334198
				19533 478615.429210264
				19534 478626.225000729
				19535 478637.020791194
				19536 478647.816581658
				19537 478658.612372123
				19538 478669.408162587
				19539 478680.203953052
				19540 478690.999743517
				19541 478701.795533981
				19542 478712.591324446
				19543 478723.38711491
				19544 478734.182905375
				19545 478744.978695839
				19546 478755.774486304
				19547 478766.570276769
				19548 478777.366067233
				19549 478788.161857698
				19550 478798.957648162
				19551 478809.753438627
				19552 478820.549229091
				19553 478831.345019556
				19554 478842.140810021
				19555 478852.936600485
				19556 478863.73239095
				19557 478874.528181414
				19558 478885.323971879
				19559 478896.119762344
				19560 478906.915552808
				19561 478917.711343273
				19562 478928.507133737
				19563 478939.302924202
				19564 478950.098714666
				19565 478960.894505131
				19566 478971.690295596
				19567 478982.48608606
				19568 478993.281876525
				19569 479004.077666989
				19570 479014.873457454
				19571 479025.669247919
				19572 479036.465038383
				19573 479047.260828848
				19574 479058.056619312
				19575 479068.852409777
				19576 479079.648200241
				19577 479090.443990706
				19578 479101.239781171
				19579 479112.035571635
				19580 479122.8313621
				19581 479133.627152564
				19582 479144.422943029
				19583 479155.218733493
				19584 479166.014523958
				19585 479176.810314423
				19586 479187.606104887
				19587 479198.401895352
				19588 479209.197685816
				19589 479219.993476281
				19590 479230.789266746
				19591 479241.58505721
				19592 479252.380847675
				19593 479263.176638139
				19594 479273.972428604
				19595 479284.768219068
				19596 479295.564009533
				19597 479306.359799998
				19598 479317.155590462
				19599 479327.951380927
				19600 479338.747171391
				19601 479349.542961856
				19602 479360.338752321
				19603 479371.134542785
				19604 479381.93033325
				19605 479392.726123714
				19606 479403.521914179
				19607 479414.317704643
				19608 479425.113495108
				19609 479435.909285573
				19610 479446.705076037
				19611 479457.500866502
				19612 479468.296656966
				19613 479479.092447431
				19614 479489.888237895
				19615 479500.68402836
				19616 479511.479818825
				19617 479522.275609289
				19618 479533.071399754
				19619 479543.867190218
				19620 479554.662980683
				19621 479565.458771147
				19622 479576.254561612
				19623 479587.050352077
				19624 479597.846142541
				19625 479608.641933006
				19626 479619.43772347
				19627 479630.233513935
				19628 479641.0293044
				19629 479651.825094864
				19630 479662.620885329
				19631 479673.416675793
				19632 479684.212466258
				19633 479695.008256722
				19634 479705.804047187
				19635 479716.599837652
				19636 479727.395628116
				19637 479738.191418581
				19638 479748.987209045
				19639 479759.78299951
				19640 479770.578789975
				19641 479781.374580439
				19642 479792.170370904
				19643 479802.966161368
				19644 479813.761951833
				19645 479824.557742297
				19646 479835.353532762
				19647 479846.149323227
				19648 479856.945113691
				19649 479867.740904156
				19650 479878.53669462
				19651 479889.332485085
				19652 479900.12827555
				19653 479910.924066014
				19654 479921.719856479
				19655 479932.515646943
				19656 479943.311437408
				19657 479954.107227872
				19658 479964.903018337
				19659 479975.698808802
				19660 479986.494599266
				19661 479997.290389731
				19662 480008.086180195
				19663 480018.88197066
				19664 480029.677761124
				19665 480040.473551589
				19666 480051.269342054
				19667 480062.065132518
				19668 480072.860922983
				19669 480083.656713447
				19670 480094.452503912
				19671 480105.248294376
				19672 480116.044084841
				19673 480126.839875306
				19674 480137.63566577
				19675 480148.431456235
				19676 480159.227246699
				19677 480170.023037164
				19678 480180.818827629
				19679 480191.614618093
				19680 480202.410408558
				19681 480213.206199022
				19682 480224.001989487
				19683 480234.797779951
				19684 480245.593570416
				19685 480256.389360881
				19686 480267.185151345
				19687 480277.98094181
				19688 480288.776732274
				19689 480299.572522739
				19690 480310.368313204
				19691 480321.164103668
				19692 480331.959894133
				19693 480342.755684597
				19694 480353.551475062
				19695 480364.347265526
				19696 480375.143055991
				19697 480385.938846456
				19698 480396.73463692
				19699 480407.530427385
				19700 480418.326217849
				19701 480429.122008314
				19702 480439.917798778
				19703 480450.713589243
				19704 480461.509379708
				19705 480472.305170172
				19706 480483.100960637
				19707 480493.896751101
				19708 480504.692541566
				19709 480515.48833203
				19710 480526.284122495
				19711 480537.07991296
				19712 480547.875703424
				19713 480558.671493889
				19714 480569.467284353
				19715 480580.263074818
				19716 480591.058865283
				19717 480601.854655747
				19718 480612.650446212
				19719 480623.446236676
				19720 480634.242027141
				19721 480645.037817605
				19722 480655.83360807
				19723 480666.629398535
				19724 480677.425188999
				19725 480688.220979464
				19726 480699.016769928
				19727 480709.812560393
				19728 480720.608350858
				19729 480731.404141322
				19730 480742.199931787
				19731 480752.995722251
				19732 480763.791512716
				19733 480774.58730318
				19734 480785.383093645
				19735 480796.17888411
				19736 480806.974674574
				19737 480817.770465039
				19738 480828.566255503
				19739 480839.362045968
				19740 480850.157836432
				19741 480860.953626897
				19742 480871.749417362
				19743 480882.545207826
				19744 480893.340998291
				19745 480904.136788755
				19746 480914.93257922
				19747 480925.728369685
				19748 480936.524160149
				19749 480947.319950614
				19750 480958.115741078
				19751 480968.911531543
				19752 480979.707322007
				19753 480990.503112472
				19754 481001.298902937
				19755 481012.094693401
				19756 481022.890483866
				19757 481033.68627433
				19758 481044.482064795
				19759 481055.277855259
				19760 481066.073645724
				19761 481076.869436189
				19762 481087.665226653
				19763 481098.461017118
				19764 481109.256807582
				19765 481120.052598047
				19766 481130.848388512
				19767 481141.644178976
				19768 481152.439969441
				19769 481163.235759905
				19770 481174.03155037
				19771 481184.827340834
				19772 481195.623131299
				19773 481206.418921764
				19774 481217.214712228
				19775 481228.010502693
				19776 481238.806293157
				19777 481249.602083622
				19778 481260.397874087
				19779 481271.193664551
				19780 481281.989455016
				19781 481292.78524548
				19782 481303.581035945
				19783 481314.376826409
				19784 481325.172616874
				19785 481335.968407339
				19786 481346.764197803
				19787 481357.559988268
				19788 481368.355778732
				19789 481379.151569197
				19790 481389.947359661
				19791 481400.743150126
				19792 481411.538940591
				19793 481422.334731055
				19794 481433.13052152
				19795 481443.926311984
				19796 481454.722102449
				19797 481465.517892914
				19798 481476.313683378
				19799 481487.109473843
				19800 481497.905264307
				19801 481508.701054772
				19802 481519.496845236
				19803 481530.292635701
				19804 481541.088426166
				19805 481551.88421663
				19806 481562.680007095
				19807 481573.475797559
				19808 481584.271588024
				19809 481595.067378488
				19810 481605.863168953
				19811 481616.658959418
				19812 481627.454749882
				19813 481638.250540347
				19814 481649.046330811
				19815 481659.842121276
				19816 481670.637911741
				19817 481681.433702205
				19818 481692.22949267
				19819 481703.025283134
				19820 481713.821073599
				19821 481724.616864063
				19822 481735.412654528
				19823 481746.208444993
				19824 481757.004235457
				19825 481767.800025922
				19826 481778.595816386
				19827 481789.391606851
				19828 481800.187397316
				19829 481810.98318778
				19830 481821.778978245
				19831 481832.574768709
				19832 481843.370559174
				19833 481854.166349638
				19834 481864.962140103
				19835 481875.757930568
				19836 481886.553721032
				19837 481897.349511497
				19838 481908.145301961
				19839 481918.941092426
				19840 481929.73688289
				19841 481940.532673355
				19842 481951.32846382
				19843 481962.124254284
				19844 481972.920044749
				19845 481983.715835213
				19846 481994.511625678
				19847 482005.307416143
				19848 482016.103206607
				19849 482026.898997072
				19850 482037.694787536
				19851 482048.490578001
				19852 482059.286368465
				19853 482070.08215893
				19854 482080.877949395
				19855 482091.673739859
				19856 482102.469530324
				19857 482113.265320788
				19858 482124.061111253
				19859 482134.856901717
				19860 482145.652692182
				19861 482156.448482647
				19862 482167.244273111
				19863 482178.040063576
				19864 482188.83585404
				19865 482199.631644505
				19866 482210.42743497
				19867 482221.223225434
				19868 482232.019015899
				19869 482242.814806363
				19870 482253.610596828
				19871 482264.406387292
				19872 482275.202177757
				19873 482285.997968222
				19874 482296.793758686
				19875 482307.589549151
				19876 482318.385339615
				19877 482329.18113008
				19878 482339.976920545
				19879 482350.772711009
				19880 482361.568501474
				19881 482372.364291938
				19882 482383.160082403
				19883 482393.955872867
				19884 482404.751663332
				19885 482415.547453797
				19886 482426.343244261
				19887 482437.139034726
				19888 482447.93482519
				19889 482458.730615655
				19890 482469.526406119
				19891 482480.322196584
				19892 482491.117987049
				19893 482501.913777513
				19894 482512.709567978
				19895 482523.505358442
				19896 482534.301148907
				19897 482545.096939372
				19898 482555.892729836
				19899 482566.688520301
				19900 482577.484310765
				19901 482588.28010123
				19902 482599.075891694
				19903 482609.871682159
				19904 482620.667472624
				19905 482631.463263088
				19906 482642.259053553
				19907 482653.054844017
				19908 482663.850634482
				19909 482674.646424946
				19910 482685.442215411
				19911 482696.238005876
				19912 482707.03379634
				19913 482717.829586805
				19914 482728.625377269
				19915 482739.421167734
				19916 482750.216958199
				19917 482761.012748663
				19918 482771.808539128
				19919 482782.604329592
				19920 482793.400120057
				19921 482804.195910521
				19922 482814.991700986
				19923 482825.787491451
				19924 482836.583281915
				19925 482847.37907238
				19926 482858.174862844
				19927 482868.970653309
				19928 482879.766443774
				19929 482890.562234238
				19930 482901.358024703
				19931 482912.153815167
				19932 482922.949605632
				19933 482933.745396096
				19934 482944.541186561
				19935 482955.336977026
				19936 482966.13276749
				19937 482976.928557955
				19938 482987.724348419
				19939 482998.520138884
				19940 483009.315929348
				19941 483020.111719813
				19942 483030.907510278
				19943 483041.703300742
				19944 483052.499091207
				19945 483063.294881671
				19946 483074.090672136
				19947 483084.8864626
				19948 483095.682253065
				19949 483106.47804353
				19950 483117.273833994
				19951 483128.069624459
				19952 483138.865414923
				19953 483149.661205388
				19954 483160.456995853
				19955 483171.252786317
				19956 483182.048576782
				19957 483192.844367246
				19958 483203.640157711
				19959 483214.435948175
				19960 483225.23173864
				19961 483236.027529105
				19962 483246.823319569
				19963 483257.619110034
				19964 483268.414900498
				19965 483279.210690963
				19966 483290.006481428
				19967 483300.802271892
				19968 483311.598062357
				19969 483322.393852821
				19970 483333.189643286
				19971 483343.98543375
				19972 483354.781224215
				19973 483365.57701468
				19974 483376.372805144
				19975 483387.168595609
				19976 483397.964386073
				19977 483408.760176538
				19978 483419.555967002
				19979 483430.351757467
				19980 483441.147547932
				19981 483451.943338396
				19982 483462.739128861
				19983 483473.534919325
				19984 483484.33070979
				19985 483495.126500254
				19986 483505.922290719
				19987 483516.718081184
				19988 483527.513871648
				19989 483538.309662113
				19990 483549.105452577
				19991 483559.901243042
				19992 483570.697033507
				19993 483581.492823971
				19994 483592.288614436
				19995 483603.0844049
				19996 483613.880195365
				19997 483624.675985829
				19998 483635.471776294
				19999 483646.267566759
			};
		\addplot [semithick, blue, dashed]
		table {%
				2160 1002898.9297589
				2161 1002900.21093887
				2162 1002901.49211884
				2163 1002902.7732988
				2164 1002904.05447877
				2165 1002905.33565873
				2166 1002906.6168387
				2167 1002907.89801866
				2168 1002909.17919863
				2169 1002910.46037859
				2170 1002911.74155856
				2171 1002913.02273852
				2172 1002914.30391849
				2173 1002915.58509846
				2174 1002916.86627842
				2175 1002918.14745839
				2176 1002919.42863835
				2177 1002920.70981832
				2178 1002921.99099828
				2179 1002923.27217825
				2180 1002924.55335821
				2181 1002925.83453818
				2182 1002927.11571814
				2183 1002928.39689811
				2184 1002929.67807808
				2185 1002930.95925804
				2186 1002932.24043801
				2187 1002933.52161797
				2188 1002934.80279794
				2189 1002936.0839779
				2190 1002937.36515787
				2191 1002938.64633783
				2192 1002939.9275178
				2193 1002941.20869776
				2194 1002942.48987773
				2195 1002943.7710577
				2196 1002945.05223766
				2197 1002946.33341763
				2198 1002947.61459759
				2199 1002948.89577756
				2200 1002950.17695752
				2201 1002951.45813749
				2202 1002952.73931745
				2203 1002954.02049742
				2204 1002955.30167738
				2205 1002956.58285735
				2206 1002957.86403732
				2207 1002959.14521728
				2208 1002960.42639725
				2209 1002961.70757721
				2210 1002962.98875718
				2211 1002964.26993714
				2212 1002965.55111711
				2213 1002966.83229707
				2214 1002968.11347704
				2215 1002969.394657
				2216 1002970.67583697
				2217 1002971.95701694
				2218 1002973.2381969
				2219 1002974.51937687
				2220 1002975.80055683
				2221 1002977.0817368
				2222 1002978.36291676
				2223 1002979.64409673
				2224 1002980.92527669
				2225 1002982.20645666
				2226 1002983.48763662
				2227 1002984.76881659
				2228 1002986.04999656
				2229 1002987.33117652
				2230 1002988.61235649
				2231 1002989.89353645
				2232 1002991.17471642
				2233 1002992.45589638
				2234 1002993.73707635
				2235 1002995.01825631
				2236 1002996.29943628
				2237 1002997.58061624
				2238 1002998.86179621
				2239 1003000.14297618
				2240 1003001.42415614
				2241 1003002.70533611
				2242 1003003.98651607
				2243 1003005.26769604
				2244 1003006.548876
				2245 1003007.83005597
				2246 1003009.11123593
				2247 1003010.3924159
				2248 1003011.67359586
				2249 1003012.95477583
				2250 1003014.2359558
				2251 1003015.51713576
				2252 1003016.79831573
				2253 1003018.07949569
				2254 1003019.36067566
				2255 1003020.64185562
				2256 1003021.92303559
				2257 1003023.20421555
				2258 1003024.48539552
				2259 1003025.76657548
				2260 1003027.04775545
				2261 1003028.32893542
				2262 1003029.61011538
				2263 1003030.89129535
				2264 1003032.17247531
				2265 1003033.45365528
				2266 1003034.73483524
				2267 1003036.01601521
				2268 1003037.29719517
				2269 1003038.57837514
				2270 1003039.8595551
				2271 1003041.14073507
				2272 1003042.42191504
				2273 1003043.703095
				2274 1003044.98427497
				2275 1003046.26545493
				2276 1003047.5466349
				2277 1003048.82781486
				2278 1003050.10899483
				2279 1003051.39017479
				2280 1003052.67135476
				2281 1003053.95253472
				2282 1003055.23371469
				2283 1003056.51489466
				2284 1003057.79607462
				2285 1003059.07725459
				2286 1003060.35843455
				2287 1003061.63961452
				2288 1003062.92079448
				2289 1003064.20197445
				2290 1003065.48315441
				2291 1003066.76433438
				2292 1003068.04551434
				2293 1003069.32669431
				2294 1003070.60787428
				2295 1003071.88905424
				2296 1003073.17023421
				2297 1003074.45141417
				2298 1003075.73259414
				2299 1003077.0137741
				2300 1003078.29495407
				2301 1003079.57613403
				2302 1003080.857314
				2303 1003082.13849396
				2304 1003083.41967393
				2305 1003084.7008539
				2306 1003085.98203386
				2307 1003087.26321383
				2308 1003088.54439379
				2309 1003089.82557376
				2310 1003091.10675372
				2311 1003092.38793369
				2312 1003093.66911365
				2313 1003094.95029362
				2314 1003096.23147358
				2315 1003097.51265355
				2316 1003098.79383352
				2317 1003100.07501348
				2318 1003101.35619345
				2319 1003102.63737341
				2320 1003103.91855338
				2321 1003105.19973334
				2322 1003106.48091331
				2323 1003107.76209327
				2324 1003109.04327324
				2325 1003110.3244532
				2326 1003111.60563317
				2327 1003112.88681314
				2328 1003114.1679931
				2329 1003115.44917307
				2330 1003116.73035303
				2331 1003118.011533
				2332 1003119.29271296
				2333 1003120.57389293
				2334 1003121.85507289
				2335 1003123.13625286
				2336 1003124.41743282
				2337 1003125.69861279
				2338 1003126.97979276
				2339 1003128.26097272
				2340 1003129.54215269
				2341 1003130.82333265
				2342 1003132.10451262
				2343 1003133.38569258
				2344 1003134.66687255
				2345 1003135.94805251
				2346 1003137.22923248
				2347 1003138.51041244
				2348 1003139.79159241
				2349 1003141.07277238
				2350 1003142.35395234
				2351 1003143.63513231
				2352 1003144.91631227
				2353 1003146.19749224
				2354 1003147.4786722
				2355 1003148.75985217
				2356 1003150.04103213
				2357 1003151.3222121
				2358 1003152.60339206
				2359 1003153.88457203
				2360 1003155.165752
				2361 1003156.44693196
				2362 1003157.72811193
				2363 1003159.00929189
				2364 1003160.29047186
				2365 1003161.57165182
				2366 1003162.85283179
				2367 1003164.13401175
				2368 1003165.41519172
				2369 1003166.69637168
				2370 1003167.97755165
				2371 1003169.25873162
				2372 1003170.53991158
				2373 1003171.82109155
				2374 1003173.10227151
				2375 1003174.38345148
				2376 1003175.66463144
				2377 1003176.94581141
				2378 1003178.22699137
				2379 1003179.50817134
				2380 1003180.7893513
				2381 1003182.07053127
				2382 1003183.35171124
				2383 1003184.6328912
				2384 1003185.91407117
				2385 1003187.19525113
				2386 1003188.4764311
				2387 1003189.75761106
				2388 1003191.03879103
				2389 1003192.31997099
				2390 1003193.60115096
				2391 1003194.88233092
				2392 1003196.16351089
				2393 1003197.44469085
				2394 1003198.72587082
				2395 1003200.00705079
				2396 1003201.28823075
				2397 1003202.56941072
				2398 1003203.85059068
				2399 1003205.13177065
				2400 1003206.41295061
				2401 1003207.69413058
				2402 1003208.97531054
				2403 1003210.25649051
				2404 1003211.53767047
				2405 1003212.81885044
				2406 1003214.10003041
				2407 1003215.38121037
				2408 1003216.66239034
				2409 1003217.9435703
				2410 1003219.22475027
				2411 1003220.50593023
				2412 1003221.7871102
				2413 1003223.06829016
				2414 1003224.34947013
				2415 1003225.63065009
				2416 1003226.91183006
				2417 1003228.19301003
				2418 1003229.47418999
				2419 1003230.75536996
				2420 1003232.03654992
				2421 1003233.31772989
				2422 1003234.59890985
				2423 1003235.88008982
				2424 1003237.16126978
				2425 1003238.44244975
				2426 1003239.72362972
				2427 1003241.00480968
				2428 1003242.28598965
				2429 1003243.56716961
				2430 1003244.84834958
				2431 1003246.12952954
				2432 1003247.41070951
				2433 1003248.69188947
				2434 1003249.97306944
				2435 1003251.2542494
				2436 1003252.53542937
				2437 1003253.81660934
				2438 1003255.0977893
				2439 1003256.37896927
				2440 1003257.66014923
				2441 1003258.9413292
				2442 1003260.22250916
				2443 1003261.50368913
				2444 1003262.78486909
				2445 1003264.06604906
				2446 1003265.34722902
				2447 1003266.62840899
				2448 1003267.90958895
				2449 1003269.19076892
				2450 1003270.47194889
				2451 1003271.75312885
				2452 1003273.03430882
				2453 1003274.31548878
				2454 1003275.59666875
				2455 1003276.87784871
				2456 1003278.15902868
				2457 1003279.44020864
				2458 1003280.72138861
				2459 1003282.00256857
				2460 1003283.28374854
				2461 1003284.56492851
				2462 1003285.84610847
				2463 1003287.12728844
				2464 1003288.4084684
				2465 1003289.68964837
				2466 1003290.97082833
				2467 1003292.2520083
				2468 1003293.53318826
				2469 1003294.81436823
				2470 1003296.09554819
				2471 1003297.37672816
				2472 1003298.65790813
				2473 1003299.93908809
				2474 1003301.22026806
				2475 1003302.50144802
				2476 1003303.78262799
				2477 1003305.06380795
				2478 1003306.34498792
				2479 1003307.62616788
				2480 1003308.90734785
				2481 1003310.18852781
				2482 1003311.46970778
				2483 1003312.75088775
				2484 1003314.03206771
				2485 1003315.31324768
				2486 1003316.59442764
				2487 1003317.87560761
				2488 1003319.15678757
				2489 1003320.43796754
				2490 1003321.7191475
				2491 1003323.00032747
				2492 1003324.28150743
				2493 1003325.5626874
				2494 1003326.84386737
				2495 1003328.12504733
				2496 1003329.4062273
				2497 1003330.68740726
				2498 1003331.96858723
				2499 1003333.24976719
				2500 1003334.53094716
				2501 1003335.81212712
				2502 1003337.09330709
				2503 1003338.37448705
				2504 1003339.65566702
				2505 1003340.93684699
				2506 1003342.21802695
				2507 1003343.49920692
				2508 1003344.78038688
				2509 1003346.06156685
				2510 1003347.34274681
				2511 1003348.62392678
				2512 1003349.90510674
				2513 1003351.18628671
				2514 1003352.46746667
				2515 1003353.74864664
				2516 1003355.02982661
				2517 1003356.31100657
				2518 1003357.59218654
				2519 1003358.8733665
				2520 1003360.15454647
				2521 1003361.43572643
				2522 1003362.7169064
				2523 1003363.99808636
				2524 1003365.27926633
				2525 1003366.56044629
				2526 1003367.84162626
				2527 1003369.12280623
				2528 1003370.40398619
				2529 1003371.68516616
				2530 1003372.96634612
				2531 1003374.24752609
				2532 1003375.52870605
				2533 1003376.80988602
				2534 1003378.09106598
				2535 1003379.37224595
				2536 1003380.65342591
				2537 1003381.93460588
				2538 1003383.21578585
				2539 1003384.49696581
				2540 1003385.77814578
				2541 1003387.05932574
				2542 1003388.34050571
				2543 1003389.62168567
				2544 1003390.90286564
				2545 1003392.1840456
				2546 1003393.46522557
				2547 1003394.74640553
				2548 1003396.0275855
				2549 1003397.30876547
				2550 1003398.58994543
				2551 1003399.8711254
				2552 1003401.15230536
				2553 1003402.43348533
				2554 1003403.71466529
				2555 1003404.99584526
				2556 1003406.27702522
				2557 1003407.55820519
				2558 1003408.83938515
				2559 1003410.12056512
				2560 1003411.40174509
				2561 1003412.68292505
				2562 1003413.96410502
				2563 1003415.24528498
				2564 1003416.52646495
				2565 1003417.80764491
				2566 1003419.08882488
				2567 1003420.37000484
				2568 1003421.65118481
				2569 1003422.93236477
				2570 1003424.21354474
				2571 1003425.49472471
				2572 1003426.77590467
				2573 1003428.05708464
				2574 1003429.3382646
				2575 1003430.61944457
				2576 1003431.90062453
				2577 1003433.1818045
				2578 1003434.46298446
				2579 1003435.74416443
				2580 1003437.02534439
				2581 1003438.30652436
				2582 1003439.58770433
				2583 1003440.86888429
				2584 1003442.15006426
				2585 1003443.43124422
				2586 1003444.71242419
				2587 1003445.99360415
				2588 1003447.27478412
				2589 1003448.55596408
				2590 1003449.83714405
				2591 1003451.11832401
				2592 1003452.39950398
				2593 1003453.68068395
				2594 1003454.96186391
				2595 1003456.24304388
				2596 1003457.52422384
				2597 1003458.80540381
				2598 1003460.08658377
				2599 1003461.36776374
				2600 1003462.6489437
				2601 1003463.93012367
				2602 1003465.21130363
				2603 1003466.4924836
				2604 1003467.77366357
				2605 1003469.05484353
				2606 1003470.3360235
				2607 1003471.61720346
				2608 1003472.89838343
				2609 1003474.17956339
				2610 1003475.46074336
				2611 1003476.74192332
				2612 1003478.02310329
				2613 1003479.30428325
				2614 1003480.58546322
				2615 1003481.86664319
				2616 1003483.14782315
				2617 1003484.42900312
				2618 1003485.71018308
				2619 1003486.99136305
				2620 1003488.27254301
				2621 1003489.55372298
				2622 1003490.83490294
				2623 1003492.11608291
				2624 1003493.39726287
				2625 1003494.67844284
				2626 1003495.95962281
				2627 1003497.24080277
				2628 1003498.52198274
				2629 1003499.8031627
				2630 1003501.08434267
				2631 1003502.36552263
				2632 1003503.6467026
				2633 1003504.92788256
				2634 1003506.20906253
				2635 1003507.49024249
				2636 1003508.77142246
				2637 1003510.05260243
				2638 1003511.33378239
				2639 1003512.61496236
				2640 1003513.89614232
				2641 1003515.17732229
				2642 1003516.45850225
				2643 1003517.73968222
				2644 1003519.02086218
				2645 1003520.30204215
				2646 1003521.58322211
				2647 1003522.86440208
				2648 1003524.14558205
				2649 1003525.42676201
				2650 1003526.70794198
				2651 1003527.98912194
				2652 1003529.27030191
				2653 1003530.55148187
				2654 1003531.83266184
				2655 1003533.1138418
				2656 1003534.39502177
				2657 1003535.67620173
				2658 1003536.9573817
				2659 1003538.23856167
				2660 1003539.51974163
				2661 1003540.8009216
				2662 1003542.08210156
				2663 1003543.36328153
				2664 1003544.64446149
				2665 1003545.92564146
				2666 1003547.20682142
				2667 1003548.48800139
				2668 1003549.76918135
				2669 1003551.05036132
				2670 1003552.33154129
				2671 1003553.61272125
				2672 1003554.89390122
				2673 1003556.17508118
				2674 1003557.45626115
				2675 1003558.73744111
				2676 1003560.01862108
				2677 1003561.29980104
				2678 1003562.58098101
				2679 1003563.86216097
				2680 1003565.14334094
				2681 1003566.42452091
				2682 1003567.70570087
				2683 1003568.98688084
				2684 1003570.2680608
				2685 1003571.54924077
				2686 1003572.83042073
				2687 1003574.1116007
				2688 1003575.39278066
				2689 1003576.67396063
				2690 1003577.95514059
				2691 1003579.23632056
				2692 1003580.51750053
				2693 1003581.79868049
				2694 1003583.07986046
				2695 1003584.36104042
				2696 1003585.64222039
				2697 1003586.92340035
				2698 1003588.20458032
				2699 1003589.48576028
				2700 1003590.76694025
				2701 1003592.04812021
				2702 1003593.32930018
				2703 1003594.61048015
				2704 1003595.89166011
				2705 1003597.17284008
				2706 1003598.45402004
				2707 1003599.73520001
				2708 1003601.01637997
				2709 1003602.29755994
				2710 1003603.5787399
				2711 1003604.85991987
				2712 1003606.14109983
				2713 1003607.4222798
				2714 1003608.70345977
				2715 1003609.98463973
				2716 1003611.2658197
				2717 1003612.54699966
				2718 1003613.82817963
				2719 1003615.10935959
				2720 1003616.39053956
				2721 1003617.67171952
				2722 1003618.95289949
				2723 1003620.23407945
				2724 1003621.51525942
				2725 1003622.79643939
				2726 1003624.07761935
				2727 1003625.35879932
				2728 1003626.63997928
				2729 1003627.92115925
				2730 1003629.20233921
				2731 1003630.48351918
				2732 1003631.76469914
				2733 1003633.04587911
				2734 1003634.32705907
				2735 1003635.60823904
				2736 1003636.88941901
				2737 1003638.17059897
				2738 1003639.45177894
				2739 1003640.7329589
				2740 1003642.01413887
				2741 1003643.29531883
				2742 1003644.5764988
				2743 1003645.85767876
				2744 1003647.13885873
				2745 1003648.42003869
				2746 1003649.70121866
				2747 1003650.98239863
				2748 1003652.26357859
				2749 1003653.54475856
				2750 1003654.82593852
				2751 1003656.10711849
				2752 1003657.38829845
				2753 1003658.66947842
				2754 1003659.95065838
				2755 1003661.23183835
				2756 1003662.51301831
				2757 1003663.79419828
				2758 1003665.07537825
				2759 1003666.35655821
				2760 1003667.63773818
				2761 1003668.91891814
				2762 1003670.20009811
				2763 1003671.48127807
				2764 1003672.76245804
				2765 1003674.043638
				2766 1003675.32481797
				2767 1003676.60599793
				2768 1003677.8871779
				2769 1003679.16835787
				2770 1003680.44953783
				2771 1003681.7307178
				2772 1003683.01189776
				2773 1003684.29307773
				2774 1003685.57425769
				2775 1003686.85543766
				2776 1003688.13661762
				2777 1003689.41779759
				2778 1003690.69897755
				2779 1003691.98015752
				2780 1003693.26133749
				2781 1003694.54251745
				2782 1003695.82369742
				2783 1003697.10487738
				2784 1003698.38605735
				2785 1003699.66723731
				2786 1003700.94841728
				2787 1003702.22959724
				2788 1003703.51077721
				2789 1003704.79195717
				2790 1003706.07313714
				2791 1003707.35431711
				2792 1003708.63549707
				2793 1003709.91667704
				2794 1003711.197857
				2795 1003712.47903697
				2796 1003713.76021693
				2797 1003715.0413969
				2798 1003716.32257686
				2799 1003717.60375683
				2800 1003718.88493679
				2801 1003720.16611676
				2802 1003721.44729673
				2803 1003722.72847669
				2804 1003724.00965666
				2805 1003725.29083662
				2806 1003726.57201659
				2807 1003727.85319655
				2808 1003729.13437652
				2809 1003730.41555648
				2810 1003731.69673645
				2811 1003732.97791641
				2812 1003734.25909638
				2813 1003735.54027635
				2814 1003736.82145631
				2815 1003738.10263628
				2816 1003739.38381624
				2817 1003740.66499621
				2818 1003741.94617617
				2819 1003743.22735614
				2820 1003744.5085361
				2821 1003745.78971607
				2822 1003747.07089603
				2823 1003748.352076
				2824 1003749.63325597
				2825 1003750.91443593
				2826 1003752.1956159
				2827 1003753.47679586
				2828 1003754.75797583
				2829 1003756.03915579
				2830 1003757.32033576
				2831 1003758.60151572
				2832 1003759.88269569
				2833 1003761.16387565
				2834 1003762.44505562
				2835 1003763.72623559
				2836 1003765.00741555
				2837 1003766.28859552
				2838 1003767.56977548
				2839 1003768.85095545
				2840 1003770.13213541
				2841 1003771.41331538
				2842 1003772.69449534
				2843 1003773.97567531
				2844 1003775.25685527
				2845 1003776.53803524
				2846 1003777.81921521
				2847 1003779.10039517
				2848 1003780.38157514
				2849 1003781.6627551
				2850 1003782.94393507
				2851 1003784.22511503
				2852 1003785.506295
				2853 1003786.78747496
				2854 1003788.06865493
				2855 1003789.34983489
				2856 1003790.63101486
				2857 1003791.91219483
				2858 1003793.19337479
				2859 1003794.47455476
				2860 1003795.75573472
				2861 1003797.03691469
				2862 1003798.31809465
				2863 1003799.59927462
				2864 1003800.88045458
				2865 1003802.16163455
				2866 1003803.44281451
				2867 1003804.72399448
				2868 1003806.00517445
				2869 1003807.28635441
				2870 1003808.56753438
				2871 1003809.84871434
				2872 1003811.12989431
				2873 1003812.41107427
				2874 1003813.69225424
				2875 1003814.9734342
				2876 1003816.25461417
				2877 1003817.53579413
				2878 1003818.8169741
				2879 1003820.09815407
				2880 1003821.37933403
				2881 1003822.660514
				2882 1003823.94169396
				2883 1003825.22287393
				2884 1003826.50405389
				2885 1003827.78523386
				2886 1003829.06641382
				2887 1003830.34759379
				2888 1003831.62877375
				2889 1003832.90995372
				2890 1003834.19113369
				2891 1003835.47231365
				2892 1003836.75349362
				2893 1003838.03467358
				2894 1003839.31585355
				2895 1003840.59703351
				2896 1003841.87821348
				2897 1003843.15939344
				2898 1003844.44057341
				2899 1003845.72175337
				2900 1003847.00293334
				2901 1003848.28411331
				2902 1003849.56529327
				2903 1003850.84647324
				2904 1003852.1276532
				2905 1003853.40883317
				2906 1003854.69001313
				2907 1003855.9711931
				2908 1003857.25237306
				2909 1003858.53355303
				2910 1003859.81473299
				2911 1003861.09591296
				2912 1003862.37709293
				2913 1003863.65827289
				2914 1003864.93945286
				2915 1003866.22063282
				2916 1003867.50181279
				2917 1003868.78299275
				2918 1003870.06417272
				2919 1003871.34535268
				2920 1003872.62653265
				2921 1003873.90771261
				2922 1003875.18889258
				2923 1003876.47007255
				2924 1003877.75125251
				2925 1003879.03243248
				2926 1003880.31361244
				2927 1003881.59479241
				2928 1003882.87597237
				2929 1003884.15715234
				2930 1003885.4383323
				2931 1003886.71951227
				2932 1003888.00069223
				2933 1003889.2818722
				2934 1003890.56305217
				2935 1003891.84423213
				2936 1003893.1254121
				2937 1003894.40659206
				2938 1003895.68777203
				2939 1003896.96895199
				2940 1003898.25013196
				2941 1003899.53131192
				2942 1003900.81249189
				2943 1003902.09367185
				2944 1003903.37485182
				2945 1003904.65603179
				2946 1003905.93721175
				2947 1003907.21839172
				2948 1003908.49957168
				2949 1003909.78075165
				2950 1003911.06193161
				2951 1003912.34311158
				2952 1003913.62429154
				2953 1003914.90547151
				2954 1003916.18665147
				2955 1003917.46783144
				2956 1003918.74901141
				2957 1003920.03019137
				2958 1003921.31137134
				2959 1003922.5925513
				2960 1003923.87373127
				2961 1003925.15491123
				2962 1003926.4360912
				2963 1003927.71727116
				2964 1003928.99845113
				2965 1003930.27963109
				2966 1003931.56081106
				2967 1003932.84199103
				2968 1003934.12317099
				2969 1003935.40435096
				2970 1003936.68553092
				2971 1003937.96671089
				2972 1003939.24789085
				2973 1003940.52907082
				2974 1003941.81025078
				2975 1003943.09143075
				2976 1003944.37261071
				2977 1003945.65379068
				2978 1003946.93497065
				2979 1003948.21615061
				2980 1003949.49733058
				2981 1003950.77851054
				2982 1003952.05969051
				2983 1003953.34087047
				2984 1003954.62205044
				2985 1003955.9032304
				2986 1003957.18441037
				2987 1003958.46559033
				2988 1003959.7467703
				2989 1003961.02795027
				2990 1003962.30913023
				2991 1003963.5903102
				2992 1003964.87149016
				2993 1003966.15267013
				2994 1003967.43385009
				2995 1003968.71503006
				2996 1003969.99621002
				2997 1003971.27738999
				2998 1003972.55856995
				2999 1003973.83974992
				3000 1003975.12092989
				3001 1003976.40210985
				3002 1003977.68328982
				3003 1003978.96446978
				3004 1003980.24564975
				3005 1003981.52682971
				3006 1003982.80800968
				3007 1003984.08918964
				3008 1003985.37036961
				3009 1003986.65154957
				3010 1003987.93272954
				3011 1003989.21390951
				3012 1003990.49508947
				3013 1003991.77626944
				3014 1003993.0574494
				3015 1003994.33862937
				3016 1003995.61980933
				3017 1003996.9009893
				3018 1003998.18216926
				3019 1003999.46334923
				3020 1004000.74452919
				3021 1004002.02570916
				3022 1004003.30688913
				3023 1004004.58806909
				3024 1004005.86924906
				3025 1004007.15042902
				3026 1004008.43160899
				3027 1004009.71278895
				3028 1004010.99396892
				3029 1004012.27514888
				3030 1004013.55632885
				3031 1004014.83750881
				3032 1004016.11868878
				3033 1004017.39986875
				3034 1004018.68104871
				3035 1004019.96222868
				3036 1004021.24340864
				3037 1004022.52458861
				3038 1004023.80576857
				3039 1004025.08694854
				3040 1004026.3681285
				3041 1004027.64930847
				3042 1004028.93048843
				3043 1004030.2116684
				3044 1004031.49284837
				3045 1004032.77402833
				3046 1004034.0552083
				3047 1004035.33638826
				3048 1004036.61756823
				3049 1004037.89874819
				3050 1004039.17992816
				3051 1004040.46110812
				3052 1004041.74228809
				3053 1004043.02346805
				3054 1004044.30464802
				3055 1004045.58582799
				3056 1004046.86700795
				3057 1004048.14818792
				3058 1004049.42936788
				3059 1004050.71054785
				3060 1004051.99172781
				3061 1004053.27290778
				3062 1004054.55408774
				3063 1004055.83526771
				3064 1004057.11644767
				3065 1004058.39762764
				3066 1004059.67880761
				3067 1004060.95998757
				3068 1004062.24116754
				3069 1004063.5223475
				3070 1004064.80352747
				3071 1004066.08470743
				3072 1004067.3658874
				3073 1004068.64706736
				3074 1004069.92824733
				3075 1004071.20942729
				3076 1004072.49060726
				3077 1004073.77178723
				3078 1004075.05296719
				3079 1004076.33414716
				3080 1004077.61532712
				3081 1004078.89650709
				3082 1004080.17768705
				3083 1004081.45886702
				3084 1004082.74004698
				3085 1004084.02122695
				3086 1004085.30240691
				3087 1004086.58358688
				3088 1004087.86476685
				3089 1004089.14594681
				3090 1004090.42712678
				3091 1004091.70830674
				3092 1004092.98948671
				3093 1004094.27066667
				3094 1004095.55184664
				3095 1004096.8330266
				3096 1004098.11420657
				3097 1004099.39538653
				3098 1004100.6765665
				3099 1004101.95774647
				3100 1004103.23892643
				3101 1004104.5201064
				3102 1004105.80128636
				3103 1004107.08246633
				3104 1004108.36364629
				3105 1004109.64482626
				3106 1004110.92600622
				3107 1004112.20718619
				3108 1004113.48836615
				3109 1004114.76954612
				3110 1004116.05072609
				3111 1004117.33190605
				3112 1004118.61308602
				3113 1004119.89426598
				3114 1004121.17544595
				3115 1004122.45662591
				3116 1004123.73780588
				3117 1004125.01898584
				3118 1004126.30016581
				3119 1004127.58134577
				3120 1004128.86252574
				3121 1004130.14370571
				3122 1004131.42488567
				3123 1004132.70606564
				3124 1004133.9872456
				3125 1004135.26842557
				3126 1004136.54960553
				3127 1004137.8307855
				3128 1004139.11196546
				3129 1004140.39314543
				3130 1004141.67432539
				3131 1004142.95550536
				3132 1004144.23668533
				3133 1004145.51786529
				3134 1004146.79904526
				3135 1004148.08022522
				3136 1004149.36140519
				3137 1004150.64258515
				3138 1004151.92376512
				3139 1004153.20494508
				3140 1004154.48612505
				3141 1004155.76730501
				3142 1004157.04848498
				3143 1004158.32966495
				3144 1004159.61084491
				3145 1004160.89202488
				3146 1004162.17320484
				3147 1004163.45438481
				3148 1004164.73556477
				3149 1004166.01674474
				3150 1004167.2979247
				3151 1004168.57910467
				3152 1004169.86028463
				3153 1004171.1414646
				3154 1004172.42264457
				3155 1004173.70382453
				3156 1004174.9850045
				3157 1004176.26618446
				3158 1004177.54736443
				3159 1004178.82854439
				3160 1004180.10972436
				3161 1004181.39090432
				3162 1004182.67208429
				3163 1004183.95326425
				3164 1004185.23444422
				3165 1004186.51562419
				3166 1004187.79680415
				3167 1004189.07798412
				3168 1004190.35916408
				3169 1004191.64034405
				3170 1004192.92152401
				3171 1004194.20270398
				3172 1004195.48388394
				3173 1004196.76506391
				3174 1004198.04624387
				3175 1004199.32742384
				3176 1004200.60860381
				3177 1004201.88978377
				3178 1004203.17096374
				3179 1004204.4521437
				3180 1004205.73332367
				3181 1004207.01450363
				3182 1004208.2956836
				3183 1004209.57686356
				3184 1004210.85804353
				3185 1004212.13922349
				3186 1004213.42040346
				3187 1004214.70158343
				3188 1004215.98276339
				3189 1004217.26394336
				3190 1004218.54512332
				3191 1004219.82630329
				3192 1004221.10748325
				3193 1004222.38866322
				3194 1004223.66984318
				3195 1004224.95102315
				3196 1004226.23220311
				3197 1004227.51338308
				3198 1004228.79456305
				3199 1004230.07574301
				3200 1004231.35692298
				3201 1004232.63810294
				3202 1004233.91928291
				3203 1004235.20046287
				3204 1004236.48164284
				3205 1004237.7628228
				3206 1004239.04400277
				3207 1004240.32518273
				3208 1004241.6063627
				3209 1004242.88754267
				3210 1004244.16872263
				3211 1004245.4499026
				3212 1004246.73108256
				3213 1004248.01226253
				3214 1004249.29344249
				3215 1004250.57462246
				3216 1004251.85580242
				3217 1004253.13698239
				3218 1004254.41816235
				3219 1004255.69934232
				3220 1004256.98052229
				3221 1004258.26170225
				3222 1004259.54288222
				3223 1004260.82406218
				3224 1004262.10524215
				3225 1004263.38642211
				3226 1004264.66760208
				3227 1004265.94878204
				3228 1004267.22996201
				3229 1004268.51114197
				3230 1004269.79232194
				3231 1004271.07350191
				3232 1004272.35468187
				3233 1004273.63586184
				3234 1004274.9170418
				3235 1004276.19822177
				3236 1004277.47940173
				3237 1004278.7605817
				3238 1004280.04176166
				3239 1004281.32294163
				3240 1004282.60412159
				3241 1004283.88530156
				3242 1004285.16648153
				3243 1004286.44766149
				3244 1004287.72884146
				3245 1004289.01002142
				3246 1004290.29120139
				3247 1004291.57238135
				3248 1004292.85356132
				3249 1004294.13474128
				3250 1004295.41592125
				3251 1004296.69710121
				3252 1004297.97828118
				3253 1004299.25946115
				3254 1004300.54064111
				3255 1004301.82182108
				3256 1004303.10300104
				3257 1004304.38418101
				3258 1004305.66536097
				3259 1004306.94654094
				3260 1004308.2277209
				3261 1004309.50890087
				3262 1004310.79008083
				3263 1004312.0712608
				3264 1004313.35244077
				3265 1004314.63362073
				3266 1004315.9148007
				3267 1004317.19598066
				3268 1004318.47716063
				3269 1004319.75834059
				3270 1004321.03952056
				3271 1004322.32070052
				3272 1004323.60188049
				3273 1004324.88306045
				3274 1004326.16424042
				3275 1004327.44542039
				3276 1004328.72660035
				3277 1004330.00778032
				3278 1004331.28896028
				3279 1004332.57014025
				3280 1004333.85132021
				3281 1004335.13250018
				3282 1004336.41368014
				3283 1004337.69486011
				3284 1004338.97604007
				3285 1004340.25722004
				3286 1004341.53840001
				3287 1004342.81957997
				3288 1004344.10075994
				3289 1004345.3819399
				3290 1004346.66311987
				3291 1004347.94429983
				3292 1004349.2254798
				3293 1004350.50665976
				3294 1004351.78783973
				3295 1004353.06901969
				3296 1004354.35019966
				3297 1004355.63137963
				3298 1004356.91255959
				3299 1004358.19373956
				3300 1004359.47491952
				3301 1004360.75609949
				3302 1004362.03727945
				3303 1004363.31845942
				3304 1004364.59963938
				3305 1004365.88081935
				3306 1004367.16199931
				3307 1004368.44317928
				3308 1004369.72435925
				3309 1004371.00553921
				3310 1004372.28671918
				3311 1004373.56789914
				3312 1004374.84907911
				3313 1004376.13025907
				3314 1004377.41143904
				3315 1004378.692619
				3316 1004379.97379897
				3317 1004381.25497893
				3318 1004382.5361589
				3319 1004383.81733887
				3320 1004385.09851883
				3321 1004386.3796988
				3322 1004387.66087876
				3323 1004388.94205873
				3324 1004390.22323869
				3325 1004391.50441866
				3326 1004392.78559862
				3327 1004394.06677859
				3328 1004395.34795855
				3329 1004396.62913852
				3330 1004397.91031849
				3331 1004399.19149845
				3332 1004400.47267842
				3333 1004401.75385838
				3334 1004403.03503835
				3335 1004404.31621831
				3336 1004405.59739828
				3337 1004406.87857824
				3338 1004408.15975821
				3339 1004409.44093817
				3340 1004410.72211814
				3341 1004412.00329811
				3342 1004413.28447807
				3343 1004414.56565804
				3344 1004415.846838
				3345 1004417.12801797
				3346 1004418.40919793
				3347 1004419.6903779
				3348 1004420.97155786
				3349 1004422.25273783
				3350 1004423.53391779
				3351 1004424.81509776
				3352 1004426.09627773
				3353 1004427.37745769
				3354 1004428.65863766
				3355 1004429.93981762
				3356 1004431.22099759
				3357 1004432.50217755
				3358 1004433.78335752
				3359 1004435.06453748
				3360 1004436.34571745
				3361 1004437.62689741
				3362 1004438.90807738
				3363 1004440.18925735
				3364 1004441.47043731
				3365 1004442.75161728
				3366 1004444.03279724
				3367 1004445.31397721
				3368 1004446.59515717
				3369 1004447.87633714
				3370 1004449.1575171
				3371 1004450.43869707
				3372 1004451.71987703
				3373 1004453.001057
				3374 1004454.28223697
				3375 1004455.56341693
				3376 1004456.8445969
				3377 1004458.12577686
				3378 1004459.40695683
				3379 1004460.68813679
				3380 1004461.96931676
				3381 1004463.25049672
				3382 1004464.53167669
				3383 1004465.81285665
				3384 1004467.09403662
				3385 1004468.37521659
				3386 1004469.65639655
				3387 1004470.93757652
				3388 1004472.21875648
				3389 1004473.49993645
				3390 1004474.78111641
				3391 1004476.06229638
				3392 1004477.34347634
				3393 1004478.62465631
				3394 1004479.90583627
				3395 1004481.18701624
				3396 1004482.46819621
				3397 1004483.74937617
				3398 1004485.03055614
				3399 1004486.3117361
				3400 1004487.59291607
				3401 1004488.87409603
				3402 1004490.155276
				3403 1004491.43645596
				3404 1004492.71763593
				3405 1004493.99881589
				3406 1004495.27999586
				3407 1004496.56117583
				3408 1004497.84235579
				3409 1004499.12353576
				3410 1004500.40471572
				3411 1004501.68589569
				3412 1004502.96707565
				3413 1004504.24825562
				3414 1004505.52943558
				3415 1004506.81061555
				3416 1004508.09179551
				3417 1004509.37297548
				3418 1004510.65415545
				3419 1004511.93533541
				3420 1004513.21651538
				3421 1004514.49769534
				3422 1004515.77887531
				3423 1004517.06005527
				3424 1004518.34123524
				3425 1004519.6224152
				3426 1004520.90359517
				3427 1004522.18477513
				3428 1004523.4659551
				3429 1004524.74713507
				3430 1004526.02831503
				3431 1004527.309495
				3432 1004528.59067496
				3433 1004529.87185493
				3434 1004531.15303489
				3435 1004532.43421486
				3436 1004533.71539482
				3437 1004534.99657479
				3438 1004536.27775475
				3439 1004537.55893472
				3440 1004538.84011469
				3441 1004540.12129465
				3442 1004541.40247462
				3443 1004542.68365458
				3444 1004543.96483455
				3445 1004545.24601451
				3446 1004546.52719448
				3447 1004547.80837444
				3448 1004549.08955441
				3449 1004550.37073437
				3450 1004551.65191434
				3451 1004552.93309431
				3452 1004554.21427427
				3453 1004555.49545424
				3454 1004556.7766342
				3455 1004558.05781417
				3456 1004559.33899413
				3457 1004560.6201741
				3458 1004561.90135406
				3459 1004563.18253403
				3460 1004564.46371399
				3461 1004565.74489396
				3462 1004567.02607393
				3463 1004568.30725389
				3464 1004569.58843386
				3465 1004570.86961382
				3466 1004572.15079379
				3467 1004573.43197375
				3468 1004574.71315372
				3469 1004575.99433368
				3470 1004577.27551365
				3471 1004578.55669361
				3472 1004579.83787358
				3473 1004581.11905355
				3474 1004582.40023351
				3475 1004583.68141348
				3476 1004584.96259344
				3477 1004586.24377341
				3478 1004587.52495337
				3479 1004588.80613334
				3480 1004590.0873133
				3481 1004591.36849327
				3482 1004592.64967323
				3483 1004593.9308532
				3484 1004595.21203317
				3485 1004596.49321313
				3486 1004597.7743931
				3487 1004599.05557306
				3488 1004600.33675303
				3489 1004601.61793299
				3490 1004602.89911296
				3491 1004604.18029292
				3492 1004605.46147289
				3493 1004606.74265285
				3494 1004608.02383282
				3495 1004609.30501279
				3496 1004610.58619275
				3497 1004611.86737272
				3498 1004613.14855268
				3499 1004614.42973265
				3500 1004615.71091261
				3501 1004616.99209258
				3502 1004618.27327254
				3503 1004619.55445251
				3504 1004620.83563247
				3505 1004622.11681244
				3506 1004623.39799241
				3507 1004624.67917237
				3508 1004625.96035234
				3509 1004627.2415323
				3510 1004628.52271227
				3511 1004629.80389223
				3512 1004631.0850722
				3513 1004632.36625216
				3514 1004633.64743213
				3515 1004634.92861209
				3516 1004636.20979206
				3517 1004637.49097203
				3518 1004638.77215199
				3519 1004640.05333196
				3520 1004641.33451192
				3521 1004642.61569189
				3522 1004643.89687185
				3523 1004645.17805182
				3524 1004646.45923178
				3525 1004647.74041175
				3526 1004649.02159171
				3527 1004650.30277168
				3528 1004651.58395165
				3529 1004652.86513161
				3530 1004654.14631158
				3531 1004655.42749154
				3532 1004656.70867151
				3533 1004657.98985147
				3534 1004659.27103144
				3535 1004660.5522114
				3536 1004661.83339137
				3537 1004663.11457133
				3538 1004664.3957513
				3539 1004665.67693127
				3540 1004666.95811123
				3541 1004668.2392912
				3542 1004669.52047116
				3543 1004670.80165113
				3544 1004672.08283109
				3545 1004673.36401106
				3546 1004674.64519102
				3547 1004675.92637099
				3548 1004677.20755095
				3549 1004678.48873092
				3550 1004679.76991089
				3551 1004681.05109085
				3552 1004682.33227082
				3553 1004683.61345078
				3554 1004684.89463075
				3555 1004686.17581071
				3556 1004687.45699068
				3557 1004688.73817064
				3558 1004690.01935061
				3559 1004691.30053057
				3560 1004692.58171054
				3561 1004693.86289051
				3562 1004695.14407047
				3563 1004696.42525044
				3564 1004697.7064304
				3565 1004698.98761037
				3566 1004700.26879033
				3567 1004701.5499703
				3568 1004702.83115026
				3569 1004704.11233023
				3570 1004705.39351019
				3571 1004706.67469016
				3572 1004707.95587013
				3573 1004709.23705009
				3574 1004710.51823006
				3575 1004711.79941002
				3576 1004713.08058999
				3577 1004714.36176995
				3578 1004715.64294992
				3579 1004716.92412988
				3580 1004718.20530985
				3581 1004719.48648981
				3582 1004720.76766978
				3583 1004722.04884975
				3584 1004723.33002971
				3585 1004724.61120968
				3586 1004725.89238964
				3587 1004727.17356961
				3588 1004728.45474957
				3589 1004729.73592954
				3590 1004731.0171095
				3591 1004732.29828947
				3592 1004733.57946943
				3593 1004734.8606494
				3594 1004736.14182937
				3595 1004737.42300933
				3596 1004738.7041893
				3597 1004739.98536926
				3598 1004741.26654923
				3599 1004742.54772919
				3600 1004743.82890916
				3601 1004745.11008912
				3602 1004746.39126909
				3603 1004747.67244905
				3604 1004748.95362902
				3605 1004750.23480899
				3606 1004751.51598895
				3607 1004752.79716892
				3608 1004754.07834888
				3609 1004755.35952885
				3610 1004756.64070881
				3611 1004757.92188878
				3612 1004759.20306874
				3613 1004760.48424871
				3614 1004761.76542867
				3615 1004763.04660864
				3616 1004764.32778861
				3617 1004765.60896857
				3618 1004766.89014854
				3619 1004768.1713285
				3620 1004769.45250847
				3621 1004770.73368843
				3622 1004772.0148684
				3623 1004773.29604836
				3624 1004774.57722833
				3625 1004775.85840829
				3626 1004777.13958826
				3627 1004778.42076822
				3628 1004779.70194819
				3629 1004780.98312816
				3630 1004782.26430812
				3631 1004783.54548809
				3632 1004784.82666805
				3633 1004786.10784802
				3634 1004787.38902798
				3635 1004788.67020795
				3636 1004789.95138791
				3637 1004791.23256788
				3638 1004792.51374785
				3639 1004793.79492781
				3640 1004795.07610778
				3641 1004796.35728774
				3642 1004797.63846771
				3643 1004798.91964767
				3644 1004800.20082764
				3645 1004801.4820076
				3646 1004802.76318757
				3647 1004804.04436753
				3648 1004805.3255475
				3649 1004806.60672747
				3650 1004807.88790743
				3651 1004809.1690874
				3652 1004810.45026736
				3653 1004811.73144733
				3654 1004813.01262729
				3655 1004814.29380726
				3656 1004815.57498722
				3657 1004816.85616719
				3658 1004818.13734715
				3659 1004819.41852712
				3660 1004820.69970709
				3661 1004821.98088705
				3662 1004823.26206702
				3663 1004824.54324698
				3664 1004825.82442695
				3665 1004827.10560691
				3666 1004828.38678688
				3667 1004829.66796684
				3668 1004830.94914681
				3669 1004832.23032677
				3670 1004833.51150674
				3671 1004834.79268671
				3672 1004836.07386667
				3673 1004837.35504664
				3674 1004838.6362266
				3675 1004839.91740657
				3676 1004841.19858653
				3677 1004842.4797665
				3678 1004843.76094646
				3679 1004845.04212643
				3680 1004846.32330639
				3681 1004847.60448636
				3682 1004848.88566632
				3683 1004850.16684629
				3684 1004851.44802626
				3685 1004852.72920622
				3686 1004854.01038619
				3687 1004855.29156615
				3688 1004856.57274612
				3689 1004857.85392608
				3690 1004859.13510605
				3691 1004860.41628601
				3692 1004861.69746598
				3693 1004862.97864594
				3694 1004864.25982591
				3695 1004865.54100588
				3696 1004866.82218584
				3697 1004868.10336581
				3698 1004869.38454577
				3699 1004870.66572574
				3700 1004871.9469057
				3701 1004873.22808567
				3702 1004874.50926563
				3703 1004875.7904456
				3704 1004877.07162556
				3705 1004878.35280553
				3706 1004879.6339855
				3707 1004880.91516546
				3708 1004882.19634543
				3709 1004883.47752539
				3710 1004884.75870536
				3711 1004886.03988532
				3712 1004887.32106529
				3713 1004888.60224525
				3714 1004889.88342522
				3715 1004891.16460518
				3716 1004892.44578515
				3717 1004893.72696512
				3718 1004895.00814508
				3719 1004896.28932505
				3720 1004897.57050501
				3721 1004898.85168498
				3722 1004900.13286494
				3723 1004901.41404491
				3724 1004902.69522487
				3725 1004903.97640484
				3726 1004905.2575848
				3727 1004906.53876477
				3728 1004907.81994474
				3729 1004909.1011247
				3730 1004910.38230467
				3731 1004911.66348463
				3732 1004912.9446646
				3733 1004914.22584456
				3734 1004915.50702453
				3735 1004916.78820449
				3736 1004918.06938446
				3737 1004919.35056442
				3738 1004920.63174439
				3739 1004921.91292436
				3740 1004923.19410432
				3741 1004924.47528429
				3742 1004925.75646425
				3743 1004927.03764422
				3744 1004928.31882418
				3745 1004929.60000415
				3746 1004930.88118411
				3747 1004932.16236408
				3748 1004933.44354404
				3749 1004934.72472401
				3750 1004936.00590398
				3751 1004937.28708394
				3752 1004938.56826391
				3753 1004939.84944387
				3754 1004941.13062384
				3755 1004942.4118038
				3756 1004943.69298377
				3757 1004944.97416373
				3758 1004946.2553437
				3759 1004947.53652366
				3760 1004948.81770363
				3761 1004950.0988836
				3762 1004951.38006356
				3763 1004952.66124353
				3764 1004953.94242349
				3765 1004955.22360346
				3766 1004956.50478342
				3767 1004957.78596339
				3768 1004959.06714335
				3769 1004960.34832332
				3770 1004961.62950328
				3771 1004962.91068325
				3772 1004964.19186322
				3773 1004965.47304318
				3774 1004966.75422315
				3775 1004968.03540311
				3776 1004969.31658308
				3777 1004970.59776304
				3778 1004971.87894301
				3779 1004973.16012297
				3780 1004974.44130294
				3781 1004975.7224829
				3782 1004977.00366287
				3783 1004978.28484284
				3784 1004979.5660228
				3785 1004980.84720277
				3786 1004982.12838273
				3787 1004983.4095627
				3788 1004984.69074266
				3789 1004985.97192263
				3790 1004987.25310259
				3791 1004988.53428256
				3792 1004989.81546252
				3793 1004991.09664249
				3794 1004992.37782246
				3795 1004993.65900242
				3796 1004994.94018239
				3797 1004996.22136235
				3798 1004997.50254232
				3799 1004998.78372228
				3800 1005000.06490225
				3801 1005001.34608221
				3802 1005002.62726218
				3803 1005003.90844214
				3804 1005005.18962211
				3805 1005006.47080208
				3806 1005007.75198204
				3807 1005009.03316201
				3808 1005010.31434197
				3809 1005011.59552194
				3810 1005012.8767019
				3811 1005014.15788187
				3812 1005015.43906183
				3813 1005016.7202418
				3814 1005018.00142176
				3815 1005019.28260173
				3816 1005020.5637817
				3817 1005021.84496166
				3818 1005023.12614163
				3819 1005024.40732159
				3820 1005025.68850156
				3821 1005026.96968152
				3822 1005028.25086149
				3823 1005029.53204145
				3824 1005030.81322142
				3825 1005032.09440138
				3826 1005033.37558135
				3827 1005034.65676132
				3828 1005035.93794128
				3829 1005037.21912125
				3830 1005038.50030121
				3831 1005039.78148118
				3832 1005041.06266114
				3833 1005042.34384111
				3834 1005043.62502107
				3835 1005044.90620104
				3836 1005046.187381
				3837 1005047.46856097
				3838 1005048.74974094
				3839 1005050.0309209
				3840 1005051.31210087
				3841 1005052.59328083
				3842 1005053.8744608
				3843 1005055.15564076
				3844 1005056.43682073
				3845 1005057.71800069
				3846 1005058.99918066
				3847 1005060.28036062
				3848 1005061.56154059
				3849 1005062.84272056
				3850 1005064.12390052
				3851 1005065.40508049
				3852 1005066.68626045
				3853 1005067.96744042
				3854 1005069.24862038
				3855 1005070.52980035
				3856 1005071.81098031
				3857 1005073.09216028
				3858 1005074.37334024
				3859 1005075.65452021
				3860 1005076.93570018
				3861 1005078.21688014
				3862 1005079.49806011
				3863 1005080.77924007
				3864 1005082.06042004
				3865 1005083.3416
				3866 1005084.62277997
				3867 1005085.90395993
				3868 1005087.1851399
				3869 1005088.46631986
				3870 1005089.74749983
				3871 1005091.0286798
				3872 1005092.30985976
				3873 1005093.59103973
				3874 1005094.87221969
				3875 1005096.15339966
				3876 1005097.43457962
				3877 1005098.71575959
				3878 1005099.99693955
				3879 1005101.27811952
				3880 1005102.55929948
				3881 1005103.84047945
				3882 1005105.12165942
				3883 1005106.40283938
				3884 1005107.68401935
				3885 1005108.96519931
				3886 1005110.24637928
				3887 1005111.52755924
				3888 1005112.80873921
				3889 1005114.08991917
				3890 1005115.37109914
				3891 1005116.6522791
				3892 1005117.93345907
				3893 1005119.21463904
				3894 1005120.495819
				3895 1005121.77699897
				3896 1005123.05817893
				3897 1005124.3393589
				3898 1005125.62053886
				3899 1005126.90171883
				3900 1005128.18289879
				3901 1005129.46407876
				3902 1005130.74525872
				3903 1005132.02643869
				3904 1005133.30761866
				3905 1005134.58879862
				3906 1005135.86997859
				3907 1005137.15115855
				3908 1005138.43233852
				3909 1005139.71351848
				3910 1005140.99469845
				3911 1005142.27587841
				3912 1005143.55705838
				3913 1005144.83823834
				3914 1005146.11941831
				3915 1005147.40059828
				3916 1005148.68177824
				3917 1005149.96295821
				3918 1005151.24413817
				3919 1005152.52531814
				3920 1005153.8064981
				3921 1005155.08767807
				3922 1005156.36885803
				3923 1005157.650038
				3924 1005158.93121796
				3925 1005160.21239793
				3926 1005161.4935779
				3927 1005162.77475786
				3928 1005164.05593783
				3929 1005165.33711779
				3930 1005166.61829776
				3931 1005167.89947772
				3932 1005169.18065769
				3933 1005170.46183765
				3934 1005171.74301762
				3935 1005173.02419758
				3936 1005174.30537755
				3937 1005175.58655752
				3938 1005176.86773748
				3939 1005178.14891745
				3940 1005179.43009741
				3941 1005180.71127738
				3942 1005181.99245734
				3943 1005183.27363731
				3944 1005184.55481727
				3945 1005185.83599724
				3946 1005187.1171772
				3947 1005188.39835717
				3948 1005189.67953714
				3949 1005190.9607171
				3950 1005192.24189707
				3951 1005193.52307703
				3952 1005194.804257
				3953 1005196.08543696
				3954 1005197.36661693
				3955 1005198.64779689
				3956 1005199.92897686
				3957 1005201.21015682
				3958 1005202.49133679
				3959 1005203.77251676
				3960 1005205.05369672
				3961 1005206.33487669
				3962 1005207.61605665
				3963 1005208.89723662
				3964 1005210.17841658
				3965 1005211.45959655
				3966 1005212.74077651
				3967 1005214.02195648
				3968 1005215.30313644
				3969 1005216.58431641
				3970 1005217.86549638
				3971 1005219.14667634
				3972 1005220.42785631
				3973 1005221.70903627
				3974 1005222.99021624
				3975 1005224.2713962
				3976 1005225.55257617
				3977 1005226.83375613
				3978 1005228.1149361
				3979 1005229.39611606
				3980 1005230.67729603
				3981 1005231.958476
				3982 1005233.23965596
				3983 1005234.52083593
				3984 1005235.80201589
				3985 1005237.08319586
				3986 1005238.36437582
				3987 1005239.64555579
				3988 1005240.92673575
				3989 1005242.20791572
				3990 1005243.48909568
				3991 1005244.77027565
				3992 1005246.05145562
				3993 1005247.33263558
				3994 1005248.61381555
				3995 1005249.89499551
				3996 1005251.17617548
				3997 1005252.45735544
				3998 1005253.73853541
				3999 1005255.01971537
				4000 1005256.30089534
				4001 1005257.5820753
				4002 1005258.86325527
				4003 1005260.14443524
				4004 1005261.4256152
				4005 1005262.70679517
				4006 1005263.98797513
				4007 1005265.2691551
				4008 1005266.55033506
				4009 1005267.83151503
				4010 1005269.11269499
				4011 1005270.39387496
				4012 1005271.67505492
				4013 1005272.95623489
				4014 1005274.23741486
				4015 1005275.51859482
				4016 1005276.79977479
				4017 1005278.08095475
				4018 1005279.36213472
				4019 1005280.64331468
				4020 1005281.92449465
				4021 1005283.20567461
				4022 1005284.48685458
				4023 1005285.76803454
				4024 1005287.04921451
				4025 1005288.33039448
				4026 1005289.61157444
				4027 1005290.89275441
				4028 1005292.17393437
				4029 1005293.45511434
				4030 1005294.7362943
				4031 1005296.01747427
				4032 1005297.29865423
				4033 1005298.5798342
				4034 1005299.86101416
				4035 1005301.14219413
				4036 1005302.4233741
				4037 1005303.70455406
				4038 1005304.98573403
				4039 1005306.26691399
				4040 1005307.54809396
				4041 1005308.82927392
				4042 1005310.11045389
				4043 1005311.39163385
				4044 1005312.67281382
				4045 1005313.95399378
				4046 1005315.23517375
				4047 1005316.51635372
				4048 1005317.79753368
				4049 1005319.07871365
				4050 1005320.35989361
				4051 1005321.64107358
				4052 1005322.92225354
				4053 1005324.20343351
				4054 1005325.48461347
				4055 1005326.76579344
				4056 1005328.0469734
				4057 1005329.32815337
				4058 1005330.60933334
				4059 1005331.8905133
				4060 1005333.17169327
				4061 1005334.45287323
				4062 1005335.7340532
				4063 1005337.01523316
				4064 1005338.29641313
				4065 1005339.57759309
				4066 1005340.85877306
				4067 1005342.13995302
				4068 1005343.42113299
				4069 1005344.70231296
				4070 1005345.98349292
				4071 1005347.26467289
				4072 1005348.54585285
				4073 1005349.82703282
				4074 1005351.10821278
				4075 1005352.38939275
				4076 1005353.67057271
				4077 1005354.95175268
				4078 1005356.23293264
				4079 1005357.51411261
				4080 1005358.79529258
				4081 1005360.07647254
				4082 1005361.35765251
				4083 1005362.63883247
				4084 1005363.92001244
				4085 1005365.2011924
				4086 1005366.48237237
				4087 1005367.76355233
				4088 1005369.0447323
				4089 1005370.32591226
				4090 1005371.60709223
				4091 1005372.8882722
				4092 1005374.16945216
				4093 1005375.45063213
				4094 1005376.73181209
				4095 1005378.01299206
				4096 1005379.29417202
				4097 1005380.57535199
				4098 1005381.85653195
				4099 1005383.13771192
				4100 1005384.41889188
				4101 1005385.70007185
				4102 1005386.98125182
				4103 1005388.26243178
				4104 1005389.54361175
				4105 1005390.82479171
				4106 1005392.10597168
				4107 1005393.38715164
				4108 1005394.66833161
				4109 1005395.94951157
				4110 1005397.23069154
				4111 1005398.5118715
				4112 1005399.79305147
				4113 1005401.07423144
				4114 1005402.3554114
				4115 1005403.63659137
				4116 1005404.91777133
				4117 1005406.1989513
				4118 1005407.48013126
				4119 1005408.76131123
				4120 1005410.04249119
				4121 1005411.32367116
				4122 1005412.60485112
				4123 1005413.88603109
				4124 1005415.16721106
				4125 1005416.44839102
				4126 1005417.72957099
				4127 1005419.01075095
				4128 1005420.29193092
				4129 1005421.57311088
				4130 1005422.85429085
				4131 1005424.13547081
				4132 1005425.41665078
				4133 1005426.69783074
				4134 1005427.97901071
				4135 1005429.26019068
				4136 1005430.54137064
				4137 1005431.82255061
				4138 1005433.10373057
				4139 1005434.38491054
				4140 1005435.6660905
				4141 1005436.94727047
				4142 1005438.22845043
				4143 1005439.5096304
				4144 1005440.79081036
				4145 1005442.07199033
				4146 1005443.3531703
				4147 1005444.63435026
				4148 1005445.91553023
				4149 1005447.19671019
				4150 1005448.47789016
				4151 1005449.75907012
				4152 1005451.04025009
				4153 1005452.32143005
				4154 1005453.60261002
				4155 1005454.88378998
				4156 1005456.16496995
				4157 1005457.44614992
				4158 1005458.72732988
				4159 1005460.00850985
				4160 1005461.28968981
				4161 1005462.57086978
				4162 1005463.85204974
				4163 1005465.13322971
				4164 1005466.41440967
				4165 1005467.69558964
				4166 1005468.9767696
				4167 1005470.25794957
				4168 1005471.53912954
				4169 1005472.8203095
				4170 1005474.10148947
				4171 1005475.38266943
				4172 1005476.6638494
				4173 1005477.94502936
				4174 1005479.22620933
				4175 1005480.50738929
				4176 1005481.78856926
				4177 1005483.06974922
				4178 1005484.35092919
				4179 1005485.63210916
				4180 1005486.91328912
				4181 1005488.19446909
				4182 1005489.47564905
				4183 1005490.75682902
				4184 1005492.03800898
				4185 1005493.31918895
				4186 1005494.60036891
				4187 1005495.88154888
				4188 1005497.16272884
				4189 1005498.44390881
				4190 1005499.72508878
				4191 1005501.00626874
				4192 1005502.28744871
				4193 1005503.56862867
				4194 1005504.84980864
				4195 1005506.1309886
				4196 1005507.41216857
				4197 1005508.69334853
				4198 1005509.9745285
				4199 1005511.25570846
				4200 1005512.53688843
				4201 1005513.8180684
				4202 1005515.09924836
				4203 1005516.38042833
				4204 1005517.66160829
				4205 1005518.94278826
				4206 1005520.22396822
				4207 1005521.50514819
				4208 1005522.78632815
				4209 1005524.06750812
				4210 1005525.34868808
				4211 1005526.62986805
				4212 1005527.91104802
				4213 1005529.19222798
				4214 1005530.47340795
				4215 1005531.75458791
				4216 1005533.03576788
				4217 1005534.31694784
				4218 1005535.59812781
				4219 1005536.87930777
				4220 1005538.16048774
				4221 1005539.4416677
				4222 1005540.72284767
				4223 1005542.00402764
				4224 1005543.2852076
				4225 1005544.56638757
				4226 1005545.84756753
				4227 1005547.1287475
				4228 1005548.40992746
				4229 1005549.69110743
				4230 1005550.97228739
				4231 1005552.25346736
				4232 1005553.53464732
				4233 1005554.81582729
				4234 1005556.09700726
				4235 1005557.37818722
				4236 1005558.65936719
				4237 1005559.94054715
				4238 1005561.22172712
				4239 1005562.50290708
				4240 1005563.78408705
				4241 1005565.06526701
				4242 1005566.34644698
				4243 1005567.62762694
				4244 1005568.90880691
				4245 1005570.18998688
				4246 1005571.47116684
				4247 1005572.75234681
				4248 1005574.03352677
				4249 1005575.31470674
				4250 1005576.5958867
				4251 1005577.87706667
				4252 1005579.15824663
				4253 1005580.4394266
				4254 1005581.72060656
				4255 1005583.00178653
				4256 1005584.2829665
				4257 1005585.56414646
				4258 1005586.84532643
				4259 1005588.12650639
				4260 1005589.40768636
				4261 1005590.68886632
				4262 1005591.97004629
				4263 1005593.25122625
				4264 1005594.53240622
				4265 1005595.81358618
				4266 1005597.09476615
				4267 1005598.37594612
				4268 1005599.65712608
				4269 1005600.93830605
				4270 1005602.21948601
				4271 1005603.50066598
				4272 1005604.78184594
				4273 1005606.06302591
				4274 1005607.34420587
				4275 1005608.62538584
				4276 1005609.9065658
				4277 1005611.18774577
				4278 1005612.46892574
				4279 1005613.7501057
				4280 1005615.03128567
				4281 1005616.31246563
				4282 1005617.5936456
				4283 1005618.87482556
				4284 1005620.15600553
				4285 1005621.43718549
				4286 1005622.71836546
				4287 1005623.99954542
				4288 1005625.28072539
				4289 1005626.56190536
				4290 1005627.84308532
				4291 1005629.12426529
				4292 1005630.40544525
				4293 1005631.68662522
				4294 1005632.96780518
				4295 1005634.24898515
				4296 1005635.53016511
				4297 1005636.81134508
				4298 1005638.09252504
				4299 1005639.37370501
				4300 1005640.65488498
				4301 1005641.93606494
				4302 1005643.21724491
				4303 1005644.49842487
				4304 1005645.77960484
				4305 1005647.0607848
				4306 1005648.34196477
				4307 1005649.62314473
				4308 1005650.9043247
				4309 1005652.18550466
				4310 1005653.46668463
				4311 1005654.7478646
				4312 1005656.02904456
				4313 1005657.31022453
				4314 1005658.59140449
				4315 1005659.87258446
				4316 1005661.15376442
				4317 1005662.43494439
				4318 1005663.71612435
				4319 1005664.99730432
				4320 1005666.27848428
				4321 1005667.55966425
				4322 1005668.84084422
				4323 1005670.12202418
				4324 1005671.40320415
				4325 1005672.68438411
				4326 1005673.96556408
				4327 1005675.24674404
				4328 1005676.52792401
				4329 1005677.80910397
				4330 1005679.09028394
				4331 1005680.3714639
				4332 1005681.65264387
				4333 1005682.93382384
				4334 1005684.2150038
				4335 1005685.49618377
				4336 1005686.77736373
				4337 1005688.0585437
				4338 1005689.33972366
				4339 1005690.62090363
				4340 1005691.90208359
				4341 1005693.18326356
				4342 1005694.46444352
				4343 1005695.74562349
				4344 1005697.02680346
				4345 1005698.30798342
				4346 1005699.58916339
				4347 1005700.87034335
				4348 1005702.15152332
				4349 1005703.43270328
				4350 1005704.71388325
				4351 1005705.99506321
				4352 1005707.27624318
				4353 1005708.55742314
				4354 1005709.83860311
				4355 1005711.11978308
				4356 1005712.40096304
				4357 1005713.68214301
				4358 1005714.96332297
				4359 1005716.24450294
				4360 1005717.5256829
				4361 1005718.80686287
				4362 1005720.08804283
				4363 1005721.3692228
				4364 1005722.65040276
				4365 1005723.93158273
				4366 1005725.2127627
				4367 1005726.49394266
				4368 1005727.77512263
				4369 1005729.05630259
				4370 1005730.33748256
				4371 1005731.61866252
				4372 1005732.89984249
				4373 1005734.18102245
				4374 1005735.46220242
				4375 1005736.74338238
				4376 1005738.02456235
				4377 1005739.30574232
				4378 1005740.58692228
				4379 1005741.86810225
				4380 1005743.14928221
				4381 1005744.43046218
				4382 1005745.71164214
				4383 1005746.99282211
				4384 1005748.27400207
				4385 1005749.55518204
				4386 1005750.836362
				4387 1005752.11754197
				4388 1005753.39872194
				4389 1005754.6799019
				4390 1005755.96108187
				4391 1005757.24226183
				4392 1005758.5234418
				4393 1005759.80462176
				4394 1005761.08580173
				4395 1005762.36698169
				4396 1005763.64816166
				4397 1005764.92934162
				4398 1005766.21052159
				4399 1005767.49170156
				4400 1005768.77288152
				4401 1005770.05406149
				4402 1005771.33524145
				4403 1005772.61642142
				4404 1005773.89760138
				4405 1005775.17878135
				4406 1005776.45996131
				4407 1005777.74114128
				4408 1005779.02232124
				4409 1005780.30350121
				4410 1005781.58468118
				4411 1005782.86586114
				4412 1005784.14704111
				4413 1005785.42822107
				4414 1005786.70940104
				4415 1005787.990581
				4416 1005789.27176097
				4417 1005790.55294093
				4418 1005791.8341209
				4419 1005793.11530086
				4420 1005794.39648083
				4421 1005795.6776608
				4422 1005796.95884076
				4423 1005798.24002073
				4424 1005799.52120069
				4425 1005800.80238066
				4426 1005802.08356062
				4427 1005803.36474059
				4428 1005804.64592055
				4429 1005805.92710052
				4430 1005807.20828048
				4431 1005808.48946045
				4432 1005809.77064042
				4433 1005811.05182038
				4434 1005812.33300035
				4435 1005813.61418031
				4436 1005814.89536028
				4437 1005816.17654024
				4438 1005817.45772021
				4439 1005818.73890017
				4440 1005820.02008014
				4441 1005821.3012601
				4442 1005822.58244007
				4443 1005823.86362004
				4444 1005825.1448
				4445 1005826.42597997
				4446 1005827.70715993
				4447 1005828.9883399
				4448 1005830.26951986
				4449 1005831.55069983
				4450 1005832.83187979
				4451 1005834.11305976
				4452 1005835.39423972
				4453 1005836.67541969
				4454 1005837.95659966
				4455 1005839.23777962
				4456 1005840.51895959
				4457 1005841.80013955
				4458 1005843.08131952
				4459 1005844.36249948
				4460 1005845.64367945
				4461 1005846.92485941
				4462 1005848.20603938
				4463 1005849.48721934
				4464 1005850.76839931
				4465 1005852.04957928
				4466 1005853.33075924
				4467 1005854.61193921
				4468 1005855.89311917
				4469 1005857.17429914
				4470 1005858.4554791
				4471 1005859.73665907
				4472 1005861.01783903
				4473 1005862.299019
				4474 1005863.58019896
				4475 1005864.86137893
				4476 1005866.1425589
				4477 1005867.42373886
				4478 1005868.70491883
				4479 1005869.98609879
				4480 1005871.26727876
				4481 1005872.54845872
				4482 1005873.82963869
				4483 1005875.11081865
				4484 1005876.39199862
				4485 1005877.67317858
				4486 1005878.95435855
				4487 1005880.23553852
				4488 1005881.51671848
				4489 1005882.79789845
				4490 1005884.07907841
				4491 1005885.36025838
				4492 1005886.64143834
				4493 1005887.92261831
				4494 1005889.20379827
				4495 1005890.48497824
				4496 1005891.7661582
				4497 1005893.04733817
				4498 1005894.32851814
				4499 1005895.6096981
				4500 1005896.89087807
				4501 1005898.17205803
				4502 1005899.453238
				4503 1005900.73441796
				4504 1005902.01559793
				4505 1005903.29677789
				4506 1005904.57795786
				4507 1005905.85913782
				4508 1005907.14031779
				4509 1005908.42149776
				4510 1005909.70267772
				4511 1005910.98385769
				4512 1005912.26503765
				4513 1005913.54621762
				4514 1005914.82739758
				4515 1005916.10857755
				4516 1005917.38975751
				4517 1005918.67093748
				4518 1005919.95211744
				4519 1005921.23329741
				4520 1005922.51447738
				4521 1005923.79565734
				4522 1005925.07683731
				4523 1005926.35801727
				4524 1005927.63919724
				4525 1005928.9203772
				4526 1005930.20155717
				4527 1005931.48273713
				4528 1005932.7639171
				4529 1005934.04509706
				4530 1005935.32627703
				4531 1005936.607457
				4532 1005937.88863696
				4533 1005939.16981693
				4534 1005940.45099689
				4535 1005941.73217686
				4536 1005943.01335682
				4537 1005944.29453679
				4538 1005945.57571675
				4539 1005946.85689672
				4540 1005948.13807668
				4541 1005949.41925665
				4542 1005950.70043662
				4543 1005951.98161658
				4544 1005953.26279655
				4545 1005954.54397651
				4546 1005955.82515648
				4547 1005957.10633644
				4548 1005958.38751641
				4549 1005959.66869637
				4550 1005960.94987634
				4551 1005962.2310563
				4552 1005963.51223627
				4553 1005964.79341624
				4554 1005966.0745962
				4555 1005967.35577617
				4556 1005968.63695613
				4557 1005969.9181361
				4558 1005971.19931606
				4559 1005972.48049603
				4560 1005973.76167599
				4561 1005975.04285596
				4562 1005976.32403592
				4563 1005977.60521589
				4564 1005978.88639586
				4565 1005980.16757582
				4566 1005981.44875579
				4567 1005982.72993575
				4568 1005984.01111572
				4569 1005985.29229568
				4570 1005986.57347565
				4571 1005987.85465561
				4572 1005989.13583558
				4573 1005990.41701554
				4574 1005991.69819551
				4575 1005992.97937548
				4576 1005994.26055544
				4577 1005995.54173541
				4578 1005996.82291537
				4579 1005998.10409534
				4580 1005999.3852753
				4581 1006000.66645527
				4582 1006001.94763523
				4583 1006003.2288152
				4584 1006004.50999516
				4585 1006005.79117513
				4586 1006007.0723551
				4587 1006008.35353506
				4588 1006009.63471503
				4589 1006010.91589499
				4590 1006012.19707496
				4591 1006013.47825492
				4592 1006014.75943489
				4593 1006016.04061485
				4594 1006017.32179482
				4595 1006018.60297478
				4596 1006019.88415475
				4597 1006021.16533472
				4598 1006022.44651468
				4599 1006023.72769465
				4600 1006025.00887461
				4601 1006026.29005458
				4602 1006027.57123454
				4603 1006028.85241451
				4604 1006030.13359447
				4605 1006031.41477444
				4606 1006032.6959544
				4607 1006033.97713437
				4608 1006035.25831434
				4609 1006036.5394943
				4610 1006037.82067427
				4611 1006039.10185423
				4612 1006040.3830342
				4613 1006041.66421416
				4614 1006042.94539413
				4615 1006044.22657409
				4616 1006045.50775406
				4617 1006046.78893402
				4618 1006048.07011399
				4619 1006049.35129396
				4620 1006050.63247392
				4621 1006051.91365389
				4622 1006053.19483385
				4623 1006054.47601382
				4624 1006055.75719378
				4625 1006057.03837375
				4626 1006058.31955371
				4627 1006059.60073368
				4628 1006060.88191364
				4629 1006062.16309361
				4630 1006063.44427358
				4631 1006064.72545354
				4632 1006066.00663351
				4633 1006067.28781347
				4634 1006068.56899344
				4635 1006069.8501734
				4636 1006071.13135337
				4637 1006072.41253333
				4638 1006073.6937133
				4639 1006074.97489326
				4640 1006076.25607323
				4641 1006077.5372532
				4642 1006078.81843316
				4643 1006080.09961313
				4644 1006081.38079309
				4645 1006082.66197306
				4646 1006083.94315302
				4647 1006085.22433299
				4648 1006086.50551295
				4649 1006087.78669292
				4650 1006089.06787288
				4651 1006090.34905285
				4652 1006091.63023282
				4653 1006092.91141278
				4654 1006094.19259275
				4655 1006095.47377271
				4656 1006096.75495268
				4657 1006098.03613264
				4658 1006099.31731261
				4659 1006100.59849257
				4660 1006101.87967254
				4661 1006103.1608525
				4662 1006104.44203247
				4663 1006105.72321244
				4664 1006107.0043924
				4665 1006108.28557237
				4666 1006109.56675233
				4667 1006110.8479323
				4668 1006112.12911226
				4669 1006113.41029223
				4670 1006114.69147219
				4671 1006115.97265216
				4672 1006117.25383212
				4673 1006118.53501209
				4674 1006119.81619206
				4675 1006121.09737202
				4676 1006122.37855199
				4677 1006123.65973195
				4678 1006124.94091192
				4679 1006126.22209188
				4680 1006127.50327185
				4681 1006128.78445181
				4682 1006130.06563178
				4683 1006131.34681174
				4684 1006132.62799171
				4685 1006133.90917168
				4686 1006135.19035164
				4687 1006136.47153161
				4688 1006137.75271157
				4689 1006139.03389154
				4690 1006140.3150715
				4691 1006141.59625147
				4692 1006142.87743143
				4693 1006144.1586114
				4694 1006145.43979136
				4695 1006146.72097133
				4696 1006148.0021513
				4697 1006149.28333126
				4698 1006150.56451123
				4699 1006151.84569119
				4700 1006153.12687116
				4701 1006154.40805112
				4702 1006155.68923109
				4703 1006156.97041105
				4704 1006158.25159102
				4705 1006159.53277098
				4706 1006160.81395095
				4707 1006162.09513092
				4708 1006163.37631088
				4709 1006164.65749085
				4710 1006165.93867081
				4711 1006167.21985078
				4712 1006168.50103074
				4713 1006169.78221071
				4714 1006171.06339067
				4715 1006172.34457064
				4716 1006173.6257506
				4717 1006174.90693057
				4718 1006176.18811054
				4719 1006177.4692905
				4720 1006178.75047047
				4721 1006180.03165043
				4722 1006181.3128304
				4723 1006182.59401036
				4724 1006183.87519033
				4725 1006185.15637029
				4726 1006186.43755026
				4727 1006187.71873022
				4728 1006188.99991019
				4729 1006190.28109016
				4730 1006191.56227012
				4731 1006192.84345009
				4732 1006194.12463005
				4733 1006195.40581002
				4734 1006196.68698998
				4735 1006197.96816995
				4736 1006199.24934991
				4737 1006200.53052988
				4738 1006201.81170984
				4739 1006203.09288981
				4740 1006204.37406978
				4741 1006205.65524974
				4742 1006206.93642971
				4743 1006208.21760967
				4744 1006209.49878964
				4745 1006210.7799696
				4746 1006212.06114957
				4747 1006213.34232953
				4748 1006214.6235095
				4749 1006215.90468946
				4750 1006217.18586943
				4751 1006218.4670494
				4752 1006219.74822936
				4753 1006221.02940933
				4754 1006222.31058929
				4755 1006223.59176926
				4756 1006224.87294922
				4757 1006226.15412919
				4758 1006227.43530915
				4759 1006228.71648912
				4760 1006229.99766908
				4761 1006231.27884905
				4762 1006232.56002902
				4763 1006233.84120898
				4764 1006235.12238895
				4765 1006236.40356891
				4766 1006237.68474888
				4767 1006238.96592884
				4768 1006240.24710881
				4769 1006241.52828877
				4770 1006242.80946874
				4771 1006244.0906487
				4772 1006245.37182867
				4773 1006246.65300864
				4774 1006247.9341886
				4775 1006249.21536857
				4776 1006250.49654853
				4777 1006251.7777285
				4778 1006253.05890846
				4779 1006254.34008843
				4780 1006255.62126839
				4781 1006256.90244836
				4782 1006258.18362832
				4783 1006259.46480829
				4784 1006260.74598826
				4785 1006262.02716822
				4786 1006263.30834819
				4787 1006264.58952815
				4788 1006265.87070812
				4789 1006267.15188808
				4790 1006268.43306805
				4791 1006269.71424801
				4792 1006270.99542798
				4793 1006272.27660794
				4794 1006273.55778791
				4795 1006274.83896788
				4796 1006276.12014784
				4797 1006277.40132781
				4798 1006278.68250777
				4799 1006279.96368774
				4800 1006281.2448677
				4801 1006282.52604767
				4802 1006283.80722763
				4803 1006285.0884076
				4804 1006286.36958756
				4805 1006287.65076753
				4806 1006288.9319475
				4807 1006290.21312746
				4808 1006291.49430743
				4809 1006292.77548739
				4810 1006294.05666736
				4811 1006295.33784732
				4812 1006296.61902729
				4813 1006297.90020725
				4814 1006299.18138722
				4815 1006300.46256718
				4816 1006301.74374715
				4817 1006303.02492712
				4818 1006304.30610708
				4819 1006305.58728705
				4820 1006306.86846701
				4821 1006308.14964698
				4822 1006309.43082694
				4823 1006310.71200691
				4824 1006311.99318687
				4825 1006313.27436684
				4826 1006314.5555468
				4827 1006315.83672677
				4828 1006317.11790674
				4829 1006318.3990867
				4830 1006319.68026667
				4831 1006320.96144663
				4832 1006322.2426266
				4833 1006323.52380656
				4834 1006324.80498653
				4835 1006326.08616649
				4836 1006327.36734646
				4837 1006328.64852642
				4838 1006329.92970639
				4839 1006331.21088636
				4840 1006332.49206632
				4841 1006333.77324629
				4842 1006335.05442625
				4843 1006336.33560622
				4844 1006337.61678618
				4845 1006338.89796615
				4846 1006340.17914611
				4847 1006341.46032608
				4848 1006342.74150604
				4849 1006344.02268601
				4850 1006345.30386598
				4851 1006346.58504594
				4852 1006347.86622591
				4853 1006349.14740587
				4854 1006350.42858584
				4855 1006351.7097658
				4856 1006352.99094577
				4857 1006354.27212573
				4858 1006355.5533057
				4859 1006356.83448566
				4860 1006358.11566563
				4861 1006359.3968456
				4862 1006360.67802556
				4863 1006361.95920553
				4864 1006363.24038549
				4865 1006364.52156546
				4866 1006365.80274542
				4867 1006367.08392539
				4868 1006368.36510535
				4869 1006369.64628532
				4870 1006370.92746528
				4871 1006372.20864525
				4872 1006373.48982522
				4873 1006374.77100518
				4874 1006376.05218515
				4875 1006377.33336511
				4876 1006378.61454508
				4877 1006379.89572504
				4878 1006381.17690501
				4879 1006382.45808497
				4880 1006383.73926494
				4881 1006385.0204449
				4882 1006386.30162487
				4883 1006387.58280484
				4884 1006388.8639848
				4885 1006390.14516477
				4886 1006391.42634473
				4887 1006392.7075247
				4888 1006393.98870466
				4889 1006395.26988463
				4890 1006396.55106459
				4891 1006397.83224456
				4892 1006399.11342452
				4893 1006400.39460449
				4894 1006401.67578446
				4895 1006402.95696442
				4896 1006404.23814439
				4897 1006405.51932435
				4898 1006406.80050432
				4899 1006408.08168428
				4900 1006409.36286425
				4901 1006410.64404421
				4902 1006411.92522418
				4903 1006413.20640414
				4904 1006414.48758411
				4905 1006415.76876408
				4906 1006417.04994404
				4907 1006418.33112401
				4908 1006419.61230397
				4909 1006420.89348394
				4910 1006422.1746639
				4911 1006423.45584387
				4912 1006424.73702383
				4913 1006426.0182038
				4914 1006427.29938376
				4915 1006428.58056373
				4916 1006429.86174369
				4917 1006431.14292366
				4918 1006432.42410363
				4919 1006433.70528359
				4920 1006434.98646356
				4921 1006436.26764352
				4922 1006437.54882349
				4923 1006438.83000345
				4924 1006440.11118342
				4925 1006441.39236338
				4926 1006442.67354335
				4927 1006443.95472331
				4928 1006445.23590328
				4929 1006446.51708325
				4930 1006447.79826321
				4931 1006449.07944318
				4932 1006450.36062314
				4933 1006451.64180311
				4934 1006452.92298307
				4935 1006454.20416304
				4936 1006455.485343
				4937 1006456.76652297
				4938 1006458.04770293
				4939 1006459.3288829
				4940 1006460.61006287
				4941 1006461.89124283
				4942 1006463.1724228
				4943 1006464.45360276
				4944 1006465.73478273
				4945 1006467.01596269
				4946 1006468.29714266
				4947 1006469.57832262
				4948 1006470.85950259
				4949 1006472.14068256
				4950 1006473.42186252
				4951 1006474.70304249
				4952 1006475.98422245
				4953 1006477.26540242
				4954 1006478.54658238
				4955 1006479.82776235
				4956 1006481.10894231
				4957 1006482.39012228
				4958 1006483.67130224
				4959 1006484.95248221
				4960 1006486.23366218
				4961 1006487.51484214
				4962 1006488.79602211
				4963 1006490.07720207
				4964 1006491.35838204
				4965 1006492.639562
				4966 1006493.92074197
				4967 1006495.20192193
				4968 1006496.4831019
				4969 1006497.76428186
				4970 1006499.04546183
				4971 1006500.32664179
				4972 1006501.60782176
				4973 1006502.88900173
				4974 1006504.17018169
				4975 1006505.45136166
				4976 1006506.73254162
				4977 1006508.01372159
				4978 1006509.29490155
				4979 1006510.57608152
				4980 1006511.85726148
				4981 1006513.13844145
				4982 1006514.41962141
				4983 1006515.70080138
				4984 1006516.98198135
				4985 1006518.26316131
				4986 1006519.54434128
				4987 1006520.82552124
				4988 1006522.10670121
				4989 1006523.38788117
				4990 1006524.66906114
				4991 1006525.9502411
				4992 1006527.23142107
				4993 1006528.51260103
				4994 1006529.793781
				4995 1006531.07496097
				4996 1006532.35614093
				4997 1006533.6373209
				4998 1006534.91850086
				4999 1006536.19968083
				5000 1006537.48086079
				5001 1006538.76204076
				5002 1006540.04322072
				5003 1006541.32440069
				5004 1006542.60558065
				5005 1006543.88676062
				5006 1006545.16794059
				5007 1006546.44912055
				5008 1006547.73030052
				5009 1006549.01148048
				5010 1006550.29266045
				5011 1006551.57384041
				5012 1006552.85502038
				5013 1006554.13620034
				5014 1006555.41738031
				5015 1006556.69856027
				5016 1006557.97974024
				5017 1006559.26092021
				5018 1006560.54210017
				5019 1006561.82328014
				5020 1006563.1044601
				5021 1006564.38564007
				5022 1006565.66682003
				5023 1006566.948
				5024 1006568.22917996
				5025 1006569.51035993
				5026 1006570.79153989
				5027 1006572.07271986
				5028 1006573.35389983
				5029 1006574.63507979
				5030 1006575.91625976
				5031 1006577.19743972
				5032 1006578.47861969
				5033 1006579.75979965
				5034 1006581.04097962
				5035 1006582.32215958
				5036 1006583.60333955
				5037 1006584.88451951
				5038 1006586.16569948
				5039 1006587.44687945
				5040 1006588.72805941
				5041 1006590.00923938
				5042 1006591.29041934
				5043 1006592.57159931
				5044 1006593.85277927
				5045 1006595.13395924
				5046 1006596.4151392
				5047 1006597.69631917
				5048 1006598.97749913
				5049 1006600.2586791
				5050 1006601.53985907
				5051 1006602.82103903
				5052 1006604.102219
				5053 1006605.38339896
				5054 1006606.66457893
				5055 1006607.94575889
				5056 1006609.22693886
				5057 1006610.50811882
				5058 1006611.78929879
				5059 1006613.07047875
				5060 1006614.35165872
				5061 1006615.63283869
				5062 1006616.91401865
				5063 1006618.19519862
				5064 1006619.47637858
				5065 1006620.75755855
				5066 1006622.03873851
				5067 1006623.31991848
				5068 1006624.60109844
				5069 1006625.88227841
				5070 1006627.16345837
				5071 1006628.44463834
				5072 1006629.72581831
				5073 1006631.00699827
				5074 1006632.28817824
				5075 1006633.5693582
				5076 1006634.85053817
				5077 1006636.13171813
				5078 1006637.4128981
				5079 1006638.69407806
				5080 1006639.97525803
				5081 1006641.25643799
				5082 1006642.53761796
				5083 1006643.81879793
				5084 1006645.09997789
				5085 1006646.38115786
				5086 1006647.66233782
				5087 1006648.94351779
				5088 1006650.22469775
				5089 1006651.50587772
				5090 1006652.78705768
				5091 1006654.06823765
				5092 1006655.34941761
				5093 1006656.63059758
				5094 1006657.91177755
				5095 1006659.19295751
				5096 1006660.47413748
				5097 1006661.75531744
				5098 1006663.03649741
				5099 1006664.31767737
				5100 1006665.59885734
				5101 1006666.8800373
				5102 1006668.16121727
				5103 1006669.44239723
				5104 1006670.7235772
				5105 1006672.00475717
				5106 1006673.28593713
				5107 1006674.5671171
				5108 1006675.84829706
				5109 1006677.12947703
				5110 1006678.41065699
				5111 1006679.69183696
				5112 1006680.97301692
				5113 1006682.25419689
				5114 1006683.53537685
				5115 1006684.81655682
				5116 1006686.09773679
				5117 1006687.37891675
				5118 1006688.66009672
				5119 1006689.94127668
				5120 1006691.22245665
				5121 1006692.50363661
				5122 1006693.78481658
				5123 1006695.06599654
				5124 1006696.34717651
				5125 1006697.62835647
				5126 1006698.90953644
				5127 1006700.19071641
				5128 1006701.47189637
				5129 1006702.75307634
				5130 1006704.0342563
				5131 1006705.31543627
				5132 1006706.59661623
				5133 1006707.8777962
				5134 1006709.15897616
				5135 1006710.44015613
				5136 1006711.72133609
				5137 1006713.00251606
				5138 1006714.28369603
				5139 1006715.56487599
				5140 1006716.84605596
				5141 1006718.12723592
				5142 1006719.40841589
				5143 1006720.68959585
				5144 1006721.97077582
				5145 1006723.25195578
				5146 1006724.53313575
				5147 1006725.81431571
				5148 1006727.09549568
				5149 1006728.37667565
				5150 1006729.65785561
				5151 1006730.93903558
				5152 1006732.22021554
				5153 1006733.50139551
				5154 1006734.78257547
				5155 1006736.06375544
				5156 1006737.3449354
				5157 1006738.62611537
				5158 1006739.90729533
				5159 1006741.1884753
				5160 1006742.46965527
				5161 1006743.75083523
				5162 1006745.0320152
				5163 1006746.31319516
				5164 1006747.59437513
				5165 1006748.87555509
				5166 1006750.15673506
				5167 1006751.43791502
				5168 1006752.71909499
				5169 1006754.00027495
				5170 1006755.28145492
				5171 1006756.56263489
				5172 1006757.84381485
				5173 1006759.12499482
				5174 1006760.40617478
				5175 1006761.68735475
				5176 1006762.96853471
				5177 1006764.24971468
				5178 1006765.53089464
				5179 1006766.81207461
				5180 1006768.09325457
				5181 1006769.37443454
				5182 1006770.65561451
				5183 1006771.93679447
				5184 1006773.21797444
				5185 1006774.4991544
				5186 1006775.78033437
				5187 1006777.06151433
				5188 1006778.3426943
				5189 1006779.62387426
				5190 1006780.90505423
				5191 1006782.18623419
				5192 1006783.46741416
				5193 1006784.74859413
				5194 1006786.02977409
				5195 1006787.31095406
				5196 1006788.59213402
				5197 1006789.87331399
				5198 1006791.15449395
				5199 1006792.43567392
				5200 1006793.71685388
				5201 1006794.99803385
				5202 1006796.27921381
				5203 1006797.56039378
				5204 1006798.84157375
				5205 1006800.12275371
				5206 1006801.40393368
				5207 1006802.68511364
				5208 1006803.96629361
				5209 1006805.24747357
				5210 1006806.52865354
				5211 1006807.8098335
				5212 1006809.09101347
				5213 1006810.37219343
				5214 1006811.6533734
				5215 1006812.93455337
				5216 1006814.21573333
				5217 1006815.4969133
				5218 1006816.77809326
				5219 1006818.05927323
				5220 1006819.34045319
				5221 1006820.62163316
				5222 1006821.90281312
				5223 1006823.18399309
				5224 1006824.46517305
				5225 1006825.74635302
				5226 1006827.02753299
				5227 1006828.30871295
				5228 1006829.58989292
				5229 1006830.87107288
				5230 1006832.15225285
				5231 1006833.43343281
				5232 1006834.71461278
				5233 1006835.99579274
				5234 1006837.27697271
				5235 1006838.55815267
				5236 1006839.83933264
				5237 1006841.12051261
				5238 1006842.40169257
				5239 1006843.68287254
				5240 1006844.9640525
				5241 1006846.24523247
				5242 1006847.52641243
				5243 1006848.8075924
				5244 1006850.08877236
				5245 1006851.36995233
				5246 1006852.65113229
				5247 1006853.93231226
				5248 1006855.21349223
				5249 1006856.49467219
				5250 1006857.77585216
				5251 1006859.05703212
				5252 1006860.33821209
				5253 1006861.61939205
				5254 1006862.90057202
				5255 1006864.18175198
				5256 1006865.46293195
				5257 1006866.74411191
				5258 1006868.02529188
				5259 1006869.30647185
				5260 1006870.58765181
				5261 1006871.86883178
				5262 1006873.15001174
				5263 1006874.43119171
				5264 1006875.71237167
				5265 1006876.99355164
				5266 1006878.2747316
				5267 1006879.55591157
				5268 1006880.83709153
				5269 1006882.1182715
				5270 1006883.39945147
				5271 1006884.68063143
				5272 1006885.9618114
				5273 1006887.24299136
				5274 1006888.52417133
				5275 1006889.80535129
				5276 1006891.08653126
				5277 1006892.36771122
				5278 1006893.64889119
				5279 1006894.93007115
				5280 1006896.21125112
				5281 1006897.49243109
				5282 1006898.77361105
				5283 1006900.05479102
				5284 1006901.33597098
				5285 1006902.61715095
				5286 1006903.89833091
				5287 1006905.17951088
				5288 1006906.46069084
				5289 1006907.74187081
				5290 1006909.02305077
				5291 1006910.30423074
				5292 1006911.58541071
				5293 1006912.86659067
				5294 1006914.14777064
				5295 1006915.4289506
				5296 1006916.71013057
				5297 1006917.99131053
				5298 1006919.2724905
				5299 1006920.55367046
				5300 1006921.83485043
				5301 1006923.11603039
				5302 1006924.39721036
				5303 1006925.67839033
				5304 1006926.95957029
				5305 1006928.24075026
				5306 1006929.52193022
				5307 1006930.80311019
				5308 1006932.08429015
				5309 1006933.36547012
				5310 1006934.64665008
				5311 1006935.92783005
				5312 1006937.20901001
				5313 1006938.49018998
				5314 1006939.77136995
				5315 1006941.05254991
				5316 1006942.33372988
				5317 1006943.61490984
				5318 1006944.89608981
				5319 1006946.17726977
				5320 1006947.45844974
				5321 1006948.7396297
				5322 1006950.02080967
				5323 1006951.30198963
				5324 1006952.5831696
				5325 1006953.86434957
				5326 1006955.14552953
				5327 1006956.4267095
				5328 1006957.70788946
				5329 1006958.98906943
				5330 1006960.27024939
				5331 1006961.55142936
				5332 1006962.83260932
				5333 1006964.11378929
				5334 1006965.39496925
				5335 1006966.67614922
				5336 1006967.95732919
				5337 1006969.23850915
				5338 1006970.51968912
				5339 1006971.80086908
				5340 1006973.08204905
				5341 1006974.36322901
				5342 1006975.64440898
				5343 1006976.92558894
				5344 1006978.20676891
				5345 1006979.48794887
				5346 1006980.76912884
				5347 1006982.05030881
				5348 1006983.33148877
				5349 1006984.61266874
				5350 1006985.8938487
				5351 1006987.17502867
				5352 1006988.45620863
				5353 1006989.7373886
				5354 1006991.01856856
				5355 1006992.29974853
				5356 1006993.58092849
				5357 1006994.86210846
				5358 1006996.14328843
				5359 1006997.42446839
				5360 1006998.70564836
				5361 1006999.98682832
				5362 1007001.26800829
				5363 1007002.54918825
				5364 1007003.83036822
				5365 1007005.11154818
				5366 1007006.39272815
				5367 1007007.67390811
				5368 1007008.95508808
				5369 1007010.23626805
				5370 1007011.51744801
				5371 1007012.79862798
				5372 1007014.07980794
				5373 1007015.36098791
				5374 1007016.64216787
				5375 1007017.92334784
				5376 1007019.2045278
				5377 1007020.48570777
				5378 1007021.76688773
				5379 1007023.0480677
				5380 1007024.32924767
				5381 1007025.61042763
				5382 1007026.8916076
				5383 1007028.17278756
				5384 1007029.45396753
				5385 1007030.73514749
				5386 1007032.01632746
				5387 1007033.29750742
				5388 1007034.57868739
				5389 1007035.85986735
				5390 1007037.14104732
				5391 1007038.42222729
				5392 1007039.70340725
				5393 1007040.98458722
				5394 1007042.26576718
				5395 1007043.54694715
				5396 1007044.82812711
				5397 1007046.10930708
				5398 1007047.39048704
				5399 1007048.67166701
				5400 1007049.95284697
				5401 1007051.23402694
				5402 1007052.51520691
				5403 1007053.79638687
				5404 1007055.07756684
				5405 1007056.3587468
				5406 1007057.63992677
				5407 1007058.92110673
				5408 1007060.2022867
				5409 1007061.48346666
				5410 1007062.76464663
				5411 1007064.04582659
				5412 1007065.32700656
				5413 1007066.60818653
				5414 1007067.88936649
				5415 1007069.17054646
				5416 1007070.45172642
				5417 1007071.73290639
				5418 1007073.01408635
				5419 1007074.29526632
				5420 1007075.57644628
				5421 1007076.85762625
				5422 1007078.13880621
				5423 1007079.41998618
				5424 1007080.70116615
				5425 1007081.98234611
				5426 1007083.26352608
				5427 1007084.54470604
				5428 1007085.82588601
				5429 1007087.10706597
				5430 1007088.38824594
				5431 1007089.6694259
				5432 1007090.95060587
				5433 1007092.23178583
				5434 1007093.5129658
				5435 1007094.79414577
				5436 1007096.07532573
				5437 1007097.3565057
				5438 1007098.63768566
				5439 1007099.91886563
				5440 1007101.20004559
				5441 1007102.48122556
				5442 1007103.76240552
				5443 1007105.04358549
				5444 1007106.32476545
				5445 1007107.60594542
				5446 1007108.88712539
				5447 1007110.16830535
				5448 1007111.44948532
				5449 1007112.73066528
				5450 1007114.01184525
				5451 1007115.29302521
				5452 1007116.57420518
				5453 1007117.85538514
				5454 1007119.13656511
				5455 1007120.41774507
				5456 1007121.69892504
				5457 1007122.98010501
				5458 1007124.26128497
				5459 1007125.54246494
				5460 1007126.8236449
				5461 1007128.10482487
				5462 1007129.38600483
				5463 1007130.6671848
				5464 1007131.94836476
				5465 1007133.22954473
				5466 1007134.51072469
				5467 1007135.79190466
				5468 1007137.07308463
				5469 1007138.35426459
				5470 1007139.63544456
				5471 1007140.91662452
				5472 1007142.19780449
				5473 1007143.47898445
				5474 1007144.76016442
				5475 1007146.04134438
				5476 1007147.32252435
				5477 1007148.60370431
				5478 1007149.88488428
				5479 1007151.16606425
				5480 1007152.44724421
				5481 1007153.72842418
				5482 1007155.00960414
				5483 1007156.29078411
				5484 1007157.57196407
				5485 1007158.85314404
				5486 1007160.134324
				5487 1007161.41550397
				5488 1007162.69668393
				5489 1007163.9778639
				5490 1007165.25904387
				5491 1007166.54022383
				5492 1007167.8214038
				5493 1007169.10258376
				5494 1007170.38376373
				5495 1007171.66494369
				5496 1007172.94612366
				5497 1007174.22730362
				5498 1007175.50848359
				5499 1007176.78966355
				5500 1007178.07084352
				5501 1007179.35202349
				5502 1007180.63320345
				5503 1007181.91438342
				5504 1007183.19556338
				5505 1007184.47674335
				5506 1007185.75792331
				5507 1007187.03910328
				5508 1007188.32028324
				5509 1007189.60146321
				5510 1007190.88264317
				5511 1007192.16382314
				5512 1007193.44500311
				5513 1007194.72618307
				5514 1007196.00736304
				5515 1007197.288543
				5516 1007198.56972297
				5517 1007199.85090293
				5518 1007201.1320829
				5519 1007202.41326286
				5520 1007203.69444283
				5521 1007204.97562279
				5522 1007206.25680276
				5523 1007207.53798273
				5524 1007208.81916269
				5525 1007210.10034266
				5526 1007211.38152262
				5527 1007212.66270259
				5528 1007213.94388255
				5529 1007215.22506252
				5530 1007216.50624248
				5531 1007217.78742245
				5532 1007219.06860241
				5533 1007220.34978238
				5534 1007221.63096235
				5535 1007222.91214231
				5536 1007224.19332228
				5537 1007225.47450224
				5538 1007226.75568221
				5539 1007228.03686217
				5540 1007229.31804214
				5541 1007230.5992221
				5542 1007231.88040207
				5543 1007233.16158203
				5544 1007234.442762
				5545 1007235.72394197
				5546 1007237.00512193
				5547 1007238.2863019
				5548 1007239.56748186
				5549 1007240.84866183
				5550 1007242.12984179
				5551 1007243.41102176
				5552 1007244.69220172
				5553 1007245.97338169
				5554 1007247.25456165
				5555 1007248.53574162
				5556 1007249.81692159
				5557 1007251.09810155
				5558 1007252.37928152
				5559 1007253.66046148
				5560 1007254.94164145
				5561 1007256.22282141
				5562 1007257.50400138
				5563 1007258.78518134
				5564 1007260.06636131
				5565 1007261.34754127
				5566 1007262.62872124
				5567 1007263.90990121
				5568 1007265.19108117
				5569 1007266.47226114
				5570 1007267.7534411
				5571 1007269.03462107
				5572 1007270.31580103
				5573 1007271.596981
				5574 1007272.87816096
				5575 1007274.15934093
				5576 1007275.44052089
				5577 1007276.72170086
				5578 1007278.00288083
				5579 1007279.28406079
				5580 1007280.56524076
				5581 1007281.84642072
				5582 1007283.12760069
				5583 1007284.40878065
				5584 1007285.68996062
				5585 1007286.97114058
				5586 1007288.25232055
				5587 1007289.53350051
				5588 1007290.81468048
				5589 1007292.09586045
				5590 1007293.37704041
				5591 1007294.65822038
				5592 1007295.93940034
				5593 1007297.22058031
				5594 1007298.50176027
				5595 1007299.78294024
				5596 1007301.0641202
				5597 1007302.34530017
				5598 1007303.62648013
				5599 1007304.9076601
				5600 1007306.18884007
				5601 1007307.47002003
				5602 1007308.7512
				5603 1007310.03237996
				5604 1007311.31355993
				5605 1007312.59473989
				5606 1007313.87591986
				5607 1007315.15709982
				5608 1007316.43827979
				5609 1007317.71945975
				5610 1007319.00063972
				5611 1007320.28181969
				5612 1007321.56299965
				5613 1007322.84417962
				5614 1007324.12535958
				5615 1007325.40653955
				5616 1007326.68771951
				5617 1007327.96889948
				5618 1007329.25007944
				5619 1007330.53125941
				5620 1007331.81243937
				5621 1007333.09361934
				5622 1007334.37479931
				5623 1007335.65597927
				5624 1007336.93715924
				5625 1007338.2183392
				5626 1007339.49951917
				5627 1007340.78069913
				5628 1007342.0618791
				5629 1007343.34305906
				5630 1007344.62423903
				5631 1007345.90541899
				5632 1007347.18659896
				5633 1007348.46777893
				5634 1007349.74895889
				5635 1007351.03013886
				5636 1007352.31131882
				5637 1007353.59249879
				5638 1007354.87367875
				5639 1007356.15485872
				5640 1007357.43603868
				5641 1007358.71721865
				5642 1007359.99839861
				5643 1007361.27957858
				5644 1007362.56075855
				5645 1007363.84193851
				5646 1007365.12311848
				5647 1007366.40429844
				5648 1007367.68547841
				5649 1007368.96665837
				5650 1007370.24783834
				5651 1007371.5290183
				5652 1007372.81019827
				5653 1007374.09137823
				5654 1007375.3725582
				5655 1007376.65373817
				5656 1007377.93491813
				5657 1007379.2160981
				5658 1007380.49727806
				5659 1007381.77845803
				5660 1007383.05963799
				5661 1007384.34081796
				5662 1007385.62199792
				5663 1007386.90317789
				5664 1007388.18435785
				5665 1007389.46553782
				5666 1007390.74671779
				5667 1007392.02789775
				5668 1007393.30907772
				5669 1007394.59025768
				5670 1007395.87143765
				5671 1007397.15261761
				5672 1007398.43379758
				5673 1007399.71497754
				5674 1007400.99615751
				5675 1007402.27733747
				5676 1007403.55851744
				5677 1007404.83969741
				5678 1007406.12087737
				5679 1007407.40205734
				5680 1007408.6832373
				5681 1007409.96441727
				5682 1007411.24559723
				5683 1007412.5267772
				5684 1007413.80795716
				5685 1007415.08913713
				5686 1007416.37031709
				5687 1007417.65149706
				5688 1007418.93267703
				5689 1007420.21385699
				5690 1007421.49503696
				5691 1007422.77621692
				5692 1007424.05739689
				5693 1007425.33857685
				5694 1007426.61975682
				5695 1007427.90093678
				5696 1007429.18211675
				5697 1007430.46329671
				5698 1007431.74447668
				5699 1007433.02565665
				5700 1007434.30683661
				5701 1007435.58801658
				5702 1007436.86919654
				5703 1007438.15037651
				5704 1007439.43155647
				5705 1007440.71273644
				5706 1007441.9939164
				5707 1007443.27509637
				5708 1007444.55627633
				5709 1007445.8374563
				5710 1007447.11863627
				5711 1007448.39981623
				5712 1007449.6809962
				5713 1007450.96217616
				5714 1007452.24335613
				5715 1007453.52453609
				5716 1007454.80571606
				5717 1007456.08689602
				5718 1007457.36807599
				5719 1007458.64925595
				5720 1007459.93043592
				5721 1007461.21161589
				5722 1007462.49279585
				5723 1007463.77397582
				5724 1007465.05515578
				5725 1007466.33633575
				5726 1007467.61751571
				5727 1007468.89869568
				5728 1007470.17987564
				5729 1007471.46105561
				5730 1007472.74223557
				5731 1007474.02341554
				5732 1007475.30459551
				5733 1007476.58577547
				5734 1007477.86695544
				5735 1007479.1481354
				5736 1007480.42931537
				5737 1007481.71049533
				5738 1007482.9916753
				5739 1007484.27285526
				5740 1007485.55403523
				5741 1007486.83521519
				5742 1007488.11639516
				5743 1007489.39757513
				5744 1007490.67875509
				5745 1007491.95993506
				5746 1007493.24111502
				5747 1007494.52229499
				5748 1007495.80347495
				5749 1007497.08465492
				5750 1007498.36583488
				5751 1007499.64701485
				5752 1007500.92819481
				5753 1007502.20937478
				5754 1007503.49055475
				5755 1007504.77173471
				5756 1007506.05291468
				5757 1007507.33409464
				5758 1007508.61527461
				5759 1007509.89645457
				5760 1007511.17763454
				5761 1007512.4588145
				5762 1007513.73999447
				5763 1007515.02117443
				5764 1007516.3023544
				5765 1007517.58353437
				5766 1007518.86471433
				5767 1007520.1458943
				5768 1007521.42707426
				5769 1007522.70825423
				5770 1007523.98943419
				5771 1007525.27061416
				5772 1007526.55179412
				5773 1007527.83297409
				5774 1007529.11415405
				5775 1007530.39533402
				5776 1007531.67651399
				5777 1007532.95769395
				5778 1007534.23887392
				5779 1007535.52005388
				5780 1007536.80123385
				5781 1007538.08241381
				5782 1007539.36359378
				5783 1007540.64477374
				5784 1007541.92595371
				5785 1007543.20713367
				5786 1007544.48831364
				5787 1007545.76949361
				5788 1007547.05067357
				5789 1007548.33185354
				5790 1007549.6130335
				5791 1007550.89421347
				5792 1007552.17539343
				5793 1007553.4565734
				5794 1007554.73775336
				5795 1007556.01893333
				5796 1007557.30011329
				5797 1007558.58129326
				5798 1007559.86247323
				5799 1007561.14365319
				5800 1007562.42483316
				5801 1007563.70601312
				5802 1007564.98719309
				5803 1007566.26837305
				5804 1007567.54955302
				5805 1007568.83073298
				5806 1007570.11191295
				5807 1007571.39309291
				5808 1007572.67427288
				5809 1007573.95545285
				5810 1007575.23663281
				5811 1007576.51781278
				5812 1007577.79899274
				5813 1007579.08017271
				5814 1007580.36135267
				5815 1007581.64253264
				5816 1007582.9237126
				5817 1007584.20489257
				5818 1007585.48607253
				5819 1007586.7672525
				5820 1007588.04843247
				5821 1007589.32961243
				5822 1007590.6107924
				5823 1007591.89197236
				5824 1007593.17315233
				5825 1007594.45433229
				5826 1007595.73551226
				5827 1007597.01669222
				5828 1007598.29787219
				5829 1007599.57905215
				5830 1007600.86023212
				5831 1007602.14141209
				5832 1007603.42259205
				5833 1007604.70377202
				5834 1007605.98495198
				5835 1007607.26613195
				5836 1007608.54731191
				5837 1007609.82849188
				5838 1007611.10967184
				5839 1007612.39085181
				5840 1007613.67203177
				5841 1007614.95321174
				5842 1007616.23439171
				5843 1007617.51557167
				5844 1007618.79675164
				5845 1007620.0779316
				5846 1007621.35911157
				5847 1007622.64029153
				5848 1007623.9214715
				5849 1007625.20265146
				5850 1007626.48383143
				5851 1007627.76501139
				5852 1007629.04619136
				5853 1007630.32737133
				5854 1007631.60855129
				5855 1007632.88973126
				5856 1007634.17091122
				5857 1007635.45209119
				5858 1007636.73327115
				5859 1007638.01445112
				5860 1007639.29563108
				5861 1007640.57681105
				5862 1007641.85799101
				5863 1007643.13917098
				5864 1007644.42035095
				5865 1007645.70153091
				5866 1007646.98271088
				5867 1007648.26389084
				5868 1007649.54507081
				5869 1007650.82625077
				5870 1007652.10743074
				5871 1007653.3886107
				5872 1007654.66979067
				5873 1007655.95097063
				5874 1007657.2321506
				5875 1007658.51333057
				5876 1007659.79451053
				5877 1007661.0756905
				5878 1007662.35687046
				5879 1007663.63805043
				5880 1007664.91923039
				5881 1007666.20041036
				5882 1007667.48159032
				5883 1007668.76277029
				5884 1007670.04395025
				5885 1007671.32513022
				5886 1007672.60631019
				5887 1007673.88749015
				5888 1007675.16867012
				5889 1007676.44985008
				5890 1007677.73103005
				5891 1007679.01221001
				5892 1007680.29338998
				5893 1007681.57456994
				5894 1007682.85574991
				5895 1007684.13692987
				5896 1007685.41810984
				5897 1007686.69928981
				5898 1007687.98046977
				5899 1007689.26164974
				5900 1007690.5428297
				5901 1007691.82400967
				5902 1007693.10518963
				5903 1007694.3863696
				5904 1007695.66754956
				5905 1007696.94872953
				5906 1007698.22990949
				5907 1007699.51108946
				5908 1007700.79226943
				5909 1007702.07344939
				5910 1007703.35462936
				5911 1007704.63580932
				5912 1007705.91698929
				5913 1007707.19816925
				5914 1007708.47934922
				5915 1007709.76052918
				5916 1007711.04170915
				5917 1007712.32288911
				5918 1007713.60406908
				5919 1007714.88524905
				5920 1007716.16642901
				5921 1007717.44760898
				5922 1007718.72878894
				5923 1007720.00996891
				5924 1007721.29114887
				5925 1007722.57232884
				5926 1007723.8535088
				5927 1007725.13468877
				5928 1007726.41586873
				5929 1007727.6970487
				5930 1007728.97822867
				5931 1007730.25940863
				5932 1007731.5405886
				5933 1007732.82176856
				5934 1007734.10294853
				5935 1007735.38412849
				5936 1007736.66530846
				5937 1007737.94648842
				5938 1007739.22766839
				5939 1007740.50884835
				5940 1007741.79002832
				5941 1007743.07120829
				5942 1007744.35238825
				5943 1007745.63356822
				5944 1007746.91474818
				5945 1007748.19592815
				5946 1007749.47710811
				5947 1007750.75828808
				5948 1007752.03946804
				5949 1007753.32064801
				5950 1007754.60182797
				5951 1007755.88300794
				5952 1007757.16418791
				5953 1007758.44536787
				5954 1007759.72654784
				5955 1007761.0077278
				5956 1007762.28890777
				5957 1007763.57008773
				5958 1007764.8512677
				5959 1007766.13244766
				5960 1007767.41362763
				5961 1007768.69480759
				5962 1007769.97598756
				5963 1007771.25716753
				5964 1007772.53834749
				5965 1007773.81952746
				5966 1007775.10070742
				5967 1007776.38188739
				5968 1007777.66306735
				5969 1007778.94424732
				5970 1007780.22542728
				5971 1007781.50660725
				5972 1007782.78778721
				5973 1007784.06896718
				5974 1007785.35014715
				5975 1007786.63132711
				5976 1007787.91250708
				5977 1007789.19368704
				5978 1007790.47486701
				5979 1007791.75604697
				5980 1007793.03722694
				5981 1007794.3184069
				5982 1007795.59958687
				5983 1007796.88076683
				5984 1007798.1619468
				5985 1007799.44312677
				5986 1007800.72430673
				5987 1007802.0054867
				5988 1007803.28666666
				5989 1007804.56784663
				5990 1007805.84902659
				5991 1007807.13020656
				5992 1007808.41138652
				5993 1007809.69256649
				5994 1007810.97374645
				5995 1007812.25492642
				5996 1007813.53610639
				5997 1007814.81728635
				5998 1007816.09846632
				5999 1007817.37964628
				6000 1007818.66082625
				6001 1007819.94200621
				6002 1007821.22318618
				6003 1007822.50436614
				6004 1007823.78554611
				6005 1007825.06672607
				6006 1007826.34790604
				6007 1007827.62908601
				6008 1007828.91026597
				6009 1007830.19144594
				6010 1007831.4726259
				6011 1007832.75380587
				6012 1007834.03498583
				6013 1007835.3161658
				6014 1007836.59734576
				6015 1007837.87852573
				6016 1007839.15970569
				6017 1007840.44088566
				6018 1007841.72206563
				6019 1007843.00324559
				6020 1007844.28442556
				6021 1007845.56560552
				6022 1007846.84678549
				6023 1007848.12796545
				6024 1007849.40914542
				6025 1007850.69032538
				6026 1007851.97150535
				6027 1007853.25268531
				6028 1007854.53386528
				6029 1007855.81504525
				6030 1007857.09622521
				6031 1007858.37740518
				6032 1007859.65858514
				6033 1007860.93976511
				6034 1007862.22094507
				6035 1007863.50212504
				6036 1007864.783305
				6037 1007866.06448497
				6038 1007867.34566493
				6039 1007868.6268449
				6040 1007869.90802487
				6041 1007871.18920483
				6042 1007872.4703848
				6043 1007873.75156476
				6044 1007875.03274473
				6045 1007876.31392469
				6046 1007877.59510466
				6047 1007878.87628462
				6048 1007880.15746459
				6049 1007881.43864455
				6050 1007882.71982452
				6051 1007884.00100449
				6052 1007885.28218445
				6053 1007886.56336442
				6054 1007887.84454438
				6055 1007889.12572435
				6056 1007890.40690431
				6057 1007891.68808428
				6058 1007892.96926424
				6059 1007894.25044421
				6060 1007895.53162417
				6061 1007896.81280414
				6062 1007898.09398411
				6063 1007899.37516407
				6064 1007900.65634404
				6065 1007901.937524
				6066 1007903.21870397
				6067 1007904.49988393
				6068 1007905.7810639
				6069 1007907.06224386
				6070 1007908.34342383
				6071 1007909.62460379
				6072 1007910.90578376
				6073 1007912.18696373
				6074 1007913.46814369
				6075 1007914.74932366
				6076 1007916.03050362
				6077 1007917.31168359
				6078 1007918.59286355
				6079 1007919.87404352
				6080 1007921.15522348
				6081 1007922.43640345
				6082 1007923.71758341
				6083 1007924.99876338
				6084 1007926.27994335
				6085 1007927.56112331
				6086 1007928.84230328
				6087 1007930.12348324
				6088 1007931.40466321
				6089 1007932.68584317
				6090 1007933.96702314
				6091 1007935.2482031
				6092 1007936.52938307
				6093 1007937.81056303
				6094 1007939.091743
				6095 1007940.37292297
				6096 1007941.65410293
				6097 1007942.9352829
				6098 1007944.21646286
				6099 1007945.49764283
				6100 1007946.77882279
				6101 1007948.06000276
				6102 1007949.34118272
				6103 1007950.62236269
				6104 1007951.90354265
				6105 1007953.18472262
				6106 1007954.46590259
				6107 1007955.74708255
				6108 1007957.02826252
				6109 1007958.30944248
				6110 1007959.59062245
				6111 1007960.87180241
				6112 1007962.15298238
				6113 1007963.43416234
				6114 1007964.71534231
				6115 1007965.99652227
				6116 1007967.27770224
				6117 1007968.55888221
				6118 1007969.84006217
				6119 1007971.12124214
				6120 1007972.4024221
				6121 1007973.68360207
				6122 1007974.96478203
				6123 1007976.245962
				6124 1007977.52714196
				6125 1007978.80832193
				6126 1007980.08950189
				6127 1007981.37068186
				6128 1007982.65186183
				6129 1007983.93304179
				6130 1007985.21422176
				6131 1007986.49540172
				6132 1007987.77658169
				6133 1007989.05776165
				6134 1007990.33894162
				6135 1007991.62012158
				6136 1007992.90130155
				6137 1007994.18248151
				6138 1007995.46366148
				6139 1007996.74484145
				6140 1007998.02602141
				6141 1007999.30720138
				6142 1008000.58838134
				6143 1008001.86956131
				6144 1008003.15074127
				6145 1008004.43192124
				6146 1008005.7131012
				6147 1008006.99428117
				6148 1008008.27546113
				6149 1008009.5566411
				6150 1008010.83782106
				6151 1008012.11900103
				6152 1008013.400181
				6153 1008014.68136096
				6154 1008015.96254093
				6155 1008017.24372089
				6156 1008018.52490086
				6157 1008019.80608082
				6158 1008021.08726079
				6159 1008022.36844075
				6160 1008023.64962072
				6161 1008024.93080069
				6162 1008026.21198065
				6163 1008027.49316062
				6164 1008028.77434058
				6165 1008030.05552055
				6166 1008031.33670051
				6167 1008032.61788048
				6168 1008033.89906044
				6169 1008035.18024041
				6170 1008036.46142037
				6171 1008037.74260034
				6172 1008039.02378031
				6173 1008040.30496027
				6174 1008041.58614024
				6175 1008042.8673202
				6176 1008044.14850017
				6177 1008045.42968013
				6178 1008046.7108601
				6179 1008047.99204006
				6180 1008049.27322003
				6181 1008050.55439999
				6182 1008051.83557996
				6183 1008053.11675993
				6184 1008054.39793989
				6185 1008055.67911986
				6186 1008056.96029982
				6187 1008058.24147979
				6188 1008059.52265975
				6189 1008060.80383972
				6190 1008062.08501968
				6191 1008063.36619965
				6192 1008064.64737961
				6193 1008065.92855958
				6194 1008067.20973955
				6195 1008068.49091951
				6196 1008069.77209948
				6197 1008071.05327944
				6198 1008072.33445941
				6199 1008073.61563937
				6200 1008074.89681934
				6201 1008076.1779993
				6202 1008077.45917927
				6203 1008078.74035923
				6204 1008080.0215392
				6205 1008081.30271916
				6206 1008082.58389913
				6207 1008083.8650791
				6208 1008085.14625906
				6209 1008086.42743903
				6210 1008087.70861899
				6211 1008088.98979896
				6212 1008090.27097892
				6213 1008091.55215889
				6214 1008092.83333885
				6215 1008094.11451882
				6216 1008095.39569878
				6217 1008096.67687875
				6218 1008097.95805872
				6219 1008099.23923868
				6220 1008100.52041865
				6221 1008101.80159861
				6222 1008103.08277858
				6223 1008104.36395854
				6224 1008105.64513851
				6225 1008106.92631847
				6226 1008108.20749844
				6227 1008109.4886784
				6228 1008110.76985837
				6229 1008112.05103834
				6230 1008113.3322183
				6231 1008114.61339827
				6232 1008115.89457823
				6233 1008117.1757582
				6234 1008118.45693816
				6235 1008119.73811813
				6236 1008121.01929809
				6237 1008122.30047806
				6238 1008123.58165802
				6239 1008124.86283799
				6240 1008126.14401796
				6241 1008127.42519792
				6242 1008128.70637789
				6243 1008129.98755785
				6244 1008131.26873782
				6245 1008132.54991778
				6246 1008133.83109775
				6247 1008135.11227771
				6248 1008136.39345768
				6249 1008137.67463765
				6250 1008138.95581761
				6251 1008140.23699758
				6252 1008141.51817754
				6253 1008142.79935751
				6254 1008144.08053747
				6255 1008145.36171744
				6256 1008146.6428974
				6257 1008147.92407737
				6258 1008149.20525733
				6259 1008150.4864373
				6260 1008151.76761726
				6261 1008153.04879723
				6262 1008154.3299772
				6263 1008155.61115716
				6264 1008156.89233713
				6265 1008158.17351709
				6266 1008159.45469706
				6267 1008160.73587702
				6268 1008162.01705699
				6269 1008163.29823695
				6270 1008164.57941692
				6271 1008165.86059688
				6272 1008167.14177685
				6273 1008168.42295682
				6274 1008169.70413678
				6275 1008170.98531675
				6276 1008172.26649671
				6277 1008173.54767668
				6278 1008174.82885664
				6279 1008176.11003661
				6280 1008177.39121657
				6281 1008178.67239654
				6282 1008179.9535765
				6283 1008181.23475647
				6284 1008182.51593644
				6285 1008183.7971164
				6286 1008185.07829637
				6287 1008186.35947633
				6288 1008187.6406563
				6289 1008188.92183626
				6290 1008190.20301623
				6291 1008191.48419619
				6292 1008192.76537616
				6293 1008194.04655612
				6294 1008195.32773609
				6295 1008196.60891606
				6296 1008197.89009602
				6297 1008199.17127599
				6298 1008200.45245595
				6299 1008201.73363592
				6300 1008203.01481588
				6301 1008204.29599585
				6302 1008205.57717581
				6303 1008206.85835578
				6304 1008208.13953574
				6305 1008209.42071571
				6306 1008210.70189568
				6307 1008211.98307564
				6308 1008213.26425561
				6309 1008214.54543557
				6310 1008215.82661554
				6311 1008217.1077955
				6312 1008218.38897547
				6313 1008219.67015543
				6314 1008220.9513354
				6315 1008222.23251536
				6316 1008223.51369533
				6317 1008224.7948753
				6318 1008226.07605526
				6319 1008227.35723523
				6320 1008228.63841519
				6321 1008229.91959516
				6322 1008231.20077512
				6323 1008232.48195509
				6324 1008233.76313505
				6325 1008235.04431502
				6326 1008236.32549498
				6327 1008237.60667495
				6328 1008238.88785492
				6329 1008240.16903488
				6330 1008241.45021485
				6331 1008242.73139481
				6332 1008244.01257478
				6333 1008245.29375474
				6334 1008246.57493471
				6335 1008247.85611467
				6336 1008249.13729464
				6337 1008250.4184746
				6338 1008251.69965457
				6339 1008252.98083454
				6340 1008254.2620145
				6341 1008255.54319447
				6342 1008256.82437443
				6343 1008258.1055544
				6344 1008259.38673436
				6345 1008260.66791433
				6346 1008261.94909429
				6347 1008263.23027426
				6348 1008264.51145422
				6349 1008265.79263419
				6350 1008267.07381416
				6351 1008268.35499412
				6352 1008269.63617409
				6353 1008270.91735405
				6354 1008272.19853402
				6355 1008273.47971398
				6356 1008274.76089395
				6357 1008276.04207391
				6358 1008277.32325388
				6359 1008278.60443384
				6360 1008279.88561381
				6361 1008281.16679378
				6362 1008282.44797374
				6363 1008283.72915371
				6364 1008285.01033367
				6365 1008286.29151364
				6366 1008287.5726936
				6367 1008288.85387357
				6368 1008290.13505353
				6369 1008291.4162335
				6370 1008292.69741346
				6371 1008293.97859343
				6372 1008295.2597734
				6373 1008296.54095336
				6374 1008297.82213333
				6375 1008299.10331329
				6376 1008300.38449326
				6377 1008301.66567322
				6378 1008302.94685319
				6379 1008304.22803315
				6380 1008305.50921312
				6381 1008306.79039308
				6382 1008308.07157305
				6383 1008309.35275302
				6384 1008310.63393298
				6385 1008311.91511295
				6386 1008313.19629291
				6387 1008314.47747288
				6388 1008315.75865284
				6389 1008317.03983281
				6390 1008318.32101277
				6391 1008319.60219274
				6392 1008320.8833727
				6393 1008322.16455267
				6394 1008323.44573264
				6395 1008324.7269126
				6396 1008326.00809257
				6397 1008327.28927253
				6398 1008328.5704525
				6399 1008329.85163246
				6400 1008331.13281243
				6401 1008332.41399239
				6402 1008333.69517236
				6403 1008334.97635232
				6404 1008336.25753229
				6405 1008337.53871226
				6406 1008338.81989222
				6407 1008340.10107219
				6408 1008341.38225215
				6409 1008342.66343212
				6410 1008343.94461208
				6411 1008345.22579205
				6412 1008346.50697201
				6413 1008347.78815198
				6414 1008349.06933194
				6415 1008350.35051191
				6416 1008351.63169188
				6417 1008352.91287184
				6418 1008354.19405181
				6419 1008355.47523177
				6420 1008356.75641174
				6421 1008358.0375917
				6422 1008359.31877167
				6423 1008360.59995163
				6424 1008361.8811316
				6425 1008363.16231156
				6426 1008364.44349153
				6427 1008365.7246715
				6428 1008367.00585146
				6429 1008368.28703143
				6430 1008369.56821139
				6431 1008370.84939136
				6432 1008372.13057132
				6433 1008373.41175129
				6434 1008374.69293125
				6435 1008375.97411122
				6436 1008377.25529118
				6437 1008378.53647115
				6438 1008379.81765112
				6439 1008381.09883108
				6440 1008382.38001105
				6441 1008383.66119101
				6442 1008384.94237098
				6443 1008386.22355094
				6444 1008387.50473091
				6445 1008388.78591087
				6446 1008390.06709084
				6447 1008391.3482708
				6448 1008392.62945077
				6449 1008393.91063074
				6450 1008395.1918107
				6451 1008396.47299067
				6452 1008397.75417063
				6453 1008399.0353506
				6454 1008400.31653056
				6455 1008401.59771053
				6456 1008402.87889049
				6457 1008404.16007046
				6458 1008405.44125042
				6459 1008406.72243039
				6460 1008408.00361036
				6461 1008409.28479032
				6462 1008410.56597029
				6463 1008411.84715025
				6464 1008413.12833022
				6465 1008414.40951018
				6466 1008415.69069015
				6467 1008416.97187011
				6468 1008418.25305008
				6469 1008419.53423004
				6470 1008420.81541001
				6471 1008422.09658998
				6472 1008423.37776994
				6473 1008424.65894991
				6474 1008425.94012987
				6475 1008427.22130984
				6476 1008428.5024898
				6477 1008429.78366977
				6478 1008431.06484973
				6479 1008432.3460297
				6480 1008433.62720966
				6481 1008434.90838963
				6482 1008436.1895696
				6483 1008437.47074956
				6484 1008438.75192953
				6485 1008440.03310949
				6486 1008441.31428946
				6487 1008442.59546942
				6488 1008443.87664939
				6489 1008445.15782935
				6490 1008446.43900932
				6491 1008447.72018928
				6492 1008449.00136925
				6493 1008450.28254922
				6494 1008451.56372918
				6495 1008452.84490915
				6496 1008454.12608911
				6497 1008455.40726908
				6498 1008456.68844904
				6499 1008457.96962901
				6500 1008459.25080897
				6501 1008460.53198894
				6502 1008461.8131689
				6503 1008463.09434887
				6504 1008464.37552884
				6505 1008465.6567088
				6506 1008466.93788877
				6507 1008468.21906873
				6508 1008469.5002487
				6509 1008470.78142866
				6510 1008472.06260863
				6511 1008473.34378859
				6512 1008474.62496856
				6513 1008475.90614852
				6514 1008477.18732849
				6515 1008478.46850846
				6516 1008479.74968842
				6517 1008481.03086839
				6518 1008482.31204835
				6519 1008483.59322832
				6520 1008484.87440828
				6521 1008486.15558825
				6522 1008487.43676821
				6523 1008488.71794818
				6524 1008489.99912814
				6525 1008491.28030811
				6526 1008492.56148808
				6527 1008493.84266804
				6528 1008495.12384801
				6529 1008496.40502797
				6530 1008497.68620794
				6531 1008498.9673879
				6532 1008500.24856787
				6533 1008501.52974783
				6534 1008502.8109278
				6535 1008504.09210776
				6536 1008505.37328773
				6537 1008506.6544677
				6538 1008507.93564766
				6539 1008509.21682763
				6540 1008510.49800759
				6541 1008511.77918756
				6542 1008513.06036752
				6543 1008514.34154749
				6544 1008515.62272745
				6545 1008516.90390742
				6546 1008518.18508738
				6547 1008519.46626735
				6548 1008520.74744732
				6549 1008522.02862728
				6550 1008523.30980725
				6551 1008524.59098721
				6552 1008525.87216718
				6553 1008527.15334714
				6554 1008528.43452711
				6555 1008529.71570707
				6556 1008530.99688704
				6557 1008532.278067
				6558 1008533.55924697
				6559 1008534.84042694
				6560 1008536.1216069
				6561 1008537.40278687
				6562 1008538.68396683
				6563 1008539.9651468
				6564 1008541.24632676
				6565 1008542.52750673
				6566 1008543.80868669
				6567 1008545.08986666
				6568 1008546.37104662
				6569 1008547.65222659
				6570 1008548.93340656
				6571 1008550.21458652
				6572 1008551.49576649
				6573 1008552.77694645
				6574 1008554.05812642
				6575 1008555.33930638
				6576 1008556.62048635
				6577 1008557.90166631
				6578 1008559.18284628
				6579 1008560.46402624
				6580 1008561.74520621
				6581 1008563.02638618
				6582 1008564.30756614
				6583 1008565.58874611
				6584 1008566.86992607
				6585 1008568.15110604
				6586 1008569.432286
				6587 1008570.71346597
				6588 1008571.99464593
				6589 1008573.2758259
				6590 1008574.55700586
				6591 1008575.83818583
				6592 1008577.1193658
				6593 1008578.40054576
				6594 1008579.68172573
				6595 1008580.96290569
				6596 1008582.24408566
				6597 1008583.52526562
				6598 1008584.80644559
				6599 1008586.08762555
				6600 1008587.36880552
				6601 1008588.64998548
				6602 1008589.93116545
				6603 1008591.21234542
				6604 1008592.49352538
				6605 1008593.77470535
				6606 1008595.05588531
				6607 1008596.33706528
				6608 1008597.61824524
				6609 1008598.89942521
				6610 1008600.18060517
				6611 1008601.46178514
				6612 1008602.7429651
				6613 1008604.02414507
				6614 1008605.30532504
				6615 1008606.586505
				6616 1008607.86768497
				6617 1008609.14886493
				6618 1008610.4300449
				6619 1008611.71122486
				6620 1008612.99240483
				6621 1008614.27358479
				6622 1008615.55476476
				6623 1008616.83594472
				6624 1008618.11712469
				6625 1008619.39830466
				6626 1008620.67948462
				6627 1008621.96066459
				6628 1008623.24184455
				6629 1008624.52302452
				6630 1008625.80420448
				6631 1008627.08538445
				6632 1008628.36656441
				6633 1008629.64774438
				6634 1008630.92892434
				6635 1008632.21010431
				6636 1008633.49128428
				6637 1008634.77246424
				6638 1008636.05364421
				6639 1008637.33482417
				6640 1008638.61600414
				6641 1008639.8971841
				6642 1008641.17836407
				6643 1008642.45954403
				6644 1008643.740724
				6645 1008645.02190396
				6646 1008646.30308393
				6647 1008647.5842639
				6648 1008648.86544386
				6649 1008650.14662383
				6650 1008651.42780379
				6651 1008652.70898376
				6652 1008653.99016372
				6653 1008655.27134369
				6654 1008656.55252365
				6655 1008657.83370362
				6656 1008659.11488358
				6657 1008660.39606355
				6658 1008661.67724352
				6659 1008662.95842348
				6660 1008664.23960345
				6661 1008665.52078341
				6662 1008666.80196338
				6663 1008668.08314334
				6664 1008669.36432331
				6665 1008670.64550327
				6666 1008671.92668324
				6667 1008673.2078632
				6668 1008674.48904317
				6669 1008675.77022314
				6670 1008677.0514031
				6671 1008678.33258307
				6672 1008679.61376303
				6673 1008680.894943
				6674 1008682.17612296
				6675 1008683.45730293
				6676 1008684.73848289
				6677 1008686.01966286
				6678 1008687.30084282
				6679 1008688.58202279
				6680 1008689.86320276
				6681 1008691.14438272
				6682 1008692.42556269
				6683 1008693.70674265
				6684 1008694.98792262
				6685 1008696.26910258
				6686 1008697.55028255
				6687 1008698.83146251
				6688 1008700.11264248
				6689 1008701.39382244
				6690 1008702.67500241
				6691 1008703.95618238
				6692 1008705.23736234
				6693 1008706.51854231
				6694 1008707.79972227
				6695 1008709.08090224
				6696 1008710.3620822
				6697 1008711.64326217
				6698 1008712.92444213
				6699 1008714.2056221
				6700 1008715.48680206
				6701 1008716.76798203
				6702 1008718.049162
				6703 1008719.33034196
				6704 1008720.61152193
				6705 1008721.89270189
				6706 1008723.17388186
				6707 1008724.45506182
				6708 1008725.73624179
				6709 1008727.01742175
				6710 1008728.29860172
				6711 1008729.57978168
				6712 1008730.86096165
				6713 1008732.14214162
				6714 1008733.42332158
				6715 1008734.70450155
				6716 1008735.98568151
				6717 1008737.26686148
				6718 1008738.54804144
				6719 1008739.82922141
				6720 1008741.11040137
				6721 1008742.39158134
				6722 1008743.6727613
				6723 1008744.95394127
				6724 1008746.23512124
				6725 1008747.5163012
				6726 1008748.79748117
				6727 1008750.07866113
				6728 1008751.3598411
				6729 1008752.64102106
				6730 1008753.92220103
				6731 1008755.20338099
				6732 1008756.48456096
				6733 1008757.76574092
				6734 1008759.04692089
				6735 1008760.32810086
				6736 1008761.60928082
				6737 1008762.89046079
				6738 1008764.17164075
				6739 1008765.45282072
				6740 1008766.73400068
				6741 1008768.01518065
				6742 1008769.29636061
				6743 1008770.57754058
				6744 1008771.85872054
				6745 1008773.13990051
				6746 1008774.42108048
				6747 1008775.70226044
				6748 1008776.98344041
				6749 1008778.26462037
				6750 1008779.54580034
				6751 1008780.8269803
				6752 1008782.10816027
				6753 1008783.38934023
				6754 1008784.6705202
				6755 1008785.95170016
				6756 1008787.23288013
				6757 1008788.5140601
				6758 1008789.79524006
				6759 1008791.07642003
				6760 1008792.35759999
				6761 1008793.63877996
				6762 1008794.91995992
				6763 1008796.20113989
				6764 1008797.48231985
				6765 1008798.76349982
				6766 1008800.04467978
				6767 1008801.32585975
				6768 1008802.60703972
				6769 1008803.88821968
				6770 1008805.16939965
				6771 1008806.45057961
				6772 1008807.73175958
				6773 1008809.01293954
				6774 1008810.29411951
				6775 1008811.57529947
				6776 1008812.85647944
				6777 1008814.1376594
				6778 1008815.41883937
				6779 1008816.70001934
				6780 1008817.9811993
				6781 1008819.26237927
				6782 1008820.54355923
				6783 1008821.8247392
				6784 1008823.10591916
				6785 1008824.38709913
				6786 1008825.66827909
				6787 1008826.94945906
				6788 1008828.23063902
				6789 1008829.51181899
				6790 1008830.79299896
				6791 1008832.07417892
				6792 1008833.35535889
				6793 1008834.63653885
				6794 1008835.91771882
				6795 1008837.19889878
				6796 1008838.48007875
				6797 1008839.76125871
				6798 1008841.04243868
				6799 1008842.32361864
				6800 1008843.60479861
				6801 1008844.88597858
				6802 1008846.16715854
				6803 1008847.44833851
				6804 1008848.72951847
				6805 1008850.01069844
				6806 1008851.2918784
				6807 1008852.57305837
				6808 1008853.85423833
				6809 1008855.1354183
				6810 1008856.41659826
				6811 1008857.69777823
				6812 1008858.9789582
				6813 1008860.26013816
				6814 1008861.54131813
				6815 1008862.82249809
				6816 1008864.10367806
				6817 1008865.38485802
				6818 1008866.66603799
				6819 1008867.94721795
				6820 1008869.22839792
				6821 1008870.50957788
				6822 1008871.79075785
				6823 1008873.07193782
				6824 1008874.35311778
				6825 1008875.63429775
				6826 1008876.91547771
				6827 1008878.19665768
				6828 1008879.47783764
				6829 1008880.75901761
				6830 1008882.04019757
				6831 1008883.32137754
				6832 1008884.6025575
				6833 1008885.88373747
				6834 1008887.16491744
				6835 1008888.4460974
				6836 1008889.72727737
				6837 1008891.00845733
				6838 1008892.2896373
				6839 1008893.57081726
				6840 1008894.85199723
				6841 1008896.13317719
				6842 1008897.41435716
				6843 1008898.69553712
				6844 1008899.97671709
				6845 1008901.25789706
				6846 1008902.53907702
				6847 1008903.82025699
				6848 1008905.10143695
				6849 1008906.38261692
				6850 1008907.66379688
				6851 1008908.94497685
				6852 1008910.22615681
				6853 1008911.50733678
				6854 1008912.78851674
				6855 1008914.06969671
				6856 1008915.35087668
				6857 1008916.63205664
				6858 1008917.91323661
				6859 1008919.19441657
				6860 1008920.47559654
				6861 1008921.7567765
				6862 1008923.03795647
				6863 1008924.31913643
				6864 1008925.6003164
				6865 1008926.88149636
				6866 1008928.16267633
				6867 1008929.4438563
				6868 1008930.72503626
				6869 1008932.00621623
				6870 1008933.28739619
				6871 1008934.56857616
				6872 1008935.84975612
				6873 1008937.13093609
				6874 1008938.41211605
				6875 1008939.69329602
				6876 1008940.97447598
				6877 1008942.25565595
				6878 1008943.53683592
				6879 1008944.81801588
				6880 1008946.09919585
				6881 1008947.38037581
				6882 1008948.66155578
				6883 1008949.94273574
				6884 1008951.22391571
				6885 1008952.50509567
				6886 1008953.78627564
				6887 1008955.0674556
				6888 1008956.34863557
				6889 1008957.62981554
				6890 1008958.9109955
				6891 1008960.19217547
				6892 1008961.47335543
				6893 1008962.7545354
				6894 1008964.03571536
				6895 1008965.31689533
				6896 1008966.59807529
				6897 1008967.87925526
				6898 1008969.16043522
				6899 1008970.44161519
				6900 1008971.72279516
				6901 1008973.00397512
				6902 1008974.28515509
				6903 1008975.56633505
				6904 1008976.84751502
				6905 1008978.12869498
				6906 1008979.40987495
				6907 1008980.69105491
				6908 1008981.97223488
				6909 1008983.25341484
				6910 1008984.53459481
				6911 1008985.81577478
				6912 1008987.09695474
				6913 1008988.37813471
				6914 1008989.65931467
				6915 1008990.94049464
				6916 1008992.2216746
				6917 1008993.50285457
				6918 1008994.78403453
				6919 1008996.0652145
				6920 1008997.34639446
				6921 1008998.62757443
				6922 1008999.9087544
				6923 1009001.18993436
				6924 1009002.47111433
				6925 1009003.75229429
				6926 1009005.03347426
				6927 1009006.31465422
				6928 1009007.59583419
				6929 1009008.87701415
				6930 1009010.15819412
				6931 1009011.43937408
				6932 1009012.72055405
				6933 1009014.00173402
				6934 1009015.28291398
				6935 1009016.56409395
				6936 1009017.84527391
				6937 1009019.12645388
				6938 1009020.40763384
				6939 1009021.68881381
				6940 1009022.96999377
				6941 1009024.25117374
				6942 1009025.5323537
				6943 1009026.81353367
				6944 1009028.09471364
				6945 1009029.3758936
				6946 1009030.65707357
				6947 1009031.93825353
				6948 1009033.2194335
				6949 1009034.50061346
				6950 1009035.78179343
				6951 1009037.06297339
				6952 1009038.34415336
				6953 1009039.62533332
				6954 1009040.90651329
				6955 1009042.18769326
				6956 1009043.46887322
				6957 1009044.75005319
				6958 1009046.03123315
				6959 1009047.31241312
				6960 1009048.59359308
				6961 1009049.87477305
				6962 1009051.15595301
				6963 1009052.43713298
				6964 1009053.71831294
				6965 1009054.99949291
				6966 1009056.28067288
				6967 1009057.56185284
				6968 1009058.84303281
				6969 1009060.12421277
				6970 1009061.40539274
				6971 1009062.6865727
				6972 1009063.96775267
				6973 1009065.24893263
				6974 1009066.5301126
				6975 1009067.81129256
				6976 1009069.09247253
				6977 1009070.3736525
				6978 1009071.65483246
				6979 1009072.93601243
				6980 1009074.21719239
				6981 1009075.49837236
				6982 1009076.77955232
				6983 1009078.06073229
				6984 1009079.34191225
				6985 1009080.62309222
				6986 1009081.90427218
				6987 1009083.18545215
				6988 1009084.46663212
				6989 1009085.74781208
				6990 1009087.02899205
				6991 1009088.31017201
				6992 1009089.59135198
				6993 1009090.87253194
				6994 1009092.15371191
				6995 1009093.43489187
				6996 1009094.71607184
				6997 1009095.9972518
				6998 1009097.27843177
				6999 1009098.55961174
				7000 1009099.8407917
				7001 1009101.12197167
				7002 1009102.40315163
				7003 1009103.6843316
				7004 1009104.96551156
				7005 1009106.24669153
				7006 1009107.52787149
				7007 1009108.80905146
				7008 1009110.09023142
				7009 1009111.37141139
				7010 1009112.65259136
				7011 1009113.93377132
				7012 1009115.21495129
				7013 1009116.49613125
				7014 1009117.77731122
				7015 1009119.05849118
				7016 1009120.33967115
				7017 1009121.62085111
				7018 1009122.90203108
				7019 1009124.18321104
				7020 1009125.46439101
				7021 1009126.74557098
				7022 1009128.02675094
				7023 1009129.30793091
				7024 1009130.58911087
				7025 1009131.87029084
				7026 1009133.1514708
				7027 1009134.43265077
				7028 1009135.71383073
				7029 1009136.9950107
				7030 1009138.27619066
				7031 1009139.55737063
				7032 1009140.8385506
				7033 1009142.11973056
				7034 1009143.40091053
				7035 1009144.68209049
				7036 1009145.96327046
				7037 1009147.24445042
				7038 1009148.52563039
				7039 1009149.80681035
				7040 1009151.08799032
				7041 1009152.36917028
				7042 1009153.65035025
				7043 1009154.93153022
				7044 1009156.21271018
				7045 1009157.49389015
				7046 1009158.77507011
				7047 1009160.05625008
				7048 1009161.33743004
				7049 1009162.61861001
				7050 1009163.89978997
				7051 1009165.18096994
				7052 1009166.4621499
				7053 1009167.74332987
				7054 1009169.02450984
				7055 1009170.3056898
				7056 1009171.58686977
				7057 1009172.86804973
				7058 1009174.1492297
				7059 1009175.43040966
				7060 1009176.71158963
				7061 1009177.99276959
				7062 1009179.27394956
				7063 1009180.55512952
				7064 1009181.83630949
				7065 1009183.11748946
				7066 1009184.39866942
				7067 1009185.67984939
				7068 1009186.96102935
				7069 1009188.24220932
				7070 1009189.52338928
				7071 1009190.80456925
				7072 1009192.08574921
				7073 1009193.36692918
				7074 1009194.64810914
				7075 1009195.92928911
				7076 1009197.21046908
				7077 1009198.49164904
				7078 1009199.77282901
				7079 1009201.05400897
				7080 1009202.33518894
				7081 1009203.6163689
				7082 1009204.89754887
				7083 1009206.17872883
				7084 1009207.4599088
				7085 1009208.74108876
				7086 1009210.02226873
				7087 1009211.3034487
				7088 1009212.58462866
				7089 1009213.86580863
				7090 1009215.14698859
				7091 1009216.42816856
				7092 1009217.70934852
				7093 1009218.99052849
				7094 1009220.27170845
				7095 1009221.55288842
				7096 1009222.83406838
				7097 1009224.11524835
				7098 1009225.39642832
				7099 1009226.67760828
				7100 1009227.95878825
				7101 1009229.23996821
				7102 1009230.52114818
				7103 1009231.80232814
				7104 1009233.08350811
				7105 1009234.36468807
				7106 1009235.64586804
				7107 1009236.927048
				7108 1009238.20822797
				7109 1009239.48940794
				7110 1009240.7705879
				7111 1009242.05176787
				7112 1009243.33294783
				7113 1009244.6141278
				7114 1009245.89530776
				7115 1009247.17648773
				7116 1009248.45766769
				7117 1009249.73884766
				7118 1009251.02002762
				7119 1009252.30120759
				7120 1009253.58238756
				7121 1009254.86356752
				7122 1009256.14474749
				7123 1009257.42592745
				7124 1009258.70710742
				7125 1009259.98828738
				7126 1009261.26946735
				7127 1009262.55064731
				7128 1009263.83182728
				7129 1009265.11300724
				7130 1009266.39418721
				7131 1009267.67536718
				7132 1009268.95654714
				7133 1009270.23772711
				7134 1009271.51890707
				7135 1009272.80008704
				7136 1009274.081267
				7137 1009275.36244697
				7138 1009276.64362693
				7139 1009277.9248069
				7140 1009279.20598686
				7141 1009280.48716683
				7142 1009281.7683468
				7143 1009283.04952676
				7144 1009284.33070673
				7145 1009285.61188669
				7146 1009286.89306666
				7147 1009288.17424662
				7148 1009289.45542659
				7149 1009290.73660655
				7150 1009292.01778652
				7151 1009293.29896648
				7152 1009294.58014645
				7153 1009295.86132642
				7154 1009297.14250638
				7155 1009298.42368635
				7156 1009299.70486631
				7157 1009300.98604628
				7158 1009302.26722624
				7159 1009303.54840621
				7160 1009304.82958617
				7161 1009306.11076614
				7162 1009307.3919461
				7163 1009308.67312607
				7164 1009309.95430604
				7165 1009311.235486
				7166 1009312.51666597
				7167 1009313.79784593
				7168 1009315.0790259
				7169 1009316.36020586
				7170 1009317.64138583
				7171 1009318.92256579
				7172 1009320.20374576
				7173 1009321.48492572
				7174 1009322.76610569
				7175 1009324.04728566
				7176 1009325.32846562
				7177 1009326.60964559
				7178 1009327.89082555
				7179 1009329.17200552
				7180 1009330.45318548
				7181 1009331.73436545
				7182 1009333.01554541
				7183 1009334.29672538
				7184 1009335.57790534
				7185 1009336.85908531
				7186 1009338.14026528
				7187 1009339.42144524
				7188 1009340.70262521
				7189 1009341.98380517
				7190 1009343.26498514
				7191 1009344.5461651
				7192 1009345.82734507
				7193 1009347.10852503
				7194 1009348.389705
				7195 1009349.67088496
				7196 1009350.95206493
				7197 1009352.2332449
				7198 1009353.51442486
				7199 1009354.79560483
				7200 1009356.07678479
				7201 1009357.35796476
				7202 1009358.63914472
				7203 1009359.92032469
				7204 1009361.20150465
				7205 1009362.48268462
				7206 1009363.76386458
				7207 1009365.04504455
				7208 1009366.32622452
				7209 1009367.60740448
				7210 1009368.88858445
				7211 1009370.16976441
				7212 1009371.45094438
				7213 1009372.73212434
				7214 1009374.01330431
				7215 1009375.29448427
				7216 1009376.57566424
				7217 1009377.8568442
				7218 1009379.13802417
				7219 1009380.41920414
				7220 1009381.7003841
				7221 1009382.98156407
				7222 1009384.26274403
				7223 1009385.543924
				7224 1009386.82510396
				7225 1009388.10628393
				7226 1009389.38746389
				7227 1009390.66864386
				7228 1009391.94982382
				7229 1009393.23100379
				7230 1009394.51218376
				7231 1009395.79336372
				7232 1009397.07454369
				7233 1009398.35572365
				7234 1009399.63690362
				7235 1009400.91808358
				7236 1009402.19926355
				7237 1009403.48044351
				7238 1009404.76162348
				7239 1009406.04280344
				7240 1009407.32398341
				7241 1009408.60516338
				7242 1009409.88634334
				7243 1009411.16752331
				7244 1009412.44870327
				7245 1009413.72988324
				7246 1009415.0110632
				7247 1009416.29224317
				7248 1009417.57342313
				7249 1009418.8546031
				7250 1009420.13578306
				7251 1009421.41696303
				7252 1009422.698143
				7253 1009423.97932296
				7254 1009425.26050293
				7255 1009426.54168289
				7256 1009427.82286286
				7257 1009429.10404282
				7258 1009430.38522279
				7259 1009431.66640275
				7260 1009432.94758272
				7261 1009434.22876268
				7262 1009435.50994265
				7263 1009436.79112262
				7264 1009438.07230258
				7265 1009439.35348255
				7266 1009440.63466251
				7267 1009441.91584248
				7268 1009443.19702244
				7269 1009444.47820241
				7270 1009445.75938237
				7271 1009447.04056234
				7272 1009448.3217423
				7273 1009449.60292227
				7274 1009450.88410224
				7275 1009452.1652822
				7276 1009453.44646217
				7277 1009454.72764213
				7278 1009456.0088221
				7279 1009457.29000206
				7280 1009458.57118203
				7281 1009459.85236199
				7282 1009461.13354196
				7283 1009462.41472192
				7284 1009463.69590189
				7285 1009464.97708186
				7286 1009466.25826182
				7287 1009467.53944179
				7288 1009468.82062175
				7289 1009470.10180172
				7290 1009471.38298168
				7291 1009472.66416165
				7292 1009473.94534161
				7293 1009475.22652158
				7294 1009476.50770154
				7295 1009477.78888151
				7296 1009479.07006148
				7297 1009480.35124144
				7298 1009481.63242141
				7299 1009482.91360137
				7300 1009484.19478134
				7301 1009485.4759613
				7302 1009486.75714127
				7303 1009488.03832123
				7304 1009489.3195012
				7305 1009490.60068116
				7306 1009491.88186113
				7307 1009493.1630411
				7308 1009494.44422106
				7309 1009495.72540103
				7310 1009497.00658099
				7311 1009498.28776096
				7312 1009499.56894092
				7313 1009500.85012089
				7314 1009502.13130085
				7315 1009503.41248082
				7316 1009504.69366078
				7317 1009505.97484075
				7318 1009507.25602072
				7319 1009508.53720068
				7320 1009509.81838065
				7321 1009511.09956061
				7322 1009512.38074058
				7323 1009513.66192054
				7324 1009514.94310051
				7325 1009516.22428047
				7326 1009517.50546044
				7327 1009518.7866404
				7328 1009520.06782037
				7329 1009521.34900034
				7330 1009522.6301803
				7331 1009523.91136027
				7332 1009525.19254023
				7333 1009526.4737202
				7334 1009527.75490016
				7335 1009529.03608013
				7336 1009530.31726009
				7337 1009531.59844006
				7338 1009532.87962002
				7339 1009534.16079999
				7340 1009535.44197996
				7341 1009536.72315992
				7342 1009538.00433989
				7343 1009539.28551985
				7344 1009540.56669982
				7345 1009541.84787978
				7346 1009543.12905975
				7347 1009544.41023971
				7348 1009545.69141968
				7349 1009546.97259964
				7350 1009548.25377961
				7351 1009549.53495958
				7352 1009550.81613954
				7353 1009552.09731951
				7354 1009553.37849947
				7355 1009554.65967944
				7356 1009555.9408594
				7357 1009557.22203937
				7358 1009558.50321933
				7359 1009559.7843993
				7360 1009561.06557926
				7361 1009562.34675923
				7362 1009563.6279392
				7363 1009564.90911916
				7364 1009566.19029913
				7365 1009567.47147909
				7366 1009568.75265906
				7367 1009570.03383902
				7368 1009571.31501899
				7369 1009572.59619895
				7370 1009573.87737892
				7371 1009575.15855888
				7372 1009576.43973885
				7373 1009577.72091882
				7374 1009579.00209878
				7375 1009580.28327875
				7376 1009581.56445871
				7377 1009582.84563868
				7378 1009584.12681864
				7379 1009585.40799861
				7380 1009586.68917857
				7381 1009587.97035854
				7382 1009589.2515385
				7383 1009590.53271847
				7384 1009591.81389844
				7385 1009593.0950784
				7386 1009594.37625837
				7387 1009595.65743833
				7388 1009596.9386183
				7389 1009598.21979826
				7390 1009599.50097823
				7391 1009600.78215819
				7392 1009602.06333816
				7393 1009603.34451812
				7394 1009604.62569809
				7395 1009605.90687806
				7396 1009607.18805802
				7397 1009608.46923799
				7398 1009609.75041795
				7399 1009611.03159792
				7400 1009612.31277788
				7401 1009613.59395785
				7402 1009614.87513781
				7403 1009616.15631778
				7404 1009617.43749774
				7405 1009618.71867771
				7406 1009619.99985768
				7407 1009621.28103764
				7408 1009622.56221761
				7409 1009623.84339757
				7410 1009625.12457754
				7411 1009626.4057575
				7412 1009627.68693747
				7413 1009628.96811743
				7414 1009630.2492974
				7415 1009631.53047736
				7416 1009632.81165733
				7417 1009634.0928373
				7418 1009635.37401726
				7419 1009636.65519723
				7420 1009637.93637719
				7421 1009639.21755716
				7422 1009640.49873712
				7423 1009641.77991709
				7424 1009643.06109705
				7425 1009644.34227702
				7426 1009645.62345698
				7427 1009646.90463695
				7428 1009648.18581692
				7429 1009649.46699688
				7430 1009650.74817685
				7431 1009652.02935681
				7432 1009653.31053678
				7433 1009654.59171674
				7434 1009655.87289671
				7435 1009657.15407667
				7436 1009658.43525664
				7437 1009659.7164366
				7438 1009660.99761657
				7439 1009662.27879653
				7440 1009663.5599765
				7441 1009664.84115647
				7442 1009666.12233643
				7443 1009667.4035164
				7444 1009668.68469636
				7445 1009669.96587633
				7446 1009671.24705629
				7447 1009672.52823626
				7448 1009673.80941622
				7449 1009675.09059619
				7450 1009676.37177615
				7451 1009677.65295612
				7452 1009678.93413609
				7453 1009680.21531605
				7454 1009681.49649602
				7455 1009682.77767598
				7456 1009684.05885595
				7457 1009685.34003591
				7458 1009686.62121588
				7459 1009687.90239584
				7460 1009689.18357581
				7461 1009690.46475577
				7462 1009691.74593574
				7463 1009693.02711571
				7464 1009694.30829567
				7465 1009695.58947564
				7466 1009696.8706556
				7467 1009698.15183557
				7468 1009699.43301553
				7469 1009700.7141955
				7470 1009701.99537546
				7471 1009703.27655543
				7472 1009704.5577354
				7473 1009705.83891536
				7474 1009707.12009533
				7475 1009708.40127529
				7476 1009709.68245526
				7477 1009710.96363522
				7478 1009712.24481519
				7479 1009713.52599515
				7480 1009714.80717512
				7481 1009716.08835508
				7482 1009717.36953505
				7483 1009718.65071502
				7484 1009719.93189498
				7485 1009721.21307495
				7486 1009722.49425491
				7487 1009723.77543488
				7488 1009725.05661484
				7489 1009726.33779481
				7490 1009727.61897477
				7491 1009728.90015474
				7492 1009730.1813347
				7493 1009731.46251467
				7494 1009732.74369463
				7495 1009734.0248746
				7496 1009735.30605457
				7497 1009736.58723453
				7498 1009737.8684145
				7499 1009739.14959446
				7500 1009740.43077443
				7501 1009741.71195439
				7502 1009742.99313436
				7503 1009744.27431432
				7504 1009745.55549429
				7505 1009746.83667425
				7506 1009748.11785422
				7507 1009749.39903419
				7508 1009750.68021415
				7509 1009751.96139412
				7510 1009753.24257408
				7511 1009754.52375405
				7512 1009755.80493401
				7513 1009757.08611398
				7514 1009758.36729394
				7515 1009759.64847391
				7516 1009760.92965387
				7517 1009762.21083384
				7518 1009763.49201381
				7519 1009764.77319377
				7520 1009766.05437374
				7521 1009767.3355537
				7522 1009768.61673367
				7523 1009769.89791363
				7524 1009771.1790936
				7525 1009772.46027356
				7526 1009773.74145353
				7527 1009775.02263349
				7528 1009776.30381346
				7529 1009777.58499343
				7530 1009778.86617339
				7531 1009780.14735336
				7532 1009781.42853332
				7533 1009782.70971329
				7534 1009783.99089325
				7535 1009785.27207322
				7536 1009786.55325318
				7537 1009787.83443315
				7538 1009789.11561311
				7539 1009790.39679308
				7540 1009791.67797305
				7541 1009792.95915301
				7542 1009794.24033298
				7543 1009795.52151294
				7544 1009796.80269291
				7545 1009798.08387287
				7546 1009799.36505284
				7547 1009800.6462328
				7548 1009801.92741277
				7549 1009803.20859273
				7550 1009804.4897727
				7551 1009805.77095267
				7552 1009807.05213263
				7553 1009808.3333126
				7554 1009809.61449256
				7555 1009810.89567253
				7556 1009812.17685249
				7557 1009813.45803246
				7558 1009814.73921242
				7559 1009816.02039239
				7560 1009817.30157235
				7561 1009818.58275232
				7562 1009819.86393229
				7563 1009821.14511225
				7564 1009822.42629222
				7565 1009823.70747218
				7566 1009824.98865215
				7567 1009826.26983211
				7568 1009827.55101208
				7569 1009828.83219204
				7570 1009830.11337201
				7571 1009831.39455197
				7572 1009832.67573194
				7573 1009833.95691191
				7574 1009835.23809187
				7575 1009836.51927184
				7576 1009837.8004518
				7577 1009839.08163177
				7578 1009840.36281173
				7579 1009841.6439917
				7580 1009842.92517166
				7581 1009844.20635163
				7582 1009845.48753159
				7583 1009846.76871156
				7584 1009848.04989153
				7585 1009849.33107149
				7586 1009850.61225146
				7587 1009851.89343142
				7588 1009853.17461139
				7589 1009854.45579135
				7590 1009855.73697132
				7591 1009857.01815128
				7592 1009858.29933125
				7593 1009859.58051121
				7594 1009860.86169118
				7595 1009862.14287115
				7596 1009863.42405111
				7597 1009864.70523108
				7598 1009865.98641104
				7599 1009867.26759101
				7600 1009868.54877097
				7601 1009869.82995094
				7602 1009871.1111309
				7603 1009872.39231087
				7604 1009873.67349083
				7605 1009874.9546708
				7606 1009876.23585077
				7607 1009877.51703073
				7608 1009878.7982107
				7609 1009880.07939066
				7610 1009881.36057063
				7611 1009882.64175059
				7612 1009883.92293056
				7613 1009885.20411052
				7614 1009886.48529049
				7615 1009887.76647045
				7616 1009889.04765042
				7617 1009890.32883039
				7618 1009891.61001035
				7619 1009892.89119032
				7620 1009894.17237028
				7621 1009895.45355025
				7622 1009896.73473021
				7623 1009898.01591018
				7624 1009899.29709014
				7625 1009900.57827011
				7626 1009901.85945007
				7627 1009903.14063004
				7628 1009904.42181001
				7629 1009905.70298997
				7630 1009906.98416994
				7631 1009908.2653499
				7632 1009909.54652987
				7633 1009910.82770983
				7634 1009912.1088898
				7635 1009913.39006976
				7636 1009914.67124973
				7637 1009915.95242969
				7638 1009917.23360966
				7639 1009918.51478963
				7640 1009919.79596959
				7641 1009921.07714956
				7642 1009922.35832952
				7643 1009923.63950949
				7644 1009924.92068945
				7645 1009926.20186942
				7646 1009927.48304938
				7647 1009928.76422935
				7648 1009930.04540931
				7649 1009931.32658928
				7650 1009932.60776925
				7651 1009933.88894921
				7652 1009935.17012918
				7653 1009936.45130914
				7654 1009937.73248911
				7655 1009939.01366907
				7656 1009940.29484904
				7657 1009941.576029
				7658 1009942.85720897
				7659 1009944.13838893
				7660 1009945.4195689
				7661 1009946.70074887
				7662 1009947.98192883
				7663 1009949.2631088
				7664 1009950.54428876
				7665 1009951.82546873
				7666 1009953.10664869
				7667 1009954.38782866
				7668 1009955.66900862
				7669 1009956.95018859
				7670 1009958.23136855
				7671 1009959.51254852
				7672 1009960.79372849
				7673 1009962.07490845
				7674 1009963.35608842
				7675 1009964.63726838
				7676 1009965.91844835
				7677 1009967.19962831
				7678 1009968.48080828
				7679 1009969.76198824
				7680 1009971.04316821
				7681 1009972.32434817
				7682 1009973.60552814
				7683 1009974.88670811
				7684 1009976.16788807
				7685 1009977.44906804
				7686 1009978.730248
				7687 1009980.01142797
				7688 1009981.29260793
				7689 1009982.5737879
				7690 1009983.85496786
				7691 1009985.13614783
				7692 1009986.41732779
				7693 1009987.69850776
				7694 1009988.97968773
				7695 1009990.26086769
				7696 1009991.54204766
				7697 1009992.82322762
				7698 1009994.10440759
				7699 1009995.38558755
				7700 1009996.66676752
				7701 1009997.94794748
				7702 1009999.22912745
				7703 1010000.51030741
				7704 1010001.79148738
				7705 1010003.07266735
				7706 1010004.35384731
				7707 1010005.63502728
				7708 1010006.91620724
				7709 1010008.19738721
				7710 1010009.47856717
				7711 1010010.75974714
				7712 1010012.0409271
				7713 1010013.32210707
				7714 1010014.60328703
				7715 1010015.884467
				7716 1010017.16564697
				7717 1010018.44682693
				7718 1010019.7280069
				7719 1010021.00918686
				7720 1010022.29036683
				7721 1010023.57154679
				7722 1010024.85272676
				7723 1010026.13390672
				7724 1010027.41508669
				7725 1010028.69626665
				7726 1010029.97744662
				7727 1010031.25862659
				7728 1010032.53980655
				7729 1010033.82098652
				7730 1010035.10216648
				7731 1010036.38334645
				7732 1010037.66452641
				7733 1010038.94570638
				7734 1010040.22688634
				7735 1010041.50806631
				7736 1010042.78924627
				7737 1010044.07042624
				7738 1010045.35160621
				7739 1010046.63278617
				7740 1010047.91396614
				7741 1010049.1951461
				7742 1010050.47632607
				7743 1010051.75750603
				7744 1010053.038686
				7745 1010054.31986596
				7746 1010055.60104593
				7747 1010056.88222589
				7748 1010058.16340586
				7749 1010059.44458583
				7750 1010060.72576579
				7751 1010062.00694576
				7752 1010063.28812572
				7753 1010064.56930569
				7754 1010065.85048565
				7755 1010067.13166562
				7756 1010068.41284558
				7757 1010069.69402555
				7758 1010070.97520551
				7759 1010072.25638548
				7760 1010073.53756545
				7761 1010074.81874541
				7762 1010076.09992538
				7763 1010077.38110534
				7764 1010078.66228531
				7765 1010079.94346527
				7766 1010081.22464524
				7767 1010082.5058252
				7768 1010083.78700517
				7769 1010085.06818513
				7770 1010086.3493651
				7771 1010087.63054507
				7772 1010088.91172503
				7773 1010090.192905
				7774 1010091.47408496
				7775 1010092.75526493
				7776 1010094.03644489
				7777 1010095.31762486
				7778 1010096.59880482
				7779 1010097.87998479
				7780 1010099.16116475
				7781 1010100.44234472
				7782 1010101.72352469
				7783 1010103.00470465
				7784 1010104.28588462
				7785 1010105.56706458
				7786 1010106.84824455
				7787 1010108.12942451
				7788 1010109.41060448
				7789 1010110.69178444
				7790 1010111.97296441
				7791 1010113.25414437
				7792 1010114.53532434
				7793 1010115.81650431
				7794 1010117.09768427
				7795 1010118.37886424
				7796 1010119.6600442
				7797 1010120.94122417
				7798 1010122.22240413
				7799 1010123.5035841
				7800 1010124.78476406
				7801 1010126.06594403
				7802 1010127.34712399
				7803 1010128.62830396
				7804 1010129.90948393
				7805 1010131.19066389
				7806 1010132.47184386
				7807 1010133.75302382
				7808 1010135.03420379
				7809 1010136.31538375
				7810 1010137.59656372
				7811 1010138.87774368
				7812 1010140.15892365
				7813 1010141.44010361
				7814 1010142.72128358
				7815 1010144.00246355
				7816 1010145.28364351
				7817 1010146.56482348
				7818 1010147.84600344
				7819 1010149.12718341
				7820 1010150.40836337
				7821 1010151.68954334
				7822 1010152.9707233
				7823 1010154.25190327
				7824 1010155.53308323
				7825 1010156.8142632
				7826 1010158.09544317
				7827 1010159.37662313
				7828 1010160.6578031
				7829 1010161.93898306
				7830 1010163.22016303
				7831 1010164.50134299
				7832 1010165.78252296
				7833 1010167.06370292
				7834 1010168.34488289
				7835 1010169.62606285
				7836 1010170.90724282
				7837 1010172.18842279
				7838 1010173.46960275
				7839 1010174.75078272
				7840 1010176.03196268
				7841 1010177.31314265
				7842 1010178.59432261
				7843 1010179.87550258
				7844 1010181.15668254
				7845 1010182.43786251
				7846 1010183.71904247
				7847 1010185.00022244
				7848 1010186.28140241
				7849 1010187.56258237
				7850 1010188.84376234
				7851 1010190.1249423
				7852 1010191.40612227
				7853 1010192.68730223
				7854 1010193.9684822
				7855 1010195.24966216
				7856 1010196.53084213
				7857 1010197.81202209
				7858 1010199.09320206
				7859 1010200.37438203
				7860 1010201.65556199
				7861 1010202.93674196
				7862 1010204.21792192
				7863 1010205.49910189
				7864 1010206.78028185
				7865 1010208.06146182
				7866 1010209.34264178
				7867 1010210.62382175
				7868 1010211.90500171
				7869 1010213.18618168
				7870 1010214.46736165
				7871 1010215.74854161
				7872 1010217.02972158
				7873 1010218.31090154
				7874 1010219.59208151
				7875 1010220.87326147
				7876 1010222.15444144
				7877 1010223.4356214
				7878 1010224.71680137
				7879 1010225.99798133
				7880 1010227.2791613
				7881 1010228.56034127
				7882 1010229.84152123
				7883 1010231.1227012
				7884 1010232.40388116
				7885 1010233.68506113
				7886 1010234.96624109
				7887 1010236.24742106
				7888 1010237.52860102
				7889 1010238.80978099
				7890 1010240.09096095
				7891 1010241.37214092
				7892 1010242.65332089
				7893 1010243.93450085
				7894 1010245.21568082
				7895 1010246.49686078
				7896 1010247.77804075
				7897 1010249.05922071
				7898 1010250.34040068
				7899 1010251.62158064
				7900 1010252.90276061
				7901 1010254.18394057
				7902 1010255.46512054
				7903 1010256.74630051
				7904 1010258.02748047
				7905 1010259.30866044
				7906 1010260.5898404
				7907 1010261.87102037
				7908 1010263.15220033
				7909 1010264.4333803
				7910 1010265.71456026
				7911 1010266.99574023
				7912 1010268.27692019
				7913 1010269.55810016
				7914 1010270.83928013
				7915 1010272.12046009
				7916 1010273.40164006
				7917 1010274.68282002
				7918 1010275.96399999
				7919 1010277.24517995
				7920 1010278.52635992
				7921 1010279.80753988
				7922 1010281.08871985
				7923 1010282.36989981
				7924 1010283.65107978
				7925 1010284.93225975
				7926 1010286.21343971
				7927 1010287.49461968
				7928 1010288.77579964
				7929 1010290.05697961
				7930 1010291.33815957
				7931 1010292.61933954
				7932 1010293.9005195
				7933 1010295.18169947
				7934 1010296.46287943
				7935 1010297.7440594
				7936 1010299.02523937
				7937 1010300.30641933
				7938 1010301.5875993
				7939 1010302.86877926
				7940 1010304.14995923
				7941 1010305.43113919
				7942 1010306.71231916
				7943 1010307.99349912
				7944 1010309.27467909
				7945 1010310.55585905
				7946 1010311.83703902
				7947 1010313.11821899
				7948 1010314.39939895
				7949 1010315.68057892
				7950 1010316.96175888
				7951 1010318.24293885
				7952 1010319.52411881
				7953 1010320.80529878
				7954 1010322.08647874
				7955 1010323.36765871
				7956 1010324.64883867
				7957 1010325.93001864
				7958 1010327.21119861
				7959 1010328.49237857
				7960 1010329.77355854
				7961 1010331.0547385
				7962 1010332.33591847
				7963 1010333.61709843
				7964 1010334.8982784
				7965 1010336.17945836
				7966 1010337.46063833
				7967 1010338.74181829
				7968 1010340.02299826
				7969 1010341.30417823
				7970 1010342.58535819
				7971 1010343.86653816
				7972 1010345.14771812
				7973 1010346.42889809
				7974 1010347.71007805
				7975 1010348.99125802
				7976 1010350.27243798
				7977 1010351.55361795
				7978 1010352.83479791
				7979 1010354.11597788
				7980 1010355.39715785
				7981 1010356.67833781
				7982 1010357.95951778
				7983 1010359.24069774
				7984 1010360.52187771
				7985 1010361.80305767
				7986 1010363.08423764
				7987 1010364.3654176
				7988 1010365.64659757
				7989 1010366.92777753
				7990 1010368.2089575
				7991 1010369.49013747
				7992 1010370.77131743
				7993 1010372.0524974
				7994 1010373.33367736
				7995 1010374.61485733
				7996 1010375.89603729
				7997 1010377.17721726
				7998 1010378.45839722
				7999 1010379.73957719
				8000 1010381.02075715
				8001 1010382.30193712
				8002 1010383.58311709
				8003 1010384.86429705
				8004 1010386.14547702
				8005 1010387.42665698
				8006 1010388.70783695
				8007 1010389.98901691
				8008 1010391.27019688
				8009 1010392.55137684
				8010 1010393.83255681
				8011 1010395.11373677
				8012 1010396.39491674
				8013 1010397.67609671
				8014 1010398.95727667
				8015 1010400.23845664
				8016 1010401.5196366
				8017 1010402.80081657
				8018 1010404.08199653
				8019 1010405.3631765
				8020 1010406.64435646
				8021 1010407.92553643
				8022 1010409.20671639
				8023 1010410.48789636
				8024 1010411.76907633
				8025 1010413.05025629
				8026 1010414.33143626
				8027 1010415.61261622
				8028 1010416.89379619
				8029 1010418.17497615
				8030 1010419.45615612
				8031 1010420.73733608
				8032 1010422.01851605
				8033 1010423.29969601
				8034 1010424.58087598
				8035 1010425.86205595
				8036 1010427.14323591
				8037 1010428.42441588
				8038 1010429.70559584
				8039 1010430.98677581
				8040 1010432.26795577
				8041 1010433.54913574
				8042 1010434.8303157
				8043 1010436.11149567
				8044 1010437.39267563
				8045 1010438.6738556
				8046 1010439.95503557
				8047 1010441.23621553
				8048 1010442.5173955
				8049 1010443.79857546
				8050 1010445.07975543
				8051 1010446.36093539
				8052 1010447.64211536
				8053 1010448.92329532
				8054 1010450.20447529
				8055 1010451.48565525
				8056 1010452.76683522
				8057 1010454.04801519
				8058 1010455.32919515
				8059 1010456.61037512
				8060 1010457.89155508
				8061 1010459.17273505
				8062 1010460.45391501
				8063 1010461.73509498
				8064 1010463.01627494
				8065 1010464.29745491
				8066 1010465.57863487
				8067 1010466.85981484
				8068 1010468.14099481
				8069 1010469.42217477
				8070 1010470.70335474
				8071 1010471.9845347
				8072 1010473.26571467
				8073 1010474.54689463
				8074 1010475.8280746
				8075 1010477.10925456
				8076 1010478.39043453
				8077 1010479.67161449
				8078 1010480.95279446
				8079 1010482.23397443
				8080 1010483.51515439
				8081 1010484.79633436
				8082 1010486.07751432
				8083 1010487.35869429
				8084 1010488.63987425
				8085 1010489.92105422
				8086 1010491.20223418
				8087 1010492.48341415
				8088 1010493.76459411
				8089 1010495.04577408
				8090 1010496.32695405
				8091 1010497.60813401
				8092 1010498.88931398
				8093 1010500.17049394
				8094 1010501.45167391
				8095 1010502.73285387
				8096 1010504.01403384
				8097 1010505.2952138
				8098 1010506.57639377
				8099 1010507.85757373
				8100 1010509.1387537
				8101 1010510.41993367
				8102 1010511.70111363
				8103 1010512.9822936
				8104 1010514.26347356
				8105 1010515.54465353
				8106 1010516.82583349
				8107 1010518.10701346
				8108 1010519.38819342
				8109 1010520.66937339
				8110 1010521.95055335
				8111 1010523.23173332
				8112 1010524.51291329
				8113 1010525.79409325
				8114 1010527.07527322
				8115 1010528.35645318
				8116 1010529.63763315
				8117 1010530.91881311
				8118 1010532.19999308
				8119 1010533.48117304
				8120 1010534.76235301
				8121 1010536.04353297
				8122 1010537.32471294
				8123 1010538.60589291
				8124 1010539.88707287
				8125 1010541.16825284
				8126 1010542.4494328
				8127 1010543.73061277
				8128 1010545.01179273
				8129 1010546.2929727
				8130 1010547.57415266
				8131 1010548.85533263
				8132 1010550.13651259
				8133 1010551.41769256
				8134 1010552.69887253
				8135 1010553.98005249
				8136 1010555.26123246
				8137 1010556.54241242
				8138 1010557.82359239
				8139 1010559.10477235
				8140 1010560.38595232
				8141 1010561.66713228
				8142 1010562.94831225
				8143 1010564.22949221
				8144 1010565.51067218
				8145 1010566.79185215
				8146 1010568.07303211
				8147 1010569.35421208
				8148 1010570.63539204
				8149 1010571.91657201
				8150 1010573.19775197
				8151 1010574.47893194
				8152 1010575.7601119
				8153 1010577.04129187
				8154 1010578.32247183
				8155 1010579.6036518
				8156 1010580.88483177
				8157 1010582.16601173
				8158 1010583.4471917
				8159 1010584.72837166
				8160 1010586.00955163
				8161 1010587.29073159
				8162 1010588.57191156
				8163 1010589.85309152
				8164 1010591.13427149
				8165 1010592.41545145
				8166 1010593.69663142
				8167 1010594.97781139
				8168 1010596.25899135
				8169 1010597.54017132
				8170 1010598.82135128
				8171 1010600.10253125
				8172 1010601.38371121
				8173 1010602.66489118
				8174 1010603.94607114
				8175 1010605.22725111
				8176 1010606.50843107
				8177 1010607.78961104
				8178 1010609.07079101
				8179 1010610.35197097
				8180 1010611.63315094
				8181 1010612.9143309
				8182 1010614.19551087
				8183 1010615.47669083
				8184 1010616.7578708
				8185 1010618.03905076
				8186 1010619.32023073
				8187 1010620.60141069
				8188 1010621.88259066
				8189 1010623.16377063
				8190 1010624.44495059
				8191 1010625.72613056
				8192 1010627.00731052
				8193 1010628.28849049
				8194 1010629.56967045
				8195 1010630.85085042
				8196 1010632.13203038
				8197 1010633.41321035
				8198 1010634.69439031
				8199 1010635.97557028
				8200 1010637.25675025
				8201 1010638.53793021
				8202 1010639.81911018
				8203 1010641.10029014
				8204 1010642.38147011
				8205 1010643.66265007
				8206 1010644.94383004
				8207 1010646.22501
				8208 1010647.50618997
				8209 1010648.78736993
				8210 1010650.0685499
				8211 1010651.34972987
				8212 1010652.63090983
				8213 1010653.9120898
				8214 1010655.19326976
				8215 1010656.47444973
				8216 1010657.75562969
				8217 1010659.03680966
				8218 1010660.31798962
				8219 1010661.59916959
				8220 1010662.88034955
				8221 1010664.16152952
				8222 1010665.44270949
				8223 1010666.72388945
				8224 1010668.00506942
				8225 1010669.28624938
				8226 1010670.56742935
				8227 1010671.84860931
				8228 1010673.12978928
				8229 1010674.41096924
				8230 1010675.69214921
				8231 1010676.97332917
				8232 1010678.25450914
				8233 1010679.53568911
				8234 1010680.81686907
				8235 1010682.09804904
				8236 1010683.379229
				8237 1010684.66040897
				8238 1010685.94158893
				8239 1010687.2227689
				8240 1010688.50394886
				8241 1010689.78512883
				8242 1010691.06630879
				8243 1010692.34748876
				8244 1010693.62866873
				8245 1010694.90984869
				8246 1010696.19102866
				8247 1010697.47220862
				8248 1010698.75338859
				8249 1010700.03456855
				8250 1010701.31574852
				8251 1010702.59692848
				8252 1010703.87810845
				8253 1010705.15928841
				8254 1010706.44046838
				8255 1010707.72164835
				8256 1010709.00282831
				8257 1010710.28400828
				8258 1010711.56518824
				8259 1010712.84636821
				8260 1010714.12754817
				8261 1010715.40872814
				8262 1010716.6899081
				8263 1010717.97108807
				8264 1010719.25226803
				8265 1010720.533448
				8266 1010721.81462797
				8267 1010723.09580793
				8268 1010724.3769879
				8269 1010725.65816786
				8270 1010726.93934783
				8271 1010728.22052779
				8272 1010729.50170776
				8273 1010730.78288772
				8274 1010732.06406769
				8275 1010733.34524765
				8276 1010734.62642762
				8277 1010735.90760759
				8278 1010737.18878755
				8279 1010738.46996752
				8280 1010739.75114748
				8281 1010741.03232745
				8282 1010742.31350741
				8283 1010743.59468738
				8284 1010744.87586734
				8285 1010746.15704731
				8286 1010747.43822727
				8287 1010748.71940724
				8288 1010750.00058721
				8289 1010751.28176717
				8290 1010752.56294714
				8291 1010753.8441271
				8292 1010755.12530707
				8293 1010756.40648703
				8294 1010757.687667
				8295 1010758.96884696
				8296 1010760.25002693
				8297 1010761.53120689
				8298 1010762.81238686
				8299 1010764.09356683
				8300 1010765.37474679
				8301 1010766.65592676
				8302 1010767.93710672
				8303 1010769.21828669
				8304 1010770.49946665
				8305 1010771.78064662
				8306 1010773.06182658
				8307 1010774.34300655
				8308 1010775.62418651
				8309 1010776.90536648
				8310 1010778.18654645
				8311 1010779.46772641
				8312 1010780.74890638
				8313 1010782.03008634
				8314 1010783.31126631
				8315 1010784.59244627
				8316 1010785.87362624
				8317 1010787.1548062
				8318 1010788.43598617
				8319 1010789.71716613
				8320 1010790.9983461
				8321 1010792.27952607
				8322 1010793.56070603
				8323 1010794.841886
				8324 1010796.12306596
				8325 1010797.40424593
				8326 1010798.68542589
				8327 1010799.96660586
				8328 1010801.24778582
				8329 1010802.52896579
				8330 1010803.81014575
				8331 1010805.09132572
				8332 1010806.37250569
				8333 1010807.65368565
				8334 1010808.93486562
				8335 1010810.21604558
				8336 1010811.49722555
				8337 1010812.77840551
				8338 1010814.05958548
				8339 1010815.34076544
				8340 1010816.62194541
				8341 1010817.90312537
				8342 1010819.18430534
				8343 1010820.46548531
				8344 1010821.74666527
				8345 1010823.02784524
				8346 1010824.3090252
				8347 1010825.59020517
				8348 1010826.87138513
				8349 1010828.1525651
				8350 1010829.43374506
				8351 1010830.71492503
				8352 1010831.99610499
				8353 1010833.27728496
				8354 1010834.55846493
				8355 1010835.83964489
				8356 1010837.12082486
				8357 1010838.40200482
				8358 1010839.68318479
				8359 1010840.96436475
				8360 1010842.24554472
				8361 1010843.52672468
				8362 1010844.80790465
				8363 1010846.08908461
				8364 1010847.37026458
				8365 1010848.65144455
				8366 1010849.93262451
				8367 1010851.21380448
				8368 1010852.49498444
				8369 1010853.77616441
				8370 1010855.05734437
				8371 1010856.33852434
				8372 1010857.6197043
				8373 1010858.90088427
				8374 1010860.18206423
				8375 1010861.4632442
				8376 1010862.74442417
				8377 1010864.02560413
				8378 1010865.3067841
				8379 1010866.58796406
				8380 1010867.86914403
				8381 1010869.15032399
				8382 1010870.43150396
				8383 1010871.71268392
				8384 1010872.99386389
				8385 1010874.27504385
				8386 1010875.55622382
				8387 1010876.83740379
				8388 1010878.11858375
				8389 1010879.39976372
				8390 1010880.68094368
				8391 1010881.96212365
				8392 1010883.24330361
				8393 1010884.52448358
				8394 1010885.80566354
				8395 1010887.08684351
				8396 1010888.36802347
				8397 1010889.64920344
				8398 1010890.93038341
				8399 1010892.21156337
				8400 1010893.49274334
				8401 1010894.7739233
				8402 1010896.05510327
				8403 1010897.33628323
				8404 1010898.6174632
				8405 1010899.89864316
				8406 1010901.17982313
				8407 1010902.46100309
				8408 1010903.74218306
				8409 1010905.02336303
				8410 1010906.30454299
				8411 1010907.58572296
				8412 1010908.86690292
				8413 1010910.14808289
				8414 1010911.42926285
				8415 1010912.71044282
				8416 1010913.99162278
				8417 1010915.27280275
				8418 1010916.55398271
				8419 1010917.83516268
				8420 1010919.11634265
				8421 1010920.39752261
				8422 1010921.67870258
				8423 1010922.95988254
				8424 1010924.24106251
				8425 1010925.52224247
				8426 1010926.80342244
				8427 1010928.0846024
				8428 1010929.36578237
				8429 1010930.64696233
				8430 1010931.9281423
				8431 1010933.20932227
				8432 1010934.49050223
				8433 1010935.7716822
				8434 1010937.05286216
				8435 1010938.33404213
				8436 1010939.61522209
				8437 1010940.89640206
				8438 1010942.17758202
				8439 1010943.45876199
				8440 1010944.73994195
				8441 1010946.02112192
				8442 1010947.30230189
				8443 1010948.58348185
				8444 1010949.86466182
				8445 1010951.14584178
				8446 1010952.42702175
				8447 1010953.70820171
				8448 1010954.98938168
				8449 1010956.27056164
				8450 1010957.55174161
				8451 1010958.83292157
				8452 1010960.11410154
				8453 1010961.39528151
				8454 1010962.67646147
				8455 1010963.95764144
				8456 1010965.2388214
				8457 1010966.52000137
				8458 1010967.80118133
				8459 1010969.0823613
				8460 1010970.36354126
				8461 1010971.64472123
				8462 1010972.92590119
				8463 1010974.20708116
				8464 1010975.48826113
				8465 1010976.76944109
				8466 1010978.05062106
				8467 1010979.33180102
				8468 1010980.61298099
				8469 1010981.89416095
				8470 1010983.17534092
				8471 1010984.45652088
				8472 1010985.73770085
				8473 1010987.01888081
				8474 1010988.30006078
				8475 1010989.58124075
				8476 1010990.86242071
				8477 1010992.14360068
				8478 1010993.42478064
				8479 1010994.70596061
				8480 1010995.98714057
				8481 1010997.26832054
				8482 1010998.5495005
				8483 1010999.83068047
				8484 1011001.11186043
				8485 1011002.3930404
				8486 1011003.67422037
				8487 1011004.95540033
				8488 1011006.2365803
				8489 1011007.51776026
				8490 1011008.79894023
				8491 1011010.08012019
				8492 1011011.36130016
				8493 1011012.64248012
				8494 1011013.92366009
				8495 1011015.20484005
				8496 1011016.48602002
				8497 1011017.76719999
				8498 1011019.04837995
				8499 1011020.32955992
				8500 1011021.61073988
				8501 1011022.89191985
				8502 1011024.17309981
				8503 1011025.45427978
				8504 1011026.73545974
				8505 1011028.01663971
				8506 1011029.29781967
				8507 1011030.57899964
				8508 1011031.86017961
				8509 1011033.14135957
				8510 1011034.42253954
				8511 1011035.7037195
				8512 1011036.98489947
				8513 1011038.26607943
				8514 1011039.5472594
				8515 1011040.82843936
				8516 1011042.10961933
				8517 1011043.39079929
				8518 1011044.67197926
				8519 1011045.95315923
				8520 1011047.23433919
				8521 1011048.51551916
				8522 1011049.79669912
				8523 1011051.07787909
				8524 1011052.35905905
				8525 1011053.64023902
				8526 1011054.92141898
				8527 1011056.20259895
				8528 1011057.48377891
				8529 1011058.76495888
				8530 1011060.04613885
				8531 1011061.32731881
				8532 1011062.60849878
				8533 1011063.88967874
				8534 1011065.17085871
				8535 1011066.45203867
				8536 1011067.73321864
				8537 1011069.0143986
				8538 1011070.29557857
				8539 1011071.57675853
				8540 1011072.8579385
				8541 1011074.13911847
				8542 1011075.42029843
				8543 1011076.7014784
				8544 1011077.98265836
				8545 1011079.26383833
				8546 1011080.54501829
				8547 1011081.82619826
				8548 1011083.10737822
				8549 1011084.38855819
				8550 1011085.66973815
				8551 1011086.95091812
				8552 1011088.23209809
				8553 1011089.51327805
				8554 1011090.79445802
				8555 1011092.07563798
				8556 1011093.35681795
				8557 1011094.63799791
				8558 1011095.91917788
				8559 1011097.20035784
				8560 1011098.48153781
				8561 1011099.76271777
				8562 1011101.04389774
				8563 1011102.32507771
				8564 1011103.60625767
				8565 1011104.88743764
				8566 1011106.1686176
				8567 1011107.44979757
				8568 1011108.73097753
				8569 1011110.0121575
				8570 1011111.29333746
				8571 1011112.57451743
				8572 1011113.85569739
				8573 1011115.13687736
				8574 1011116.41805733
				8575 1011117.69923729
				8576 1011118.98041726
				8577 1011120.26159722
				8578 1011121.54277719
				8579 1011122.82395715
				8580 1011124.10513712
				8581 1011125.38631708
				8582 1011126.66749705
				8583 1011127.94867701
				8584 1011129.22985698
				8585 1011130.51103695
				8586 1011131.79221691
				8587 1011133.07339688
				8588 1011134.35457684
				8589 1011135.63575681
				8590 1011136.91693677
				8591 1011138.19811674
				8592 1011139.4792967
				8593 1011140.76047667
				8594 1011142.04165663
				8595 1011143.3228366
				8596 1011144.60401657
				8597 1011145.88519653
				8598 1011147.1663765
				8599 1011148.44755646
				8600 1011149.72873643
				8601 1011151.00991639
				8602 1011152.29109636
				8603 1011153.57227632
				8604 1011154.85345629
				8605 1011156.13463625
				8606 1011157.41581622
				8607 1011158.69699619
				8608 1011159.97817615
				8609 1011161.25935612
				8610 1011162.54053608
				8611 1011163.82171605
				8612 1011165.10289601
				8613 1011166.38407598
				8614 1011167.66525594
				8615 1011168.94643591
				8616 1011170.22761587
				8617 1011171.50879584
				8618 1011172.78997581
				8619 1011174.07115577
				8620 1011175.35233574
				8621 1011176.6335157
				8622 1011177.91469567
				8623 1011179.19587563
				8624 1011180.4770556
				8625 1011181.75823556
				8626 1011183.03941553
				8627 1011184.32059549
				8628 1011185.60177546
				8629 1011186.88295543
				8630 1011188.16413539
				8631 1011189.44531536
				8632 1011190.72649532
				8633 1011192.00767529
				8634 1011193.28885525
				8635 1011194.57003522
				8636 1011195.85121518
				8637 1011197.13239515
				8638 1011198.41357511
				8639 1011199.69475508
				8640 1011200.97593505
				8641 1011202.25711501
				8642 1011203.53829498
				8643 1011204.81947494
				8644 1011206.10065491
				8645 1011207.38183487
				8646 1011208.66301484
				8647 1011209.9441948
				8648 1011211.22537477
				8649 1011212.50655473
				8650 1011213.7877347
				8651 1011215.06891467
				8652 1011216.35009463
				8653 1011217.6312746
				8654 1011218.91245456
				8655 1011220.19363453
				8656 1011221.47481449
				8657 1011222.75599446
				8658 1011224.03717442
				8659 1011225.31835439
				8660 1011226.59953435
				8661 1011227.88071432
				8662 1011229.16189429
				8663 1011230.44307425
				8664 1011231.72425422
				8665 1011233.00543418
				8666 1011234.28661415
				8667 1011235.56779411
				8668 1011236.84897408
				8669 1011238.13015404
				8670 1011239.41133401
				8671 1011240.69251397
				8672 1011241.97369394
				8673 1011243.2548739
				8674 1011244.53605387
				8675 1011245.81723384
				8676 1011247.0984138
				8677 1011248.37959377
				8678 1011249.66077373
				8679 1011250.9419537
				8680 1011252.22313366
				8681 1011253.50431363
				8682 1011254.78549359
				8683 1011256.06667356
				8684 1011257.34785353
				8685 1011258.62903349
				8686 1011259.91021346
				8687 1011261.19139342
				8688 1011262.47257339
				8689 1011263.75375335
				8690 1011265.03493332
				8691 1011266.31611328
				8692 1011267.59729325
				8693 1011268.87847321
				8694 1011270.15965318
				8695 1011271.44083315
				8696 1011272.72201311
				8697 1011274.00319308
				8698 1011275.28437304
				8699 1011276.56555301
				8700 1011277.84673297
				8701 1011279.12791294
				8702 1011280.4090929
				8703 1011281.69027287
				8704 1011282.97145283
				8705 1011284.2526328
				8706 1011285.53381277
				8707 1011286.81499273
				8708 1011288.0961727
				8709 1011289.37735266
				8710 1011290.65853263
				8711 1011291.93971259
				8712 1011293.22089256
				8713 1011294.50207252
				8714 1011295.78325249
				8715 1011297.06443245
				8716 1011298.34561242
				8717 1011299.62679239
				8718 1011300.90797235
				8719 1011302.18915232
				8720 1011303.47033228
				8721 1011304.75151225
				8722 1011306.03269221
				8723 1011307.31387218
				8724 1011308.59505214
				8725 1011309.87623211
				8726 1011311.15741207
				8727 1011312.43859204
				8728 1011313.719772
				8729 1011315.00095197
				8730 1011316.28213194
				8731 1011317.5633119
				8732 1011318.84449187
				8733 1011320.12567183
				8734 1011321.4068518
				8735 1011322.68803176
				8736 1011323.96921173
				8737 1011325.25039169
				8738 1011326.53157166
				8739 1011327.81275162
				8740 1011329.09393159
				8741 1011330.37511156
				8742 1011331.65629152
				8743 1011332.93747149
				8744 1011334.21865145
				8745 1011335.49983142
				8746 1011336.78101138
				8747 1011338.06219135
				8748 1011339.34337131
				8749 1011340.62455128
				8750 1011341.90573124
				8751 1011343.18691121
				8752 1011344.46809118
				8753 1011345.74927114
				8754 1011347.03045111
				8755 1011348.31163107
				8756 1011349.59281104
				8757 1011350.873991
				8758 1011352.15517097
				8759 1011353.43635093
				8760 1011354.7175309
				8761 1011355.99871086
				8762 1011357.27989083
				8763 1011358.5610708
				8764 1011359.84225076
				8765 1011361.12343073
				8766 1011362.40461069
				8767 1011363.68579066
				8768 1011364.96697062
				8769 1011366.24815059
				8770 1011367.52933055
				8771 1011368.81051052
				8772 1011370.09169049
				8773 1011371.37287045
				8774 1011372.65405042
				8775 1011373.93523038
				8776 1011375.21641035
				8777 1011376.49759031
				8778 1011377.77877028
				8779 1011379.05995024
				8780 1011380.34113021
				8781 1011381.62231017
				8782 1011382.90349014
				8783 1011384.1846701
				8784 1011385.46585007
				8785 1011386.74703004
				8786 1011388.02821
				8787 1011389.30938997
				8788 1011390.59056993
				8789 1011391.8717499
				8790 1011393.15292986
				8791 1011394.43410983
				8792 1011395.71528979
				8793 1011396.99646976
				8794 1011398.27764972
				8795 1011399.55882969
				8796 1011400.84000966
				8797 1011402.12118962
				8798 1011403.40236959
				8799 1011404.68354955
				8800 1011405.96472952
				8801 1011407.24590948
				8802 1011408.52708945
				8803 1011409.80826941
				8804 1011411.08944938
				8805 1011412.37062934
				8806 1011413.65180931
				8807 1011414.93298928
				8808 1011416.21416924
				8809 1011417.49534921
				8810 1011418.77652917
				8811 1011420.05770914
				8812 1011421.3388891
				8813 1011422.62006907
				8814 1011423.90124903
				8815 1011425.182429
				8816 1011426.46360896
				8817 1011427.74478893
				8818 1011429.0259689
				8819 1011430.30714886
				8820 1011431.58832883
				8821 1011432.86950879
				8822 1011434.15068876
				8823 1011435.43186872
				8824 1011436.71304869
				8825 1011437.99422865
				8826 1011439.27540862
				8827 1011440.55658858
				8828 1011441.83776855
				8829 1011443.11894852
				8830 1011444.40012848
				8831 1011445.68130845
				8832 1011446.96248841
				8833 1011448.24366838
				8834 1011449.52484834
				8835 1011450.80602831
				8836 1011452.08720827
				8837 1011453.36838824
				8838 1011454.6495682
				8839 1011455.93074817
				8840 1011457.21192814
				8841 1011458.4931081
				8842 1011459.77428807
				8843 1011461.05546803
				8844 1011462.336648
				8845 1011463.61782796
				8846 1011464.89900793
				8847 1011466.18018789
				8848 1011467.46136786
				8849 1011468.74254782
				8850 1011470.02372779
				8851 1011471.30490776
				8852 1011472.58608772
				8853 1011473.86726769
				8854 1011475.14844765
				8855 1011476.42962762
				8856 1011477.71080758
				8857 1011478.99198755
				8858 1011480.27316751
				8859 1011481.55434748
				8860 1011482.83552744
				8861 1011484.11670741
				8862 1011485.39788738
				8863 1011486.67906734
				8864 1011487.96024731
				8865 1011489.24142727
				8866 1011490.52260724
				8867 1011491.8037872
				8868 1011493.08496717
				8869 1011494.36614713
				8870 1011495.6473271
				8871 1011496.92850706
				8872 1011498.20968703
				8873 1011499.490867
				8874 1011500.77204696
				8875 1011502.05322693
				8876 1011503.33440689
				8877 1011504.61558686
				8878 1011505.89676682
				8879 1011507.17794679
				8880 1011508.45912675
				8881 1011509.74030672
				8882 1011511.02148668
				8883 1011512.30266665
				8884 1011513.58384662
				8885 1011514.86502658
				8886 1011516.14620655
				8887 1011517.42738651
				8888 1011518.70856648
				8889 1011519.98974644
				8890 1011521.27092641
				8891 1011522.55210637
				8892 1011523.83328634
				8893 1011525.1144663
				8894 1011526.39564627
				8895 1011527.67682624
				8896 1011528.9580062
				8897 1011530.23918617
				8898 1011531.52036613
				8899 1011532.8015461
				8900 1011534.08272606
				8901 1011535.36390603
				8902 1011536.64508599
				8903 1011537.92626596
				8904 1011539.20744592
				8905 1011540.48862589
				8906 1011541.76980586
				8907 1011543.05098582
				8908 1011544.33216579
				8909 1011545.61334575
				8910 1011546.89452572
				8911 1011548.17570568
				8912 1011549.45688565
				8913 1011550.73806561
				8914 1011552.01924558
				8915 1011553.30042554
				8916 1011554.58160551
				8917 1011555.86278548
				8918 1011557.14396544
				8919 1011558.42514541
				8920 1011559.70632537
				8921 1011560.98750534
				8922 1011562.2686853
				8923 1011563.54986527
				8924 1011564.83104523
				8925 1011566.1122252
				8926 1011567.39340516
				8927 1011568.67458513
				8928 1011569.9557651
				8929 1011571.23694506
				8930 1011572.51812503
				8931 1011573.79930499
				8932 1011575.08048496
				8933 1011576.36166492
				8934 1011577.64284489
				8935 1011578.92402485
				8936 1011580.20520482
				8937 1011581.48638478
				8938 1011582.76756475
				8939 1011584.04874472
				8940 1011585.32992468
				8941 1011586.61110465
				8942 1011587.89228461
				8943 1011589.17346458
				8944 1011590.45464454
				8945 1011591.73582451
				8946 1011593.01700447
				8947 1011594.29818444
				8948 1011595.5793644
				8949 1011596.86054437
				8950 1011598.14172434
				8951 1011599.4229043
				8952 1011600.70408427
				8953 1011601.98526423
				8954 1011603.2664442
				8955 1011604.54762416
				8956 1011605.82880413
				8957 1011607.10998409
				8958 1011608.39116406
				8959 1011609.67234402
				8960 1011610.95352399
				8961 1011612.23470396
				8962 1011613.51588392
				8963 1011614.79706389
				8964 1011616.07824385
				8965 1011617.35942382
				8966 1011618.64060378
				8967 1011619.92178375
				8968 1011621.20296371
				8969 1011622.48414368
				8970 1011623.76532364
				8971 1011625.04650361
				8972 1011626.32768358
				8973 1011627.60886354
				8974 1011628.89004351
				8975 1011630.17122347
				8976 1011631.45240344
				8977 1011632.7335834
				8978 1011634.01476337
				8979 1011635.29594333
				8980 1011636.5771233
				8981 1011637.85830326
				8982 1011639.13948323
				8983 1011640.4206632
				8984 1011641.70184316
				8985 1011642.98302313
				8986 1011644.26420309
				8987 1011645.54538306
				8988 1011646.82656302
				8989 1011648.10774299
				8990 1011649.38892295
				8991 1011650.67010292
				8992 1011651.95128288
				8993 1011653.23246285
				8994 1011654.51364282
				8995 1011655.79482278
				8996 1011657.07600275
				8997 1011658.35718271
				8998 1011659.63836268
				8999 1011660.91954264
				9000 1011662.20072261
				9001 1011663.48190257
				9002 1011664.76308254
				9003 1011666.0442625
				9004 1011667.32544247
				9005 1011668.60662244
				9006 1011669.8878024
				9007 1011671.16898237
				9008 1011672.45016233
				9009 1011673.7313423
				9010 1011675.01252226
				9011 1011676.29370223
				9012 1011677.57488219
				9013 1011678.85606216
				9014 1011680.13724212
				9015 1011681.41842209
				9016 1011682.69960206
				9017 1011683.98078202
				9018 1011685.26196199
				9019 1011686.54314195
				9020 1011687.82432192
				9021 1011689.10550188
				9022 1011690.38668185
				9023 1011691.66786181
				9024 1011692.94904178
				9025 1011694.23022174
				9026 1011695.51140171
				9027 1011696.79258168
				9028 1011698.07376164
				9029 1011699.35494161
				9030 1011700.63612157
				9031 1011701.91730154
				9032 1011703.1984815
				9033 1011704.47966147
				9034 1011705.76084143
				9035 1011707.0420214
				9036 1011708.32320136
				9037 1011709.60438133
				9038 1011710.8855613
				9039 1011712.16674126
				9040 1011713.44792123
				9041 1011714.72910119
				9042 1011716.01028116
				9043 1011717.29146112
				9044 1011718.57264109
				9045 1011719.85382105
				9046 1011721.13500102
				9047 1011722.41618098
				9048 1011723.69736095
				9049 1011724.97854092
				9050 1011726.25972088
				9051 1011727.54090085
				9052 1011728.82208081
				9053 1011730.10326078
				9054 1011731.38444074
				9055 1011732.66562071
				9056 1011733.94680067
				9057 1011735.22798064
				9058 1011736.5091606
				9059 1011737.79034057
				9060 1011739.07152054
				9061 1011740.3527005
				9062 1011741.63388047
				9063 1011742.91506043
				9064 1011744.1962404
				9065 1011745.47742036
				9066 1011746.75860033
				9067 1011748.03978029
				9068 1011749.32096026
				9069 1011750.60214022
				9070 1011751.88332019
				9071 1011753.16450016
				9072 1011754.44568012
				9073 1011755.72686009
				9074 1011757.00804005
				9075 1011758.28922002
				9076 1011759.57039998
				9077 1011760.85157995
				9078 1011762.13275991
				9079 1011763.41393988
				9080 1011764.69511984
				9081 1011765.97629981
				9082 1011767.25747978
				9083 1011768.53865974
				9084 1011769.81983971
				9085 1011771.10101967
				9086 1011772.38219964
				9087 1011773.6633796
				9088 1011774.94455957
				9089 1011776.22573953
				9090 1011777.5069195
				9091 1011778.78809946
				9092 1011780.06927943
				9093 1011781.3504594
				9094 1011782.63163936
				9095 1011783.91281933
				9096 1011785.19399929
				9097 1011786.47517926
				9098 1011787.75635922
				9099 1011789.03753919
				9100 1011790.31871915
				9101 1011791.59989912
				9102 1011792.88107908
				9103 1011794.16225905
				9104 1011795.44343902
				9105 1011796.72461898
				9106 1011798.00579895
				9107 1011799.28697891
				9108 1011800.56815888
				9109 1011801.84933884
				9110 1011803.13051881
				9111 1011804.41169877
				9112 1011805.69287874
				9113 1011806.9740587
				9114 1011808.25523867
				9115 1011809.53641864
				9116 1011810.8175986
				9117 1011812.09877857
				9118 1011813.37995853
				9119 1011814.6611385
				9120 1011815.94231846
				9121 1011817.22349843
				9122 1011818.50467839
				9123 1011819.78585836
				9124 1011821.06703832
				9125 1011822.34821829
				9126 1011823.62939826
				9127 1011824.91057822
				9128 1011826.19175819
				9129 1011827.47293815
				9130 1011828.75411812
				9131 1011830.03529808
				9132 1011831.31647805
				9133 1011832.59765801
				9134 1011833.87883798
				9135 1011835.16001794
				9136 1011836.44119791
				9137 1011837.72237788
				9138 1011839.00355784
				9139 1011840.28473781
				9140 1011841.56591777
				9141 1011842.84709774
				9142 1011844.1282777
				9143 1011845.40945767
				9144 1011846.69063763
				9145 1011847.9718176
				9146 1011849.25299756
				9147 1011850.53417753
				9148 1011851.8153575
				9149 1011853.09653746
				9150 1011854.37771743
				9151 1011855.65889739
				9152 1011856.94007736
				9153 1011858.22125732
				9154 1011859.50243729
				9155 1011860.78361725
				9156 1011862.06479722
				9157 1011863.34597718
				9158 1011864.62715715
				9159 1011865.90833712
				9160 1011867.18951708
				9161 1011868.47069705
				9162 1011869.75187701
				9163 1011871.03305698
				9164 1011872.31423694
				9165 1011873.59541691
				9166 1011874.87659687
				9167 1011876.15777684
				9168 1011877.4389568
				9169 1011878.72013677
				9170 1011880.00131674
				9171 1011881.2824967
				9172 1011882.56367667
				9173 1011883.84485663
				9174 1011885.1260366
				9175 1011886.40721656
				9176 1011887.68839653
				9177 1011888.96957649
				9178 1011890.25075646
				9179 1011891.53193642
				9180 1011892.81311639
				9181 1011894.09429636
				9182 1011895.37547632
				9183 1011896.65665629
				9184 1011897.93783625
				9185 1011899.21901622
				9186 1011900.50019618
				9187 1011901.78137615
				9188 1011903.06255611
				9189 1011904.34373608
				9190 1011905.62491604
				9191 1011906.90609601
				9192 1011908.18727598
				9193 1011909.46845594
				9194 1011910.74963591
				9195 1011912.03081587
				9196 1011913.31199584
				9197 1011914.5931758
				9198 1011915.87435577
				9199 1011917.15553573
				9200 1011918.4367157
				9201 1011919.71789566
				9202 1011920.99907563
				9203 1011922.2802556
				9204 1011923.56143556
				9205 1011924.84261553
				9206 1011926.12379549
				9207 1011927.40497546
				9208 1011928.68615542
				9209 1011929.96733539
				9210 1011931.24851535
				9211 1011932.52969532
				9212 1011933.81087528
				9213 1011935.09205525
				9214 1011936.37323522
				9215 1011937.65441518
				9216 1011938.93559515
				9217 1011940.21677511
				9218 1011941.49795508
				9219 1011942.77913504
				9220 1011944.06031501
				9221 1011945.34149497
				9222 1011946.62267494
				9223 1011947.9038549
				9224 1011949.18503487
				9225 1011950.46621484
				9226 1011951.7473948
				9227 1011953.02857477
				9228 1011954.30975473
				9229 1011955.5909347
				9230 1011956.87211466
				9231 1011958.15329463
				9232 1011959.43447459
				9233 1011960.71565456
				9234 1011961.99683452
				9235 1011963.27801449
				9236 1011964.55919446
				9237 1011965.84037442
				9238 1011967.12155439
				9239 1011968.40273435
				9240 1011969.68391432
				9241 1011970.96509428
				9242 1011972.24627425
				9243 1011973.52745421
				9244 1011974.80863418
				9245 1011976.08981414
				9246 1011977.37099411
				9247 1011978.65217408
				9248 1011979.93335404
				9249 1011981.21453401
				9250 1011982.49571397
				9251 1011983.77689394
				9252 1011985.0580739
				9253 1011986.33925387
				9254 1011987.62043383
				9255 1011988.9016138
				9256 1011990.18279376
				9257 1011991.46397373
				9258 1011992.7451537
				9259 1011994.02633366
				9260 1011995.30751363
				9261 1011996.58869359
				9262 1011997.86987356
				9263 1011999.15105352
				9264 1012000.43223349
				9265 1012001.71341345
				9266 1012002.99459342
				9267 1012004.27577338
				9268 1012005.55695335
				9269 1012006.83813332
				9270 1012008.11931328
				9271 1012009.40049325
				9272 1012010.68167321
				9273 1012011.96285318
				9274 1012013.24403314
				9275 1012014.52521311
				9276 1012015.80639307
				9277 1012017.08757304
				9278 1012018.368753
				9279 1012019.64993297
				9280 1012020.93111294
				9281 1012022.2122929
				9282 1012023.49347287
				9283 1012024.77465283
				9284 1012026.0558328
				9285 1012027.33701276
				9286 1012028.61819273
				9287 1012029.89937269
				9288 1012031.18055266
				9289 1012032.46173262
				9290 1012033.74291259
				9291 1012035.02409256
				9292 1012036.30527252
				9293 1012037.58645249
				9294 1012038.86763245
				9295 1012040.14881242
				9296 1012041.42999238
				9297 1012042.71117235
				9298 1012043.99235231
				9299 1012045.27353228
				9300 1012046.55471224
				9301 1012047.83589221
				9302 1012049.11707218
				9303 1012050.39825214
				9304 1012051.67943211
				9305 1012052.96061207
				9306 1012054.24179204
				9307 1012055.522972
				9308 1012056.80415197
				9309 1012058.08533193
				9310 1012059.3665119
				9311 1012060.64769186
				9312 1012061.92887183
				9313 1012063.2100518
				9314 1012064.49123176
				9315 1012065.77241173
				9316 1012067.05359169
				9317 1012068.33477166
				9318 1012069.61595162
				9319 1012070.89713159
				9320 1012072.17831155
				9321 1012073.45949152
				9322 1012074.74067148
				9323 1012076.02185145
				9324 1012077.30303142
				9325 1012078.58421138
				9326 1012079.86539135
				9327 1012081.14657131
				9328 1012082.42775128
				9329 1012083.70893124
				9330 1012084.99011121
				9331 1012086.27129117
				9332 1012087.55247114
				9333 1012088.8336511
				9334 1012090.11483107
				9335 1012091.39601104
				9336 1012092.677191
				9337 1012093.95837097
				9338 1012095.23955093
				9339 1012096.5207309
				9340 1012097.80191086
				9341 1012099.08309083
				9342 1012100.36427079
				9343 1012101.64545076
				9344 1012102.92663072
				9345 1012104.20781069
				9346 1012105.48899066
				9347 1012106.77017062
				9348 1012108.05135059
				9349 1012109.33253055
				9350 1012110.61371052
				9351 1012111.89489048
				9352 1012113.17607045
				9353 1012114.45725041
				9354 1012115.73843038
				9355 1012117.01961034
				9356 1012118.30079031
				9357 1012119.58197028
				9358 1012120.86315024
				9359 1012122.14433021
				9360 1012123.42551017
				9361 1012124.70669014
				9362 1012125.9878701
				9363 1012127.26905007
				9364 1012128.55023003
				9365 1012129.83141
				9366 1012131.11258996
				9367 1012132.39376993
				9368 1012133.6749499
				9369 1012134.95612986
				9370 1012136.23730983
				9371 1012137.51848979
				9372 1012138.79966976
				9373 1012140.08084972
				9374 1012141.36202969
				9375 1012142.64320965
				9376 1012143.92438962
				9377 1012145.20556958
				9378 1012146.48674955
				9379 1012147.76792952
				9380 1012149.04910948
				9381 1012150.33028945
				9382 1012151.61146941
				9383 1012152.89264938
				9384 1012154.17382934
				9385 1012155.45500931
				9386 1012156.73618927
				9387 1012158.01736924
				9388 1012159.2985492
				9389 1012160.57972917
				9390 1012161.86090914
				9391 1012163.1420891
				9392 1012164.42326907
				9393 1012165.70444903
				9394 1012166.985629
				9395 1012168.26680896
				9396 1012169.54798893
				9397 1012170.82916889
				9398 1012172.11034886
				9399 1012173.39152882
				9400 1012174.67270879
				9401 1012175.95388876
				9402 1012177.23506872
				9403 1012178.51624869
				9404 1012179.79742865
				9405 1012181.07860862
				9406 1012182.35978858
				9407 1012183.64096855
				9408 1012184.92214851
				9409 1012186.20332848
				9410 1012187.48450844
				9411 1012188.76568841
				9412 1012190.04686838
				9413 1012191.32804834
				9414 1012192.60922831
				9415 1012193.89040827
				9416 1012195.17158824
				9417 1012196.4527682
				9418 1012197.73394817
				9419 1012199.01512813
				9420 1012200.2963081
				9421 1012201.57748806
				9422 1012202.85866803
				9423 1012204.139848
				9424 1012205.42102796
				9425 1012206.70220793
				9426 1012207.98338789
				9427 1012209.26456786
				9428 1012210.54574782
				9429 1012211.82692779
				9430 1012213.10810775
				9431 1012214.38928772
				9432 1012215.67046768
				9433 1012216.95164765
				9434 1012218.23282762
				9435 1012219.51400758
				9436 1012220.79518755
				9437 1012222.07636751
				9438 1012223.35754748
				9439 1012224.63872744
				9440 1012225.91990741
				9441 1012227.20108737
				9442 1012228.48226734
				9443 1012229.7634473
				9444 1012231.04462727
				9445 1012232.32580724
				9446 1012233.6069872
				9447 1012234.88816717
				9448 1012236.16934713
				9449 1012237.4505271
				9450 1012238.73170706
				9451 1012240.01288703
				9452 1012241.29406699
				9453 1012242.57524696
				9454 1012243.85642692
				9455 1012245.13760689
				9456 1012246.41878686
				9457 1012247.69996682
				9458 1012248.98114679
				9459 1012250.26232675
				9460 1012251.54350672
				9461 1012252.82468668
				9462 1012254.10586665
				9463 1012255.38704661
				9464 1012256.66822658
				9465 1012257.94940654
				9466 1012259.23058651
				9467 1012260.51176648
				9468 1012261.79294644
				9469 1012263.07412641
				9470 1012264.35530637
				9471 1012265.63648634
				9472 1012266.9176663
				9473 1012268.19884627
				9474 1012269.48002623
				9475 1012270.7612062
				9476 1012272.04238616
				9477 1012273.32356613
				9478 1012274.6047461
				9479 1012275.88592606
				9480 1012277.16710603
				9481 1012278.44828599
				9482 1012279.72946596
				9483 1012281.01064592
				9484 1012282.29182589
				9485 1012283.57300585
				9486 1012284.85418582
				9487 1012286.13536578
				9488 1012287.41654575
				9489 1012288.69772572
				9490 1012289.97890568
				9491 1012291.26008565
				9492 1012292.54126561
				9493 1012293.82244558
				9494 1012295.10362554
				9495 1012296.38480551
				9496 1012297.66598547
				9497 1012298.94716544
				9498 1012300.2283454
				9499 1012301.50952537
				9500 1012302.79070534
				9501 1012304.0718853
				9502 1012305.35306527
				9503 1012306.63424523
				9504 1012307.9154252
				9505 1012309.19660516
				9506 1012310.47778513
				9507 1012311.75896509
				9508 1012313.04014506
				9509 1012314.32132502
				9510 1012315.60250499
				9511 1012316.88368496
				9512 1012318.16486492
				9513 1012319.44604489
				9514 1012320.72722485
				9515 1012322.00840482
				9516 1012323.28958478
				9517 1012324.57076475
				9518 1012325.85194471
				9519 1012327.13312468
				9520 1012328.41430464
				9521 1012329.69548461
				9522 1012330.97666458
				9523 1012332.25784454
				9524 1012333.53902451
				9525 1012334.82020447
				9526 1012336.10138444
				9527 1012337.3825644
				9528 1012338.66374437
				9529 1012339.94492433
				9530 1012341.2261043
				9531 1012342.50728426
				9532 1012343.78846423
				9533 1012345.0696442
				9534 1012346.35082416
				9535 1012347.63200413
				9536 1012348.91318409
				9537 1012350.19436406
				9538 1012351.47554402
				9539 1012352.75672399
				9540 1012354.03790395
				9541 1012355.31908392
				9542 1012356.60026388
				9543 1012357.88144385
				9544 1012359.16262382
				9545 1012360.44380378
				9546 1012361.72498375
				9547 1012363.00616371
				9548 1012364.28734368
				9549 1012365.56852364
				9550 1012366.84970361
				9551 1012368.13088357
				9552 1012369.41206354
				9553 1012370.6932435
				9554 1012371.97442347
				9555 1012373.25560344
				9556 1012374.5367834
				9557 1012375.81796337
				9558 1012377.09914333
				9559 1012378.3803233
				9560 1012379.66150326
				9561 1012380.94268323
				9562 1012382.22386319
				9563 1012383.50504316
				9564 1012384.78622312
				9565 1012386.06740309
				9566 1012387.34858306
				9567 1012388.62976302
				9568 1012389.91094299
				9569 1012391.19212295
				9570 1012392.47330292
				9571 1012393.75448288
				9572 1012395.03566285
				9573 1012396.31684281
				9574 1012397.59802278
				9575 1012398.87920274
				9576 1012400.16038271
				9577 1012401.44156268
				9578 1012402.72274264
				9579 1012404.00392261
				9580 1012405.28510257
				9581 1012406.56628254
				9582 1012407.8474625
				9583 1012409.12864247
				9584 1012410.40982243
				9585 1012411.6910024
				9586 1012412.97218236
				9587 1012414.25336233
				9588 1012415.5345423
				9589 1012416.81572226
				9590 1012418.09690223
				9591 1012419.37808219
				9592 1012420.65926216
				9593 1012421.94044212
				9594 1012423.22162209
				9595 1012424.50280205
				9596 1012425.78398202
				9597 1012427.06516198
				9598 1012428.34634195
				9599 1012429.62752192
				9600 1012430.90870188
				9601 1012432.18988185
				9602 1012433.47106181
				9603 1012434.75224178
				9604 1012436.03342174
				9605 1012437.31460171
				9606 1012438.59578167
				9607 1012439.87696164
				9608 1012441.1581416
				9609 1012442.43932157
				9610 1012443.72050154
				9611 1012445.0016815
				9612 1012446.28286147
				9613 1012447.56404143
				9614 1012448.8452214
				9615 1012450.12640136
				9616 1012451.40758133
				9617 1012452.68876129
				9618 1012453.96994126
				9619 1012455.25112122
				9620 1012456.53230119
				9621 1012457.81348116
				9622 1012459.09466112
				9623 1012460.37584109
				9624 1012461.65702105
				9625 1012462.93820102
				9626 1012464.21938098
				9627 1012465.50056095
				9628 1012466.78174091
				9629 1012468.06292088
				9630 1012469.34410084
				9631 1012470.62528081
				9632 1012471.90646078
				9633 1012473.18764074
				9634 1012474.46882071
				9635 1012475.75000067
				9636 1012477.03118064
				9637 1012478.3123606
				9638 1012479.59354057
				9639 1012480.87472053
				9640 1012482.1559005
				9641 1012483.43708046
				9642 1012484.71826043
				9643 1012485.9994404
				9644 1012487.28062036
				9645 1012488.56180033
				9646 1012489.84298029
				9647 1012491.12416026
				9648 1012492.40534022
				9649 1012493.68652019
				9650 1012494.96770015
				9651 1012496.24888012
				9652 1012497.53006008
				9653 1012498.81124005
				9654 1012500.09242002
				9655 1012501.37359998
				9656 1012502.65477995
				9657 1012503.93595991
				9658 1012505.21713988
				9659 1012506.49831984
				9660 1012507.77949981
				9661 1012509.06067977
				9662 1012510.34185974
				9663 1012511.6230397
				9664 1012512.90421967
				9665 1012514.18539964
				9666 1012515.4665796
				9667 1012516.74775957
				9668 1012518.02893953
				9669 1012519.3101195
				9670 1012520.59129946
				9671 1012521.87247943
				9672 1012523.15365939
				9673 1012524.43483936
				9674 1012525.71601932
				9675 1012526.99719929
				9676 1012528.27837926
				9677 1012529.55955922
				9678 1012530.84073919
				9679 1012532.12191915
				9680 1012533.40309912
				9681 1012534.68427908
				9682 1012535.96545905
				9683 1012537.24663901
				9684 1012538.52781898
				9685 1012539.80899894
				9686 1012541.09017891
				9687 1012542.37135888
				9688 1012543.65253884
				9689 1012544.93371881
				9690 1012546.21489877
				9691 1012547.49607874
				9692 1012548.7772587
				9693 1012550.05843867
				9694 1012551.33961863
				9695 1012552.6207986
				9696 1012553.90197856
				9697 1012555.18315853
				9698 1012556.4643385
				9699 1012557.74551846
				9700 1012559.02669843
				9701 1012560.30787839
				9702 1012561.58905836
				9703 1012562.87023832
				9704 1012564.15141829
				9705 1012565.43259825
				9706 1012566.71377822
				9707 1012567.99495818
				9708 1012569.27613815
				9709 1012570.55731812
				9710 1012571.83849808
				9711 1012573.11967805
				9712 1012574.40085801
				9713 1012575.68203798
				9714 1012576.96321794
				9715 1012578.24439791
				9716 1012579.52557787
				9717 1012580.80675784
				9718 1012582.0879378
				9719 1012583.36911777
				9720 1012584.65029774
				9721 1012585.9314777
				9722 1012587.21265767
				9723 1012588.49383763
				9724 1012589.7750176
				9725 1012591.05619756
				9726 1012592.33737753
				9727 1012593.61855749
				9728 1012594.89973746
				9729 1012596.18091742
				9730 1012597.46209739
				9731 1012598.74327736
				9732 1012600.02445732
				9733 1012601.30563729
				9734 1012602.58681725
				9735 1012603.86799722
				9736 1012605.14917718
				9737 1012606.43035715
				9738 1012607.71153711
				9739 1012608.99271708
				9740 1012610.27389704
				9741 1012611.55507701
				9742 1012612.83625698
				9743 1012614.11743694
				9744 1012615.39861691
				9745 1012616.67979687
				9746 1012617.96097684
				9747 1012619.2421568
				9748 1012620.52333677
				9749 1012621.80451673
				9750 1012623.0856967
				9751 1012624.36687666
				9752 1012625.64805663
				9753 1012626.9292366
				9754 1012628.21041656
				9755 1012629.49159653
				9756 1012630.77277649
				9757 1012632.05395646
				9758 1012633.33513642
				9759 1012634.61631639
				9760 1012635.89749635
				9761 1012637.17867632
				9762 1012638.45985628
				9763 1012639.74103625
				9764 1012641.02221622
				9765 1012642.30339618
				9766 1012643.58457615
				9767 1012644.86575611
				9768 1012646.14693608
				9769 1012647.42811604
				9770 1012648.70929601
				9771 1012649.99047597
				9772 1012651.27165594
				9773 1012652.5528359
				9774 1012653.83401587
				9775 1012655.11519584
				9776 1012656.3963758
				9777 1012657.67755577
				9778 1012658.95873573
				9779 1012660.2399157
				9780 1012661.52109566
				9781 1012662.80227563
				9782 1012664.08345559
				9783 1012665.36463556
				9784 1012666.64581552
				9785 1012667.92699549
				9786 1012669.20817546
				9787 1012670.48935542
				9788 1012671.77053539
				9789 1012673.05171535
				9790 1012674.33289532
				9791 1012675.61407528
				9792 1012676.89525525
				9793 1012678.17643521
				9794 1012679.45761518
				9795 1012680.73879514
				9796 1012682.01997511
				9797 1012683.30115508
				9798 1012684.58233504
				9799 1012685.86351501
				9800 1012687.14469497
				9801 1012688.42587494
				9802 1012689.7070549
				9803 1012690.98823487
				9804 1012692.26941483
				9805 1012693.5505948
				9806 1012694.83177476
				9807 1012696.11295473
				9808 1012697.3941347
				9809 1012698.67531466
				9810 1012699.95649463
				9811 1012701.23767459
				9812 1012702.51885456
				9813 1012703.80003452
				9814 1012705.08121449
				9815 1012706.36239445
				9816 1012707.64357442
				9817 1012708.92475438
				9818 1012710.20593435
				9819 1012711.48711432
				9820 1012712.76829428
				9821 1012714.04947425
				9822 1012715.33065421
				9823 1012716.61183418
				9824 1012717.89301414
				9825 1012719.17419411
				9826 1012720.45537407
				9827 1012721.73655404
				9828 1012723.017734
				9829 1012724.29891397
				9830 1012725.58009394
				9831 1012726.8612739
				9832 1012728.14245387
				9833 1012729.42363383
				9834 1012730.7048138
				9835 1012731.98599376
				9836 1012733.26717373
				9837 1012734.54835369
				9838 1012735.82953366
				9839 1012737.11071362
				9840 1012738.39189359
				9841 1012739.67307356
				9842 1012740.95425352
				9843 1012742.23543349
				9844 1012743.51661345
				9845 1012744.79779342
				9846 1012746.07897338
				9847 1012747.36015335
				9848 1012748.64133331
				9849 1012749.92251328
				9850 1012751.20369324
				9851 1012752.48487321
				9852 1012753.76605318
				9853 1012755.04723314
				9854 1012756.32841311
				9855 1012757.60959307
				9856 1012758.89077304
				9857 1012760.171953
				9858 1012761.45313297
				9859 1012762.73431293
				9860 1012764.0154929
				9861 1012765.29667286
				9862 1012766.57785283
				9863 1012767.8590328
				9864 1012769.14021276
				9865 1012770.42139273
				9866 1012771.70257269
				9867 1012772.98375266
				9868 1012774.26493262
				9869 1012775.54611259
				9870 1012776.82729255
				9871 1012778.10847252
				9872 1012779.38965248
				9873 1012780.67083245
				9874 1012781.95201242
				9875 1012783.23319238
				9876 1012784.51437235
				9877 1012785.79555231
				9878 1012787.07673228
				9879 1012788.35791224
				9880 1012789.63909221
				9881 1012790.92027217
				9882 1012792.20145214
				9883 1012793.4826321
				9884 1012794.76381207
				9885 1012796.04499204
				9886 1012797.326172
				9887 1012798.60735197
				9888 1012799.88853193
				9889 1012801.1697119
				9890 1012802.45089186
				9891 1012803.73207183
				9892 1012805.01325179
				9893 1012806.29443176
				9894 1012807.57561172
				9895 1012808.85679169
				9896 1012810.13797166
				9897 1012811.41915162
				9898 1012812.70033159
				9899 1012813.98151155
				9900 1012815.26269152
				9901 1012816.54387148
				9902 1012817.82505145
				9903 1012819.10623141
				9904 1012820.38741138
				9905 1012821.66859134
				9906 1012822.94977131
				9907 1012824.23095128
				9908 1012825.51213124
				9909 1012826.79331121
				9910 1012828.07449117
				9911 1012829.35567114
				9912 1012830.6368511
				9913 1012831.91803107
				9914 1012833.19921103
				9915 1012834.480391
				9916 1012835.76157096
				9917 1012837.04275093
				9918 1012838.3239309
				9919 1012839.60511086
				9920 1012840.88629083
				9921 1012842.16747079
				9922 1012843.44865076
				9923 1012844.72983072
				9924 1012846.01101069
				9925 1012847.29219065
				9926 1012848.57337062
				9927 1012849.85455058
				9928 1012851.13573055
				9929 1012852.41691052
				9930 1012853.69809048
				9931 1012854.97927045
				9932 1012856.26045041
				9933 1012857.54163038
				9934 1012858.82281034
				9935 1012860.10399031
				9936 1012861.38517027
				9937 1012862.66635024
				9938 1012863.9475302
				9939 1012865.22871017
				9940 1012866.50989014
				9941 1012867.7910701
				9942 1012869.07225007
				9943 1012870.35343003
				9944 1012871.63461
				9945 1012872.91578996
				9946 1012874.19696993
				9947 1012875.47814989
				9948 1012876.75932986
				9949 1012878.04050982
				9950 1012879.32168979
				9951 1012880.60286976
				9952 1012881.88404972
				9953 1012883.16522969
				9954 1012884.44640965
				9955 1012885.72758962
				9956 1012887.00876958
				9957 1012888.28994955
				9958 1012889.57112951
				9959 1012890.85230948
				9960 1012892.13348944
				9961 1012893.41466941
				9962 1012894.69584937
				9963 1012895.97702934
				9964 1012897.25820931
				9965 1012898.53938927
				9966 1012899.82056924
				9967 1012901.1017492
				9968 1012902.38292917
				9969 1012903.66410913
				9970 1012904.9452891
				9971 1012906.22646906
				9972 1012907.50764903
				9973 1012908.78882899
				9974 1012910.07000896
				9975 1012911.35118893
				9976 1012912.63236889
				9977 1012913.91354886
				9978 1012915.19472882
				9979 1012916.47590879
				9980 1012917.75708875
				9981 1012919.03826872
				9982 1012920.31944868
				9983 1012921.60062865
				9984 1012922.88180861
				9985 1012924.16298858
				9986 1012925.44416855
				9987 1012926.72534851
				9988 1012928.00652848
				9989 1012929.28770844
				9990 1012930.56888841
				9991 1012931.85006837
				9992 1012933.13124834
				9993 1012934.4124283
				9994 1012935.69360827
				9995 1012936.97478824
				9996 1012938.2559682
				9997 1012939.53714817
				9998 1012940.81832813
				9999 1012942.0995081
				10000 1012943.38068806
				10001 1012944.66186803
				10002 1012945.94304799
				10003 1012947.22422796
				10004 1012948.50540792
				10005 1012949.78658789
				10006 1012951.06776786
				10007 1012952.34894782
				10008 1012953.63012779
				10009 1012954.91130775
				10010 1012956.19248772
				10011 1012957.47366768
				10012 1012958.75484765
				10013 1012960.03602761
				10014 1012961.31720758
				10015 1012962.59838754
				10016 1012963.87956751
				10017 1012965.16074747
				10018 1012966.44192744
				10019 1012967.72310741
				10020 1012969.00428737
				10021 1012970.28546734
				10022 1012971.5666473
				10023 1012972.84782727
				10024 1012974.12900723
				10025 1012975.4101872
				10026 1012976.69136716
				10027 1012977.97254713
				10028 1012979.25372709
				10029 1012980.53490706
				10030 1012981.81608703
				10031 1012983.09726699
				10032 1012984.37844696
				10033 1012985.65962692
				10034 1012986.94080689
				10035 1012988.22198685
				10036 1012989.50316682
				10037 1012990.78434678
				10038 1012992.06552675
				10039 1012993.34670671
				10040 1012994.62788668
				10041 1012995.90906665
				10042 1012997.19024661
				10043 1012998.47142658
				10044 1012999.75260654
				10045 1013001.03378651
				10046 1013002.31496647
				10047 1013003.59614644
				10048 1013004.8773264
				10049 1013006.15850637
				10050 1013007.43968633
				10051 1013008.7208663
				10052 1013010.00204627
				10053 1013011.28322623
				10054 1013012.5644062
				10055 1013013.84558616
				10056 1013015.12676613
				10057 1013016.40794609
				10058 1013017.68912606
				10059 1013018.97030602
				10060 1013020.25148599
				10061 1013021.53266595
				10062 1013022.81384592
				10063 1013024.09502589
				10064 1013025.37620585
				10065 1013026.65738582
				10066 1013027.93856578
				10067 1013029.21974575
				10068 1013030.50092571
				10069 1013031.78210568
				10070 1013033.06328564
				10071 1013034.34446561
				10072 1013035.62564557
				10073 1013036.90682554
				10074 1013038.18800551
				10075 1013039.46918547
				10076 1013040.75036544
				10077 1013042.0315454
				10078 1013043.31272537
				10079 1013044.59390533
				10080 1013045.8750853
				10081 1013047.15626526
				10082 1013048.43744523
				10083 1013049.71862519
				10084 1013050.99980516
				10085 1013052.28098513
				10086 1013053.56216509
				10087 1013054.84334506
				10088 1013056.12452502
				10089 1013057.40570499
				10090 1013058.68688495
				10091 1013059.96806492
				10092 1013061.24924488
				10093 1013062.53042485
				10094 1013063.81160481
				10095 1013065.09278478
				10096 1013066.37396475
				10097 1013067.65514471
				10098 1013068.93632468
				10099 1013070.21750464
				10100 1013071.49868461
				10101 1013072.77986457
				10102 1013074.06104454
				10103 1013075.3422245
				10104 1013076.62340447
				10105 1013077.90458443
				10106 1013079.1857644
				10107 1013080.46694437
				10108 1013081.74812433
				10109 1013083.0293043
				10110 1013084.31048426
				10111 1013085.59166423
				10112 1013086.87284419
				10113 1013088.15402416
				10114 1013089.43520412
				10115 1013090.71638409
				10116 1013091.99756405
				10117 1013093.27874402
				10118 1013094.55992399
				10119 1013095.84110395
				10120 1013097.12228392
				10121 1013098.40346388
				10122 1013099.68464385
				10123 1013100.96582381
				10124 1013102.24700378
				10125 1013103.52818374
				10126 1013104.80936371
				10127 1013106.09054367
				10128 1013107.37172364
				10129 1013108.65290361
				10130 1013109.93408357
				10131 1013111.21526354
				10132 1013112.4964435
				10133 1013113.77762347
				10134 1013115.05880343
				10135 1013116.3399834
				10136 1013117.62116336
				10137 1013118.90234333
				10138 1013120.18352329
				10139 1013121.46470326
				10140 1013122.74588323
				10141 1013124.02706319
				10142 1013125.30824316
				10143 1013126.58942312
				10144 1013127.87060309
				10145 1013129.15178305
				10146 1013130.43296302
				10147 1013131.71414298
				10148 1013132.99532295
				10149 1013134.27650291
				10150 1013135.55768288
				10151 1013136.83886285
				10152 1013138.12004281
				10153 1013139.40122278
				10154 1013140.68240274
				10155 1013141.96358271
				10156 1013143.24476267
				10157 1013144.52594264
				10158 1013145.8071226
				10159 1013147.08830257
				10160 1013148.36948253
				10161 1013149.6506625
				10162 1013150.93184247
				10163 1013152.21302243
				10164 1013153.4942024
				10165 1013154.77538236
				10166 1013156.05656233
				10167 1013157.33774229
				10168 1013158.61892226
				10169 1013159.90010222
				10170 1013161.18128219
				10171 1013162.46246215
				10172 1013163.74364212
				10173 1013165.02482209
				10174 1013166.30600205
				10175 1013167.58718202
				10176 1013168.86836198
				10177 1013170.14954195
				10178 1013171.43072191
				10179 1013172.71190188
				10180 1013173.99308184
				10181 1013175.27426181
				10182 1013176.55544177
				10183 1013177.83662174
				10184 1013179.11780171
				10185 1013180.39898167
				10186 1013181.68016164
				10187 1013182.9613416
				10188 1013184.24252157
				10189 1013185.52370153
				10190 1013186.8048815
				10191 1013188.08606146
				10192 1013189.36724143
				10193 1013190.64842139
				10194 1013191.92960136
				10195 1013193.21078133
				10196 1013194.49196129
				10197 1013195.77314126
				10198 1013197.05432122
				10199 1013198.33550119
				10200 1013199.61668115
				10201 1013200.89786112
				10202 1013202.17904108
				10203 1013203.46022105
				10204 1013204.74140101
				10205 1013206.02258098
				10206 1013207.30376095
				10207 1013208.58494091
				10208 1013209.86612088
				10209 1013211.14730084
				10210 1013212.42848081
				10211 1013213.70966077
				10212 1013214.99084074
				10213 1013216.2720207
				10214 1013217.55320067
				10215 1013218.83438063
				10216 1013220.1155606
				10217 1013221.39674057
				10218 1013222.67792053
				10219 1013223.9591005
				10220 1013225.24028046
				10221 1013226.52146043
				10222 1013227.80264039
				10223 1013229.08382036
				10224 1013230.36500032
				10225 1013231.64618029
				10226 1013232.92736025
				10227 1013234.20854022
				10228 1013235.48972019
				10229 1013236.77090015
				10230 1013238.05208012
				10231 1013239.33326008
				10232 1013240.61444005
				10233 1013241.89562001
				10234 1013243.17679998
				10235 1013244.45797994
				10236 1013245.73915991
				10237 1013247.02033987
				10238 1013248.30151984
				10239 1013249.58269981
				10240 1013250.86387977
				10241 1013252.14505974
				10242 1013253.4262397
				10243 1013254.70741967
				10244 1013255.98859963
				10245 1013257.2697796
				10246 1013258.55095956
				10247 1013259.83213953
				10248 1013261.11331949
				10249 1013262.39449946
				10250 1013263.67567943
				10251 1013264.95685939
				10252 1013266.23803936
				10253 1013267.51921932
				10254 1013268.80039929
				10255 1013270.08157925
				10256 1013271.36275922
				10257 1013272.64393918
				10258 1013273.92511915
				10259 1013275.20629911
				10260 1013276.48747908
				10261 1013277.76865905
				10262 1013279.04983901
				10263 1013280.33101898
				10264 1013281.61219894
				10265 1013282.89337891
				10266 1013284.17455887
				10267 1013285.45573884
				10268 1013286.7369188
				10269 1013288.01809877
				10270 1013289.29927873
				10271 1013290.5804587
				10272 1013291.86163867
				10273 1013293.14281863
				10274 1013294.4239986
				10275 1013295.70517856
				10276 1013296.98635853
				10277 1013298.26753849
				10278 1013299.54871846
				10279 1013300.82989842
				10280 1013302.11107839
				10281 1013303.39225835
				10282 1013304.67343832
				10283 1013305.95461829
				10284 1013307.23579825
				10285 1013308.51697822
				10286 1013309.79815818
				10287 1013311.07933815
				10288 1013312.36051811
				10289 1013313.64169808
				10290 1013314.92287804
				10291 1013316.20405801
				10292 1013317.48523797
				10293 1013318.76641794
				10294 1013320.04759791
				10295 1013321.32877787
				10296 1013322.60995784
				10297 1013323.8911378
				10298 1013325.17231777
				10299 1013326.45349773
				10300 1013327.7346777
				10301 1013329.01585766
				10302 1013330.29703763
				10303 1013331.57821759
				10304 1013332.85939756
				10305 1013334.14057753
				10306 1013335.42175749
				10307 1013336.70293746
				10308 1013337.98411742
				10309 1013339.26529739
				10310 1013340.54647735
				10311 1013341.82765732
				10312 1013343.10883728
				10313 1013344.39001725
				10314 1013345.67119721
				10315 1013346.95237718
				10316 1013348.23355715
				10317 1013349.51473711
				10318 1013350.79591708
				10319 1013352.07709704
				10320 1013353.35827701
				10321 1013354.63945697
				10322 1013355.92063694
				10323 1013357.2018169
				10324 1013358.48299687
				10325 1013359.76417683
				10326 1013361.0453568
				10327 1013362.32653677
				10328 1013363.60771673
				10329 1013364.8888967
				10330 1013366.17007666
				10331 1013367.45125663
				10332 1013368.73243659
				10333 1013370.01361656
				10334 1013371.29479652
				10335 1013372.57597649
				10336 1013373.85715645
				10337 1013375.13833642
				10338 1013376.41951639
				10339 1013377.70069635
				10340 1013378.98187632
				10341 1013380.26305628
				10342 1013381.54423625
				10343 1013382.82541621
				10344 1013384.10659618
				10345 1013385.38777614
				10346 1013386.66895611
				10347 1013387.95013607
				10348 1013389.23131604
				10349 1013390.51249601
				10350 1013391.79367597
				10351 1013393.07485594
				10352 1013394.3560359
				10353 1013395.63721587
				10354 1013396.91839583
				10355 1013398.1995758
				10356 1013399.48075576
				10357 1013400.76193573
				10358 1013402.04311569
				10359 1013403.32429566
				10360 1013404.60547563
				10361 1013405.88665559
				10362 1013407.16783556
				10363 1013408.44901552
				10364 1013409.73019549
				10365 1013411.01137545
				10366 1013412.29255542
				10367 1013413.57373538
				10368 1013414.85491535
				10369 1013416.13609531
				10370 1013417.41727528
				10371 1013418.69845525
				10372 1013419.97963521
				10373 1013421.26081518
				10374 1013422.54199514
				10375 1013423.82317511
				10376 1013425.10435507
				10377 1013426.38553504
				10378 1013427.666715
				10379 1013428.94789497
				10380 1013430.22907493
				10381 1013431.5102549
				10382 1013432.79143487
				10383 1013434.07261483
				10384 1013435.3537948
				10385 1013436.63497476
				10386 1013437.91615473
				10387 1013439.19733469
				10388 1013440.47851466
				10389 1013441.75969462
				10390 1013443.04087459
				10391 1013444.32205455
				10392 1013445.60323452
				10393 1013446.88441449
				10394 1013448.16559445
				10395 1013449.44677442
				10396 1013450.72795438
				10397 1013452.00913435
				10398 1013453.29031431
				10399 1013454.57149428
				10400 1013455.85267424
				10401 1013457.13385421
				10402 1013458.41503417
				10403 1013459.69621414
				10404 1013460.97739411
				10405 1013462.25857407
				10406 1013463.53975404
				10407 1013464.820934
				10408 1013466.10211397
				10409 1013467.38329393
				10410 1013468.6644739
				10411 1013469.94565386
				10412 1013471.22683383
				10413 1013472.50801379
				10414 1013473.78919376
				10415 1013475.07037373
				10416 1013476.35155369
				10417 1013477.63273366
				10418 1013478.91391362
				10419 1013480.19509359
				10420 1013481.47627355
				10421 1013482.75745352
				10422 1013484.03863348
				10423 1013485.31981345
				10424 1013486.60099341
				10425 1013487.88217338
				10426 1013489.16335335
				10427 1013490.44453331
				10428 1013491.72571328
				10429 1013493.00689324
				10430 1013494.28807321
				10431 1013495.56925317
				10432 1013496.85043314
				10433 1013498.1316131
				10434 1013499.41279307
				10435 1013500.69397303
				10436 1013501.975153
				10437 1013503.25633297
				10438 1013504.53751293
				10439 1013505.8186929
				10440 1013507.09987286
				10441 1013508.38105283
				10442 1013509.66223279
				10443 1013510.94341276
				10444 1013512.22459272
				10445 1013513.50577269
				10446 1013514.78695265
				10447 1013516.06813262
				10448 1013517.34931259
				10449 1013518.63049255
				10450 1013519.91167252
				10451 1013521.19285248
				10452 1013522.47403245
				10453 1013523.75521241
				10454 1013525.03639238
				10455 1013526.31757234
				10456 1013527.59875231
				10457 1013528.87993227
				10458 1013530.16111224
				10459 1013531.44229221
				10460 1013532.72347217
				10461 1013534.00465214
				10462 1013535.2858321
				10463 1013536.56701207
				10464 1013537.84819203
				10465 1013539.129372
				10466 1013540.41055196
				10467 1013541.69173193
				10468 1013542.97291189
				10469 1013544.25409186
				10470 1013545.53527183
				10471 1013546.81645179
				10472 1013548.09763176
				10473 1013549.37881172
				10474 1013550.65999169
				10475 1013551.94117165
				10476 1013553.22235162
				10477 1013554.50353158
				10478 1013555.78471155
				10479 1013557.06589151
				10480 1013558.34707148
				10481 1013559.62825145
				10482 1013560.90943141
				10483 1013562.19061138
				10484 1013563.47179134
				10485 1013564.75297131
				10486 1013566.03415127
				10487 1013567.31533124
				10488 1013568.5965112
				10489 1013569.87769117
				10490 1013571.15887113
				10491 1013572.4400511
				10492 1013573.72123107
				10493 1013575.00241103
				10494 1013576.283591
				10495 1013577.56477096
				10496 1013578.84595093
				10497 1013580.12713089
				10498 1013581.40831086
				10499 1013582.68949082
				10500 1013583.97067079
				10501 1013585.25185075
				10502 1013586.53303072
				10503 1013587.81421069
				10504 1013589.09539065
				10505 1013590.37657062
				10506 1013591.65775058
				10507 1013592.93893055
				10508 1013594.22011051
				10509 1013595.50129048
				10510 1013596.78247044
				10511 1013598.06365041
				10512 1013599.34483037
				10513 1013600.62601034
				10514 1013601.90719031
				10515 1013603.18837027
				10516 1013604.46955024
				10517 1013605.7507302
				10518 1013607.03191017
				10519 1013608.31309013
				10520 1013609.5942701
				10521 1013610.87545006
				10522 1013612.15663003
				10523 1013613.43780999
				10524 1013614.71898996
				10525 1013616.00016993
				10526 1013617.28134989
				10527 1013618.56252986
				10528 1013619.84370982
				10529 1013621.12488979
				10530 1013622.40606975
				10531 1013623.68724972
				10532 1013624.96842968
				10533 1013626.24960965
				10534 1013627.53078961
				10535 1013628.81196958
				10536 1013630.09314955
				10537 1013631.37432951
				10538 1013632.65550948
				10539 1013633.93668944
				10540 1013635.21786941
				10541 1013636.49904937
				10542 1013637.78022934
				10543 1013639.0614093
				10544 1013640.34258927
				10545 1013641.62376923
				10546 1013642.9049492
				10547 1013644.18612917
				10548 1013645.46730913
				10549 1013646.7484891
				10550 1013648.02966906
				10551 1013649.31084903
				10552 1013650.59202899
				10553 1013651.87320896
				10554 1013653.15438892
				10555 1013654.43556889
				10556 1013655.71674885
				10557 1013656.99792882
				10558 1013658.27910879
				10559 1013659.56028875
				10560 1013660.84146872
				10561 1013662.12264868
				10562 1013663.40382865
				10563 1013664.68500861
				10564 1013665.96618858
				10565 1013667.24736854
				10566 1013668.52854851
				10567 1013669.80972847
				10568 1013671.09090844
				10569 1013672.37208841
				10570 1013673.65326837
				10571 1013674.93444834
				10572 1013676.2156283
				10573 1013677.49680827
				10574 1013678.77798823
				10575 1013680.0591682
				10576 1013681.34034816
				10577 1013682.62152813
				10578 1013683.90270809
				10579 1013685.18388806
				10580 1013686.46506803
				10581 1013687.74624799
				10582 1013689.02742796
				10583 1013690.30860792
				10584 1013691.58978789
				10585 1013692.87096785
				10586 1013694.15214782
				10587 1013695.43332778
				10588 1013696.71450775
				10589 1013697.99568771
				10590 1013699.27686768
				10591 1013700.55804765
				10592 1013701.83922761
				10593 1013703.12040758
				10594 1013704.40158754
				10595 1013705.68276751
				10596 1013706.96394747
				10597 1013708.24512744
				10598 1013709.5263074
				10599 1013710.80748737
				10600 1013712.08866733
				10601 1013713.3698473
				10602 1013714.65102727
				10603 1013715.93220723
				10604 1013717.2133872
				10605 1013718.49456716
				10606 1013719.77574713
				10607 1013721.05692709
				10608 1013722.33810706
				10609 1013723.61928702
				10610 1013724.90046699
				10611 1013726.18164695
				10612 1013727.46282692
				10613 1013728.74400689
				10614 1013730.02518685
				10615 1013731.30636682
				10616 1013732.58754678
				10617 1013733.86872675
				10618 1013735.14990671
				10619 1013736.43108668
				10620 1013737.71226664
				10621 1013738.99344661
				10622 1013740.27462657
				10623 1013741.55580654
				10624 1013742.83698651
				10625 1013744.11816647
				10626 1013745.39934644
				10627 1013746.6805264
				10628 1013747.96170637
				10629 1013749.24288633
				10630 1013750.5240663
				10631 1013751.80524626
				10632 1013753.08642623
				10633 1013754.36760619
				10634 1013755.64878616
				10635 1013756.92996613
				10636 1013758.21114609
				10637 1013759.49232606
				10638 1013760.77350602
				10639 1013762.05468599
				10640 1013763.33586595
				10641 1013764.61704592
				10642 1013765.89822588
				10643 1013767.17940585
				10644 1013768.46058581
				10645 1013769.74176578
				10646 1013771.02294575
				10647 1013772.30412571
				10648 1013773.58530568
				10649 1013774.86648564
				10650 1013776.14766561
				10651 1013777.42884557
				10652 1013778.71002554
				10653 1013779.9912055
				10654 1013781.27238547
				10655 1013782.55356543
				10656 1013783.8347454
				10657 1013785.11592537
				10658 1013786.39710533
				10659 1013787.6782853
				10660 1013788.95946526
				10661 1013790.24064523
				10662 1013791.52182519
				10663 1013792.80300516
				10664 1013794.08418512
				10665 1013795.36536509
				10666 1013796.64654505
				10667 1013797.92772502
				10668 1013799.20890499
				10669 1013800.49008495
				10670 1013801.77126492
				10671 1013803.05244488
				10672 1013804.33362485
				10673 1013805.61480481
				10674 1013806.89598478
				10675 1013808.17716474
				10676 1013809.45834471
				10677 1013810.73952467
				10678 1013812.02070464
				10679 1013813.30188461
				10680 1013814.58306457
				10681 1013815.86424454
				10682 1013817.1454245
				10683 1013818.42660447
				10684 1013819.70778443
				10685 1013820.9889644
				10686 1013822.27014436
				10687 1013823.55132433
				10688 1013824.83250429
				10689 1013826.11368426
				10690 1013827.39486423
				10691 1013828.67604419
				10692 1013829.95722416
				10693 1013831.23840412
				10694 1013832.51958409
				10695 1013833.80076405
				10696 1013835.08194402
				10697 1013836.36312398
				10698 1013837.64430395
				10699 1013838.92548391
				10700 1013840.20666388
				10701 1013841.48784385
				10702 1013842.76902381
				10703 1013844.05020378
				10704 1013845.33138374
				10705 1013846.61256371
				10706 1013847.89374367
				10707 1013849.17492364
				10708 1013850.4561036
				10709 1013851.73728357
				10710 1013853.01846353
				10711 1013854.2996435
				10712 1013855.58082347
				10713 1013856.86200343
				10714 1013858.1431834
				10715 1013859.42436336
				10716 1013860.70554333
				10717 1013861.98672329
				10718 1013863.26790326
				10719 1013864.54908322
				10720 1013865.83026319
				10721 1013867.11144315
				10722 1013868.39262312
				10723 1013869.67380309
				10724 1013870.95498305
				10725 1013872.23616302
				10726 1013873.51734298
				10727 1013874.79852295
				10728 1013876.07970291
				10729 1013877.36088288
				10730 1013878.64206284
				10731 1013879.92324281
				10732 1013881.20442277
				10733 1013882.48560274
				10734 1013883.76678271
				10735 1013885.04796267
				10736 1013886.32914264
				10737 1013887.6103226
				10738 1013888.89150257
				10739 1013890.17268253
				10740 1013891.4538625
				10741 1013892.73504246
				10742 1013894.01622243
				10743 1013895.29740239
				10744 1013896.57858236
				10745 1013897.85976233
				10746 1013899.14094229
				10747 1013900.42212226
				10748 1013901.70330222
				10749 1013902.98448219
				10750 1013904.26566215
				10751 1013905.54684212
				10752 1013906.82802208
				10753 1013908.10920205
				10754 1013909.39038201
				10755 1013910.67156198
				10756 1013911.95274195
				10757 1013913.23392191
				10758 1013914.51510188
				10759 1013915.79628184
				10760 1013917.07746181
				10761 1013918.35864177
				10762 1013919.63982174
				10763 1013920.9210017
				10764 1013922.20218167
				10765 1013923.48336163
				10766 1013924.7645416
				10767 1013926.04572157
				10768 1013927.32690153
				10769 1013928.6080815
				10770 1013929.88926146
				10771 1013931.17044143
				10772 1013932.45162139
				10773 1013933.73280136
				10774 1013935.01398132
				10775 1013936.29516129
				10776 1013937.57634125
				10777 1013938.85752122
				10778 1013940.13870119
				10779 1013941.41988115
				10780 1013942.70106112
				10781 1013943.98224108
				10782 1013945.26342105
				10783 1013946.54460101
				10784 1013947.82578098
				10785 1013949.10696094
				10786 1013950.38814091
				10787 1013951.66932087
				10788 1013952.95050084
				10789 1013954.23168081
				10790 1013955.51286077
				10791 1013956.79404074
				10792 1013958.0752207
				10793 1013959.35640067
				10794 1013960.63758063
				10795 1013961.9187606
				10796 1013963.19994056
				10797 1013964.48112053
				10798 1013965.76230049
				10799 1013967.04348046
				10800 1013968.32466043
				10801 1013969.60584039
				10802 1013970.88702036
				10803 1013972.16820032
				10804 1013973.44938029
				10805 1013974.73056025
				10806 1013976.01174022
				10807 1013977.29292018
				10808 1013978.57410015
				10809 1013979.85528011
				10810 1013981.13646008
				10811 1013982.41764005
				10812 1013983.69882001
				10813 1013984.97999998
				10814 1013986.26117994
				10815 1013987.54235991
				10816 1013988.82353987
				10817 1013990.10471984
				10818 1013991.3858998
				10819 1013992.66707977
				10820 1013993.94825973
				10821 1013995.2294397
				10822 1013996.51061967
				10823 1013997.79179963
				10824 1013999.0729796
				10825 1014000.35415956
				10826 1014001.63533953
				10827 1014002.91651949
				10828 1014004.19769946
				10829 1014005.47887942
				10830 1014006.76005939
				10831 1014008.04123935
				10832 1014009.32241932
				10833 1014010.60359929
				10834 1014011.88477925
				10835 1014013.16595922
				10836 1014014.44713918
				10837 1014015.72831915
				10838 1014017.00949911
				10839 1014018.29067908
				10840 1014019.57185904
				10841 1014020.85303901
				10842 1014022.13421897
				10843 1014023.41539894
				10844 1014024.69657891
				10845 1014025.97775887
				10846 1014027.25893884
				10847 1014028.5401188
				10848 1014029.82129877
				10849 1014031.10247873
				10850 1014032.3836587
				10851 1014033.66483866
				10852 1014034.94601863
				10853 1014036.22719859
				10854 1014037.50837856
				10855 1014038.78955853
				10856 1014040.07073849
				10857 1014041.35191846
				10858 1014042.63309842
				10859 1014043.91427839
				10860 1014045.19545835
				10861 1014046.47663832
				10862 1014047.75781828
				10863 1014049.03899825
				10864 1014050.32017821
				10865 1014051.60135818
				10866 1014052.88253815
				10867 1014054.16371811
				10868 1014055.44489808
				10869 1014056.72607804
				10870 1014058.00725801
				10871 1014059.28843797
				10872 1014060.56961794
				10873 1014061.8507979
				10874 1014063.13197787
				10875 1014064.41315783
				10876 1014065.6943378
				10877 1014066.97551777
				10878 1014068.25669773
				10879 1014069.5378777
				10880 1014070.81905766
				10881 1014072.10023763
				10882 1014073.38141759
				10883 1014074.66259756
				10884 1014075.94377752
				10885 1014077.22495749
				10886 1014078.50613745
				10887 1014079.78731742
				10888 1014081.06849739
				10889 1014082.34967735
				10890 1014083.63085732
				10891 1014084.91203728
				10892 1014086.19321725
				10893 1014087.47439721
				10894 1014088.75557718
				10895 1014090.03675714
				10896 1014091.31793711
				10897 1014092.59911707
				10898 1014093.88029704
				10899 1014095.16147701
				10900 1014096.44265697
				10901 1014097.72383694
				10902 1014099.0050169
				10903 1014100.28619687
				10904 1014101.56737683
				10905 1014102.8485568
				10906 1014104.12973676
				10907 1014105.41091673
				10908 1014106.69209669
				10909 1014107.97327666
				10910 1014109.25445663
				10911 1014110.53563659
				10912 1014111.81681656
				10913 1014113.09799652
				10914 1014114.37917649
				10915 1014115.66035645
				10916 1014116.94153642
				10917 1014118.22271638
				10918 1014119.50389635
				10919 1014120.78507631
				10920 1014122.06625628
				10921 1014123.34743625
				10922 1014124.62861621
				10923 1014125.90979618
				10924 1014127.19097614
				10925 1014128.47215611
				10926 1014129.75333607
				10927 1014131.03451604
				10928 1014132.315696
				10929 1014133.59687597
				10930 1014134.87805593
				10931 1014136.1592359
				10932 1014137.44041587
				10933 1014138.72159583
				10934 1014140.0027758
				10935 1014141.28395576
				10936 1014142.56513573
				10937 1014143.84631569
				10938 1014145.12749566
				10939 1014146.40867562
				10940 1014147.68985559
				10941 1014148.97103555
				10942 1014150.25221552
				10943 1014151.53339549
				10944 1014152.81457545
				10945 1014154.09575542
				10946 1014155.37693538
				10947 1014156.65811535
				10948 1014157.93929531
				10949 1014159.22047528
				10950 1014160.50165524
				10951 1014161.78283521
				10952 1014163.06401517
				10953 1014164.34519514
				10954 1014165.62637511
				10955 1014166.90755507
				10956 1014168.18873504
				10957 1014169.469915
				10958 1014170.75109497
				10959 1014172.03227493
				10960 1014173.3134549
				10961 1014174.59463486
				10962 1014175.87581483
				10963 1014177.15699479
				10964 1014178.43817476
				10965 1014179.71935473
				10966 1014181.00053469
				10967 1014182.28171466
				10968 1014183.56289462
				10969 1014184.84407459
				10970 1014186.12525455
				10971 1014187.40643452
				10972 1014188.68761448
				10973 1014189.96879445
				10974 1014191.24997441
				10975 1014192.53115438
				10976 1014193.81233435
				10977 1014195.09351431
				10978 1014196.37469428
				10979 1014197.65587424
				10980 1014198.93705421
				10981 1014200.21823417
				10982 1014201.49941414
				10983 1014202.7805941
				10984 1014204.06177407
				10985 1014205.34295403
				10986 1014206.624134
				10987 1014207.90531397
				10988 1014209.18649393
				10989 1014210.4676739
				10990 1014211.74885386
				10991 1014213.03003383
				10992 1014214.31121379
				10993 1014215.59239376
				10994 1014216.87357372
				10995 1014218.15475369
				10996 1014219.43593365
				10997 1014220.71711362
				10998 1014221.99829359
				10999 1014223.27947355
				11000 1014224.56065352
				11001 1014225.84183348
				11002 1014227.12301345
				11003 1014228.40419341
				11004 1014229.68537338
				11005 1014230.96655334
				11006 1014232.24773331
				11007 1014233.52891327
				11008 1014234.81009324
				11009 1014236.09127321
				11010 1014237.37245317
				11011 1014238.65363314
				11012 1014239.9348131
				11013 1014241.21599307
				11014 1014242.49717303
				11015 1014243.778353
				11016 1014245.05953296
				11017 1014246.34071293
				11018 1014247.62189289
				11019 1014248.90307286
				11020 1014250.18425283
				11021 1014251.46543279
				11022 1014252.74661276
				11023 1014254.02779272
				11024 1014255.30897269
				11025 1014256.59015265
				11026 1014257.87133262
				11027 1014259.15251258
				11028 1014260.43369255
				11029 1014261.71487251
				11030 1014262.99605248
				11031 1014264.27723245
				11032 1014265.55841241
				11033 1014266.83959238
				11034 1014268.12077234
				11035 1014269.40195231
				11036 1014270.68313227
				11037 1014271.96431224
				11038 1014273.2454922
				11039 1014274.52667217
				11040 1014275.80785213
				11041 1014277.0890321
				11042 1014278.37021207
				11043 1014279.65139203
				11044 1014280.932572
				11045 1014282.21375196
				11046 1014283.49493193
				11047 1014284.77611189
				11048 1014286.05729186
				11049 1014287.33847182
				11050 1014288.61965179
				11051 1014289.90083175
				11052 1014291.18201172
				11053 1014292.46319169
				11054 1014293.74437165
				11055 1014295.02555162
				11056 1014296.30673158
				11057 1014297.58791155
				11058 1014298.86909151
				11059 1014300.15027148
				11060 1014301.43145144
				11061 1014302.71263141
				11062 1014303.99381137
				11063 1014305.27499134
				11064 1014306.55617131
				11065 1014307.83735127
				11066 1014309.11853124
				11067 1014310.3997112
				11068 1014311.68089117
				11069 1014312.96207113
				11070 1014314.2432511
				11071 1014315.52443106
				11072 1014316.80561103
				11073 1014318.08679099
				11074 1014319.36797096
				11075 1014320.64915093
				11076 1014321.93033089
				11077 1014323.21151086
				11078 1014324.49269082
				11079 1014325.77387079
				11080 1014327.05505075
				11081 1014328.33623072
				11082 1014329.61741068
				11083 1014330.89859065
				11084 1014332.17977061
				11085 1014333.46095058
				11086 1014334.74213055
				11087 1014336.02331051
				11088 1014337.30449048
				11089 1014338.58567044
				11090 1014339.86685041
				11091 1014341.14803037
				11092 1014342.42921034
				11093 1014343.7103903
				11094 1014344.99157027
				11095 1014346.27275023
				11096 1014347.5539302
				11097 1014348.83511017
				11098 1014350.11629013
				11099 1014351.3974701
				11100 1014352.67865006
				11101 1014353.95983003
				11102 1014355.24100999
				11103 1014356.52218996
				11104 1014357.80336992
				11105 1014359.08454989
				11106 1014360.36572985
				11107 1014361.64690982
				11108 1014362.92808979
				11109 1014364.20926975
				11110 1014365.49044972
				11111 1014366.77162968
				11112 1014368.05280965
				11113 1014369.33398961
				11114 1014370.61516958
				11115 1014371.89634954
				11116 1014373.17752951
				11117 1014374.45870947
				11118 1014375.73988944
				11119 1014377.02106941
				11120 1014378.30224937
				11121 1014379.58342934
				11122 1014380.8646093
				11123 1014382.14578927
				11124 1014383.42696923
				11125 1014384.7081492
				11126 1014385.98932916
				11127 1014387.27050913
				11128 1014388.55168909
				11129 1014389.83286906
				11130 1014391.11404903
				11131 1014392.39522899
				11132 1014393.67640896
				11133 1014394.95758892
				11134 1014396.23876889
				11135 1014397.51994885
				11136 1014398.80112882
				11137 1014400.08230878
				11138 1014401.36348875
				11139 1014402.64466871
				11140 1014403.92584868
				11141 1014405.20702865
				11142 1014406.48820861
				11143 1014407.76938858
				11144 1014409.05056854
				11145 1014410.33174851
				11146 1014411.61292847
				11147 1014412.89410844
				11148 1014414.1752884
				11149 1014415.45646837
				11150 1014416.73764833
				11151 1014418.0188283
				11152 1014419.30000827
				11153 1014420.58118823
				11154 1014421.8623682
				11155 1014423.14354816
				11156 1014424.42472813
				11157 1014425.70590809
				11158 1014426.98708806
				11159 1014428.26826802
				11160 1014429.54944799
				11161 1014430.83062795
				11162 1014432.11180792
				11163 1014433.39298789
				11164 1014434.67416785
				11165 1014435.95534782
				11166 1014437.23652778
				11167 1014438.51770775
				11168 1014439.79888771
				11169 1014441.08006768
				11170 1014442.36124764
				11171 1014443.64242761
				11172 1014444.92360757
				11173 1014446.20478754
				11174 1014447.48596751
				11175 1014448.76714747
				11176 1014450.04832744
				11177 1014451.3295074
				11178 1014452.61068737
				11179 1014453.89186733
				11180 1014455.1730473
				11181 1014456.45422726
				11182 1014457.73540723
				11183 1014459.01658719
				11184 1014460.29776716
				11185 1014461.57894713
				11186 1014462.86012709
				11187 1014464.14130706
				11188 1014465.42248702
				11189 1014466.70366699
				11190 1014467.98484695
				11191 1014469.26602692
				11192 1014470.54720688
				11193 1014471.82838685
				11194 1014473.10956681
				11195 1014474.39074678
				11196 1014475.67192674
				11197 1014476.95310671
				11198 1014478.23428668
				11199 1014479.51546664
				11200 1014480.79664661
				11201 1014482.07782657
				11202 1014483.35900654
				11203 1014484.6401865
				11204 1014485.92136647
				11205 1014487.20254643
				11206 1014488.4837264
				11207 1014489.76490637
				11208 1014491.04608633
				11209 1014492.3272663
				11210 1014493.60844626
				11211 1014494.88962623
				11212 1014496.17080619
				11213 1014497.45198616
				11214 1014498.73316612
				11215 1014500.01434609
				11216 1014501.29552605
				11217 1014502.57670602
				11218 1014503.85788599
				11219 1014505.13906595
				11220 1014506.42024592
				11221 1014507.70142588
				11222 1014508.98260585
				11223 1014510.26378581
				11224 1014511.54496578
				11225 1014512.82614574
				11226 1014514.10732571
				11227 1014515.38850567
				11228 1014516.66968564
				11229 1014517.95086561
				11230 1014519.23204557
				11231 1014520.51322554
				11232 1014521.7944055
				11233 1014523.07558547
				11234 1014524.35676543
				11235 1014525.6379454
				11236 1014526.91912536
				11237 1014528.20030533
				11238 1014529.48148529
				11239 1014530.76266526
				11240 1014532.04384523
				11241 1014533.32502519
				11242 1014534.60620516
				11243 1014535.88738512
				11244 1014537.16856509
				11245 1014538.44974505
				11246 1014539.73092502
				11247 1014541.01210498
				11248 1014542.29328495
				11249 1014543.57446491
				11250 1014544.85564488
				11251 1014546.13682484
				11252 1014547.41800481
				11253 1014548.69918478
				11254 1014549.98036474
				11255 1014551.26154471
				11256 1014552.54272467
				11257 1014553.82390464
				11258 1014555.1050846
				11259 1014556.38626457
				11260 1014557.66744453
				11261 1014558.9486245
				11262 1014560.22980446
				11263 1014561.51098443
				11264 1014562.7921644
				11265 1014564.07334436
				11266 1014565.35452433
				11267 1014566.63570429
				11268 1014567.91688426
				11269 1014569.19806422
				11270 1014570.47924419
				11271 1014571.76042415
				11272 1014573.04160412
				11273 1014574.32278408
				11274 1014575.60396405
				11275 1014576.88514402
				11276 1014578.16632398
				11277 1014579.44750395
				11278 1014580.72868391
				11279 1014582.00986388
				11280 1014583.29104384
				11281 1014584.57222381
				11282 1014585.85340377
				11283 1014587.13458374
				11284 1014588.4157637
				11285 1014589.69694367
				11286 1014590.97812364
				11287 1014592.2593036
				11288 1014593.54048357
				11289 1014594.82166353
				11290 1014596.1028435
				11291 1014597.38402346
				11292 1014598.66520343
				11293 1014599.94638339
				11294 1014601.22756336
				11295 1014602.50874333
				11296 1014603.78992329
				11297 1014605.07110326
				11298 1014606.35228322
				11299 1014607.63346319
				11300 1014608.91464315
				11301 1014610.19582312
				11302 1014611.47700308
				11303 1014612.75818305
				11304 1014614.03936301
				11305 1014615.32054298
				11306 1014616.60172294
				11307 1014617.88290291
				11308 1014619.16408288
				11309 1014620.44526284
				11310 1014621.72644281
				11311 1014623.00762277
				11312 1014624.28880274
				11313 1014625.5699827
				11314 1014626.85116267
				11315 1014628.13234263
				11316 1014629.4135226
				11317 1014630.69470256
				11318 1014631.97588253
				11319 1014633.2570625
				11320 1014634.53824246
				11321 1014635.81942243
				11322 1014637.10060239
				11323 1014638.38178236
				11324 1014639.66296232
				11325 1014640.94414229
				11326 1014642.22532225
				11327 1014643.50650222
				11328 1014644.78768218
				11329 1014646.06886215
				11330 1014647.35004212
				11331 1014648.63122208
				11332 1014649.91240205
				11333 1014651.19358201
				11334 1014652.47476198
				11335 1014653.75594194
				11336 1014655.03712191
				11337 1014656.31830187
				11338 1014657.59948184
				11339 1014658.8806618
				11340 1014660.16184177
				11341 1014661.44302174
				11342 1014662.7242017
				11343 1014664.00538167
				11344 1014665.28656163
				11345 1014666.5677416
				11346 1014667.84892156
				11347 1014669.13010153
				11348 1014670.41128149
				11349 1014671.69246146
				11350 1014672.97364142
				11351 1014674.25482139
				11352 1014675.53600136
				11353 1014676.81718132
				11354 1014678.09836129
				11355 1014679.37954125
				11356 1014680.66072122
				11357 1014681.94190118
				11358 1014683.22308115
				11359 1014684.50426111
				11360 1014685.78544108
				11361 1014687.06662104
				11362 1014688.34780101
				11363 1014689.62898098
				11364 1014690.91016094
				11365 1014692.19134091
				11366 1014693.47252087
				11367 1014694.75370084
				11368 1014696.0348808
				11369 1014697.31606077
				11370 1014698.59724073
				11371 1014699.8784207
				11372 1014701.15960066
				11373 1014702.44078063
				11374 1014703.7219606
				11375 1014705.00314056
				11376 1014706.28432053
				11377 1014707.56550049
				11378 1014708.84668046
				11379 1014710.12786042
				11380 1014711.40904039
				11381 1014712.69022035
				11382 1014713.97140032
				11383 1014715.25258028
				11384 1014716.53376025
				11385 1014717.81494022
				11386 1014719.09612018
				11387 1014720.37730015
				11388 1014721.65848011
				11389 1014722.93966008
				11390 1014724.22084004
				11391 1014725.50202001
				11392 1014726.78319997
				11393 1014728.06437994
				11394 1014729.3455599
				11395 1014730.62673987
				11396 1014731.90791984
				11397 1014733.1890998
				11398 1014734.47027977
				11399 1014735.75145973
				11400 1014737.0326397
				11401 1014738.31381966
				11402 1014739.59499963
				11403 1014740.87617959
				11404 1014742.15735956
				11405 1014743.43853952
				11406 1014744.71971949
				11407 1014746.00089946
				11408 1014747.28207942
				11409 1014748.56325939
				11410 1014749.84443935
				11411 1014751.12561932
				11412 1014752.40679928
				11413 1014753.68797925
				11414 1014754.96915921
				11415 1014756.25033918
				11416 1014757.53151914
				11417 1014758.81269911
				11418 1014760.09387908
				11419 1014761.37505904
				11420 1014762.65623901
				11421 1014763.93741897
				11422 1014765.21859894
				11423 1014766.4997789
				11424 1014767.78095887
				11425 1014769.06213883
				11426 1014770.3433188
				11427 1014771.62449876
				11428 1014772.90567873
				11429 1014774.1868587
				11430 1014775.46803866
				11431 1014776.74921863
				11432 1014778.03039859
				11433 1014779.31157856
				11434 1014780.59275852
				11435 1014781.87393849
				11436 1014783.15511845
				11437 1014784.43629842
				11438 1014785.71747838
				11439 1014786.99865835
				11440 1014788.27983832
				11441 1014789.56101828
				11442 1014790.84219825
				11443 1014792.12337821
				11444 1014793.40455818
				11445 1014794.68573814
				11446 1014795.96691811
				11447 1014797.24809807
				11448 1014798.52927804
				11449 1014799.810458
				11450 1014801.09163797
				11451 1014802.37281794
				11452 1014803.6539979
				11453 1014804.93517787
				11454 1014806.21635783
				11455 1014807.4975378
				11456 1014808.77871776
				11457 1014810.05989773
				11458 1014811.34107769
				11459 1014812.62225766
				11460 1014813.90343762
				11461 1014815.18461759
				11462 1014816.46579756
				11463 1014817.74697752
				11464 1014819.02815749
				11465 1014820.30933745
				11466 1014821.59051742
				11467 1014822.87169738
				11468 1014824.15287735
				11469 1014825.43405731
				11470 1014826.71523728
				11471 1014827.99641724
				11472 1014829.27759721
				11473 1014830.55877718
				11474 1014831.83995714
				11475 1014833.12113711
				11476 1014834.40231707
				11477 1014835.68349704
				11478 1014836.964677
				11479 1014838.24585697
				11480 1014839.52703693
				11481 1014840.8082169
				11482 1014842.08939686
				11483 1014843.37057683
				11484 1014844.6517568
				11485 1014845.93293676
				11486 1014847.21411673
				11487 1014848.49529669
				11488 1014849.77647666
				11489 1014851.05765662
				11490 1014852.33883659
				11491 1014853.62001655
				11492 1014854.90119652
				11493 1014856.18237648
				11494 1014857.46355645
				11495 1014858.74473642
				11496 1014860.02591638
				11497 1014861.30709635
				11498 1014862.58827631
				11499 1014863.86945628
				11500 1014865.15063624
				11501 1014866.43181621
				11502 1014867.71299617
				11503 1014868.99417614
				11504 1014870.2753561
				11505 1014871.55653607
				11506 1014872.83771604
				11507 1014874.118896
				11508 1014875.40007597
				11509 1014876.68125593
				11510 1014877.9624359
				11511 1014879.24361586
				11512 1014880.52479583
				11513 1014881.80597579
				11514 1014883.08715576
				11515 1014884.36833572
				11516 1014885.64951569
				11517 1014886.93069566
				11518 1014888.21187562
				11519 1014889.49305559
				11520 1014890.77423555
				11521 1014892.05541552
				11522 1014893.33659548
				11523 1014894.61777545
				11524 1014895.89895541
				11525 1014897.18013538
				11526 1014898.46131534
				11527 1014899.74249531
				11528 1014901.02367528
				11529 1014902.30485524
				11530 1014903.58603521
				11531 1014904.86721517
				11532 1014906.14839514
				11533 1014907.4295751
				11534 1014908.71075507
				11535 1014909.99193503
				11536 1014911.273115
				11537 1014912.55429496
				11538 1014913.83547493
				11539 1014915.1166549
				11540 1014916.39783486
				11541 1014917.67901483
				11542 1014918.96019479
				11543 1014920.24137476
				11544 1014921.52255472
				11545 1014922.80373469
				11546 1014924.08491465
				11547 1014925.36609462
				11548 1014926.64727458
				11549 1014927.92845455
				11550 1014929.20963452
				11551 1014930.49081448
				11552 1014931.77199445
				11553 1014933.05317441
				11554 1014934.33435438
				11555 1014935.61553434
				11556 1014936.89671431
				11557 1014938.17789427
				11558 1014939.45907424
				11559 1014940.7402542
				11560 1014942.02143417
				11561 1014943.30261414
				11562 1014944.5837941
				11563 1014945.86497407
				11564 1014947.14615403
				11565 1014948.427334
				11566 1014949.70851396
				11567 1014950.98969393
				11568 1014952.27087389
				11569 1014953.55205386
				11570 1014954.83323382
				11571 1014956.11441379
				11572 1014957.39559376
				11573 1014958.67677372
				11574 1014959.95795369
				11575 1014961.23913365
				11576 1014962.52031362
				11577 1014963.80149358
				11578 1014965.08267355
				11579 1014966.36385351
				11580 1014967.64503348
				11581 1014968.92621344
				11582 1014970.20739341
				11583 1014971.48857338
				11584 1014972.76975334
				11585 1014974.05093331
				11586 1014975.33211327
				11587 1014976.61329324
				11588 1014977.8944732
				11589 1014979.17565317
				11590 1014980.45683313
				11591 1014981.7380131
				11592 1014983.01919306
				11593 1014984.30037303
				11594 1014985.581553
				11595 1014986.86273296
				11596 1014988.14391293
				11597 1014989.42509289
				11598 1014990.70627286
				11599 1014991.98745282
				11600 1014993.26863279
				11601 1014994.54981275
				11602 1014995.83099272
				11603 1014997.11217268
				11604 1014998.39335265
				11605 1014999.67453262
				11606 1015000.95571258
				11607 1015002.23689255
				11608 1015003.51807251
				11609 1015004.79925248
				11610 1015006.08043244
				11611 1015007.36161241
				11612 1015008.64279237
				11613 1015009.92397234
				11614 1015011.2051523
				11615 1015012.48633227
				11616 1015013.76751224
				11617 1015015.0486922
				11618 1015016.32987217
				11619 1015017.61105213
				11620 1015018.8922321
				11621 1015020.17341206
				11622 1015021.45459203
				11623 1015022.73577199
				11624 1015024.01695196
				11625 1015025.29813192
				11626 1015026.57931189
				11627 1015027.86049186
				11628 1015029.14167182
				11629 1015030.42285179
				11630 1015031.70403175
				11631 1015032.98521172
				11632 1015034.26639168
				11633 1015035.54757165
				11634 1015036.82875161
				11635 1015038.10993158
				11636 1015039.39111154
				11637 1015040.67229151
				11638 1015041.95347148
				11639 1015043.23465144
				11640 1015044.51583141
				11641 1015045.79701137
				11642 1015047.07819134
				11643 1015048.3593713
				11644 1015049.64055127
				11645 1015050.92173123
				11646 1015052.2029112
				11647 1015053.48409116
				11648 1015054.76527113
				11649 1015056.0464511
				11650 1015057.32763106
				11651 1015058.60881103
				11652 1015059.88999099
				11653 1015061.17117096
				11654 1015062.45235092
				11655 1015063.73353089
				11656 1015065.01471085
				11657 1015066.29589082
				11658 1015067.57707078
				11659 1015068.85825075
				11660 1015070.13943072
				11661 1015071.42061068
				11662 1015072.70179065
				11663 1015073.98297061
				11664 1015075.26415058
				11665 1015076.54533054
				11666 1015077.82651051
				11667 1015079.10769047
				11668 1015080.38887044
				11669 1015081.6700504
				11670 1015082.95123037
				11671 1015084.23241034
				11672 1015085.5135903
				11673 1015086.79477027
				11674 1015088.07595023
				11675 1015089.3571302
				11676 1015090.63831016
				11677 1015091.91949013
				11678 1015093.20067009
				11679 1015094.48185006
				11680 1015095.76303002
				11681 1015097.04420999
				11682 1015098.32538996
				11683 1015099.60656992
				11684 1015100.88774989
				11685 1015102.16892985
				11686 1015103.45010982
				11687 1015104.73128978
				11688 1015106.01246975
				11689 1015107.29364971
				11690 1015108.57482968
				11691 1015109.85600964
				11692 1015111.13718961
				11693 1015112.41836958
				11694 1015113.69954954
				11695 1015114.98072951
				11696 1015116.26190947
				11697 1015117.54308944
				11698 1015118.8242694
				11699 1015120.10544937
				11700 1015121.38662933
				11701 1015122.6678093
				11702 1015123.94898926
				11703 1015125.23016923
				11704 1015126.5113492
				11705 1015127.79252916
				11706 1015129.07370913
				11707 1015130.35488909
				11708 1015131.63606906
				11709 1015132.91724902
				11710 1015134.19842899
				11711 1015135.47960895
				11712 1015136.76078892
				11713 1015138.04196888
				11714 1015139.32314885
				11715 1015140.60432882
				11716 1015141.88550878
				11717 1015143.16668875
				11718 1015144.44786871
				11719 1015145.72904868
				11720 1015147.01022864
				11721 1015148.29140861
				11722 1015149.57258857
				11723 1015150.85376854
				11724 1015152.1349485
				11725 1015153.41612847
				11726 1015154.69730844
				11727 1015155.9784884
				11728 1015157.25966837
				11729 1015158.54084833
				11730 1015159.8220283
				11731 1015161.10320826
				11732 1015162.38438823
				11733 1015163.66556819
				11734 1015164.94674816
				11735 1015166.22792812
				11736 1015167.50910809
				11737 1015168.79028806
				11738 1015170.07146802
				11739 1015171.35264799
				11740 1015172.63382795
				11741 1015173.91500792
				11742 1015175.19618788
				11743 1015176.47736785
				11744 1015177.75854781
				11745 1015179.03972778
				11746 1015180.32090774
				11747 1015181.60208771
				11748 1015182.88326768
				11749 1015184.16444764
				11750 1015185.44562761
				11751 1015186.72680757
				11752 1015188.00798754
				11753 1015189.2891675
				11754 1015190.57034747
				11755 1015191.85152743
				11756 1015193.1327074
				11757 1015194.41388736
				11758 1015195.69506733
				11759 1015196.9762473
				11760 1015198.25742726
				11761 1015199.53860723
				11762 1015200.81978719
				11763 1015202.10096716
				11764 1015203.38214712
				11765 1015204.66332709
				11766 1015205.94450705
				11767 1015207.22568702
				11768 1015208.50686698
				11769 1015209.78804695
				11770 1015211.06922692
				11771 1015212.35040688
				11772 1015213.63158685
				11773 1015214.91276681
				11774 1015216.19394678
				11775 1015217.47512674
				11776 1015218.75630671
				11777 1015220.03748667
				11778 1015221.31866664
				11779 1015222.5998466
				11780 1015223.88102657
				11781 1015225.16220654
				11782 1015226.4433865
				11783 1015227.72456647
				11784 1015229.00574643
				11785 1015230.2869264
				11786 1015231.56810636
				11787 1015232.84928633
				11788 1015234.13046629
				11789 1015235.41164626
				11790 1015236.69282622
				11791 1015237.97400619
				11792 1015239.25518616
				11793 1015240.53636612
				11794 1015241.81754609
				11795 1015243.09872605
				11796 1015244.37990602
				11797 1015245.66108598
				11798 1015246.94226595
				11799 1015248.22344591
				11800 1015249.50462588
				11801 1015250.78580584
				11802 1015252.06698581
				11803 1015253.34816578
				11804 1015254.62934574
				11805 1015255.91052571
				11806 1015257.19170567
				11807 1015258.47288564
				11808 1015259.7540656
				11809 1015261.03524557
				11810 1015262.31642553
				11811 1015263.5976055
				11812 1015264.87878546
				11813 1015266.15996543
				11814 1015267.4411454
				11815 1015268.72232536
				11816 1015270.00350533
				11817 1015271.28468529
				11818 1015272.56586526
				11819 1015273.84704522
				11820 1015275.12822519
				11821 1015276.40940515
				11822 1015277.69058512
				11823 1015278.97176508
				11824 1015280.25294505
				11825 1015281.53412502
				11826 1015282.81530498
				11827 1015284.09648495
				11828 1015285.37766491
				11829 1015286.65884488
				11830 1015287.94002484
				11831 1015289.22120481
				11832 1015290.50238477
				11833 1015291.78356474
				11834 1015293.0647447
				11835 1015294.34592467
				11836 1015295.62710464
				11837 1015296.9082846
				11838 1015298.18946457
				11839 1015299.47064453
				11840 1015300.7518245
				11841 1015302.03300446
				11842 1015303.31418443
				11843 1015304.59536439
				11844 1015305.87654436
				11845 1015307.15772432
				11846 1015308.43890429
				11847 1015309.72008426
				11848 1015311.00126422
				11849 1015312.28244419
				11850 1015313.56362415
				11851 1015314.84480412
				11852 1015316.12598408
				11853 1015317.40716405
				11854 1015318.68834401
				11855 1015319.96952398
				11856 1015321.25070394
				11857 1015322.53188391
				11858 1015323.81306388
				11859 1015325.09424384
				11860 1015326.37542381
				11861 1015327.65660377
				11862 1015328.93778374
				11863 1015330.2189637
				11864 1015331.50014367
				11865 1015332.78132363
				11866 1015334.0625036
				11867 1015335.34368356
				11868 1015336.62486353
				11869 1015337.9060435
				11870 1015339.18722346
				11871 1015340.46840343
				11872 1015341.74958339
				11873 1015343.03076336
				11874 1015344.31194332
				11875 1015345.59312329
				11876 1015346.87430325
				11877 1015348.15548322
				11878 1015349.43666318
				11879 1015350.71784315
				11880 1015351.99902312
				11881 1015353.28020308
				11882 1015354.56138305
				11883 1015355.84256301
				11884 1015357.12374298
				11885 1015358.40492294
				11886 1015359.68610291
				11887 1015360.96728287
				11888 1015362.24846284
				11889 1015363.5296428
				11890 1015364.81082277
				11891 1015366.09200274
				11892 1015367.3731827
				11893 1015368.65436267
				11894 1015369.93554263
				11895 1015371.2167226
				11896 1015372.49790256
				11897 1015373.77908253
				11898 1015375.06026249
				11899 1015376.34144246
				11900 1015377.62262242
				11901 1015378.90380239
				11902 1015380.18498236
				11903 1015381.46616232
				11904 1015382.74734229
				11905 1015384.02852225
				11906 1015385.30970222
				11907 1015386.59088218
				11908 1015387.87206215
				11909 1015389.15324211
				11910 1015390.43442208
				11911 1015391.71560204
				11912 1015392.99678201
				11913 1015394.27796198
				11914 1015395.55914194
				11915 1015396.84032191
				11916 1015398.12150187
				11917 1015399.40268184
				11918 1015400.6838618
				11919 1015401.96504177
				11920 1015403.24622173
				11921 1015404.5274017
				11922 1015405.80858166
				11923 1015407.08976163
				11924 1015408.3709416
				11925 1015409.65212156
				11926 1015410.93330153
				11927 1015412.21448149
				11928 1015413.49566146
				11929 1015414.77684142
				11930 1015416.05802139
				11931 1015417.33920135
				11932 1015418.62038132
				11933 1015419.90156128
				11934 1015421.18274125
				11935 1015422.46392122
				11936 1015423.74510118
				11937 1015425.02628115
				11938 1015426.30746111
				11939 1015427.58864108
				11940 1015428.86982104
				11941 1015430.15100101
				11942 1015431.43218097
				11943 1015432.71336094
				11944 1015433.9945409
				11945 1015435.27572087
				11946 1015436.55690084
				11947 1015437.8380808
				11948 1015439.11926077
				11949 1015440.40044073
				11950 1015441.6816207
				11951 1015442.96280066
				11952 1015444.24398063
				11953 1015445.52516059
				11954 1015446.80634056
				11955 1015448.08752052
				11956 1015449.36870049
				11957 1015450.64988046
				11958 1015451.93106042
				11959 1015453.21224039
				11960 1015454.49342035
				11961 1015455.77460032
				11962 1015457.05578028
				11963 1015458.33696025
				11964 1015459.61814021
				11965 1015460.89932018
				11966 1015462.18050014
				11967 1015463.46168011
				11968 1015464.74286008
				11969 1015466.02404004
				11970 1015467.30522001
				11971 1015468.58639997
				11972 1015469.86757994
				11973 1015471.1487599
				11974 1015472.42993987
				11975 1015473.71111983
				11976 1015474.9922998
				11977 1015476.27347976
				11978 1015477.55465973
				11979 1015478.8358397
				11980 1015480.11701966
				11981 1015481.39819963
				11982 1015482.67937959
				11983 1015483.96055956
				11984 1015485.24173952
				11985 1015486.52291949
				11986 1015487.80409945
				11987 1015489.08527942
				11988 1015490.36645938
				11989 1015491.64763935
				11990 1015492.92881932
				11991 1015494.20999928
				11992 1015495.49117925
				11993 1015496.77235921
				11994 1015498.05353918
				11995 1015499.33471914
				11996 1015500.61589911
				11997 1015501.89707907
				11998 1015503.17825904
				11999 1015504.459439
				12000 1015505.74061897
				12001 1015507.02179894
				12002 1015508.3029789
				12003 1015509.58415887
				12004 1015510.86533883
				12005 1015512.1465188
				12006 1015513.42769876
				12007 1015514.70887873
				12008 1015515.99005869
				12009 1015517.27123866
				12010 1015518.55241862
				12011 1015519.83359859
				12012 1015521.11477856
				12013 1015522.39595852
				12014 1015523.67713849
				12015 1015524.95831845
				12016 1015526.23949842
				12017 1015527.52067838
				12018 1015528.80185835
				12019 1015530.08303831
				12020 1015531.36421828
				12021 1015532.64539824
				12022 1015533.92657821
				12023 1015535.20775818
				12024 1015536.48893814
				12025 1015537.77011811
				12026 1015539.05129807
				12027 1015540.33247804
				12028 1015541.613658
				12029 1015542.89483797
				12030 1015544.17601793
				12031 1015545.4571979
				12032 1015546.73837786
				12033 1015548.01955783
				12034 1015549.3007378
				12035 1015550.58191776
				12036 1015551.86309773
				12037 1015553.14427769
				12038 1015554.42545766
				12039 1015555.70663762
				12040 1015556.98781759
				12041 1015558.26899755
				12042 1015559.55017752
				12043 1015560.83135748
				12044 1015562.11253745
				12045 1015563.39371742
				12046 1015564.67489738
				12047 1015565.95607735
				12048 1015567.23725731
				12049 1015568.51843728
				12050 1015569.79961724
				12051 1015571.08079721
				12052 1015572.36197717
				12053 1015573.64315714
				12054 1015574.9243371
				12055 1015576.20551707
				12056 1015577.48669704
				12057 1015578.767877
				12058 1015580.04905697
				12059 1015581.33023693
				12060 1015582.6114169
				12061 1015583.89259686
				12062 1015585.17377683
				12063 1015586.45495679
				12064 1015587.73613676
				12065 1015589.01731672
				12066 1015590.29849669
				12067 1015591.57967666
				12068 1015592.86085662
				12069 1015594.14203659
				12070 1015595.42321655
				12071 1015596.70439652
				12072 1015597.98557648
				12073 1015599.26675645
				12074 1015600.54793641
				12075 1015601.82911638
				12076 1015603.11029634
				12077 1015604.39147631
				12078 1015605.67265628
				12079 1015606.95383624
				12080 1015608.23501621
				12081 1015609.51619617
				12082 1015610.79737614
				12083 1015612.0785561
				12084 1015613.35973607
				12085 1015614.64091603
				12086 1015615.922096
				12087 1015617.20327596
				12088 1015618.48445593
				12089 1015619.7656359
				12090 1015621.04681586
				12091 1015622.32799583
				12092 1015623.60917579
				12093 1015624.89035576
				12094 1015626.17153572
				12095 1015627.45271569
				12096 1015628.73389565
				12097 1015630.01507562
				12098 1015631.29625558
				12099 1015632.57743555
				12100 1015633.85861552
				12101 1015635.13979548
				12102 1015636.42097545
				12103 1015637.70215541
				12104 1015638.98333538
				12105 1015640.26451534
				12106 1015641.54569531
				12107 1015642.82687527
				12108 1015644.10805524
				12109 1015645.3892352
				12110 1015646.67041517
				12111 1015647.95159514
				12112 1015649.2327751
				12113 1015650.51395507
				12114 1015651.79513503
				12115 1015653.076315
				12116 1015654.35749496
				12117 1015655.63867493
				12118 1015656.91985489
				12119 1015658.20103486
				12120 1015659.48221482
				12121 1015660.76339479
				12122 1015662.04457476
				12123 1015663.32575472
				12124 1015664.60693469
				12125 1015665.88811465
				12126 1015667.16929462
				12127 1015668.45047458
				12128 1015669.73165455
				12129 1015671.01283451
				12130 1015672.29401448
				12131 1015673.57519444
				12132 1015674.85637441
				12133 1015676.13755438
				12134 1015677.41873434
				12135 1015678.69991431
				12136 1015679.98109427
				12137 1015681.26227424
				12138 1015682.5434542
				12139 1015683.82463417
				12140 1015685.10581413
				12141 1015686.3869941
				12142 1015687.66817406
				12143 1015688.94935403
				12144 1015690.230534
				12145 1015691.51171396
				12146 1015692.79289393
				12147 1015694.07407389
				12148 1015695.35525386
				12149 1015696.63643382
				12150 1015697.91761379
				12151 1015699.19879375
				12152 1015700.47997372
				12153 1015701.76115368
				12154 1015703.04233365
				12155 1015704.32351362
				12156 1015705.60469358
				12157 1015706.88587355
				12158 1015708.16705351
				12159 1015709.44823348
				12160 1015710.72941344
				12161 1015712.01059341
				12162 1015713.29177337
				12163 1015714.57295334
				12164 1015715.8541333
				12165 1015717.13531327
				12166 1015718.41649324
				12167 1015719.6976732
				12168 1015720.97885317
				12169 1015722.26003313
				12170 1015723.5412131
				12171 1015724.82239306
				12172 1015726.10357303
				12173 1015727.38475299
				12174 1015728.66593296
				12175 1015729.94711292
				12176 1015731.22829289
				12177 1015732.50947286
				12178 1015733.79065282
				12179 1015735.07183279
				12180 1015736.35301275
				12181 1015737.63419272
				12182 1015738.91537268
				12183 1015740.19655265
				12184 1015741.47773261
				12185 1015742.75891258
				12186 1015744.04009254
				12187 1015745.32127251
				12188 1015746.60245248
				12189 1015747.88363244
				12190 1015749.16481241
				12191 1015750.44599237
				12192 1015751.72717234
				12193 1015753.0083523
				12194 1015754.28953227
				12195 1015755.57071223
				12196 1015756.8518922
				12197 1015758.13307216
				12198 1015759.41425213
				12199 1015760.6954321
				12200 1015761.97661206
				12201 1015763.25779203
				12202 1015764.53897199
				12203 1015765.82015196
				12204 1015767.10133192
				12205 1015768.38251189
				12206 1015769.66369185
				12207 1015770.94487182
				12208 1015772.22605178
				12209 1015773.50723175
				12210 1015774.78841172
				12211 1015776.06959168
				12212 1015777.35077165
				12213 1015778.63195161
				12214 1015779.91313158
				12215 1015781.19431154
				12216 1015782.47549151
				12217 1015783.75667147
				12218 1015785.03785144
				12219 1015786.3190314
				12220 1015787.60021137
				12221 1015788.88139134
				12222 1015790.1625713
				12223 1015791.44375127
				12224 1015792.72493123
				12225 1015794.0061112
				12226 1015795.28729116
				12227 1015796.56847113
				12228 1015797.84965109
				12229 1015799.13083106
				12230 1015800.41201102
				12231 1015801.69319099
				12232 1015802.97437096
				12233 1015804.25555092
				12234 1015805.53673089
				12235 1015806.81791085
				12236 1015808.09909082
				12237 1015809.38027078
				12238 1015810.66145075
				12239 1015811.94263071
				12240 1015813.22381068
				12241 1015814.50499064
				12242 1015815.78617061
				12243 1015817.06735058
				12244 1015818.34853054
				12245 1015819.62971051
				12246 1015820.91089047
				12247 1015822.19207044
				12248 1015823.4732504
				12249 1015824.75443037
				12250 1015826.03561033
				12251 1015827.3167903
				12252 1015828.59797026
				12253 1015829.87915023
				12254 1015831.1603302
				12255 1015832.44151016
				12256 1015833.72269013
				12257 1015835.00387009
				12258 1015836.28505006
				12259 1015837.56623002
				12260 1015838.84740999
				12261 1015840.12858995
				12262 1015841.40976992
				12263 1015842.69094988
				12264 1015843.97212985
				12265 1015845.25330982
				12266 1015846.53448978
				12267 1015847.81566975
				12268 1015849.09684971
				12269 1015850.37802968
				12270 1015851.65920964
				12271 1015852.94038961
				12272 1015854.22156957
				12273 1015855.50274954
				12274 1015856.7839295
				12275 1015858.06510947
				12276 1015859.34628944
				12277 1015860.6274694
				12278 1015861.90864937
				12279 1015863.18982933
				12280 1015864.4710093
				12281 1015865.75218926
				12282 1015867.03336923
				12283 1015868.31454919
				12284 1015869.59572916
				12285 1015870.87690912
				12286 1015872.15808909
				12287 1015873.43926906
				12288 1015874.72044902
				12289 1015876.00162899
				12290 1015877.28280895
				12291 1015878.56398892
				12292 1015879.84516888
				12293 1015881.12634885
				12294 1015882.40752881
				12295 1015883.68870878
				12296 1015884.96988874
				12297 1015886.25106871
				12298 1015887.53224868
				12299 1015888.81342864
				12300 1015890.09460861
				12301 1015891.37578857
				12302 1015892.65696854
				12303 1015893.9381485
				12304 1015895.21932847
				12305 1015896.50050843
				12306 1015897.7816884
				12307 1015899.06286836
				12308 1015900.34404833
				12309 1015901.6252283
				12310 1015902.90640826
				12311 1015904.18758823
				12312 1015905.46876819
				12313 1015906.74994816
				12314 1015908.03112812
				12315 1015909.31230809
				12316 1015910.59348805
				12317 1015911.87466802
				12318 1015913.15584798
				12319 1015914.43702795
				12320 1015915.71820792
				12321 1015916.99938788
				12322 1015918.28056785
				12323 1015919.56174781
				12324 1015920.84292778
				12325 1015922.12410774
				12326 1015923.40528771
				12327 1015924.68646767
				12328 1015925.96764764
				12329 1015927.2488276
				12330 1015928.53000757
				12331 1015929.81118754
				12332 1015931.0923675
				12333 1015932.37354747
				12334 1015933.65472743
				12335 1015934.9359074
				12336 1015936.21708736
				12337 1015937.49826733
				12338 1015938.77944729
				12339 1015940.06062726
				12340 1015941.34180722
				12341 1015942.62298719
				12342 1015943.90416716
				12343 1015945.18534712
				12344 1015946.46652709
				12345 1015947.74770705
				12346 1015949.02888702
				12347 1015950.31006698
				12348 1015951.59124695
				12349 1015952.87242691
				12350 1015954.15360688
				12351 1015955.43478684
				12352 1015956.71596681
				12353 1015957.99714678
				12354 1015959.27832674
				12355 1015960.55950671
				12356 1015961.84068667
				12357 1015963.12186664
				12358 1015964.4030466
				12359 1015965.68422657
				12360 1015966.96540653
				12361 1015968.2465865
				12362 1015969.52776646
				12363 1015970.80894643
				12364 1015972.0901264
				12365 1015973.37130636
				12366 1015974.65248633
				12367 1015975.93366629
				12368 1015977.21484626
				12369 1015978.49602622
				12370 1015979.77720619
				12371 1015981.05838615
				12372 1015982.33956612
				12373 1015983.62074608
				12374 1015984.90192605
				12375 1015986.18310602
				12376 1015987.46428598
				12377 1015988.74546595
				12378 1015990.02664591
				12379 1015991.30782588
				12380 1015992.58900584
				12381 1015993.87018581
				12382 1015995.15136577
				12383 1015996.43254574
				12384 1015997.7137257
				12385 1015998.99490567
				12386 1016000.27608564
				12387 1016001.5572656
				12388 1016002.83844557
				12389 1016004.11962553
				12390 1016005.4008055
				12391 1016006.68198546
				12392 1016007.96316543
				12393 1016009.24434539
				12394 1016010.52552536
				12395 1016011.80670532
				12396 1016013.08788529
				12397 1016014.36906526
				12398 1016015.65024522
				12399 1016016.93142519
				12400 1016018.21260515
				12401 1016019.49378512
				12402 1016020.77496508
				12403 1016022.05614505
				12404 1016023.33732501
				12405 1016024.61850498
				12406 1016025.89968494
				12407 1016027.18086491
				12408 1016028.46204488
				12409 1016029.74322484
				12410 1016031.02440481
				12411 1016032.30558477
				12412 1016033.58676474
				12413 1016034.8679447
				12414 1016036.14912467
				12415 1016037.43030463
				12416 1016038.7114846
				12417 1016039.99266456
				12418 1016041.27384453
				12419 1016042.5550245
				12420 1016043.83620446
				12421 1016045.11738443
				12422 1016046.39856439
				12423 1016047.67974436
				12424 1016048.96092432
				12425 1016050.24210429
				12426 1016051.52328425
				12427 1016052.80446422
				12428 1016054.08564418
				12429 1016055.36682415
				12430 1016056.64800412
				12431 1016057.92918408
				12432 1016059.21036405
				12433 1016060.49154401
				12434 1016061.77272398
				12435 1016063.05390394
				12436 1016064.33508391
				12437 1016065.61626387
				12438 1016066.89744384
				12439 1016068.1786238
				12440 1016069.45980377
				12441 1016070.74098374
				12442 1016072.0221637
				12443 1016073.30334367
				12444 1016074.58452363
				12445 1016075.8657036
				12446 1016077.14688356
				12447 1016078.42806353
				12448 1016079.70924349
				12449 1016080.99042346
				12450 1016082.27160342
				12451 1016083.55278339
				12452 1016084.83396336
				12453 1016086.11514332
				12454 1016087.39632329
				12455 1016088.67750325
				12456 1016089.95868322
				12457 1016091.23986318
				12458 1016092.52104315
				12459 1016093.80222311
				12460 1016095.08340308
				12461 1016096.36458304
				12462 1016097.64576301
				12463 1016098.92694298
				12464 1016100.20812294
				12465 1016101.48930291
				12466 1016102.77048287
				12467 1016104.05166284
				12468 1016105.3328428
				12469 1016106.61402277
				12470 1016107.89520273
				12471 1016109.1763827
				12472 1016110.45756266
				12473 1016111.73874263
				12474 1016113.0199226
				12475 1016114.30110256
				12476 1016115.58228253
				12477 1016116.86346249
				12478 1016118.14464246
				12479 1016119.42582242
				12480 1016120.70700239
				12481 1016121.98818235
				12482 1016123.26936232
				12483 1016124.55054228
				12484 1016125.83172225
				12485 1016127.11290221
				12486 1016128.39408218
				12487 1016129.67526215
				12488 1016130.95644211
				12489 1016132.23762208
				12490 1016133.51880204
				12491 1016134.79998201
				12492 1016136.08116197
				12493 1016137.36234194
				12494 1016138.6435219
				12495 1016139.92470187
				12496 1016141.20588183
				12497 1016142.4870618
				12498 1016143.76824177
				12499 1016145.04942173
				12500 1016146.3306017
				12501 1016147.61178166
				12502 1016148.89296163
				12503 1016150.17414159
				12504 1016151.45532156
				12505 1016152.73650152
				12506 1016154.01768149
				12507 1016155.29886145
				12508 1016156.58004142
				12509 1016157.86122139
				12510 1016159.14240135
				12511 1016160.42358132
				12512 1016161.70476128
				12513 1016162.98594125
				12514 1016164.26712121
				12515 1016165.54830118
				12516 1016166.82948114
				12517 1016168.11066111
				12518 1016169.39184108
				12519 1016170.67302104
				12520 1016171.95420101
				12521 1016173.23538097
				12522 1016174.51656094
				12523 1016175.7977409
				12524 1016177.07892087
				12525 1016178.36010083
				12526 1016179.6412808
				12527 1016180.92246076
				12528 1016182.20364073
				12529 1016183.4848207
				12530 1016184.76600066
				12531 1016186.04718063
				12532 1016187.32836059
				12533 1016188.60954056
				12534 1016189.89072052
				12535 1016191.17190049
				12536 1016192.45308045
				12537 1016193.73426042
				12538 1016195.01544038
				12539 1016196.29662035
				12540 1016197.57780031
				12541 1016198.85898028
				12542 1016200.14016025
				12543 1016201.42134021
				12544 1016202.70252018
				12545 1016203.98370014
				12546 1016205.26488011
				12547 1016206.54606007
				12548 1016207.82724004
				12549 1016209.10842
				12550 1016210.38959997
				12551 1016211.67077993
				12552 1016212.9519599
				12553 1016214.23313987
				12554 1016215.51431983
				12555 1016216.7954998
				12556 1016218.07667976
				12557 1016219.35785973
				12558 1016220.63903969
				12559 1016221.92021966
				12560 1016223.20139962
				12561 1016224.48257959
				12562 1016225.76375955
				12563 1016227.04493952
				12564 1016228.32611949
				12565 1016229.60729945
				12566 1016230.88847942
				12567 1016232.16965938
				12568 1016233.45083935
				12569 1016234.73201931
				12570 1016236.01319928
				12571 1016237.29437924
				12572 1016238.57555921
				12573 1016239.85673917
				12574 1016241.13791914
				12575 1016242.41909911
				12576 1016243.70027907
				12577 1016244.98145904
				12578 1016246.262639
				12579 1016247.54381897
				12580 1016248.82499893
				12581 1016250.1061789
				12582 1016251.38735886
				12583 1016252.66853883
				12584 1016253.94971879
				12585 1016255.23089876
				12586 1016256.51207873
				12587 1016257.79325869
				12588 1016259.07443866
				12589 1016260.35561862
				12590 1016261.63679859
				12591 1016262.91797855
				12592 1016264.19915852
				12593 1016265.48033848
				12594 1016266.76151845
				12595 1016268.04269841
				12596 1016269.32387838
				12597 1016270.60505835
				12598 1016271.88623831
				12599 1016273.16741828
				12600 1016274.44859824
				12601 1016275.72977821
				12602 1016277.01095817
				12603 1016278.29213814
				12604 1016279.5733181
				12605 1016280.85449807
				12606 1016282.13567803
				12607 1016283.416858
				12608 1016284.69803797
				12609 1016285.97921793
				12610 1016287.2603979
				12611 1016288.54157786
				12612 1016289.82275783
				12613 1016291.10393779
				12614 1016292.38511776
				12615 1016293.66629772
				12616 1016294.94747769
				12617 1016296.22865765
				12618 1016297.50983762
				12619 1016298.79101759
				12620 1016300.07219755
				12621 1016301.35337752
				12622 1016302.63455748
				12623 1016303.91573745
				12624 1016305.19691741
				12625 1016306.47809738
				12626 1016307.75927734
				12627 1016309.04045731
				12628 1016310.32163727
				12629 1016311.60281724
				12630 1016312.88399721
				12631 1016314.16517717
				12632 1016315.44635714
				12633 1016316.7275371
				12634 1016318.00871707
				12635 1016319.28989703
				12636 1016320.571077
				12637 1016321.85225696
				12638 1016323.13343693
				12639 1016324.41461689
				12640 1016325.69579686
				12641 1016326.97697683
				12642 1016328.25815679
				12643 1016329.53933676
				12644 1016330.82051672
				12645 1016332.10169669
				12646 1016333.38287665
				12647 1016334.66405662
				12648 1016335.94523658
				12649 1016337.22641655
				12650 1016338.50759651
				12651 1016339.78877648
				12652 1016341.06995645
				12653 1016342.35113641
				12654 1016343.63231638
				12655 1016344.91349634
				12656 1016346.19467631
				12657 1016347.47585627
				12658 1016348.75703624
				12659 1016350.0382162
				12660 1016351.31939617
				12661 1016352.60057613
				12662 1016353.8817561
				12663 1016355.16293607
				12664 1016356.44411603
				12665 1016357.725296
				12666 1016359.00647596
				12667 1016360.28765593
				12668 1016361.56883589
				12669 1016362.85001586
				12670 1016364.13119582
				12671 1016365.41237579
				12672 1016366.69355575
				12673 1016367.97473572
				12674 1016369.25591569
				12675 1016370.53709565
				12676 1016371.81827562
				12677 1016373.09945558
				12678 1016374.38063555
				12679 1016375.66181551
				12680 1016376.94299548
				12681 1016378.22417544
				12682 1016379.50535541
				12683 1016380.78653537
				12684 1016382.06771534
				12685 1016383.34889531
				12686 1016384.63007527
				12687 1016385.91125524
				12688 1016387.1924352
				12689 1016388.47361517
				12690 1016389.75479513
				12691 1016391.0359751
				12692 1016392.31715506
				12693 1016393.59833503
				12694 1016394.87951499
				12695 1016396.16069496
				12696 1016397.44187493
				12697 1016398.72305489
				12698 1016400.00423486
				12699 1016401.28541482
				12700 1016402.56659479
				12701 1016403.84777475
				12702 1016405.12895472
				12703 1016406.41013468
				12704 1016407.69131465
				12705 1016408.97249461
				12706 1016410.25367458
				12707 1016411.53485455
				12708 1016412.81603451
				12709 1016414.09721448
				12710 1016415.37839444
				12711 1016416.65957441
				12712 1016417.94075437
				12713 1016419.22193434
				12714 1016420.5031143
				12715 1016421.78429427
				12716 1016423.06547423
				12717 1016424.3466542
				12718 1016425.62783417
				12719 1016426.90901413
				12720 1016428.1901941
				12721 1016429.47137406
				12722 1016430.75255403
				12723 1016432.03373399
				12724 1016433.31491396
				12725 1016434.59609392
				12726 1016435.87727389
				12727 1016437.15845385
				12728 1016438.43963382
				12729 1016439.72081379
				12730 1016441.00199375
				12731 1016442.28317372
				12732 1016443.56435368
				12733 1016444.84553365
				12734 1016446.12671361
				12735 1016447.40789358
				12736 1016448.68907354
				12737 1016449.97025351
				12738 1016451.25143347
				12739 1016452.53261344
				12740 1016453.81379341
				12741 1016455.09497337
				12742 1016456.37615334
				12743 1016457.6573333
				12744 1016458.93851327
				12745 1016460.21969323
				12746 1016461.5008732
				12747 1016462.78205316
				12748 1016464.06323313
				12749 1016465.34441309
				12750 1016466.62559306
				12751 1016467.90677303
				12752 1016469.18795299
				12753 1016470.46913296
				12754 1016471.75031292
				12755 1016473.03149289
				12756 1016474.31267285
				12757 1016475.59385282
				12758 1016476.87503278
				12759 1016478.15621275
				12760 1016479.43739271
				12761 1016480.71857268
				12762 1016481.99975265
				12763 1016483.28093261
				12764 1016484.56211258
				12765 1016485.84329254
				12766 1016487.12447251
				12767 1016488.40565247
				12768 1016489.68683244
				12769 1016490.9680124
				12770 1016492.24919237
				12771 1016493.53037233
				12772 1016494.8115523
				12773 1016496.09273227
				12774 1016497.37391223
				12775 1016498.6550922
				12776 1016499.93627216
				12777 1016501.21745213
				12778 1016502.49863209
				12779 1016503.77981206
				12780 1016505.06099202
				12781 1016506.34217199
				12782 1016507.62335195
				12783 1016508.90453192
				12784 1016510.18571189
				12785 1016511.46689185
				12786 1016512.74807182
				12787 1016514.02925178
				12788 1016515.31043175
				12789 1016516.59161171
				12790 1016517.87279168
				12791 1016519.15397164
				12792 1016520.43515161
				12793 1016521.71633157
				12794 1016522.99751154
				12795 1016524.27869151
				12796 1016525.55987147
				12797 1016526.84105144
				12798 1016528.1222314
				12799 1016529.40341137
				12800 1016530.68459133
				12801 1016531.9657713
				12802 1016533.24695126
				12803 1016534.52813123
				12804 1016535.80931119
				12805 1016537.09049116
				12806 1016538.37167113
				12807 1016539.65285109
				12808 1016540.93403106
				12809 1016542.21521102
				12810 1016543.49639099
				12811 1016544.77757095
				12812 1016546.05875092
				12813 1016547.33993088
				12814 1016548.62111085
				12815 1016549.90229081
				12816 1016551.18347078
				12817 1016552.46465075
				12818 1016553.74583071
				12819 1016555.02701068
				12820 1016556.30819064
				12821 1016557.58937061
				12822 1016558.87055057
				12823 1016560.15173054
				12824 1016561.4329105
				12825 1016562.71409047
				12826 1016563.99527043
				12827 1016565.2764504
				12828 1016566.55763037
				12829 1016567.83881033
				12830 1016569.1199903
				12831 1016570.40117026
				12832 1016571.68235023
				12833 1016572.96353019
				12834 1016574.24471016
				12835 1016575.52589012
				12836 1016576.80707009
				12837 1016578.08825005
				12838 1016579.36943002
				12839 1016580.65060999
				12840 1016581.93178995
				12841 1016583.21296992
				12842 1016584.49414988
				12843 1016585.77532985
				12844 1016587.05650981
				12845 1016588.33768978
				12846 1016589.61886974
				12847 1016590.90004971
				12848 1016592.18122967
				12849 1016593.46240964
				12850 1016594.74358961
				12851 1016596.02476957
				12852 1016597.30594954
				12853 1016598.5871295
				12854 1016599.86830947
				12855 1016601.14948943
				12856 1016602.4306694
				12857 1016603.71184936
				12858 1016604.99302933
				12859 1016606.27420929
				12860 1016607.55538926
				12861 1016608.83656923
				12862 1016610.11774919
				12863 1016611.39892916
				12864 1016612.68010912
				12865 1016613.96128909
				12866 1016615.24246905
				12867 1016616.52364902
				12868 1016617.80482898
				12869 1016619.08600895
				12870 1016620.36718891
				12871 1016621.64836888
				12872 1016622.92954885
				12873 1016624.21072881
				12874 1016625.49190878
				12875 1016626.77308874
				12876 1016628.05426871
				12877 1016629.33544867
				12878 1016630.61662864
				12879 1016631.8978086
				12880 1016633.17898857
				12881 1016634.46016853
				12882 1016635.7413485
				12883 1016637.02252847
				12884 1016638.30370843
				12885 1016639.5848884
				12886 1016640.86606836
				12887 1016642.14724833
				12888 1016643.42842829
				12889 1016644.70960826
				12890 1016645.99078822
				12891 1016647.27196819
				12892 1016648.55314815
				12893 1016649.83432812
				12894 1016651.11550809
				12895 1016652.39668805
				12896 1016653.67786802
				12897 1016654.95904798
				12898 1016656.24022795
				12899 1016657.52140791
				12900 1016658.80258788
				12901 1016660.08376784
				12902 1016661.36494781
				12903 1016662.64612777
				12904 1016663.92730774
				12905 1016665.20848771
				12906 1016666.48966767
				12907 1016667.77084764
				12908 1016669.0520276
				12909 1016670.33320757
				12910 1016671.61438753
				12911 1016672.8955675
				12912 1016674.17674746
				12913 1016675.45792743
				12914 1016676.73910739
				12915 1016678.02028736
				12916 1016679.30146733
				12917 1016680.58264729
				12918 1016681.86382726
				12919 1016683.14500722
				12920 1016684.42618719
				12921 1016685.70736715
				12922 1016686.98854712
				12923 1016688.26972708
				12924 1016689.55090705
				12925 1016690.83208701
				12926 1016692.11326698
				12927 1016693.39444695
				12928 1016694.67562691
				12929 1016695.95680688
				12930 1016697.23798684
				12931 1016698.51916681
				12932 1016699.80034677
				12933 1016701.08152674
				12934 1016702.3627067
				12935 1016703.64388667
				12936 1016704.92506663
				12937 1016706.2062466
				12938 1016707.48742657
				12939 1016708.76860653
				12940 1016710.0497865
				12941 1016711.33096646
				12942 1016712.61214643
				12943 1016713.89332639
				12944 1016715.17450636
				12945 1016716.45568632
				12946 1016717.73686629
				12947 1016719.01804625
				12948 1016720.29922622
				12949 1016721.58040619
				12950 1016722.86158615
				12951 1016724.14276612
				12952 1016725.42394608
				12953 1016726.70512605
				12954 1016727.98630601
				12955 1016729.26748598
				12956 1016730.54866594
				12957 1016731.82984591
				12958 1016733.11102587
				12959 1016734.39220584
				12960 1016735.67338581
				12961 1016736.95456577
				12962 1016738.23574574
				12963 1016739.5169257
				12964 1016740.79810567
				12965 1016742.07928563
				12966 1016743.3604656
				12967 1016744.64164556
				12968 1016745.92282553
				12969 1016747.20400549
				12970 1016748.48518546
				12971 1016749.76636543
				12972 1016751.04754539
				12973 1016752.32872536
				12974 1016753.60990532
				12975 1016754.89108529
				12976 1016756.17226525
				12977 1016757.45344522
				12978 1016758.73462518
				12979 1016760.01580515
				12980 1016761.29698511
				12981 1016762.57816508
				12982 1016763.85934505
				12983 1016765.14052501
				12984 1016766.42170498
				12985 1016767.70288494
				12986 1016768.98406491
				12987 1016770.26524487
				12988 1016771.54642484
				12989 1016772.8276048
				12990 1016774.10878477
				12991 1016775.38996473
				12992 1016776.6711447
				12993 1016777.95232467
				12994 1016779.23350463
				12995 1016780.5146846
				12996 1016781.79586456
				12997 1016783.07704453
				12998 1016784.35822449
				12999 1016785.63940446
				13000 1016786.92058442
				13001 1016788.20176439
				13002 1016789.48294435
				13003 1016790.76412432
				13004 1016792.04530429
				13005 1016793.32648425
				13006 1016794.60766422
				13007 1016795.88884418
				13008 1016797.17002415
				13009 1016798.45120411
				13010 1016799.73238408
				13011 1016801.01356404
				13012 1016802.29474401
				13013 1016803.57592397
				13014 1016804.85710394
				13015 1016806.13828391
				13016 1016807.41946387
				13017 1016808.70064384
				13018 1016809.9818238
				13019 1016811.26300377
				13020 1016812.54418373
				13021 1016813.8253637
				13022 1016815.10654366
				13023 1016816.38772363
				13024 1016817.66890359
				13025 1016818.95008356
				13026 1016820.23126353
				13027 1016821.51244349
				13028 1016822.79362346
				13029 1016824.07480342
				13030 1016825.35598339
				13031 1016826.63716335
				13032 1016827.91834332
				13033 1016829.19952328
				13034 1016830.48070325
				13035 1016831.76188321
				13036 1016833.04306318
				13037 1016834.32424315
				13038 1016835.60542311
				13039 1016836.88660308
				13040 1016838.16778304
				13041 1016839.44896301
				13042 1016840.73014297
				13043 1016842.01132294
				13044 1016843.2925029
				13045 1016844.57368287
				13046 1016845.85486283
				13047 1016847.1360428
				13048 1016848.41722277
				13049 1016849.69840273
				13050 1016850.9795827
				13051 1016852.26076266
				13052 1016853.54194263
				13053 1016854.82312259
				13054 1016856.10430256
				13055 1016857.38548252
				13056 1016858.66666249
				13057 1016859.94784245
				13058 1016861.22902242
				13059 1016862.51020239
				13060 1016863.79138235
				13061 1016865.07256232
				13062 1016866.35374228
				13063 1016867.63492225
				13064 1016868.91610221
				13065 1016870.19728218
				13066 1016871.47846214
				13067 1016872.75964211
				13068 1016874.04082207
				13069 1016875.32200204
				13070 1016876.60318201
				13071 1016877.88436197
				13072 1016879.16554194
				13073 1016880.4467219
				13074 1016881.72790187
				13075 1016883.00908183
				13076 1016884.2902618
				13077 1016885.57144176
				13078 1016886.85262173
				13079 1016888.13380169
				13080 1016889.41498166
				13081 1016890.69616163
				13082 1016891.97734159
				13083 1016893.25852156
				13084 1016894.53970152
				13085 1016895.82088149
				13086 1016897.10206145
				13087 1016898.38324142
				13088 1016899.66442138
				13089 1016900.94560135
				13090 1016902.22678131
				13091 1016903.50796128
				13092 1016904.78914125
				13093 1016906.07032121
				13094 1016907.35150118
				13095 1016908.63268114
				13096 1016909.91386111
				13097 1016911.19504107
				13098 1016912.47622104
				13099 1016913.757401
				13100 1016915.03858097
				13101 1016916.31976093
				13102 1016917.6009409
				13103 1016918.88212087
				13104 1016920.16330083
				13105 1016921.4444808
				13106 1016922.72566076
				13107 1016924.00684073
				13108 1016925.28802069
				13109 1016926.56920066
				13110 1016927.85038062
				13111 1016929.13156059
				13112 1016930.41274055
				13113 1016931.69392052
				13114 1016932.97510049
				13115 1016934.25628045
				13116 1016935.53746042
				13117 1016936.81864038
				13118 1016938.09982035
				13119 1016939.38100031
				13120 1016940.66218028
				13121 1016941.94336024
				13122 1016943.22454021
				13123 1016944.50572017
				13124 1016945.78690014
				13125 1016947.06808011
				13126 1016948.34926007
				13127 1016949.63044004
				13128 1016950.91162
				13129 1016952.19279997
				13130 1016953.47397993
				13131 1016954.7551599
				13132 1016956.03633986
				13133 1016957.31751983
				13134 1016958.59869979
				13135 1016959.87987976
				13136 1016961.16105973
				13137 1016962.44223969
				13138 1016963.72341966
				13139 1016965.00459962
				13140 1016966.28577959
				13141 1016967.56695955
				13142 1016968.84813952
				13143 1016970.12931948
				13144 1016971.41049945
				13145 1016972.69167941
				13146 1016973.97285938
				13147 1016975.25403935
				13148 1016976.53521931
				13149 1016977.81639928
				13150 1016979.09757924
				13151 1016980.37875921
				13152 1016981.65993917
				13153 1016982.94111914
				13154 1016984.2222991
				13155 1016985.50347907
				13156 1016986.78465903
				13157 1016988.065839
				13158 1016989.34701897
				13159 1016990.62819893
				13160 1016991.9093789
				13161 1016993.19055886
				13162 1016994.47173883
				13163 1016995.75291879
				13164 1016997.03409876
				13165 1016998.31527872
				13166 1016999.59645869
				13167 1017000.87763865
				13168 1017002.15881862
				13169 1017003.43999859
				13170 1017004.72117855
				13171 1017006.00235852
				13172 1017007.28353848
				13173 1017008.56471845
				13174 1017009.84589841
				13175 1017011.12707838
				13176 1017012.40825834
				13177 1017013.68943831
				13178 1017014.97061827
				13179 1017016.25179824
				13180 1017017.53297821
				13181 1017018.81415817
				13182 1017020.09533814
				13183 1017021.3765181
				13184 1017022.65769807
				13185 1017023.93887803
				13186 1017025.220058
				13187 1017026.50123796
				13188 1017027.78241793
				13189 1017029.06359789
				13190 1017030.34477786
				13191 1017031.62595783
				13192 1017032.90713779
				13193 1017034.18831776
				13194 1017035.46949772
				13195 1017036.75067769
				13196 1017038.03185765
				13197 1017039.31303762
				13198 1017040.59421758
				13199 1017041.87539755
				13200 1017043.15657751
				13201 1017044.43775748
				13202 1017045.71893745
				13203 1017047.00011741
				13204 1017048.28129738
				13205 1017049.56247734
				13206 1017050.84365731
				13207 1017052.12483727
				13208 1017053.40601724
				13209 1017054.6871972
				13210 1017055.96837717
				13211 1017057.24955713
				13212 1017058.5307371
				13213 1017059.81191707
				13214 1017061.09309703
				13215 1017062.374277
				13216 1017063.65545696
				13217 1017064.93663693
				13218 1017066.21781689
				13219 1017067.49899686
				13220 1017068.78017682
				13221 1017070.06135679
				13222 1017071.34253675
				13223 1017072.62371672
				13224 1017073.90489669
				13225 1017075.18607665
				13226 1017076.46725662
				13227 1017077.74843658
				13228 1017079.02961655
				13229 1017080.31079651
				13230 1017081.59197648
				13231 1017082.87315644
				13232 1017084.15433641
				13233 1017085.43551637
				13234 1017086.71669634
				13235 1017087.99787631
				13236 1017089.27905627
				13237 1017090.56023624
				13238 1017091.8414162
				13239 1017093.12259617
				13240 1017094.40377613
				13241 1017095.6849561
				13242 1017096.96613606
				13243 1017098.24731603
				13244 1017099.52849599
				13245 1017100.80967596
				13246 1017102.09085593
				13247 1017103.37203589
				13248 1017104.65321586
				13249 1017105.93439582
				13250 1017107.21557579
				13251 1017108.49675575
				13252 1017109.77793572
				13253 1017111.05911568
				13254 1017112.34029565
				13255 1017113.62147561
				13256 1017114.90265558
				13257 1017116.18383555
				13258 1017117.46501551
				13259 1017118.74619548
				13260 1017120.02737544
				13261 1017121.30855541
				13262 1017122.58973537
				13263 1017123.87091534
				13264 1017125.1520953
				13265 1017126.43327527
				13266 1017127.71445523
				13267 1017128.9956352
				13268 1017130.27681517
				13269 1017131.55799513
				13270 1017132.8391751
				13271 1017134.12035506
				13272 1017135.40153503
				13273 1017136.68271499
				13274 1017137.96389496
				13275 1017139.24507492
				13276 1017140.52625489
				13277 1017141.80743485
				13278 1017143.08861482
				13279 1017144.36979479
				13280 1017145.65097475
				13281 1017146.93215472
				13282 1017148.21333468
				13283 1017149.49451465
				13284 1017150.77569461
				13285 1017152.05687458
				13286 1017153.33805454
				13287 1017154.61923451
				13288 1017155.90041447
				13289 1017157.18159444
				13290 1017158.46277441
				13291 1017159.74395437
				13292 1017161.02513434
				13293 1017162.3063143
				13294 1017163.58749427
				13295 1017164.86867423
				13296 1017166.1498542
				13297 1017167.43103416
				13298 1017168.71221413
				13299 1017169.99339409
				13300 1017171.27457406
				13301 1017172.55575403
				13302 1017173.83693399
				13303 1017175.11811396
				13304 1017176.39929392
				13305 1017177.68047389
				13306 1017178.96165385
				13307 1017180.24283382
				13308 1017181.52401378
				13309 1017182.80519375
				13310 1017184.08637371
				13311 1017185.36755368
				13312 1017186.64873365
				13313 1017187.92991361
				13314 1017189.21109358
				13315 1017190.49227354
				13316 1017191.77345351
				13317 1017193.05463347
				13318 1017194.33581344
				13319 1017195.6169934
				13320 1017196.89817337
				13321 1017198.17935333
				13322 1017199.4605333
				13323 1017200.74171327
				13324 1017202.02289323
				13325 1017203.3040732
				13326 1017204.58525316
				13327 1017205.86643313
				13328 1017207.14761309
				13329 1017208.42879306
				13330 1017209.70997302
				13331 1017210.99115299
				13332 1017212.27233295
				13333 1017213.55351292
				13334 1017214.83469289
				13335 1017216.11587285
				13336 1017217.39705282
				13337 1017218.67823278
				13338 1017219.95941275
				13339 1017221.24059271
				13340 1017222.52177268
				13341 1017223.80295264
				13342 1017225.08413261
				13343 1017226.36531257
				13344 1017227.64649254
				13345 1017228.92767251
				13346 1017230.20885247
				13347 1017231.49003244
				13348 1017232.7712124
				13349 1017234.05239237
				13350 1017235.33357233
				13351 1017236.6147523
				13352 1017237.89593226
				13353 1017239.17711223
				13354 1017240.45829219
				13355 1017241.73947216
				13356 1017243.02065213
				13357 1017244.30183209
				13358 1017245.58301206
				13359 1017246.86419202
				13360 1017248.14537199
				13361 1017249.42655195
				13362 1017250.70773192
				13363 1017251.98891188
				13364 1017253.27009185
				13365 1017254.55127181
				13366 1017255.83245178
				13367 1017257.11363175
				13368 1017258.39481171
				13369 1017259.67599168
				13370 1017260.95717164
				13371 1017262.23835161
				13372 1017263.51953157
				13373 1017264.80071154
				13374 1017266.0818915
				13375 1017267.36307147
				13376 1017268.64425143
				13377 1017269.9254314
				13378 1017271.20661137
				13379 1017272.48779133
				13380 1017273.7689713
				13381 1017275.05015126
				13382 1017276.33133123
				13383 1017277.61251119
				13384 1017278.89369116
				13385 1017280.17487112
				13386 1017281.45605109
				13387 1017282.73723105
				13388 1017284.01841102
				13389 1017285.29959099
				13390 1017286.58077095
				13391 1017287.86195092
				13392 1017289.14313088
				13393 1017290.42431085
				13394 1017291.70549081
				13395 1017292.98667078
				13396 1017294.26785074
				13397 1017295.54903071
				13398 1017296.83021067
				13399 1017298.11139064
				13400 1017299.39257061
				13401 1017300.67375057
				13402 1017301.95493054
				13403 1017303.2361105
				13404 1017304.51729047
				13405 1017305.79847043
				13406 1017307.0796504
				13407 1017308.36083036
				13408 1017309.64201033
				13409 1017310.92319029
				13410 1017312.20437026
				13411 1017313.48555023
				13412 1017314.76673019
				13413 1017316.04791016
				13414 1017317.32909012
				13415 1017318.61027009
				13416 1017319.89145005
				13417 1017321.17263002
				13418 1017322.45380998
				13419 1017323.73498995
				13420 1017325.01616991
				13421 1017326.29734988
				13422 1017327.57852985
				13423 1017328.85970981
				13424 1017330.14088978
				13425 1017331.42206974
				13426 1017332.70324971
				13427 1017333.98442967
				13428 1017335.26560964
				13429 1017336.5467896
				13430 1017337.82796957
				13431 1017339.10914953
				13432 1017340.3903295
				13433 1017341.67150947
				13434 1017342.95268943
				13435 1017344.2338694
				13436 1017345.51504936
				13437 1017346.79622933
				13438 1017348.07740929
				13439 1017349.35858926
				13440 1017350.63976922
				13441 1017351.92094919
				13442 1017353.20212915
				13443 1017354.48330912
				13444 1017355.76448909
				13445 1017357.04566905
				13446 1017358.32684902
				13447 1017359.60802898
				13448 1017360.88920895
				13449 1017362.17038891
				13450 1017363.45156888
				13451 1017364.73274884
				13452 1017366.01392881
				13453 1017367.29510877
				13454 1017368.57628874
				13455 1017369.85746871
				13456 1017371.13864867
				13457 1017372.41982864
				13458 1017373.7010086
				13459 1017374.98218857
				13460 1017376.26336853
				13461 1017377.5445485
				13462 1017378.82572846
				13463 1017380.10690843
				13464 1017381.38808839
				13465 1017382.66926836
				13466 1017383.95044833
				13467 1017385.23162829
				13468 1017386.51280826
				13469 1017387.79398822
				13470 1017389.07516819
				13471 1017390.35634815
				13472 1017391.63752812
				13473 1017392.91870808
				13474 1017394.19988805
				13475 1017395.48106801
				13476 1017396.76224798
				13477 1017398.04342795
				13478 1017399.32460791
				13479 1017400.60578788
				13480 1017401.88696784
				13481 1017403.16814781
				13482 1017404.44932777
				13483 1017405.73050774
				13484 1017407.0116877
				13485 1017408.29286767
				13486 1017409.57404763
				13487 1017410.8552276
				13488 1017412.13640757
				13489 1017413.41758753
				13490 1017414.6987675
				13491 1017415.97994746
				13492 1017417.26112743
				13493 1017418.54230739
				13494 1017419.82348736
				13495 1017421.10466732
				13496 1017422.38584729
				13497 1017423.66702725
				13498 1017424.94820722
				13499 1017426.22938719
				13500 1017427.51056715
				13501 1017428.79174712
				13502 1017430.07292708
				13503 1017431.35410705
				13504 1017432.63528701
				13505 1017433.91646698
				13506 1017435.19764694
				13507 1017436.47882691
				13508 1017437.76000687
				13509 1017439.04118684
				13510 1017440.32236681
				13511 1017441.60354677
				13512 1017442.88472674
				13513 1017444.1659067
				13514 1017445.44708667
				13515 1017446.72826663
				13516 1017448.0094466
				13517 1017449.29062656
				13518 1017450.57180653
				13519 1017451.85298649
				13520 1017453.13416646
				13521 1017454.41534643
				13522 1017455.69652639
				13523 1017456.97770636
				13524 1017458.25888632
				13525 1017459.54006629
				13526 1017460.82124625
				13527 1017462.10242622
				13528 1017463.38360618
				13529 1017464.66478615
				13530 1017465.94596611
				13531 1017467.22714608
				13532 1017468.50832605
				13533 1017469.78950601
				13534 1017471.07068598
				13535 1017472.35186594
				13536 1017473.63304591
				13537 1017474.91422587
				13538 1017476.19540584
				13539 1017477.4765858
				13540 1017478.75776577
				13541 1017480.03894573
				13542 1017481.3201257
				13543 1017482.60130567
				13544 1017483.88248563
				13545 1017485.1636656
				13546 1017486.44484556
				13547 1017487.72602553
				13548 1017489.00720549
				13549 1017490.28838546
				13550 1017491.56956542
				13551 1017492.85074539
				13552 1017494.13192535
				13553 1017495.41310532
				13554 1017496.69428529
				13555 1017497.97546525
				13556 1017499.25664522
				13557 1017500.53782518
				13558 1017501.81900515
				13559 1017503.10018511
				13560 1017504.38136508
				13561 1017505.66254504
				13562 1017506.94372501
				13563 1017508.22490497
				13564 1017509.50608494
				13565 1017510.78726491
				13566 1017512.06844487
				13567 1017513.34962484
				13568 1017514.6308048
				13569 1017515.91198477
				13570 1017517.19316473
				13571 1017518.4743447
				13572 1017519.75552466
				13573 1017521.03670463
				13574 1017522.31788459
				13575 1017523.59906456
				13576 1017524.88024453
				13577 1017526.16142449
				13578 1017527.44260446
				13579 1017528.72378442
				13580 1017530.00496439
				13581 1017531.28614435
				13582 1017532.56732432
				13583 1017533.84850428
				13584 1017535.12968425
				13585 1017536.41086421
				13586 1017537.69204418
				13587 1017538.97322415
				13588 1017540.25440411
				13589 1017541.53558408
				13590 1017542.81676404
				13591 1017544.09794401
				13592 1017545.37912397
				13593 1017546.66030394
				13594 1017547.9414839
				13595 1017549.22266387
				13596 1017550.50384383
				13597 1017551.7850238
				13598 1017553.06620377
				13599 1017554.34738373
				13600 1017555.6285637
				13601 1017556.90974366
				13602 1017558.19092363
				13603 1017559.47210359
				13604 1017560.75328356
				13605 1017562.03446352
				13606 1017563.31564349
				13607 1017564.59682345
				13608 1017565.87800342
				13609 1017567.15918339
				13610 1017568.44036335
				13611 1017569.72154332
				13612 1017571.00272328
				13613 1017572.28390325
				13614 1017573.56508321
				13615 1017574.84626318
				13616 1017576.12744314
				13617 1017577.40862311
				13618 1017578.68980307
				13619 1017579.97098304
				13620 1017581.25216301
				13621 1017582.53334297
				13622 1017583.81452294
				13623 1017585.0957029
				13624 1017586.37688287
				13625 1017587.65806283
				13626 1017588.9392428
				13627 1017590.22042276
				13628 1017591.50160273
				13629 1017592.78278269
				13630 1017594.06396266
				13631 1017595.34514263
				13632 1017596.62632259
				13633 1017597.90750256
				13634 1017599.18868252
				13635 1017600.46986249
				13636 1017601.75104245
				13637 1017603.03222242
				13638 1017604.31340238
				13639 1017605.59458235
				13640 1017606.87576231
				13641 1017608.15694228
				13642 1017609.43812225
				13643 1017610.71930221
				13644 1017612.00048218
				13645 1017613.28166214
				13646 1017614.56284211
				13647 1017615.84402207
				13648 1017617.12520204
				13649 1017618.406382
				13650 1017619.68756197
				13651 1017620.96874193
				13652 1017622.2499219
				13653 1017623.53110187
				13654 1017624.81228183
				13655 1017626.0934618
				13656 1017627.37464176
				13657 1017628.65582173
				13658 1017629.93700169
				13659 1017631.21818166
				13660 1017632.49936162
				13661 1017633.78054159
				13662 1017635.06172155
				13663 1017636.34290152
				13664 1017637.62408149
				13665 1017638.90526145
				13666 1017640.18644142
				13667 1017641.46762138
				13668 1017642.74880135
				13669 1017644.02998131
				13670 1017645.31116128
				13671 1017646.59234124
				13672 1017647.87352121
				13673 1017649.15470117
				13674 1017650.43588114
				13675 1017651.71706111
				13676 1017652.99824107
				13677 1017654.27942104
				13678 1017655.560601
				13679 1017656.84178097
				13680 1017658.12296093
				13681 1017659.4041409
				13682 1017660.68532086
				13683 1017661.96650083
				13684 1017663.24768079
				13685 1017664.52886076
				13686 1017665.81004073
				13687 1017667.09122069
				13688 1017668.37240066
				13689 1017669.65358062
				13690 1017670.93476059
				13691 1017672.21594055
				13692 1017673.49712052
				13693 1017674.77830048
				13694 1017676.05948045
				13695 1017677.34066041
				13696 1017678.62184038
				13697 1017679.90302035
				13698 1017681.18420031
				13699 1017682.46538028
				13700 1017683.74656024
				13701 1017685.02774021
				13702 1017686.30892017
				13703 1017687.59010014
				13704 1017688.8712801
				13705 1017690.15246007
				13706 1017691.43364003
				13707 1017692.71482
				13708 1017693.99599997
				13709 1017695.27717993
				13710 1017696.5583599
				13711 1017697.83953986
				13712 1017699.12071983
				13713 1017700.40189979
				13714 1017701.68307976
				13715 1017702.96425972
				13716 1017704.24543969
				13717 1017705.52661965
				13718 1017706.80779962
				13719 1017708.08897958
				13720 1017709.37015955
				13721 1017710.65133952
				13722 1017711.93251948
				13723 1017713.21369945
				13724 1017714.49487941
				13725 1017715.77605938
				13726 1017717.05723934
				13727 1017718.33841931
				13728 1017719.61959927
				13729 1017720.90077924
				13730 1017722.18195921
				13731 1017723.46313917
				13732 1017724.74431914
				13733 1017726.0254991
				13734 1017727.30667907
				13735 1017728.58785903
				13736 1017729.869039
				13737 1017731.15021896
				13738 1017732.43139893
				13739 1017733.71257889
				13740 1017734.99375886
				13741 1017736.27493883
				13742 1017737.55611879
				13743 1017738.83729876
				13744 1017740.11847872
				13745 1017741.39965869
				13746 1017742.68083865
				13747 1017743.96201862
				13748 1017745.24319858
				13749 1017746.52437855
				13750 1017747.80555851
				13751 1017749.08673848
				13752 1017750.36791845
				13753 1017751.64909841
				13754 1017752.93027838
				13755 1017754.21145834
				13756 1017755.49263831
				13757 1017756.77381827
				13758 1017758.05499824
				13759 1017759.3361782
				13760 1017760.61735817
				13761 1017761.89853813
				13762 1017763.1797181
				13763 1017764.46089807
				13764 1017765.74207803
				13765 1017767.023258
				13766 1017768.30443796
				13767 1017769.58561793
				13768 1017770.86679789
				13769 1017772.14797786
				13770 1017773.42915782
				13771 1017774.71033779
				13772 1017775.99151775
				13773 1017777.27269772
				13774 1017778.55387768
				13775 1017779.83505765
				13776 1017781.11623762
				13777 1017782.39741758
				13778 1017783.67859755
				13779 1017784.95977751
				13780 1017786.24095748
				13781 1017787.52213744
				13782 1017788.80331741
				13783 1017790.08449737
				13784 1017791.36567734
				13785 1017792.6468573
				13786 1017793.92803727
				13787 1017795.20921724
				13788 1017796.4903972
				13789 1017797.77157717
				13790 1017799.05275713
				13791 1017800.3339371
				13792 1017801.61511706
				13793 1017802.89629703
				13794 1017804.17747699
				13795 1017805.45865696
				13796 1017806.73983692
				13797 1017808.02101689
				13798 1017809.30219686
				13799 1017810.58337682
				13800 1017811.86455679
				13801 1017813.14573675
				13802 1017814.42691672
				13803 1017815.70809668
				13804 1017816.98927665
				13805 1017818.27045661
				13806 1017819.55163658
				13807 1017820.83281654
				13808 1017822.11399651
				13809 1017823.39517648
				13810 1017824.67635644
				13811 1017825.95753641
				13812 1017827.23871637
				13813 1017828.51989634
				13814 1017829.8010763
				13815 1017831.08225627
				13816 1017832.36343623
				13817 1017833.6446162
				13818 1017834.92579617
				13819 1017836.20697613
				13820 1017837.4881561
				13821 1017838.76933606
				13822 1017840.05051603
				13823 1017841.33169599
				13824 1017842.61287596
				13825 1017843.89405592
				13826 1017845.17523589
				13827 1017846.45641585
				13828 1017847.73759582
				13829 1017849.01877578
				13830 1017850.29995575
				13831 1017851.58113572
				13832 1017852.86231568
				13833 1017854.14349565
				13834 1017855.42467561
				13835 1017856.70585558
				13836 1017857.98703554
				13837 1017859.26821551
				13838 1017860.54939547
				13839 1017861.83057544
				13840 1017863.1117554
				13841 1017864.39293537
				13842 1017865.67411534
				13843 1017866.9552953
				13844 1017868.23647527
				13845 1017869.51765523
				13846 1017870.7988352
				13847 1017872.08001516
				13848 1017873.36119513
				13849 1017874.64237509
				13850 1017875.92355506
				13851 1017877.20473502
				13852 1017878.48591499
				13853 1017879.76709496
				13854 1017881.04827492
				13855 1017882.32945489
				13856 1017883.61063485
				13857 1017884.89181482
				13858 1017886.17299478
				13859 1017887.45417475
				13860 1017888.73535471
				13861 1017890.01653468
				13862 1017891.29771464
				13863 1017892.57889461
				13864 1017893.86007458
				13865 1017895.14125454
				13866 1017896.42243451
				13867 1017897.70361447
				13868 1017898.98479444
				13869 1017900.2659744
				13870 1017901.54715437
				13871 1017902.82833433
				13872 1017904.1095143
				13873 1017905.39069426
				13874 1017906.67187423
				13875 1017907.9530542
				13876 1017909.23423416
				13877 1017910.51541413
				13878 1017911.79659409
				13879 1017913.07777406
				13880 1017914.35895402
				13881 1017915.64013399
				13882 1017916.92131395
				13883 1017918.20249392
				13884 1017919.48367388
				13885 1017920.76485385
				13886 1017922.04603382
				13887 1017923.32721378
				13888 1017924.60839375
				13889 1017925.88957371
				13890 1017927.17075368
				13891 1017928.45193364
				13892 1017929.73311361
				13893 1017931.01429357
				13894 1017932.29547354
				13895 1017933.5766535
				13896 1017934.85783347
				13897 1017936.13901344
				13898 1017937.4201934
				13899 1017938.70137337
				13900 1017939.98255333
				13901 1017941.2637333
				13902 1017942.54491326
				13903 1017943.82609323
				13904 1017945.10727319
				13905 1017946.38845316
				13906 1017947.66963312
				13907 1017948.95081309
				13908 1017950.23199306
				13909 1017951.51317302
				13910 1017952.79435299
				13911 1017954.07553295
				13912 1017955.35671292
				13913 1017956.63789288
				13914 1017957.91907285
				13915 1017959.20025281
				13916 1017960.48143278
				13917 1017961.76261274
				13918 1017963.04379271
				13919 1017964.32497268
				13920 1017965.60615264
				13921 1017966.88733261
				13922 1017968.16851257
				13923 1017969.44969254
				13924 1017970.7308725
				13925 1017972.01205247
				13926 1017973.29323243
				13927 1017974.5744124
				13928 1017975.85559236
				13929 1017977.13677233
				13930 1017978.4179523
				13931 1017979.69913226
				13932 1017980.98031223
				13933 1017982.26149219
				13934 1017983.54267216
				13935 1017984.82385212
				13936 1017986.10503209
				13937 1017987.38621205
				13938 1017988.66739202
				13939 1017989.94857198
				13940 1017991.22975195
				13941 1017992.51093192
				13942 1017993.79211188
				13943 1017995.07329185
				13944 1017996.35447181
				13945 1017997.63565178
				13946 1017998.91683174
				13947 1018000.19801171
				13948 1018001.47919167
				13949 1018002.76037164
				13950 1018004.0415516
				13951 1018005.32273157
				13952 1018006.60391154
				13953 1018007.8850915
				13954 1018009.16627147
				13955 1018010.44745143
				13956 1018011.7286314
				13957 1018013.00981136
				13958 1018014.29099133
				13959 1018015.57217129
				13960 1018016.85335126
				13961 1018018.13453122
				13962 1018019.41571119
				13963 1018020.69689116
				13964 1018021.97807112
				13965 1018023.25925109
				13966 1018024.54043105
				13967 1018025.82161102
				13968 1018027.10279098
				13969 1018028.38397095
				13970 1018029.66515091
				13971 1018030.94633088
				13972 1018032.22751084
				13973 1018033.50869081
				13974 1018034.78987078
				13975 1018036.07105074
				13976 1018037.35223071
				13977 1018038.63341067
				13978 1018039.91459064
				13979 1018041.1957706
				13980 1018042.47695057
				13981 1018043.75813053
				13982 1018045.0393105
				13983 1018046.32049046
				13984 1018047.60167043
				13985 1018048.8828504
				13986 1018050.16403036
				13987 1018051.44521033
				13988 1018052.72639029
				13989 1018054.00757026
				13990 1018055.28875022
				13991 1018056.56993019
				13992 1018057.85111015
				13993 1018059.13229012
				13994 1018060.41347008
				13995 1018061.69465005
				13996 1018062.97583002
				13997 1018064.25700998
				13998 1018065.53818995
				13999 1018066.81936991
				14000 1018068.10054988
				14001 1018069.38172984
				14002 1018070.66290981
				14003 1018071.94408977
				14004 1018073.22526974
				14005 1018074.5064497
				14006 1018075.78762967
				14007 1018077.06880964
				14008 1018078.3499896
				14009 1018079.63116957
				14010 1018080.91234953
				14011 1018082.1935295
				14012 1018083.47470946
				14013 1018084.75588943
				14014 1018086.03706939
				14015 1018087.31824936
				14016 1018088.59942932
				14017 1018089.88060929
				14018 1018091.16178926
				14019 1018092.44296922
				14020 1018093.72414919
				14021 1018095.00532915
				14022 1018096.28650912
				14023 1018097.56768908
				14024 1018098.84886905
				14025 1018100.13004901
				14026 1018101.41122898
				14027 1018102.69240894
				14028 1018103.97358891
				14029 1018105.25476888
				14030 1018106.53594884
				14031 1018107.81712881
				14032 1018109.09830877
				14033 1018110.37948874
				14034 1018111.6606687
				14035 1018112.94184867
				14036 1018114.22302863
				14037 1018115.5042086
				14038 1018116.78538856
				14039 1018118.06656853
				14040 1018119.3477485
				14041 1018120.62892846
				14042 1018121.91010843
				14043 1018123.19128839
				14044 1018124.47246836
				14045 1018125.75364832
				14046 1018127.03482829
				14047 1018128.31600825
				14048 1018129.59718822
				14049 1018130.87836818
				14050 1018132.15954815
				14051 1018133.44072812
				14052 1018134.72190808
				14053 1018136.00308805
				14054 1018137.28426801
				14055 1018138.56544798
				14056 1018139.84662794
				14057 1018141.12780791
				14058 1018142.40898787
				14059 1018143.69016784
				14060 1018144.9713478
				14061 1018146.25252777
				14062 1018147.53370774
				14063 1018148.8148877
				14064 1018150.09606767
				14065 1018151.37724763
				14066 1018152.6584276
				14067 1018153.93960756
				14068 1018155.22078753
				14069 1018156.50196749
				14070 1018157.78314746
				14071 1018159.06432742
				14072 1018160.34550739
				14073 1018161.62668736
				14074 1018162.90786732
				14075 1018164.18904729
				14076 1018165.47022725
				14077 1018166.75140722
				14078 1018168.03258718
				14079 1018169.31376715
				14080 1018170.59494711
				14081 1018171.87612708
				14082 1018173.15730704
				14083 1018174.43848701
				14084 1018175.71966698
				14085 1018177.00084694
				14086 1018178.28202691
				14087 1018179.56320687
				14088 1018180.84438684
				14089 1018182.1255668
				14090 1018183.40674677
				14091 1018184.68792673
				14092 1018185.9691067
				14093 1018187.25028666
				14094 1018188.53146663
				14095 1018189.8126466
				14096 1018191.09382656
				14097 1018192.37500653
				14098 1018193.65618649
				14099 1018194.93736646
				14100 1018196.21854642
				14101 1018197.49972639
				14102 1018198.78090635
				14103 1018200.06208632
				14104 1018201.34326628
				14105 1018202.62444625
				14106 1018203.90562622
				14107 1018205.18680618
				14108 1018206.46798615
				14109 1018207.74916611
				14110 1018209.03034608
				14111 1018210.31152604
				14112 1018211.59270601
				14113 1018212.87388597
				14114 1018214.15506594
				14115 1018215.4362459
				14116 1018216.71742587
				14117 1018217.99860584
				14118 1018219.2797858
				14119 1018220.56096577
				14120 1018221.84214573
				14121 1018223.1233257
				14122 1018224.40450566
				14123 1018225.68568563
				14124 1018226.96686559
				14125 1018228.24804556
				14126 1018229.52922552
				14127 1018230.81040549
				14128 1018232.09158546
				14129 1018233.37276542
				14130 1018234.65394539
				14131 1018235.93512535
				14132 1018237.21630532
				14133 1018238.49748528
				14134 1018239.77866525
				14135 1018241.05984521
				14136 1018242.34102518
				14137 1018243.62220514
				14138 1018244.90338511
				14139 1018246.18456508
				14140 1018247.46574504
				14141 1018248.74692501
				14142 1018250.02810497
				14143 1018251.30928494
				14144 1018252.5904649
				14145 1018253.87164487
				14146 1018255.15282483
				14147 1018256.4340048
				14148 1018257.71518476
				14149 1018258.99636473
				14150 1018260.2775447
				14151 1018261.55872466
				14152 1018262.83990463
				14153 1018264.12108459
				14154 1018265.40226456
				14155 1018266.68344452
				14156 1018267.96462449
				14157 1018269.24580445
				14158 1018270.52698442
				14159 1018271.80816438
				14160 1018273.08934435
				14161 1018274.37052432
				14162 1018275.65170428
				14163 1018276.93288425
				14164 1018278.21406421
				14165 1018279.49524418
				14166 1018280.77642414
				14167 1018282.05760411
				14168 1018283.33878407
				14169 1018284.61996404
				14170 1018285.901144
				14171 1018287.18232397
				14172 1018288.46350394
				14173 1018289.7446839
				14174 1018291.02586387
				14175 1018292.30704383
				14176 1018293.5882238
				14177 1018294.86940376
				14178 1018296.15058373
				14179 1018297.43176369
				14180 1018298.71294366
				14181 1018299.99412362
				14182 1018301.27530359
				14183 1018302.55648356
				14184 1018303.83766352
				14185 1018305.11884349
				14186 1018306.40002345
				14187 1018307.68120342
				14188 1018308.96238338
				14189 1018310.24356335
				14190 1018311.52474331
				14191 1018312.80592328
				14192 1018314.08710324
				14193 1018315.36828321
				14194 1018316.64946318
				14195 1018317.93064314
				14196 1018319.21182311
				14197 1018320.49300307
				14198 1018321.77418304
				14199 1018323.055363
				14200 1018324.33654297
				14201 1018325.61772293
				14202 1018326.8989029
				14203 1018328.18008286
				14204 1018329.46126283
				14205 1018330.7424428
				14206 1018332.02362276
				14207 1018333.30480273
				14208 1018334.58598269
				14209 1018335.86716266
				14210 1018337.14834262
				14211 1018338.42952259
				14212 1018339.71070255
				14213 1018340.99188252
				14214 1018342.27306248
				14215 1018343.55424245
				14216 1018344.83542242
				14217 1018346.11660238
				14218 1018347.39778235
				14219 1018348.67896231
				14220 1018349.96014228
				14221 1018351.24132224
				14222 1018352.52250221
				14223 1018353.80368217
				14224 1018355.08486214
				14225 1018356.3660421
				14226 1018357.64722207
				14227 1018358.92840204
				14228 1018360.209582
				14229 1018361.49076197
				14230 1018362.77194193
				14231 1018364.0531219
				14232 1018365.33430186
				14233 1018366.61548183
				14234 1018367.89666179
				14235 1018369.17784176
				14236 1018370.45902172
				14237 1018371.74020169
				14238 1018373.02138166
				14239 1018374.30256162
				14240 1018375.58374159
				14241 1018376.86492155
				14242 1018378.14610152
				14243 1018379.42728148
				14244 1018380.70846145
				14245 1018381.98964141
				14246 1018383.27082138
				14247 1018384.55200134
				14248 1018385.83318131
				14249 1018387.11436128
				14250 1018388.39554124
				14251 1018389.67672121
				14252 1018390.95790117
				14253 1018392.23908114
				14254 1018393.5202611
				14255 1018394.80144107
				14256 1018396.08262103
				14257 1018397.363801
				14258 1018398.64498096
				14259 1018399.92616093
				14260 1018401.2073409
				14261 1018402.48852086
				14262 1018403.76970083
				14263 1018405.05088079
				14264 1018406.33206076
				14265 1018407.61324072
				14266 1018408.89442069
				14267 1018410.17560065
				14268 1018411.45678062
				14269 1018412.73796058
				14270 1018414.01914055
				14271 1018415.30032052
				14272 1018416.58150048
				14273 1018417.86268045
				14274 1018419.14386041
				14275 1018420.42504038
				14276 1018421.70622034
				14277 1018422.98740031
				14278 1018424.26858027
				14279 1018425.54976024
				14280 1018426.8309402
				14281 1018428.11212017
				14282 1018429.39330014
				14283 1018430.6744801
				14284 1018431.95566007
				14285 1018433.23684003
				14286 1018434.51802
				14287 1018435.79919996
				14288 1018437.08037993
				14289 1018438.36155989
				14290 1018439.64273986
				14291 1018440.92391982
				14292 1018442.20509979
				14293 1018443.48627976
				14294 1018444.76745972
				14295 1018446.04863969
				14296 1018447.32981965
				14297 1018448.61099962
				14298 1018449.89217958
				14299 1018451.17335955
				14300 1018452.45453951
				14301 1018453.73571948
				14302 1018455.01689944
				14303 1018456.29807941
				14304 1018457.57925938
				14305 1018458.86043934
				14306 1018460.14161931
				14307 1018461.42279927
				14308 1018462.70397924
				14309 1018463.9851592
				14310 1018465.26633917
				14311 1018466.54751913
				14312 1018467.8286991
				14313 1018469.10987906
				14314 1018470.39105903
				14315 1018471.672239
				14316 1018472.95341896
				14317 1018474.23459893
				14318 1018475.51577889
				14319 1018476.79695886
				14320 1018478.07813882
				14321 1018479.35931879
				14322 1018480.64049875
				14323 1018481.92167872
				14324 1018483.20285868
				14325 1018484.48403865
				14326 1018485.76521862
				14327 1018487.04639858
				14328 1018488.32757855
				14329 1018489.60875851
				14330 1018490.88993848
				14331 1018492.17111844
				14332 1018493.45229841
				14333 1018494.73347837
				14334 1018496.01465834
				14335 1018497.2958383
				14336 1018498.57701827
				14337 1018499.85819824
				14338 1018501.1393782
				14339 1018502.42055817
				14340 1018503.70173813
				14341 1018504.9829181
				14342 1018506.26409806
				14343 1018507.54527803
				14344 1018508.82645799
				14345 1018510.10763796
				14346 1018511.38881792
				14347 1018512.66999789
				14348 1018513.95117786
				14349 1018515.23235782
				14350 1018516.51353779
				14351 1018517.79471775
				14352 1018519.07589772
				14353 1018520.35707768
				14354 1018521.63825765
				14355 1018522.91943761
				14356 1018524.20061758
				14357 1018525.48179754
				14358 1018526.76297751
				14359 1018528.04415748
				14360 1018529.32533744
				14361 1018530.60651741
				14362 1018531.88769737
				14363 1018533.16887734
				14364 1018534.4500573
				14365 1018535.73123727
				14366 1018537.01241723
				14367 1018538.2935972
				14368 1018539.57477716
				14369 1018540.85595713
				14370 1018542.1371371
				14371 1018543.41831706
				14372 1018544.69949703
				14373 1018545.98067699
				14374 1018547.26185696
				14375 1018548.54303692
				14376 1018549.82421689
				14377 1018551.10539685
				14378 1018552.38657682
				14379 1018553.66775678
				14380 1018554.94893675
				14381 1018556.23011672
				14382 1018557.51129668
				14383 1018558.79247665
				14384 1018560.07365661
				14385 1018561.35483658
				14386 1018562.63601654
				14387 1018563.91719651
				14388 1018565.19837647
				14389 1018566.47955644
				14390 1018567.7607364
				14391 1018569.04191637
				14392 1018570.32309634
				14393 1018571.6042763
				14394 1018572.88545627
				14395 1018574.16663623
				14396 1018575.4478162
				14397 1018576.72899616
				14398 1018578.01017613
				14399 1018579.29135609
				14400 1018580.57253606
				14401 1018581.85371602
				14402 1018583.13489599
				14403 1018584.41607596
				14404 1018585.69725592
				14405 1018586.97843589
				14406 1018588.25961585
				14407 1018589.54079582
				14408 1018590.82197578
				14409 1018592.10315575
				14410 1018593.38433571
				14411 1018594.66551568
				14412 1018595.94669564
				14413 1018597.22787561
				14414 1018598.50905558
				14415 1018599.79023554
				14416 1018601.07141551
				14417 1018602.35259547
				14418 1018603.63377544
				14419 1018604.9149554
				14420 1018606.19613537
				14421 1018607.47731533
				14422 1018608.7584953
				14423 1018610.03967526
				14424 1018611.32085523
				14425 1018612.6020352
				14426 1018613.88321516
				14427 1018615.16439513
				14428 1018616.44557509
				14429 1018617.72675506
				14430 1018619.00793502
				14431 1018620.28911499
				14432 1018621.57029495
				14433 1018622.85147492
				14434 1018624.13265488
				14435 1018625.41383485
				14436 1018626.69501482
				14437 1018627.97619478
				14438 1018629.25737475
				14439 1018630.53855471
				14440 1018631.81973468
				14441 1018633.10091464
				14442 1018634.38209461
				14443 1018635.66327457
				14444 1018636.94445454
				14445 1018638.2256345
				14446 1018639.50681447
				14447 1018640.78799444
				14448 1018642.0691744
				14449 1018643.35035437
				14450 1018644.63153433
				14451 1018645.9127143
				14452 1018647.19389426
				14453 1018648.47507423
				14454 1018649.75625419
				14455 1018651.03743416
				14456 1018652.31861412
				14457 1018653.59979409
				14458 1018654.88097406
				14459 1018656.16215402
				14460 1018657.44333399
				14461 1018658.72451395
				14462 1018660.00569392
				14463 1018661.28687388
				14464 1018662.56805385
				14465 1018663.84923381
				14466 1018665.13041378
				14467 1018666.41159374
				14468 1018667.69277371
				14469 1018668.97395368
				14470 1018670.25513364
				14471 1018671.53631361
				14472 1018672.81749357
				14473 1018674.09867354
				14474 1018675.3798535
				14475 1018676.66103347
				14476 1018677.94221343
				14477 1018679.2233934
				14478 1018680.50457336
				14479 1018681.78575333
				14480 1018683.0669333
				14481 1018684.34811326
				14482 1018685.62929323
				14483 1018686.91047319
				14484 1018688.19165316
				14485 1018689.47283312
				14486 1018690.75401309
				14487 1018692.03519305
				14488 1018693.31637302
				14489 1018694.59755298
				14490 1018695.87873295
				14491 1018697.15991292
				14492 1018698.44109288
				14493 1018699.72227285
				14494 1018701.00345281
				14495 1018702.28463278
				14496 1018703.56581274
				14497 1018704.84699271
				14498 1018706.12817267
				14499 1018707.40935264
				14500 1018708.6905326
				14501 1018709.97171257
				14502 1018711.25289254
				14503 1018712.5340725
				14504 1018713.81525247
				14505 1018715.09643243
				14506 1018716.3776124
				14507 1018717.65879236
				14508 1018718.93997233
				14509 1018720.22115229
				14510 1018721.50233226
				14511 1018722.78351222
				14512 1018724.06469219
				14513 1018725.34587216
				14514 1018726.62705212
				14515 1018727.90823209
				14516 1018729.18941205
				14517 1018730.47059202
				14518 1018731.75177198
				14519 1018733.03295195
				14520 1018734.31413191
				14521 1018735.59531188
				14522 1018736.87649184
				14523 1018738.15767181
				14524 1018739.43885178
				14525 1018740.72003174
				14526 1018742.00121171
				14527 1018743.28239167
				14528 1018744.56357164
				14529 1018745.8447516
				14530 1018747.12593157
				14531 1018748.40711153
				14532 1018749.6882915
				14533 1018750.96947146
				14534 1018752.25065143
				14535 1018753.5318314
				14536 1018754.81301136
				14537 1018756.09419133
				14538 1018757.37537129
				14539 1018758.65655126
				14540 1018759.93773122
				14541 1018761.21891119
				14542 1018762.50009115
				14543 1018763.78127112
				14544 1018765.06245108
				14545 1018766.34363105
				14546 1018767.62481102
				14547 1018768.90599098
				14548 1018770.18717095
				14549 1018771.46835091
				14550 1018772.74953088
				14551 1018774.03071084
				14552 1018775.31189081
				14553 1018776.59307077
				14554 1018777.87425074
				14555 1018779.1554307
				14556 1018780.43661067
				14557 1018781.71779064
				14558 1018782.9989706
				14559 1018784.28015057
				14560 1018785.56133053
				14561 1018786.8425105
				14562 1018788.12369046
				14563 1018789.40487043
				14564 1018790.68605039
				14565 1018791.96723036
				14566 1018793.24841032
				14567 1018794.52959029
				14568 1018795.81077026
				14569 1018797.09195022
				14570 1018798.37313019
				14571 1018799.65431015
				14572 1018800.93549012
				14573 1018802.21667008
				14574 1018803.49785005
				14575 1018804.77903001
				14576 1018806.06020998
				14577 1018807.34138994
				14578 1018808.62256991
				14579 1018809.90374988
				14580 1018811.18492984
				14581 1018812.46610981
				14582 1018813.74728977
				14583 1018815.02846974
				14584 1018816.3096497
				14585 1018817.59082967
				14586 1018818.87200963
				14587 1018820.1531896
				14588 1018821.43436956
				14589 1018822.71554953
				14590 1018823.9967295
				14591 1018825.27790946
				14592 1018826.55908943
				14593 1018827.84026939
				14594 1018829.12144936
				14595 1018830.40262932
				14596 1018831.68380929
				14597 1018832.96498925
				14598 1018834.24616922
				14599 1018835.52734918
				14600 1018836.80852915
				14601 1018838.08970912
				14602 1018839.37088908
				14603 1018840.65206905
				14604 1018841.93324901
				14605 1018843.21442898
				14606 1018844.49560894
				14607 1018845.77678891
				14608 1018847.05796887
				14609 1018848.33914884
				14610 1018849.6203288
				14611 1018850.90150877
				14612 1018852.18268874
				14613 1018853.4638687
				14614 1018854.74504867
				14615 1018856.02622863
				14616 1018857.3074086
				14617 1018858.58858856
				14618 1018859.86976853
				14619 1018861.15094849
				14620 1018862.43212846
				14621 1018863.71330842
				14622 1018864.99448839
				14623 1018866.27566836
				14624 1018867.55684832
				14625 1018868.83802829
				14626 1018870.11920825
				14627 1018871.40038822
				14628 1018872.68156818
				14629 1018873.96274815
				14630 1018875.24392811
				14631 1018876.52510808
				14632 1018877.80628804
				14633 1018879.08746801
				14634 1018880.36864798
				14635 1018881.64982794
				14636 1018882.93100791
				14637 1018884.21218787
				14638 1018885.49336784
				14639 1018886.7745478
				14640 1018888.05572777
				14641 1018889.33690773
				14642 1018890.6180877
				14643 1018891.89926766
				14644 1018893.18044763
				14645 1018894.4616276
				14646 1018895.74280756
				14647 1018897.02398753
				14648 1018898.30516749
				14649 1018899.58634746
				14650 1018900.86752742
				14651 1018902.14870739
				14652 1018903.42988735
				14653 1018904.71106732
				14654 1018905.99224728
				14655 1018907.27342725
				14656 1018908.55460722
				14657 1018909.83578718
				14658 1018911.11696715
				14659 1018912.39814711
				14660 1018913.67932708
				14661 1018914.96050704
				14662 1018916.24168701
				14663 1018917.52286697
				14664 1018918.80404694
				14665 1018920.0852269
				14666 1018921.36640687
				14667 1018922.64758684
				14668 1018923.9287668
				14669 1018925.20994677
				14670 1018926.49112673
				14671 1018927.7723067
				14672 1018929.05348666
				14673 1018930.33466663
				14674 1018931.61584659
				14675 1018932.89702656
				14676 1018934.17820652
				14677 1018935.45938649
				14678 1018936.74056646
				14679 1018938.02174642
				14680 1018939.30292639
				14681 1018940.58410635
				14682 1018941.86528632
				14683 1018943.14646628
				14684 1018944.42764625
				14685 1018945.70882621
				14686 1018946.99000618
				14687 1018948.27118614
				14688 1018949.55236611
				14689 1018950.83354608
				14690 1018952.11472604
				14691 1018953.39590601
				14692 1018954.67708597
				14693 1018955.95826594
				14694 1018957.2394459
				14695 1018958.52062587
				14696 1018959.80180583
				14697 1018961.0829858
				14698 1018962.36416576
				14699 1018963.64534573
				14700 1018964.9265257
				14701 1018966.20770566
				14702 1018967.48888563
				14703 1018968.77006559
				14704 1018970.05124556
				14705 1018971.33242552
				14706 1018972.61360549
				14707 1018973.89478545
				14708 1018975.17596542
				14709 1018976.45714538
				14710 1018977.73832535
				14711 1018979.01950532
				14712 1018980.30068528
				14713 1018981.58186525
				14714 1018982.86304521
				14715 1018984.14422518
				14716 1018985.42540514
				14717 1018986.70658511
				14718 1018987.98776507
				14719 1018989.26894504
				14720 1018990.550125
				14721 1018991.83130497
				14722 1018993.11248494
				14723 1018994.3936649
				14724 1018995.67484487
				14725 1018996.95602483
				14726 1018998.2372048
				14727 1018999.51838476
				14728 1019000.79956473
				14729 1019002.08074469
				14730 1019003.36192466
				14731 1019004.64310462
				14732 1019005.92428459
				14733 1019007.20546456
				14734 1019008.48664452
				14735 1019009.76782449
				14736 1019011.04900445
				14737 1019012.33018442
				14738 1019013.61136438
				14739 1019014.89254435
				14740 1019016.17372431
				14741 1019017.45490428
				14742 1019018.73608424
				14743 1019020.01726421
				14744 1019021.29844418
				14745 1019022.57962414
				14746 1019023.86080411
				14747 1019025.14198407
				14748 1019026.42316404
				14749 1019027.704344
				14750 1019028.98552397
				14751 1019030.26670393
				14752 1019031.5478839
				14753 1019032.82906386
				14754 1019034.11024383
				14755 1019035.3914238
				14756 1019036.67260376
				14757 1019037.95378373
				14758 1019039.23496369
				14759 1019040.51614366
				14760 1019041.79732362
				14761 1019043.07850359
				14762 1019044.35968355
				14763 1019045.64086352
				14764 1019046.92204348
				14765 1019048.20322345
				14766 1019049.48440342
				14767 1019050.76558338
				14768 1019052.04676335
				14769 1019053.32794331
				14770 1019054.60912328
				14771 1019055.89030324
				14772 1019057.17148321
				14773 1019058.45266317
				14774 1019059.73384314
				14775 1019061.0150231
				14776 1019062.29620307
				14777 1019063.57738304
				14778 1019064.858563
				14779 1019066.13974297
				14780 1019067.42092293
				14781 1019068.7021029
				14782 1019069.98328286
				14783 1019071.26446283
				14784 1019072.54564279
				14785 1019073.82682276
				14786 1019075.10800272
				14787 1019076.38918269
				14788 1019077.67036266
				14789 1019078.95154262
				14790 1019080.23272259
				14791 1019081.51390255
				14792 1019082.79508252
				14793 1019084.07626248
				14794 1019085.35744245
				14795 1019086.63862241
				14796 1019087.91980238
				14797 1019089.20098234
				14798 1019090.48216231
				14799 1019091.76334228
				14800 1019093.04452224
				14801 1019094.32570221
				14802 1019095.60688217
				14803 1019096.88806214
				14804 1019098.1692421
				14805 1019099.45042207
				14806 1019100.73160203
				14807 1019102.012782
				14808 1019103.29396196
				14809 1019104.57514193
				14810 1019105.8563219
				14811 1019107.13750186
				14812 1019108.41868183
				14813 1019109.69986179
				14814 1019110.98104176
				14815 1019112.26222172
				14816 1019113.54340169
				14817 1019114.82458165
				14818 1019116.10576162
				14819 1019117.38694158
				14820 1019118.66812155
				14821 1019119.94930152
				14822 1019121.23048148
				14823 1019122.51166145
				14824 1019123.79284141
				14825 1019125.07402138
				14826 1019126.35520134
				14827 1019127.63638131
				14828 1019128.91756127
				14829 1019130.19874124
				14830 1019131.4799212
				14831 1019132.76110117
				14832 1019134.04228114
				14833 1019135.3234611
				14834 1019136.60464107
				14835 1019137.88582103
				14836 1019139.167001
				14837 1019140.44818096
				14838 1019141.72936093
				14839 1019143.01054089
				14840 1019144.29172086
				14841 1019145.57290082
				14842 1019146.85408079
				14843 1019148.13526076
				14844 1019149.41644072
				14845 1019150.69762069
				14846 1019151.97880065
				14847 1019153.25998062
				14848 1019154.54116058
				14849 1019155.82234055
				14850 1019157.10352051
				14851 1019158.38470048
				14852 1019159.66588044
				14853 1019160.94706041
				14854 1019162.22824038
				14855 1019163.50942034
				14856 1019164.79060031
				14857 1019166.07178027
				14858 1019167.35296024
				14859 1019168.6341402
				14860 1019169.91532017
				14861 1019171.19650013
				14862 1019172.4776801
				14863 1019173.75886006
				14864 1019175.04004003
				14865 1019176.32122
				14866 1019177.60239996
				14867 1019178.88357993
				14868 1019180.16475989
				14869 1019181.44593986
				14870 1019182.72711982
				14871 1019184.00829979
				14872 1019185.28947975
				14873 1019186.57065972
				14874 1019187.85183968
				14875 1019189.13301965
				14876 1019190.41419962
				14877 1019191.69537958
				14878 1019192.97655955
				14879 1019194.25773951
				14880 1019195.53891948
				14881 1019196.82009944
				14882 1019198.10127941
				14883 1019199.38245937
				14884 1019200.66363934
				14885 1019201.9448193
				14886 1019203.22599927
				14887 1019204.50717924
				14888 1019205.7883592
				14889 1019207.06953917
				14890 1019208.35071913
				14891 1019209.6318991
				14892 1019210.91307906
				14893 1019212.19425903
				14894 1019213.47543899
				14895 1019214.75661896
				14896 1019216.03779892
				14897 1019217.31897889
				14898 1019218.60015886
				14899 1019219.88133882
				14900 1019221.16251879
				14901 1019222.44369875
				14902 1019223.72487872
				14903 1019225.00605868
				14904 1019226.28723865
				14905 1019227.56841861
				14906 1019228.84959858
				14907 1019230.13077854
				14908 1019231.41195851
				14909 1019232.69313848
				14910 1019233.97431844
				14911 1019235.25549841
				14912 1019236.53667837
				14913 1019237.81785834
				14914 1019239.0990383
				14915 1019240.38021827
				14916 1019241.66139823
				14917 1019242.9425782
				14918 1019244.22375816
				14919 1019245.50493813
				14920 1019246.7861181
				14921 1019248.06729806
				14922 1019249.34847803
				14923 1019250.62965799
				14924 1019251.91083796
				14925 1019253.19201792
				14926 1019254.47319789
				14927 1019255.75437785
				14928 1019257.03555782
				14929 1019258.31673778
				14930 1019259.59791775
				14931 1019260.87909772
				14932 1019262.16027768
				14933 1019263.44145765
				14934 1019264.72263761
				14935 1019266.00381758
				14936 1019267.28499754
				14937 1019268.56617751
				14938 1019269.84735747
				14939 1019271.12853744
				14940 1019272.4097174
				14941 1019273.69089737
				14942 1019274.97207734
				14943 1019276.2532573
				14944 1019277.53443727
				14945 1019278.81561723
				14946 1019280.0967972
				14947 1019281.37797716
				14948 1019282.65915713
				14949 1019283.94033709
				14950 1019285.22151706
				14951 1019286.50269702
				14952 1019287.78387699
				14953 1019289.06505696
				14954 1019290.34623692
				14955 1019291.62741689
				14956 1019292.90859685
				14957 1019294.18977682
				14958 1019295.47095678
				14959 1019296.75213675
				14960 1019298.03331671
				14961 1019299.31449668
				14962 1019300.59567664
				14963 1019301.87685661
				14964 1019303.15803658
				14965 1019304.43921654
				14966 1019305.72039651
				14967 1019307.00157647
				14968 1019308.28275644
				14969 1019309.5639364
				14970 1019310.84511637
				14971 1019312.12629633
				14972 1019313.4074763
				14973 1019314.68865626
				14974 1019315.96983623
				14975 1019317.2510162
				14976 1019318.53219616
				14977 1019319.81337613
				14978 1019321.09455609
				14979 1019322.37573606
				14980 1019323.65691602
				14981 1019324.93809599
				14982 1019326.21927595
				14983 1019327.50045592
				14984 1019328.78163588
				14985 1019330.06281585
				14986 1019331.34399582
				14987 1019332.62517578
				14988 1019333.90635575
				14989 1019335.18753571
				14990 1019336.46871568
				14991 1019337.74989564
				14992 1019339.03107561
				14993 1019340.31225557
				14994 1019341.59343554
				14995 1019342.8746155
				14996 1019344.15579547
				14997 1019345.43697544
				14998 1019346.7181554
				14999 1019347.99933537
				15000 1019349.28051533
				15001 1019350.5616953
				15002 1019351.84287526
				15003 1019353.12405523
				15004 1019354.40523519
				15005 1019355.68641516
				15006 1019356.96759512
				15007 1019358.24877509
				15008 1019359.52995505
				15009 1019360.81113502
				15010 1019362.09231499
				15011 1019363.37349495
				15012 1019364.65467492
				15013 1019365.93585488
				15014 1019367.21703485
				15015 1019368.49821481
				15016 1019369.77939478
				15017 1019371.06057474
				15018 1019372.34175471
				15019 1019373.62293467
				15020 1019374.90411464
				15021 1019376.18529461
				15022 1019377.46647457
				15023 1019378.74765454
				15024 1019380.0288345
				15025 1019381.31001447
				15026 1019382.59119443
				15027 1019383.8723744
				15028 1019385.15355436
				15029 1019386.43473433
				15030 1019387.7159143
				15031 1019388.99709426
				15032 1019390.27827423
				15033 1019391.55945419
				15034 1019392.84063416
				15035 1019394.12181412
				15036 1019395.40299409
				15037 1019396.68417405
				15038 1019397.96535402
				15039 1019399.24653398
				15040 1019400.52771395
				15041 1019401.80889392
				15042 1019403.09007388
				15043 1019404.37125385
				15044 1019405.65243381
				15045 1019406.93361378
				15046 1019408.21479374
				15047 1019409.49597371
				15048 1019410.77715367
				15049 1019412.05833364
				15050 1019413.3395136
				15051 1019414.62069357
				15052 1019415.90187354
				15053 1019417.1830535
				15054 1019418.46423347
				15055 1019419.74541343
				15056 1019421.0265934
				15057 1019422.30777336
				15058 1019423.58895333
				15059 1019424.87013329
				15060 1019426.15131326
				15061 1019427.43249322
				15062 1019428.71367319
				15063 1019429.99485315
				15064 1019431.27603312
				15065 1019432.55721309
				15066 1019433.83839305
				15067 1019435.11957302
				15068 1019436.40075298
				15069 1019437.68193295
				15070 1019438.96311291
				15071 1019440.24429288
				15072 1019441.52547284
				15073 1019442.80665281
				15074 1019444.08783277
				15075 1019445.36901274
				15076 1019446.65019271
				15077 1019447.93137267
				15078 1019449.21255264
				15079 1019450.4937326
				15080 1019451.77491257
				15081 1019453.05609253
				15082 1019454.3372725
				15083 1019455.61845246
				15084 1019456.89963243
				15085 1019458.18081239
				15086 1019459.46199236
				15087 1019460.74317233
				15088 1019462.02435229
				15089 1019463.30553226
				15090 1019464.58671222
				15091 1019465.86789219
				15092 1019467.14907215
				15093 1019468.43025212
				15094 1019469.71143208
				15095 1019470.99261205
				15096 1019472.27379201
				15097 1019473.55497198
				15098 1019474.83615195
				15099 1019476.11733191
				15100 1019477.39851188
				15101 1019478.67969184
				15102 1019479.96087181
				15103 1019481.24205177
				15104 1019482.52323174
				15105 1019483.8044117
				15106 1019485.08559167
				15107 1019486.36677163
				15108 1019487.6479516
				15109 1019488.92913157
				15110 1019490.21031153
				15111 1019491.4914915
				15112 1019492.77267146
				15113 1019494.05385143
				15114 1019495.33503139
				15115 1019496.61621136
				15116 1019497.89739132
				15117 1019499.17857129
				15118 1019500.45975125
				15119 1019501.74093122
				15120 1019503.02211119
				15121 1019504.30329115
				15122 1019505.58447112
				15123 1019506.86565108
				15124 1019508.14683105
				15125 1019509.42801101
				15126 1019510.70919098
				15127 1019511.99037094
				15128 1019513.27155091
				15129 1019514.55273087
				15130 1019515.83391084
				15131 1019517.11509081
				15132 1019518.39627077
				15133 1019519.67745074
				15134 1019520.9586307
				15135 1019522.23981067
				15136 1019523.52099063
				15137 1019524.8021706
				15138 1019526.08335056
				15139 1019527.36453053
				15140 1019528.64571049
				15141 1019529.92689046
				15142 1019531.20807043
				15143 1019532.48925039
				15144 1019533.77043036
				15145 1019535.05161032
				15146 1019536.33279029
				15147 1019537.61397025
				15148 1019538.89515022
				15149 1019540.17633018
				15150 1019541.45751015
				15151 1019542.73869011
				15152 1019544.01987008
				15153 1019545.30105005
				15154 1019546.58223001
				15155 1019547.86340998
				15156 1019549.14458994
				15157 1019550.42576991
				15158 1019551.70694987
				15159 1019552.98812984
				15160 1019554.2693098
				15161 1019555.55048977
				15162 1019556.83166973
				15163 1019558.1128497
				15164 1019559.39402967
				15165 1019560.67520963
				15166 1019561.9563896
				15167 1019563.23756956
				15168 1019564.51874953
				15169 1019565.79992949
				15170 1019567.08110946
				15171 1019568.36228942
				15172 1019569.64346939
				15173 1019570.92464935
				15174 1019572.20582932
				15175 1019573.48700929
				15176 1019574.76818925
				15177 1019576.04936922
				15178 1019577.33054918
				15179 1019578.61172915
				15180 1019579.89290911
				15181 1019581.17408908
				15182 1019582.45526904
				15183 1019583.73644901
				15184 1019585.01762897
				15185 1019586.29880894
				15186 1019587.57998891
				15187 1019588.86116887
				15188 1019590.14234884
				15189 1019591.4235288
				15190 1019592.70470877
				15191 1019593.98588873
				15192 1019595.2670687
				15193 1019596.54824866
				15194 1019597.82942863
				15195 1019599.11060859
				15196 1019600.39178856
				15197 1019601.67296853
				15198 1019602.95414849
				15199 1019604.23532846
				15200 1019605.51650842
				15201 1019606.79768839
				15202 1019608.07886835
				15203 1019609.36004832
				15204 1019610.64122828
				15205 1019611.92240825
				15206 1019613.20358821
				15207 1019614.48476818
				15208 1019615.76594815
				15209 1019617.04712811
				15210 1019618.32830808
				15211 1019619.60948804
				15212 1019620.89066801
				15213 1019622.17184797
				15214 1019623.45302794
				15215 1019624.7342079
				15216 1019626.01538787
				15217 1019627.29656783
				15218 1019628.5777478
				15219 1019629.85892777
				15220 1019631.14010773
				15221 1019632.4212877
				15222 1019633.70246766
				15223 1019634.98364763
				15224 1019636.26482759
				15225 1019637.54600756
				15226 1019638.82718752
				15227 1019640.10836749
				15228 1019641.38954745
				15229 1019642.67072742
				15230 1019643.95190739
				15231 1019645.23308735
				15232 1019646.51426732
				15233 1019647.79544728
				15234 1019649.07662725
				15235 1019650.35780721
				15236 1019651.63898718
				15237 1019652.92016714
				15238 1019654.20134711
				15239 1019655.48252707
				15240 1019656.76370704
				15241 1019658.04488701
				15242 1019659.32606697
				15243 1019660.60724694
				15244 1019661.8884269
				15245 1019663.16960687
				15246 1019664.45078683
				15247 1019665.7319668
				15248 1019667.01314676
				15249 1019668.29432673
				15250 1019669.57550669
				15251 1019670.85668666
				15252 1019672.13786663
				15253 1019673.41904659
				15254 1019674.70022656
				15255 1019675.98140652
				15256 1019677.26258649
				15257 1019678.54376645
				15258 1019679.82494642
				15259 1019681.10612638
				15260 1019682.38730635
				15261 1019683.66848631
				15262 1019684.94966628
				15263 1019686.23084625
				15264 1019687.51202621
				15265 1019688.79320618
				15266 1019690.07438614
				15267 1019691.35556611
				15268 1019692.63674607
				15269 1019693.91792604
				15270 1019695.199106
				15271 1019696.48028597
				15272 1019697.76146593
				15273 1019699.0426459
				15274 1019700.32382587
				15275 1019701.60500583
				15276 1019702.8861858
				15277 1019704.16736576
				15278 1019705.44854573
				15279 1019706.72972569
				15280 1019708.01090566
				15281 1019709.29208562
				15282 1019710.57326559
				15283 1019711.85444555
				15284 1019713.13562552
				15285 1019714.41680549
				15286 1019715.69798545
				15287 1019716.97916542
				15288 1019718.26034538
				15289 1019719.54152535
				15290 1019720.82270531
				15291 1019722.10388528
				15292 1019723.38506524
				15293 1019724.66624521
				15294 1019725.94742517
				15295 1019727.22860514
				15296 1019728.50978511
				15297 1019729.79096507
				15298 1019731.07214504
				15299 1019732.353325
				15300 1019733.63450497
				15301 1019734.91568493
				15302 1019736.1968649
				15303 1019737.47804486
				15304 1019738.75922483
				15305 1019740.04040479
				15306 1019741.32158476
				15307 1019742.60276473
				15308 1019743.88394469
				15309 1019745.16512466
				15310 1019746.44630462
				15311 1019747.72748459
				15312 1019749.00866455
				15313 1019750.28984452
				15314 1019751.57102448
				15315 1019752.85220445
				15316 1019754.13338441
				15317 1019755.41456438
				15318 1019756.69574435
				15319 1019757.97692431
				15320 1019759.25810428
				15321 1019760.53928424
				15322 1019761.82046421
				15323 1019763.10164417
				15324 1019764.38282414
				15325 1019765.6640041
				15326 1019766.94518407
				15327 1019768.22636403
				15328 1019769.507544
				15329 1019770.78872397
				15330 1019772.06990393
				15331 1019773.3510839
				15332 1019774.63226386
				15333 1019775.91344383
				15334 1019777.19462379
				15335 1019778.47580376
				15336 1019779.75698372
				15337 1019781.03816369
				15338 1019782.31934365
				15339 1019783.60052362
				15340 1019784.88170359
				15341 1019786.16288355
				15342 1019787.44406352
				15343 1019788.72524348
				15344 1019790.00642345
				15345 1019791.28760341
				15346 1019792.56878338
				15347 1019793.84996334
				15348 1019795.13114331
				15349 1019796.41232327
				15350 1019797.69350324
				15351 1019798.97468321
				15352 1019800.25586317
				15353 1019801.53704314
				15354 1019802.8182231
				15355 1019804.09940307
				15356 1019805.38058303
				15357 1019806.661763
				15358 1019807.94294296
				15359 1019809.22412293
				15360 1019810.50530289
				15361 1019811.78648286
				15362 1019813.06766283
				15363 1019814.34884279
				15364 1019815.63002276
				15365 1019816.91120272
				15366 1019818.19238269
				15367 1019819.47356265
				15368 1019820.75474262
				15369 1019822.03592258
				15370 1019823.31710255
				15371 1019824.59828251
				15372 1019825.87946248
				15373 1019827.16064245
				15374 1019828.44182241
				15375 1019829.72300238
				15376 1019831.00418234
				15377 1019832.28536231
				15378 1019833.56654227
				15379 1019834.84772224
				15380 1019836.1289022
				15381 1019837.41008217
				15382 1019838.69126213
				15383 1019839.9724421
				15384 1019841.25362207
				15385 1019842.53480203
				15386 1019843.815982
				15387 1019845.09716196
				15388 1019846.37834193
				15389 1019847.65952189
				15390 1019848.94070186
				15391 1019850.22188182
				15392 1019851.50306179
				15393 1019852.78424175
				15394 1019854.06542172
				15395 1019855.34660169
				15396 1019856.62778165
				15397 1019857.90896162
				15398 1019859.19014158
				15399 1019860.47132155
				15400 1019861.75250151
				15401 1019863.03368148
				15402 1019864.31486144
				15403 1019865.59604141
				15404 1019866.87722137
				15405 1019868.15840134
				15406 1019869.43958131
				15407 1019870.72076127
				15408 1019872.00194124
				15409 1019873.2831212
				15410 1019874.56430117
				15411 1019875.84548113
				15412 1019877.1266611
				15413 1019878.40784106
				15414 1019879.68902103
				15415 1019880.97020099
				15416 1019882.25138096
				15417 1019883.53256093
				15418 1019884.81374089
				15419 1019886.09492086
				15420 1019887.37610082
				15421 1019888.65728079
				15422 1019889.93846075
				15423 1019891.21964072
				15424 1019892.50082068
				15425 1019893.78200065
				15426 1019895.06318061
				15427 1019896.34436058
				15428 1019897.62554055
				15429 1019898.90672051
				15430 1019900.18790048
				15431 1019901.46908044
				15432 1019902.75026041
				15433 1019904.03144037
				15434 1019905.31262034
				15435 1019906.5938003
				15436 1019907.87498027
				15437 1019909.15616023
				15438 1019910.4373402
				15439 1019911.71852017
				15440 1019912.99970013
				15441 1019914.2808801
				15442 1019915.56206006
				15443 1019916.84324003
				15444 1019918.12441999
				15445 1019919.40559996
				15446 1019920.68677992
				15447 1019921.96795989
				15448 1019923.24913985
				15449 1019924.53031982
				15450 1019925.81149979
				15451 1019927.09267975
				15452 1019928.37385972
				15453 1019929.65503968
				15454 1019930.93621965
				15455 1019932.21739961
				15456 1019933.49857958
				15457 1019934.77975954
				15458 1019936.06093951
				15459 1019937.34211947
				15460 1019938.62329944
				15461 1019939.90447941
				15462 1019941.18565937
				15463 1019942.46683934
				15464 1019943.7480193
				15465 1019945.02919927
				15466 1019946.31037923
				15467 1019947.5915592
				15468 1019948.87273916
				15469 1019950.15391913
				15470 1019951.43509909
				15471 1019952.71627906
				15472 1019953.99745903
				15473 1019955.27863899
				15474 1019956.55981896
				15475 1019957.84099892
				15476 1019959.12217889
				15477 1019960.40335885
				15478 1019961.68453882
				15479 1019962.96571878
				15480 1019964.24689875
				15481 1019965.52807871
				15482 1019966.80925868
				15483 1019968.09043865
				15484 1019969.37161861
				15485 1019970.65279858
				15486 1019971.93397854
				15487 1019973.21515851
				15488 1019974.49633847
				15489 1019975.77751844
				15490 1019977.0586984
				15491 1019978.33987837
				15492 1019979.62105833
				15493 1019980.9022383
				15494 1019982.18341827
				15495 1019983.46459823
				15496 1019984.7457782
				15497 1019986.02695816
				15498 1019987.30813813
				15499 1019988.58931809
				15500 1019989.87049806
				15501 1019991.15167802
				15502 1019992.43285799
				15503 1019993.71403795
				15504 1019994.99521792
				15505 1019996.27639789
				15506 1019997.55757785
				15507 1019998.83875782
				15508 1020000.11993778
				15509 1020001.40111775
				15510 1020002.68229771
				15511 1020003.96347768
				15512 1020005.24465764
				15513 1020006.52583761
				15514 1020007.80701757
				15515 1020009.08819754
				15516 1020010.36937751
				15517 1020011.65055747
				15518 1020012.93173744
				15519 1020014.2129174
				15520 1020015.49409737
				15521 1020016.77527733
				15522 1020018.0564573
				15523 1020019.33763726
				15524 1020020.61881723
				15525 1020021.89999719
				15526 1020023.18117716
				15527 1020024.46235713
				15528 1020025.74353709
				15529 1020027.02471706
				15530 1020028.30589702
				15531 1020029.58707699
				15532 1020030.86825695
				15533 1020032.14943692
				15534 1020033.43061688
				15535 1020034.71179685
				15536 1020035.99297681
				15537 1020037.27415678
				15538 1020038.55533675
				15539 1020039.83651671
				15540 1020041.11769668
				15541 1020042.39887664
				15542 1020043.68005661
				15543 1020044.96123657
				15544 1020046.24241654
				15545 1020047.5235965
				15546 1020048.80477647
				15547 1020050.08595643
				15548 1020051.3671364
				15549 1020052.64831637
				15550 1020053.92949633
				15551 1020055.2106763
				15552 1020056.49185626
				15553 1020057.77303623
				15554 1020059.05421619
				15555 1020060.33539616
				15556 1020061.61657612
				15557 1020062.89775609
				15558 1020064.17893605
				15559 1020065.46011602
				15560 1020066.74129599
				15561 1020068.02247595
				15562 1020069.30365592
				15563 1020070.58483588
				15564 1020071.86601585
				15565 1020073.14719581
				15566 1020074.42837578
				15567 1020075.70955574
				15568 1020076.99073571
				15569 1020078.27191567
				15570 1020079.55309564
				15571 1020080.83427561
				15572 1020082.11545557
				15573 1020083.39663554
				15574 1020084.6778155
				15575 1020085.95899547
				15576 1020087.24017543
				15577 1020088.5213554
				15578 1020089.80253536
				15579 1020091.08371533
				15580 1020092.36489529
				15581 1020093.64607526
				15582 1020094.92725523
				15583 1020096.20843519
				15584 1020097.48961516
				15585 1020098.77079512
				15586 1020100.05197509
				15587 1020101.33315505
				15588 1020102.61433502
				15589 1020103.89551498
				15590 1020105.17669495
				15591 1020106.45787491
				15592 1020107.73905488
				15593 1020109.02023485
				15594 1020110.30141481
				15595 1020111.58259478
				15596 1020112.86377474
				15597 1020114.14495471
				15598 1020115.42613467
				15599 1020116.70731464
				15600 1020117.9884946
				15601 1020119.26967457
				15602 1020120.55085453
				15603 1020121.8320345
				15604 1020123.11321447
				15605 1020124.39439443
				15606 1020125.6755744
				15607 1020126.95675436
				15608 1020128.23793433
				15609 1020129.51911429
				15610 1020130.80029426
				15611 1020132.08147422
				15612 1020133.36265419
				15613 1020134.64383415
				15614 1020135.92501412
				15615 1020137.20619409
				15616 1020138.48737405
				15617 1020139.76855402
				15618 1020141.04973398
				15619 1020142.33091395
				15620 1020143.61209391
				15621 1020144.89327388
				15622 1020146.17445384
				15623 1020147.45563381
				15624 1020148.73681377
				15625 1020150.01799374
				15626 1020151.29917371
				15627 1020152.58035367
				15628 1020153.86153364
				15629 1020155.1427136
				15630 1020156.42389357
				15631 1020157.70507353
				15632 1020158.9862535
				15633 1020160.26743346
				15634 1020161.54861343
				15635 1020162.82979339
				15636 1020164.11097336
				15637 1020165.39215333
				15638 1020166.67333329
				15639 1020167.95451326
				15640 1020169.23569322
				15641 1020170.51687319
				15642 1020171.79805315
				15643 1020173.07923312
				15644 1020174.36041308
				15645 1020175.64159305
				15646 1020176.92277301
				15647 1020178.20395298
				15648 1020179.48513295
				15649 1020180.76631291
				15650 1020182.04749288
				15651 1020183.32867284
				15652 1020184.60985281
				15653 1020185.89103277
				15654 1020187.17221274
				15655 1020188.4533927
				15656 1020189.73457267
				15657 1020191.01575263
				15658 1020192.2969326
				15659 1020193.57811257
				15660 1020194.85929253
				15661 1020196.1404725
				15662 1020197.42165246
				15663 1020198.70283243
				15664 1020199.98401239
				15665 1020201.26519236
				15666 1020202.54637232
				15667 1020203.82755229
				15668 1020205.10873225
				15669 1020206.38991222
				15670 1020207.67109219
				15671 1020208.95227215
				15672 1020210.23345212
				15673 1020211.51463208
				15674 1020212.79581205
				15675 1020214.07699201
				15676 1020215.35817198
				15677 1020216.63935194
				15678 1020217.92053191
				15679 1020219.20171187
				15680 1020220.48289184
				15681 1020221.76407181
				15682 1020223.04525177
				15683 1020224.32643174
				15684 1020225.6076117
				15685 1020226.88879167
				15686 1020228.16997163
				15687 1020229.4511516
				15688 1020230.73233156
				15689 1020232.01351153
				15690 1020233.29469149
				15691 1020234.57587146
				15692 1020235.85705143
				15693 1020237.13823139
				15694 1020238.41941136
				15695 1020239.70059132
				15696 1020240.98177129
				15697 1020242.26295125
				15698 1020243.54413122
				15699 1020244.82531118
				15700 1020246.10649115
				15701 1020247.38767111
				15702 1020248.66885108
				15703 1020249.95003105
				15704 1020251.23121101
				15705 1020252.51239098
				15706 1020253.79357094
				15707 1020255.07475091
				15708 1020256.35593087
				15709 1020257.63711084
				15710 1020258.9182908
				15711 1020260.19947077
				15712 1020261.48065073
				15713 1020262.7618307
				15714 1020264.04301067
				15715 1020265.32419063
				15716 1020266.6053706
				15717 1020267.88655056
				15718 1020269.16773053
				15719 1020270.44891049
				15720 1020271.73009046
				15721 1020273.01127042
				15722 1020274.29245039
				15723 1020275.57363035
				15724 1020276.85481032
				15725 1020278.13599029
				15726 1020279.41717025
				15727 1020280.69835022
				15728 1020281.97953018
				15729 1020283.26071015
				15730 1020284.54189011
				15731 1020285.82307008
				15732 1020287.10425004
				15733 1020288.38543001
				15734 1020289.66660997
				15735 1020290.94778994
				15736 1020292.22896991
				15737 1020293.51014987
				15738 1020294.79132984
				15739 1020296.0725098
				15740 1020297.35368977
				15741 1020298.63486973
				15742 1020299.9160497
				15743 1020301.19722966
				15744 1020302.47840963
				15745 1020303.75958959
				15746 1020305.04076956
				15747 1020306.32194953
				15748 1020307.60312949
				15749 1020308.88430946
				15750 1020310.16548942
				15751 1020311.44666939
				15752 1020312.72784935
				15753 1020314.00902932
				15754 1020315.29020928
				15755 1020316.57138925
				15756 1020317.85256921
				15757 1020319.13374918
				15758 1020320.41492915
				15759 1020321.69610911
				15760 1020322.97728908
				15761 1020324.25846904
				15762 1020325.53964901
				15763 1020326.82082897
				15764 1020328.10200894
				15765 1020329.3831889
				15766 1020330.66436887
				15767 1020331.94554883
				15768 1020333.2267288
				15769 1020334.50790877
				15770 1020335.78908873
				15771 1020337.0702687
				15772 1020338.35144866
				15773 1020339.63262863
				15774 1020340.91380859
				15775 1020342.19498856
				15776 1020343.47616852
				15777 1020344.75734849
				15778 1020346.03852845
				15779 1020347.31970842
				15780 1020348.60088839
				15781 1020349.88206835
				15782 1020351.16324832
				15783 1020352.44442828
				15784 1020353.72560825
				15785 1020355.00678821
				15786 1020356.28796818
				15787 1020357.56914814
				15788 1020358.85032811
				15789 1020360.13150807
				15790 1020361.41268804
				15791 1020362.69386801
				15792 1020363.97504797
				15793 1020365.25622794
				15794 1020366.5374079
				15795 1020367.81858787
				15796 1020369.09976783
				15797 1020370.3809478
				15798 1020371.66212776
				15799 1020372.94330773
				15800 1020374.22448769
				15801 1020375.50566766
				15802 1020376.78684763
				15803 1020378.06802759
				15804 1020379.34920756
				15805 1020380.63038752
				15806 1020381.91156749
				15807 1020383.19274745
				15808 1020384.47392742
				15809 1020385.75510738
				15810 1020387.03628735
				15811 1020388.31746731
				15812 1020389.59864728
				15813 1020390.87982725
				15814 1020392.16100721
				15815 1020393.44218718
				15816 1020394.72336714
				15817 1020396.00454711
				15818 1020397.28572707
				15819 1020398.56690704
				15820 1020399.848087
				15821 1020401.12926697
				15822 1020402.41044693
				15823 1020403.6916269
				15824 1020404.97280687
				15825 1020406.25398683
				15826 1020407.5351668
				15827 1020408.81634676
				15828 1020410.09752673
				15829 1020411.37870669
				15830 1020412.65988666
				15831 1020413.94106662
				15832 1020415.22224659
				15833 1020416.50342655
				15834 1020417.78460652
				15835 1020419.06578649
				15836 1020420.34696645
				15837 1020421.62814642
				15838 1020422.90932638
				15839 1020424.19050635
				15840 1020425.47168631
				15841 1020426.75286628
				15842 1020428.03404624
				15843 1020429.31522621
				15844 1020430.59640617
				15845 1020431.87758614
				15846 1020433.15876611
				15847 1020434.43994607
				15848 1020435.72112604
				15849 1020437.002306
				15850 1020438.28348597
				15851 1020439.56466593
				15852 1020440.8458459
				15853 1020442.12702586
				15854 1020443.40820583
				15855 1020444.68938579
				15856 1020445.97056576
				15857 1020447.25174573
				15858 1020448.53292569
				15859 1020449.81410566
				15860 1020451.09528562
				15861 1020452.37646559
				15862 1020453.65764555
				15863 1020454.93882552
				15864 1020456.22000548
				15865 1020457.50118545
				15866 1020458.78236541
				15867 1020460.06354538
				15868 1020461.34472535
				15869 1020462.62590531
				15870 1020463.90708528
				15871 1020465.18826524
				15872 1020466.46944521
				15873 1020467.75062517
				15874 1020469.03180514
				15875 1020470.3129851
				15876 1020471.59416507
				15877 1020472.87534503
				15878 1020474.156525
				15879 1020475.43770497
				15880 1020476.71888493
				15881 1020478.0000649
				15882 1020479.28124486
				15883 1020480.56242483
				15884 1020481.84360479
				15885 1020483.12478476
				15886 1020484.40596472
				15887 1020485.68714469
				15888 1020486.96832465
				15889 1020488.24950462
				15890 1020489.53068459
				15891 1020490.81186455
				15892 1020492.09304452
				15893 1020493.37422448
				15894 1020494.65540445
				15895 1020495.93658441
				15896 1020497.21776438
				15897 1020498.49894434
				15898 1020499.78012431
				15899 1020501.06130427
				15900 1020502.34248424
				15901 1020503.62366421
				15902 1020504.90484417
				15903 1020506.18602414
				15904 1020507.4672041
				15905 1020508.74838407
				15906 1020510.02956403
				15907 1020511.310744
				15908 1020512.59192396
				15909 1020513.87310393
				15910 1020515.15428389
				15911 1020516.43546386
				15912 1020517.71664383
				15913 1020518.99782379
				15914 1020520.27900376
				15915 1020521.56018372
				15916 1020522.84136369
				15917 1020524.12254365
				15918 1020525.40372362
				15919 1020526.68490358
				15920 1020527.96608355
				15921 1020529.24726351
				15922 1020530.52844348
				15923 1020531.80962345
				15924 1020533.09080341
				15925 1020534.37198338
				15926 1020535.65316334
				15927 1020536.93434331
				15928 1020538.21552327
				15929 1020539.49670324
				15930 1020540.7778832
				15931 1020542.05906317
				15932 1020543.34024313
				15933 1020544.6214231
				15934 1020545.90260307
				15935 1020547.18378303
				15936 1020548.464963
				15937 1020549.74614296
				15938 1020551.02732293
				15939 1020552.30850289
				15940 1020553.58968286
				15941 1020554.87086282
				15942 1020556.15204279
				15943 1020557.43322275
				15944 1020558.71440272
				15945 1020559.99558269
				15946 1020561.27676265
				15947 1020562.55794262
				15948 1020563.83912258
				15949 1020565.12030255
				15950 1020566.40148251
				15951 1020567.68266248
				15952 1020568.96384244
				15953 1020570.24502241
				15954 1020571.52620237
				15955 1020572.80738234
				15956 1020574.08856231
				15957 1020575.36974227
				15958 1020576.65092224
				15959 1020577.9321022
				15960 1020579.21328217
				15961 1020580.49446213
				15962 1020581.7756421
				15963 1020583.05682206
				15964 1020584.33800203
				15965 1020585.61918199
				15966 1020586.90036196
				15967 1020588.18154193
				15968 1020589.46272189
				15969 1020590.74390186
				15970 1020592.02508182
				15971 1020593.30626179
				15972 1020594.58744175
				15973 1020595.86862172
				15974 1020597.14980168
				15975 1020598.43098165
				15976 1020599.71216161
				15977 1020600.99334158
				15978 1020602.27452155
				15979 1020603.55570151
				15980 1020604.83688148
				15981 1020606.11806144
				15982 1020607.39924141
				15983 1020608.68042137
				15984 1020609.96160134
				15985 1020611.2427813
				15986 1020612.52396127
				15987 1020613.80514123
				15988 1020615.0863212
				15989 1020616.36750117
				15990 1020617.64868113
				15991 1020618.9298611
				15992 1020620.21104106
				15993 1020621.49222103
				15994 1020622.77340099
				15995 1020624.05458096
				15996 1020625.33576092
				15997 1020626.61694089
				15998 1020627.89812085
				15999 1020629.17930082
				16000 1020630.46048079
				16001 1020631.74166075
				16002 1020633.02284072
				16003 1020634.30402068
				16004 1020635.58520065
				16005 1020636.86638061
				16006 1020638.14756058
				16007 1020639.42874054
				16008 1020640.70992051
				16009 1020641.99110047
				16010 1020643.27228044
				16011 1020644.55346041
				16012 1020645.83464037
				16013 1020647.11582034
				16014 1020648.3970003
				16015 1020649.67818027
				16016 1020650.95936023
				16017 1020652.2405402
				16018 1020653.52172016
				16019 1020654.80290013
				16020 1020656.08408009
				16021 1020657.36526006
				16022 1020658.64644003
				16023 1020659.92761999
				16024 1020661.20879996
				16025 1020662.48997992
				16026 1020663.77115989
				16027 1020665.05233985
				16028 1020666.33351982
				16029 1020667.61469978
				16030 1020668.89587975
				16031 1020670.17705971
				16032 1020671.45823968
				16033 1020672.73941965
				16034 1020674.02059961
				16035 1020675.30177958
				16036 1020676.58295954
				16037 1020677.86413951
				16038 1020679.14531947
				16039 1020680.42649944
				16040 1020681.7076794
				16041 1020682.98885937
				16042 1020684.27003933
				16043 1020685.5512193
				16044 1020686.83239927
				16045 1020688.11357923
				16046 1020689.3947592
				16047 1020690.67593916
				16048 1020691.95711913
				16049 1020693.23829909
				16050 1020694.51947906
				16051 1020695.80065902
				16052 1020697.08183899
				16053 1020698.36301895
				16054 1020699.64419892
				16055 1020700.92537889
				16056 1020702.20655885
				16057 1020703.48773882
				16058 1020704.76891878
				16059 1020706.05009875
				16060 1020707.33127871
				16061 1020708.61245868
				16062 1020709.89363864
				16063 1020711.17481861
				16064 1020712.45599857
				16065 1020713.73717854
				16066 1020715.01835851
				16067 1020716.29953847
				16068 1020717.58071844
				16069 1020718.8618984
				16070 1020720.14307837
				16071 1020721.42425833
				16072 1020722.7054383
				16073 1020723.98661826
				16074 1020725.26779823
				16075 1020726.54897819
				16076 1020727.83015816
				16077 1020729.11133813
				16078 1020730.39251809
				16079 1020731.67369806
				16080 1020732.95487802
				16081 1020734.23605799
				16082 1020735.51723795
				16083 1020736.79841792
				16084 1020738.07959788
				16085 1020739.36077785
				16086 1020740.64195781
				16087 1020741.92313778
				16088 1020743.20431775
				16089 1020744.48549771
				16090 1020745.76667768
				16091 1020747.04785764
				16092 1020748.32903761
				16093 1020749.61021757
				16094 1020750.89139754
				16095 1020752.1725775
				16096 1020753.45375747
				16097 1020754.73493743
				16098 1020756.0161174
				16099 1020757.29729737
				16100 1020758.57847733
				16101 1020759.8596573
				16102 1020761.14083726
				16103 1020762.42201723
				16104 1020763.70319719
				16105 1020764.98437716
				16106 1020766.26555712
				16107 1020767.54673709
				16108 1020768.82791705
				16109 1020770.10909702
				16110 1020771.39027699
				16111 1020772.67145695
				16112 1020773.95263692
				16113 1020775.23381688
				16114 1020776.51499685
				16115 1020777.79617681
				16116 1020779.07735678
				16117 1020780.35853674
				16118 1020781.63971671
				16119 1020782.92089667
				16120 1020784.20207664
				16121 1020785.48325661
				16122 1020786.76443657
				16123 1020788.04561654
				16124 1020789.3267965
				16125 1020790.60797647
				16126 1020791.88915643
				16127 1020793.1703364
				16128 1020794.45151636
				16129 1020795.73269633
				16130 1020797.01387629
				16131 1020798.29505626
				16132 1020799.57623623
				16133 1020800.85741619
				16134 1020802.13859616
				16135 1020803.41977612
				16136 1020804.70095609
				16137 1020805.98213605
				16138 1020807.26331602
				16139 1020808.54449598
				16140 1020809.82567595
				16141 1020811.10685591
				16142 1020812.38803588
				16143 1020813.66921585
				16144 1020814.95039581
				16145 1020816.23157578
				16146 1020817.51275574
				16147 1020818.79393571
				16148 1020820.07511567
				16149 1020821.35629564
				16150 1020822.6374756
				16151 1020823.91865557
				16152 1020825.19983553
				16153 1020826.4810155
				16154 1020827.76219547
				16155 1020829.04337543
				16156 1020830.3245554
				16157 1020831.60573536
				16158 1020832.88691533
				16159 1020834.16809529
				16160 1020835.44927526
				16161 1020836.73045522
				16162 1020838.01163519
				16163 1020839.29281515
				16164 1020840.57399512
				16165 1020841.85517509
				16166 1020843.13635505
				16167 1020844.41753502
				16168 1020845.69871498
				16169 1020846.97989495
				16170 1020848.26107491
				16171 1020849.54225488
				16172 1020850.82343484
				16173 1020852.10461481
				16174 1020853.38579477
				16175 1020854.66697474
				16176 1020855.94815471
				16177 1020857.22933467
				16178 1020858.51051464
				16179 1020859.7916946
				16180 1020861.07287457
				16181 1020862.35405453
				16182 1020863.6352345
				16183 1020864.91641446
				16184 1020866.19759443
				16185 1020867.47877439
				16186 1020868.75995436
				16187 1020870.04113433
				16188 1020871.32231429
				16189 1020872.60349426
				16190 1020873.88467422
				16191 1020875.16585419
				16192 1020876.44703415
				16193 1020877.72821412
				16194 1020879.00939408
				16195 1020880.29057405
				16196 1020881.57175401
				16197 1020882.85293398
				16198 1020884.13411395
				16199 1020885.41529391
				16200 1020886.69647388
				16201 1020887.97765384
				16202 1020889.25883381
				16203 1020890.54001377
				16204 1020891.82119374
				16205 1020893.1023737
				16206 1020894.38355367
				16207 1020895.66473363
				16208 1020896.9459136
				16209 1020898.22709357
				16210 1020899.50827353
				16211 1020900.7894535
				16212 1020902.07063346
				16213 1020903.35181343
				16214 1020904.63299339
				16215 1020905.91417336
				16216 1020907.19535332
				16217 1020908.47653329
				16218 1020909.75771325
				16219 1020911.03889322
				16220 1020912.32007319
				16221 1020913.60125315
				16222 1020914.88243312
				16223 1020916.16361308
				16224 1020917.44479305
				16225 1020918.72597301
				16226 1020920.00715298
				16227 1020921.28833294
				16228 1020922.56951291
				16229 1020923.85069287
				16230 1020925.13187284
				16231 1020926.41305281
				16232 1020927.69423277
				16233 1020928.97541274
				16234 1020930.2565927
				16235 1020931.53777267
				16236 1020932.81895263
				16237 1020934.1001326
				16238 1020935.38131256
				16239 1020936.66249253
				16240 1020937.94367249
				16241 1020939.22485246
				16242 1020940.50603242
				16243 1020941.78721239
				16244 1020943.06839236
				16245 1020944.34957232
				16246 1020945.63075229
				16247 1020946.91193225
				16248 1020948.19311222
				16249 1020949.47429218
				16250 1020950.75547215
				16251 1020952.03665211
				16252 1020953.31783208
				16253 1020954.59901205
				16254 1020955.88019201
				16255 1020957.16137198
				16256 1020958.44255194
				16257 1020959.72373191
				16258 1020961.00491187
				16259 1020962.28609184
				16260 1020963.5672718
				16261 1020964.84845177
				16262 1020966.12963173
				16263 1020967.4108117
				16264 1020968.69199167
				16265 1020969.97317163
				16266 1020971.2543516
				16267 1020972.53553156
				16268 1020973.81671153
				16269 1020975.09789149
				16270 1020976.37907146
				16271 1020977.66025142
				16272 1020978.94143139
				16273 1020980.22261135
				16274 1020981.50379132
				16275 1020982.78497129
				16276 1020984.06615125
				16277 1020985.34733122
				16278 1020986.62851118
				16279 1020987.90969115
				16280 1020989.19087111
				16281 1020990.47205108
				16282 1020991.75323104
				16283 1020993.03441101
				16284 1020994.31559097
				16285 1020995.59677094
				16286 1020996.87795091
				16287 1020998.15913087
				16288 1020999.44031084
				16289 1021000.7214908
				16290 1021002.00267077
				16291 1021003.28385073
				16292 1021004.5650307
				16293 1021005.84621066
				16294 1021007.12739063
				16295 1021008.40857059
				16296 1021009.68975056
				16297 1021010.97093052
				16298 1021012.25211049
				16299 1021013.53329046
				16300 1021014.81447042
				16301 1021016.09565039
				16302 1021017.37683035
				16303 1021018.65801032
				16304 1021019.93919028
				16305 1021021.22037025
				16306 1021022.50155021
				16307 1021023.78273018
				16308 1021025.06391014
				16309 1021026.34509011
				16310 1021027.62627008
				16311 1021028.90745004
				16312 1021030.18863001
				16313 1021031.46980997
				16314 1021032.75098994
				16315 1021034.0321699
				16316 1021035.31334987
				16317 1021036.59452983
				16318 1021037.8757098
				16319 1021039.15688976
				16320 1021040.43806973
				16321 1021041.7192497
				16322 1021043.00042966
				16323 1021044.28160963
				16324 1021045.56278959
				16325 1021046.84396956
				16326 1021048.12514952
				16327 1021049.40632949
				16328 1021050.68750945
				16329 1021051.96868942
				16330 1021053.24986938
				16331 1021054.53104935
				16332 1021055.81222932
				16333 1021057.09340928
				16334 1021058.37458925
				16335 1021059.65576921
				16336 1021060.93694918
				16337 1021062.21812914
				16338 1021063.49930911
				16339 1021064.78048907
				16340 1021066.06166904
				16341 1021067.34284901
				16342 1021068.62402897
				16343 1021069.90520894
				16344 1021071.1863889
				16345 1021072.46756887
				16346 1021073.74874883
				16347 1021075.0299288
				16348 1021076.31110876
				16349 1021077.59228873
				16350 1021078.87346869
				16351 1021080.15464866
				16352 1021081.43582862
				16353 1021082.71700859
				16354 1021083.99818856
				16355 1021085.27936852
				16356 1021086.56054849
				16357 1021087.84172845
				16358 1021089.12290842
				16359 1021090.40408838
				16360 1021091.68526835
				16361 1021092.96644831
				16362 1021094.24762828
				16363 1021095.52880824
				16364 1021096.80998821
				16365 1021098.09116818
				16366 1021099.37234814
				16367 1021100.65352811
				16368 1021101.93470807
				16369 1021103.21588804
				16370 1021104.497068
				16371 1021105.77824797
				16372 1021107.05942793
				16373 1021108.3406079
				16374 1021109.62178786
				16375 1021110.90296783
				16376 1021112.1841478
				16377 1021113.46532776
				16378 1021114.74650773
				16379 1021116.02768769
				16380 1021117.30886766
				16381 1021118.59004762
				16382 1021119.87122759
				16383 1021121.15240755
				16384 1021122.43358752
				16385 1021123.71476748
				16386 1021124.99594745
				16387 1021126.27712742
				16388 1021127.55830738
				16389 1021128.83948735
				16390 1021130.12066731
				16391 1021131.40184728
				16392 1021132.68302724
				16393 1021133.96420721
				16394 1021135.24538717
				16395 1021136.52656714
				16396 1021137.8077471
				16397 1021139.08892707
				16398 1021140.37010704
				16399 1021141.651287
				16400 1021142.93246697
				16401 1021144.21364693
				16402 1021145.4948269
				16403 1021146.77600686
				16404 1021148.05718683
				16405 1021149.33836679
				16406 1021150.61954676
				16407 1021151.90072672
				16408 1021153.18190669
				16409 1021154.46308666
				16410 1021155.74426662
				16411 1021157.02544659
				16412 1021158.30662655
				16413 1021159.58780652
				16414 1021160.86898648
				16415 1021162.15016645
				16416 1021163.43134641
				16417 1021164.71252638
				16418 1021165.99370634
				16419 1021167.27488631
				16420 1021168.55606628
				16421 1021169.83724624
				16422 1021171.11842621
				16423 1021172.39960617
				16424 1021173.68078614
				16425 1021174.9619661
				16426 1021176.24314607
				16427 1021177.52432603
				16428 1021178.805506
				16429 1021180.08668596
				16430 1021181.36786593
				16431 1021182.6490459
				16432 1021183.93022586
				16433 1021185.21140583
				16434 1021186.49258579
				16435 1021187.77376576
				16436 1021189.05494572
				16437 1021190.33612569
				16438 1021191.61730565
				16439 1021192.89848562
				16440 1021194.17966558
				16441 1021195.46084555
				16442 1021196.74202552
				16443 1021198.02320548
				16444 1021199.30438545
				16445 1021200.58556541
				16446 1021201.86674538
				16447 1021203.14792534
				16448 1021204.42910531
				16449 1021205.71028527
				16450 1021206.99146524
				16451 1021208.2726452
				16452 1021209.55382517
				16453 1021210.83500514
				16454 1021212.1161851
				16455 1021213.39736507
				16456 1021214.67854503
				16457 1021215.959725
				16458 1021217.24090496
				16459 1021218.52208493
				16460 1021219.80326489
				16461 1021221.08444486
				16462 1021222.36562482
				16463 1021223.64680479
				16464 1021224.92798476
				16465 1021226.20916472
				16466 1021227.49034469
				16467 1021228.77152465
				16468 1021230.05270462
				16469 1021231.33388458
				16470 1021232.61506455
				16471 1021233.89624451
				16472 1021235.17742448
				16473 1021236.45860444
				16474 1021237.73978441
				16475 1021239.02096438
				16476 1021240.30214434
				16477 1021241.58332431
				16478 1021242.86450427
				16479 1021244.14568424
				16480 1021245.4268642
				16481 1021246.70804417
				16482 1021247.98922413
				16483 1021249.2704041
				16484 1021250.55158406
				16485 1021251.83276403
				16486 1021253.113944
				16487 1021254.39512396
				16488 1021255.67630393
				16489 1021256.95748389
				16490 1021258.23866386
				16491 1021259.51984382
				16492 1021260.80102379
				16493 1021262.08220375
				16494 1021263.36338372
				16495 1021264.64456368
				16496 1021265.92574365
				16497 1021267.20692362
				16498 1021268.48810358
				16499 1021269.76928355
				16500 1021271.05046351
				16501 1021272.33164348
				16502 1021273.61282344
				16503 1021274.89400341
				16504 1021276.17518337
				16505 1021277.45636334
				16506 1021278.7375433
				16507 1021280.01872327
				16508 1021281.29990324
				16509 1021282.5810832
				16510 1021283.86226317
				16511 1021285.14344313
				16512 1021286.4246231
				16513 1021287.70580306
				16514 1021288.98698303
				16515 1021290.26816299
				16516 1021291.54934296
				16517 1021292.83052292
				16518 1021294.11170289
				16519 1021295.39288286
				16520 1021296.67406282
				16521 1021297.95524279
				16522 1021299.23642275
				16523 1021300.51760272
				16524 1021301.79878268
				16525 1021303.07996265
				16526 1021304.36114261
				16527 1021305.64232258
				16528 1021306.92350254
				16529 1021308.20468251
				16530 1021309.48586248
				16531 1021310.76704244
				16532 1021312.04822241
				16533 1021313.32940237
				16534 1021314.61058234
				16535 1021315.8917623
				16536 1021317.17294227
				16537 1021318.45412223
				16538 1021319.7353022
				16539 1021321.01648216
				16540 1021322.29766213
				16541 1021323.5788421
				16542 1021324.86002206
				16543 1021326.14120203
				16544 1021327.42238199
				16545 1021328.70356196
				16546 1021329.98474192
				16547 1021331.26592189
				16548 1021332.54710185
				16549 1021333.82828182
				16550 1021335.10946178
				16551 1021336.39064175
				16552 1021337.67182172
				16553 1021338.95300168
				16554 1021340.23418165
				16555 1021341.51536161
				16556 1021342.79654158
				16557 1021344.07772154
				16558 1021345.35890151
				16559 1021346.64008147
				16560 1021347.92126144
				16561 1021349.2024414
				16562 1021350.48362137
				16563 1021351.76480134
				16564 1021353.0459813
				16565 1021354.32716127
				16566 1021355.60834123
				16567 1021356.8895212
				16568 1021358.17070116
				16569 1021359.45188113
				16570 1021360.73306109
				16571 1021362.01424106
				16572 1021363.29542102
				16573 1021364.57660099
				16574 1021365.85778096
				16575 1021367.13896092
				16576 1021368.42014089
				16577 1021369.70132085
				16578 1021370.98250082
				16579 1021372.26368078
				16580 1021373.54486075
				16581 1021374.82604071
				16582 1021376.10722068
				16583 1021377.38840064
				16584 1021378.66958061
				16585 1021379.95076058
				16586 1021381.23194054
				16587 1021382.51312051
				16588 1021383.79430047
				16589 1021385.07548044
				16590 1021386.3566604
				16591 1021387.63784037
				16592 1021388.91902033
				16593 1021390.2002003
				16594 1021391.48138026
				16595 1021392.76256023
				16596 1021394.0437402
				16597 1021395.32492016
				16598 1021396.60610013
				16599 1021397.88728009
				16600 1021399.16846006
				16601 1021400.44964002
				16602 1021401.73081999
				16603 1021403.01199995
				16604 1021404.29317992
				16605 1021405.57435988
				16606 1021406.85553985
				16607 1021408.13671982
				16608 1021409.41789978
				16609 1021410.69907975
				16610 1021411.98025971
				16611 1021413.26143968
				16612 1021414.54261964
				16613 1021415.82379961
				16614 1021417.10497957
				16615 1021418.38615954
				16616 1021419.6673395
				16617 1021420.94851947
				16618 1021422.22969944
				16619 1021423.5108794
				16620 1021424.79205937
				16621 1021426.07323933
				16622 1021427.3544193
				16623 1021428.63559926
				16624 1021429.91677923
				16625 1021431.19795919
				16626 1021432.47913916
				16627 1021433.76031912
				16628 1021435.04149909
				16629 1021436.32267906
				16630 1021437.60385902
				16631 1021438.88503899
				16632 1021440.16621895
				16633 1021441.44739892
				16634 1021442.72857888
				16635 1021444.00975885
				16636 1021445.29093881
				16637 1021446.57211878
				16638 1021447.85329874
				16639 1021449.13447871
				16640 1021450.41565868
				16641 1021451.69683864
				16642 1021452.97801861
				16643 1021454.25919857
				16644 1021455.54037854
				16645 1021456.8215585
				16646 1021458.10273847
				16647 1021459.38391843
				16648 1021460.6650984
				16649 1021461.94627836
				16650 1021463.22745833
				16651 1021464.5086383
				16652 1021465.78981826
				16653 1021467.07099823
				16654 1021468.35217819
				16655 1021469.63335816
				16656 1021470.91453812
				16657 1021472.19571809
				16658 1021473.47689805
				16659 1021474.75807802
				16660 1021476.03925798
				16661 1021477.32043795
				16662 1021478.60161792
				16663 1021479.88279788
				16664 1021481.16397785
				16665 1021482.44515781
				16666 1021483.72633778
				16667 1021485.00751774
				16668 1021486.28869771
				16669 1021487.56987767
				16670 1021488.85105764
				16671 1021490.1322376
				16672 1021491.41341757
				16673 1021492.69459754
				16674 1021493.9757775
				16675 1021495.25695747
				16676 1021496.53813743
				16677 1021497.8193174
				16678 1021499.10049736
				16679 1021500.38167733
				16680 1021501.66285729
				16681 1021502.94403726
				16682 1021504.22521722
				16683 1021505.50639719
				16684 1021506.78757716
				16685 1021508.06875712
				16686 1021509.34993709
				16687 1021510.63111705
				16688 1021511.91229702
				16689 1021513.19347698
				16690 1021514.47465695
				16691 1021515.75583691
				16692 1021517.03701688
				16693 1021518.31819684
				16694 1021519.59937681
				16695 1021520.88055678
				16696 1021522.16173674
				16697 1021523.44291671
				16698 1021524.72409667
				16699 1021526.00527664
				16700 1021527.2864566
				16701 1021528.56763657
				16702 1021529.84881653
				16703 1021531.1299965
				16704 1021532.41117646
				16705 1021533.69235643
				16706 1021534.9735364
				16707 1021536.25471636
				16708 1021537.53589633
				16709 1021538.81707629
				16710 1021540.09825626
				16711 1021541.37943622
				16712 1021542.66061619
				16713 1021543.94179615
				16714 1021545.22297612
				16715 1021546.50415608
				16716 1021547.78533605
				16717 1021549.06651602
				16718 1021550.34769598
				16719 1021551.62887595
				16720 1021552.91005591
				16721 1021554.19123588
				16722 1021555.47241584
				16723 1021556.75359581
				16724 1021558.03477577
				16725 1021559.31595574
				16726 1021560.5971357
				16727 1021561.87831567
				16728 1021563.15949564
				16729 1021564.4406756
				16730 1021565.72185557
				16731 1021567.00303553
				16732 1021568.2842155
				16733 1021569.56539546
				16734 1021570.84657543
				16735 1021572.12775539
				16736 1021573.40893536
				16737 1021574.69011532
				16738 1021575.97129529
				16739 1021577.25247526
				16740 1021578.53365522
				16741 1021579.81483519
				16742 1021581.09601515
				16743 1021582.37719512
				16744 1021583.65837508
				16745 1021584.93955505
				16746 1021586.22073501
				16747 1021587.50191498
				16748 1021588.78309494
				16749 1021590.06427491
				16750 1021591.34545488
				16751 1021592.62663484
				16752 1021593.90781481
				16753 1021595.18899477
				16754 1021596.47017474
				16755 1021597.7513547
				16756 1021599.03253467
				16757 1021600.31371463
				16758 1021601.5948946
				16759 1021602.87607456
				16760 1021604.15725453
				16761 1021605.4384345
				16762 1021606.71961446
				16763 1021608.00079443
				16764 1021609.28197439
				16765 1021610.56315436
				16766 1021611.84433432
				16767 1021613.12551429
				16768 1021614.40669425
				16769 1021615.68787422
				16770 1021616.96905418
				16771 1021618.25023415
				16772 1021619.53141412
				16773 1021620.81259408
				16774 1021622.09377405
				16775 1021623.37495401
				16776 1021624.65613398
				16777 1021625.93731394
				16778 1021627.21849391
				16779 1021628.49967387
				16780 1021629.78085384
				16781 1021631.0620338
				16782 1021632.34321377
				16783 1021633.62439374
				16784 1021634.9055737
				16785 1021636.18675367
				16786 1021637.46793363
				16787 1021638.7491136
				16788 1021640.03029356
				16789 1021641.31147353
				16790 1021642.59265349
				16791 1021643.87383346
				16792 1021645.15501342
				16793 1021646.43619339
				16794 1021647.71737336
				16795 1021648.99855332
				16796 1021650.27973329
				16797 1021651.56091325
				16798 1021652.84209322
				16799 1021654.12327318
				16800 1021655.40445315
				16801 1021656.68563311
				16802 1021657.96681308
				16803 1021659.24799304
				16804 1021660.52917301
				16805 1021661.81035298
				16806 1021663.09153294
				16807 1021664.37271291
				16808 1021665.65389287
				16809 1021666.93507284
				16810 1021668.2162528
				16811 1021669.49743277
				16812 1021670.77861273
				16813 1021672.0597927
				16814 1021673.34097266
				16815 1021674.62215263
				16816 1021675.9033326
				16817 1021677.18451256
				16818 1021678.46569253
				16819 1021679.74687249
				16820 1021681.02805246
				16821 1021682.30923242
				16822 1021683.59041239
				16823 1021684.87159235
				16824 1021686.15277232
				16825 1021687.43395228
				16826 1021688.71513225
				16827 1021689.99631222
				16828 1021691.27749218
				16829 1021692.55867215
				16830 1021693.83985211
				16831 1021695.12103208
				16832 1021696.40221204
				16833 1021697.68339201
				16834 1021698.96457197
				16835 1021700.24575194
				16836 1021701.5269319
				16837 1021702.80811187
				16838 1021704.08929184
				16839 1021705.3704718
				16840 1021706.65165177
				16841 1021707.93283173
				16842 1021709.2140117
				16843 1021710.49519166
				16844 1021711.77637163
				16845 1021713.05755159
				16846 1021714.33873156
				16847 1021715.61991152
				16848 1021716.90109149
				16849 1021718.18227146
				16850 1021719.46345142
				16851 1021720.74463139
				16852 1021722.02581135
				16853 1021723.30699132
				16854 1021724.58817128
				16855 1021725.86935125
				16856 1021727.15053121
				16857 1021728.43171118
				16858 1021729.71289114
				16859 1021730.99407111
				16860 1021732.27525108
				16861 1021733.55643104
				16862 1021734.83761101
				16863 1021736.11879097
				16864 1021737.39997094
				16865 1021738.6811509
				16866 1021739.96233087
				16867 1021741.24351083
				16868 1021742.5246908
				16869 1021743.80587076
				16870 1021745.08705073
				16871 1021746.3682307
				16872 1021747.64941066
				16873 1021748.93059063
				16874 1021750.21177059
				16875 1021751.49295056
				16876 1021752.77413052
				16877 1021754.05531049
				16878 1021755.33649045
				16879 1021756.61767042
				16880 1021757.89885038
				16881 1021759.18003035
				16882 1021760.46121032
				16883 1021761.74239028
				16884 1021763.02357025
				16885 1021764.30475021
				16886 1021765.58593018
				16887 1021766.86711014
				16888 1021768.14829011
				16889 1021769.42947007
				16890 1021770.71065004
				16891 1021771.99183
				16892 1021773.27300997
				16893 1021774.55418994
				16894 1021775.8353699
				16895 1021777.11654987
				16896 1021778.39772983
				16897 1021779.6789098
				16898 1021780.96008976
				16899 1021782.24126973
				16900 1021783.52244969
				16901 1021784.80362966
				16902 1021786.08480962
				16903 1021787.36598959
				16904 1021788.64716956
				16905 1021789.92834952
				16906 1021791.20952949
				16907 1021792.49070945
				16908 1021793.77188942
				16909 1021795.05306938
				16910 1021796.33424935
				16911 1021797.61542931
				16912 1021798.89660928
				16913 1021800.17778924
				16914 1021801.45896921
				16915 1021802.74014918
				16916 1021804.02132914
				16917 1021805.30250911
				16918 1021806.58368907
				16919 1021807.86486904
				16920 1021809.146049
				16921 1021810.42722897
				16922 1021811.70840893
				16923 1021812.9895889
				16924 1021814.27076886
				16925 1021815.55194883
				16926 1021816.8331288
				16927 1021818.11430876
				16928 1021819.39548873
				16929 1021820.67666869
				16930 1021821.95784866
				16931 1021823.23902862
				16932 1021824.52020859
				16933 1021825.80138855
				16934 1021827.08256852
				16935 1021828.36374848
				16936 1021829.64492845
				16937 1021830.92610842
				16938 1021832.20728838
				16939 1021833.48846835
				16940 1021834.76964831
				16941 1021836.05082828
				16942 1021837.33200824
				16943 1021838.61318821
				16944 1021839.89436817
				16945 1021841.17554814
				16946 1021842.4567281
				16947 1021843.73790807
				16948 1021845.01908804
				16949 1021846.300268
				16950 1021847.58144797
				16951 1021848.86262793
				16952 1021850.1438079
				16953 1021851.42498786
				16954 1021852.70616783
				16955 1021853.98734779
				16956 1021855.26852776
				16957 1021856.54970772
				16958 1021857.83088769
				16959 1021859.11206766
				16960 1021860.39324762
				16961 1021861.67442759
				16962 1021862.95560755
				16963 1021864.23678752
				16964 1021865.51796748
				16965 1021866.79914745
				16966 1021868.08032741
				16967 1021869.36150738
				16968 1021870.64268734
				16969 1021871.92386731
				16970 1021873.20504728
				16971 1021874.48622724
				16972 1021875.76740721
				16973 1021877.04858717
				16974 1021878.32976714
				16975 1021879.6109471
				16976 1021880.89212707
				16977 1021882.17330703
				16978 1021883.454487
				16979 1021884.73566696
				16980 1021886.01684693
				16981 1021887.2980269
				16982 1021888.57920686
				16983 1021889.86038683
				16984 1021891.14156679
				16985 1021892.42274676
				16986 1021893.70392672
				16987 1021894.98510669
				16988 1021896.26628665
				16989 1021897.54746662
				16990 1021898.82864658
				16991 1021900.10982655
				16992 1021901.39100652
				16993 1021902.67218648
				16994 1021903.95336645
				16995 1021905.23454641
				16996 1021906.51572638
				16997 1021907.79690634
				16998 1021909.07808631
				16999 1021910.35926627
				17000 1021911.64044624
				17001 1021912.9216262
				17002 1021914.20280617
				17003 1021915.48398614
				17004 1021916.7651661
				17005 1021918.04634607
				17006 1021919.32752603
				17007 1021920.608706
				17008 1021921.88988596
				17009 1021923.17106593
				17010 1021924.45224589
				17011 1021925.73342586
				17012 1021927.01460582
				17013 1021928.29578579
				17014 1021929.57696576
				17015 1021930.85814572
				17016 1021932.13932569
				17017 1021933.42050565
				17018 1021934.70168562
				17019 1021935.98286558
				17020 1021937.26404555
				17021 1021938.54522551
				17022 1021939.82640548
				17023 1021941.10758544
				17024 1021942.38876541
				17025 1021943.66994538
				17026 1021944.95112534
				17027 1021946.23230531
				17028 1021947.51348527
				17029 1021948.79466524
				17030 1021950.0758452
				17031 1021951.35702517
				17032 1021952.63820513
				17033 1021953.9193851
				17034 1021955.20056506
				17035 1021956.48174503
				17036 1021957.762925
				17037 1021959.04410496
				17038 1021960.32528493
				17039 1021961.60646489
				17040 1021962.88764486
				17041 1021964.16882482
				17042 1021965.45000479
				17043 1021966.73118475
				17044 1021968.01236472
				17045 1021969.29354468
				17046 1021970.57472465
				17047 1021971.85590462
				17048 1021973.13708458
				17049 1021974.41826455
				17050 1021975.69944451
				17051 1021976.98062448
				17052 1021978.26180444
				17053 1021979.54298441
				17054 1021980.82416437
				17055 1021982.10534434
				17056 1021983.3865243
				17057 1021984.66770427
				17058 1021985.94888424
				17059 1021987.2300642
				17060 1021988.51124417
				17061 1021989.79242413
				17062 1021991.0736041
				17063 1021992.35478406
				17064 1021993.63596403
				17065 1021994.91714399
				17066 1021996.19832396
				17067 1021997.47950392
				17068 1021998.76068389
				17069 1022000.04186386
				17070 1022001.32304382
				17071 1022002.60422379
				17072 1022003.88540375
				17073 1022005.16658372
				17074 1022006.44776368
				17075 1022007.72894365
				17076 1022009.01012361
				17077 1022010.29130358
				17078 1022011.57248354
				17079 1022012.85366351
				17080 1022014.13484348
				17081 1022015.41602344
				17082 1022016.69720341
				17083 1022017.97838337
				17084 1022019.25956334
				17085 1022020.5407433
				17086 1022021.82192327
				17087 1022023.10310323
				17088 1022024.3842832
				17089 1022025.66546316
				17090 1022026.94664313
				17091 1022028.2278231
				17092 1022029.50900306
				17093 1022030.79018303
				17094 1022032.07136299
				17095 1022033.35254296
				17096 1022034.63372292
				17097 1022035.91490289
				17098 1022037.19608285
				17099 1022038.47726282
				17100 1022039.75844278
				17101 1022041.03962275
				17102 1022042.32080272
				17103 1022043.60198268
				17104 1022044.88316265
				17105 1022046.16434261
				17106 1022047.44552258
				17107 1022048.72670254
				17108 1022050.00788251
				17109 1022051.28906247
				17110 1022052.57024244
				17111 1022053.8514224
				17112 1022055.13260237
				17113 1022056.41378234
				17114 1022057.6949623
				17115 1022058.97614227
				17116 1022060.25732223
				17117 1022061.5385022
				17118 1022062.81968216
				17119 1022064.10086213
				17120 1022065.38204209
				17121 1022066.66322206
				17122 1022067.94440202
				17123 1022069.22558199
				17124 1022070.50676196
				17125 1022071.78794192
				17126 1022073.06912189
				17127 1022074.35030185
				17128 1022075.63148182
				17129 1022076.91266178
				17130 1022078.19384175
				17131 1022079.47502171
				17132 1022080.75620168
				17133 1022082.03738164
				17134 1022083.31856161
				17135 1022084.59974158
				17136 1022085.88092154
				17137 1022087.16210151
				17138 1022088.44328147
				17139 1022089.72446144
				17140 1022091.0056414
				17141 1022092.28682137
				17142 1022093.56800133
				17143 1022094.8491813
				17144 1022096.13036126
				17145 1022097.41154123
				17146 1022098.6927212
				17147 1022099.97390116
				17148 1022101.25508113
				17149 1022102.53626109
				17150 1022103.81744106
				17151 1022105.09862102
				17152 1022106.37980099
				17153 1022107.66098095
				17154 1022108.94216092
				17155 1022110.22334088
				17156 1022111.50452085
				17157 1022112.78570082
				17158 1022114.06688078
				17159 1022115.34806075
				17160 1022116.62924071
				17161 1022117.91042068
				17162 1022119.19160064
				17163 1022120.47278061
				17164 1022121.75396057
				17165 1022123.03514054
				17166 1022124.3163205
				17167 1022125.59750047
				17168 1022126.87868044
				17169 1022128.1598604
				17170 1022129.44104037
				17171 1022130.72222033
				17172 1022132.0034003
				17173 1022133.28458026
				17174 1022134.56576023
				17175 1022135.84694019
				17176 1022137.12812016
				17177 1022138.40930012
				17178 1022139.69048009
				17179 1022140.97166006
				17180 1022142.25284002
				17181 1022143.53401999
				17182 1022144.81519995
				17183 1022146.09637992
				17184 1022147.37755988
				17185 1022148.65873985
				17186 1022149.93991981
				17187 1022151.22109978
				17188 1022152.50227974
				17189 1022153.78345971
				17190 1022155.06463968
				17191 1022156.34581964
				17192 1022157.62699961
				17193 1022158.90817957
				17194 1022160.18935954
				17195 1022161.4705395
				17196 1022162.75171947
				17197 1022164.03289943
				17198 1022165.3140794
				17199 1022166.59525936
				17200 1022167.87643933
				17201 1022169.1576193
				17202 1022170.43879926
				17203 1022171.71997923
				17204 1022173.00115919
				17205 1022174.28233916
				17206 1022175.56351912
				17207 1022176.84469909
				17208 1022178.12587905
				17209 1022179.40705902
				17210 1022180.68823898
				17211 1022181.96941895
				17212 1022183.25059892
				17213 1022184.53177888
				17214 1022185.81295885
				17215 1022187.09413881
				17216 1022188.37531878
				17217 1022189.65649874
				17218 1022190.93767871
				17219 1022192.21885867
				17220 1022193.50003864
				17221 1022194.7812186
				17222 1022196.06239857
				17223 1022197.34357854
				17224 1022198.6247585
				17225 1022199.90593847
				17226 1022201.18711843
				17227 1022202.4682984
				17228 1022203.74947836
				17229 1022205.03065833
				17230 1022206.31183829
				17231 1022207.59301826
				17232 1022208.87419822
				17233 1022210.15537819
				17234 1022211.43655816
				17235 1022212.71773812
				17236 1022213.99891809
				17237 1022215.28009805
				17238 1022216.56127802
				17239 1022217.84245798
				17240 1022219.12363795
				17241 1022220.40481791
				17242 1022221.68599788
				17243 1022222.96717784
				17244 1022224.24835781
				17245 1022225.52953778
				17246 1022226.81071774
				17247 1022228.09189771
				17248 1022229.37307767
				17249 1022230.65425764
				17250 1022231.9354376
				17251 1022233.21661757
				17252 1022234.49779753
				17253 1022235.7789775
				17254 1022237.06015746
				17255 1022238.34133743
				17256 1022239.6225174
				17257 1022240.90369736
				17258 1022242.18487733
				17259 1022243.46605729
				17260 1022244.74723726
				17261 1022246.02841722
				17262 1022247.30959719
				17263 1022248.59077715
				17264 1022249.87195712
				17265 1022251.15313708
				17266 1022252.43431705
				17267 1022253.71549702
				17268 1022254.99667698
				17269 1022256.27785695
				17270 1022257.55903691
				17271 1022258.84021688
				17272 1022260.12139684
				17273 1022261.40257681
				17274 1022262.68375677
				17275 1022263.96493674
				17276 1022265.2461167
				17277 1022266.52729667
				17278 1022267.80847664
				17279 1022269.0896566
				17280 1022270.37083657
				17281 1022271.65201653
				17282 1022272.9331965
				17283 1022274.21437646
				17284 1022275.49555643
				17285 1022276.77673639
				17286 1022278.05791636
				17287 1022279.33909632
				17288 1022280.62027629
				17289 1022281.90145626
				17290 1022283.18263622
				17291 1022284.46381619
				17292 1022285.74499615
				17293 1022287.02617612
				17294 1022288.30735608
				17295 1022289.58853605
				17296 1022290.86971601
				17297 1022292.15089598
				17298 1022293.43207594
				17299 1022294.71325591
				17300 1022295.99443588
				17301 1022297.27561584
				17302 1022298.55679581
				17303 1022299.83797577
				17304 1022301.11915574
				17305 1022302.4003357
				17306 1022303.68151567
				17307 1022304.96269563
				17308 1022306.2438756
				17309 1022307.52505556
				17310 1022308.80623553
				17311 1022310.0874155
				17312 1022311.36859546
				17313 1022312.64977543
				17314 1022313.93095539
				17315 1022315.21213536
				17316 1022316.49331532
				17317 1022317.77449529
				17318 1022319.05567525
				17319 1022320.33685522
				17320 1022321.61803518
				17321 1022322.89921515
				17322 1022324.18039512
				17323 1022325.46157508
				17324 1022326.74275505
				17325 1022328.02393501
				17326 1022329.30511498
				17327 1022330.58629494
				17328 1022331.86747491
				17329 1022333.14865487
				17330 1022334.42983484
				17331 1022335.7110148
				17332 1022336.99219477
				17333 1022338.27337474
				17334 1022339.5545547
				17335 1022340.83573467
				17336 1022342.11691463
				17337 1022343.3980946
				17338 1022344.67927456
				17339 1022345.96045453
				17340 1022347.24163449
				17341 1022348.52281446
				17342 1022349.80399442
				17343 1022351.08517439
				17344 1022352.36635436
				17345 1022353.64753432
				17346 1022354.92871429
				17347 1022356.20989425
				17348 1022357.49107422
				17349 1022358.77225418
				17350 1022360.05343415
				17351 1022361.33461411
				17352 1022362.61579408
				17353 1022363.89697404
				17354 1022365.17815401
				17355 1022366.45933398
				17356 1022367.74051394
				17357 1022369.02169391
				17358 1022370.30287387
				17359 1022371.58405384
				17360 1022372.8652338
				17361 1022374.14641377
				17362 1022375.42759373
				17363 1022376.7087737
				17364 1022377.98995366
				17365 1022379.27113363
				17366 1022380.5523136
				17367 1022381.83349356
				17368 1022383.11467353
				17369 1022384.39585349
				17370 1022385.67703346
				17371 1022386.95821342
				17372 1022388.23939339
				17373 1022389.52057335
				17374 1022390.80175332
				17375 1022392.08293328
				17376 1022393.36411325
				17377 1022394.64529322
				17378 1022395.92647318
				17379 1022397.20765315
				17380 1022398.48883311
				17381 1022399.77001308
				17382 1022401.05119304
				17383 1022402.33237301
				17384 1022403.61355297
				17385 1022404.89473294
				17386 1022406.1759129
				17387 1022407.45709287
				17388 1022408.73827284
				17389 1022410.0194528
				17390 1022411.30063277
				17391 1022412.58181273
				17392 1022413.8629927
				17393 1022415.14417266
				17394 1022416.42535263
				17395 1022417.70653259
				17396 1022418.98771256
				17397 1022420.26889252
				17398 1022421.55007249
				17399 1022422.83125246
				17400 1022424.11243242
				17401 1022425.39361239
				17402 1022426.67479235
				17403 1022427.95597232
				17404 1022429.23715228
				17405 1022430.51833225
				17406 1022431.79951221
				17407 1022433.08069218
				17408 1022434.36187214
				17409 1022435.64305211
				17410 1022436.92423208
				17411 1022438.20541204
				17412 1022439.48659201
				17413 1022440.76777197
				17414 1022442.04895194
				17415 1022443.3301319
				17416 1022444.61131187
				17417 1022445.89249183
				17418 1022447.1736718
				17419 1022448.45485176
				17420 1022449.73603173
				17421 1022451.0172117
				17422 1022452.29839166
				17423 1022453.57957163
				17424 1022454.86075159
				17425 1022456.14193156
				17426 1022457.42311152
				17427 1022458.70429149
				17428 1022459.98547145
				17429 1022461.26665142
				17430 1022462.54783138
				17431 1022463.82901135
				17432 1022465.11019132
				17433 1022466.39137128
				17434 1022467.67255125
				17435 1022468.95373121
				17436 1022470.23491118
				17437 1022471.51609114
				17438 1022472.79727111
				17439 1022474.07845107
				17440 1022475.35963104
				17441 1022476.640811
				17442 1022477.92199097
				17443 1022479.20317094
				17444 1022480.4843509
				17445 1022481.76553087
				17446 1022483.04671083
				17447 1022484.3278908
				17448 1022485.60907076
				17449 1022486.89025073
				17450 1022488.17143069
				17451 1022489.45261066
				17452 1022490.73379062
				17453 1022492.01497059
				17454 1022493.29615056
				17455 1022494.57733052
				17456 1022495.85851049
				17457 1022497.13969045
				17458 1022498.42087042
				17459 1022499.70205038
				17460 1022500.98323035
				17461 1022502.26441031
				17462 1022503.54559028
				17463 1022504.82677024
				17464 1022506.10795021
				17465 1022507.38913018
				17466 1022508.67031014
				17467 1022509.95149011
				17468 1022511.23267007
				17469 1022512.51385004
				17470 1022513.79503
				17471 1022515.07620997
				17472 1022516.35738993
				17473 1022517.6385699
				17474 1022518.91974986
				17475 1022520.20092983
				17476 1022521.4821098
				17477 1022522.76328976
				17478 1022524.04446973
				17479 1022525.32564969
				17480 1022526.60682966
				17481 1022527.88800962
				17482 1022529.16918959
				17483 1022530.45036955
				17484 1022531.73154952
				17485 1022533.01272948
				17486 1022534.29390945
				17487 1022535.57508942
				17488 1022536.85626938
				17489 1022538.13744935
				17490 1022539.41862931
				17491 1022540.69980928
				17492 1022541.98098924
				17493 1022543.26216921
				17494 1022544.54334917
				17495 1022545.82452914
				17496 1022547.1057091
				17497 1022548.38688907
				17498 1022549.66806904
				17499 1022550.949249
				17500 1022552.23042897
				17501 1022553.51160893
				17502 1022554.7927889
				17503 1022556.07396886
				17504 1022557.35514883
				17505 1022558.63632879
				17506 1022559.91750876
				17507 1022561.19868872
				17508 1022562.47986869
				17509 1022563.76104866
				17510 1022565.04222862
				17511 1022566.32340859
				17512 1022567.60458855
				17513 1022568.88576852
				17514 1022570.16694848
				17515 1022571.44812845
				17516 1022572.72930841
				17517 1022574.01048838
				17518 1022575.29166834
				17519 1022576.57284831
				17520 1022577.85402828
				17521 1022579.13520824
				17522 1022580.41638821
				17523 1022581.69756817
				17524 1022582.97874814
				17525 1022584.2599281
				17526 1022585.54110807
				17527 1022586.82228803
				17528 1022588.103468
				17529 1022589.38464796
				17530 1022590.66582793
				17531 1022591.94700789
				17532 1022593.22818786
				17533 1022594.50936783
				17534 1022595.79054779
				17535 1022597.07172776
				17536 1022598.35290772
				17537 1022599.63408769
				17538 1022600.91526765
				17539 1022602.19644762
				17540 1022603.47762758
				17541 1022604.75880755
				17542 1022606.03998751
				17543 1022607.32116748
				17544 1022608.60234745
				17545 1022609.88352741
				17546 1022611.16470738
				17547 1022612.44588734
				17548 1022613.72706731
				17549 1022615.00824727
				17550 1022616.28942724
				17551 1022617.5706072
				17552 1022618.85178717
				17553 1022620.13296713
				17554 1022621.4141471
				17555 1022622.69532707
				17556 1022623.97650703
				17557 1022625.257687
				17558 1022626.53886696
				17559 1022627.82004693
				17560 1022629.10122689
				17561 1022630.38240686
				17562 1022631.66358682
				17563 1022632.94476679
				17564 1022634.22594676
				17565 1022635.50712672
				17566 1022636.78830669
				17567 1022638.06948665
				17568 1022639.35066662
				17569 1022640.63184658
				17570 1022641.91302655
				17571 1022643.19420651
				17572 1022644.47538648
				17573 1022645.75656644
				17574 1022647.03774641
				17575 1022648.31892638
				17576 1022649.60010634
				17577 1022650.88128631
				17578 1022652.16246627
				17579 1022653.44364624
				17580 1022654.7248262
				17581 1022656.00600617
				17582 1022657.28718613
				17583 1022658.5683661
				17584 1022659.84954606
				17585 1022661.13072603
				17586 1022662.41190599
				17587 1022663.69308596
				17588 1022664.97426593
				17589 1022666.25544589
				17590 1022667.53662586
				17591 1022668.81780582
				17592 1022670.09898579
				17593 1022671.38016575
				17594 1022672.66134572
				17595 1022673.94252568
				17596 1022675.22370565
				17597 1022676.50488561
				17598 1022677.78606558
				17599 1022679.06724555
				17600 1022680.34842551
				17601 1022681.62960548
				17602 1022682.91078544
				17603 1022684.19196541
				17604 1022685.47314537
				17605 1022686.75432534
				17606 1022688.0355053
				17607 1022689.31668527
				17608 1022690.59786523
				17609 1022691.8790452
				17610 1022693.16022517
				17611 1022694.44140513
				17612 1022695.7225851
				17613 1022697.00376506
				17614 1022698.28494503
				17615 1022699.56612499
				17616 1022700.84730496
				17617 1022702.12848492
				17618 1022703.40966489
				17619 1022704.69084485
				17620 1022705.97202482
				17621 1022707.25320479
				17622 1022708.53438475
				17623 1022709.81556472
				17624 1022711.09674468
				17625 1022712.37792465
				17626 1022713.65910461
				17627 1022714.94028458
				17628 1022716.22146454
				17629 1022717.50264451
				17630 1022718.78382447
				17631 1022720.06500444
				17632 1022721.34618441
				17633 1022722.62736437
				17634 1022723.90854434
				17635 1022725.1897243
				17636 1022726.47090427
				17637 1022727.75208423
				17638 1022729.0332642
				17639 1022730.31444416
				17640 1022731.59562413
				17641 1022732.87680409
				17642 1022734.15798406
				17643 1022735.43916403
				17644 1022736.72034399
				17645 1022738.00152396
				17646 1022739.28270392
				17647 1022740.56388389
				17648 1022741.84506385
				17649 1022743.12624382
				17650 1022744.40742378
				17651 1022745.68860375
				17652 1022746.96978371
				17653 1022748.25096368
				17654 1022749.53214365
				17655 1022750.81332361
				17656 1022752.09450358
				17657 1022753.37568354
				17658 1022754.65686351
				17659 1022755.93804347
				17660 1022757.21922344
				17661 1022758.5004034
				17662 1022759.78158337
				17663 1022761.06276333
				17664 1022762.3439433
				17665 1022763.62512327
				17666 1022764.90630323
				17667 1022766.1874832
				17668 1022767.46866316
				17669 1022768.74984313
				17670 1022770.03102309
				17671 1022771.31220306
				17672 1022772.59338302
				17673 1022773.87456299
				17674 1022775.15574295
				17675 1022776.43692292
				17676 1022777.71810289
				17677 1022778.99928285
				17678 1022780.28046282
				17679 1022781.56164278
				17680 1022782.84282275
				17681 1022784.12400271
				17682 1022785.40518268
				17683 1022786.68636264
				17684 1022787.96754261
				17685 1022789.24872257
				17686 1022790.52990254
				17687 1022791.81108251
				17688 1022793.09226247
				17689 1022794.37344244
				17690 1022795.6546224
				17691 1022796.93580237
				17692 1022798.21698233
				17693 1022799.4981623
				17694 1022800.77934226
				17695 1022802.06052223
				17696 1022803.34170219
				17697 1022804.62288216
				17698 1022805.90406213
				17699 1022807.18524209
				17700 1022808.46642206
				17701 1022809.74760202
				17702 1022811.02878199
				17703 1022812.30996195
				17704 1022813.59114192
				17705 1022814.87232188
				17706 1022816.15350185
				17707 1022817.43468181
				17708 1022818.71586178
				17709 1022819.99704175
				17710 1022821.27822171
				17711 1022822.55940168
				17712 1022823.84058164
				17713 1022825.12176161
				17714 1022826.40294157
				17715 1022827.68412154
				17716 1022828.9653015
				17717 1022830.24648147
				17718 1022831.52766143
				17719 1022832.8088414
				17720 1022834.09002137
				17721 1022835.37120133
				17722 1022836.6523813
				17723 1022837.93356126
				17724 1022839.21474123
				17725 1022840.49592119
				17726 1022841.77710116
				17727 1022843.05828112
				17728 1022844.33946109
				17729 1022845.62064105
				17730 1022846.90182102
				17731 1022848.18300099
				17732 1022849.46418095
				17733 1022850.74536092
				17734 1022852.02654088
				17735 1022853.30772085
				17736 1022854.58890081
				17737 1022855.87008078
				17738 1022857.15126074
				17739 1022858.43244071
				17740 1022859.71362067
				17741 1022860.99480064
				17742 1022862.27598061
				17743 1022863.55716057
				17744 1022864.83834054
				17745 1022866.1195205
				17746 1022867.40070047
				17747 1022868.68188043
				17748 1022869.9630604
				17749 1022871.24424036
				17750 1022872.52542033
				17751 1022873.80660029
				17752 1022875.08778026
				17753 1022876.36896023
				17754 1022877.65014019
				17755 1022878.93132016
				17756 1022880.21250012
				17757 1022881.49368009
				17758 1022882.77486005
				17759 1022884.05604002
				17760 1022885.33721998
				17761 1022886.61839995
				17762 1022887.89957991
				17763 1022889.18075988
				17764 1022890.46193985
				17765 1022891.74311981
				17766 1022893.02429978
				17767 1022894.30547974
				17768 1022895.58665971
				17769 1022896.86783967
				17770 1022898.14901964
				17771 1022899.4301996
				17772 1022900.71137957
				17773 1022901.99255953
				17774 1022903.2737395
				17775 1022904.55491947
				17776 1022905.83609943
				17777 1022907.1172794
				17778 1022908.39845936
				17779 1022909.67963933
				17780 1022910.96081929
				17781 1022912.24199926
				17782 1022913.52317922
				17783 1022914.80435919
				17784 1022916.08553915
				17785 1022917.36671912
				17786 1022918.64789909
				17787 1022919.92907905
				17788 1022921.21025902
				17789 1022922.49143898
				17790 1022923.77261895
				17791 1022925.05379891
				17792 1022926.33497888
				17793 1022927.61615884
				17794 1022928.89733881
				17795 1022930.17851877
				17796 1022931.45969874
				17797 1022932.74087871
				17798 1022934.02205867
				17799 1022935.30323864
				17800 1022936.5844186
				17801 1022937.86559857
				17802 1022939.14677853
				17803 1022940.4279585
				17804 1022941.70913846
				17805 1022942.99031843
				17806 1022944.27149839
				17807 1022945.55267836
				17808 1022946.83385833
				17809 1022948.11503829
				17810 1022949.39621826
				17811 1022950.67739822
				17812 1022951.95857819
				17813 1022953.23975815
				17814 1022954.52093812
				17815 1022955.80211808
				17816 1022957.08329805
				17817 1022958.36447801
				17818 1022959.64565798
				17819 1022960.92683795
				17820 1022962.20801791
				17821 1022963.48919788
				17822 1022964.77037784
				17823 1022966.05155781
				17824 1022967.33273777
				17825 1022968.61391774
				17826 1022969.8950977
				17827 1022971.17627767
				17828 1022972.45745763
				17829 1022973.7386376
				17830 1022975.01981757
				17831 1022976.30099753
				17832 1022977.5821775
				17833 1022978.86335746
				17834 1022980.14453743
				17835 1022981.42571739
				17836 1022982.70689736
				17837 1022983.98807732
				17838 1022985.26925729
				17839 1022986.55043725
				17840 1022987.83161722
				17841 1022989.11279719
				17842 1022990.39397715
				17843 1022991.67515712
				17844 1022992.95633708
				17845 1022994.23751705
				17846 1022995.51869701
				17847 1022996.79987698
				17848 1022998.08105694
				17849 1022999.36223691
				17850 1023000.64341687
				17851 1023001.92459684
				17852 1023003.20577681
				17853 1023004.48695677
				17854 1023005.76813674
				17855 1023007.0493167
				17856 1023008.33049667
				17857 1023009.61167663
				17858 1023010.8928566
				17859 1023012.17403656
				17860 1023013.45521653
				17861 1023014.73639649
				17862 1023016.01757646
				17863 1023017.29875643
				17864 1023018.57993639
				17865 1023019.86111636
				17866 1023021.14229632
				17867 1023022.42347629
				17868 1023023.70465625
				17869 1023024.98583622
				17870 1023026.26701618
				17871 1023027.54819615
				17872 1023028.82937611
				17873 1023030.11055608
				17874 1023031.39173605
				17875 1023032.67291601
				17876 1023033.95409598
				17877 1023035.23527594
				17878 1023036.51645591
				17879 1023037.79763587
				17880 1023039.07881584
				17881 1023040.3599958
				17882 1023041.64117577
				17883 1023042.92235573
				17884 1023044.2035357
				17885 1023045.48471567
				17886 1023046.76589563
				17887 1023048.0470756
				17888 1023049.32825556
				17889 1023050.60943553
				17890 1023051.89061549
				17891 1023053.17179546
				17892 1023054.45297542
				17893 1023055.73415539
				17894 1023057.01533535
				17895 1023058.29651532
				17896 1023059.57769529
				17897 1023060.85887525
				17898 1023062.14005522
				17899 1023063.42123518
				17900 1023064.70241515
				17901 1023065.98359511
				17902 1023067.26477508
				17903 1023068.54595504
				17904 1023069.82713501
				17905 1023071.10831497
				17906 1023072.38949494
				17907 1023073.67067491
				17908 1023074.95185487
				17909 1023076.23303484
				17910 1023077.5142148
				17911 1023078.79539477
				17912 1023080.07657473
				17913 1023081.3577547
				17914 1023082.63893466
				17915 1023083.92011463
				17916 1023085.20129459
				17917 1023086.48247456
				17918 1023087.76365453
				17919 1023089.04483449
				17920 1023090.32601446
				17921 1023091.60719442
				17922 1023092.88837439
				17923 1023094.16955435
				17924 1023095.45073432
				17925 1023096.73191428
				17926 1023098.01309425
				17927 1023099.29427421
				17928 1023100.57545418
				17929 1023101.85663415
				17930 1023103.13781411
				17931 1023104.41899408
				17932 1023105.70017404
				17933 1023106.98135401
				17934 1023108.26253397
				17935 1023109.54371394
				17936 1023110.8248939
				17937 1023112.10607387
				17938 1023113.38725383
				17939 1023114.6684338
				17940 1023115.94961377
				17941 1023117.23079373
				17942 1023118.5119737
				17943 1023119.79315366
				17944 1023121.07433363
				17945 1023122.35551359
				17946 1023123.63669356
				17947 1023124.91787352
				17948 1023126.19905349
				17949 1023127.48023345
				17950 1023128.76141342
				17951 1023130.04259339
				17952 1023131.32377335
				17953 1023132.60495332
				17954 1023133.88613328
				17955 1023135.16731325
				17956 1023136.44849321
				17957 1023137.72967318
				17958 1023139.01085314
				17959 1023140.29203311
				17960 1023141.57321307
				17961 1023142.85439304
				17962 1023144.13557301
				17963 1023145.41675297
				17964 1023146.69793294
				17965 1023147.9791129
				17966 1023149.26029287
				17967 1023150.54147283
				17968 1023151.8226528
				17969 1023153.10383276
				17970 1023154.38501273
				17971 1023155.66619269
				17972 1023156.94737266
				17973 1023158.22855263
				17974 1023159.50973259
				17975 1023160.79091256
				17976 1023162.07209252
				17977 1023163.35327249
				17978 1023164.63445245
				17979 1023165.91563242
				17980 1023167.19681238
				17981 1023168.47799235
				17982 1023169.75917231
				17983 1023171.04035228
				17984 1023172.32153225
				17985 1023173.60271221
				17986 1023174.88389218
				17987 1023176.16507214
				17988 1023177.44625211
				17989 1023178.72743207
				17990 1023180.00861204
				17991 1023181.289792
				17992 1023182.57097197
				17993 1023183.85215193
				17994 1023185.1333319
				17995 1023186.41451187
				17996 1023187.69569183
				17997 1023188.9768718
				17998 1023190.25805176
				17999 1023191.53923173
				18000 1023192.82041169
				18001 1023194.10159166
				18002 1023195.38277162
				18003 1023196.66395159
				18004 1023197.94513155
				18005 1023199.22631152
				18006 1023200.50749149
				18007 1023201.78867145
				18008 1023203.06985142
				18009 1023204.35103138
				18010 1023205.63221135
				18011 1023206.91339131
				18012 1023208.19457128
				18013 1023209.47575124
				18014 1023210.75693121
				18015 1023212.03811117
				18016 1023213.31929114
				18017 1023214.60047111
				18018 1023215.88165107
				18019 1023217.16283104
				18020 1023218.444011
				18021 1023219.72519097
				18022 1023221.00637093
				18023 1023222.2875509
				18024 1023223.56873086
				18025 1023224.84991083
				18026 1023226.13109079
				18027 1023227.41227076
				18028 1023228.69345073
				18029 1023229.97463069
				18030 1023231.25581066
				18031 1023232.53699062
				18032 1023233.81817059
				18033 1023235.09935055
				18034 1023236.38053052
				18035 1023237.66171048
				18036 1023238.94289045
				18037 1023240.22407041
				18038 1023241.50525038
				18039 1023242.78643035
				18040 1023244.06761031
				18041 1023245.34879028
				18042 1023246.62997024
				18043 1023247.91115021
				18044 1023249.19233017
				18045 1023250.47351014
				18046 1023251.7546901
				18047 1023253.03587007
				18048 1023254.31705003
				18049 1023255.59823
				18050 1023256.87940997
				18051 1023258.16058993
				18052 1023259.4417699
				18053 1023260.72294986
				18054 1023262.00412983
				18055 1023263.28530979
				18056 1023264.56648976
				18057 1023265.84766972
				18058 1023267.12884969
				18059 1023268.41002965
				18060 1023269.69120962
				18061 1023270.97238959
				18062 1023272.25356955
				18063 1023273.53474952
				18064 1023274.81592948
				18065 1023276.09710945
				18066 1023277.37828941
				18067 1023278.65946938
				18068 1023279.94064934
				18069 1023281.22182931
				18070 1023282.50300927
				18071 1023283.78418924
				18072 1023285.06536921
				18073 1023286.34654917
				18074 1023287.62772914
				18075 1023288.9089091
				18076 1023290.19008907
				18077 1023291.47126903
				18078 1023292.752449
				18079 1023294.03362896
				18080 1023295.31480893
				18081 1023296.59598889
				18082 1023297.87716886
				18083 1023299.15834883
				18084 1023300.43952879
				18085 1023301.72070876
				18086 1023303.00188872
				18087 1023304.28306869
				18088 1023305.56424865
				18089 1023306.84542862
				18090 1023308.12660858
				18091 1023309.40778855
				18092 1023310.68896851
				18093 1023311.97014848
				18094 1023313.25132845
				18095 1023314.53250841
				18096 1023315.81368838
				18097 1023317.09486834
				18098 1023318.37604831
				18099 1023319.65722827
				18100 1023320.93840824
				18101 1023322.2195882
				18102 1023323.50076817
				18103 1023324.78194813
				18104 1023326.0631281
				18105 1023327.34430807
				18106 1023328.62548803
				18107 1023329.906668
				18108 1023331.18784796
				18109 1023332.46902793
				18110 1023333.75020789
				18111 1023335.03138786
				18112 1023336.31256782
				18113 1023337.59374779
				18114 1023338.87492775
				18115 1023340.15610772
				18116 1023341.43728769
				18117 1023342.71846765
				18118 1023343.99964762
				18119 1023345.28082758
				18120 1023346.56200755
				18121 1023347.84318751
				18122 1023349.12436748
				18123 1023350.40554744
				18124 1023351.68672741
				18125 1023352.96790737
				18126 1023354.24908734
				18127 1023355.53026731
				18128 1023356.81144727
				18129 1023358.09262724
				18130 1023359.3738072
				18131 1023360.65498717
				18132 1023361.93616713
				18133 1023363.2173471
				18134 1023364.49852706
				18135 1023365.77970703
				18136 1023367.06088699
				18137 1023368.34206696
				18138 1023369.62324693
				18139 1023370.90442689
				18140 1023372.18560686
				18141 1023373.46678682
				18142 1023374.74796679
				18143 1023376.02914675
				18144 1023377.31032672
				18145 1023378.59150668
				18146 1023379.87268665
				18147 1023381.15386661
				18148 1023382.43504658
				18149 1023383.71622655
				18150 1023384.99740651
				18151 1023386.27858648
				18152 1023387.55976644
				18153 1023388.84094641
				18154 1023390.12212637
				18155 1023391.40330634
				18156 1023392.6844863
				18157 1023393.96566627
				18158 1023395.24684623
				18159 1023396.5280262
				18160 1023397.80920617
				18161 1023399.09038613
				18162 1023400.3715661
				18163 1023401.65274606
				18164 1023402.93392603
				18165 1023404.21510599
				18166 1023405.49628596
				18167 1023406.77746592
				18168 1023408.05864589
				18169 1023409.33982585
				18170 1023410.62100582
				18171 1023411.90218579
				18172 1023413.18336575
				18173 1023414.46454572
				18174 1023415.74572568
				18175 1023417.02690565
				18176 1023418.30808561
				18177 1023419.58926558
				18178 1023420.87044554
				18179 1023422.15162551
				18180 1023423.43280547
				18181 1023424.71398544
				18182 1023425.99516541
				18183 1023427.27634537
				18184 1023428.55752534
				18185 1023429.8387053
				18186 1023431.11988527
				18187 1023432.40106523
				18188 1023433.6822452
				18189 1023434.96342516
				18190 1023436.24460513
				18191 1023437.52578509
				18192 1023438.80696506
				18193 1023440.08814503
				18194 1023441.36932499
				18195 1023442.65050496
				18196 1023443.93168492
				18197 1023445.21286489
				18198 1023446.49404485
				18199 1023447.77522482
				18200 1023449.05640478
				18201 1023450.33758475
				18202 1023451.61876471
				18203 1023452.89994468
				18204 1023454.18112465
				18205 1023455.46230461
				18206 1023456.74348458
				18207 1023458.02466454
				18208 1023459.30584451
				18209 1023460.58702447
				18210 1023461.86820444
				18211 1023463.1493844
				18212 1023464.43056437
				18213 1023465.71174433
				18214 1023466.9929243
				18215 1023468.27410427
				18216 1023469.55528423
				18217 1023470.8364642
				18218 1023472.11764416
				18219 1023473.39882413
				18220 1023474.68000409
				18221 1023475.96118406
				18222 1023477.24236402
				18223 1023478.52354399
				18224 1023479.80472395
				18225 1023481.08590392
				18226 1023482.36708389
				18227 1023483.64826385
				18228 1023484.92944382
				18229 1023486.21062378
				18230 1023487.49180375
				18231 1023488.77298371
				18232 1023490.05416368
				18233 1023491.33534364
				18234 1023492.61652361
				18235 1023493.89770357
				18236 1023495.17888354
				18237 1023496.46006351
				18238 1023497.74124347
				18239 1023499.02242344
				18240 1023500.3036034
				18241 1023501.58478337
				18242 1023502.86596333
				18243 1023504.1471433
				18244 1023505.42832326
				18245 1023506.70950323
				18246 1023507.99068319
				18247 1023509.27186316
				18248 1023510.55304313
				18249 1023511.83422309
				18250 1023513.11540306
				18251 1023514.39658302
				18252 1023515.67776299
				18253 1023516.95894295
				18254 1023518.24012292
				18255 1023519.52130288
				18256 1023520.80248285
				18257 1023522.08366281
				18258 1023523.36484278
				18259 1023524.64602275
				18260 1023525.92720271
				18261 1023527.20838268
				18262 1023528.48956264
				18263 1023529.77074261
				18264 1023531.05192257
				18265 1023532.33310254
				18266 1023533.6142825
				18267 1023534.89546247
				18268 1023536.17664243
				18269 1023537.4578224
				18270 1023538.73900237
				18271 1023540.02018233
				18272 1023541.3013623
				18273 1023542.58254226
				18274 1023543.86372223
				18275 1023545.14490219
				18276 1023546.42608216
				18277 1023547.70726212
				18278 1023548.98844209
				18279 1023550.26962205
				18280 1023551.55080202
				18281 1023552.83198199
				18282 1023554.11316195
				18283 1023555.39434192
				18284 1023556.67552188
				18285 1023557.95670185
				18286 1023559.23788181
				18287 1023560.51906178
				18288 1023561.80024174
				18289 1023563.08142171
				18290 1023564.36260167
				18291 1023565.64378164
				18292 1023566.92496161
				18293 1023568.20614157
				18294 1023569.48732154
				18295 1023570.7685015
				18296 1023572.04968147
				18297 1023573.33086143
				18298 1023574.6120414
				18299 1023575.89322136
				18300 1023577.17440133
				18301 1023578.45558129
				18302 1023579.73676126
				18303 1023581.01794123
				18304 1023582.29912119
				18305 1023583.58030116
				18306 1023584.86148112
				18307 1023586.14266109
				18308 1023587.42384105
				18309 1023588.70502102
				18310 1023589.98620098
				18311 1023591.26738095
				18312 1023592.54856091
				18313 1023593.82974088
				18314 1023595.11092085
				18315 1023596.39210081
				18316 1023597.67328078
				18317 1023598.95446074
				18318 1023600.23564071
				18319 1023601.51682067
				18320 1023602.79800064
				18321 1023604.0791806
				18322 1023605.36036057
				18323 1023606.64154053
				18324 1023607.9227205
				18325 1023609.20390047
				18326 1023610.48508043
				18327 1023611.7662604
				18328 1023613.04744036
				18329 1023614.32862033
				18330 1023615.60980029
				18331 1023616.89098026
				18332 1023618.17216022
				18333 1023619.45334019
				18334 1023620.73452015
				18335 1023622.01570012
				18336 1023623.29688009
				18337 1023624.57806005
				18338 1023625.85924002
				18339 1023627.14041998
				18340 1023628.42159995
				18341 1023629.70277991
				18342 1023630.98395988
				18343 1023632.26513984
				18344 1023633.54631981
				18345 1023634.82749977
				18346 1023636.10867974
				18347 1023637.38985971
				18348 1023638.67103967
				18349 1023639.95221964
				18350 1023641.2333996
				18351 1023642.51457957
				18352 1023643.79575953
				18353 1023645.0769395
				18354 1023646.35811946
				18355 1023647.63929943
				18356 1023648.92047939
				18357 1023650.20165936
				18358 1023651.48283933
				18359 1023652.76401929
				18360 1023654.04519926
				18361 1023655.32637922
				18362 1023656.60755919
				18363 1023657.88873915
				18364 1023659.16991912
				18365 1023660.45109908
				18366 1023661.73227905
				18367 1023663.01345901
				18368 1023664.29463898
				18369 1023665.57581895
				18370 1023666.85699891
				18371 1023668.13817888
				18372 1023669.41935884
				18373 1023670.70053881
				18374 1023671.98171877
				18375 1023673.26289874
				18376 1023674.5440787
				18377 1023675.82525867
				18378 1023677.10643863
				18379 1023678.3876186
				18380 1023679.66879857
				18381 1023680.94997853
				18382 1023682.2311585
				18383 1023683.51233846
				18384 1023684.79351843
				18385 1023686.07469839
				18386 1023687.35587836
				18387 1023688.63705832
				18388 1023689.91823829
				18389 1023691.19941825
				18390 1023692.48059822
				18391 1023693.76177819
				18392 1023695.04295815
				18393 1023696.32413812
				18394 1023697.60531808
				18395 1023698.88649805
				18396 1023700.16767801
				18397 1023701.44885798
				18398 1023702.73003794
				18399 1023704.01121791
				18400 1023705.29239787
				18401 1023706.57357784
				18402 1023707.85475781
				18403 1023709.13593777
				18404 1023710.41711774
				18405 1023711.6982977
				18406 1023712.97947767
				18407 1023714.26065763
				18408 1023715.5418376
				18409 1023716.82301756
				18410 1023718.10419753
				18411 1023719.38537749
				18412 1023720.66655746
				18413 1023721.94773743
				18414 1023723.22891739
				18415 1023724.51009736
				18416 1023725.79127732
				18417 1023727.07245729
				18418 1023728.35363725
				18419 1023729.63481722
				18420 1023730.91599718
				18421 1023732.19717715
				18422 1023733.47835711
				18423 1023734.75953708
				18424 1023736.04071705
				18425 1023737.32189701
				18426 1023738.60307698
				18427 1023739.88425694
				18428 1023741.16543691
				18429 1023742.44661687
				18430 1023743.72779684
				18431 1023745.0089768
				18432 1023746.29015677
				18433 1023747.57133673
				18434 1023748.8525167
				18435 1023750.13369667
				18436 1023751.41487663
				18437 1023752.6960566
				18438 1023753.97723656
				18439 1023755.25841653
				18440 1023756.53959649
				18441 1023757.82077646
				18442 1023759.10195642
				18443 1023760.38313639
				18444 1023761.66431635
				18445 1023762.94549632
				18446 1023764.22667629
				18447 1023765.50785625
				18448 1023766.78903622
				18449 1023768.07021618
				18450 1023769.35139615
				18451 1023770.63257611
				18452 1023771.91375608
				18453 1023773.19493604
				18454 1023774.47611601
				18455 1023775.75729597
				18456 1023777.03847594
				18457 1023778.31965591
				18458 1023779.60083587
				18459 1023780.88201584
				18460 1023782.1631958
				18461 1023783.44437577
				18462 1023784.72555573
				18463 1023786.0067357
				18464 1023787.28791566
				18465 1023788.56909563
				18466 1023789.85027559
				18467 1023791.13145556
				18468 1023792.41263553
				18469 1023793.69381549
				18470 1023794.97499546
				18471 1023796.25617542
				18472 1023797.53735539
				18473 1023798.81853535
				18474 1023800.09971532
				18475 1023801.38089528
				18476 1023802.66207525
				18477 1023803.94325521
				18478 1023805.22443518
				18479 1023806.50561515
				18480 1023807.78679511
				18481 1023809.06797508
				18482 1023810.34915504
				18483 1023811.63033501
				18484 1023812.91151497
				18485 1023814.19269494
				18486 1023815.4738749
				18487 1023816.75505487
				18488 1023818.03623483
				18489 1023819.3174148
				18490 1023820.59859477
				18491 1023821.87977473
				18492 1023823.1609547
				18493 1023824.44213466
				18494 1023825.72331463
				18495 1023827.00449459
				18496 1023828.28567456
				18497 1023829.56685452
				18498 1023830.84803449
				18499 1023832.12921445
				18500 1023833.41039442
				18501 1023834.69157439
				18502 1023835.97275435
				18503 1023837.25393432
				18504 1023838.53511428
				18505 1023839.81629425
				18506 1023841.09747421
				18507 1023842.37865418
				18508 1023843.65983414
				18509 1023844.94101411
				18510 1023846.22219407
				18511 1023847.50337404
				18512 1023848.78455401
				18513 1023850.06573397
				18514 1023851.34691394
				18515 1023852.6280939
				18516 1023853.90927387
				18517 1023855.19045383
				18518 1023856.4716338
				18519 1023857.75281376
				18520 1023859.03399373
				18521 1023860.31517369
				18522 1023861.59635366
				18523 1023862.87753363
				18524 1023864.15871359
				18525 1023865.43989356
				18526 1023866.72107352
				18527 1023868.00225349
				18528 1023869.28343345
				18529 1023870.56461342
				18530 1023871.84579338
				18531 1023873.12697335
				18532 1023874.40815331
				18533 1023875.68933328
				18534 1023876.97051325
				18535 1023878.25169321
				18536 1023879.53287318
				18537 1023880.81405314
				18538 1023882.09523311
				18539 1023883.37641307
				18540 1023884.65759304
				18541 1023885.938773
				18542 1023887.21995297
				18543 1023888.50113293
				18544 1023889.7823129
				18545 1023891.06349287
				18546 1023892.34467283
				18547 1023893.6258528
				18548 1023894.90703276
				18549 1023896.18821273
				18550 1023897.46939269
				18551 1023898.75057266
				18552 1023900.03175262
				18553 1023901.31293259
				18554 1023902.59411255
				18555 1023903.87529252
				18556 1023905.15647249
				18557 1023906.43765245
				18558 1023907.71883242
				18559 1023909.00001238
				18560 1023910.28119235
				18561 1023911.56237231
				18562 1023912.84355228
				18563 1023914.12473224
				18564 1023915.40591221
				18565 1023916.68709217
				18566 1023917.96827214
				18567 1023919.24945211
				18568 1023920.53063207
				18569 1023921.81181204
				18570 1023923.092992
				18571 1023924.37417197
				18572 1023925.65535193
				18573 1023926.9365319
				18574 1023928.21771186
				18575 1023929.49889183
				18576 1023930.78007179
				18577 1023932.06125176
				18578 1023933.34243173
				18579 1023934.62361169
				18580 1023935.90479166
				18581 1023937.18597162
				18582 1023938.46715159
				18583 1023939.74833155
				18584 1023941.02951152
				18585 1023942.31069148
				18586 1023943.59187145
				18587 1023944.87305141
				18588 1023946.15423138
				18589 1023947.43541135
				18590 1023948.71659131
				18591 1023949.99777128
				18592 1023951.27895124
				18593 1023952.56013121
				18594 1023953.84131117
				18595 1023955.12249114
				18596 1023956.4036711
				18597 1023957.68485107
				18598 1023958.96603103
				18599 1023960.247211
				18600 1023961.52839097
				18601 1023962.80957093
				18602 1023964.0907509
				18603 1023965.37193086
				18604 1023966.65311083
				18605 1023967.93429079
				18606 1023969.21547076
				18607 1023970.49665072
				18608 1023971.77783069
				18609 1023973.05901065
				18610 1023974.34019062
				18611 1023975.62137059
				18612 1023976.90255055
				18613 1023978.18373052
				18614 1023979.46491048
				18615 1023980.74609045
				18616 1023982.02727041
				18617 1023983.30845038
				18618 1023984.58963034
				18619 1023985.87081031
				18620 1023987.15199027
				18621 1023988.43317024
				18622 1023989.71435021
				18623 1023990.99553017
				18624 1023992.27671014
				18625 1023993.5578901
				18626 1023994.83907007
				18627 1023996.12025003
				18628 1023997.40143
				18629 1023998.68260996
				18630 1023999.96378993
				18631 1024001.24496989
				18632 1024002.52614986
				18633 1024003.80732983
				18634 1024005.08850979
				18635 1024006.36968976
				18636 1024007.65086972
				18637 1024008.93204969
				18638 1024010.21322965
				18639 1024011.49440962
				18640 1024012.77558958
				18641 1024014.05676955
				18642 1024015.33794951
				18643 1024016.61912948
				18644 1024017.90030945
				18645 1024019.18148941
				18646 1024020.46266938
				18647 1024021.74384934
				18648 1024023.02502931
				18649 1024024.30620927
				18650 1024025.58738924
				18651 1024026.8685692
				18652 1024028.14974917
				18653 1024029.43092913
				18654 1024030.7121091
				18655 1024031.99328907
				18656 1024033.27446903
				18657 1024034.555649
				18658 1024035.83682896
				18659 1024037.11800893
				18660 1024038.39918889
				18661 1024039.68036886
				18662 1024040.96154882
				18663 1024042.24272879
				18664 1024043.52390875
				18665 1024044.80508872
				18666 1024046.08626869
				18667 1024047.36744865
				18668 1024048.64862862
				18669 1024049.92980858
				18670 1024051.21098855
				18671 1024052.49216851
				18672 1024053.77334848
				18673 1024055.05452844
				18674 1024056.33570841
				18675 1024057.61688837
				18676 1024058.89806834
				18677 1024060.17924831
				18678 1024061.46042827
				18679 1024062.74160824
				18680 1024064.0227882
				18681 1024065.30396817
				18682 1024066.58514813
				18683 1024067.8663281
				18684 1024069.14750806
				18685 1024070.42868803
				18686 1024071.70986799
				18687 1024072.99104796
				18688 1024074.27222793
				18689 1024075.55340789
				18690 1024076.83458786
				18691 1024078.11576782
				18692 1024079.39694779
				18693 1024080.67812775
				18694 1024081.95930772
				18695 1024083.24048768
				18696 1024084.52166765
				18697 1024085.80284761
				18698 1024087.08402758
				18699 1024088.36520755
				18700 1024089.64638751
				18701 1024090.92756748
				18702 1024092.20874744
				18703 1024093.48992741
				18704 1024094.77110737
				18705 1024096.05228734
				18706 1024097.3334673
				18707 1024098.61464727
				18708 1024099.89582723
				18709 1024101.1770072
				18710 1024102.45818717
				18711 1024103.73936713
				18712 1024105.0205471
				18713 1024106.30172706
				18714 1024107.58290703
				18715 1024108.86408699
				18716 1024110.14526696
				18717 1024111.42644692
				18718 1024112.70762689
				18719 1024113.98880685
				18720 1024115.26998682
				18721 1024116.55116679
				18722 1024117.83234675
				18723 1024119.11352672
				18724 1024120.39470668
				18725 1024121.67588665
				18726 1024122.95706661
				18727 1024124.23824658
				18728 1024125.51942654
				18729 1024126.80060651
				18730 1024128.08178647
				18731 1024129.36296644
				18732 1024130.64414641
				18733 1024131.92532637
				18734 1024133.20650634
				18735 1024134.4876863
				18736 1024135.76886627
				18737 1024137.05004623
				18738 1024138.3312262
				18739 1024139.61240616
				18740 1024140.89358613
				18741 1024142.17476609
				18742 1024143.45594606
				18743 1024144.73712603
				18744 1024146.01830599
				18745 1024147.29948596
				18746 1024148.58066592
				18747 1024149.86184589
				18748 1024151.14302585
				18749 1024152.42420582
				18750 1024153.70538578
				18751 1024154.98656575
				18752 1024156.26774571
				18753 1024157.54892568
				18754 1024158.83010565
				18755 1024160.11128561
				18756 1024161.39246558
				18757 1024162.67364554
				18758 1024163.95482551
				18759 1024165.23600547
				18760 1024166.51718544
				18761 1024167.7983654
				18762 1024169.07954537
				18763 1024170.36072533
				18764 1024171.6419053
				18765 1024172.92308526
				18766 1024174.20426523
				18767 1024175.4854452
				18768 1024176.76662516
				18769 1024178.04780513
				18770 1024179.32898509
				18771 1024180.61016506
				18772 1024181.89134502
				18773 1024183.17252499
				18774 1024184.45370495
				18775 1024185.73488492
				18776 1024187.01606489
				18777 1024188.29724485
				18778 1024189.57842482
				18779 1024190.85960478
				18780 1024192.14078475
				18781 1024193.42196471
				18782 1024194.70314468
				18783 1024195.98432464
				18784 1024197.26550461
				18785 1024198.54668457
				18786 1024199.82786454
				18787 1024201.10904451
				18788 1024202.39022447
				18789 1024203.67140444
				18790 1024204.9525844
				18791 1024206.23376437
				18792 1024207.51494433
				18793 1024208.7961243
				18794 1024210.07730426
				18795 1024211.35848423
				18796 1024212.63966419
				18797 1024213.92084416
				18798 1024215.20202413
				18799 1024216.48320409
				18800 1024217.76438406
				18801 1024219.04556402
				18802 1024220.32674399
				18803 1024221.60792395
				18804 1024222.88910392
				18805 1024224.17028388
				18806 1024225.45146385
				18807 1024226.73264381
				18808 1024228.01382378
				18809 1024229.29500375
				18810 1024230.57618371
				18811 1024231.85736368
				18812 1024233.13854364
				18813 1024234.41972361
				18814 1024235.70090357
				18815 1024236.98208354
				18816 1024238.2632635
				18817 1024239.54444347
				18818 1024240.82562343
				18819 1024242.1068034
				18820 1024243.38798336
				18821 1024244.66916333
				18822 1024245.9503433
				18823 1024247.23152326
				18824 1024248.51270323
				18825 1024249.79388319
				18826 1024251.07506316
				18827 1024252.35624312
				18828 1024253.63742309
				18829 1024254.91860305
				18830 1024256.19978302
				18831 1024257.48096298
				18832 1024258.76214295
				18833 1024260.04332292
				18834 1024261.32450288
				18835 1024262.60568285
				18836 1024263.88686281
				18837 1024265.16804278
				18838 1024266.44922274
				18839 1024267.73040271
				18840 1024269.01158267
				18841 1024270.29276264
				18842 1024271.5739426
				18843 1024272.85512257
				18844 1024274.13630254
				18845 1024275.4174825
				18846 1024276.69866247
				18847 1024277.97984243
				18848 1024279.2610224
				18849 1024280.54220236
				18850 1024281.82338233
				18851 1024283.10456229
				18852 1024284.38574226
				18853 1024285.66692222
				18854 1024286.94810219
				18855 1024288.22928216
				18856 1024289.51046212
				18857 1024290.79164209
				18858 1024292.07282205
				18859 1024293.35400202
				18860 1024294.63518198
				18861 1024295.91636195
				18862 1024297.19754191
				18863 1024298.47872188
				18864 1024299.75990185
				18865 1024301.04108181
				18866 1024302.32226178
				18867 1024303.60344174
				18868 1024304.88462171
				18869 1024306.16580167
				18870 1024307.44698164
				18871 1024308.7281616
				18872 1024310.00934157
				18873 1024311.29052153
				18874 1024312.5717015
				18875 1024313.85288146
				18876 1024315.13406143
				18877 1024316.4152414
				18878 1024317.69642136
				18879 1024318.97760133
				18880 1024320.25878129
				18881 1024321.53996126
				18882 1024322.82114122
				18883 1024324.10232119
				18884 1024325.38350115
				18885 1024326.66468112
				18886 1024327.94586108
				18887 1024329.22704105
				18888 1024330.50822102
				18889 1024331.78940098
				18890 1024333.07058095
				18891 1024334.35176091
				18892 1024335.63294088
				18893 1024336.91412084
				18894 1024338.19530081
				18895 1024339.47648077
				18896 1024340.75766074
				18897 1024342.0388407
				18898 1024343.32002067
				18899 1024344.60120064
				18900 1024345.8823806
				18901 1024347.16356057
				18902 1024348.44474053
				18903 1024349.7259205
				18904 1024351.00710046
				18905 1024352.28828043
				18906 1024353.56946039
				18907 1024354.85064036
				18908 1024356.13182032
				18909 1024357.41300029
				18910 1024358.69418026
				18911 1024359.97536022
				18912 1024361.25654019
				18913 1024362.53772015
				18914 1024363.81890012
				18915 1024365.10008008
				18916 1024366.38126005
				18917 1024367.66244001
				18918 1024368.94361998
				18919 1024370.22479994
				18920 1024371.50597991
				18921 1024372.78715988
				18922 1024374.06833984
				18923 1024375.34951981
				18924 1024376.63069977
				18925 1024377.91187974
				18926 1024379.1930597
				18927 1024380.47423967
				18928 1024381.75541963
				18929 1024383.0365996
				18930 1024384.31777956
				18931 1024385.59895953
				18932 1024386.8801395
				18933 1024388.16131946
				18934 1024389.44249943
				18935 1024390.72367939
				18936 1024392.00485936
				18937 1024393.28603932
				18938 1024394.56721929
				18939 1024395.84839925
				18940 1024397.12957922
				18941 1024398.41075918
				18942 1024399.69193915
				18943 1024400.97311912
				18944 1024402.25429908
				18945 1024403.53547905
				18946 1024404.81665901
				18947 1024406.09783898
				18948 1024407.37901894
				18949 1024408.66019891
				18950 1024409.94137887
				18951 1024411.22255884
				18952 1024412.5037388
				18953 1024413.78491877
				18954 1024415.06609874
				18955 1024416.3472787
				18956 1024417.62845867
				18957 1024418.90963863
				18958 1024420.1908186
				18959 1024421.47199856
				18960 1024422.75317853
				18961 1024424.03435849
				18962 1024425.31553846
				18963 1024426.59671842
				18964 1024427.87789839
				18965 1024429.15907836
				18966 1024430.44025832
				18967 1024431.72143829
				18968 1024433.00261825
				18969 1024434.28379822
				18970 1024435.56497818
				18971 1024436.84615815
				18972 1024438.12733811
				18973 1024439.40851808
				18974 1024440.68969804
				18975 1024441.97087801
				18976 1024443.25205798
				18977 1024444.53323794
				18978 1024445.81441791
				18979 1024447.09559787
				18980 1024448.37677784
				18981 1024449.6579578
				18982 1024450.93913777
				18983 1024452.22031773
				18984 1024453.5014977
				18985 1024454.78267766
				18986 1024456.06385763
				18987 1024457.3450376
				18988 1024458.62621756
				18989 1024459.90739753
				18990 1024461.18857749
				18991 1024462.46975746
				18992 1024463.75093742
				18993 1024465.03211739
				18994 1024466.31329735
				18995 1024467.59447732
				18996 1024468.87565728
				18997 1024470.15683725
				18998 1024471.43801722
				18999 1024472.71919718
				19000 1024474.00037715
				19001 1024475.28155711
				19002 1024476.56273708
				19003 1024477.84391704
				19004 1024479.12509701
				19005 1024480.40627697
				19006 1024481.68745694
				19007 1024482.9686369
				19008 1024484.24981687
				19009 1024485.53099684
				19010 1024486.8121768
				19011 1024488.09335677
				19012 1024489.37453673
				19013 1024490.6557167
				19014 1024491.93689666
				19015 1024493.21807663
				19016 1024494.49925659
				19017 1024495.78043656
				19018 1024497.06161652
				19019 1024498.34279649
				19020 1024499.62397646
				19021 1024500.90515642
				19022 1024502.18633639
				19023 1024503.46751635
				19024 1024504.74869632
				19025 1024506.02987628
				19026 1024507.31105625
				19027 1024508.59223621
				19028 1024509.87341618
				19029 1024511.15459614
				19030 1024512.43577611
				19031 1024513.71695608
				19032 1024514.99813604
				19033 1024516.27931601
				19034 1024517.56049597
				19035 1024518.84167594
				19036 1024520.1228559
				19037 1024521.40403587
				19038 1024522.68521583
				19039 1024523.9663958
				19040 1024525.24757576
				19041 1024526.52875573
				19042 1024527.8099357
				19043 1024529.09111566
				19044 1024530.37229563
				19045 1024531.65347559
				19046 1024532.93465556
				19047 1024534.21583552
				19048 1024535.49701549
				19049 1024536.77819545
				19050 1024538.05937542
				19051 1024539.34055538
				19052 1024540.62173535
				19053 1024541.90291532
				19054 1024543.18409528
				19055 1024544.46527525
				19056 1024545.74645521
				19057 1024547.02763518
				19058 1024548.30881514
				19059 1024549.58999511
				19060 1024550.87117507
				19061 1024552.15235504
				19062 1024553.433535
				19063 1024554.71471497
				19064 1024555.99589494
				19065 1024557.2770749
				19066 1024558.55825487
				19067 1024559.83943483
				19068 1024561.1206148
				19069 1024562.40179476
				19070 1024563.68297473
				19071 1024564.96415469
				19072 1024566.24533466
				19073 1024567.52651462
				19074 1024568.80769459
				19075 1024570.08887456
				19076 1024571.37005452
				19077 1024572.65123449
				19078 1024573.93241445
				19079 1024575.21359442
				19080 1024576.49477438
				19081 1024577.77595435
				19082 1024579.05713431
				19083 1024580.33831428
				19084 1024581.61949424
				19085 1024582.90067421
				19086 1024584.18185418
				19087 1024585.46303414
				19088 1024586.74421411
				19089 1024588.02539407
				19090 1024589.30657404
				19091 1024590.587754
				19092 1024591.86893397
				19093 1024593.15011393
				19094 1024594.4312939
				19095 1024595.71247386
				19096 1024596.99365383
				19097 1024598.2748338
				19098 1024599.55601376
				19099 1024600.83719373
				19100 1024602.11837369
				19101 1024603.39955366
				19102 1024604.68073362
				19103 1024605.96191359
				19104 1024607.24309355
				19105 1024608.52427352
				19106 1024609.80545348
				19107 1024611.08663345
				19108 1024612.36781342
				19109 1024613.64899338
				19110 1024614.93017335
				19111 1024616.21135331
				19112 1024617.49253328
				19113 1024618.77371324
				19114 1024620.05489321
				19115 1024621.33607317
				19116 1024622.61725314
				19117 1024623.8984331
				19118 1024625.17961307
				19119 1024626.46079304
				19120 1024627.741973
				19121 1024629.02315297
				19122 1024630.30433293
				19123 1024631.5855129
				19124 1024632.86669286
				19125 1024634.14787283
				19126 1024635.42905279
				19127 1024636.71023276
				19128 1024637.99141272
				19129 1024639.27259269
				19130 1024640.55377266
				19131 1024641.83495262
				19132 1024643.11613259
				19133 1024644.39731255
				19134 1024645.67849252
				19135 1024646.95967248
				19136 1024648.24085245
				19137 1024649.52203241
				19138 1024650.80321238
				19139 1024652.08439234
				19140 1024653.36557231
				19141 1024654.64675228
				19142 1024655.92793224
				19143 1024657.20911221
				19144 1024658.49029217
				19145 1024659.77147214
				19146 1024661.0526521
				19147 1024662.33383207
				19148 1024663.61501203
				19149 1024664.896192
				19150 1024666.17737196
				19151 1024667.45855193
				19152 1024668.7397319
				19153 1024670.02091186
				19154 1024671.30209183
				19155 1024672.58327179
				19156 1024673.86445176
				19157 1024675.14563172
				19158 1024676.42681169
				19159 1024677.70799165
				19160 1024678.98917162
				19161 1024680.27035158
				19162 1024681.55153155
				19163 1024682.83271152
				19164 1024684.11389148
				19165 1024685.39507145
				19166 1024686.67625141
				19167 1024687.95743138
				19168 1024689.23861134
				19169 1024690.51979131
				19170 1024691.80097127
				19171 1024693.08215124
				19172 1024694.3633312
				19173 1024695.64451117
				19174 1024696.92569114
				19175 1024698.2068711
				19176 1024699.48805107
				19177 1024700.76923103
				19178 1024702.050411
				19179 1024703.33159096
				19180 1024704.61277093
				19181 1024705.89395089
				19182 1024707.17513086
				19183 1024708.45631082
				19184 1024709.73749079
				19185 1024711.01867076
				19186 1024712.29985072
				19187 1024713.58103069
				19188 1024714.86221065
				19189 1024716.14339062
				19190 1024717.42457058
				19191 1024718.70575055
				19192 1024719.98693051
				19193 1024721.26811048
				19194 1024722.54929044
				19195 1024723.83047041
				19196 1024725.11165038
				19197 1024726.39283034
				19198 1024727.67401031
				19199 1024728.95519027
				19200 1024730.23637024
				19201 1024731.5175502
				19202 1024732.79873017
				19203 1024734.07991013
				19204 1024735.3610901
				19205 1024736.64227006
				19206 1024737.92345003
				19207 1024739.20463
				19208 1024740.48580996
				19209 1024741.76698993
				19210 1024743.04816989
				19211 1024744.32934986
				19212 1024745.61052982
				19213 1024746.89170979
				19214 1024748.17288975
				19215 1024749.45406972
				19216 1024750.73524968
				19217 1024752.01642965
				19218 1024753.29760962
				19219 1024754.57878958
				19220 1024755.85996955
				19221 1024757.14114951
				19222 1024758.42232948
				19223 1024759.70350944
				19224 1024760.98468941
				19225 1024762.26586937
				19226 1024763.54704934
				19227 1024764.8282293
				19228 1024766.10940927
				19229 1024767.39058924
				19230 1024768.6717692
				19231 1024769.95294917
				19232 1024771.23412913
				19233 1024772.5153091
				19234 1024773.79648906
				19235 1024775.07766903
				19236 1024776.35884899
				19237 1024777.64002896
				19238 1024778.92120892
				19239 1024780.20238889
				19240 1024781.48356886
				19241 1024782.76474882
				19242 1024784.04592879
				19243 1024785.32710875
				19244 1024786.60828872
				19245 1024787.88946868
				19246 1024789.17064865
				19247 1024790.45182861
				19248 1024791.73300858
				19249 1024793.01418854
				19250 1024794.29536851
				19251 1024795.57654848
				19252 1024796.85772844
				19253 1024798.13890841
				19254 1024799.42008837
				19255 1024800.70126834
				19256 1024801.9824483
				19257 1024803.26362827
				19258 1024804.54480823
				19259 1024805.8259882
				19260 1024807.10716816
				19261 1024808.38834813
				19262 1024809.6695281
				19263 1024810.95070806
				19264 1024812.23188803
				19265 1024813.51306799
				19266 1024814.79424796
				19267 1024816.07542792
				19268 1024817.35660789
				19269 1024818.63778785
				19270 1024819.91896782
				19271 1024821.20014778
				19272 1024822.48132775
				19273 1024823.76250772
				19274 1024825.04368768
				19275 1024826.32486765
				19276 1024827.60604761
				19277 1024828.88722758
				19278 1024830.16840754
				19279 1024831.44958751
				19280 1024832.73076747
				19281 1024834.01194744
				19282 1024835.2931274
				19283 1024836.57430737
				19284 1024837.85548734
				19285 1024839.1366673
				19286 1024840.41784727
				19287 1024841.69902723
				19288 1024842.9802072
				19289 1024844.26138716
				19290 1024845.54256713
				19291 1024846.82374709
				19292 1024848.10492706
				19293 1024849.38610702
				19294 1024850.66728699
				19295 1024851.94846696
				19296 1024853.22964692
				19297 1024854.51082689
				19298 1024855.79200685
				19299 1024857.07318682
				19300 1024858.35436678
				19301 1024859.63554675
				19302 1024860.91672671
				19303 1024862.19790668
				19304 1024863.47908664
				19305 1024864.76026661
				19306 1024866.04144658
				19307 1024867.32262654
				19308 1024868.60380651
				19309 1024869.88498647
				19310 1024871.16616644
				19311 1024872.4473464
				19312 1024873.72852637
				19313 1024875.00970633
				19314 1024876.2908863
				19315 1024877.57206626
				19316 1024878.85324623
				19317 1024880.1344262
				19318 1024881.41560616
				19319 1024882.69678613
				19320 1024883.97796609
				19321 1024885.25914606
				19322 1024886.54032602
				19323 1024887.82150599
				19324 1024889.10268595
				19325 1024890.38386592
				19326 1024891.66504588
				19327 1024892.94622585
				19328 1024894.22740582
				19329 1024895.50858578
				19330 1024896.78976575
				19331 1024898.07094571
				19332 1024899.35212568
				19333 1024900.63330564
				19334 1024901.91448561
				19335 1024903.19566557
				19336 1024904.47684554
				19337 1024905.7580255
				19338 1024907.03920547
				19339 1024908.32038544
				19340 1024909.6015654
				19341 1024910.88274537
				19342 1024912.16392533
				19343 1024913.4451053
				19344 1024914.72628526
				19345 1024916.00746523
				19346 1024917.28864519
				19347 1024918.56982516
				19348 1024919.85100512
				19349 1024921.13218509
				19350 1024922.41336506
				19351 1024923.69454502
				19352 1024924.97572499
				19353 1024926.25690495
				19354 1024927.53808492
				19355 1024928.81926488
				19356 1024930.10044485
				19357 1024931.38162481
				19358 1024932.66280478
				19359 1024933.94398474
				19360 1024935.22516471
				19361 1024936.50634468
				19362 1024937.78752464
				19363 1024939.06870461
				19364 1024940.34988457
				19365 1024941.63106454
				19366 1024942.9122445
				19367 1024944.19342447
				19368 1024945.47460443
				19369 1024946.7557844
				19370 1024948.03696436
				19371 1024949.31814433
				19372 1024950.5993243
				19373 1024951.88050426
				19374 1024953.16168423
				19375 1024954.44286419
				19376 1024955.72404416
				19377 1024957.00522412
				19378 1024958.28640409
				19379 1024959.56758405
				19380 1024960.84876402
				19381 1024962.12994398
				19382 1024963.41112395
				19383 1024964.69230392
				19384 1024965.97348388
				19385 1024967.25466385
				19386 1024968.53584381
				19387 1024969.81702378
				19388 1024971.09820374
				19389 1024972.37938371
				19390 1024973.66056367
				19391 1024974.94174364
				19392 1024976.2229236
				19393 1024977.50410357
				19394 1024978.78528354
				19395 1024980.0664635
				19396 1024981.34764347
				19397 1024982.62882343
				19398 1024983.9100034
				19399 1024985.19118336
				19400 1024986.47236333
				19401 1024987.75354329
				19402 1024989.03472326
				19403 1024990.31590322
				19404 1024991.59708319
				19405 1024992.87826316
				19406 1024994.15944312
				19407 1024995.44062309
				19408 1024996.72180305
				19409 1024998.00298302
				19410 1024999.28416298
				19411 1025000.56534295
				19412 1025001.84652291
				19413 1025003.12770288
				19414 1025004.40888284
				19415 1025005.69006281
				19416 1025006.97124278
				19417 1025008.25242274
				19418 1025009.53360271
				19419 1025010.81478267
				19420 1025012.09596264
				19421 1025013.3771426
				19422 1025014.65832257
				19423 1025015.93950253
				19424 1025017.2206825
				19425 1025018.50186246
				19426 1025019.78304243
				19427 1025021.0642224
				19428 1025022.34540236
				19429 1025023.62658233
				19430 1025024.90776229
				19431 1025026.18894226
				19432 1025027.47012222
				19433 1025028.75130219
				19434 1025030.03248215
				19435 1025031.31366212
				19436 1025032.59484208
				19437 1025033.87602205
				19438 1025035.15720202
				19439 1025036.43838198
				19440 1025037.71956195
				19441 1025039.00074191
				19442 1025040.28192188
				19443 1025041.56310184
				19444 1025042.84428181
				19445 1025044.12546177
				19446 1025045.40664174
				19447 1025046.6878217
				19448 1025047.96900167
				19449 1025049.25018164
				19450 1025050.5313616
				19451 1025051.81254157
				19452 1025053.09372153
				19453 1025054.3749015
				19454 1025055.65608146
				19455 1025056.93726143
				19456 1025058.21844139
				19457 1025059.49962136
				19458 1025060.78080132
				19459 1025062.06198129
				19460 1025063.34316126
				19461 1025064.62434122
				19462 1025065.90552119
				19463 1025067.18670115
				19464 1025068.46788112
				19465 1025069.74906108
				19466 1025071.03024105
				19467 1025072.31142101
				19468 1025073.59260098
				19469 1025074.87378094
				19470 1025076.15496091
				19471 1025077.43614088
				19472 1025078.71732084
				19473 1025079.99850081
				19474 1025081.27968077
				19475 1025082.56086074
				19476 1025083.8420407
				19477 1025085.12322067
				19478 1025086.40440063
				19479 1025087.6855806
				19480 1025088.96676056
				19481 1025090.24794053
				19482 1025091.5291205
				19483 1025092.81030046
				19484 1025094.09148043
				19485 1025095.37266039
				19486 1025096.65384036
				19487 1025097.93502032
				19488 1025099.21620029
				19489 1025100.49738025
				19490 1025101.77856022
				19491 1025103.05974018
				19492 1025104.34092015
				19493 1025105.62210012
				19494 1025106.90328008
				19495 1025108.18446005
				19496 1025109.46564001
				19497 1025110.74681998
				19498 1025112.02799994
				19499 1025113.30917991
				19500 1025114.59035987
				19501 1025115.87153984
				19502 1025117.1527198
				19503 1025118.43389977
				19504 1025119.71507974
				19505 1025120.9962597
				19506 1025122.27743967
				19507 1025123.55861963
				19508 1025124.8397996
				19509 1025126.12097956
				19510 1025127.40215953
				19511 1025128.68333949
				19512 1025129.96451946
				19513 1025131.24569942
				19514 1025132.52687939
				19515 1025133.80805936
				19516 1025135.08923932
				19517 1025136.37041929
				19518 1025137.65159925
				19519 1025138.93277922
				19520 1025140.21395918
				19521 1025141.49513915
				19522 1025142.77631911
				19523 1025144.05749908
				19524 1025145.33867904
				19525 1025146.61985901
				19526 1025147.90103898
				19527 1025149.18221894
				19528 1025150.46339891
				19529 1025151.74457887
				19530 1025153.02575884
				19531 1025154.3069388
				19532 1025155.58811877
				19533 1025156.86929873
				19534 1025158.1504787
				19535 1025159.43165866
				19536 1025160.71283863
				19537 1025161.9940186
				19538 1025163.27519856
				19539 1025164.55637853
				19540 1025165.83755849
				19541 1025167.11873846
				19542 1025168.39991842
				19543 1025169.68109839
				19544 1025170.96227835
				19545 1025172.24345832
				19546 1025173.52463828
				19547 1025174.80581825
				19548 1025176.08699822
				19549 1025177.36817818
				19550 1025178.64935815
				19551 1025179.93053811
				19552 1025181.21171808
				19553 1025182.49289804
				19554 1025183.77407801
				19555 1025185.05525797
				19556 1025186.33643794
				19557 1025187.6176179
				19558 1025188.89879787
				19559 1025190.17997784
				19560 1025191.4611578
				19561 1025192.74233777
				19562 1025194.02351773
				19563 1025195.3046977
				19564 1025196.58587766
				19565 1025197.86705763
				19566 1025199.14823759
				19567 1025200.42941756
				19568 1025201.71059752
				19569 1025202.99177749
				19570 1025204.27295746
				19571 1025205.55413742
				19572 1025206.83531739
				19573 1025208.11649735
				19574 1025209.39767732
				19575 1025210.67885728
				19576 1025211.96003725
				19577 1025213.24121721
				19578 1025214.52239718
				19579 1025215.80357714
				19580 1025217.08475711
				19581 1025218.36593708
				19582 1025219.64711704
				19583 1025220.92829701
				19584 1025222.20947697
				19585 1025223.49065694
				19586 1025224.7718369
				19587 1025226.05301687
				19588 1025227.33419683
				19589 1025228.6153768
				19590 1025229.89655676
				19591 1025231.17773673
				19592 1025232.4589167
				19593 1025233.74009666
				19594 1025235.02127663
				19595 1025236.30245659
				19596 1025237.58363656
				19597 1025238.86481652
				19598 1025240.14599649
				19599 1025241.42717645
				19600 1025242.70835642
				19601 1025243.98953638
				19602 1025245.27071635
				19603 1025246.55189632
				19604 1025247.83307628
				19605 1025249.11425625
				19606 1025250.39543621
				19607 1025251.67661618
				19608 1025252.95779614
				19609 1025254.23897611
				19610 1025255.52015607
				19611 1025256.80133604
				19612 1025258.082516
				19613 1025259.36369597
				19614 1025260.64487594
				19615 1025261.9260559
				19616 1025263.20723587
				19617 1025264.48841583
				19618 1025265.7695958
				19619 1025267.05077576
				19620 1025268.33195573
				19621 1025269.61313569
				19622 1025270.89431566
				19623 1025272.17549562
				19624 1025273.45667559
				19625 1025274.73785556
				19626 1025276.01903552
				19627 1025277.30021549
				19628 1025278.58139545
				19629 1025279.86257542
				19630 1025281.14375538
				19631 1025282.42493535
				19632 1025283.70611531
				19633 1025284.98729528
				19634 1025286.26847524
				19635 1025287.54965521
				19636 1025288.83083518
				19637 1025290.11201514
				19638 1025291.39319511
				19639 1025292.67437507
				19640 1025293.95555504
				19641 1025295.236735
				19642 1025296.51791497
				19643 1025297.79909493
				19644 1025299.0802749
				19645 1025300.36145486
				19646 1025301.64263483
				19647 1025302.9238148
				19648 1025304.20499476
				19649 1025305.48617473
				19650 1025306.76735469
				19651 1025308.04853466
				19652 1025309.32971462
				19653 1025310.61089459
				19654 1025311.89207455
				19655 1025313.17325452
				19656 1025314.45443448
				19657 1025315.73561445
				19658 1025317.01679442
				19659 1025318.29797438
				19660 1025319.57915435
				19661 1025320.86033431
				19662 1025322.14151428
				19663 1025323.42269424
				19664 1025324.70387421
				19665 1025325.98505417
				19666 1025327.26623414
				19667 1025328.5474141
				19668 1025329.82859407
				19669 1025331.10977404
				19670 1025332.390954
				19671 1025333.67213397
				19672 1025334.95331393
				19673 1025336.2344939
				19674 1025337.51567386
				19675 1025338.79685383
				19676 1025340.07803379
				19677 1025341.35921376
				19678 1025342.64039372
				19679 1025343.92157369
				19680 1025345.20275366
				19681 1025346.48393362
				19682 1025347.76511359
				19683 1025349.04629355
				19684 1025350.32747352
				19685 1025351.60865348
				19686 1025352.88983345
				19687 1025354.17101341
				19688 1025355.45219338
				19689 1025356.73337334
				19690 1025358.01455331
				19691 1025359.29573328
				19692 1025360.57691324
				19693 1025361.85809321
				19694 1025363.13927317
				19695 1025364.42045314
				19696 1025365.7016331
				19697 1025366.98281307
				19698 1025368.26399303
				19699 1025369.545173
				19700 1025370.82635296
				19701 1025372.10753293
				19702 1025373.3887129
				19703 1025374.66989286
				19704 1025375.95107283
				19705 1025377.23225279
				19706 1025378.51343276
				19707 1025379.79461272
				19708 1025381.07579269
				19709 1025382.35697265
				19710 1025383.63815262
				19711 1025384.91933258
				19712 1025386.20051255
				19713 1025387.48169252
				19714 1025388.76287248
				19715 1025390.04405245
				19716 1025391.32523241
				19717 1025392.60641238
				19718 1025393.88759234
				19719 1025395.16877231
				19720 1025396.44995227
				19721 1025397.73113224
				19722 1025399.0123122
				19723 1025400.29349217
				19724 1025401.57467214
				19725 1025402.8558521
				19726 1025404.13703207
				19727 1025405.41821203
				19728 1025406.699392
				19729 1025407.98057196
				19730 1025409.26175193
				19731 1025410.54293189
				19732 1025411.82411186
				19733 1025413.10529182
				19734 1025414.38647179
				19735 1025415.66765176
				19736 1025416.94883172
				19737 1025418.23001169
				19738 1025419.51119165
				19739 1025420.79237162
				19740 1025422.07355158
				19741 1025423.35473155
				19742 1025424.63591151
				19743 1025425.91709148
				19744 1025427.19827144
				19745 1025428.47945141
				19746 1025429.76063138
				19747 1025431.04181134
				19748 1025432.32299131
				19749 1025433.60417127
				19750 1025434.88535124
				19751 1025436.1665312
				19752 1025437.44771117
				19753 1025438.72889113
				19754 1025440.0100711
				19755 1025441.29125106
				19756 1025442.57243103
				19757 1025443.853611
				19758 1025445.13479096
				19759 1025446.41597093
				19760 1025447.69715089
				19761 1025448.97833086
				19762 1025450.25951082
				19763 1025451.54069079
				19764 1025452.82187075
				19765 1025454.10305072
				19766 1025455.38423068
				19767 1025456.66541065
				19768 1025457.94659062
				19769 1025459.22777058
				19770 1025460.50895055
				19771 1025461.79013051
				19772 1025463.07131048
				19773 1025464.35249044
				19774 1025465.63367041
				19775 1025466.91485037
				19776 1025468.19603034
				19777 1025469.4772103
				19778 1025470.75839027
				19779 1025472.03957024
				19780 1025473.3207502
				19781 1025474.60193017
				19782 1025475.88311013
				19783 1025477.1642901
				19784 1025478.44547006
				19785 1025479.72665003
				19786 1025481.00782999
				19787 1025482.28900996
				19788 1025483.57018992
				19789 1025484.85136989
				19790 1025486.13254986
				19791 1025487.41372982
				19792 1025488.69490979
				19793 1025489.97608975
				19794 1025491.25726972
				19795 1025492.53844968
				19796 1025493.81962965
				19797 1025495.10080961
				19798 1025496.38198958
				19799 1025497.66316954
				19800 1025498.94434951
				19801 1025500.22552948
				19802 1025501.50670944
				19803 1025502.78788941
				19804 1025504.06906937
				19805 1025505.35024934
				19806 1025506.6314293
				19807 1025507.91260927
				19808 1025509.19378923
				19809 1025510.4749692
				19810 1025511.75614916
				19811 1025513.03732913
				19812 1025514.3185091
				19813 1025515.59968906
				19814 1025516.88086903
				19815 1025518.16204899
				19816 1025519.44322896
				19817 1025520.72440892
				19818 1025522.00558889
				19819 1025523.28676885
				19820 1025524.56794882
				19821 1025525.84912878
				19822 1025527.13030875
				19823 1025528.41148872
				19824 1025529.69266868
				19825 1025530.97384865
				19826 1025532.25502861
				19827 1025533.53620858
				19828 1025534.81738854
				19829 1025536.09856851
				19830 1025537.37974847
				19831 1025538.66092844
				19832 1025539.9421084
				19833 1025541.22328837
				19834 1025542.50446834
				19835 1025543.7856483
				19836 1025545.06682827
				19837 1025546.34800823
				19838 1025547.6291882
				19839 1025548.91036816
				19840 1025550.19154813
				19841 1025551.47272809
				19842 1025552.75390806
				19843 1025554.03508802
				19844 1025555.31626799
				19845 1025556.59744796
				19846 1025557.87862792
				19847 1025559.15980789
				19848 1025560.44098785
				19849 1025561.72216782
				19850 1025563.00334778
				19851 1025564.28452775
				19852 1025565.56570771
				19853 1025566.84688768
				19854 1025568.12806764
				19855 1025569.40924761
				19856 1025570.69042758
				19857 1025571.97160754
				19858 1025573.25278751
				19859 1025574.53396747
				19860 1025575.81514744
				19861 1025577.0963274
				19862 1025578.37750737
				19863 1025579.65868733
				19864 1025580.9398673
				19865 1025582.22104726
				19866 1025583.50222723
				19867 1025584.7834072
				19868 1025586.06458716
				19869 1025587.34576713
				19870 1025588.62694709
				19871 1025589.90812706
				19872 1025591.18930702
				19873 1025592.47048699
				19874 1025593.75166695
				19875 1025595.03284692
				19876 1025596.31402688
				19877 1025597.59520685
				19878 1025598.87638682
				19879 1025600.15756678
				19880 1025601.43874675
				19881 1025602.71992671
				19882 1025604.00110668
				19883 1025605.28228664
				19884 1025606.56346661
				19885 1025607.84464657
				19886 1025609.12582654
				19887 1025610.4070065
				19888 1025611.68818647
				19889 1025612.96936644
				19890 1025614.2505464
				19891 1025615.53172637
				19892 1025616.81290633
				19893 1025618.0940863
				19894 1025619.37526626
				19895 1025620.65644623
				19896 1025621.93762619
				19897 1025623.21880616
				19898 1025624.49998612
				19899 1025625.78116609
				19900 1025627.06234606
				19901 1025628.34352602
				19902 1025629.62470599
				19903 1025630.90588595
				19904 1025632.18706592
				19905 1025633.46824588
				19906 1025634.74942585
				19907 1025636.03060581
				19908 1025637.31178578
				19909 1025638.59296574
				19910 1025639.87414571
				19911 1025641.15532568
				19912 1025642.43650564
				19913 1025643.71768561
				19914 1025644.99886557
				19915 1025646.28004554
				19916 1025647.5612255
				19917 1025648.84240547
				19918 1025650.12358543
				19919 1025651.4047654
				19920 1025652.68594536
				19921 1025653.96712533
				19922 1025655.2483053
				19923 1025656.52948526
				19924 1025657.81066523
				19925 1025659.09184519
				19926 1025660.37302516
				19927 1025661.65420512
				19928 1025662.93538509
				19929 1025664.21656505
				19930 1025665.49774502
				19931 1025666.77892498
				19932 1025668.06010495
				19933 1025669.34128492
				19934 1025670.62246488
				19935 1025671.90364485
				19936 1025673.18482481
				19937 1025674.46600478
				19938 1025675.74718474
				19939 1025677.02836471
				19940 1025678.30954467
				19941 1025679.59072464
				19942 1025680.8719046
				19943 1025682.15308457
				19944 1025683.43426454
				19945 1025684.7154445
				19946 1025685.99662447
				19947 1025687.27780443
				19948 1025688.5589844
				19949 1025689.84016436
				19950 1025691.12134433
				19951 1025692.40252429
				19952 1025693.68370426
				19953 1025694.96488422
				19954 1025696.24606419
				19955 1025697.52724416
				19956 1025698.80842412
				19957 1025700.08960409
				19958 1025701.37078405
				19959 1025702.65196402
				19960 1025703.93314398
				19961 1025705.21432395
				19962 1025706.49550391
				19963 1025707.77668388
				19964 1025709.05786384
				19965 1025710.33904381
				19966 1025711.62022378
				19967 1025712.90140374
				19968 1025714.18258371
				19969 1025715.46376367
				19970 1025716.74494364
				19971 1025718.0261236
				19972 1025719.30730357
				19973 1025720.58848353
				19974 1025721.8696635
				19975 1025723.15084346
				19976 1025724.43202343
				19977 1025725.7132034
				19978 1025726.99438336
				19979 1025728.27556333
				19980 1025729.55674329
				19981 1025730.83792326
				19982 1025732.11910322
				19983 1025733.40028319
				19984 1025734.68146315
				19985 1025735.96264312
				19986 1025737.24382308
				19987 1025738.52500305
				19988 1025739.80618302
				19989 1025741.08736298
				19990 1025742.36854295
				19991 1025743.64972291
				19992 1025744.93090288
				19993 1025746.21208284
				19994 1025747.49326281
				19995 1025748.77444277
				19996 1025750.05562274
				19997 1025751.3368027
				19998 1025752.61798267
				19999 1025753.89916264
			};
	\end{axis}

\end{tikzpicture}
\end{document}

	\caption{asdf}
	\label{fig:rebuild_times}
\end{figure}



\section{Computational Overhead}
% TODO: integrate this
To exemplify the importance of optimizing the trigger routines, \autoref{fig:optimization_speedup} illustrates the runtime differences between naive and optimized triggers in the equilibrium scenario. The naive version recalculates the average over all samples each iteration, whereas the optimized version uses a ring buffer and running summation to reduce computational cost.
The speedup experienced is not only due to a lowering of computational overhead, but also due to less tuning iterations. This fact can be explained by the aforementioned self influence of the triggers: higher overhead might lead to higher fluctuation in iteration runtime which in turn leads to unstable trigger behavior, especially in the averaging trigger.

% TODO: real overhead hard to quantify?
% TODO: check if this makes any sense

\begin{figure}[htpb]
	\centering
	\begin{tikzpicture}
		\tikzset{
			barstyle1/.style={fill=tumblueaccmedium, draw=chaptertumblue},
			barstyle2/.style={pattern color=tumblueaccmedium, draw=chaptertumblue, pattern=north east lines},
		}
		\def\barwidth{15pt}
		\begin{axis}[
			height=0.4\textwidth,
			xlabel={Trigger Strategy},
			ylabel={Total Runtime in \unit{ns}},
			enlarge x limits=0.25,
			symbolic x coords={Avg,Split,Regression},
			xtick=data,
			bar width=\barwidth,
			legend entries={Unoptimized, Optimized}
			]
			
			\addlegendimage{legend image code/.code={
					\draw[barstyle1, anchor=center] (0cm, -0.15cm)  rectangle (0.3cm,0.15cm);
			}}
			\addlegendentry{\scriptsize Unoptimized}
			\addlegendimage{legend image code/.code={
					\draw[barstyle2, anchor=center] (0cm, -0.15cm)  rectangle (0.3cm,0.15cm);
			}}
			\addlegendentry{\scriptsize Optimized}
			
			% unoptimized average
			\addplot[ybar, bar shift=-0.5*\barwidth, barstyle1] coordinates {
				(Avg,1694063817546)
				(Split,1141728867654)
				(Regression,1112338563086)
			};
			% optimized average
			\addplot[ybar, bar shift=0.5*\barwidth, barstyle2] coordinates {
				(Avg,962648240758)
				(Split,999681150940)
				(Regression,889891999680)
			};
		\end{axis}
	\end{tikzpicture}
	\caption{Average Speedup between unoptimized and optimized runs for the \texttt{TimeBasedAverage}, \texttt{TimeBasedSplit} and \texttt{TimeBasedRegression} strategies.}
	\label{fig:optimization_speedup}
\end{figure}
%TODO: add speedup after optimization (self influence of runtime)


\tikzset{
	linestyleA/.style={chaptertumblue, densely dotted, thick},
	linestyleB/.style={tumblueaccdark, densely dashed, thick},
	linestyleC/.style={tumblueaccmedium, solid, thick},
	linestyleD/.style={tumblueacclight, densely dashdotted, thick},
}

\pgfplotsset{
	triggerplot/.style={
			height=0.7\textwidth,
			width=\textwidth,
			xlabel={Trigger Factor $\lambda$},
			xtick={1.25, 1.5, 1.75},
			legend style={font=\small},
			legend cell align=center,
			legend columns=3,
			legend style={at={(0.5,1.03)}, anchor=south, fill=none, draw=none, align=center},
			legend image post style={xscale=0.5},
			ylabel near ticks},
	logtriggerplot/.style={
			triggerplot,
			log basis y = 10,
			log ticks with fixed point}
}




\section{Benchmarking Results}

% TODO: text
% Formula for deviation from static baseline:
% ROUND((stat_val/dyn_val - 1)*100)
% TODO: explicitly state speedup formula

The blue bars in the graphs represent the runtime of that particular iteration.
In the configuration plots, the colored background identifies the used configuration: same configurations map to the same color. The gaps in the plot are where tuning iterations have been logged -- as their runtime is not relevant for the scenario change and would distort the actual runtime plot, they are not reported here. The red vertical lines indicate the start of a tuning phase.

\subsection{Equilibrium}

\begin{figure}[htpb]
	\centering
	\begin{subfigure}{0.45\textwidth}
		\begin{tikzpicture}
			\begin{axis}[
					triggerplot,
					legend columns=2,
					ylabel={Speedup \%},
				]
				% TimeBasedSimple
				\addplot[linestyleD] coordinates{
						(1.25,-5)
						(1.5,12)
						(1.75,5)};
				% TimeBasedAverage 1000
				\addplot[linestyleA] coordinates{
						(1.25,21)
						(1.5,44)
						(1.75,47)};
				% TimeBasedAverage 500
				\addplot[linestyleB] coordinates{
						(1.25,10)
						(1.5,34)
						(1.75,47)};
				% TimeBasedAverage 250
				\addplot[linestyleC] coordinates{
						(1.25,17)
						(1.5,26)
						(1.75,34)};
				\legend{Simple,Avg-1000,Avg-500,Avg-250}
			\end{axis}
		\end{tikzpicture}
		%		\subcaption{Simple and Average Triggers}
	\end{subfigure}
	\hspace{0.05\textwidth}
	\begin{subfigure}{0.45\textwidth}
		\begin{tikzpicture}
			\begin{axis}[
					logtriggerplot,
					legend columns=2,
					ylabel={Tuning Iterations \%},
					ymode=log,
					ymin=1,
					ymax=100,
				]
				% TimeBasedSimple
				\addplot[tumblueacclight, thick] coordinates{
						(1.25,29.37)
						(1.5,2.26)
						(1.75,2.26)};
				% TimeBasedAverage 1000
				\addplot[linestyleA] coordinates{
						(1.25,2.26)
						(1.5,1.51)
						(1.75,1.51)};
				% TimeBasedAverage 500
				\addplot[linestyleB] coordinates{
						(1.25,2.26)
						(1.5,2.26)
						(1.75,1.51)};
				% TimeBasedAverage 250
				\addplot[linestyleC] coordinates{
						(1.25,2.26)
						(1.5,1.51)
						(1.75,1.51)};
				\legend{Simple,Avg-1000,Avg-500,Avg-250}
			\end{axis}
		\end{tikzpicture}
		%		\subcaption{Simple and Average Triggers}
	\end{subfigure}
	\begin{subfigure}{0.45\textwidth}
		\begin{tikzpicture}
			\begin{axis}[
					triggerplot,
					ylabel={Speedup \%},
				]
				% TimeBasedSplit 1000
				\addplot[linestyleA] coordinates{
						(1.25,42)
						(1.5,46)
						(1.75,35)};
				% TimeBasedSplit 500
				\addplot[linestyleB] coordinates{
						(1.25,43)
						(1.5,38)
						(1.75,40)};
				% TimeBasedSplit 250
				\addplot[linestyleC] coordinates{
						(1.25,-6)
						(1.5,28)
						(1.75,-7)};
				\legend{Split-1000,Split-500,Split-250}
			\end{axis}
		\end{tikzpicture}
		%		\subcaption{Split Trigger}
	\end{subfigure}
	\hspace{0.05\textwidth}
	\begin{subfigure}{0.45\textwidth}
		\begin{tikzpicture}
			\begin{axis}[
					logtriggerplot,
					ylabel={Tuning Iterations \%},
					ymode=log,
					ymin=1,
					ymax=100,
				]
				% TimeBasedSplit 1000
				\addplot[linestyleA] coordinates{
						(1.25,3.77)
						(1.5,3.01)
						(1.75,3.77)};
				% TimeBasedSplit 500
				\addplot[linestyleB] coordinates{
						(1.25,4.52)
						(1.5,4.52)
						(1.75,3.77)};
				% TimeBasedSplit 250
				\addplot[linestyleC] coordinates{
						(1.25,81.7)
						(1.5,6.03)
						(1.75,81.9)};
				\legend{Split-1000,Split-500,Split-250}
			\end{axis}
		\end{tikzpicture}
		%		\subcaption{Split Trigger}
	\end{subfigure}
	\begin{subfigure}{0.45\textwidth}
		\begin{tikzpicture}
			\begin{axis}[
					triggerplot,
					legend columns=2,
					ylabel={Speedup \%},
				]
				% TimeBasedRegression 2000
				\addplot[linestyleA] coordinates{
						(1.25,-3)
						(1.5,13)
						(1.75,14)};

				% TimeBasedRegression 1500
				\addplot[linestyleB] coordinates{
						(1.25,-4)
						(1.5,14)
						(1.75,15)};
						
				% TimeBasedRegression 1000
				\addplot[linestyleC] coordinates{
						(1.25,7)
						(1.5,15)
						(1.75,14)};
				
				% TimeBasedRegression 500		
				\addplot[linestyleD] coordinates{
					(1.25,5)
					(1.5,16)
					(1.75,15)};

				\legend{Reg-2000,Reg-1500,Reg-1000,Reg-500}
			\end{axis}
		\end{tikzpicture}
		%		\subcaption{Regression Trigger}
	\end{subfigure}%
	\hspace{0.05\textwidth}
	\begin{subfigure}{0.45\textwidth}
		\begin{tikzpicture}
			\begin{axis}[
					logtriggerplot,
					legend columns=2,
					ylabel={Tuning Iterations \%},
					ymode=log,
					ymin=1,
					ymax=100,
				]
				% TimeBasedRegression 2000
				\addplot[linestyleA] coordinates{
						(1.25,18.72)
						(1.5,4.32)
						(1.75,1.44)};

				% TimeBasedRegression 1500
				\addplot[linestyleB] coordinates{
						(1.25,19.37)
						(1.5,1.44)
						(1.75,1.44)};
						
				% TimeBasedRegression 1000
				\addplot[linestyleC] coordinates{
						(1.25,7.88)
						(1.5,1.44)
						(1.75,1.44)};
				
				% TimeBasedRegression 500		
				\addplot[linestyleD] coordinates{
					(1.25,10.08)
					(1.5,1.44)
					(1.75,1.44)};

				\legend{Reg-2000,Reg-1500,Reg-1000,Reg-500}
			\end{axis}
		\end{tikzpicture}
		%		\subcaption{Regression Trigger}
	\end{subfigure}%
	\caption{Trigger behavior in the equilibrium scenario, the numbers in the legends refer to the number of samples $n$ considered. Note the logarithmic scale in the plots on the right hand side.}
	\label{fig:params_equil}
\end{figure}

As can be seen in \autoref{fig:params_equil}, a trigger factor of $\lambda=1.5$ leads to increased speedup compared to $\lambda=1.25$ in nearly all triggering strategies. This is however mainly due to the nature of the equilibrium scenario: after the initial configuration selection, the optimal configuration is not expected to change.
Therefore, not initiating any new tuning phases will lead to a decrease in total simulation runtime. That the speedup is indeed  a result of the decreased number of tuning iterations can be verified by looking at the right-hand side plots; for the simple and averaging trigger it is most noticeable.
Additionally, triggers with a larger sample size will typically trigger less frequently, as more of the variability in iteration runtime is smoothed out. For a too large number of samples, the speedup decreases however, as the computational overhead is directly proportional to the number of samples.
% TODO: rewrite
%This is best seen in the plots for the regression trigger, as the share of tuning iterations remains constant for all sample sizes, but the speedup decreases. Especially for the regression triggers, the computations required per sample and in each iteration are significant.

%The increase of th enumber of tuning iterations as seen in the split trigger with $n=250$ can be explained by \textellipsis
% TODO: self-interaction
% TODO: longer intervals -> slower reaction?

The collected data suggests default parameters as presented in \autoref{tab:equil_defaults}.
\begin{table}[htpb]
	\centering
	\begin{tabular}{lcc}
		\toprule
		\textbf{Trigger}                      & \textbf{Trigger factor $\lambda$} & \textbf{Number of samples $n$} \\ [0em]
		\midrule
		\texttt{TimeBasedSimple}     & $1.5$                   & --                     \\
		\texttt{TimeBasedAverage}    & $1.75$                   & 500                   \\
		\texttt{TimeBasedSplit}      & $1.5$                    & 1000                  \\
		\texttt{TimeBasedRegression} & $1.5$                    & 500                  \\
		\bottomrule
	\end{tabular}
	\caption{Suggested default parameters for the equilibrium scenario.}
	\label{tab:equil_defaults}
\end{table}


% TODO: equilibrium_dynamic_TimeBasedRegression_1.25_500 looks promising
% TODO: equilibrium_dynamic_TimeBasedRegression_1.25_1000 looks even better

\subsection{Exploding Liquid}
\subsection{Heating Sphere}



%\section{Trigger Behavior}


\subsection{Runtime and Number of Tuning Iterations}
\subsection{Optimality}



